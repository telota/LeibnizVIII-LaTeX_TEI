%   % !TEX root = ../../VIII,3_Rahmen-TeX_9.tex
%  
%   Band VIII, 3		Rubrik STOSS
%
%   Signatur/Tex-Datei:	LH_35_10_08_017
%
%   RK-Nr. 	60069		
%
%   Überschrift: 	(keine)
%   
%   Unterrubrik:			Auszüge
%
%
\selectlanguage{ngerman}
\frenchspacing
%
\begin{ledgroupsized}[r]{120mm}
\footnotesize
\pstart
\noindent\textbf{Überlieferung:}
\pend
\end{ledgroupsized}
%
\begin{ledgroupsized}[r]{114mm}
\footnotesize
\pstart \parindent -6mm
\makebox[6mm][l]{\textit{L}}%
Notiz:
LH~XXXV~10, 8~Bl.~17. 
Ein unregelmäßig beschnittener Papierstreifen (ca.~17~x~1,5 cm).
Eine Seite auf Bl.~17~r\textsuperscript{o}; Bl. 17~v\textsuperscript{o} leer.
\pend
\end{ledgroupsized}
%
%
\vspace{5mm}
\begin{ledgroup}
\footnotesize
\pstart
\noindent%
\textbf{Datierungsgründe:}
Leibniz nimmt 
%
\protect\index{Namensregister}{\textso{Marci}, Johannes Marek 1595\textendash1667}Marcus Marci von Kronland 
%
in vorliegender Notiz als einen Autor in Sachen Stoß zur Kenntnis. 
%
In anderen Zusammenhängen zitiert er ihn mehrfach seit seiner Mainzer Zeit, aber erst im 
%
\textit{Specimen Dynamicum}\cite{02032} (\cite{01023}\textit{AE}, April 1695, S.~150; \textit{LMG} VI, S.~239\,f.) 
%
und in Briefen der darauf folgenden Jahre (\title{LSB} II,~3 N.~194, S.~503\cite{01514}; \title{LSB} III,~7 N.~1, S.~15\cite{01515}) 
%
wird Marci als einer derjenigen namentlich genannt, die Vorarbeiten zur Bewegungs- und Stoßlehre geleistet
%
bzw.\ sich auf diesen Gebieten versucht haben. 
%
Etwas kürzer fällt diese Reihe an Namen noch in der \textit{Brevis demonstratio} (Januar 1686) aus, 
%
die am Schluss einen ähnlichen Verweis auf Vorarbeiten liefert, Marci aber unerwähnt lässt (\cite{01099}\textit{LSB} VI, 4 N.~369, S.~2030). 
%
Die Datierung erfolgt unter der Annahme, dass Leibniz in diesem Zusammenhang auf Marci 
%
erst in der Zeit zwischen der \cite{01099}\textit{Brevis demonstratio} und dem \textit{Specimen Dynamicum}\cite{02032} 
%
aufmerksam geworden ist, wovon vorliegende Notiz zeugen könnte. 
%
Eine frühere oder spätere Datierung ist nicht auszuschließen.  
%Anfang 1686
%Anfang 1695
\pend 
\end{ledgroup}
%
%
\selectlanguage{latin}
\frenchspacing
% \newpage%
\vspace{8mm}
\pstart%
\normalsize%
\noindent%
\lbrack17~r\textsuperscript{o}\rbrack\
\edtext{Marcus Marci\protect\index{Namensregister}{\textso{Marci}, Johannes Marek 1595\textendash1667} in lib.\ de motu,}%
{\lemma{Marcus Marci in lib.\ de motu}%
\Cfootnote{\textsc{J.\,M.~Marci}, \textit{De proportione motus figurarum rectilinearum}, Prag 1648\cite{00539}, Theorema XV\textendash XX: \glqq De Proportione Motus Orbiculorum tam ad se, quam ad Motum Orbiculi Contigui, a quo Impellitur\grqq, Q2\textendash R1.}} 
%
agit ingeniose de 
orbiculorum\protect\index{Sachverzeichnis}{orbiculum} impulsorum 
motibus.\protect\index{Sachverzeichnis}{motus orbiculorum impulsorum}
\pend
\count\Bfootins=1200%
\count\Afootins=1200%
\count\Cfootins=1200 