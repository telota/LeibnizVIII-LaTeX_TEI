%
%
%   Band VIII, 3 N.~??X (N.~??X.1 bis ??X.5)
%   Signatur/Tex-Datei: CogitationesDeSono_intro
%   Einleitung zu RK-Nr. 38526 + 38527 + 38528 + 38533 + 38536 + 38537 + 38539
%   Überschrift: [Cogitationes de sono]
%   Datierung: [zweite Hälfte August 1681 – erste Hälfte 1685]
%   WZ: ()
%.  SZ: ()
%.  Bilddateien (PDF): ()
%
%
\footnotesize%
\pstart%
\noindent%
\label{cogitationesnovae-intro}%
Bei den folgenden Stücken N.~12\textsubscript{1} bis 12\textsubscript{5} handelt es sich um Texte, die sowohl ihrem Inhalt wie ihrer Entstehung nach eng miteinander zusammenhängen.
Obwohl sie von Leibniz nicht datiert wurden, lässt sich ihre Reihenfolge und Entstehungszeit anhand des Briefwechsels genau bestimmen. 
\pend\pstart%
Der Helmstedter Medizinprofessor G.\,C. Schelhammer\protect\index{Namensregister}{\textso{Schelhammer} (Schelhammerus), Günther Christoph 1649\textendash1716} kündigte in seinem Brief an Leibniz vom 8. (18.) November 1680 eine \textit{auditus explicationem plane novam} an (\cite{01279}\textit{LSB} III,~3 N.~124, S.~286.17\textendash18), womit er offenbar seine (erst 1684 erschienene) Abhandlung \cite{01204}\textit{De auditu} meinte.
In seiner Antwort vom 6. (16.) Dezember 1680 erwähnte Leibniz angebliche ältere Aufzeichnungen \textit{de modo, quo fit sonus ac propagatur} (nicht ermittelt; siehe aber die Datierungsbegründung von N.~1), % cujus vera naturam naturam nemo hacetnus distincte explicuit 
und behauptete zudem, die Gesetze der Schwingungen \textit{ex intima Geometria} ergründet zu haben (\cite{01275}\textit{LSB} III,~3 N.~139, S.~305.3\textendash5), womit er wohl auf N.~9 anspielte.
In seiner Entgegnung vom 31. Dezember 1680 (10. Januar 1681) äußerte Schelhammer den Wunsch, mehr von Leibnizens Überlegungen zu erfahren (\cite{01280}\textit{LSB} III,~3 N.~153, S.~318.19\textendash21).
Die Antwort an Schelhammer von Februar/März 1681 überliefert Leibnizens erste (erhaltene) ausführliche Darstellung, wie der Schall aus vibrierenden elastischen Körpern entstehe, sich in die Luft als elastisches Fluidum ausbreite und ins Ohr aufgenommen werde (\cite{01194}\textit{LSB} III,~3 N.~182, S.~355\textendash361).
Die Texte N.~12\textsubscript{1}, 12\textsubscript{2} und 12\textsubscript{3} knüpfen sämtlich an diese Darstellung an. % unmittelbar
Schelhammer drückte in seiner weiteren Entgegnung vom 13. (23.) April 1681 Zustimmung und Tadel aus:
Unter anderem hielt er fest, dass selbst ein unelastischer Körper wie etwa ein Kissen, geschlagen, % (\textit{culcitra}) 
einen Klang erzeuge (\cite{01200}\textit{LSB} III,~3 N.~206, S.~395.13\textendash396.5).
Leibniz erhielt Schelhammers Brief erst am 4. (14.) August 1681 und erwiderte auf den % dort formulierten 
Einwand in einer Randbemerkung (ebd., S. 396.12\textendash14).
Dieselbe Erwiderung findet sich dann in N.~12\textsubscript{1}, 12\textsubscript{2} und 12\textsubscript{3} (S.~\refpassage{LH_37_01_018r_Schelh-1}{LH_37_01_018r_Schelh-2}; \refpassage{LH_37_01_020r_culcitra-1}{LH_37_01_020r_culcitra-2}; % \refpassage{LH_37_01_001r_culcitra-1}{LH_37_01_001r_culcitra-2}; 
\refpassage{LH_37_01_004r_culcitra-1}{LH_37_01_004r_culcitra-2}) wieder.
Die zweite Hälfte August 1681 ist daher als Terminus post quem der Gesamtdatierung von N.~12 anzusehen.
\pend
\pstart%
Schon im März/April 1681 hatte E.~Mariotte\protect\index{Namensregister}{\textso{Mariotte}, Edme, Seigneur de Chazeuil ca. 1620\textendash1684} an Leibniz von seinem Vortrag \textit{sur l’usage des organes de l’ouye} an der Pariser Akademie\protect\index{Sachverzeichnis}{Académie Royale des Sciences} geschrieben (\cite{01209}\textit{LSB} III,~3 N.~193, S.~375.1\textendash14).
Die von Mariotte mitgeteilten Einzelheiten über die Anatomie des Ohres ergänzten nun die Darstellung akustischer Phänomene, die Leibniz kurz davor % (im Februar/März) 
an Schelhammer\protect\index{Namensregister}{\textso{Schelhammer} (Schelhammerus), Günther Christoph 1649\textendash1716} gesendet hatte.
Am 8. August 1681 berich\-te\-te Mariotte ferner über einen vom Arzt J.-G. Duverney\protect\index{Namensregister}{\textso{Duverney} (Duvernejus, Duvernaeus), Joseph-Guichard 1648\textendash1730} an der Pariser Akademie gehaltenen Vortrag zur Anatomie des Ohres (\cite{01210}\textit{LSB} III,~3 N.~262, S.~464.7-18), woraus Duverneys Abhandlung \textit{De l'organe de l'ouie} (1683) entstehen sollte.\cite{01202}
In der zweiten Hälfte August 1681 fasste Leibniz für Mariotte seine Ansichten über Entstehung, Übertragung und Aufnahme des Schalls zusammen, wobei er auch auf Schelhammers \glqq Kissen\grqq-Einwand einging (\cite{01193}\textit{LSB} III,~3 N.~269, bes. S.~479.16\textendash20).
Mariotte teilte Leibniz am 29. November 1681 seinen Beifall mit (\cite{01222}\textit{LSB} III,~3 N.~297, S.~518.14-519.1).
\pend
\pstart%
Vor dem Jahresende verfasste Leibniz noch eine ausführliche Antwort auf Schelhammers Brief vom 13. (23.) April, in der er auf sämtliche Einwände des Helmstedter Professors einging (\cite{01195}\textit{LSB} III,~3 N.~311, \textit{L\textsuperscript{1}}). Der Tod von Schelhammers Schwiegervater H.~Conring\protect\index{Namensregister}{\textso{Conring}, Hermann 1606\textendash1681} am 22. % 12. (22.) 
Dezember 1681 veranlasste Leibniz jedoch dazu, diese erste Fassung seiner Antwort durch eine zweite, kürzere und weniger detaillierte (ebd., \textit{L\textsuperscript{2}}) zu ersetzen, die am 13. (23.) Januar 1682 auch versendet wurde. 
\pend
\pstart%
Auf die von Leibniz 1681 verfassten Briefe an Schelhammer\protect\index{Namensregister}{\textso{Schelhammer} (Schelhammerus), Günther Christoph 1649\textendash1716} und Mariotte\protect\index{Namensregister}{\textso{Mariotte}, Edme, Seigneur de Chazeuil ca. 1620\textendash1684} bezieht sich das Konzept N.~12\textsubscript{1} in seiner Überschrift ausdrücklich: \textit{De soni generatione, propagatione et expressione in organo, Mechanice explicatis; excerpta ex Epistolis G.G.L. ad viros quosdam clarissimos, qui in Germania Galliaque idem argumentum versant}.
Gemeint sind damit zweifelsohne die Briefe an Schelhammer von Februar/März (\textit{LSB} III,~3 N.~182) und an Mariotte von der zweiten Hälfte August (ebd. N.~269).
Dabei
\pend
\newpage
\pstart
\noindent könnte N.~12\textsubscript{1} älter sein als die erste Fassung des Briefes an Schelhammer vom Januar 1682 (\textit{LSB} III,~3 N.~311, \textit{L\textsuperscript{1}}), in der Leibniz den Ausdruck \textit{sonus clappans} als Bezeich\-nung für einen zu hohen, stumpfen Klang ver\-wendet (ebd. N.~311, \textit{L\textsuperscript{1}}, S.~547.22\textendash25); ähnlich äußert er sich im späteren Text N.~12\textsubscript{3} (%
% S.~\refpassage{LH_37_01_002v_clappans-1}{LH_37_01_002v_clappans-2}) und N.~??X\textsubscript{4} (
S.~\refpassage{LH_37_01_006v_clappans-1}{LH_37_01_006v_clappans-2}), nicht aber in N.~12\textsubscript{1} oder 12\textsubscript{2}.
Daher dürfte N.~12\textsubscript{1} zwischen der zweiten Hälfte August und dem Jahresende 1681 als Überarbeitung der in den erwähnten Briefen dargestellten Inhalte entstanden sein.
Da das Phänomen des \textit{sonus atonus} aber bereits im Brief an Mariotte\protect\index{Namensregister}{\textso{Mariotte}, Edme, Seigneur de Chazeuil ca. 1620\textendash1684} aus der zweiten Hälfte August 1681 geschildert wird (jedoch ohne die Bezeichnung \textit{sonus clappans}: \textit{LSB} \cite{01193}III,~3 N.~269, S.~480.2\textendash6), ist nicht auszuschließen, dass N.~12\textsubscript{1} doch nach Leibnizens Brief an Schelhammer N.~311 verfasst wurde, d.h. im Laufe des Jahres 1682, jedenfalls aber vor dem Konzept N.~12\textsubscript{3},~\textit{L\textsuperscript{1}}. 
\pend\pstart%
Das titellose Konzept N.~12\textsubscript{2}, dem editorisch die Überschrift \textit{Explicatio mechanica soni} zugewiesen wird, ist wohl im selben Zeitraum wie N.~12\textsubscript{1}, aber der Reihe nach als zweites entstanden.
Denn obschon N.~12\textsubscript{2} im Anfangsteil einen stichwortartigen Charakter aufweist und insgesamt weniger aus\-führlich als N.~12\textsubscript{1} ist, % erweist sich das titellose Konzept als vollständiger, wenn man N.~??X\textsubscript{1} und N.~??X\textsubscript{2} mit N.~??X\textsubscript{3} vergleicht.
wird in N.~12\textsubscript{2} eine größere Anzahl an Einzelheiten berücksichtigt, die später in N.~12\textsubscript{3} Eingang finden;
unberührt bleibt in N.~12\textsubscript{2} allerdings (wie in N.~12\textsubscript{1}) das dritte und letzte in N.~12\textsubscript{3} behandelte Thema, nämlich die Aufnahme des Schalls ins innere Ohr.
Ferner sind in N.~12\textsubscript{2} Quellen erwähnt, die in N.~12\textsubscript{3} besprochen werden, in N.~12\textsubscript{1} hingegen unbenannt bleiben: etwa Otto von Guerickes \protect\index{Namensregister}{\textso{Guericke} (Gerickius, Gerick.), Otto von 1602\textendash1686} \textit{Experimenta nova} (Amsterdam 1672).\cite{00055}
Somit ist N.~12\textsubscript{2} wahrscheinlich erst nach N.~12\textsubscript{1} entstanden und kann \textendash\ 
nebst N.~12\textsubscript{1} \textendash\ 
als unvollständiger Entwurf zu N.~12\textsubscript{3} betrachtet werden. 
\pend
\pstart%
Das\edlabel{explicatiosoni_difuvg-1} Konzept N.~12\textsubscript{3},~\textit{L\textsuperscript{1}} nimmt in der Überschrift \textit{Explicatio Soni et auditus ex epistola ad amicum hyeme superiori scripta} offenbar auf Leibnizens Brief an Schelhammer vom 13. (23.) Januar 1682 Bezug.
Die Angabe \textit{hyeme superiori} legt nahe, dass das Konzept % N.~??X\textsubscript{3},~\textit{L\textsuperscript{1}} 
frühestens im Frühjahr 1682 und spätestens im Winter 1682/1683 angefertigt wurde, jedenfalls bevor es vom Sekretär J.\,D. Brands\-hagen\protect\index{Namensregister}{\textso{Brandshagen} (Brondhaguen), Jobst Dietrich 1659\textendash nach 1716} abgeschrieben wurde.
In seinem Brief vom 28. April 1682 an C.~Pfautz,\protect\index{Namensregister}{\textso{Pfautz} (Pfauzius), Christoph 1645\textendash1711} Mitherausgeber der \textit{Acta eruditorum}, bezieht sich Leibniz wohl auf das Konzept N.~12\textsubscript{3},~\textit{L\textsuperscript{1}}, als er mit Blick auf eine mögliche Veröffentlichung berichtet, bei sich eine \textit{dissertatiuncula} zu haben, die gut zwei Bogen umfasse und zum ersten Mal die Entstehung des Schalls \textit{plane mechanice} erkläre (\cite{01281}\textit{LSB} III,~3 N.~345, S.~596.20\textendash597.2).
Daher ist anzunehmen, dass Ende April 1682 das Konzept N.~12\textsubscript{3},~\textit{L\textsuperscript{1}} gerade entstanden oder in Entstehung war.\edlabel{explicatiosoni_difuvg-2}
\pend
\pstart%
Brandshagens Reinschrift N.~12\textsubscript{3},~\textit{l}, die auf dem Konzept \textit{L\textsuperscript{1}} beruht, hat Leibniz mit der neuen Überschrift \textit{Cogitationes novae, quomodo formetur sonus, et per aerem propagetur, atque in organo auditus exprimatur} versehen und vornehmlich im Schlussteil überarbeitet (Textschicht \textit{Lil}).
Da sich Brandshagen\protect\index{Namensregister}{\textso{Brandshagen} (Brondhaguen), Jobst Dietrich 1659\textendash nach 1716} aber vom Spätherbst 1681 bis zum September 1683 in Kopenhagen aufhielt, % (siehe den in \textit{LSB} III,~3 und III,~4 edierten Briefwechsel zwischen beiden),
wurde die Reinschrift N.~12\textsubscript{3},~\textit{l} frühestens im Herbst 1683 und spätestens vor ihrer Überarbeitung durch Leibniz abgefasst, % d.h. vor Mitte Juni 1684.
welche nicht vor Mitte Juni 1684 begonnen haben dürfte (siehe unten).
Dies wird ferner von einem der in den Textträgern von N.~12\textsubscript{3},~\textit{l} vorliegenden Wasserzeichen bestätigt, das im Leibniz-Nachlass nach heutigem Wissensstand nur für die Jahre 1683 und 1684 belegt ist.
\pend
\pstart%
Um\edlabel{MariottesTod_dafjh-1}
die überarbeitete Fassung der Reinschrift N.~12\textsubscript{3},~\textit{l} (Textschicht \textit{Lil}) zu datieren, erweist sich einerseits als ausschlaggebend, dass Leibniz in der neuen Schlusspassage (S.~\refpassage{LH_37_01_008v_MT-1}{LH_37_01_008v_MT-2}) Mariottes\protect\index{Namensregister}{\textso{Mariotte}, Edme, Seigneur de Chazeuil ca. 1620\textendash1684} Tod (12.~Mai 1684) erwähnt.
Hiervon hat er vermutlich erst nach Mitte Juni 1684 durch die an ihn gerichteten Briefe von C.~Brosseau\protect\index{Namensregister}{\textso{Brosseau}, Christophe 1630\textendash1717} und N.~Douceur\protect\index{Namensregister}{\textso{Douceur}, Noel} erfahren (\textit{LSB} \cite{01245}I,~4 N.~381, S.~468.4; \cite{01246}III,~4 N.~56, S.~119.2);
er selbst weist auf Mariottes Tod erstmals in seinen Briefen an J.-B. Du Hamel\protect\index{Namensregister}{\textso{Du Hamel}, Jean-Baptiste 1624\textendash1706} und Brosseau vom 11. (21.) Juli 1684 hin (\textit{LSB} \cite{01282}III,~4 N.~62, S.~130.12; \cite{01283}I,~4 N.~390, S.~475.4\textendash6).
Demnach dürfte die Überarbeitung von N.~12\textsubscript{3},~\textit{l} (Textschicht \textit{Lil}) 
nicht vor der zweiten Hälfte Juni 1684 begonnen haben.%
\edlabel{MariottesTod_dafjh-2}
Andererseits kann sie aber nicht vor der Anfertigung des Teilkonzeptes N.~12\textsubscript{3},~\textit{L\textsuperscript{2}} abgeschlossen worden sein, das weder vor März 1685 noch viel später als Mitte April desselben Jahres entstanden sein dürfte (siehe unten).
\pend
\newpage
\pstart
Diese Feststellung ermöglicht ferner die Datierung der Stücke N.~12\textsubscript{4} und N.~12\textsubscript{5}, d.h. der Aus\-züge aus Schelhammers\protect\index{Namensregister}{\textso{Schelhammer} (Schelhammerus), Günther Christoph 1649\textendash1716} Abhand\-lung \textit{De auditu}\cite{01204} und Duverneys\protect\index{Namensregister}{\textso{Duverney} (Duvernejus, Duvernaeus), Joseph-Guichard 1648\textendash1730} \textit{Tractatus de organo auditus}.\cite{01203}
Leibniz hat diese Auszüge bei seiner Überarbeitung der Reinschrift N.~12\textsubscript{3},~\textit{l} (Textschicht \textit{Lil}) mit einbezogen.
Hierüber gilt es Folgendes zu bemerken:
\pend
\pstart%
Bei den in N.~12\textsubscript{4} edierten Auszügen aus Schelhammers \textit{De auditu}\cite{01204} (Leiden 1684) geht es vor\-wie\-gend um die von G.~Fracastoro\protect\index{Namensregister}{\textso{Fracastoro} (Fracastorius), Girolamo 1478\textendash1553} in \textit{De sympathia}\cite{01215} (Venedig 1546) abgegebene Erklärung der Schall\-aus\-brei\-tung sowie um den in Helmstedt durchgeführten Versuch zum \glqq Paradoxon\grqq\ der gleich\-förmigen Schall\-ge\-schwin\-dig\-keit.
Leibniz schreibt am 6. (16.) Mai 1684 an F.~Schrader,\protect\index{Namensregister}{\textso{Schrader}, Friedrich 1657\textendash1704} Schelhammers Buch \glqq gesehen\grqq\ zu haben, und geht auf eben beide Themen, die Gegenstand der Auszüge sind, ein (\textit{LSB} III,~4 N.~55,\cite{01224} S.~116.5\textendash16; 117.5\textendash17).
Leibniz verweist auf Schel\-hammers Buch ferner in der überarbeiteten Fassung der Reinschrift N.~12\textsubscript{3},~\textit{l} (Textschicht \textit{Lil}) und nimmt dort auf die in N.~12\textsubscript{4} exzerpierten Stellen erneut Bezug (S.~\refpassage{LH_37_01_007v_w1}{LH_37_01_007v_w2}; \refpassage{LH_37_01_007v_y1}{LH_37_01_007v_y2}).
Im überarbeiteten Schlussteil der Reinschrift % N.~??X\textsubscript{3},~\textit{l} (Textschicht \textit{Lil}) 
erwähnt er das Erscheinen von Schelhammers Buch ausdrücklich, jedoch nicht als unmittelbar vorausgegangenes Ereignis (S.~\refpassage{LH_37_01_008v_novissime-1}{LH_37_01_008v_novissime-2}).
Aus diesen Eckdaten lässt sich schließen, dass N.~12\textsubscript{4} % die Auszüge aus Schelhammers\protect\index{Namensregister}{\textso{Schelhammer} (Schelhammerus), Günther Christoph 1649\textendash1716} Abhandlung 
wahrscheinlich kurz vor dem Brief an Schrader, d.h. im Frühjahr 1684, verfasst wurde, jedenfalls nicht nach der Überarbeitung der Reinschrift N.~12\textsubscript{3},~\textit{l}% (Textschicht \textit{Lil})
, d.h. spätestens in der ersten Hälfte 1685. % , entstanden sind.
\pend
\pstart%
Die in N.~12\textsubscript{5} edierten Auszüge sind der lateinischen Übersetzung von Duverneys\protect\index{Namensregister}{\textso{Duverney} (Duvernejus, Duvernaeus), Joseph-Guichard 1648\textendash1730} Abhandlung über das Gehörorgan (\cite{01203}\textit{Tractatus de organo auditus}, Nürnberg 1684) entnommen.
Von diesen Auszügen rührte zunächst das Teilkonzept N.~12\textsubscript{3},~\textit{L\textsuperscript{2}} her, das Leibniz dann als Vorlage für die Überarbeitung eines umfangreichen Textabschnitts im Schlussteil der Reinschrift N.~12\textsubscript{3},~\textit{l} verwendet hat (S.~\refpassage{LH_37_01_008v_g-1}{LH_37_01_008v_g-2}, Textschicht \textit{Lil}).
Er selbst räumt im neuen Schlussteil % der Reinschrift N.~??X\textsubscript{3},~\textit{l} 
ein, viel von Duverneys Ausführungen profitiert zu haben (S.~\refpassage{LH_37_01_008v_profeci-1}{LH_37_01_008v_profeci-2},~\textit{Lil}).
Die Auszüge aus Duverneys Buch müssen folglich zu dem Zeit\-punkt vorgelegen haben, als Leibniz das Teilkonzept N.~12\textsubscript{3},~\textit{L\textsuperscript{2}} verfasst hat, d.h. frühestens im März 1685 und nicht viel später als Mitte April desselben Jahres (siehe unten).
Dem 1684 veröffentlichten Katalog der Leipziger und Frankfurter Buchmessen entnimmt man ferner, dass die lateinische Fassung von Duverneys Abhandlung über das Gehörorgan zwar bereits (bei ungenauer Angabe des Titels) auf der Ostermesse angekündigt, erst aber auf der Michaelismesse aufgestellt wurde (vgl. \textit{Catalogus universalis}, Osterheft, S.~E3~r\textsuperscript{o}; Herbstheft, S.~A4~v\textsuperscript{o}).\cite{01244}
Daher ist auszuschließen, dass sie vor dem Osterfest (30. März bzw. 2. April) 1684 erschienen war.
Die Aus\-züge aus Duverneys Buch sind demnach zwischen April 1684 und Mitte April 1685 zu datieren.
Da Leibniz ferner im überarbeiteten Schluss\-teil der Reinschrift N.~12\textsubscript{3},~\textit{l} % (Textschicht \textit{Lil})
auf Duverneys Abhandlung als \textit{novissime} erschienen anspielt (S.~\refpassage{LH_37_01_008v_novissime-1}{LH_37_01_008v_novissime-2},~\textit{Lil}), ist zwischen dem Entstehen von N.~12\textsubscript{5} und der Überarbeitung der Reinschrift N.~12\textsubscript{3},~\textit{l} kein weiterer Zeitabstand anzunehmen.
\pend
\pstart%
Die Datierung des Teilkonzeptes N.~12\textsubscript{3},~\textit{L\textsuperscript{2}}, welches auf den Auszügen N.~12\textsubscript{5} beruht und als unmittelbare Vorlage eines umfangreichen Abschnitts des überarbeiteten Schlussteils der Reinschrift N.~12\textsubscript{3},~\textit{l} (S.~\refpassage{LH_37_01_008v_g-1}{LH_37_01_008v_g-2}, Textschicht \textit{Lil}) gedient hat, lässt sich unter Berücksichtigung folgender Umstände bestimmen.
Das Teilkonzept ist auf einem Umschlag überliefert, der die Anschrift von J.\,F. Leibnizens Hand trägt.\protect\index{Namensregister}{\textso{Leibniz} (Leibnütz), Johann Friedrich 1632\textendash1696}
Den dazugehörigen Brief an G.\,W. Leibniz dürfte sein Halbbruder zwischen dem 28. Februar (10. März) und dem 8. (18.) April 1685 gesendet haben (\textit{LSB} I,~4 N.~573;\cite{01319} 574;\cite{01320} 577\cite{01321}).
Auf demselben Umschlag liegen neben dem Teilkonzept N.~12\textsubscript{3},~\textit{L\textsuperscript{2}} auch Hilfsrechnungen von G.\,W. Leibnizens Hand vor, welche in unmittelbarem Zusammenhang mit einem Brief stehen, den G.\,W. Leibniz am 14. (24.) April 1685 an den Rechtsanwalt Q.\,F.\,S. Rivinus\protect\index{Namensregister}{\textso{Rivinus}, Quintus Septimius Florens 1651\textendash1713} 
gesendet hat (\textit{LSB} I,~4 N.~581\cite{01322}).
Hieraus lässt sich schließen, dass das Teilkonzept N.~12\textsubscript{3},~\textit{L\textsuperscript{2}} höchstwahrscheinlich weder vor März 1685 noch viel später als Mitte April desselben Jahres verfasst wurde.
Auch die gesamte Überarbeitung der Reinschrift N.~12\textsubscript{3},~\textit{l} % (Textschicht \textit{Lil}) 
kann folglich nicht vor dieser Zeitspanne abgeschlossen worden sein.
\pend
\newpage
\pstart
Das Gesamtstück N.~12 ist demzufolge in einer Zeitspanne entstanden, die sich von der zweiten Augusthälfte 1681 bis zur ersten Hälfte 1685 erstreckt.
Aus welchem Grund die \textit{dissertatiuncula} N.~12\textsubscript{3} % , \textit{L\textsuperscript{1}} bzw. \textit{l} 
nach der in der zweiten Hälfte 1684 oder der ersten Hälfte 1685 durchgeführten Überarbeitung nicht zur beabsichtigten Veröffentlichung % in den \textit{Acta eruditorum} 
gelangte, ist nicht bekannt.
Aufgrund der in N.~12 behandelten Thematik wird dem Gesamtstück die redaktionelle Überschrift \textit{Cogitationes novae de sono} zugewiesen.
\pend
\newpage % vorläufig
\normalsize%