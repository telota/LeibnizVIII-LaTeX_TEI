%   % !TEX root = ../../VIII,3_Rahmen-TeX_8-1.tex
%  
%   Band VIII, 3		Rubrik STOSS
%
%   Signatur/Tex-Datei:	LH_35_09_16_006
%
%   RK-Nr. 	41176
%
%   \ref{RK41176}
%
%   Überschrift: 	(keine) [De aequilibrio in corporum concursu]
%   
%   Unterrubrik:			Überblick??
%
%   edlabels:			1: LH_37_05_097_naturaictus-1/2
%
%   Diagramme: 		0
%
%
%   NB: 						(Anmerkungen)					??
%
%
%
\selectlanguage{ngerman}
\frenchspacing
%
\begin{ledgroupsized}[r]{120mm}
\footnotesize
\pstart
\noindent\textbf{Überlieferung:}
\pend
\end{ledgroupsized}
%
\begin{ledgroupsized}[r]{114mm}
\footnotesize
\pstart \parindent -6mm
\makebox[6mm][l]{\textit{L}}%
Notiz:
LH~XXXV~9, 16~Bl.~6. 
Ein Zettel (etwa~9~x~5~cm);
rechter und unterer Rand beschnitten.
Eine Seite auf Bl.~6~v\textsuperscript{o};
auf Bl. 6~r\textsuperscript{o} Zeilenanfänge eines abgeschnittenen Texts von Leibnizens Hand:
\textit{Weilen ich v}\textlangle\textendash\textrangle\ \lbrack/\rbrack\ 
\textit{aestimation ein} \textlangle\textendash\textrangle\ \lbrack/\rbrack\ 
\textit{Hertze von der} \textlangle\textendash\textrangle\ \lbrack/\rbrack\ 
\textit{mir den gefallen} \textlangle\textendash\textrangle\ \lbrack/\rbrack\ 
\textit{dieß werck zu be}\textlangle\textendash\textrangle\ \lbrack/\rbrack\ 
\textit{nehmen, mithin} \textlangle\textendash\textrangle\ \lbrack/\rbrack\ 
\textit{in einigen fällen} \textlangle\textendash\textrangle\ \lbrack/\rbrack\ 
\textit{von den Rechnun}\textlangle\textendash\textrangle\ \lbrack/\rbrack\ 
\textit{richtung geben,} \textlangle\textendash\textrangle\ \lbrack/\rbrack\ 
\sout{\textit{lieffe zum Schluß g}}\textlangle\textendash\textrangle\ \lbrack/\rbrack\ 
\textit{oder doch in den}\textlangle\textendash\textrangle\ \lbrack/\rbrack\ 
\textit{Cammer gelangen} \textlangle\textendash\textrangle.
\pend
\end{ledgroupsized}
%
%
\vspace{5mm}
\begin{ledgroup}
\footnotesize
\pstart
\noindent%
\textbf{Datierungsgründe:}
%Kein Wasserzeichen.
%Der Text auf der Vorderseite, vermutlich ein Briefanfang, konnte nicht näher bestimmt werden (Suche im Korpus blieb ohne Erfolg).%
%\pend
%%
%\pstart
%Benutzung des Ausdruckes \textit{vis viva} bislang bekannt für die Zeit ab ?? (ab 1678 oder später). %
%\pend
%%
%\pstart
%\textbf{Titel ändern?} Eigentlich keine Nennung von ,Statik‘, dafür Aussagen über \textit{vis viva et mortua}.
%
Anlass zur vorliegenden Notiz N.~\ref{RK41176} gibt die Feststellung, dass der direkte zentrale Stoß zweier elastischer Körper, deren Geschwindigkeiten sich umgekehrt wie die Massen (\textit{magnitudines}) verhalten, dem Zustand des Gleichgewichts in einem statischen System entspricht.
In einem solchen Stoßfall verhalte es sich nämlich sowohl mit den \glqq lebendigen\grqq\ wie mit den \glqq toten\grqq\ Kräften gleichermaßen.
(Beim Gleichgewicht einer Balkenwaage sind es indes nur die \glqq toten Kräfte\grqq, die sich balancieren, d.h. bloß die Drehmomente).
Hierfür gibt Leibniz die % \textendash\ eher kryptische \textendash\ 
Begründung ab, dass die zwei sich stoßenden Körper \glqq am Anfang lediglich mit ihren toten Kräften gegeneinander kämpfen, obwohl sie auch lebendige Kräfte besitzen,\grqq\
könnten doch diese letzteren nicht \glqq augenblicklich\grqq\ wirken.
\pend%
%
\pstart%
Die Unterscheidung von \textit{vis viva} und \textit{vis mortua} lässt sich bei Leibniz nach heutigem Kenntnisstand bis in die Pariser Zeit zurückverfolgen (vgl. \textit{LSB} III,~1 N.~25, S.~107; 109\,f.;\cite{01374}
VIII,~2 N.~9, S.~119.4;\cite{01375}
N.~12, S.~134.4\textendash7;\cite{01352}
N.~18, S.~162.4\textendash8;\cite{01376}
siehe \textsc{Antognazza} %, M.\,R., \textit{Leibniz}, Cambridge 
2009, S.~314, Anm.~94.)\cite{01381}
Das Textfragment auf der Vorderseite \textendash\ vermutlich ein Briefentwurf, der in Zusammenhang mit Leibnizens Bergwerkunternehmungen im Harz\protect\index{Ortsregister}{Harz} gestanden haben könnte (vgl. das ähnlich anlautende Promemoria vom September 1686 für das Bergamt zu Clausthal\protect\index{Ortsregister}{Clausthal}, \textit{LSB} I,~4 N.~245, S.~284.7\textendash8)\cite{01377} \textendash\ schließt jedoch eine Entstehung der vorliegenden Notiz vor Leibnizens Rückkehr nach Deutschland\protect\index{Ortsregister}{Deutschland (Germania, Duitsland)} Ende 1676 aus.
Ferner ist bekannt, dass Leibniz nicht sofort nach seiner Entdeckung des Kraftmaßes $mv^2$ Anfang 1678 begonnen hat, dieses systematisch als \glqq lebendige Kraft\grqq\ zu bezeichnen (siehe \textsc{Fichant} 1994, S.~64).\cite{01056}
Vielmehr gewann das Begriffspaar \textit{vis viva}~/ \textit{vis mortua} vorwiegend in späteren Jahren an Bedeutung, als Leibniz seine Kräftelehre ausbaute.
In Druckschriften tritt es nach heutigem Wissensstand erstmals im \textit{Tentamen de motuum coelestium causis}, §~10 auf (\textit{AE}, Februar 1689, S.~87f;
\textit{LMG} VI, S.~153;\cite{00543}
siehe \textsc{Bertoloni Meli} 1993, S.~100\,f.).\cite{01357}
In der handschriftlichen Überlieferung kommt es besonders häufig im Briefwechsel der Jahre 1675\textendash1677 zwischen Leibniz und D.~Papin%
\protect\index{Namensregister}{\textso{Papin} (Papinus), Denis 1647\textendash?1712}
vor, in dem sich um die \textit{force vive} und \textit{force morte} ein reger Meinungsstreit zwischen beiden Gelehrten entwickelt.
Die Unterscheidung wird dort in Leibnizens Brief an Papin%
\protect\index{Namensregister}{\textso{Papin} (Papinus), Denis 1647\textendash?1712}
vom 17. November 1695 eingeführt, und zwar wieder im Zusammenhang mit der Deutung des elastischen Stoßes als Gleichgewichtszustand
(vgl. \textit{LSB} III,~6 N.~172, S.~540.15\textendash22;\cite{01378}
siehe auch den Brief an G.\,F. de l'Hospital vom 4./14. Dezember 1696: III,~7 N.~56, S.~215.14\textendash18).\cite{01379}%
\pend%
\pstart%
Bereits im Laufe der 1680er Jahre finden sich aber im Nachlass Texte, die Verwandtschaft mit der vorliegenden Notiz N.~\ref{RK41176} aufweisen.
% , wobei anstelle von \textit{vis} zuweilen der Begriff \textit{potentia} auftritt.
Dies trifft etwa auf die in diesem Band edierten Entwürfe N.~\ref{38538} (Ende Januar 1683 [?] bis Juli 1686 [?]), N.~\ref{RK60323} (Herbst 1688) und N.~\ref{RK60320} % \textit{Principia universalia ad concursus determinandos} anzutreffen ist
(1686 [?] bis Oktober 1687; vgl. dort bes. S.~\refpassage{LH_37_05_097_naturaictus-1}{LH_37_05_097_naturaictus-2})
%
sowie vorwiegend auf den Entwurf \textit{LSB} VI,~4 N.~379 % \textit{De potentiae absolutae conservatione} 
zu, der mutmaßlich im Winter 1689/1690 verfasst wurde (die Datierung beruht auf der Feststellung, dass der Entwurf inhaltlich Leibnizens Konzeption der Dynamik, wie sie im Laufe der Italienreise Gestalt angenommen habe, voraussetze; ebd., S.~2077.12\textendash14).\cite{01380}
Die Begriffe \textit{potentia viva}~/ \textit{potentia mortua} werden auch in diesem letzteren Text im Zusammenhang mit dem Stoß zweier elastischer Körper, deren Geschwindigkeiten sich umgekehrt wie die Massen verhalten, eingeführt und erläutert (ebd., S.~2077.23\textendash2078.9).\cite{01380}
Die anschließende Ausführung über das Zusammenspiel der \glqq lebendigen\grqq\ und \glqq toten\grqq\ Kräfte beider Körper unmittelbar nach dem Stoß (ebd., S.~2078.11\textendash31) entfaltet einen ähnlichen Gedanken wie den, den Leibniz am Ende von N.~\ref{RK41176} als Begründung andeutet.
% (am Anfang \glqq kämpfen\grqq die zwei Körper gegeneinenander lediglich mit ihren toten Kräften, d.h. mit ihren Bewegungsgrößen).
\pend%
%
\pstart%
Ausschlaggebend für die Datierung der vorliegenden Notiz dürfte trotzdem sein, dass die Annahme, von der N.~\ref{RK41176} unmittelbar ausgeht, sich auch für die Aufzeichnung N.~\ref{RK41205} als grundlegend erweist: nämlich, dass die Gesetze des Gleichgewichts sich auf den Fall des Stoßes übertragen ließen.
Das Begriffspaar \textit{vis viva}~/ \textit{vis mortua} tritt in N.~\ref{RK41205} freilich nicht ausdrücklich auf, die Unterscheidung \glqq lebendig~/ tot\grqq\ wird dort aber auf verwandte Begriffe angewendet (\textit{impetus}, \textit{conatus}; vgl. S.~\refpassage{LH_35_09_21_007v_par3-1}{LH_35_09_21_007v_par3-2}).
Auch der in N.~\ref{RK41176} abschließende Gedanke \textendash\ im ersten Augenblick nach dem Stoß \glqq kämpfen\grqq\ die zwei Körper gegeneinander lediglich mit ihren toten Kräften, d.h. mit ihren bloßen Bewegungsgrößen \textendash\ weist Ähnlichkeit mit einer Ausführung in N.~\ref{RK41205} auf (vgl. S.~\refpassage{LH_35_09_21_007v_par1-1}{LH_35_09_21_007v_par1-2}).
Diese inhaltliche Übereinstimmung lässt die Vermutung zu, dass die vorliegende Notiz N.~\ref{RK41176} im Zusammenhang mit der Aufzeichnung N.~\ref{RK41205} und somit in demselben Zeitraum entstanden sein könnte, d.h. zwischen 1680 und 1685 (siehe die Datierungsgründe auf S.~\pageref{LH_35_09_21_007_Datierung}).
Weder eine frühere (ab 1677) noch vor allem eine spätere Datierung lassen sich jedoch ausschließen.%
\pend%
%
%\pstart%
%Vorkommnisse des Begriffspaares, die enge Verwandtschaft mit der vorliegenden Notiz N.~\ref{RK41176} aufweisen, finden sich aber auch früher im Nachlass, wobei anstelle von \textit{vis} zuweilen der Begriff \textit{potentia} auftritt.
%Dies trifft etwa auf den Entwurf \textit{LSB} VI,~4 N.~379 \textit{De potentiae absolutae conservatione} zu, der mutmaßlich im Winter 1689/1690 verfasst wurde (die Datierung beruht auf der Feststellung, dass der Entwurf inhaltlich Leibnizens Konzeption der Dynamik, wie sie im Laufe der Italienreise Gestalt angenommen habe, voraussetze; ebd., S.~2077.12\textendash14).\cite{01380}
%Das Begriffspaar wird auch dort im Zusammenhang mit dem Stoß zweier elastischer Körper eingeführt, deren Geschwindigkeiten sich umgekehrt wie die Massen verhalten
%%: \textit{Potentia absoluta mortua fit ex ductu celeritatis mortuae in molem. Potentia viva ex ductu celeritatis vivae impetu quaesitae. Celeritas mortua potest dici conatus, viva impetus.} 
%(ebd., S.~2077.23\textendash2078.9).\cite{01380}
%Die anschließende Ausführung über das Zusammenspiel der \glqq lebendigen\grqq\ und \glqq toten\grqq\ Kräfte beider Körper unmittelbar nach dem Stoß (ebd., S.~2078.11\textendash31) lässt sich auch als Erläuterung der dunklen Begründung lesen, die Leibniz am Ende von N.~\ref{RK41176} abgibt (am Anfang \glqq kämpfen\grqq die zwei Körper gegeneinenander lediglich mit ihren toten Kräften, d.h. mit ihren Bewegungsgrößen).
%Im Hintergrund dieser Ausführung dürfte wiederum eine Betrachtung des elastischen Verhaltens sich stoßender Körper stehen, wie sie in diesem Band etwa im Entwurf N.~\ref{RK60320} \textit{Principia universalia ad concursus determinandos} anzutreffen ist (vgl. bes. S.~\refpassage{LH_37_05_097_naturaictus-1}{LH_37_05_097_naturaictus-2}??).
%Dieser Text kann auf die Zeitspanne zwischen 1686 (?) und Oktober 1687 datiert werden (siehe die Begründung dort).
%Die Begriffe \textit{vis viva} und \textit{vis mortua} treten auch in weiteren in diesem Band edierten Texten der 1680er Jahre auf, etwa in N.~\ref{RK41205}, N.~\ref{38538} und N.~\ref{RK60323}.%
%\pend%
%%
%\pstart%
%Aus diesen Gründen erweist sich als plausibel, dass auch die vorliegende Notiz N.~\ref{RK41176} im Zeitraum zwischen den Entwürfen \textit{Principia universalia ad concursus determinandos}
%% N.~\ref{RK60320} in diesem Band 
%und \textit{De potentiae absolutae conservatione}
%% \textit{LSB} VI,~4 N.~379\cite{01380} 
%entstand, d.h. zwischen 1686 und dem Winter 1689/1690.
%Eine frühere (ab 1677) oder spätere Datierung lässt sich jedoch nicht ausschließen.%
%\pend
\end{ledgroup}
%
%
\selectlanguage{latin}
\frenchspacing
% \newpage%
\vspace{8mm}
\pstart%
\normalsize%
\noindent%
\lbrack6~v\textsuperscript{o}\rbrack\
%
%
Quando duo corpora%
\protect\index{Sachverzeichnis}{corpora concurrentia}
%
\edtext{concurrunt,
reciprocis}{%
\lemma{concurrunt,}\Bfootnote{\hspace{-0,5mm}%
\textbar~quoad directionem \textit{gestr.}~%
\textbar\ reciprocis%
~\textit{L}}} 
%
magnitudini velocitatibus%
\lbrack,\rbrack%
\protect\index{Sachverzeichnis}{velocitas reciproca magnitudini}
%
in aequilibrio%
\protect\index{Sachverzeichnis}{aequilibrium}
sunt, 
%
\edtext{et perinde res procedit in vi perfecta%
\protect\index{Sachverzeichnis}{vis perfecta}
seu viva,%
\protect\index{Sachverzeichnis}{vis viva}
%
ac in imperfecta%
\protect\index{Sachverzeichnis}{vis imperfecta}
seu mortua,%
\protect\index{Sachverzeichnis}{vis mortua}%
}{%
\lemma{et}\Bfootnote{%
\textit{(1)}~tantum praestat
\textbar~vis \textit{streicht Hrsg.}~%
\textbar\ mortua seu imperfecta,
\textit{(2)}~perinde res \lbrack...\rbrack\ seu mortua,%
~\textit{L}}}
%
impetu%
\protect\index{Sachverzeichnis}{impetus}
acquisito carente.
%
Cujus rei ratio esse videtur,
quod etsi vim habeant vivam,%
\protect\index{Sachverzeichnis}{vis viva}
tamen non nisi mortua%
\protect\index{Sachverzeichnis}{vis mortua} 
%
initio colluctentur,
neque enim alterum in altero
effectum%
\protect\index{Sachverzeichnis}{effectus potentiae vivae}
vivae potentiae%
\protect\index{Sachverzeichnis}{potentia viva} 
%
momento imprimere potest,%
\protect\index{Sachverzeichnis}{momentum temporis}
%
quippe quem ipsummet momento acquirere non potuit.
%
\pend
\count\Bfootins=1200%
\count\Afootins=1200%
\count\Cfootins=1200
%
%
% % % %    Ende tes Textes auf Bl. 6v
%
%