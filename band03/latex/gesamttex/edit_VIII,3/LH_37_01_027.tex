%   ##
%
%   Band VIII, 3 N.~??A08
%   Signatur/Tex-Datei: LH_37_01_027
%   RK-Nr. 38541
%   Überschrift: [Aus und zu Joachim Jungius, Harmonica] [[ex: De sonorum intervallis]]
%   Datierung: [1683--1685 (ex: Ende Juni 1678--Ende Mai 1687)]
%   WZ: Bl. 27: RK-Wz 1555
%   SZ: (keins)
%   Bilddateien (PDF): (keine)
%
%
\begin{ledgroupsized}[r]{120mm}
\footnotesize
\pstart
\noindent\textbf{Überlieferung:}
\pend
\end{ledgroupsized}
\begin{ledgroupsized}[r]{114mm}
\footnotesize
\pstart \parindent -6mm
\makebox[6mm][l]{\textit{L}}%
Auszüge mit Bemerkungen aus
\cite{01266}J.~\textsc{Jungius}, \textit{Harmonica}
\lbrack Hamburg 1678\rbrack:
LH~XXXVII~1 Bl.~27.
Ein Blatt~2\textsuperscript{o};
ein Wasserzeichen;
geringfügiger Textverlust an den Rändern: Papiererhaltungsmaßnahmen.
Eine voll beschriebene 
Seite auf Bl.~27~r\textsuperscript{o}
und drei Zeilen auf Bl.~27~v\textsuperscript{o};
der Rest ist leer.
\pend
\end{ledgroupsized}
% NICHT BEI GERLAND 1906
%
\vspace*{5mm}
\begin{ledgroup}
\footnotesize
\pstart
\noindent\footnotesize%
\textbf{Datierungsgründe:}
Jungius'\protect\index{Namensregister}{\textso{Jungius}, Joachim 1587\textendash1657} \cite{01266}\textit{Harmonica}, eine musiktheoretische Abhandlung, erschien posthum 1678 in Hamburg\protect\index{Ortsregister}{Hamburg}, von J.~Vagetius\protect\index{Namensregister}{\textso{Vagetius}, Johannes 1633\textendash1691} herausgegeben.
Ein Jahr später wurde sie,
samt Jungius' \cite{01268}\textit{Isagoge phytoscopica},
der ebendort veröffentlichten Ausgabe von Jungius' \cite{01267}\textit{Praecipuae opiniones physicae ... ex recensione et distinctione Martinii Fogelii}\protect\index{Namensregister}{\textso{Fogel} (Vogelius, Vogel), Martin 1634\textendash1675} beigebunden.
H.~Siver\protect\index{Namensregister}{\textso{Siver} (Siverus), Heinrich 1626\textendash1691} übersandte Leibniz mit seinem Brief vom 6. (16.) Juni 1678 ein Exemplar der frisch gedruckten \textit{Harmonica} und kündigte ihm zugleich die bevorstehende Veröffentlichung der \textit{Isagoge phytoscopica} an (\cite{01269}\textit{LSB} II,~1 \lbrack2006\rbrack\ N.~179,
 S.~626.17\textendash18; 628.14\textendash15).
Bei Sivers Sendung handelt es sich vermutlich um das Exemplar Nm–A 418 der GWLB Hannover, welches etliche Marginalien von Leibnizens Hand enthält, die zuweilen die vorliegenden Auszüge N.~13 betreffen.
\pend
\pstart
Leibniz berichtete Mitte Juli 1678 dem Herausgeber des \textit{Journal des Sçavans}, J.\,P. de La Roque,\protect\index{Namensregister}{\textso{La Roque} (Larroque), Jean Paul de, gest. 1691} über diese Veröffentlichungen (\cite{01270}\textit{LSB} III,~2 N.~186, S.~473).
Im folgenden Heft der Zeitschrift erschien eine davon herrührende anonyme Ankündigung:
\textit{Harmonica et Phytoscopica, scripta posthuma Ioachimi Iungii Hamburg. 1678. Ce Iungius estoit sans contredit un des plus grands Mathematiciens et Philosophes de son temps et un des plus habiles hommes que l’Allemagne ayt jamais eu. Il y a pourtant esté peu connu pendant sa vie, et beaucoup moins ailleurs, parce qu’il n’a jamais voulu rien publier de son vivant, ne pouvant pas se contenter soy-même sur ses propres Ouvrages. Quand nous aurons receu ce livre de Hambourg, où il doit estre publié, nous ferons part de ce qu’il contient} (\cite{01271}\textit{JS}, 22. August 1678, Pariser Ausgabe: S.~342). % ; Amsterdamer Ausgabe: S.~364
Seine Mitteilung an La Roque\protect\index{Namensregister}{\textso{La Roque} (Larroque), Jean Paul de, gest. 1691} sowie deren
(angeblich untreue) 
Wiedergabe im \textit{Journal des Sçavans} erwähnt Leibniz in einer späteren Aufzeichnung über die 1681 veröffentlichte zweite Ausgabe von Jungius' \textit{Logica Hamburgensis} (\cite{01272}\textit{LSB} VI,~4 N.~233, S.~1121.1\textendash7).
Aus diesen Umständen lässt sich schließen, dass Leibniz die Auszüge N.~13 bereits ab Ende Juni 1678 verfasst haben könnte.
\pend
\pstart
In einem Brief an Vagetius\protect\index{Namensregister}{\textso{Vagetius}, Johannes 1633\textendash1691} von Ende Mai 1687, in dem erneut von der Veröffentlichung des Jun\-gius-Nach\-las\-ses\protect\index{Namensregister}{\textso{Jungius}, Joachim 1587\textendash1657} die Rede ist, nimmt Leibniz ausdrücklich auf die \textit{Harmonica} Bezug und zeigt hierbei, die Abhandlung eigenständig zu kennen (\cite{01273}\textit{LSB} II,~2 N.~47, S.~210.17\textendash20).
Spätestens zu diesem Zeitpunkt müssten also Leibnizens kommentierte Auszüge aus Jungius' \textit{Harmonica} vorgelegen haben.
\pend
\pstart
Es gilt dennoch zu bemerken, dass das im Textträger von N.~13 anzutreffende Wasserzeichen im Leibniz-Nachlass nach heutigem Kenntnisstand lediglich für die Jahre 1683\textendash1685 belegt ist.
Diese Zeit\-spanne ist demnach als die am meisten wahrscheinliche Datierung von N.~13 zu betrachten, wobei eine frühere oder spätere Datierung zwischen Ende Juni 1678 und Ende Mai 1687 nicht auszuschließen ist.
Auszuschließen ist hingegen aufgrund des Wasserzeichens eine aus inhaltlichen Gründen in Frage kommende Datierung der Auszüge auf die Zeit von Leibnizens Briefwechsel mit C.~Henfling\protect\index{Namensregister}{\textso{Henfling}, Conrad 1648\textendash1716} über Fragen der Musiklehre (1705\textendash1711), wenngleich sich Leibniz dort explizit auf Jungius' \textit{Harmonica} bezieht.
Handschriften, die inhaltlich im Zusammenhang mit N.~13 stehen, sind im Konvolut LBr 390 enthalten, welches u.a. den Briefwechsel mit Henfling überliefert.
\pend
\end{ledgroup}
\newpage%
%
% \vspace*{8mm}
%
%
\count\Bfootins=1000
\count\Afootins=1000
\count\Cfootins=1000
%
%
\pstart%
\normalsize%
\noindent%
%
\lbrack27~r\textsuperscript{o}\rbrack\ % Blatt 27r
%
\edtext{\textit{\textso{Phthongus \protect\index{Sachverzeichnis}{phthongus}}(\phantom)\hspace*{-1.2mm}%
uniformis sonus\protect\index{Sachverzeichnis}{sonus uniformis}%
\phantom(\hspace*{-1.2mm})
est,
cujus initium
\edtext{fini unisonum}{%
\lemma{fini}\Bfootnote{%
\hspace{-0,5mm}\textbar~\textit{est} \textit{gestr.}~%
\textbar\ \textit{unisonum},\protect\index{Sachverzeichnis}{unisonus}%
~\textit{L}}},}}{%
\lemma{\textit{\textso{Phthongus}} \lbrack...\rbrack\ \textit{unisonum}}\Cfootnote{%
\cite{01266}J.\,\textsc{Jungius}, \textit{Harmonica} \lbrack hrsg.\,v.\,J.\,\textsc{Vagetius}, Hamburg 1678\rbrack, n.\,1.
Der unpaginierte Druck ist in der Regel gebunden an: \cite{01267}\textsc{Ders.}, \textit{Praecipuae opiniones physicae ... ex recensione et distinctione M. Fogelii} \lbrack hrsg.\,v.\,J.\,\textsc{Vagetius}\rbrack, Hamburg 1679.}}
%
\protect\index{Sachverzeichnis}{diphthongus}alias\textso{ Diphthongus.}
\edtext{Per\textso{ sonum }intelligimus Uniformem.\protect\index{Sachverzeichnis}{sonus uniformis}}{%
\lemma{Per \lbrack...\rbrack\ Uniformem}\Cfootnote{%
\cite{01266}\textsc{Jungius}, \textit{Harmonica}, n.~3.}}
%
\pend%
%
\pstart%
\edtext{\textit{\textso{Intervallum,}\protect\index{Sachverzeichnis}{intervallum} \pgrk{diásthma}
est sonorum duorum differentia\protect\index{Sachverzeichnis}{differentia sonorum}}
(+\phantom)\hspace*{-1.2mm}~%
ratio\protect\index{Sachverzeichnis}{ratio sonorum} potius
seu logarithmorum\protect\index{Sachverzeichnis}{differentia logarithmorum} differentia~%
\phantom(\hspace*{-1.2mm}\lbrack+\rbrack)
\textit{secundum acumen\protect\index{Sachverzeichnis}{acumen} et gravitatem},\protect\index{Sachverzeichnis}{gravitas}}{%
\lemma{\textit{\textso{Intervallum}} \lbrack...\rbrack\ \textit{gravitatem}}\Cfootnote{%
\cite{01266}a.a.O., n.~3. Zitat mit Auslassungen.% \hspace*{10mm}
}}
%
ut octava\protect\index{Sachverzeichnis}{octava}\lbrack,\rbrack\ quinta.\protect\index{Sachverzeichnis}{quinta}
\pend%
%
\pstart%
\edtext{\textso{Soni consoni }\protect\index{Sachverzeichnis}{sonus consonus}%
(\phantom)\hspace*{-1.2mm}%
\textso{dissoni}\protect\index{Sachverzeichnis}{sonus dissonus}%
\phantom(\hspace*{-1.2mm})
auditui\protect\index{Sachverzeichnis}{auditus} grati,\protect\index{Sachverzeichnis}{sonus gratus}
(\phantom)\hspace*{-1.2mm}%
ingrati\protect\index{Sachverzeichnis}{sonus ingratus}%
\phantom(\hspace*{-1.2mm}),
seu quorum intervalla\textso{ concinna }\protect\index{Sachverzeichnis}{intervallum concinne}%
(\phantom)\hspace*{-1.2mm}%
\textso{inconcinna}\protect\index{Sachverzeichnis}{intervallum inconcinne}%
\phantom(\hspace*{-1.2mm}).}{%
\lemma{\textso{Soni} \lbrack...\rbrack\ (\phantom)\hspace*{-1.2mm}\textso{inconcinna}\phantom(\hspace*{-1.2mm})}\Cfootnote{%
\cite{01266}a.a.O., n.~8\,f. u. 5\,f.% \hspace*{10mm}
}}
%
\pend%
%
\pstart%
%
\edtext{\textit{\textso{Experientia:}%
\protect\index{Sachverzeichnis}{experientia}
\edlabel{LH_37_01_027r_hypoth-1}duorum corporum materia\protect\index{Sachverzeichnis}{materia} convenientium}
\edtext{\lbrack\textit{et}\rbrack}{%
\lemma{\textit{et}}\Bfootnote{%
\textit{erg. Hrsg. nach Vorlage}}}
\textit{crassitie\protect\index{Sachverzeichnis}{crassities} aequalium}\lbrack,\rbrack\
\textit{gravius sonat id quod longius est caeteris paribus.
Ut in tibiis\protect\index{Sachverzeichnis}{tibia}
quo longius ab\textso{ Hyph\-ol\-mio }\protect\index{Sachverzeichnis}{hypholmium}%
seu\textso{ lingula }\protect\index{Sachverzeichnis}{cingulum}distat foramen
per quod aer sonorus\protect\index{Sachverzeichnis}{aer sonorus} exit;
in chordis\protect\index{Sachverzeichnis}{chorda}
quo longius corpora chordam terminantia distant.}}{%
\lemma{\textit{\textso{Experientia}} \lbrack...\rbrack\ \textit{distant}}\Cfootnote{%
\cite{01266}a.a.O., n.~12\textendash14. Zitat mit Auslassungen.% \hspace*{28mm}
}}
%
\pend%
%
\pstart%
\edtext{\textit{\textso{Hypothesis:}\protect\index{Sachverzeichnis}{hypothesis}
quae proportio longitudinis sonantium praesertim chordarum,%
\protect\index{Sachverzeichnis}{chorda sonans}\protect\index{Sachverzeichnis}{longitudo chordae}
ea est sonorum.}\edlabel{LH_37_01_027r_hypoth-2}\protect\index{Sachverzeichnis}{proportio sonorum}%
}{%
\lemma{\textit{\textso{Hypothesis}} \lbrack...\rbrack\ \textit{sonorum}}\Cfootnote{%
\cite{01266}a.a.O., n.~15. Zitat mit Aus\-las\-sung. Gemeint ist hierbei eine umgekehrte Proportionalität, wie die unmittelbar 
vor\-aus\-ge\-hen\-de \glqq Erfahrung\grqq\ deutlich macht.%
% An der \glqq Hypothese\grqq\ eines umgekehrten Verhältnisses zwischen der Länge des schwingenden Körpers und der Höhe des er\-zeug\-ten Tones knüpft auch N.~??A09 an.
}}
%
\pend%
%
\pstart%
\edtext{\textit{\textso{Monochordum }\protect\index{Sachverzeichnis}{monochordum}est instrumentum\protect\index{Sachverzeichnis}{instrumentum}
in quo sonorum proportio\protect\index{Sachverzeichnis}{proportio sonorum}
vel una chorda vel pluribus unisonis\protect\index{Sachverzeichnis}{chorda unisona} exploratur.%
\textso{ Magadium }\protect\index{Sachverzeichnis}{magadium}%
seu\textso{ suppositorium }\protect\index{Sachverzeichnis}{suppositorium}%
est corpus chordam terminans estque fixum vel mobile.}}{%
\lemma{\textit{\textso{Monochordum}} \lbrack...\rbrack\ \textit{mobile}}\Cfootnote{%
% \cite{01266}\textsc{Jungius}, \textit{Harmonica}, n.~16\,f. 
a.a.O., n.~16\,f. Zitat mit Auslassungen.}}
%
\pend%
%
\pstart%
\edtext{\textso{Octava }\protect\index{Sachverzeichnis}{octava}%
$1 : 2$
\quad
\textso{ Quinta }\protect\index{Sachverzeichnis}{quinta}%
$2 : 3$
\quad
\textso{ Quarta }\protect\index{Sachverzeichnis}{quarta}%
$3 : 4$
\quad
\textso{ Tertia major }\protect\index{Sachverzeichnis}{tertia major}%
$4 : 5$
\quad
\textso{ Tertia minor }\protect\index{Sachverzeichnis}{tertia minor}%
\edlabel{LH_37_01_027r_sodfvh-1}$5 : 6$}{%
\lemma{\textso{Octava} \lbrack...\rbrack\ $5 : 6$\,}\Cfootnote{%
\cite{01266}a.a.O., n.~18\textendash22.}}%
\edtext{}{%
{\xxref{LH_37_01_027r_sodfvh-1}{LH_37_01_027r_sodfvh-2}}%
{\lemma{$5 : 6$}\Bfootnote{%
\textit{(1)}~(+\phantom)\hspace*{-1.2mm}~Ex harum consonantiarum\protect\index{Sachverzeichnis}{consonantia} intervallis
\textit{(2)}~Intervalla%
~\textit{L}}}}
%
\pend%
%
\pstart%
Intervalla\edlabel{LH_37_01_027r_sodfvh-2}%
%
\textso{ componi }dicuntur,\protect\index{Sachverzeichnis}{compositio intervallorum}
cum rationes eorum componuntur,\protect\index{Sachverzeichnis}{ratio intervallorum}
(+\phantom)\hspace*{-1.2mm}~%
sive cum logarithmi sibi adduntur~\protect\index{Sachverzeichnis}{additio logarithmorum}%
\phantom(\hspace*{-1.2mm}+)%
\textso{ auferri }a sese
cum rationes auferuntur
(+\phantom)\hspace*{-1.2mm}~%
hoc est cum quaeritur
Ratio rationis,
seu cum logarithmus subtrahitur a logarithmo~\protect\index{Sachverzeichnis}{subtractio logarithmorum}%
\phantom(\hspace*{-1.2mm}+)
et differentiae intervallorum\protect\index{Sachverzeichnis}{differentia intervallorum}
dicuntur esse eorum
\edtext{intervalla.\protect\index{Sachverzeichnis}{intervallum intervallorum}
\edtext{Ita intervallum octavae\protect\index{Sachverzeichnis}{octava} et quintae\protect\index{Sachverzeichnis}{quinta}
est quarta.\protect\index{Sachverzeichnis}{quarta}}{%
\lemma{Ita \lbrack...\rbrack\ quarta}\Cfootnote{%
\cite{01266}a.a.O., n.~27.}} %%%
\edtext{Intervallum quintae\protect\index{Sachverzeichnis}{quinta} et quartae\protect\index{Sachverzeichnis}{quarta}
est\textso{ Tonus major,}\protect\index{Sachverzeichnis}{tonus major} $2 : 3\, \squaredots\, 3 : 4$ est $8 : 9.$}{%
\lemma{Intervallum \lbrack...\rbrack\ $8 : 9$\,}\Cfootnote{%
\cite{01266}a.a.O., n.~30.\hspace{-2mm}}} %%%%%%%%%%%%%%%%%%
\edtext{Quartae\protect\index{Sachverzeichnis}{quarta} et tertiae minoris\protect\index{Sachverzeichnis}{tertia minor}
est\textso{ Tonus minor,}\protect\index{Sachverzeichnis}{tonus minor} $3 : 4\, \squaredots\, 5 : 6$ est $9 : 10.$}{%
\lemma{Quartae \lbrack...\rbrack\ $9 : 10$\,}\Cfootnote{%
\cite{01266}a.a.O., n.~31.\hspace{-2mm}}} %%%%
\edtext{Quartae\protect\index{Sachverzeichnis}{quarta} et tertiae majoris\protect\index{Sachverzeichnis}{tertia major}
est\textso{ semitonium naturale }\protect\index{Sachverzeichnis}{semitonium naturale}$15 : 16$}{%
\lemma{Quartae \lbrack...\rbrack\ $15 : 16$\,}\Cfootnote{%
\cite{01266}a.a.O., n.~32.}} %%%%
(+\phantom)\hspace*{-1.2mm}~%
tertiae majoris\protect\index{Sachverzeichnis}{tertia major} et minoris\protect\index{Sachverzeichnis}{tertia minor}
intervallum est apotome minor\protect\index{Sachverzeichnis}{apotome minor} $24 : 25$~%
\lbrack\phantom(\hspace*{-1.2mm}+).\rbrack}{%
\lemma{intervalla.}\Bfootnote{%
\textit{(1)}~Ita intervallum octavae et quintae est quarta
\textit{(2)}~Interv. oct. et quint. est quarta, intervallum oct. et quart. est quinta, quint. et 3t. maj. est 3tia minor,
\textit{(3)}~Ita intervallum \lbrack...\rbrack\ est quarta
\textit{(a)}~, quintae et
\textbar~tertiae majoris est $\langle$tertia$\rangle$ \textit{erg.}~%
\textbar\ mi$\langle$nor$\rangle$
\textit{(b)}~et
\textit{(c)}~. Intervallum quintae \lbrack...\rbrack\ minor $24 : 25$%
~\textit{L}}}
Hinc componendo vicissim,\protect\index{Sachverzeichnis}{compositio intervallorum}
et semitonium\protect\index{Sachverzeichnis}{semitonium} ac tonos\protect\index{Sachverzeichnis}{tonus}
sumendo tanquam elementa,\protect\index{Sachverzeichnis}{elementum}
%
\edtext{}{%
{\xxref{LH_37_01_027r_dsuhcp-1}{LH_37_01_027r_dsuhcp-2}}%
{\lemma{Tertia \lbrack...\rbrack\ majore}\Cfootnote{%
\cite{01266}a.a.O., n.~34.}}}%
\edlabel{LH_37_01_027r_dsuhcp-1}Tertia minor\protect\index{Sachverzeichnis}{tertia minor} est%
\textso{ sesquitonus naturalis,}\protect\index{Sachverzeichnis}{sesquitonus naturalis}
seu
\edtext{componitur ex semitonio\protect\index{Sachverzeichnis}{semitonium}
et tono majore;\protect\index{Sachverzeichnis}{tonus major}\edlabel{LH_37_01_027r_dsuhcp-2}}{%
\lemma{componitur}\Bfootnote{%
\hspace{-0,5mm}ex
\textit{(1)}~majore tono et
\textit{(2)}~semitonio et tono majore;%
~\textit{L}}}
%
\edtext{Tertia major\protect\index{Sachverzeichnis}{tertia major}
est\textso{ ditonus naturalis }\protect\index{Sachverzeichnis}{ditonus naturalis}%
seu componitur ex tono minore\protect\index{Sachverzeichnis}{tonus minor}
et majore;\protect\index{Sachverzeichnis}{tonus major}}{%
\lemma{Tertia \lbrack...\rbrack\ majore}\Cfootnote{%
\cite{01266}a.a.O., n.~33.}}
%
\edtext{quarta\protect\index{Sachverzeichnis}{quarta}
ex semitonio\protect\index{Sachverzeichnis}{semitonium} et ditono;\protect\index{Sachverzeichnis}{ditonus}}{{%
\lemma{quarta}\Bfootnote{%
\hspace{-0,5mm}ex
\textit{(1)}~ditono
\textit{(2)}~semitonio
\textit{(3)}~semitonio et ditono;%
~\textit{L}}%
}{%
\lemma{quarta \lbrack...\rbrack\ ditono}\Cfootnote{%
\cite{01266}a.a.O., n.~35.}}}
%
\edtext{quinta\protect\index{Sachverzeichnis}{quinta}
ex sesquitono\protect\index{Sachverzeichnis}{sesquitonus} et ditono;\protect\index{Sachverzeichnis}{ditonus}}{%
\lemma{quinta \lbrack...\rbrack\ ditono}\Cfootnote{%
\cite{01266}a.a.O., n.~37.}}
%
\edtext{Octava\protect\index{Sachverzeichnis}{octava} ex quarta\protect\index{Sachverzeichnis}{quarta} et
quinta\protect\index{Sachverzeichnis}{quinta}}{%
\lemma{Octava \lbrack...\rbrack\ quinta}\Cfootnote{%
\cite{01266}a.a.O., n.~40.}}%
\edtext{. Vel aliter enuntiando: ex}{%
\lemma{quinta.}\Bfootnote{%
\textit{(1)}~Seu ex
\textit{(2)}~Vel aliter enuntiando: ex%
~\textit{L}}}
%
semitonio\protect\index{Sachverzeichnis}{semitonium} et tono majore\protect\index{Sachverzeichnis}{tonus major}
sesquitonus\protect\index{Sachverzeichnis}{sesquitonus} seu tertia minor;\protect\index{Sachverzeichnis}{tertia minor}
ex tono minore\protect\index{Sachverzeichnis}{tonus minor} et majore,\protect\index{Sachverzeichnis}{tonus major}
ditonus\protect\index{Sachverzeichnis}{ditonus} seu tertia major;\protect\index{Sachverzeichnis}{tertia major}
\edtext{ex semitonio,\protect\index{Sachverzeichnis}{semitonium} tono minore\protect\index{Sachverzeichnis}{tonus minor}
et tono majore,\protect\index{Sachverzeichnis}{tonus major} quarta;\protect\index{Sachverzeichnis}{quarta}}{%
\lemma{ex semitonio \lbrack...\rbrack\ quarta}\Cfootnote{%
\cite{01266}a.a.O., n.~36.}}
%
\edtext{ex semitonio,\protect\index{Sachverzeichnis}{semitonium} tono minore,\protect\index{Sachverzeichnis}{tonus minor}
duobus majoribus,\protect\index{Sachverzeichnis}{tonus major} quinta;\protect\index{Sachverzeichnis}{quinta}}{%
\lemma{ex semitonio \lbrack...\rbrack\ quinta}\Cfootnote{%
\cite{01266}a.a.O., n.~38.}}
%
ex duobus semitoniis,\protect\index{Sachverzeichnis}{semitonium} duobus tonis minoribus,\protect\index{Sachverzeichnis}{tonus minor}
et tribus tonis majoribus\protect\index{Sachverzeichnis}{tonus major} Octava.\protect\index{Sachverzeichnis}{octava}
\pend%
%
\pstart%
Dixi semitonium naturale,\protect\index{Sachverzeichnis}{semitonium naturale}
(\phantom)\hspace*{-1.2mm}%
ac proinde sesquitonus naturalis,\protect\index{Sachverzeichnis}{sesquitonus naturalis}
ditonus naturalis\protect\index{Sachverzeichnis}{ditonus naturalis}%
\phantom(\hspace*{-1.2mm})
quia licet revera semitonium\protect\index{Sachverzeichnis}{semitonium} non sit dimidius
\edtext{tonus,\protect\index{Sachverzeichnis}{tonus dimidius}
(\phantom)\hspace*{-1.2mm}%
qui deberet esse intervallum sonorum,\protect\index{Sachverzeichnis}{intervallum sonorum}
quorum ratio\protect\index{Sachverzeichnis}{ratio sonorum}}{%
\lemma{tonus,}\Bfootnote{%
\textit{(1)}~(\phantom)\hspace*{-1.2mm} 
\textit{(a)}~tonus
\textit{(b)}~ratio
\textit{(2)}~(\phantom)\hspace*{-1.2mm}qui deberet \lbrack...\rbrack\ quorum ratio%
~\textit{L}}}
esset subduplicata seu dimidii logarithmi,\protect\index{Sachverzeichnis}{logarithmus}
rationis sonorum\protect\index{Sachverzeichnis}{ratio sonorum}
quorum intervallum\protect\index{Sachverzeichnis}{intervallum sonorum}
est tonus\protect\index{Sachverzeichnis}{tonus}%
\phantom(\hspace*{-1.2mm})
tamen cum
\edtext{\lbrack sumatur\rbrack}{%
\lemma{sumatur}\Bfootnote{%
\textit{erg. \mbox{Hrsg.}}}}
pro eo concinnitatis\protect\index{Sachverzeichnis}{concinnitas}
causa,
et quia natura\protect\index{Sachverzeichnis}{natura} in ipsum
\edtext{canendo ferimur,}{%
\lemma{canendo}\Bfootnote{%
\textit{(1)}~feritur\lbrack\textit{!}\rbrack\,
\textit{(2)}~ferimur,%
~\textit{L}}}
pro dimidio tono\protect\index{Sachverzeichnis}{tonus dimidius} substituitur.
\edtext{Quidam%
\edtext{ sesquiditonum\protect\index{Sachverzeichnis}{sesquiditonus}
vocant semiditonum,\protect\index{Sachverzeichnis}{semiditonus}%
}{\lemma{Quidam \lbrack...\rbrack\ semiditonum}\Cfootnote{%
Die Bezeichnung \textit{sesquiditonus} kommt in einer Liste der Intervallen bei M.~\textsc{Mersenne}, \textit{Harmonicorum libri XII}, liber III de instrumentis harmonicis, prop.~22 (Paris 1648, S.~131) vor.\cite{01336}
}}
sed non apte.}{%
\lemma{Quidam}\Bfootnote{%
\hspace{-0,5mm}sesquiditonum \lbrack...\rbrack\ non apte.%
~\textit{erg.~L}}}
\pend%
%
%\newpage
\pstart%
(+\phantom)\hspace*{-1.2mm}~%
Patet et plura adhuc intervalla\protect\index{Sachverzeichnis}{intervallum} vacua manere,
ita nondum componitur semitonium\protect\index{Sachverzeichnis}{semitonium} cum tono minore,\protect\index{Sachverzeichnis}{tonus minor}
fit
\edtext{enim $27 : 32$
quod est paulo inconcinnius.\protect\index{Sachverzeichnis}{intervallum inconcinne}}{%
\lemma{enim}\Bfootnote{%
\textit{(1)}~$24 : 25$
\textit{(2)}~$27 : 32$ quod est paulo inconcinnius
\textbar~est Apotome m \textit{erg. u. gestr.}~%
\textbar~.%
~\textit{L}}}~%
% \edtext{ }{\lemma{\textit{Über} est paulo inconcinnius\textit{, gestrichen und abbrechend:}%
% }\Afootnote{\footnotesize{%
% est Apotome\protect\index{Sachverzeichnis}{apotome} m % \textsuperscript{\lbrack a\rbrack}%
% \newline%
% \newline%
% \textsuperscript{\lbrack a\rbrack} est Apotome m: Möglicherweise Anspielung auf S.~\refpassage{}{}.%
% }}}%
% % % % ANFANG DER GROSSEN STREICHUNG
\edtext{\phantom(\hspace*{-1.2mm}+)%
\textso{ }%
\edtext{\textso{Sexta major }\protect\index{Sachverzeichnis}{sexta major}%
(\phantom)\hspace*{-1.2mm}%
intervallum\protect\index{Sachverzeichnis}{intervallum}
octavae\protect\index{Sachverzeichnis}{octava} et tertiae minoris\protect\index{Sachverzeichnis}{tertia minor}%
\phantom(\hspace*{-1.2mm})
$3 : 5.$}{%
\lemma{\textso{Sexta} \lbrack...\rbrack\ $3 : 5$\,}\Cfootnote{%
\cite{01266}\textsc{Jungius}, \textit{Harmonica},
 n.~107 und n.~110.}}%%%%
\textso{ }%
\edtext{\textso{Sexta minor }\protect\index{Sachverzeichnis}{sexta minor}%
(\phantom)\hspace*{-1.2mm}%
intervallum\protect\index{Sachverzeichnis}{intervallum}
octavae\protect\index{Sachverzeichnis}{octava} et tertiae majoris\protect\index{Sachverzeichnis}{tertia major}%
\phantom(\hspace*{-1.2mm})
$5 : 8.$}{%
\lemma{\textso{Sexta} \lbrack...\rbrack\ $5 : 8$\,}\Cfootnote{%
\cite{01266}a.a.O., n.~106 und n.~109.\hspace{10mm}}} %%%%
\edtext{Ambae consonae.\protect\index{Sachverzeichnis}{intervallum consonum}}{%
\lemma{Ambae consonae}\Cfootnote{%
\cite{01266}a.a.O., n.~108.}} %%%%
Quarta\protect\index{Sachverzeichnis}{quarta} et 3tia minor\protect\index{Sachverzeichnis}{tertia minor}
additae faciunt sextam minorem.\protect\index{Sachverzeichnis}{sexta minor}}{%
\lemma{\phantom(\hspace*{-1.2mm}+)}\Bfootnote{% inconcinnius.~
% gestrichene Variante:
\textit{(1)}~(+\phantom)\hspace*{-1.2mm}~%
\textso{Nomina,} octavae, quintae etc. forte ex ordine\protect\index{Sachverzeichnis}{ordo intervallorum}
apud veteres,\protect\index{Sachverzeichnis}{veteres}
de quo inquirendum:
an sic \textso{octava} $1 : 2,$
\textlbrackdbl\,\textso{septima} quid? quaerendum.\,\textrbrackdbl\
(\phantom)\hspace*{-1.2mm}\textso{Sexta} major $3 : 5,$ minor $5 : 8$\phantom(\hspace*{-1.2mm})
\textso{quinta} $2 : 3,$ \textso{quarta} $3 : 4,$
\textso{tertia} major $4 : 5,$ minor $5 : 6,$
\textlbrackdbl\,\textso{secunda}\protect\index{Sachverzeichnis}{secunda} quid? quaerendum an
$6 : 7$ secunda major,\protect\index{Sachverzeichnis}{secunda major}
$7 : 8$ minor.\protect\index{Sachverzeichnis}{secunda minor}\,\textrbrackdbl\
\textso{Prima} sive Tonus major $8 : 9,$ minor $9 : 10.$\,\phantom(\hspace*{-1.2mm}+)\,\,
% gültige Variante:
\textit{(2)}~\textso{Sexta major} \lbrack...\rbrack\ sextam minorem.%
~\textit{L}}}
% % % % ENDE DER GROSSEN STREICHUNG
\pend%
%
\count\Bfootins=900
\count\Afootins=900
\count\Cfootins=900
\pstart%
\edtext{\textit{\textso{Hypothesis.}\protect\index{Sachverzeichnis}{hypothesis}
Toni\protect\index{Sachverzeichnis}{tonus} eodem
\edtext{durare intervallo\protect\index{Sachverzeichnis}{intervallum} judicantur,}{%
\lemma{durare}\Bfootnote{%
\textit{(1)}~dicuntur
\textit{(2)}~\textit{intervallo judicantur,}%
~\textit{L}}}
quorum eadem est proportio.}\protect\index{Sachverzeichnis}{proportio intervallorum}}{%
\lemma{\textit{\textso{Hypothesis}} \lbrack...\rbrack\ \textit{proportio}}\Cfootnote{%
\cite{01266}a.a.O., n.~58. Zitat mit Auslassung.}}
\pend%
%
\pstart%
\edtext{Si soni unius intervalli sint graviores sonis alterius intervalli\protect\index{Sachverzeichnis}{intervallum} respondentibus,
% \edtext{}{%
% \lemma{soni}\Bfootnote{%
% \textit{(1)}~responde
% \textit{(2)}~unius intervalli \lbrack...\rbrack\ intervalli respondentibus,%
% ~\textit{L}}}
\edtext{acutiorem\protect\index{Sachverzeichnis}{sonus acutus}
(\phantom)\hspace*{-1.2mm}%
\lbrack graviorem\protect\index{Sachverzeichnis}{sonus gravis}\rbrack%
\phantom(\hspace*{-1.2mm})
unius cum acutiore\protect\index{Sachverzeichnis}{sonus acutus}
(\phantom)\hspace*{-1.2mm}%
graviore\protect\index{Sachverzeichnis}{sonus gravis}%
\phantom(\hspace*{-1.2mm})}{%
\lemma{acutiorem}\Bfootnote{%
\textit{(1)}~cum ac
\textit{(2)}~(\phantom)\hspace*{-1.2mm}graviorem\phantom(\hspace*{-1.2mm})
\textit{(3)}~unius cum acutiore
\textit{(4)}~(\phantom)\hspace*{-1.2mm}gratiores\phantom(\hspace*{-1.2mm})
\textit{(5)}~(\phantom)\hspace*{-1.2mm}~%
\textbar~gratiorem \textit{ändert Hrsg.}~%
\textbar~\phantom(\hspace*{-1.2mm}) unius cum acutiore (\phantom)\hspace*{-1.2mm}graviore\phantom(\hspace*{-1.2mm})%
~\textit{L}}}
alterius comparando,
intervallum illud dicitur intervallo hoc\textso{ inferius.}\protect\index{Sachverzeichnis}{intervallum inferius}}{%
\lemma{Si soni \lbrack...\rbrack\ \textso{inferius}}\Cfootnote{%
\cite{01266}a.a.O., n.~59.}}
%
\pend%
%
\pstart%
\edtext{\textit{\textso{Intervallum ex }\edtext{\textso{intervallis componi }\protect\index{Sachverzeichnis}{compositio intervallorum}dicitur,}{%
\lemma{\textso{intervallis}}\Bfootnote{%
\textit{(1)}~\textso{componitur},
\textit{(2)}~\textit{\textso{componi} dicitur,}%
~\textit{L}}}
si idem sonus
\edtext{sit acutior\protect\index{Sachverzeichnis}{sonus acutus}}{%
\lemma{sit}\Bfootnote{%
\textit{(1)}~gravior
\textit{(2)}~\textit{acutior}%
~\textit{L}}}
inferioris\protect\index{Sachverzeichnis}{intervallum inferius}
et gravior\protect\index{Sachverzeichnis}{sonus gravis} superioris}
intervalli\protect\index{Sachverzeichnis}{intervallum superius}}{%
\lemma{\textit{\textso{Intervallum}} \lbrack...\rbrack\ intervalli}\Cfootnote{%
\cite{01266}a.a.O., n.~60.}}
%
(+\phantom)\hspace*{-1.2mm}~%
imo et aliter,
ut si Tonum majorem\protect\index{Sachverzeichnis}{tonus major} et semitonium\protect\index{Sachverzeichnis}{semitonium} componas,
quod facit sesquitonum.\protect\index{Sachverzeichnis}{sesquitonus}~%
\phantom(\hspace*{-1.2mm}+)
\pend%
%
\pstart%
\edtext{\textit{\textso{Hypothesis.}\protect\index{Sachverzeichnis}{hypothesis}
Intervalla superparticularis proportionis\protect\index{Sachverzeichnis}{proportio superparticularis}
magis habentur concinna\protect\index{Sachverzeichnis}{intervallum concinne}
quam superpartientis,\protect\index{Sachverzeichnis}{proportio superpartiens}
sunt enim consonis intervallis\protect\index{Sachverzeichnis}{intervallum consonum} magis cognata}}{%
\lemma{\textit{\textso{Hypothesis}} \lbrack...\rbrack\ \textit{cognata}}\Cfootnote{%
\cite{01266}a.a.O., n.~70\,f. Zitat mit Auslassungen.}}
%
\edtext{(+\phantom)\hspace*{-1.2mm}~%
seu citius pervenitur ad mensuram communem\protect\index{Sachverzeichnis}{mensura communis intervallorum}%
~\phantom(\hspace*{-1.2mm}+).}{%
\lemma{(+\protect\vphantom)~seu \lbrack...\rbrack\ communem~\protect\vphantom(+)}\Cfootnote{%
Offenbar Anspielung auf die Koinzidenztheorie.
Vgl. N.~12\textsubscript{1}, S.~\refpassage{LH_37_01_018r_koinzidenztheorie-1}{LH_37_01_018r_koinzidenztheorie-2};
N.~12\textsubscript{3}, S.~\refpassage{LH_37_01_007v-koinzidenztheorie-1}{LH_37_01_007v-koinzidenztheorie-2}.}}
%(+\phantom)\hspace*{-1.2mm}~%
%seu citius pervenitur ad mensuram communem\protect\index{Sachverzeichnis}{mensura communis intervallorum}%
%~\phantom(\hspace*{-1.2mm}+).
\pend%
%\newpage% % % % Rein vorläufig !!!!
%
\pstart%
%Quoniam%
%\edlabel{LH_37_01_027_auf/ab-1}%
%\edtext{}{{\xxref{LH_37_01_027_auf/ab-1}{LH_37_01_027_auf/ab-2}}%
%\lemma{Quoniam \lbrack...\rbrack\ excedunt}\Cfootnote{%
%Die hier angegebenen Verhältnisse unter Intervallen gelten nur dann,
%wenn die betreffenden Intervalle \glqq absteigend\grqq\ dargestellt werden,
%d.h. mit Hilfe unechter Brüche,
%bei denen der Zählerwert größer ist als der Nennerwert.
%Stellt man die Intervalle umgekehrt \glqq aufsteigend\grqq\ dar,
%d.h. mit Hilfe echter Brüche,
%so gelten vielmehr die umgekehrten Verhältnisse.}}
Quoniam%
\edlabel{LH_37_01_027_auf/ab-1}%
\edtext{}{{\xxref{LH_37_01_027_auf/ab-1}{LH_37_01_027_auf/ab-2}}%
\lemma{Quoniam \lbrack...\rbrack\ contra}\Cfootnote{%
Mithilfe arithmetischer Brüche können Intervalle in zwei entgegengesetzten Arten dargestellt werden:
entweder (nach der herkömmlich-pythagoreischen, auf Saitenlängen beruhenden Methode) \textit{aufsteigend},
d.h. mithilfe echter Brüche, bei denen der Zähler kleiner als der Nenner ist;
oder umgekehrt (nach der in der Frühneuzeit entstandenen, auf Frequenzen beruhenden Methode) \textit{absteigend},
d.h. mithilfe unechter Brüche.
% , bei denen der Zähler größer als der Nenner ist.
In dieser Passage sowie auch später im Text wechselt Leibniz unvermittelt von einer Darstellungsweise zur anderen.}}
%
\edtext{\textit{intervallum\protect\index{Sachverzeichnis}{intervallum compositum} ex
\edtext{duobus semitoniis naturalibus \protect\index{Sachverzeichnis}{semitonium naturale} compositum}{%
\lemma{duobus}\Bfootnote{%
\textit{(1)}~tonis
\textit{(2)}~\textit{semitoniis}
\textbar~\textit{naturalibus} \textit{erg.}~%
\textbar\ \textit{compositum}%
~\textit{L}}}
majus est tono majore},}{%
\lemma{\textit{intervallum} \lbrack...\rbrack\ \textit{majore}}\Cfootnote{%
\cite{01266}\textsc{Jungius}, \textit{Harmonica},
n.~72.}}\protect\index{Sachverzeichnis}{tonus major}
%
\edtext{(\phantom)\hspace*{-1.2mm}%
ideoque et minore\protect\index{Sachverzeichnis}{tonus minor}%
\phantom(\hspace*{-1.2mm}),}{%
\lemma{(\phantom)\hspace*{-1.2mm}ideoque et minore\protect\index{Sachverzeichnis}{tonus minor}\phantom(\hspace*{-1.2mm})}\Cfootnote{%
\cite{01266}a.a.O., n.~73.}}
%
hinc
\edtext{\textit{excessus\protect\index{Sachverzeichnis}{excessus} toni\protect\index{Sachverzeichnis}{tonus} super}
\edtext{\textit{semitonium}\protect\index{Sachverzeichnis}{semitonium} \lbrack\textit{naturale}\rbrack\ dicitur}{%
\lemma{\textit{semitonium}}\Bfootnote{%
\hspace{-0,5mm}\textbar~(\phantom)\hspace*{-1.2mm}%
qui idem est cum intervallo tertiae majoris\protect\index{Sachverzeichnis}{tertia major}
et minoris\protect\index{Sachverzeichnis}{tertia minor}%
\phantom(\hspace*{-1.2mm}) \textit{erg.~u. gestr.}~%
\textbar\ \textit{naturale} \textit{erg. Hrsg. nach Vorlage}~%
\textbar\ dicitur%
~\textit{L}}}%
\textso{ \textit{Apotome},}}{%
\lemma{\textit{excessus} \lbrack...\rbrack\ \textit{\textso{Apotome}}\,}\Cfootnote{%
\cite{01266}a.a.O., n.~76.}}
%
quae\protect\index{Sachverzeichnis}{apotome} utique minor est semitonio naturali.\protect\index{Sachverzeichnis}{semitonium naturale}
Unde et
\edtext{hoc dicitur\textso{ semitonium majus,}}{%
\lemma{hoc \lbrack...\rbrack\ \textso{majus}\,}\Cfootnote{%
\cite{01266}a.a.O., n.~75.}}\protect\index{Sachverzeichnis}{semitonium majus}
%
at
\edtext{Apotome dicitur\textso{ semitonium minus.}}{%
\lemma{Apotome \lbrack...\rbrack\ \textso{minus}\,}\Cfootnote{%
\cite{01266}a.a.O., n.~79.}}\protect\index{Sachverzeichnis}{semitonium minus}
%
\edtext{Est
\edtext{autem\textso{ Apotome major }\protect\index{Sachverzeichnis}{apotome major}%
excessus\protect\index{Sachverzeichnis}{excessus}}{%
\lemma{autem}\Bfootnote{%
\hspace{-0,5mm}\textso{Apotome}
\textit{(1)}~vel \textso{major} excessus toni ma
\textit{(2)}~\textso{major}
\textit{(a)}~vel
\textit{(b)}~vel \textso{minor}
\textit{(c)}~excessus~%
~\textit{L}}}
\edtext{Toni majoris,\protect\index{Sachverzeichnis}{tonus major}}{%
\lemma{Toni}\Bfootnote{%
\textit{(1)}~majoris
\textit{(2)}~minoris
\textit{(3)}~majoris,%
~\textit{L}}}
supra semitonium naturale;\protect\index{Sachverzeichnis}{semitonium naturale}%
\textso{ Apotome minor }\protect\index{Sachverzeichnis}{apotome minor}vero est excessus\protect\index{Sachverzeichnis}{excessus}
\edtext{Toni minoris\protect\index{Sachverzeichnis}{tonus minor}
super semitonium,\protect\index{Sachverzeichnis}{semitonium}
quae idem est cum intervallo\protect\index{Sachverzeichnis}{intervallum}
tertiae majoris\protect\index{Sachverzeichnis}{tertia major} et minoris.\protect\index{Sachverzeichnis}{tertia minor}%
}{%
\lemma{Toni}\Bfootnote{%
\textit{(1)}~majoris
\textit{(2)}~minoris
\textbar~super semitonium \lbrack...\rbrack\ et minoris \textit{erg.}~%
\textbar~.%
~\textit{L}}}%
}{%
\lemma{Est \lbrack...\rbrack\ minoris}\Cfootnote{%
\cite{01266}a.a.O., n.~77\,f.}}
%
\edtext{Apotome Minor\protect\index{Sachverzeichnis}{apotome minor} est $24 : 25.$
Major\protect\index{Sachverzeichnis}{apotome major} est $128 : 135,$
unde Minor est concinnior.\protect\index{Sachverzeichnis}{intervallum concinne}}{%
\lemma{Apotome \lbrack...\rbrack\ concinnior}\Cfootnote{%
\cite{01266}a.a.O., n.~81\textendash83.\hspace{15mm}}}
%
\edtext{(+\phantom)\hspace*{-1.2mm}~%
Deberet ergo semitonium naturale\protect\index{Sachverzeichnis}{semitonium naturale} dici\textso{ maximum,}
Apotome major,\protect\index{Sachverzeichnis}{apotome major}\textso{ medium,}
minor,\protect\index{Sachverzeichnis}{apotome minor}\textso{ minimum.}\,%
\phantom(\hspace*{-1.2mm}+)}{%
\lemma{(+\phantom)\hspace*{-1.2mm}~Deberet}\Bfootnote{%
\hspace{-0,5mm}ergo \lbrack...\rbrack\ minor, \textso{minimum.}\,\phantom(\hspace*{-1.2mm}+)
\textit{erg.~L}}}%
\edtext{}{\lemma{\textit{Am Rand, umkreist:}}%
\Afootnote{Nonnullis Apotome\protect\index{Sachverzeichnis}{apotome}
dicitur semitonium\textso{ chromaticum,}\protect\index{Sachverzeichnis}{semitonium chromaticum}
quibusve Mutum.\protect\index{Sachverzeichnis}{semitonium mutum}
Vid. infra.\textsuperscript{\lbrack a\rbrack}\vspace{-0.2em}\\%
\newline%
\footnotesize{
\textsuperscript{\lbrack a\rbrack} infra: S.~\refpassage{LH_37_01_027_chromat-1}{LH_37_01_027_chromat-2}.\vspace{-1.8em}%
}}}
\pend%
\count\Bfootins=1100
\count\Afootins=1100
\count\Cfootins=1100
%
\pstart%
% \lbrack(\phantom)\hspace*{-1.2mm}\rbrack
Notandum quod ex calculo\protect\index{Sachverzeichnis}{calculus} facile
demonstrari
%
\edtext{}{%
{\xxref{LH_37_01_027_cuwidz-1}{LH_37_01_027_cuwidz-2}}%
{\lemma{\textit{sex} \lbrack...\rbrack\ \textit{excedere}}\Cfootnote{%
\cite{01266}a.a.O., n.~84.}}}%
\edtext{potero \edlabel{LH_37_01_027_cuwidz-1}\textit{sex}}{%
\lemma{potero}\Bfootnote{%
\hspace{-0,5mm}\textbar~\phantom(\hspace*{-1.0mm}) \textit{streicht Hrsg.}~%
\textbar\ \textit{sex}%
~\textit{L}}}
\textit{tonos majores\protect\index{Sachverzeichnis}{tonus major} octavam\protect\index{Sachverzeichnis}{octava} excedere},\edlabel{LH_37_01_027_cuwidz-2}
%
\edtext{\textit{tres ditonos naturales\protect\index{Sachverzeichnis}{ditonus naturalis} octavam non} attingere,}{%
\lemma{\textit{tres} \lbrack...\rbrack\ attingere}\Cfootnote{%
\cite{01266}a.a.O., n.~87.}}
%
\edtext{multo minus ergo sex toni minores\protect\index{Sachverzeichnis}{tonus minor}
\edtext{octavam\protect\index{Sachverzeichnis}{octava} explebunt.}{%
\lemma{octavam}\Bfootnote{%
\hspace{-0,5mm}\textbar~non \textit{gestr.}~%
\textbar\ explebunt.%
~\textit{L}}}%
}{\lemma{multo \lbrack...\rbrack\ explebunt}\Cfootnote{%
\cite{01266}a.a.O., n.~85.}}
%
\edtext{\textit{Tria semitonia naturalia}\protect\index{Sachverzeichnis}{semitonium naturale}
\edtext{seu majora\protect\index{Sachverzeichnis}{semitonium majus}}{%
\lemma{seu}\Bfootnote{%
\hspace{-0,5mm}majora \textit{erg.~L}}}
\textit{sesquitonum\protect\index{Sachverzeichnis}{sesquitonus} excedunt}.%
%
}{\lemma{\textit{Tria} \lbrack...\rbrack\ \textit{excedunt}}\Cfootnote{%
\cite{01266}a.a.O., n.~88.}}
%
(+\phantom)\hspace*{-1.2mm}~%
Tres
\edtext{\lbrack Apotomae\rbrack}{%
\lemma{Apotome}\Bfootnote{\textit{L~ändert Hrsg.}}}
minores\protect\index{Sachverzeichnis}{apotome minor} sesquitonum\protect\index{Sachverzeichnis}{sesquitonus} non
\edtext{explent,
$25^3 : 24^3$ aeq. $15625 : 13824$}{%
\lemma{explent}\Bfootnote{%
\textit{(1)}~$24^3 : 25^3$ aeq. $13824 : 15625$
\textit{(2)}~$25^3 : 24^3$ aeq. $15625 : 13824$%
~\textit{L}}}
quod est $\displaystyle1\frac{1}{7 + \frac{1217}{1861}}$
\protect\rule[-5mm]{0mm}{10mm}adeoque minus 
\edtext{quam $6 : 5$ seu $\displaystyle1\frac{1}{5}.$}{%
\lemma{quam}\Bfootnote{%
\textit{(1)}~$6 : 5$ seu
\textit{(2)}~$6 : 5$
\textit{(3)}~$6 : 5$ seu $1 + \displaystyle1\frac{1}{5}$
\textit{(4)}~$6 : 5$ seu $\displaystyle1\frac{1}{5}.$%
~\textit{L}}}
Possunt et addi
\edtext{inter se apotome major\protect\index{Sachverzeichnis}{apotome major}\lbrack,\rbrack\
minor\protect\index{Sachverzeichnis}{apotome minor} \lbrack et\rbrack\ semitonium naturale;\protect\index{Sachverzeichnis}{semitonium naturale}
item duae \lbrack apotomae\rbrack\ majores\protect\index{Sachverzeichnis}{apotome major}
cum semitonio,\protect\index{Sachverzeichnis}{semitonium naturale} vel duae}{%
\lemma{inter se}\Bfootnote{%
\textit{(1)}~duae apotomae et unum
\textit{(2)}~apotome major minor
\textbar~et \textit{erg. Hrsg.}~%
\textbar\ semitonium naturale;
\textit{(a)}~et quidem duae apotomae vel similes
\textit{(b)}~item duae
\textbar~apotome \textit{ändert Hrsg.}~%
\textbar\ majores cum semitonio, vel duae%
~\textit{L}}}
%
apotomae minores\protect\index{Sachverzeichnis}{apotome minor}
cum semitonio,\protect\index{Sachverzeichnis}{semitonium naturale}
vel duae apotomae minores\protect\index{Sachverzeichnis}{apotome minor}
et una major,\protect\index{Sachverzeichnis}{apotome major}
vel contra.%
\edlabel{LH_37_01_027_auf/ab-2}
Sed haec calculo\protect\index{Sachverzeichnis}{calculus} persequi nunc non vacat.
Adhibitis logarithmis,\protect\index{Sachverzeichnis}{logarithmus}
vel \edlabel{LH_37_01_027r_linealogarithmicedivisa_litx-1}%
\edtext{linea logarithmice divisa\protect\index{Sachverzeichnis}{linea logarithmice divisa}%
\edlabel{LH_37_01_027r_linealogarithmicedivisa_litx-2}}{%
\lemma{linea logarithmice divisa}\Cfootnote{%
Damit ist wohl dasselbe wie die \textit{linea musica} gemeint,
von der unten die Rede ist
(S.~\refpassage{LH_37_01_027r_linealogarithmicedivisa_kjrx-1}{LH_37_01_027r_linealogarithmicedivisa_kjrx-2};
\refpassage{LH_37_01_027r_linealogarithmicedivisa_awgr-1}{LH_37_01_027r_linealogarithmicedivisa_awgr-2}).}}
minor erit labor.~%
\phantom(\hspace*{-1.2mm}+)
\pend%
% \newpage% % % % % Rein vorläufig !!!!
%
\pstart%
(+\phantom)\hspace*{-1.2mm}~%
Notandum ex consonantiarum primariarum\protect\index{Sachverzeichnis}{consonantia primaria}
intervallis,\protect\index{Sachverzeichnis}{intervallum} quomodo caetera
\edtext{deriventur:
primo intervalla\protect\index{Sachverzeichnis}{intervallum}
inter proximas:\protect\index{Sachverzeichnis}{consonantia proxima}
inter oct.\protect\index{Sachverzeichnis}{octava} et quint.\protect\index{Sachverzeichnis}{quinta}
est quarta,\protect\index{Sachverzeichnis}{quarta}}{%
\lemma{deriventur:}\Bfootnote{%
\textit{(1)}~octava 
\textit{(2)}~intervallum oct. et quint. est 
\textit{(3)}~primus
\textit{(4)}~primo intervalla inter proximas: 
\textit{(a)}~oct. et quint.
\textit{(b)}~inter oct. et quint. est quarta,%
~\textit{L}}}
inter quint.\protect\index{Sachverzeichnis}{quinta} et quart.\protect\index{Sachverzeichnis}{quarta}
est tonus major,\protect\index{Sachverzeichnis}{tonus major}
inter quart.\protect\index{Sachverzeichnis}{quarta} et tert. maj.\protect\index{Sachverzeichnis}{tertia major} est
\edtext{semiton.\protect\index{Sachverzeichnis}{semitonium}%
\textso{ inter tert. maj. et 3t. minorem intervallum est Apotome minor. NB.}%
\protect\index{Sachverzeichnis}{tertia major}%
\protect\index{Sachverzeichnis}{tertia minor}%
\protect\index{Sachverzeichnis}{apotome minor}
Deinde intervalla\protect\index{Sachverzeichnis}{intervallum}
per unum saltum:\protect\index{Sachverzeichnis}{saltus}}{%
\lemma{semiton.}\Bfootnote{%
\textit{(1)}~Per saltum
\textit{(2)}~\textso{inter tert.} \lbrack...\rbrack\ intervalla per
\textbar~unum \textit{erg.}~%
\textbar\ saltum:%
~\textit{L}}}
inter oct.\protect\index{Sachverzeichnis}{octava} et quartam\protect\index{Sachverzeichnis}{quarta}
est quinta,\protect\index{Sachverzeichnis}{quinta}
inter quint.\protect\index{Sachverzeichnis}{quinta} et tert. maj.\protect\index{Sachverzeichnis}{tertia major}
est tertia minor,\protect\index{Sachverzeichnis}{tertia minor}
inter quart.\protect\index{Sachverzeichnis}{quarta} et 3t. minor.\protect\index{Sachverzeichnis}{tertia minor}
est tonus minor.\protect\index{Sachverzeichnis}{tonus minor}
At intervalla per duplicem saltum\protect\index{Sachverzeichnis}{saltus duplex} erunt
inter oct.\protect\index{Sachverzeichnis}{octava} et tert. maj.\protect\index{Sachverzeichnis}{tertia major} est
\edtext{sexta minor;\protect\index{Sachverzeichnis}{sexta minor}}{%
\lemma{sexta}\Bfootnote{%
\textit{(1)}~major;
\textit{(2)}~minor;%
~\textit{L}}}
inter
\edtext{quint.\protect\index{Sachverzeichnis}{quinta} et 3t.}{%
\lemma{quint.}\Bfootnote{%
\hspace{-0,5mm}et
\textit{(1)}~4t.
\textit{(2)}~3t.%
~\textit{L}}}
min.\protect\index{Sachverzeichnis}{tertia minor} est
\edtext{tertia major.\protect\index{Sachverzeichnis}{tertia major}
Denique intervallum\protect\index{Sachverzeichnis}{intervallum}}{%
\lemma{tertia}\Bfootnote{%
\hspace{-0,5mm}major
\textit{(1)}~, inter.
\textit{(2)}~. Interva
\textit{(3)}~. Denique intervallum%
~\textit{L}}}
per triplicem saltum\protect\index{Sachverzeichnis}{saltus triplex}
est non nisi unicum inter oct.\protect\index{Sachverzeichnis}{octava}
et tert. minorem,\protect\index{Sachverzeichnis}{tertia minor}
quod est
\edtext{sexta major.\protect\index{Sachverzeichnis}{sexta major}}{%
\lemma{sexta}\Bfootnote{%
\textit{(1)}~minor.
\textit{(2)}~major.%
~\textit{L}}}
Atque ita ex quatuor consonantiis primariis\protect\index{Sachverzeichnis}{consonantia primaria} invicem subtractis,
secundum omnes combinationes possibiles,\protect\index{Sachverzeichnis}{combinatio possibilis}
nihil aliud quam sextam majorem,\protect\index{Sachverzeichnis}{sexta major}
sextam minorem,\protect\index{Sachverzeichnis}{sexta minor}
tonum majorem,\protect\index{Sachverzeichnis}{tonus major}
tonum minorem,\protect\index{Sachverzeichnis}{tonus minor}
semitonium naturale,\protect\index{Sachverzeichnis}{semitonium naturale}
et apotomen minorem\protect\index{Sachverzeichnis}{apotome minor} habemus.
Quae omnia intervalla satis concinna\protect\index{Sachverzeichnis}{intervallum concinne}
\edtext{sunt.~%
\phantom(\hspace*{-1.2mm}+)
\newline%
\indent%
\edtext{\textso{Comma }\protect\index{Sachverzeichnis}{comma}%
est intervallum\protect\index{Sachverzeichnis}{intervallum}
inter Tonum majorem\protect\index{Sachverzeichnis}{tonus major}
et minorem,\protect\index{Sachverzeichnis}{tonus minor}}{%
\lemma{\hspace{-1mm}\textso{Comma} \lbrack...\rbrack\ minorem}\Cfootnote{%
\cite{01266}\textsc{Jungius}, \textit{Harmonica}, n.~117.\hspace{-3mm}}}%
}{%
\lemma{sunt.~\phantom(\hspace*{-1.2mm}+)}\Bfootnote{%
\textit{(1)}~\textit{Comma} denique \textit{est intervallum quo tonus major excedit}
\textit{(2)}~\textso{Comma} est \lbrack...\rbrack\ majorem et minorem,%
~\textit{L}}}
$80 : 81.$
\pend%
\count\Bfootins=1000
\count\Afootins=1000
\count\Cfootins=1000
%
\pstart%
\edtext{\textit{\textso{Experientia.}\protect\index{Sachverzeichnis}{experientia}
Comma\protect\index{Sachverzeichnis}{comma}
est intervallum sensile.}\protect\index{Sachverzeichnis}{intervallum sensile}}{%
\lemma{\hspace{-1mm}\textit{\textso{Experientia}} \lbrack...\rbrack\ \textit{sensile}}\Cfootnote{%
\cite{01266}a.a.O., n.~132. Zitat mit Auslassung.}}
%
\edtext{\textit{Utraque Sexta\protect\index{Sachverzeichnis}{sexta major}\protect\index{Sachverzeichnis}{sexta minor} est
\edtext{intervallum}{%
\lemma{intervallum}\Bfootnote{\textit{erg.~L}}}
consonum.}\protect\index{Sachverzeichnis}{intervallum consonum}}{%
\lemma{\textit{Utraque} \lbrack...\rbrack\ \textit{consonum}}\Cfootnote{%
\cite{01266}a.a.O., n.~108.}}%
%
\textso{ }%
\edtext{\textit{\textso{Quartula}}\textso{ }\protect\index{Sachverzeichnis}{quartula}%
est \textit{quarta pars commatis},}{%
\lemma{\textit{\textso{Quartula}} \lbrack...\rbrack\ \textit{commatis}}\Cfootnote{%
\cite{01266}a.a.O., n.~118.}}
seu ratio in qua consistit est subquadruplicata rationis\protect\index{Sachverzeichnis}{ratio subquadruplicata}
in qua consistit comma.\protect\index{Sachverzeichnis}{comma}
\pend%
%
\pstart%
\edtext{\textso{Tonus medius }\protect\index{Sachverzeichnis}{tonus medius}%
est qui tantum distat a majore\protect\index{Sachverzeichnis}{tonus major}
quantum a minore,\protect\index{Sachverzeichnis}{tonus minor}
(\phantom)\hspace*{-1.2mm}%
nempe ab utroque duobus quartulis,\protect\index{Sachverzeichnis}{quartula}%
\phantom(\hspace*{-1.2mm})}{%
\lemma{\textso{Tonus} \lbrack...\rbrack\ quartulis}\Cfootnote{%
\cite{01266}a.a.O., n.~119\textendash121.}}
%
adeoque est dimidius ditonus.\protect\index{Sachverzeichnis}{ditonus dimidius}
\edtext{(+\phantom)\hspace*{-1.2mm}~%
Posset dici Tonus reformatus.\protect\index{Sachverzeichnis}{tonus reformatus}~%
\phantom(\hspace*{-1.2mm}+)}{%
\lemma{(+\phantom)\hspace*{-1.2mm}~Posset}\Bfootnote{%
\hspace{-0,5mm}dici Tonus reformatus.~\phantom(\hspace*{-1.2mm}+)
\textit{erg.~L}}}%
\pend%
%
\count\Bfootins=1100
\count\Afootins=1100
\count\Cfootins=1100
\pstart%
\edtext{\textit{\textso{Semitonium reformatum }\protect\index{Sachverzeichnis}{semitonium reformatum}%
componitur ex semitonio naturali\protect\index{Sachverzeichnis}{semitonium naturale}
et quartula.}\protect\index{Sachverzeichnis}{quartula}}{%
\lemma{\textit{\textso{Semitonium}} \lbrack...\rbrack\ \textit{quartula}}\Cfootnote{%
\cite{01266}a.a.O., n.~122. Zitat mit Auslassungen.}}
%
\edtext{(+\phantom)\hspace*{-1.2mm}~%
Cur non potius est \lbrack semitonium medium\rbrack.~\protect\index{Sachverzeichnis}{semitonium medium}%
\phantom(\hspace*{-1.2mm}+)}{%
\lemma{(+\phantom)\hspace*{-1.2mm}~Cur}\Bfootnote{%
\hspace{-0,5mm}non potius est
\textbar~semitonius medius \textit{ändert Hrsg.}~%
\textbar~.~\phantom(\hspace*{-1.2mm}+)
\textit{erg.~L}}}%
%
\textso{ }\edlabel{LH_37_01_027r_veulbvq-1}%
\textit{\textso{Apotome reformata}}\textso{ }\protect\index{Sachverzeichnis}{apotome reformata}%
seu \textit{semitonium chromaticum}\protect\index{Sachverzeichnis}{semitonium chromaticum reformatum}
\edtext{\textit{reformatum} est
\lbrack \textit{intervallum}\rbrack\
\textit{quod cum semitonio reformato\protect\index{Sachverzeichnis}{semitonium reformatum} componit}}{%
\lemma{\textit{reformatum}}\Bfootnote{%
\textit{(1)}~componitur ex
\textit{(2)}~est
\textbar~\textit{intervallum} \textit{erg. Hrsg. nach Vorlage}~%
\textbar\ \textit{quod cum semitonio reformato componit}%
~\textit{L}}}
\textit{tonum}%
%
\edtext{}{%
{\xxref{LH_37_01_027r_veulbvq-1}{LH_37_01_027r_veulbvq-2}}%
{\lemma{\textit{\textso{Apotome}} \lbrack...\rbrack\ \textit{Medium.}}\Cfootnote{%
\cite{01266}a.a.O., n.~123\,f. Zitat mit Auslassungen.}}}
\edtext{\textit{Medium.\edlabel{LH_37_01_027r_veulbvq-2}}\protect\index{Sachverzeichnis}{tonus medius}
%
\edtext{}{%
{\xxref{LH_37_01_027r_xzjop-1}{LH_37_01_027r_xzjop-2}}%
{\lemma{Sesquitonus \lbrack...\rbrack\ reformato}\Cfootnote{%
\cite{01266}a.a.O., n.~125.\hspace*{4mm}}}}% \hspace*{4mm}
%
\edlabel{LH_37_01_027r_xzjop-1}Sesquitonus\protect\index{Sachverzeichnis}{sesquitonus reformatus}}{%
\lemma{\textit{Medium.}}\Bfootnote{%
\hspace{-0,5mm}\textbar~Ditonus legitimus dicitur qui aequatur duobus tonis mediis. \textit{gestr.}~%
\textbar\ Sesquitonus%
~\textit{L}}}
reformatus fit ex tono medio\protect\index{Sachverzeichnis}{tonus medius}
(\phantom)\hspace*{-1.2mm}%
seu reformato\protect\index{Sachverzeichnis}{tonus reformatus}%
\phantom(\hspace*{-1.2mm})
et semitonio reformato.\edlabel{LH_37_01_027r_xzjop-2}%
\protect\index{Sachverzeichnis}{semitonium reformatum}%
%
\textso{ }\edtext{\textso{Quarta reformata }\protect\index{Sachverzeichnis}{quarta reformata}%
fit ex ditono\protect\index{Sachverzeichnis}{ditonus}
et semitonio reformato;\protect\index{Sachverzeichnis}{semitonium reformatum}}{%
\lemma{\textso{Quarta} \lbrack...\rbrack\ reformato}\Cfootnote{%
\cite{01266}a.a.O., n.~126.}}% \hspace*{4mm}
%
\textso{ }\edtext{\textit{\textso{quinta reformata }\protect\index{Sachverzeichnis}{quinta reformata}%
ex tribus
\edtext{tonis mediis\protect\index{Sachverzeichnis}{tonus medius}
et semitonio reformato\protect\index{Sachverzeichnis}{semitonium reformatum}}{%
\lemma{tonis}\Bfootnote{%
\textit{(1)}~majoribus et mediis
\textit{(2)}~\textit{mediis et semitonio reformato}%
~\textit{L}}}}}{%
\lemma{\textit{\textso{quinta}} \lbrack...\rbrack\ \textit{reformato}}\Cfootnote{%
\cite{01266}a.a.O., n.~127. Zitat mit Aus\-lassungen.}}
%
\edtext{vel ex ditono\protect\index{Sachverzeichnis}{ditonus}
et semitonio reform.\protect\index{Sachverzeichnis}{semitonium reformatum}}{%
\lemma{vel}\Bfootnote{%
\hspace{-0,5mm}ex ditono et semitonio reform.
\textit{erg.~L}}}
%
Ex his sesquitonus reformatus\protect\index{Sachverzeichnis}{sesquitonus reformatus}
quartula\protect\index{Sachverzeichnis}{quartula} deficit
a sesquitono legitimo\protect\index{Sachverzeichnis}{sesquitonus legitimus}
(+\phantom)\hspace*{-1.2mm}~%
calculare hinc licebit quis ille;
an ipsa tertia minor?\protect\index{Sachverzeichnis}{tertia minor}%
~\phantom(\hspace*{-1.2mm}\lbrack+\rbrack)
%
\edtext{\textit{Quarta reformata\protect\index{Sachverzeichnis}{quarta reformata}
legitimam\protect\index{Sachverzeichnis}{quarta legitima}
superat quartula};\protect\index{Sachverzeichnis}{quinta}}{%
\lemma{\textit{Quarta} \lbrack...\rbrack\ \textit{quartula}}\Cfootnote{%
\cite{01266}a.a.O., n.~137. Zitat mit Auslassung.}}
%
\edtext{}{%
{\xxref{LH_37_01_027_vcurlw-1}{LH_37_01_027_vcurlw-2}}%
{\lemma{\textit{quinta} \lbrack...\rbrack\ \textit{quartula}}\Cfootnote{%
\cite{01266}a.a.O., n.~138. Zitat mit Auslassung.}}}%
\edtext{\edlabel{LH_37_01_027_vcurlw-1}\textit{quinta reformata\protect\index{Sachverzeichnis}{quinta reformata}
a legitima}\protect\index{Sachverzeichnis}{quinta legitima}}{%
\lemma{\textit{quinta}}\Bfootnote{%
\textit{(1)}~legitima
\textit{(2)}~\textit{reformata a legitima}%
~\textit{L}}}
\textit{deficit quartula.}\edlabel{LH_37_01_027_vcurlw-2}\protect\index{Sachverzeichnis}{quartula}
%
\edtext{\textit{Octava\protect\index{Sachverzeichnis}{octava}
ex quarta\protect\index{Sachverzeichnis}{quarta reformata}
et quinta reformatis\protect\index{Sachverzeichnis}{quinta reformata}
aequatur octavae legitimae}\protect\index{Sachverzeichnis}{octava legitima}}{%
\lemma{\textit{Octava} \lbrack...\rbrack\ \textit{legitimae}}\Cfootnote{%
\cite{01266}a.a.O., n.~139. Zitat mit Auslassung.}}
%
(+\phantom)\hspace*{-1.2mm}~%
\edtext{seu octava}{\lemma{}\Afootnote{\textit{Über der Zeile, zwischen} seu \textit{und} octava: NB\vspace{-1.8em}}}
reformata\protect\index{Sachverzeichnis}{octava reformata}
et legitima\protect\index{Sachverzeichnis}{octava legitima}
seu communis\protect\index{Sachverzeichnis}{octava communis} sunt una eademque.%
~\phantom(\hspace*{-1.2mm}+)
%
\edtext{\textit{Sexta major\protect\index{Sachverzeichnis}{sexta major}
cum sesquitono reformato\protect\index{Sachverzeichnis}{sesquitonus reformatus}
octavam\protect\index{Sachverzeichnis}{octava} componens abundat quartula.}\protect\index{Sachverzeichnis}{quartula}}{%
\lemma{\textit{Sexta} \lbrack...\rbrack\ \textit{quartula}}\Cfootnote{%
\cite{01266}a.a.O., n.~140.}}
%
(+\phantom)\hspace*{-1.2mm}~%
\edlabel{LH_37_01_027r_linealogarithmicedivisa_kjrx-1}%
Alios usus
\edtext{reformatarum\protect\index{Sachverzeichnis}{intervallum reformatum}
notavi in schedula de}{%
\lemma{reformatarum}\Bfootnote{%
\textit{(1)}~de
\textit{(2)}~notavi in schedula de%
~\textit{L}}}%
\edtext{ linea Musica.\protect\index{Sachverzeichnis}{linea musica}}{%
\lemma{de linea Musica}\Cfootnote{%
Keine \textit{schedula} mit dieser Überschrift ist ermittelt worden.
Allerdings dürfte es sich hierbei sowie bei der \textit{linea logarithmice divisa}
(S.~\refpassage{LH_37_01_027r_linealogarithmicedivisa_litx-1}{LH_37_01_027r_linealogarithmicedivisa_litx-2})
um eine systematische Darstellung der Strukturintervalle,
wie sie im Kern in LBr~390 Bl.~81~r\textsuperscript{o} anzutreffen ist.}}%
\edlabel{LH_37_01_027r_linealogarithmicedivisa_kjrx-2}%
~\phantom(\hspace*{-1.2mm}+)
%
\pend%
%\newpage%
%
\pstart%
\edtext{}{%
{\xxref{LH_37_01_027_oebchz-1}{LH_37_01_027_oebchz-2}}%
{\lemma{\textso{Limma} \lbrack...\rbrack\ \textit{semitonium minus}}\Cfootnote{%
\cite{01266}\textsc{Jungius}, \textit{Harmonica}, n.~113\textendash115. Zitate mit Auslassungen.%\hspace*{25mm}
}}}%
\edtext{\textso{Limma }\edlabel{LH_37_01_027_oebchz-1}\protect\index{Sachverzeichnis}{limma}%
\textit{(\phantom)\hspace*{-1.2mm}%
quasi residuum\protect\index{Sachverzeichnis}{residuum}%
\phantom(\hspace*{-1.2mm})
est}}{%
\lemma{\textso{Limma}}\Bfootnote{%
\textit{(1)}~est
\textit{(2)}~\textit{(\phantom)\hspace*{-1.2mm}quasi residuum\phantom(\hspace*{-1.2mm}) est}%
~\textit{L}}}
\edtext{\textit{excessus\protect\index{Sachverzeichnis}{excessus}
Quartae\protect\index{Sachverzeichnis}{quarta} super duos}}{%
\lemma{\textit{excessus}}\Bfootnote{%
\textit{(1)}~Quartulae inter duos
\textit{(2)}~\textit{Quartae super duos}%
~\textit{L}}}
\textit{tonos majores},\protect\index{Sachverzeichnis}{tonus major}
estque
\edtext{\lbrack$243 : 256.$\rbrack}{%
\lemma{$243 : 246$}\Bfootnote{\textit{L~ändert Hrsg. nach Vorlage}}}
%\edtext{}{\lemma{$243 : 246$}\Cfootnote{Das (aufsteigende) pythagoreische Limma ist $\displaystyle\frac{243}{256},$ d.h. $\displaystyle\frac{3^5}{2^8}.$}}
\edtext{\textit{Olim habitum est pro semitonio naturali\protect\index{Sachverzeichnis}{semitonium naturale}
et dictum est semitonium minus.}\edlabel{LH_37_01_027_oebchz-2}\protect\index{Sachverzeichnis}{semitonium minus}}{%
\lemma{\textit{Olim} \lbrack...\rbrack\ \textit{minus}}\Cfootnote{%
In der pythagoreischen Skala. Siehe a.a.O, n.~158\textendash163 (\textit{scala diatonica vetus}).}}
%
(+\phantom)\hspace*{-1.2mm}~%
Cadit inter Apotomen majorem\protect\index{Sachverzeichnis}{apotome major}
et minorem.\protect\index{Sachverzeichnis}{apotome minor}~%
\phantom(\hspace*{-1.2mm}\lbrack+\rbrack)
\pend%
% \newpage% % % % Rein vorläufig !!!!
%
\pstart%
\edtext{\textso{Tritonus }\protect\index{Sachverzeichnis}{tritonus}%
intervallum\protect\index{Sachverzeichnis}{intervallum}
quod tres continet
\edtext{tonos,\protect\index{Sachverzeichnis}{tonus}%
\textso{ major }\protect\index{Sachverzeichnis}{tritonus major}%
majores,\protect\index{Sachverzeichnis}{tonus major}%
\textso{ minor }\protect\index{Sachverzeichnis}{tritonus minor}%
minores.\protect\index{Sachverzeichnis}{tonus minor}}{%
\lemma{tonos,}\Bfootnote{%
\textit{(1)}~\textso{majus}
\textit{(2)}~\textso{major} majores
\textit{(a)}~\textso{minus}
\textit{(b)}~\textso{minor} minores.%
~\textit{L}}}%
}{%
\lemma{\textso{Tritonus} \lbrack...\rbrack\ minores}\Cfootnote{%
\cite{01266}a.a.O.,
 n.~93\textendash95.}}
\pend%
%
\pstart%
\edtext{Semitonium naturale\protect\index{Sachverzeichnis}{semitonium naturale}
interdum ponitur pro tono,\protect\index{Sachverzeichnis}{tonus}
eoque casu dicitur%
\textso{ semitonium extraordinarium }\protect\index{Sachverzeichnis}{semitonium extraordinarium}%
vel etiam fictum,\protect\index{Sachverzeichnis}{semitonium fictum}}{%
\lemma{Semitonium \lbrack...\rbrack\ fictum}\Cfootnote{%
\cite{01266}a.a.O., n.~89\,f.% Siehe ?????? ?????? ?????? ??????.
}}
%
ita\edtext{\textit{%
\textso{ quinta vulgaris imperfecta }\protect\index{Sachverzeichnis}{quinta vulgaris imperfecta}%
sive deficiens\protect\index{Sachverzeichnis}{quinta vulgaris deficiens}
constat duobus tonis\protect\index{Sachverzeichnis}{tonus} et duobus semitoniis,\protect\index{Sachverzeichnis}{semitonium}}
cum tamen vera quinta\protect\index{Sachverzeichnis}{quinta vera}
componatur ex ditono,\protect\index{Sachverzeichnis}{ditonus}
tono majore\protect\index{Sachverzeichnis}{tonus major}
et semitonio.\protect\index{Sachverzeichnis}{semitonium}}{%
\lemma{\textit{\textso{quinta}} \lbrack...\rbrack\ semitonio}\Cfootnote{%
\cite{01266} a.a.O., n.~96\,f. Zitat mit Auslassungen.%\hspace*{25mm}
}}
%
Hinc
\edtext{\textit{semitonium} istud
\textit{extraordinarium\protect\index{Sachverzeichnis}{semitonium extraordinarium}
parit necessario\textso{ }%
\edlabel{LH_37_01_027_chromat-1}%
\textso{Apotomen }}\protect\index{Sachverzeichnis}{apotome}%
(\phantom)\hspace*{-1.2mm}%
\edtext{seu intervallum\protect\index{Sachverzeichnis}{intervallum}}{%
\lemma{seu}\Bfootnote{%
\textit{(1)}~differt
\textit{(2)}~intervallum%
~\textit{L}}}
semitonii\protect\index{Sachverzeichnis}{semitonium} et toni\protect\index{Sachverzeichnis}{tonus}%
\phantom(\hspace*{-1.2mm})
\edtext{\textit{quae aliis\textso{ chromaticum }\protect\index{Sachverzeichnis}{semitonium chromaticum}%
aliis\textso{ mutum semitonium }\protect\index{Sachverzeichnis}{semitonium mutum}dicitur.%
\edlabel{LH_37_01_027_chromat-2}}}{%
\lemma{\textit{quae} \lbrack...\rbrack\ \textit{dicitur}}\Cfootnote{%
Quelle nicht nachgewiesen.}}%
}{\lemma{\textit{semitonium} \lbrack...\rbrack\ \textit{dicitur}}\Cfootnote{%
\cite{01266}a.a.O., n.~98\textendash100.}}
%
\edtext{\textit{Semitonium extraordinarium\protect\index{Sachverzeichnis}{semitonium extraordinarium}
prodest Tritono\protect\index{Sachverzeichnis}{tritonus}
mutando in legitimam quartam}\protect\index{Sachverzeichnis}{quarta legitima}
(+\phantom)\hspace*{-1.2mm}~%
si scilicet opinor in tritono\protect\index{Sachverzeichnis}{tritonus}
pro tono\protect\index{Sachverzeichnis}{tonus} substituas semitonium\protect\index{Sachverzeichnis}{semitonium}~%
\phantom(\hspace*{-1.2mm}\lbrack+\rbrack)
\textit{et vulgari deficienti quintae\protect\index{Sachverzeichnis}{quinta vulgaris deficiens}
in legitimam\protect\index{Sachverzeichnis}{quarta legitima} restituendae}}{%
\lemma{\textit{Semitonium} \lbrack...\rbrack\ \textit{restituendae}}\Cfootnote{%
\cite{01266}\textsc{Jungius}, \textit{Harmonica}, n.~101\,f.}}
%
(+\phantom)\hspace*{-1.2mm}~%
non satis apparet~%
\phantom(\hspace*{-1.2mm}+).%
\edtext{}{\lemma{\textit{Am Rand:}}\Afootnote{\Denarius\vspace{-1.8em}}}
\pend%
\count\Bfootins=1100
\count\Afootins=1100
\count\Cfootins=1100
%
\pstart%
\edtext{}{{\xxref{LH_37_01_027_auf/ab-3}{LH_37_01_027_auf/ab-4}}%
\lemma{\textit{Octo} \lbrack...\rbrack\ \textit{excedunt}}\Cfootnote{%
\cite{01266}a.a.O., n.~216\textendash218. Zitat mit Auslassung.
Auch hier gelten die angegebenen Verhältnisse nur insofern,
als die betreffenden Intervalle \textit{absteigend} dargestellt werden.
Siehe hierüber die Erläuterung zu S.~\refpassage{LH_37_01_027_auf/ab-1}{LH_37_01_027_auf/ab-2}.}}%
\edlabel{LH_37_01_027_auf/ab-3}%
\textit{Octo
commata\protect\index{Sachverzeichnis}{comma}
tonum minorem\protect\index{Sachverzeichnis}{tonus minor} non explent,
novem commata\protect\index{Sachverzeichnis}{comma}
minorem\protect\index{Sachverzeichnis}{tonus minor} excedunt,
majorem\protect\index{Sachverzeichnis}{tonus major} non complent,
decem commata\protect\index{Sachverzeichnis}{comma}
tonum majorem}\protect\index{Sachverzeichnis}{tonus major}
\edtext{\textit{excedunt.}%
\edlabel{LH_37_01_027_auf/ab-4}
\edtext{(+\phantom)\hspace*{-1.2mm}~%
Semicomma\protect\index{Sachverzeichnis}{semicomma}}{%
\lemma{(+\phantom)\hspace*{-1.2mm}~Semicomma}\Cfootnote{%
Die Klammer bleibt im Text offen.}}
erit $\displaystyle 9 : 4\sqrt{5}.$}{%
\lemma{\textit{excedunt}}\Bfootnote{%
\textit{(1)}~. (+\phantom)\hspace*{-1.2mm}~Etiam Comma\protect\index{Sachverzeichnis}{comma} posset sumi pro minima mensura,%
\protect\index{Sachverzeichnis}{mensura minima intervallorum} faciendo Tonum 9
\textit{(2)}~et
\textit{(3)}~. $\displaystyle9 : 8\, \squaredots\, 10 : 9$ est $81 : 80.$
\textit{(a)}~Ut inveniatur quot num
\textit{(b)}~$\displaystyle\overline{81 : 80}^{v}$ aeq. $9 : 8.$
$\displaystyle\overline{81 : 80}^{\omega}$ aeq. $10 : 9.$
fiet: $\displaystyle\sqrt[v]{9 : 8}$ aeq. $\displaystyle\sqrt[\omega]{10 : 9}$
\textit{(4)}~. (+\phantom)\hspace*{-1.2mm}~Semicomma erit $\displaystyle 9 : 4\sqrt{5}.$%
~\textit{L}}}
Habet ergo connexionem cum sectione divina\protect\index{Sachverzeichnis}{sectio divina} dicta,
extremae et mediae rationis.
Nam si $a + b : a\, \squaredots\,a : b,$
fit $aa$ aeq. $ab + bb$ seu $\displaystyle aa + \frac{1}{4}aa$ aeq.
\edtext{$Q.$}{\lemma{$Q$}\Cfootnote{%
Hier wird ein q-artiges Zeichen verwendet,
das folgenden Ausdruck ersetzt:
$\displaystyle b^2 + ab + \frac{1}{4}a^2$ bzw. $\displaystyle(b + \frac{1}{2}a)^2.$}}
Ergo
\edtext{erit $\displaystyle\frac{1}{2}a\sqrt{5}$ aeq. $\displaystyle b + \frac{1}{2}a,$}{%
\lemma{erit}\Bfootnote{%
\textit{(1)}~$a$ aeq. $\displaystyle b + \frac{1}{2}$
\textit{(2)}~$\displaystyle\frac{1}{2}a\sqrt{5}$ aeq. $\displaystyle b + \frac{1}{2}a,$%
~\textit{L}}}
seu $\displaystyle\frac{1}{2}a\overline{\sqrt{5} - 1}$ aeq. $b,$
seu $\displaystyle a : b\, \squaredots\,2 : \overline{\sqrt{5} - 1},$
seu $\displaystyle\overline{\sqrt{5} + 1} : 2\, \squaredots\,\overline{a + b} : a.$
Si $9 : 8,$ et $\displaystyle 2 : \sqrt{5}$ in se ducantur,
fit semicomma.\protect\index{Sachverzeichnis}{semicomma}
Jam $\displaystyle 2 : \sqrt{5}$ est ut $\displaystyle a : b + \frac{1}{2}a.$
\edtext{Itaque si recta secetur}{%
\lemma{Itaque}\Bfootnote{%
\hspace{-0,5mm}si
\textit{(1)}~tonus major comparetur
\textit{(2)}~recta secetur%
~\textit{L}}}
extrema et media ratione,
atque inde auferatur dimidium segmenti majoris
(\phantom)\hspace*{-1.2mm}%
nempe $\displaystyle\frac{1}{2}a,$ ut restet $\displaystyle\frac{1}{2}a + b$%
\phantom(\hspace*{-1.2mm})
residuique\protect\index{Sachverzeichnis}{residuum} ratio ad segmentum medii
auferatur a ratione in qua consistit tonus
\edtext{major\protect\index{Sachverzeichnis}{tonus major}}{%
\lemma{major}\Bfootnote{\textit{erg.~L}}}
$\displaystyle 9 : 8 \squaredots \sqrt{5} : 2,$
habebitur semicomma\protect\index{Sachverzeichnis}{semicomma}
$\displaystyle 9 : 4\sqrt{5}.$
Quod si liceat assumere semicomma
 pro minima
\edtext{mensura,\protect\index{Sachverzeichnis}{mensura minima intervallorum} fient}{%
\lemma{mensura,}\Bfootnote{%
\textit{(1)}~fiet
\textit{(2)}~fient%
~\textit{L}}}
19 semicommata\protect\index{Sachverzeichnis}{semicomma} aequalia
\edtext{tono quasi medio,\protect\index{Sachverzeichnis}{tonus quasi medius}
$\displaystyle 9 \frac{1}{2}$ quasi semitonio;\protect\index{Sachverzeichnis}{quasi semitonium}}{%
\lemma{tono}\Bfootnote{%
\textit{(1)}~majori
\textit{(2)}~minori
\textit{(3)}~\textbar~quasi \textit{erg.}~\textbar\ medio, $\displaystyle 9 \frac{1}{2}$
\textit{(a)}~semitonio
\textit{(b)}~quasi semitonio;%
~\textit{L}}}
ex quibus caetera componentur.\protect\index{Sachverzeichnis}{compositio intervallorum}
Quartula\protect\index{Sachverzeichnis}{quartula}
\edtext{erit $\displaystyle 2 \sqrt[4]{5} : 3.$
Tonus medius\protect\index{Sachverzeichnis}{tonus medius}
est dimidius ditonus,\protect\index{Sachverzeichnis}{ditonus dimidius}}{%
\lemma{erit}\Bfootnote{%
\textit{(1)}~$\displaystyle 3 : 2\sqrt[4]{5}$
\textit{(2)}~$\displaystyle 2\sqrt[4]{5} : 3.$
\textit{(a)}~Cum
\textit{(b)}~Tonus medius
\textit{(aa)}~vel semiditonus
\textit{(bb)}~est dimidius ditonus,%
~\textit{L}}}
sive $\displaystyle 2 : \sqrt[2]{5}.$
Semitonium rigorosum\protect\index{Sachverzeichnis}{semitonium rigorosum} foret
\edtext{$\displaystyle\sqrt[2]{2} : \sqrt[4]{5},$}{%
\lemma{$\displaystyle\sqrt[2]{2} : \sqrt[4]{5}$}\Cfootnote{%
Vom aufsteigenden Mittelton $\displaystyle\frac{2}{\sqrt{5}}$ ausgehend,
ist der richtige Wert $\displaystyle\frac{\sqrt[4]{5}}{\sqrt{2}}.$}}
dimidius scilicet tonus medius.\protect\index{Sachverzeichnis}{tonus medius dimidius}
Semitonium reformatum\protect\index{Sachverzeichnis}{semitonium reformatum}
est $\displaystyle\overline{15 : 16}\,.\,\overline{2\sqrt[4]{5}} : 3$
seu $\displaystyle 5\sqrt[4]{5} : 8.$
Apotome reformata\protect\index{Sachverzeichnis}{apotome reformata}
est $\displaystyle 16 : 5\sqrt[2]{5}\sqrt[4]{5}.$
Unde patet has reformationes\protect\index{Sachverzeichnis}{reformatio} satis esse perplexas.
\pend%
\newpage% % % % Rein vorläufig !!!!
%
\pstart%
Si logarithmis\protect\index{Sachverzeichnis}{logarithmus} utamur,
\edtext{logarithmus 1 seu unisoni\protect\index{Sachverzeichnis}{unisonus} est 0.}{%
\lemma{logarithmus}\Bfootnote{%
\textit{(1)}~1 est 0,
\textit{(2)}~1 seu unisoni est 0.%
~\textit{L}}}
Octavae\protect\index{Sachverzeichnis}{octava} est 0.30103.
\edtext{Sextae ma:\protect\index{Sachverzeichnis}{sexta major} 0.22185.
Sext. min.\protect\index{Sachverzeichnis}{sexta minor} 0.20412.}{%
\lemma{Sextae}\Bfootnote{%
\hspace{-0,5mm}ma: 0.22185. Sext. min. 0.20412.
\textit{erg.~L}}}
Quintae\protect\index{Sachverzeichnis}{quinta} est 0.17609.
Quartae,\protect\index{Sachverzeichnis}{quarta} 0.12494.
Tertiae majoris\protect\index{Sachverzeichnis}{tertia major}
\edtext{seu ditoni\protect\index{Sachverzeichnis}{ditonus}}{%
\lemma{seu}\Bfootnote{%
\hspace{-0,5mm}ditoni \textit{erg.~L}}}
0.09691.
Tertiae minoris\protect\index{Sachverzeichnis}{tertia minor}
\edtext{seu sesquitoni\protect\index{Sachverzeichnis}{sesquitonus}}{%
\lemma{seu}\Bfootnote{%
\hspace{-0,5mm}sesquitoni \textit{erg.~L}}}
\edtext{0.07924%
\edtext{.}{\lemma{0.07924}\Cfootnote{%
Der richtige Wert ist $\log 6 - \log 5 = 0.07918.$}}
Toni majoris\protect\index{Sachverzeichnis}{tonus major}}{%
\lemma{0.07924.}\Bfootnote{%
\textit{(1)}~Sextae majoris
\textit{(2)}~Toni majoris%
~\textit{L}}}
0.05115.
Toni minoris\protect\index{Sachverzeichnis}{tonus minor}
\edtext{0.04575.}{\lemma{0.04575}\Cfootnote{%
Der richtige Wert ist $\log 10 - \log 9 = 0.04576.$}}
Semitonii naturalis\protect\index{Sachverzeichnis}{semitonium naturale} 0.02803.
\edtext{Commatis:\protect\index{Sachverzeichnis}{comma} 0.00539%
\edtext{.}{\lemma{0.00539}\Cfootnote{%
Der richtige Wert ist $\log 81 - \log 80 = 0.00540.$}}}{%
\lemma{Commatis:}\Bfootnote{%
\textit{(1)}~0.00540.
\textit{(2)}~0.00539.%
~\textit{L}}}
\edtext{Semicommatis\protect\index{Sachverzeichnis}{semicomma} 0.00269%
\edtext{.}{\lemma{0.00269}\Cfootnote{%
Der richtige Wert ist $\displaystyle\frac{\log 81 - \log 80}{2} = 0.00270.$}}}{%
\lemma{Semicommatis}\Bfootnote{%
\textit{(1)}~0.00270.
\textit{(2)}~0.00269.%
~\textit{L}}}
\edtext{Quartulae\protect\index{Sachverzeichnis}{quartula} 0.00134%
\edtext{.}{\lemma{0.00134}\Cfootnote{%
Der richtige Wert ist $\displaystyle\frac{\log 81 - \log 80}{4} = 0.00135.$}}
Toni medii\protect\index{Sachverzeichnis}{tonus medius}}{%
\lemma{Quartulae}\Bfootnote{%
\textit{(1)}~0.00154.
\textit{(2)}~0.00135.
\textit{(3)}~0.00134.
\textit{(a)}~Semitonii reform
\textit{(b)}~Toni medii%
~\textit{L}}}
0.04845.
Semitonii rigorosi\protect\index{Sachverzeichnis}{semitonium rigorosum} 0.02422.
Semitonii\protect\index{Sachverzeichnis}{semitonium reformatum} 0.02937. 
Apotomae reformatae\protect\index{Sachverzeichnis}{apotome reformata} 0.01908.
Sesquitoni\protect\index{Sachverzeichnis}{sesquitonus reformatus}
\edtext{\lbrack seu\rbrack}{%
\lemma{seu}\Bfootnote{%
\textit{erg. Hrsg.}}}
\edtext{3. minor.\protect\index{Sachverzeichnis}{tertia minor reformata}}{%
\lemma{3.}\Bfootnote{%
\hspace{-0,5mm}minor.
\textit{erg.~L}}}
reformati
\edtext{0.07582.}{%
\lemma{0.07582}\Cfootnote{%
Der richtige Wert ist $0.04845 + 0.02937 = 0.07782.$}}
Quartae reformatae\protect\index{Sachverzeichnis}{quarta reformata} 0.12628.
Quintae reformatae\protect\index{Sachverzeichnis}{quinta reformata}
\edtext{0.17472.
Videamus an quarta\protect\index{Sachverzeichnis}{quarta reformata}
et quinta\protect\index{Sachverzeichnis}{quinta reformata}}{%
\lemma{0.17472.}\Bfootnote{%
\textit{(1)}~quarta et qu
\textit{(2)}~Videamus an quarta et quinta%
~\textit{L}}}
reformatae a legitimis\protect\index{Sachverzeichnis}{quarta legitima}\protect\index{Sachverzeichnis}{quinta legitima} distent
\edtext{quartula,\protect\index{Sachverzeichnis}{quartula}
quae fuit proba calculi:\protect\index{Sachverzeichnis}{proba calculi}}{%
\lemma{quartula,}\Bfootnote{%
\textit{(1)}~ quae fuit proba calculi:
\textit{(2)}~ut supra asserebatur:
\textit{(3)}~quae fuit proba calculi:%
~\textit{L}}}
$0.17609 -
\edtext{0.17472$ aeq. $0.00137.\,$
$0.12628 - 0.12494$ aeq. \lbrack$0.00134.$\rbrack\
Verum}{%
\lemma{0.17472}\Bfootnote{%
\hspace{-0,5mm}aeq.
\textit{(1)}~$0.0137$
\textit{(2)}~$0.00137.$
\textit{(a)}~$0.12494 - 0.12628$ aeq. 66
\textit{(b)}~$0.12628 - 0.12494$ aeq.
\textbar~$-\,0.00134.$ \textit{ändert Hrsg.}~%
\textbar\ Verum%
~\textit{L}}}
est ergo.
Apotome major\protect\index{Sachverzeichnis}{apotome major} 0.02312.
Minor\protect\index{Sachverzeichnis}{apotome minor} 0.01772.
Limma,\protect\index{Sachverzeichnis}{limma} 0.02264
(\lbrack+\rbrack\phantom)\hspace*{-1.2mm}~%
Limma\protect\index{Sachverzeichnis}{limma} circiter
dimidius tonus minor\protect\index{Sachverzeichnis}{tonus minor dimidius}%
~\phantom(\hspace*{-1.2mm}+)
%
\lbrack27~v\textsuperscript{o}\rbrack\ % Blatt 27v
%
\pend%
%\newpage
%
\pstart%
\edtext{Hinc\edlabel{LH_37_01_027r_linealogarithmicedivisa_awgr-1}
possumus lineam Musicam\protect\index{Sachverzeichnis}{linea musica} parare,
metientem intervalla
\edtext{sonorum,\protect\index{Sachverzeichnis}{intervallum sonorum} dissimulatis}{%
\lemma{sonorum,}\Bfootnote{%
\textit{(1)}~ubi
\textit{(2)}~dissimulatis%
~\textit{L}}}
logarithmis\protect\index{Sachverzeichnis}{logarithmus} divisam,
cujus ope facile omnis generis systemata\protect\index{Sachverzeichnis}{systema} componentur.
Ea minimum dividenda erit in
\edtext{partes 3010.\edlabel{LH_37_01_027r_linealogarithmicedivisa_awgr-2}}{%
\lemma{partes}\Bfootnote{%
\textit{(1)}~310
\textit{(2)}~3010.%
~\textit{L}}}%
}{\lemma{Hinc \lbrack...\rbrack\ 3010}\Cfootnote{%
Siehe über die \textit{linea musica} als Methode zur Darstellung der Strukturintervalle die Erläuterung zu S.~\refpassage{LH_37_01_027r_linealogarithmicedivisa_kjrx-1}{LH_37_01_027r_linealogarithmicedivisa_kjrx-2} sowie LBr~390, Bl.~81~r\textsuperscript{o}.}}
\pend%
\vspace{1.0em}% % % % % Rein vorläufig !!!!
 \newpage% % % % % Rein vorläufig !!!!
%
\pstart%
\noindent%
\lbrack\textit{Am unteren rechten Rand von Bl.~27~r\textsuperscript{o}:}\rbrack\ 
%\newline%
\pend%
\vspace*{1.0em}%
\pstart%
\noindent%
%
\hspace*{6mm}$\phantom{\log{1}}15$ \quad 1.17609 \\ 
\hspace*{6mm}$\phantom{\log{1}}16$ \quad 1.20412 \\
% \hspace*{6mm}$\phantom{\log{1}16}$ \quad \\
\protect\index{Sachverzeichnis}{logarithmus}\hspace*{6mm}$\hspace*{3,5mm}\log{1}$ \quad 0.00000 \\ 
\hspace*{6mm}$\phantom{\log{10}}2$ \quad 0.30103 \\ 
\hspace*{6mm}$\phantom{\log{10}}3$ \quad 0.47712 \\ 
\hspace*{6mm}$\phantom{\log{10}}4$ \quad 0.60206 \\ 
\hspace*{6mm}$\phantom{\log{10}}5$ \quad 0.69897 \\ 
\hspace*{6mm}$\phantom{\log{10}}6$ \quad 0.77815 \\ 
\hspace*{6mm}$\phantom{\log{10}}7$ \quad 0.84509 \\ 
\hspace*{6mm}$\phantom{\log{10}}8$ \quad 0.90309 \\ 
\hspace*{6mm}$\phantom{\log{10}}9$ \quad 0.95424 \\ 
\hspace*{6mm}$\phantom{\log{1}}10$ \quad 1.00000 \\ 
\hspace*{6mm}$\phantom{\log{1}}11$ \quad 1.04139 \\ 
\hspace*{6mm}$\phantom{\log{1}}12$ \quad 1.07918 \\ 
\hspace*{6mm}$\phantom{\log{1}}80$ \quad 1.90309 \\ 
\hspace*{6mm}$\phantom{\log{1}}81$ \quad %
%\hspace*{-0,90mm}
\edtext{1.90948}{%
\lemma{1.90948}\Cfootnote{%
Der richtige Wert für $\log 81$ ist 1.90849.}}
%
\pend
\count\Bfootins=1200
\count\Afootins=1200
\count\Cfootins=1200
%
% ENDE DES STÜCKES auf Blatt 27v