%   % !TEX root = ../../VIII,3_Rahmen-TeX_8-1.tex
%  
%   Band VIII, 3		Rubrik STOSS
%
%   Signatur/Tex-Datei:	LH_35_10_08_018
%
%   RK-Nr. 	60070	\ref{RK60070}
%
%   Überschrift: 	[Experimenta domini Regnault pendulis facta]
%   
%   Unterrubrik:			DCC plus
%
%   Datierung:		[Herbst 1674 (?) bis Januar 1678]
%
%   edlabels:			0
%
%   Diagramme: 		0
%
%
%   NB: 						(Anmerkungen)					??
%
%
%
\selectlanguage{ngerman}
\frenchspacing
%
\begin{ledgroupsized}[r]{120mm}
\footnotesize
\pstart
\noindent\textbf{Überlieferung:}
\pend
\end{ledgroupsized}
%
\begin{ledgroupsized}[r]{114mm}
\footnotesize
\pstart \parindent -6mm
\makebox[6mm][l]{\textit{L}}%
Notiz:
LH~XXXV~10,~8 Bl.~18.
Ein Zettel (ca.~6~x~17,5~cm.);
oberer und linker Rand beschnitten.
Eine Seiten auf Bl.~18~r\textsuperscript{o}; Bl.~18~v\textsuperscript{o} leer.
\pend
\end{ledgroupsized}
%
%
\vspace{5mm}
\begin{ledgroup}
\footnotesize
\pstart
\noindent%
\textbf{Datierungsgründe:}
In der vorliegenden Notiz N.~\ref{RK60070} %??S06 
greift Leibniz, wohl erstmals in Bezug auf die Stoßlehre, Inhalte aus B.~de Monconys'%
\protect\index{Namensregister}{\textso{Monconys} (Monconisius), Balthasar de 1611\textendash1665}
\title{Journal de voyages} (3~Teile, Lyon 1666)\cite{00118} auf.
Hierbei bezieht er sich insbesondere auf F.~Regnaulds%
\protect\index{Namensregister}{\textso{Regnauld} (Regnaud; Regnaldus), Fran\c{c}ois de 1626\textendash1689}
dort veröffentlichten Brief vom 21. Dezember 1655, in dem ausführlich über Experimente zum Stoß kugelförmiger Pendel berichtet wird (ebd., Teil III: \glqq Lettres escrittes à Monsieur de Monconys\grqq, S.~52\textendash56).\cite{02021} %, separate Paginierung
In seinem Schreiben stellt Regnauld in tabellarischer Form auch empirische Messwerte dar, denen Leibnizens Aufmerksamkeit in der Notiz N.~\ref{RK60070} %??S06 
am meisten gilt.
\pend%
%
\pstart%
%Monconys'%
%\protect\index{Namensregister}{\textso{Monconys} (Monconisius), Balthasar de 1611\textendash1665}
Den \title{Journal de voyages}\cite{00118} zieht Leibniz auch in drei Pariser Texten heran, die insgesamt auf den Zeitraum zwischen den letzten Monaten 1674 und April 1675 datierbar sind:
\title{LSB} VIII,~2 N.~8;\cite{01343} N.~10;\cite{01344} N.~32.\cite{01345}
Darunter beruft sich der mit der Reibungslehre befasste und von Leibniz auf April 1675 datierte Entwurf N.~32 auf ebenden % in der Notiz N.~\ref{RK60070} %??S06 zitierten 
Brief Regnaulds%
\protect\index{Namensregister}{\textso{Regnauld} (Regnaud; Regnaldus), Fran\c{c}ois de 1626\textendash1689} (vgl. VIII,~2, S.~273.1\textendash4).\cite{01345}
Der Tenor dieser Andeutung lässt keinen Zweifel zu, dass Leibniz zu dem Zeitpunkt, als er N.~32 verfasste, den Brief gelesen hatte und Regnaulds empirische Ergebnisse als Herausforderung für die Stoßlehre ansah. % (\textit{concursuum labyrinthi quos exhibet Regnaldus apud Monconisium}). 
Die Notiz N.~\ref{RK60070} %??S06 
könnte demnach ebenfalls im Frühjahr 1675 oder noch in den letzten Monaten des vorherigen Jahres entstanden sein, als Leibniz sich möglicherweise mit Monconys' Werk (wieder) beschäftigte.
Dass er sich zu diesem Zeitpunkt bereits lange damit auskannte, ist nicht von der Hand zu weisen, da der \title{Journal de voyages}\cite{00118} seit spätestens dem Frühjahr 1671 mehrfach und zuweilen auch ausführlich im Leibniz-Nachlass Erwähnung findet (vgl. etwa
\textit{LSB} II,~1 \lbrack2006\rbrack\ N.~57, S.~168.20\textendash23;\cite{01346} % An Oldenburg, 29. IV. (9. V.) 1671
IV,~1 \lbrack1983\rbrack\ N.~15, S.~276.5\textendash14, 642.13\,f.;\cite{01347} % Justa dissertatio [Consilium Aegyptiacum, Winter 1671/1672]
VIII,~2 N.~84, S.~716.15\,f.;\cite{01348} % Chronologia. Efficere horologia accurata [2. Hälfte 1672]
N.~87, S.~722.10\,f.;\cite{01349} % Quomodo penduli motus magnete effici possit [2. Hälfte 1672]
III,~1 N.~4, S.~24.10\textendash18;\cite{01350} % Für die Royal Society, 3. (13.) II. 1673; S.~22\textendash29
VII,~4 N.~11, S.~167.4\,ff., 171.8\,ff.\cite{01353}). % De chordis in circulo. De hemisphaerii et sphaeroeidum superficiebus [Frühjahr 1673]
Eine frühere Entstehungszeit ist folglich auch bei der Aufzeichnung N.~\ref{RK60070} %??S06 
nicht auszuschließen; sie erweist sich jedoch als weniger wahrscheinlich, weil Leibnizens Augenmerk in N.~\ref{RK60070} %??S06 
Inhalten gilt, die eine größere Verwandtschaft mit den drei oben genannten Pariser Texten aufweisen:
An erster Stelle ist die Erhaltung bzw. Absorption der Bewegung beim Stoß zu nennen (vgl. S.~\refpassage{LH_35_10_08_018r_absorptio-1}{LH_35_10_08_018r_absorptio-2}).
Auch die Bemerkung im Schlussteil von N.~\ref{RK60070} %??S06 
\textendash\ Regnaulds empirische Messungen zum Stoß würden zeigen, dass eine Wirkung gegebenenfalls mehr Bewegungsgröße umfassen könne als ihre Ursache (S.~\refpassage{LH_35_10_08_018r_effectuscausa-1}{LH_35_10_08_018r_effectuscausa-2}) \textendash\ dürfte an die Überlegungen über die Äquipollenz von Ursache und Wirkung anknüpfen, die auf die späte Pariser Zeit zurückgehen (vgl.
\title{LSB} VIII,~2 N.~52;\cite{01351} % De ictuum quantitate (V. 1675)
N.~12\cite{01352}). % De arcanis motus [II-IX 1676]
Als wahrscheinlichster Terminus post quem der Datierung von N.~\ref{RK60070} %??S06 
ist demgemäß die Entstehungszeit der drei einschlägigen Pariser Texte % (\textit{LSB} VIII,~2 N.~8, N.~10 und N.~32) 
anzusehen. 
%LSB VIII,2 N. 8: letzte Monate 1674 / erste Monate 1675
%LSB VIII,2 N. 10: eigh. Dez. 1674
%LSB VIII,2 N. 32: eigh. April 1675
%daher faute de mieux: April 1675 bis Ende 1677
\pend%
%
\pstart%
Die in Regnaulds Brief angeführten experimentellen Messwerte greift Leibniz erneut und ausführlicher im Januar 1678 auf, als er in Hannover die \textit{Scheda VI-II de corporum concursu} verfasst (N.~\ref{dcc_06-2}, %??S01\textsubscript{8} 
S.~\pageref{37_04_059r_Regnauld_Anfang}\textendash\pageref{37_04_060r_Regnauld_Ende}).
Regnaulds Daten werden dort \textendash\ zum Teil mit Hilfe eines Schreibers \textendash\ zusammengefasst und tabellarisch dargestellt.
Hierbei verfolgt Leibniz wohl das Ziel, die Gleichungen über den zentralen Stoß, die er im ersten Teil von N.~\ref{dcc_06-2} %??S01\textsubscript{8} 
berechnet hat, durch einen Vergleich mit empirischen Ergebnissen aus der \glqq Literatur\grqq\ zu prüfen (siehe die editorische Vorbemerkung zu N.~\ref{dcc_00}, S.~\refpassage{dcc_intro_VI-II_fzr-1}{dcc_intro_VI-II_fzr-2}). %??S01 
Somit erweist sich Regnaulds Brief als eine zentrale Quelle für Leibnizens Stoßlehre im Textkomplex \textit{De corporum concursu}.
Es wäre demgemäß zu erwarten, dass diese so bedeutsame Verwendung von Regnaulds Daten % bei der Abfassung von N.~\ref{dcc_06-2} % %??S01\textsubscript{8}
ausdrückliche Erwähnung gefunden hätte, wenn die Notiz N.~\ref{RK60070} %??S06 
nach N.~\ref{dcc_06-2} %??S01\textsubscript{8} 
und somit nach Januar 1678 verfasst worden wäre.
Zudem hätte sich Leibniz nach Januar 1678 nicht mehr ohne weiteres darüber gewundert, dass die Bewegungsgröße nach dem Stoß gegebenenfalls zu wachsen scheine, wie dies aus dem zweiten in N.~\ref{RK60070} %??S06 
angeführten Beispiel resultiere (S.~\refpassage{LH_35_10_08_018r_MVaugmentat-1}{LH_35_10_08_018r_MVaugmentat-2}).
Denn zu der Zeit war seine Entdeckung, dass beim direkten zentralen Stoß nicht die Bewegungsgröße \textit{mv}, sondern die \glqq Kraft\grqq\ $mv^2$ erhalten bleibe, bereits vollzogen (vgl. N~\ref{dcc_08}, %??S01\textsubscript{10}
S.~\refpassage{LH_37_05_086r_reformatio_idzg-1}{LH_37_05_086r_reformatio_idzg-2}). 
Januar 1678 ist aus diesen Gründen als (spätester) Terminus ante quem der Datierung anzusehen.
Daraus ergibt sich für die Notiz N.~\ref{RK60070} %??S06 
insgesamt die vorgeschlagene Entstehungszeit.%
\pend 
\end{ledgroup}
%
%
\count\Bfootins=1000%
\count\Afootins=1200%
\count\Cfootins=1000
\selectlanguage{latin}
\frenchspacing
% \newpage%
\vspace{8mm}
\pstart%
\normalsize%
\noindent%
\lbrack18~r\textsuperscript{o}\rbrack\
%
\edlabel{LH_35_10_08_018r_Quelle-1}%
\edtext{}{%
{\xxref{LH_35_10_08_018r_Quelle-1}{LH_35_10_08_018r_Quelle-2}}%
{\lemma{Sub finem \lbrack...\rbrack\ experimentis}\Cfootnote{%
\textsc{F.~Regnauld}, Brief an B.~de Monconys vom 21.~Dezember 1655, in: \textsc{B.~de Monconys}, \title{Journal des voyages}, Teil III: \glqq Lettres escrittes à Monsieur de Monconys\grqq, Lyon 1666, S.~52\textendash56 (separate Seitenzählung).%
\cite{00118}\cite{02021}
Leibniz hat im Januar 1678 diese Quelle beim Verfassen der \textit{Scheda IX de corporum concursu} exzerpiert; siehe N.~\ref{dcc_06-2}, %??S01\textsubscript{8} 
S.~\pageref{37_04_059r_Regnauld_Anfang}\textendash\pageref{37_04_060r_Regnauld_Ende}.%
}}}%
Sub finem itinerarii Monconisiani%
\protect\index{Namensregister}{\textso{Monconys} (Monconisius), Balthasar de 1611\textendash1665}
inter Epistolas ei scriptas habetur una Domini Regnault
\protect\index{Namensregister}{\textso{Regnauld} (Regnaud; Regnaldus), Fran\c{c}ois de 1626\textendash1689}
ubi
%
\edtext{ex}{%
\lemma{ex}\Bfootnote{%
\textit{erg.~L}}}
%
pendulorum\protect\index{Sachverzeichnis}{pendulum} conflictu\protect\index{Sachverzeichnis}{conflictus pendulorum}
judicare aggressus est
de percussione\protect\index{Sachverzeichnis}{percussio}
jam anno 1655 subjectis 
%
\edtext{experimentis.\protect\index{Sachverzeichnis}{experimentum}%
\edlabel{LH_35_10_08_018r_Quelle-2}
Ex ligno duro\protect\index{Sachverzeichnis}{lignum durum} fuere}{%
\lemma{experimentis}\Bfootnote{%
\textit{(1)}~sed non addit ex qua materia fuerint
\textit{(2)}~Ex ligno duro fuere%
~\textit{L}}}
%
pilae\protect\index{Sachverzeichnis}{pila confligens} 
%
\edtext{confligentes.%
\edlabel{LH_35_10_08_018r_absorptio-1}
Magnam}
{\lemma{confligentes.}\Bfootnote{%
\textit{(1)}~Certe
\textit{(2)}~Magnam%
~\textit{L}}}
%
percussionis\protect\index{Sachverzeichnis}{percussio}
partem ab ipsis absorberi video
pilis;\protect\index{Sachverzeichnis}{pila confligens}
%
\edtext{v.\,g.}{\lemma{v.\,g.}\Cfootnote{%
Vgl. zum folgenden Beispiel:
\textsc{Regnauld}, Brief an Monconys vom 21.~Dezember 1655, S.~53;
die Wiedergabe ist nicht getreu.%
\cite{00118}\cite{02021}
Siehe zudem Leibnizens Auszug in N.~\ref{dcc_06-2}, %??S01\textsubscript{8} 
S.~\pageref{LH_37_04_059r_tabelle1in16}.%
}}
%
cum tota altitudo\protect\index{Sachverzeichnis}{altitudo}
seu semidiameter funependuli\protect\index{Sachverzeichnis}{funependulum}
divisa fuisset in 100 partes,
et pila librae\protect\index{Sachverzeichnis}{libra}
unius in pilam\protect\index{Sachverzeichnis}{pila confligens}
16 librarum\protect\index{Sachverzeichnis}{libra}
incidisset ex altitudine\protect\index{Sachverzeichnis}{altitudo}
%
\edtext{partium 4,}{%
\lemma{partium}\Bfootnote{%
\textit{(1)}~8,
\textit{(2)}~64,
\textit{(3)}~4,%
~\textit{L}}}
%
tunc ipsa quidem pila\protect\index{Sachverzeichnis}{pila accipiens} 
%
\edtext{accipiens immota mansit,
incurrens autem reflexa%
\protect\index{Sachverzeichnis}{pila incurrens}%
\protect\index{Sachverzeichnis}{pila reflexa}
tantum assurexit}{%
\lemma{accipiens}\Bfootnote{%
\textit{(1)}~nullo modo ass
textit{(2)}~immota mansit, incurrens autem assurexit%
~\textit{L}}}
%
ad altitudinem\protect\index{Sachverzeichnis}{altitudo}
unius partis.
Reliquum ergo impetus\protect\index{Sachverzeichnis}{impetus reliquus}
absorptum fuit.%
\edlabel{LH_35_10_08_018r_absorptio-2}%
\protect\index{Sachverzeichnis}{impetus absorptus}%
\pend%
%
\pstart%
Interea%
\edlabel{LH_35_10_08_018r_MVaugmentat-1}%
\edlabel{LH_35_10_08_018r_effectuscausa-1}
accurate ibi demonstratur quantitatem motus%
\protect\index{Sachverzeichnis}{quantitas motus}
aliquando majorem esse in effectu%
\protect\index{Sachverzeichnis}{effectus}
quam in causa,%
\edlabel{LH_35_10_08_018r_effectuscausa-2}%
\protect\index{Sachverzeichnis}{causa}
%
\edtext{ex.\ grat.}{%
\lemma{ex.\ grat.}\Cfootnote{%
Vgl. zum folgenden Beispiel:
\textsc{Regnauld}, Brief an Monconys vom 21.~Dezember 1655, S.~54;
die Wiedergabe ist nicht getreu.%
\cite{00118}\cite{02021}
Siehe zudem Leibnizens Auszug in N.~\ref{dcc_06-2}, %??S01\textsubscript{8} 
S.~\pageref{LH_37_04_059v_Tab(1)in(4)}.%
}}
%
Agens%
\protect\index{Sachverzeichnis}{agens}
ut 1 descendit in patiens quiescens%
\protect\index{Sachverzeichnis}{patiens quiescens}
%
\edtext{ut
\lbrack4\rbrack,
celeritate%
\protect\index{Sachverzeichnis}{celeritas}%
}{%
\lemma{ut}\Bfootnote{\hspace{-0,5mm}%
\textbar~2 \textit{ändert Hrsg.}~\textbar\
\textit{(1)}~ex altitudine
\textit{(2)}~celeritate%
~\textit{L}}}
%
ut 4,
quantitas motus\protect\index{Sachverzeichnis}{quantitas motus}
in causa\protect\index{Sachverzeichnis}{causa}
est 1 in 4 seu 4.
Post ictum incurrens%
\protect\index{Sachverzeichnis}{pila incurrens}
reflectitur celeritate ut~1,
\edtext{quantitate motus%
\protect\index{Sachverzeichnis}{quantitas motus}
ut 1,}{%
\lemma{quantitate}\Bfootnote{%
\hspace{-0,5mm}motus ut 1,
\textit{erg.~L}}}
%\lbrack,\rbrack\
%
excipiens%
\protect\index{Sachverzeichnis}{pila excipiens}
vero assurgit celeritate%
\protect\index{Sachverzeichnis}{celeritas}
$1\frac{1}{4},$
seu quantitate motus~5.%
\protect\index{Sachverzeichnis}{quantitas motus}
Ergo
%
\edtext{q.\,m.%
\edlabel{LH_35_10_08_018r_MVaugmentat-2}%
\protect\index{Sachverzeichnis}{quantitas motus}%
}{\lemma{q.\,m.}\Cfootnote{%
\textit{quantitas motus}%
}}
%
est in summa 6.%
\pend
\count\Bfootins=1200%
\count\Afootins=1200%
\count\Cfootins=1200
%
%
%
%    %    %    %    Ende des Textes auf Blatt 18r
%
%
%