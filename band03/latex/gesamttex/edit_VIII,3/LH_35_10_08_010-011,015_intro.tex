%   % !TEX root = ../../VIII,3_Rahmen-TeX_8-1.tex
%
%
%   Band VIII, 3 N.~??Z (= N.~??Z.1 + N.~??Z.2)
%   Signatur/Tex-Datei: LH_35_10_08_010-011 + LH_35_10_08_015 
%   RK-Nr. 
%   Überschrift: 
%   Datierung: 
%   WZ: LEd8-WZ (insgesamt )
%.  SZ: ()
%.  Bilddateien (PDF): ()
%
%
\selectlanguage{ngerman}%
\frenchspacing%
%
\footnotesize%
\pstart%
\noindent%
\label{LH_35_10_08_010-011+LH_35_10_08_015_Vorbemerkung}%
Beide folgenden Entwürfe N.~27\textsubscript{1} und 27\textsubscript{2} sind von Leibniz eigenhändig datiert worden: N.~27\textsubscript{1} auf Juli 1686; N.~27\textsubscript{2} auf den 27. August 1689.
Die Datierung von N.~27\textsubscript{2} geht also auf die Zeit zurück, als Leibniz sich in Rom\protect\index{Ortsregister}{Rom} aufhielt (der Textträger selbst kann mangels eines Wasserzeichens nicht datiert werden), wohingegen N.~27\textsubscript{1} zweifelsohne in Deutschland\protect\index{Ortsregister}{Deutschland (Germania, Duitsland)} entstanden war (das Papier stammt wohl aus dem Harz).\protect\index{Ortsregister}{Harz} 
Trotz ihres zeitlichen und räumlichen Abstands weisen beide Entwürfe eine enge inhaltliche Verwandtschaft auf.
Der erste Teil von N.~27\textsubscript{2} (S.~\refpassage{LH_35_10_08_010-011_ersterTeil-1}{LH_35_10_08_010-011_ersterTeil-2}) ist eine ausführlichere und sorgfältigere Wiederaufnahme der in N.~27\textsubscript{1} durchgeführten Untersuchung, bei der die elastische Kraft einer Luftmasse, die in einem verschlossenen Behälter zunehmend komprimiert wird, zu bestimmen ist.
Im zweiten, symmetrischen Teil von N.~27\textsubscript{2}, für den es in N.~27\textsubscript{1} keine Entsprechung gibt, wird der komplementäre Fall untersucht, in dem die Kraft eines sich entladenden elastischen Körpers im Mittelpunkt steht.
Demnach lässt sich N.~27\textsubscript{2} als Weiterentwicklung und Vervollständigung von N.~27\textsubscript{1} betrachten.
Beide Texte weisen zudem inhaltliche Verwandtschaft mit dem Entwurf N.~21 \textit{De vibrationibus aeris tensi} auf: Bei allen steht im Hintergrund das Gedankenexperiment, mit dem in N.~14\textsubscript{3} (S.~\refpassage{LH_35_09_16_020v_kolbenmodell-1}{LH_35_09_16_020v_kolbenmodell-2}) und N.~14\textsubscript{7} (S.~\refpassage{LH_35_09_16_002_Beweis-1}{LH_35_09_16_002_Beweis-2}) sowie im Brief an E.~Mariotte von März/April 1683 (\textit{LSB} III,~3 N.~456, S.~795.21–796.3)\cite{01262} die Proportionalität zwischen Spannungskräften und Dehnungen elastischer Körper nachgewiesen werden soll.
\pend%
\pstart%
Auf den engen Zusammenhang der Entwürfe N.~27\textsubscript{1} und N.~27\textsubscript{2} weist Leibniz selbst in einer Randbemerkung am Kopf von N.~27\textsubscript{2} (S.~\pageref{LH_35_10_08_010r_Marg_1686}) hin: \textit{Haec accurate constituta Jul. 1686.}
Es ist auf den ersten Blick naheliegend, diesen Hinweis als eine unmittelbare Anspielung auf N.~27\textsubscript{1} zu deuten.
Allerdings könnte der Hinweis auch so gedeutet werden, dass der Text N.~27\textsubscript{2} bereits im Juli 1686 \textendash\ im Anschluss an N.~27\textsubscript{1} \textendash\ verfasst worden war und am 27. August 1687 nur überarbeitet wurde; zu dieser späteren Textschicht würden dann hauptsächlich die Randbemerkungen zu S.~\refpassage{LH_35_10_08_010r_suurihuomaus-1}{LH_35_10_08_010r_suurihuomaus-2}, S.~\refpassage{LH_35_10_08_010v_zewiteMarg-1}{LH_35_10_08_010v_zewiteMarg-2} und S.~\refpassage{LH_35_10_08_011r_dritteMarg-1}{LH_35_10_08_011r_dritteMarg-2} sowie der Schlussteil, S.~\refpassage{LH_35_10_08_011v_finisschedae_avdf-1}{LH_35_10_08_011v_finisschedae_avdf-2} gehören.
Diese zweite Deutung ist nach heutigem Wissensstand nicht auszuschließen.
Bemerkenswert ist, dass Leibniz in beiden Fällen mindestens einen der beiden Entwürfe auf seine Reise nach Italien\protect\index{Ortsregister}{Italien} mitgenommen und in Rom\protect\index{Ortsregister}{Rom} weiter bearbeitet haben muss.
Die enge Verbindung, die sowohl den Inhalt wie auch die Entstehung beider Texte umfasst, rechtfertigt jedoch in beiden Fällen die Entscheidung der Herausgeber, N.~27\textsubscript{1} und N.~27\textsubscript{2} zusammenhängend zu edieren.
\pend
%
% \newpage % vorläufig
\normalsize
\selectlanguage{latin}%
\frenchspacing%
%