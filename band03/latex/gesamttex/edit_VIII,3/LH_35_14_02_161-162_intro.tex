%   % !TEX root = ../../VIII,3_Rahmen-TeX_8-1.tex
%
%
%   Band VIII, 3 N.~??A34
%   Signatur/Tex-Datei: LH_35_14_02_161-162_intro
%   RK-Nr. 58256
%   Überschrift: De motu elaterii se restituentis
%   Modul: Mechanik / EAF (Elastizität)
%   Datierung: [Anfang August bis zweite Hälfte November 1689]
%   WZ: ––––
%   SZ: ––––
%   Bilddateien (PDF): ––––
%
%
\selectlanguage{ngerman}%
\frenchspacing%
%
\footnotesize%
\pstart%
\noindent%
Beide folgenden Aufzeichnungen N.~28\textsubscript{1} und 28\textsubscript{2} sind auf demselben Träger (dem Oktavbogen LH~XXXV~14,~2 Bl.~161\textendash162) überliefert und erweisen sich auch inhaltlich als eng miteinander verbunden.
In beiden Fällen handelt es sich um die mathematische Beschreibung der verzögerten Bewegung, mit der sich ein gespannter elastischer Körper zusammenzieht; dieser Körper wird in N.~28\textsubscript{1} als eine Saite (\textit{chorda}) beschrieben.
Aufgrund der gemeinsamen Überlieferung und der thematischen Verbindung bietet sich an, N.~28\textsubscript{1} und 28\textsubscript{2} zusammenhängend zu edieren.
\pend
\pstart
Auf Bl.~162~r\textsuperscript{o} ist ein von der Aufzeichnung N.~28\textsubscript{1} überschriebenes Satzfragment überliefert, das wohl den ersten Entwurf eines Briefes darstellt, den Leibniz am 6. August 1689 aus Rom an C.~Gudenus, kurmainzischen Residenten in Wien,\protect\index{Namensregister}{\textso{Gudenus} Johann Christoph 1632\textendash1705} sendete (\textit{LSB} I,~5 N.~250;\cite{01297} auf S.~46.8\textendash10 kommt eine nahezu gleiche Formulierung wie im Satzfragment auf Bl.~162~r\textsuperscript{o}\! vor).
Der Träger der Texte N.~28\textsubscript{1} und 28\textsubscript{2} weist zudem das gleiche Wasserzeichen wie die Handschrift LBr~425 Bl.~57\textendash58 auf, welche das vollständige Konzept des Briefes an Gudenus überliefert.
Hieraus lässt sich schließen, dass die Texte N.~28\textsubscript{1} und 28\textsubscript{2} zwischen Anfang August 1689 und spätestens Leibnizens Abreise aus Rom\protect\index{Ortsregister}{Rom} \textendash\ vermutlich am 21. November desselben Jahres (\textit{Chronik}, S.~98\cite{01236}) \textendash\ entstanden sind, wahrscheinlich aber noch etwa zu der Zeit, als der Brief an Gudenus abgefasst wurde.
Dagegen ist unwahrscheinlich, dass Leibniz den Oktavbogen nach Oberitalien\protect\index{Ortsregister}{Italien} mitgenommen hätte, wenn dort die Aufzeichnungen N.~28\textsubscript{1} und 28\textsubscript{2} nicht schon gestanden hätten.
\pend
\pstart
Mit Blick auf die relative Chronologie ist anzunehmen, dass gemäß der Anordnung der zwei Texte auf ihrem gemeinsamen Träger N.~28\textsubscript{1} wohl als erster verfasst wurde. 
Schließlich gilt es zu bemerken, dass beide Aufzeichnungen eine besondere inhaltliche Verwandtschaft mit N.~25 \textit{De restitutionis potentia} aufweisen.
Auch dort steht die verzögerte Bewegung eines sich zusammenziehenden elastischen Körpers im Mittelpunkt der Betrachtung; und auch N.~25 knüpft ausdrücklich und mit Tadel auf R.~Hookes\protect\index{Namensregister}{\textso{Hooke} (Hookius, Hook), Robert 1635\textendash1703} elastizitätstheoretische Ausführungen an. 
\pend
%
\newpage % REIN VORLÄUFIG ????
\selectlanguage{latin}%
\frenchspacing%
%
\normalsize