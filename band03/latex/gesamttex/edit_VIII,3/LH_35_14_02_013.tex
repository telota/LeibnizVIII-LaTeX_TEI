%   % !TEX root = ../../VIII,3_Rahmen-TeX_9-0.tex
%  
%   Band VIII, 3 N.~?? 	
%
%   RK-Nr. 	58226
%   Überschrift:			Regulae motus Cartesii
%   Datierung:		März 1689 bis März 1690 (Italienaufenthalt)
%
%   WZ: 				keins, aber ital. Papier
%
%
%
%
\selectlanguage{ngerman}
\frenchspacing
%
\begin{ledgroupsized}[r]{120mm}
\footnotesize
\pstart
\noindent\textbf{Überlieferung:}
\pend
\end{ledgroupsized}
%
\begin{ledgroupsized}[r]{114mm}
\footnotesize
\pstart \parindent -6mm
\makebox[6mm][l]{\textit{L}}%
Auszüge mit Bemerkungen aus 
\protect\index{Namensregister}{\textso{Descartes} (Cartesius, des Cartes), Ren\'{e} 1596\textendash1650}\textsc{R.~Descartes},
\cite{00035}\textit{Principia Philosophiae}, Pars~II, §§46\textendash52, Amsterdam 1644, S.~59\textendash61:
LH~XXXV~14, 2~Bl.~13. 
Ein Blatt~4\textsuperscript{o};
italienisches Papier;
oberer und linker Rand beschnitten.
Eine Seite auf Bl.~13~r\textsuperscript{o}; Rückseite leer.
\pend
\end{ledgroupsized}
%
%
\vspace{5mm}
\begin{ledgroup}
\footnotesize
\pstart
\noindent%
\textbf{Datierungsgründe:}
%
Bereits in den Pariser Jahren (wohl um 1675/1676) hatte Leibniz die sieben Stoßregeln aus dem zweiten Teil von
\protect\index{Namensregister}{\textso{Descartes} (Cartesius, des Cartes), Ren\'{e} 1596\textendash1650}Descartes' \cite{00035}\textit{Principia philosophiae} 
%
exzerpiert (\cite{01302}\textit{LSB} VI,~3 N.~15, bes.\ S.~216f.).
%
Die vorliegenden Auszüge der §§46\textendash52 sind allerdings höchstwahrscheinlich während Leibnizens Italienaufenthalts (März 1689 bis März 1690) entstanden, da sie auf italienischem Papier verfasst sind.
%
Zu dieser Zeit arbeitete Leibniz an einer umfassenden kritischen Darstellung von
%
\protect\index{Namensregister}{\textso{Descartes} (Cartesius, des Cartes), Ren\'{e} 1596\textendash1650}Descartes' Leben und Lehre. 
%
Davon zeugen verschiedene Stücke, die aufgrund des italienischen Wasserzeichens 
%
auf Frühjahr bis Herbst 1689 datierbar sind: \textit{LSB} VI, 4 N.~373\textendash375 
%
und vor allem die ausführlichen, in drei Fassungen überlieferten, 
%
\cite{02065}\glqq Notata quaedam G.\,G.\,L.\ circa vitam et doctrinam Cartesii\grqq\ (ebd.\ N.~376).
%
\pend
%
\pstart
Die Auseinandersetzung mit \protect\index{Namensregister}{\textso{Descartes} (Cartesius, des Cartes), Ren\'{e} 1596\textendash1650}Descartes
%
während des Italienaufenthalts wurde ab 1691 in den
%
\cite{02038}\glqq Anim\-adversiones ad Cartesii \textit{Principia}\grqq\
%
(erscheint in \textit{LSB} VI,~5; siehe auch \cite{02039}\textit{LPG}~IV, S.~354\textendash392)
%
fortgeführt. 
%
Darin besprach Leibniz ebenfalls die cartesischen Stoßregeln (\cite{02038}siehe z.B.\ LH~IV~1, 4a~Bl.~8~r\textsuperscript{o}\textendash9~v\textsuperscript{o}). 
%
\pend 
%
\pstart
%
Leibnizens Marginalien in seinem Handexemplar (GWLB, Leibn.\ Marg.~6) der \cite{00035}\textit{Principia} 
%
(\protect\vphantom)\cite{01301}\textit{LSB} VI, 4 N.~335\textsubscript{1}\protect\vphantom(),
%
die hauptsächlich dem Vergleich der Stoßlehren
\protect\index{Namensregister}{\textso{Descartes} (Cartesius, des Cartes), Ren\'{e} 1596\textendash1650}Descartes' 
%
und \protect\index{Namensregister}{\textso{Malebranche}, Nicolas 1638\textendash1715}Malebranches 
%
gewidmet sind,
%
dürften mit den vorliegenden Auszügen nicht näher verwandt sein, 
%
sondern vielmehr spätestens um 1687 entstanden sein,
%
im Vorfeld der Diskussion über das Kontinuitätsprinzip und die Fehler 
%
in \protect\index{Namensregister}{\textso{Descartes} (Cartesius, des Cartes), Ren\'{e} 1596\textendash1650}Descartes' 
%
und \protect\index{Namensregister}{\textso{Malebranche}, Nicolas 1638\textendash1715}Malebranches 
Stoßregeln 
%
(\protect\vphantom)\cite{02054}\textit{Extrait d'une Lettre de M.~L.\ sur un Principe Générale} in 
\cite{02002}\textit{Nouvelles de la République des lettres}, Juli 1687, S.~744\textendash753, \cite{02039}\textit{LPG} III, S.~51\textendash55; sowie 
\cite{02055}\textit{Principium quoddam generale}, 
\textit{LSB} VI, 4 N.~371\protect\vphantom();
%
siehe die editorische Vorbemerkung zu \cite{01301}\textit{LSB} VI, 4 N.~335.
\pend
%
\end{ledgroup}
%
%
\selectlanguage{latin}
\frenchspacing
% \newpage%
\vspace{8mm}
\pstart%
\normalsize%
\noindent%
\lbrack13~r\textsuperscript{o}\rbrack\
\pend
%
% Überschrift
\pstart
\centering
\edtext{Regulae motus Cartesii%
\protect\index{Sachverzeichnis}{regulae motus Cartesii}}{%
\lemma{Regulae motus Cartesii}%
\Cfootnote{%
\protect\index{Namensregister}{\textso{Descartes} (Cartesius, des Cartes), Ren\'{e} 1596\textendash1650}\textsc{R.~Descartes}, \cite{00035}\textit{Principia Philosophiae}, Pars~II, §§46\textendash52, Amsterdam 1644, S.~59\textendash61 (\cite{00120}\textit{DO} VIII.1, S.~68\textendash70).}}
\pend
\vspace{1.0em}
%
\pstart
%
\textso{Reg.~1}. Si duo corpora aequalia\protect\index{Sachverzeichnis}{corpora aequalia} 
%
aequali velocitate in partes contrarias tendentia concurrant%
\protect\index{Sachverzeichnis}{corpora aequalia aequali velocitate concurrentia}, 
%
ambo reflectentur qua venere celeritate et directione.\pend
%
 \pstart 
%
\textso{Reg.~2}. Si duo corpora 
%
\edtext{inaequalia%
\protect\index{Sachverzeichnis}{corpora inaequalia}}{\lemma{}\Bfootnote{ inaequalia \textit{erg. L}}} 
%
sibi aequali velocitate occurrant\protect\index{Sachverzeichnis}{corpora aequalia sibi inaequali velocitate occurrentia}
%
ambo eadem 
%
\edtext{qua ante}{\lemma{}\Bfootnote{qua ante \textit{erg. L}}}
%
velocitate ferentur et directione majoris. 
\pend 
\newpage
%
\pstart 
%
\textso{Reg.~3}. Si duo %
corpora aequalia\protect\index{Sachverzeichnis}{corpora aequalia} inaequali velocitate sibi 
%
\edtext{occurrant,\protect\index{Sachverzeichnis}{corpora inaequalia sibi aequali velocitate occurrentia} ambo}%
{\lemma{occurrant,}%
\Bfootnote{\textit{(1)} ambo pergent \textit{(2)} celerita \textit{(3)} directione \textit{(a)} et velocitate celerioris \textit{(b)} et velocitate \textit{(4)} ambo \textit{(a)} eadem velocitate \textit{(b)} aequ \textit{(5)} ambo \textit{L}}}  
%
aequali velocitate ferentur, et directione celerioris. Et tardius
%
\edtext{accipiet dimidium}{\lemma{accipiet}\Bfootnote{\textit{(1)} medium \textit{(2)}~dimidium \textit{L}}} 
%
differentiae celeritatum\protect\index{Sachverzeichnis}{differentia celeritatum}. 
%
\pend 
%
\pstart 
%
Reg.~4. Si corpus incurrat in quiescens\protect\index{Sachverzeichnis}{corpus quiescens} paulo majus,%
\protect\index{Sachverzeichnis}{incursus corporis in quiescens paulo majus} 
%
ipsum non movebit sed repelletur qua venit celeritate, 
%
\edtext{\textit{quia %
corpus quiescens\protect\index{Sachverzeichnis}{corpus quiescens} magis resistit magnae celeritati quam parvae}.}{%
\lemma{\textit{quia} \lbrack...\rbrack\ \textit{parvae}}%
\Cfootnote{%
\cite{00035}a.a.O., §49, S.~60.}}
%
 \pend 
%
\pstart 
%
Reg.~5. Si 
%
\edtext{corpus incurrat}{\lemma{corpus}\Bfootnote{\textit{(1)} \textbar\ in \textit{streicht Hrsg.} \textbar\ \textit{(a)}~quie \textit{(b)}~cu \textit{(2)}~incurrat \textit{L}}} 
%
in quiescens\protect\index{Sachverzeichnis}{corpus quiescens} paulo 
%
\edtext{minus\protect\index{Sachverzeichnis}{incursus corporis in quiescens paulo minus} ibunt}%
{\lemma{minus}\Bfootnote{\textit{(1)}~ita \textit{(2)} ibunt \textit{L}}} 
%
\edtext{simul}{\lemma{}\Bfootnote{simul \textit{erg. L}}} 
%
directione incurrentis, et velocitate tali, 
%
\edtext{ut motus}{%
\lemma{}%
\Bfootnote{%
ut \textbar\  quantitas \textit{streicht Hrsg.}\ \textbar\ motus \textit{L}}}
%
sit idem qui ante. Ut si corpus incurrens sit duplum, motus erit tertia parte celeritatis. 
%
\pend 
%
\pstart 
%
Reg.~6. Si corpus incurrat in quiescens\protect\index{Sachverzeichnis}{corpus quiescens} aequale%
\protect\index{Sachverzeichnis}{incursus corporis in quiescens aequale}
%
partim impellet, partim reflecteretur, 
%
ut si habeat 4 gradus velocitatis, dabit 
%
\edtext{ipsi excipienti}{%
\lemma{ipsi}%
\Bfootnote{%
\textit{(1)} \textit{C} unum \textit{(2)} excipienti~\textit{L}%
}}
%
\edtext{\textit{unum gradum, et cum tribus residuis reflectetur}.}{%
\lemma{\textit{unum} \lbrack...\rbrack\ \textit{reflectetur}}%
\Cfootnote{%
\cite{00035}a.a.O., §51, S.~61.}}
%
(+ Non intelligo hanc regulam. \lbrack+\rbrack)
%
\pend 
%
\pstart 
%
Reg.~7. Si corpus assequatur aliud, et celerius esset minus, sed %
excessus celeritatis\protect\index{Sachverzeichnis}{excessus celeritatis} 
%
major quam %
magnitudinis\protect\index{Sachverzeichnis}{excessus magnitudinis}\lbrack,\rbrack\ ita transferetur celeritas, 
%
ut ambo simul in easdem partes moveantur; sin %
excessus celeritatis\protect\index{Sachverzeichnis}{excessus celeritatis} 
%
minor quam %
magnitudinis\protect\index{Sachverzeichnis}{excessus magnitudinis} incurrens in contrarias partes toto suo motu reflecteretur.
%
\pend
\count\Afootins=1200%
\count\Bfootins=1200%
\count\Cfootins=1200