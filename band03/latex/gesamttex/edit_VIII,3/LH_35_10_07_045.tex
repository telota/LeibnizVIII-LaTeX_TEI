%   % !TEX root = ../../VIII,3_Rahmen-TeX_9.tex
%  
%   Band VIII, 3		Rubrik STOSS
%
%   Signatur/Tex-Datei:	LH_35_10_07_045
%
%   RK-Nr. 	60060
%
%   Überschrift: 	(keine)
%   
%   Unterrubrik:			Nachträge zu VIII, 2?
%
%   edlabels:			1
%
%   Diagramme: 		0
%
%
%   NB: 						(Anmerkungen)					??
%
%
%
\selectlanguage{ngerman}
\frenchspacing
%
\begin{ledgroupsized}[r]{120mm}
\footnotesize
\pstart
\noindent\textbf{Überlieferung:}
\pend
\end{ledgroupsized}
%
\begin{ledgroupsized}[r]{114mm}
\footnotesize
\pstart \parindent -6mm
\makebox[6mm][l]{\textit{L}}%
Konzept:
LH~XXXV~10, 7~Bl.~45.
Ein Zettel (ca.~3~x~5,5~cm.);
unterer Rand ausgefranst.
Zwei Seiten.
\pend
\end{ledgroupsized}
%
%
\vspace{5mm}
\begin{ledgroup}
\footnotesize
\pstart
\noindent%
\textbf{Datierungsgründe:}  %Datierung: 1671\textendash September 1677
Das Verhalten zweier sich bewegender Körper nach ihrem Zusammenstoß wird in vorliegendem Stück so beschrieben, wie es den grundlegenden Stoßregeln entspricht, die Leibniz im ersten auf 1671 (2.\,H.) datierten Entwurf seiner \textit{Summa Hypotheseos physicae novae} (\textit{LSB} VI, 2 N.~48\textsubscript1\cite{01276}) aufstellt: Gemäß dem dort formulierten dritten Fall (\textit{LSB} VI, 2 N.~48\textsubscript1, S.~338.27\,f.) bewegen sich die hier (Randbemerkung zu S.~\refpassage{35_10_07_045_1a}{35_10_07_045_1b}) beschriebenen Körper \textit{A} und \textit{B} in Richtung des ursprünglich schnelleren Körpers mit der Differenz ihrer Ausgangsgeschwindigkeiten; laut dem zweiten in der \textit{Summa} angeführten Fall (\textit{LSB} VI, 2 N. 48\textsubscript1, S. 338.17\,f.) kommen zwei Körper, die mit gleicher Geschwindigkeit zusammenstoßen, zur Ruhe, ganz wie in vorliegendem Stück zu lesen (S.~\refpassage{LH_35_10_07_045_Ruhe-1}{LH_35_10_07_045_Ruhe-2}). 
Als Terminus post quem für N.~\ref{RK60060} wird damit 1671 gewählt, das Jahr, in welchem die \textit{Hypothesis physica nova} (\textit{LSB} VI, 2 N. 40\cite{00256}) ihren Abschluss findet, von der die \textit{Summa} ihren Ausgang nimmt.
Auf die Zeit vor Paris geht auch die hier angesprochene rein mathematische Betrachtungsweise (\textit{leges pure} oder \textit{mere mathematicae}) zurück (\textit{Theoria motus abstracti}, Winter 1670/71: \textit{LSB} VI, 2 N.~41\cite{00259}), die Leibniz einer physikalischen Betrachtung der Stoßbewegung gegenüberstellt (\textit{leges systematicae}, \textit{Systema}), in der konkrete Bedingungen wie die der Elastizität zu berücksichtigen sind (\glqq Theoria motus concreti\grqq\ bzw. \textit{Hypothesis physica nova}, 1670/71: \textit{LSB} VI, 2 N. 40\cite{00256}). 
Beide Betrachtungsweisen lassen sich auch noch in seinen Auszügen \textit{Zu Descartes’ Principia philosophiae} ausmachen, die Leibniz gegen Ende seiner Pariser Zeit anfertigt (\textit{LSB} VI, 3 N.~15\cite{01302}, hier S. 215.31\textendash216.2, S. 216.13\textendash17, S. 216.26\,~f.), doch verlieren sie im Laufe der Pariser Jahre offenbar an Gegensätzlichkeit, während die Phänomene selbst mehr und mehr Leibnizens Interesse finden, woran die Auseinandersetzung mit Mariottes\protect\index{Namensregister}{\textso{Mariotte}, Edme, Seigneur de Chazeuil ca. 1620\textendash1684} \textit{Traité de la percussion}\cite{00311} (\textit{LSB} VIII, 2 N. 50\cite{01292}) ganz entscheidend Anteil gehabt haben dürfte. Nach der Pariser Zeit stellt sich für Leibniz die konkret physikalische Natur nicht mehr als Gegensatz zu einer mathematischen Beschreibung der Phänomene dar, sondern vielmehr soll Erfahrung dafür die Voraussetzung liefern.
Dieser Ansatz dürfte ihn nicht nur zu einer erneuten Auseinandersetzung mit Mariotte\protect\index{Namensregister}{\textso{Mariotte}, Edme, Seigneur de Chazeuil ca. 1620\textendash1684} sowie auch mit den Stoßgesetzen von Huygens\protect\index{Namensregister}{\textso{Huygens} (Hugenius, Ugenius, Hugens, Huguens), Christiaan 1629\textendash1695} geführt haben, die im Frühjahr 1677 erfolgt, und sich in Stücken (N.~\ref{RK60070}, N.~\ref{RK60127}, N.~\ref{RK57266-1}) niederschlägt, die davor und danach entstanden sind und gleichsam eine Brücke zwischen der Pariser Zeit und der Entstehungszeit des \textit{De corporum concursu} (N.~\ref{dcc_00}) schlagen.
Geradezu programmatisch kommt dieser Ansatz, mit dem die Gegensätzlichkeit sich auflöst, in einem wahrscheinlich für Jean Bertet\protect\index{Namensregister}{\textso{Bertet} (Berthet), Jean 1622\textendash1692} bestimmten \cite{02047}Brief von September 1677, der nicht abgegangen ist und sich nur als Fragment erhalten hat, zum Ausdruck (\textit{LSB} II, 1\ \lbrack2.\,Aufl.\rbrack, 1 N.~158a, siehe hierzu auch die Datierungsgründe in N.~\ref{RK52278}).
Darin macht Leibniz rückblickend seine Heimreise nach Deutschland zum Ausgangspunkt neuerlicher Überlegungen zu den Gesetzen der Bewegung und nimmt sich vor, diese gleichsam als erster so vollständig wie korrekt zu formulieren und empirisch zu beweisen; er müsse hierfür aber noch grundlegende Erfahrungen bzw. Experimente (\textit{experiences fondamentales}) anstellen, könne dabei aber auch auf schon gemachte oder leicht auszuführende Versuche zurückgreifen (\cite{02047}\textit{LSB} II, 1\ \lbrack2.\,Aufl.\rbrack, 1 N.~158a, hier S. 572.12\textendash16.\,22\,f.). Eine Auflistung solcher \textit{Expériences à faire sur le mouvement} hat sich in N.~\ref{RK52278} erhalten, das sich ähnlich datieren lässt wie das Brieffragment. Die rein mathematische Betrachtungsweise ist damit nicht vergessen, wie das Akustik-Stück N.~\ref{60273} zeigen kann, wo Leibniz jedoch zugleich Abstand davon nimmt (siehe die Datierungsgründe dort). Es ist aber wenig wahrscheinlich, dass sie weiterhin, wie in vorliegendem Stück, sich als Gegensatz oder getrennt zu einer physikalischen Betrachtung der Stoßphänomene darstellt. In Stücken der Jahre 1677 und später verfolgt Leibniz den im September 1677 formulierten Vorsatz, der hier als Terminus ante quem für die Datierung von N.~\ref{60060} gewählt wird.
\pend 
\end{ledgroup}
%
\selectlanguage{latin}
\frenchspacing
% \newpage%
\vspace{8mm}
\pstart%
\normalsize%
\noindent%
\lbrack45~r\textsuperscript{o}\rbrack\
%
\edlabel{35_10_07_045_1a}%
\edtext{}{% A-Footnote
{\xxref%
{35_10_07_045_1a}{35_10_07_045_1b}}%
\lemma{}%
\Afootnote{%
\textit{An den Rändern, tlw.\ quer zur Schreibrichtung}: Ex legibus mere mathematicis\protect\index{Sachverzeichnis}{leges merae mathematicae}, 
si corpus \textit{A} celeritate 3 et corpus \textit{B} celeritate 2 concurrant, pergent ambo
versu\edtext{\lbrack s\rbrack}{\lemma{}\Bfootnote{versu \textit{L \"{a}ndert Hrsg.}}}
3\textsuperscript{\lbrack a\rbrack} celeritate 1. 
Video concursum\protect\index{Sachverzeichnis}{concursus} 
aequalium fingendum aequalem\lbrack,\rbrack\ alioqui nullus esset ictus\protect\index{Sachverzeichnis}{ictus}\textsuperscript{\lbrack b\rbrack} in \textlangle concurrentibus.\textrangle
\newline
\newline
{\footnotesize \textsuperscript{\lbrack a\rbrack}\glqq 3\grqq\ bezeichnet die Bewegungsrichtung des schnelleren Körpers.\quad \textsuperscript{\lbrack b\rbrack}versu\,\textit{L} \textit{ändert Hrsg.}\quad \textsuperscript{\lbrack c\rbrack}ictus \textbar~in \textit{streicht Hrsg.} \textbar\ in~\textit{L}}}}%
%
In omni ictu\protect\index{Sachverzeichnis}{ictus} aequales sunt
%
 corporum concurrentium\protect\index{Sachverzeichnis}{corpora concurrentia} vires,\protect\index{Sachverzeichnis}{vis} et omne corpus non nisi propria sua 
%
\edtext{\lbrack vi\rbrack\protect\index{Sachverzeichnis}{vis}}{%
\lemma{}%
\Bfootnote{%
via %
\textit{L ändert Hrsg.}%
}}
%
ab alio resilit. Et omnis corporis \textit{A} a corpore \edtext{\textit{B}}{\lemma{}\Bfootnote{\textit{B} \textit{erg. L}}} 
%
impulsus\protect\index{Sachverzeichnis}{impulsus} nihil aliud est quam vis\protect\index{Sachverzeichnis}{vis corporis} corporis
%
\textit{A} ab alio \edtext{\textit{B}}{\lemma{}\Bfootnote{\textit{B} \textit{erg.~L}}} 
resilientis.%
\edlabel{35_10_07_045_1b}
%
\lbrack45~v\textsuperscript{o}\rbrack\
Possibile est fortasse concipi eundem in corporibus servari conatum\protect\index{Sachverzeichnis}{conatus} secundum 
%
leges \edtext{compositionum motus\protect\index{Sachverzeichnis}{compositio motus} pure}{\lemma{compositionum}\Bfootnote{\textit{(1)}~pure \textit{(2)}~motus pure~\textit{L}}}
%
mathematicas,\protect\index{Sachverzeichnis}{leges compositionum motus pure mathematicae}
%
sed quia nulla sunt corpora perfecte solida\protect\index{Sachverzeichnis}{corpus perfecte solidum}
%
ideo fit, ut hoc non obstante leges systematicae\protect\index{Sachverzeichnis}{leges systematicae} 
%
in mundo serventur\lbrack,\rbrack\ ut si concipiamus constare ex meris solidis globis\protect\index{Sachverzeichnis}{globus solidus} aequalibus, 
%
et fluido intercurrente\protect\index{Sachverzeichnis}{fluidum intercurrens} a quo vis \edtext{Elastica\protect\index{Sachverzeichnis}{vis elastica}, oriuntur}{\lemma{Elastica,}\Bfootnote{\textit{(1)}~oritur haec ipsa \textit{(2)}~oriuntur~\textit{L}}} 
%
hae leges motus\protect\index{Sachverzeichnis}{leges motus} quas Systema\protect\index{Sachverzeichnis}{Systema} 
%
requirit ita ut \edlabel{LH_35_10_07_045_Ruhe-1}duo corpora $=$ aequali celeritate concurrentia\protect\index{Sachverzeichnis}{corpora duo aequali celeritate concurrentia} quiescere ponantur.\edlabel{LH_35_10_07_045_Ruhe-2} \pend