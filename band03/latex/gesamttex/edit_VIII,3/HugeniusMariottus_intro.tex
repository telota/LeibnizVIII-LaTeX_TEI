%   % !TEX root = ../../VIII,3_Rahmen-TeX_9-0.tex
%  
%   Vorbemerkung zu RK 	57267_1 + 60343
%
%   Datierung:		März bis Mai 1677 (a. St.?)
%
%
%
%
\selectlanguage{ngerman}
\frenchspacing
%
%\vspace{5mm}
\begin{ledgroup}
\footnotesize
\pstart
\noindent
%
Die vorliegenden drei Stücke weisen eine gemeinsame Entstehungsgeschichte auf:
%
N.~\ref{57267_2} und N.~\ref{57267_3}, in sehr kurzem zeitlichen Abstand bzw.\ unmittelbar nacheinander auf demselben Träger verfasst,
%
halten, in Form von kommentierten Auszügen, die Ergebnisse von Leibnizens erster bekannter Auseinandersetzung mit den Stoßgesetzen von 
%
\protect\index{Namensregister}{\textso{Huygens} (Hugenius, Ugenius, Hugens, Huguens), Christiaan 1629\textendash1695}Huygens und \protect\index{Namensregister}{\textso{Mariotte}, Edme, Seigneur de Chazeuil ca. 1620\textendash1684}Mariotte 
%
in der Hannoveraner Zeit fest;
%
N.~\ref{60343} ist die kurze, summarische Reinschrift eines Abschnitts der Besprechung von
%
\protect\index{Namensregister}{\textso{Huygens} (Hugenius, Ugenius, Hugens, Huguens), Christiaan 1629\textendash1695}Huygens in N.~\ref{57267_2}.
\pend
%
\pstart
N.~\ref{57267_2} enthält einen fast vollständigen Auszug aus, und eine kritische Besprechung von, Huygens' französischem Aufsatz
%
\cite{00529}\glqq Regles du mouvement dans la rencontre des corps\grqq\ (\cite{00157}\textit{JS} vom 18.~März 1669).
%
Bereits in der Mainzer Zeit, wenige Monate nach Erscheinen der lateinischen Fassung von \protect\index{Namensregister}{\textso{Huygens} (Hugenius, Ugenius, Hugens, Huguens), Christiaan 1629\textendash1695}Huygens' Aufsatz
%
(\cite{01067}\glqq A summary account of the laws of motion\grqq, \cite{00158}\textit{PT} von April 1669),
%
hatte Leibniz sich mit dessen Stoßlehre befasst, wovon das Stück 
%
\glqq De rationibus motus\grqq\ (\cite{02025}\textit{LSB} VI, 2 N.~38\textsubscript{1}) zeugt. 
%
N.~\ref{57267_3} ist \protect\index{Namensregister}{\textso{Mariotte}, Edme, Seigneur de Chazeuil ca. 1620\textendash1684}Mariottes \cite{00311}\textit{Traité de la percussion} (Paris 1673) gewidmet.
%
Leibnizens erste ausführliche Auseinandersetzung mit \protect\index{Namensregister}{\textso{Mariotte}, Edme, Seigneur de Chazeuil ca. 1620\textendash1684}Mariottes \cite{00311}Abhandlung
%
fand in der Pariser Zeit statt und schlug sich in den kommentierten Auszügen aus den letzten Monaten von 1674 nieder (\cite{01292}\textit{LSB} VIII, 2 N.~50).
\pend
%
\pstart
Zur Datierung der kommentierten Auszüge N.~\ref{57267_2} und N.~\ref{57267_3} 
%
lassen sich ihre Verhältnisse mit den anderen beiden auf diesem Bogen überlieferten, von Leibniz eigenhändig datierten Stücken heranziehen.
%
Auf dem Bogen finden vier Stücke Platz: 
%
Auf der ersten Seite (Bl.~144~r\textsuperscript{o}) befindet sich hauptsächlich N.~\ref{57267_1} von März 1677 (hinzu kommt der Schlussteil von N.~\ref{57268}); 
%
auf der zweiten und dritten Seite folgen N.~\ref{57267_2} und N.~\ref{57267_3} unmittelbar und nahtlos aufeinander; 
%
die vierte Seite (Bl.~145~v\textsuperscript{o}) überliefert den Hauptteil von N.~\ref{57268} von Mai 1677.
%
Unter Annahme einer durchgehenden Beschreibung des Bogens erscheint die Hypothese einer Entstehung der mittleren Stücke, N.~\ref{57267_2} und  N.~\ref{57267_3}, zwischen den beiden übrigen (also zwischen März und Mai 1677), zunächst plausibel. Folgende zwei Umstände bekräftigen die Hypothese.
%
\pend
%
\pstart
Erstens bezeugt die Lage einer auf Bl.~144~r\textsuperscript{o} befindlichen, dennoch zu N.~\ref{57267_2} zugehörigen
%
Berechnung eines \glqq siebten Falls\grqq\
%
(S.~\refpassage{37_05_144-145_19a}{37_05_144-145_19b}) die spätere Entstehung von N.~\ref{57267_2} relativ zu N.~\ref{57267_1}.
%
Diese Rechnung wurde auf Bl.~144~r\textsuperscript{o}, nach Abfassung des dort befindlichen Hauptteils von N.~\ref{57267_1} (März 1677),
%
am frei gebliebenen linken Seitenrand quer zur Schreibrichtung nachgetragen. 
%
Sie kann allerdings nicht N.~\ref{57267_1} zugeordnet werden; vielmehr ist sie als Bestandteil von N.~\ref{57267_2} anzusehen, 
%
da Leibniz in der zweiten Hälfte von N.~\ref{57267_2}
%
die Regel aus \protect\index{Namensregister}{\textso{Huygens} (Hugenius, Ugenius, Hugens, Huguens), Christiaan 1629\textendash1695}Huygens' \cite{00529}Aufsatz
%
algebraisch formalisiert, seine zehn Fälle auflistet und sie einzeln bewertet. Der Randtext auf Bl.~144~r\textsuperscript{o} entspricht tatsächlich 
%
\protect\index{Namensregister}{\textso{Huygens} (Hugenius, Ugenius, Hugens, Huguens), Christiaan 1629\textendash1695}Huygens' siebtem Fall.
%
Wahrscheinlich hat Leibniz diese Rechnung während der Abfassung von N.~\ref{57267_2} wegen Platzmangels auf der Rückseite, d.h.\ auf Bl.~144~r\textsuperscript{o}, niedergeschrieben.
%
Erst danach brachte er im noch unbeschriebenen Bereich am linken Rand von Bl.~144~r\textsuperscript{o}den Schlussteil von N.~\ref{57268} (Mai 1677) unter.
%
\pend
%
\pstart
Zweitens gibt die Randanmerkung am oberen linken Rand von Bl.~145~v\textsuperscript{o} 
%
(S.~\refpassage{37_05_144r-145v_8a}{37_05_144r-145v_8b}), 
%
mit der Leibniz das Stück N.~\ref{57268} auf Mai 1677 datiert, 
%
zugleich Aufschluss über die Entstehung von N.~\ref{57267_3}.
%
Dort merkt er an,
%
dass die \glqq soeben erst\grqq\ (also wohl in N.~\ref{57268}) gesicherten Erkenntnisse zum Stoß 
%
\glqq in den vorangehenden Seiten\grqq\ (d.h.\ Bl.~144~r\textsuperscript{o}\textendash145~r\textsuperscript{o}) 
%
noch nicht berücksichtigt worden waren. 
%
Dieser Umstand bestätigt, dass Leibniz N.~\ref{57268} erst nach N.~\ref{57267_3} 
%
(somit auch nach N.~\ref{57267_2} und N.~\ref{57267_1}) abgefasst hat.
\pend
%
\pstart
Demnach müssen N.~\ref{57267_2} und N.~\ref{57267_3} zwischen  N.~\ref{57267_1} und N.~\ref{57268}, also zwischen März und Mai 1677 entstanden sein. %
%
Einen weiteren Beleg von Leibnizens direkter Auseinandersetzung mit 
%
\protect\index{Namensregister}{\textso{Huygens} (Hugenius, Ugenius, Hugens, Huguens), Christiaan 1629\textendash1695}Huygens' \cite{00529}\glqq Regles\grqq\  
%
zu dieser Zeit bietet das kurze Exzerpt der §§5f.\ des Absatzes in N.~\ref{57266_1} von März 1677 
(S.~\refpassage{37_05_161-162r_9a}{37_05_161-162r_9b}).
\pend 
\newpage
\pstart
Während der Abfassung von N.~\ref{57267_2}, also ebenfalls zwischen März und Mai 1677, 
%
entstand mit hoher Wahrscheinlichkeit auch das kurze, undatierte Stück N.~\ref{60343}, 
%
das auf einem Zettel überliefert ist und editorisch mit dem Titel \glqq Schedula de decem casibus Hugenianis\grqq\ versehen wird.
%
Darin hält Leibniz die Ergebnisse seiner mathematischen Prüfung der \protect\index{Namensregister}{\textso{Huygens} (Hugenius, Ugenius, Hugens, Huguens), Christiaan 1629\textendash1695}Huygens'schen Regel summarisch fest. 
%
Zwar enthält das Stück weder eine ausdrückliche Nennung von Huygens, 
%
noch bietet es einen Kontext für die Rechnungen; 
%
dennoch kann es zweifelsfrei als weitgehend wörtliche Reinschrift 
%
dreier Passagen aus dem Mittelteil von N.~\ref{57267_2} eingeordnet werden: 
S.~\refpassage{37_05_144-145_25a}{37_05_144-145_25b}, 
S.~\refpassage{37_05_144-145_26a}{37_05_144-145_26b}
und S.~\refpassage{37_05_144-145_27a}{37_05_144-145_27b}. 
%
Dabei wird die Huygens'sche Stoßregel zunächst algebraisch ausgedrückt
%
und die Formel anschließend 
%
auf dessen zehn Fälle einzeln angewendet, was Leibniz zufolge 
%
jedes Mal zu einem falschen oder ungewissen Ergebnis führt.
%
Allerdings entspringen sämtliche Fehler weder Huygens' Regel noch ihrer an sich korrekten Formalisierung. 
%
Sie beruhen vielmehr, in N.~\ref{60343} wie bereits in N.~\ref{57267_2}, 
%
auf Leibnizens fehlerhafter Handhabung der Vorzeichen in der Vektoraddition.
%
\pend
%
\pstart
Sowohl die beträchtliche Nähe im Wortlaut als auch der Umstand, dass Leibniz 
%
seinen grundlegenden Fehler noch nicht aufgedeckt hat,
%
legen für N.~\ref{60343} 
%
eine Entstehung unmittelbar nach Abfassung der entsprechenden Passagen von N.~\ref{57267_2} nahe. 
%
Letzteres Stück deutet in einem kurz darauffolgenden Abschnitt (S.~\refpassage{37_05_144-145_29a}{37_05_144-145_29b}) 
%
eine Rehabilitierung der \protect\index{Namensregister}{\textso{Huygens} (Hugenius, Ugenius, Hugens, Huguens), Christiaan 1629\textendash1695}Huygens'schen 
%
Berechnungen in einigen Fällen an; daher ist N.~\ref{60343} womöglich noch vor diesem Abschnitt entstanden.
\pend
%
\pstart
Einen Hinweis auf die mögliche Bestimmung von N.~\ref{60343} bieten die letzten Sätze von N.~\ref{57267_2} (S.~\refpassage{37_05_144-145_28a}{37_05_144-145_28b}):
%
Dort äußert Leibniz die Absicht, \protect\index{Namensregister}{\textso{Huygens} (Hugenius, Ugenius, Hugens, Huguens), Christiaan 1629\textendash1695}Huygens 
%
auf die von ihm aufgedeckte (vermeintliche) Inkonsistenz seines Ansatzes aufmerksam zu machen.
%
Möglicherweise war die \textit{Schedula} als Vorlage für eine dahingehende Mitteilung an Huygens angedacht.
%
Allerdings war seit Huygens' Abreise aus Paris im Juli 1676 der briefliche Kontakt abgebrochen; er lebte erst mit Leibnizens Brief vom 8.\ (18.) September 1679 wieder auf (\cite{02046}\textit{LSB} III, 2 N.~346).
%
\pend
\end{ledgroup}