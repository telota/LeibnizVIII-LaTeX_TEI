%   % !TEX root = ../../VIII,3_Rahmen-TeX_9-0.tex
%  
%   Band VIII, 3 N.~?? 	
%   Signatur/Tex-Datei:	LH_AA_BB_CC_252-256
%   RK-Nr. 	61042
%			
%   Titel: 			Rezension zu A. Parent, Élémens, 1700
%   Datierung:		Juni 1701
%   Diagramme: 		0
%   edlabels:			2
%
%   Erstaufnahme:			P. Just 2020
%
%   NB: 						(Anmerkungen)					??
%
%
%
\selectlanguage{ngerman}
\frenchspacing
%
\begin{ledgroupsized}[r]{120mm}
\footnotesize
\pstart
\noindent\textbf{Überlieferung:}
\pend
\end{ledgroupsized}
%
\begin{ledgroupsized}[r]{114mm}
\footnotesize
\pstart
\parindent -6mm
\makebox[6mm][l]{\textit{E}}%
Rezension:
\cite{01023}\textit{Acta Eruditorum}, Juni 1701, S.~252\textendash256.
Wie zu Leibnizens Zeiten üblich erschien die Rezension anonym.
\pend%
\end{ledgroupsized}
%
%
\selectlanguage{latin}
\frenchspacing
% \newpage%
\vspace{8mm}
\pstart%
\normalsize%
\noindent%
\lbrack S.~252\rbrack\
\pend 
%
\pstart
\centering{\textso{Élémens de méchanique et de physique, %keine Errata/Corrigenda
où l'on donne Geometriquement les principes du choc et des equilibres entre toutes sortes des corps,
avec l'explication naturelle des machines fondamentales,
par M.~Parent, de l'Academie Royale des Sciences.}}\protect\index{Sachverzeichnis}{Académie Royale des Sciences}
%
\pend
%\vspace{3mm}
%
\pstart\centering
i.e. 
\pend
%\vspace{3mm}
%
\pstart
\centering
\textso{Elementa mechanicae et physicae,}\\
quibus principia conflictus et aequilibrii in omni corporum genere\\
Geometrice traduntur, una cum explicatione naturali machinarum fundamentalium:\\
Autore Dn.\ \protect\index{Namensregister}{\textso{Parent}, Antoine 1666\textendash1716}\textso{Parent,} 
e\\
Regia Scientiarum Academia.\protect\index{Sachverzeichnis}{Academia regia (Académie Royale des Sciences)}\\
Parisiis apud Flor.\ et Petr.\ Delaulne, 1700. 12.\\
Constant plag.~21, et fig.\ aen.\ plag.~4.  
\pend
\vspace{1em}
\count\Bfootins=1100%
\count\Afootins=1200%
\count\Cfootins=1100
%
\pstart \noindent Quoniam Physica pendent ex Mechanicis seu ex motuum legibus, 
%
hinc Autor ingeniosus, explicans doctrinam percussionis corporum, 
%
librum censuit jure vocari \textit{Elementa Mechanica et Physica}. 
%
\edtext{Leges percussionis derivat ex principio concursus aequilibrium facientis
%
cum aliis motibus compositi.}{%
\lemma{Leges \lbrack...\rbrack\ compositi}%
\Cfootnote{%
\textsc{A.~Parent}, \textit{Élémens de méchanique et de physique}, Paris 1700, Preface, S.~\lbrack1\rbrack.}}
%
\edtext{Et ait se initio putasse, primo sibi istam methodum occurrisse, 
%
sed postea comperisse, \protect\index{Namensregister}{\textso{Huygens} (Hugenius, Ugenius, Hugens, Huguens), Christiaan 1629\textendash1695}Hugenium,  
\protect\index{Namensregister}{\textso{Wallis} (Wallisius), John 1616\textendash1703}Wallisium 
et  \protect\index{Namensregister}{\textso{Mariotte}, Edme, Seigneur de Chazeuil ca. 1620\textendash1684}Mariottum eodem principio usos:
%
quoniam tamen non satis omnibus innotuisse videatur, et ideo celebres autores principia peculiaria 
%
habeant, et inter se dissentiant, ideo sibi operae pretium visum hanc \lbrack S.~253\rbrack\  
%
Methodum fusius exponere, et ad varia applicare.}{%
\lemma{Et ait \lbrack...\rbrack\ applicare}%
\Cfootnote{%
a.a.O., S.~\lbrack3f.\rbrack.\cite{01500}}}  
%
\edtext{Multum quoque se debere ait Domini \protect\index{Namensregister}{\textso{Renau d'Eli\c{c}agaray}, Bernard, 1652\textendash1719}Renaut libro de gubernatione navis,  
%
et Mechanicae Domini \protect\index{Namensregister}{\textso{La Hire}, Philippe de 1640\textendash1718}De la Hire, 
%
sed maxime Domino \protect\index{Namensregister}{\textso{Sauveur}, Joseph 1653\textendash1716}Sauveur,  
%
cujus magna sint in Mechanicen et Physicam merita.}{%
\lemma{Multum \lbrack...\rbrack\ merita}%
\Cfootnote{%
a.a.O., S.~\lbrack4f.\rbrack.\cite{01500} Siehe 						
%
\protect\index{Namensregister}{\textso{Renau d'Eli\c{c}agaray}, Bernard, 1652\textendash1719}
\textsc{B.~Renau d'Eliçagaray}, %
\cite{02027}\title{Théorie de la manoeuvre des vaisseaux}, Paris 1688 und 
%
\protect\index{Namensregister}{\textso{La Hire}, Philippe de 1640\textendash1718}
\textsc{P.~de~la~Hire}, %
\cite{02026}\title{Traité de mécanique}, Paris 1695. 		
}}
%
\edlabel{AE_1701_252-256_1a}%
\edtext{}{%
{\xxref%
{AE_1701_252-256_1a}{AE_1701_252-256_1b}}%
\lemma{Scripsisse \lbrack...\rbrack\ liberis}%
\Cfootnote{%
\textsc{Parent}, \textit{Élémens}, S.~\lbrack7f.\rbrack.\cite{01500}%
}}%
Scripsisse se octo abhinc annis librum 
%
fundatum super principio parallelogrammorum, per quod explicuerit 
%
omnem doctrinam aequilibrii circa punctum fixum et centra gravitatis 
%
sed cum eum tractatum Domino \protect\index{Namensregister}{\textso{Sauveur}, Joseph 1653\textendash1716}Sauveur 
%
ostendisset, ut Academiae Regiae\protect\index{Sachverzeichnis}{Academia regia (Académie Royale des Sciences)} offerretur, hunc monuisse, 
%
\edtext{jam Dominum \protect\index{Namensregister}{\textso{Varignon}, Pierre de 1654\textendash1722}Varignonium Mechanicen suam eidem principio inaedificasse.}{%
\lemma{jam \lbrack...\rbrack\ inaedificasse}%
\Cfootnote{%
\protect\index{Namensregister}{\textso{Varignon}, Pierre de 1654\textendash1722}
\textsc{P.~de~Varignon}, %
\cite{02028}\title{Projet d'une nouvelle méchanique}, Paris 1687.}}
%
Itaque cum praeterea intellexisset, 
%
Patrem \protect\index{Namensregister}{\textso{Lamy} (L'Amy, Lami), Bernard (Bernhard) 1640\textendash1715}Lamy, 
%
et adhuc alios hanc inventionem sibi tribuere,
%
ideo sese ab eo labore destitisse, et maluisse talia Mechanica deducere 
%
ex principio de maximis et minimis, gloria inventi aliis relicta, 
%
etsi non minus quam illi eam sibi tribuere posset, cum nec ipsi fuerint 
%
primi; praeterea principium parallelogrammorum non succedere 
%
pro concursibus liberis.%
\edlabel{AE_1701_252-256_1b}
%
\edtext{Galilaei\protect\index{Namensregister}{\textso{Galilei} (Galilaeus, Galileus), Galileo 1564\textendash1642} propositiones se omisisse, quod sint in 
%
imaginatione ipsius unice fundatae, etsi pulchrae videantur et experientiis 
%
satis consentiant. Tempus enim, quo  \protect\index{Namensregister}{\textso{Galilei} (Galilaeus, Galileus), Galileo 1564\textendash1642}Galilaeus utatur in acceleratione 
%
corporum descendentium, esse rem imaginariam, qua possimus 
%
carere in ratiocinando.}{%
\lemma{Galilaei \lbrack...\rbrack\ ratiocinando}%
\Cfootnote{%
\textsc{Parent}, \textit{Élémens}, S.~\lbrack8f.\rbrack.\cite{01500}}}
%
\pend
%
\pstart Librum suum 
%
\edtext{dividit \textso{in tres partes}.}{%
\lemma{\textso{in tres partes}}%
\Cfootnote{%
\protect\index{Namensregister}{\textso{Parent}, Antoine 1666\textendash1716}Parents \cite{01500}Werk hat eigentlich vier Teile.}}
%
\textso{Prima} agit de motu et concursu rectilineo. 
%
\edtext{Ibi cap.~4 definit tempus, quod sit effectus successivus
%
et uniformis, stabilitus vel assumtus ad mensurandum quicquid est 
%
successivum. Ita spatium quod homo absolvat aequali passu, esse tempus, 
%
sic aquam ex clepsydra ejusdem semper altitudinis elapsam, esse tempus etc.}{%
\lemma{Ibi cap.~4 \lbrack...\rbrack\ etc.}%
\Cfootnote{% 
\textsc{Parent}, \textit{Élémens}, S.~9\textendash11.\cite{01500}}}
%
\edtext{Potentiam vel vim (force) definit cap.~5 statum
%
praesentem motus corporis, comparatum statui tali alterius corporis. 
%
Hos status non posse differe, nisi respectu massae et celeritatis.}{%
\lemma{Potentiam \lbrack...\rbrack\ celeritatis}%
\Cfootnote{%
a.a.O., S.~11\textendash15.\cite{01500}}}
%
\edtext{Ex qua definitione inferre se posse putat, corpus duplo celerius 
%
altero aequali, esse etiam duplo fortius, praesertim quoniam
%
in aequilibrio massa et celeritas se mutuo compensant, cap.~6. 
%
Ubi etiam putat, vim destrui in corporibus cum aequilibrio concurrentibus, 
%
nec perfecte elasticis.}{%
\lemma{}%
\Cfootnote{%
\hspace{-2.45mm}Ex qua \lbrack...\rbrack\ elasticis: \hspace{1mm}a.a.O., S.~15\textendash19.\cite{01500}}} 
%
\edtext{Ponit deinde 12 Axiomata (c.~8) iisque 
%
in supplemento septem alia adjungit quanquam fortasse ad pauciora multo reduci possent.}{%
\lemma{Ponit \lbrack...\rbrack\ possent}%
\Cfootnote{%
a.a.O., S.~22\textendash24.\cite{01500}}}
%
\edtext{Cap.~11 explicans motus reciprocos, qui 
%
scilicet cuilibet corpori attribui possunt, statuit systemata \hspace{-0.3mm}\protect\index{Namensregister}{\textso{Kopernikus}, Nikolaus 1473\textendash1543}Copernici
%
\hspace{-0.3mm}et \hspace{-0.3mm}\protect\index{Namensregister}{\textso{Brahe}, Tycho (Tyco) 1546\textendash1601}Tychonis \hspace{-0.3mm}coexistere, \hspace{-0.3mm}seu \hspace{-0.3mm}coincidere,}{%
\lemma{Cap.~11 \lbrack...\rbrack\ coincidere}%
\Cfootnote{%
a.a.O., S.~33\textendash37.\cite{01500}}}
%
\hspace{-0.3mm}etsi \hspace{-0.3mm}in \hspace{-0.3mm}eo \hspace{-0.3mm}maneat \hspace{-0.3mm}discrimen \hspace{-0.3mm}\lbrack S.~254\rbrack\  
%
reale, quod 
%
\edtext{in hypothesi \protect\index{Namensregister}{\textso{Brahe}, Tycho (Tyco) 1546\textendash1601}Tychonis Sol ingreditur in orbitam Martis, 
%
in \protect\index{Namensregister}{\textso{Kopernikus}, Nikolaus 1473\textendash1543}Copernicana 
%
\hspace{-0.4mm}vero \hspace{-0.4mm}nihil \hspace{-0.3mm}tale \hspace{-0.3mm}contingit, \hspace{-0.3mm}ut \hspace{-0.3mm}Mars \hspace{-0.3mm}veniat \hspace{-0.3mm}in \hspace{-0.3mm}locum, ubi
%
\hspace{-0.3mm}aliquando \hspace{-0.3mm}Sol \hspace{-0.3mm}fuit.}{%
\lemma{in hypothesi \lbrack...\rbrack\ Sol fuit}%
\Cfootnote{%
\protect\index{Namensregister}{\textso{Brahe}, Tycho (Tyco) 1546\textendash1601}\textsc{T.~Brahe}, 
\cite{00327}\textit{De mundi aetherei recentioris phaenomenis Liber secundus}, Prag 1603, hier S.~185\textendash201; 
\protect\index{Namensregister}{\textso{Kopernikus}, Nikolaus 1473\textendash1543}\textsc{N.~Kopernikus}, 
\cite{02059}\textit{De revolutionibus},
Nürnberg 1543, 
Lib.~I cap.~X, fol.~7~v\textsuperscript{o}\textendash10~r\textsuperscript{o}.}}
%
\hspace{-0.3mm}Nimirum \hspace{-0.3mm}phaenomena%
\pend
\newpage
\count\Bfootins=1200%
\count\Afootins=1200%
\count\Cfootins=1200
\pstart
\noindent ostendunt, Martem interdum 
%
nobis esse propiorem, quam est Sol. 
%
\edtext{Reliquis hujus partis capitibus 
%
examinat casus concursuum rectilineorum tam cum, quam sine 
%
oppositione inter corpora mollia, semidura seu semielastica, et perfecte 
%
elastica.}{%
\lemma{Reliquis \lbrack...\rbrack\ elastica}%
\Cfootnote{%
\cite{01500}\textsc{Parent}, \textit{Élémens}, chap.~XII\textendash XXIV, S.~37\textendash82.}}
%
\edlabel{AE_1701_252-256_2a}\edtext{In quibus non dissentit quoad conclusiones ab iis, quae 
%
\protect\index{Namensregister}{\textso{Mariotte}, Edme, Seigneur de Chazeuil ca. 1620\textendash1684}Mariottus aliique dedere.}{%
\lemma{In quibus \lbrack...\rbrack\ dedere}%
\Cfootnote{%
\protect\index{Namensregister}{\textso{Mariotte}, Edme, Seigneur de Chazeuil ca. 1620\textendash1684}
\textsc{E.~Mariotte}, \cite{00311}\textit{Traité de la percussion}, Paris 1673; %
%
\protect\index{Namensregister}{\textso{Wallis} (Wallisius), John 1616\textendash1703}\textsc{J.~Wallis}, \cite{01065}\glqq A summary account \lbrack...\rbrack\ of the general laws of motion\grqq, \cite{00158}\textit{PT} III (1668\textendash1669), Januar 1669, S.~864\textendash866; 
%
\protect\index{Namensregister}{\textso{Wallis} (Wallisius), John 1616\textendash1703}\textsc{Ders.}, \cite{00301}\textit{Mechanica}, London 1670\textendash1671, Pars~III, Cap.~XI, S.~660\textendash682 (\cite{01008}\textit{WO} I, S.~1002\textendash1015) sowie Cap.~XIII, S.~686\textendash707 (\cite{01008}\textit{WO} I, S.~1018\textendash1031); 
%
\protect\index{Namensregister}{\textso{Wren} (Wrennus), Christopher 1632\textendash1723}\textsc{C.~Wren}, \cite{01066}\glqq Theory concerning the same subject\grqq, \cite{00158}\textit{PT} III (1668\textendash1669), Januar 1669, S.~867f.;
%
\protect\index{Namensregister}{\textso{Huygens} (Hugenius, Ugenius, Hugens, Huguens), Christiaan 1629\textendash1695}\textsc{C.~Huygens}, \cite{00529}\glqq Regles du mouvement dans la rencontre des corps\grqq, \cite{00157}\textit{JS} (Pariser Ausgabe), 18.~März 1669, S.~22\textendash24 (\cite{00113}\textit{HO} XVI, S.~179\textendash181).}}\edlabel{AE_1701_252-256_2b}
%
\edtext{Vim ictus aestimat a concursu cum aequilibrio,  
%
et ita definit c.~17, ut si \textit{A} et \textit{B} cum aequilibrio concurrat in \textit{D}, sit vis 
%
ictus factum ex \textit{A} in \textit{AD}, vel ex \textit{B} in \textit{BD}.}{%
\lemma{Vim ictus \lbrack...\rbrack\ \textit{BD}}%
\Cfootnote{%
\cite{01500}\textsc{Parent}, \textit{Élémens}, S.~56\textendash58.}} 
\edtext{Vim relativam quidem agnoscit manere eandem, sed c.~16 et 18 defendit, vim absolutam crescere vel diminui per concursum. Agnoscit etiam, si corpora sint homogenea, centrum gravitatis suam celeritatem et directionem non mutare, mutare tamen eam in heterogeneis;}{%
\lemma{Vim relativam \lbrack...\rbrack\ heterogeneis}%
\Cfootnote{%
a.a.O., S.~50\textendash56 und S.~58\textendash65.\cite{01500}}} 
%
hinc ut utramque una regula complecteretur, ideo \edtext{in \textit{Diario Eruditorum} Parisino 4 Maji 1699}{%
\lemma{in \textit{Diario Eruditorum} Parisino 4 Maji 1699}%
\Cfootnote{%
\protect\index{Namensregister}{\textso{Parent}, Antoine 1666\textendash1716}%
\textsc{A.~Parent}, %	
\cite{02029}\glqq Loy universelle pour quelque multitude de corps que ce soit\grqq, 	%
\cite{00157}\textit{JS} (Pariser Ausgabe), 4.~Mai 1699, %
S.~197\textendash200.}}
%
hanc proposuerat: in omnibus concursibus in eadem recta corpora 
%
conservant legem aequilibrii respectu puncti immensitatis, quod 
%
movetur eadem celeritate post ictum, qua ante ictum ibat centrum 
%
commune massae. 
%
Sed ne hic quidem explicuit, quid intelligatur per \textso{punctum immensitatis}, 
%
\edtext{etsi regulam repetat sub finem capitis 20.}{%
\lemma{etsi \lbrack...\rbrack\ 20}%
\Cfootnote{%
\cite{01500}\textsc{Parent}, \textit{Élémens}, S.~68\textendash73.}}
\pend
%
\pstart \textso{Parte secunda} agitur de concursu corporum, quae circulari motu 
%
feruntur, ubi eodem fere modo, ut in rectilineis concursibus, procedit. 
%
\edtext{Radium massae vocat cap.~2, in quo corpora concurrerent cum aequilibrio: hinc radii hujus celeritatem non mutari.}{%
\lemma{Radium \lbrack...\rbrack\ mutari}%
\Cfootnote{%
a.a.O., S.~84\textendash86.\cite{01500}}} 
%
\edtext{Inde cap.~9 agit de centro virium, quod  \protect\index{Namensregister}{\textso{Mariotte}, Edme, Seigneur de Chazeuil ca. 1620\textendash1684}Mariottus et alii appellarant centrum percussionis;}{%
\lemma{Inde \lbrack...\rbrack\ percussionis}%
\Cfootnote{%
\cite{01500}a.a.O., S.~104\textendash109; siehe \protect\index{Namensregister}{\textso{Mariotte}, Edme, Seigneur de Chazeuil ca. 1620\textendash1684}
\textsc{Mariotte}, \cite{00311}\textit{Traité de la percussion},
Seconde partie, Prop.~XIV,
S.~267\textendash271
und
\protect\index{Namensregister}{\textso{Wallis} (Wallisius), John 1616\textendash1703}\textsc{Wallis}, \cite{00301}\textit{Mechanica}, Pars~III, Cap.~XI, Prop.~XV S.~677\textendash682 (\cite{01008}\textit{WO} I, S.~1012\textendash1015).}}
%\pend
%%\newpage
%%\count\Bfootins=1200%
%%\count\Afootins=1200%
%%\count\Cfootins=1200
%\pstart
%\noindent 
\makebox[1.0\textwidth][s]{\edtext{quemadmodum c.~14 centrum temporis vocat appellatione generaliore, }{%
\lemma{que}%
\Cfootnote{%
\hspace{-2.4mm}madmodum \lbrack...\rbrack\ generaliore: \hspace{1mm}\protect\index{Namensregister}{\textso{Parent}, Antoine 1666\textendash1716}\textsc{Parent}, \textit{Élémens}, S.~122\textendash127.}}%
\edlabel{KZeitz111}\edtext{}{{\xxref{KZeitz111}{KZeitz112}}%
{%	
\lemma{quod \lbrack...\rbrack\ oscillationis}%
\Cfootnote{%
\protect\index{Namensregister}{\textso{Huygens} (Hugenius, Ugenius, Hugens, Huguens), Christiaan 1629\textendash1695}%
\textsc{C.~Huygens}, %
\cite{00123}\title{Horologium Oscillatorium}, 	
 Paris 1673, Pars IV, S.~91\textendash156 %
(\cite{00113}\textit{HO} %
XVIII, S.~242\textendash359).}}}%
quod \protect\index{Namensregister}{\textso{Huygens} (Hugenius, Ugenius, Hugens, Huguens), Christiaan 1629\textendash1695}Hugenius}
\pend
\newpage
\pstart
\noindent ad gravitatem respiciens, vocaret centrum oscillationis.\edlabel{KZeitz112}
%
%
\edlabel{KZeitz109}\edtext{}{{\xxref{KZeitz109}{KZeitz110}}%
{\lemma{Putat \lbrack...\rbrack\ virium}%
\Cfootnote{%
\textsc{Parent}, \textit{Élémens}, S.~127\textendash131.\cite{01500}}}}%
Putat autem et Autor cap.~15,
%\pend
%\newpage
%\pstart
%\noindent 
naturam gravitatis ex hoc ipso
valde illustrari, quod experientia ostendat, in oscillationibus
idem esse centrum temporis et centrum percussionis seu virium;\edlabel{KZeitz110}
%
uti sane 
%
\edtext{Mariottus\protect\index{Namensregister}{\textso{Mariotte}, Edme, Seigneur de Chazeuil ca. 1620\textendash1684} ostenderat, centrum suum percussionis idem esse cum 
%
centro oscillationis vel agitationis  \protect\index{Namensregister}{\textso{Huygens} (Hugenius, Ugenius, Hugens, Huguens), Christiaan 1629\textendash1695}Hugeniano.}{%	
\lemma{Mariottus \lbrack...\rbrack\ Hugeniano}%
\Cfootnote{%
\protect\index{Namensregister}{\textso{Mariotte}, Edme, Seigneur de Chazeuil ca. 1620\textendash1684}\textsc{Mariotte},
\cite{00311}\title{Traité de la percussion},
Seconde partie, Prop.~XIX,
S.~288\textendash\lbrack291\rbrack.\hspace{-2mm}}}
%
\edtext{Et cap.~16 notat, centrum massae in corporibus gravibus cum centro gravitatis coincidere.}{%
\lemma{Et cap.~16 \lbrack...\rbrack\ coincidere}%
\Cfootnote{%
\textsc{Parent}, \textit{Élémens}, S.~131\textendash137.\cite{01500}}}
%
\edtext{Unde et de centris gravitatis\lbrack,\rbrack\ virium et temporis in variis figuris nonnihil tractat.}{%
\lemma{Unde \lbrack...\rbrack\ tractat}%
\Cfootnote{%
a.a.O., S.~138\textendash140.\cite{01500}\hspace{-2mm}}}
%
Inde \textso{parte tertia} transit ad motus obliquos, et compositos. 
%
\edtext{Ubi notat, duos motus perpendiculares inter se composi\lbrack S.~255\rbrack tos  
%
id habere singulare, quod se mutuo nec juvant nec impediunt, cum 
%
acutum inter se angulum facientes sese adjuvent, at obtusum facientes 
%
sibi sint contrarii, cap.~4.}{%
\lemma{Ubi \lbrack...\rbrack\ cap.~4}%
\Cfootnote{%
a.a.O., S.~159\textendash161.\cite{01500}\hspace{-2mm}}}  
%
\edtext{Et eadem transfert ad motus circulares, cap.~5.}{%
\lemma{Et \lbrack...\rbrack\ cap.~5}%
\Cfootnote{%
a.a.O., S.~161\textendash166.\cite{01500}}}
%
\edtext{Inde agit de reflexione corporis oblique incidentis cap.~6,}{%
\lemma{Inde \lbrack...\rbrack\  cap.~6}%
\Cfootnote{%
a.a.O., S.~166\textendash172.\cite{01500}}} 
%
\edtext{et de variis corporum in idem incursibus, cap.~7 seqq.\ ubi rursus tuetur, centrum massae inter homogenea motum servare, et vires relativas manere, sed absolutas imminui aut crescere;}{%
\lemma{cap.~7 \lbrack...\rbrack\ crescere}%
\Cfootnote{%
a.a.O., chap.~VII, S.~172\textendash174 und chap.~XIII, S.~197\textendash203.\cite{01500}}}	
%
sed non applicat huc punctum immensitatis, quo supra erat usus. 
%
\edtext{Porro et cap.~18 agit de corporibus plicatilibus, qualia sunt catenae, aut funes, abstrahendo hoc 
%
loco ab eorum massa propria; ubi quoque agit de trochleis.}{%
\lemma{Porro \lbrack...\rbrack\ trochleis}%
\Cfootnote{%
a.a.O., S.~223\textendash238.\cite{01500}\hspace{-2mm}}}
%
\edtext{Inde cap.~19 et 20 procedit ad vectes et rotas,}{%
\lemma{Inde \lbrack...\rbrack\ rotas}%
\Cfootnote{%
a.a.O., S.~238\textendash248 und S.~248\textendash258.\cite{01500}}}
%
\edtext{et c.~21 ad plana inclinata,}{%
\lemma{et \lbrack...\rbrack\ inclinata}%
\Cfootnote{%
a.a.O., S.~258\textendash262.\cite{01500}\hspace{-4mm}}}
%
\edtext{et c.~22 (quod est partis tertiae ultimum) de polyedro findente seu de cuneo agit.}{%
\lemma{et c.~22 \lbrack...\rbrack\ agit}%
\Cfootnote{%
a.a.O., S.~262\textendash273.\cite{01500}\hspace{-5mm}}}
\pend
%
\pstart \textso{Pars quarta} destinata est corporibus fluidis inter se vel cum solido combinatis; 
%
\edtext{ubi cap.~3 promittit Dominus Autor, se aliquando tractatum de Hydraulicis editurum.}{%
\lemma{u}%
\Cfootnote{\hspace{-2.4mm}bi cap.~3 \lbrack...\rbrack\ editurum: \hspace{1mm}Eigentlich \cite{01500}chap.~II, S.~290.\cite{01500}}}
%
\edtext{Notat cap.~1, concursus fluidorum hoc habere, quod sunt permanentes seu durabiles, cum solidorum concursus sint quasi instantanei.}{%							
\lemma{Notat \lbrack...\rbrack\ instantanei}%
\Cfootnote{%
a.a.O., S.~274\textendash286.\cite{01500}}} 
%
Hinc in fluidis operae pretium 
%
futurum est considerare, quantum ictus duret, et quod tanto major 
%
est quantitas agentis fluidi, quanto major velocitas. 
%
Ubi post alios notat, maximam velocitatem, quam grave in fluido per 
%
accelerationem acquirere possit, eam esse, qua fluidum sursum motum 
%
(ut in \makebox[1.0\textwidth][s]{aquae jactibus) hoc grave sustineret. 
%
\edtext{Attingit etiam aequilibrium liquorum cap.~5.}{%
\lemma{Attingit \lbrack...\rbrack\ cap.~5}%
\Cfootnote{%
a.a.O., S.~316\textendash337.\cite{01500}}}}
\pend
\newpage
\pstart
\noindent
%
\edtext{Et paucis agit c.~6 de rotis, quae trahuntur ab animalibus, affirmatque magnas rotas hic exiguis aequipollere.}{%
\lemma{Et paucis \lbrack...\rbrack\ aequipollere}%
\Cfootnote{%
a.a.O., S.~338\textendash342.\cite{01500}}} 
%
\edtext{Et cap.~7 tractat de lineis curvis formatis per corpora flexibilia, quae non extenduntur: ubi concipit ea trahi vel impelli aut conatibus parallelis inter se, aut perpendicularibus ad ipsas curvae partes.}{%
\lemma{Et cap.~7 \lbrack...\rbrack\ partes}%
\Cfootnote{%
a.a.O., S.~343\textendash358.\cite{01500}}} 
%
Parallelis inter se agunt pondera, quae intelligi possunt vel esse (primo) ipsae curvae 		
%
partes ut in catena, vel haec ponderis exortia fingendo, intelligi possunt 
%
pondera imposita, quae vel sint inter se (secundo) ut spatia figurae respondentia seu	
%
superimminentia inde a recta horizonti parallela; vel sint (tertio) inter se aequalia. 
%
\edtext{In primo casu ait elementa distantiarum a dicta recta horizonti parallela, seu \textit{dy} esse ut arcus a curva abscissos inde a vertice, usque ad punctum, ubi est elementum \textit{dy}.}{%
\lemma{In primo \lbrack...\rbrack\ elementum \textit{dy}}%
\Cfootnote{%
a.a.O., S.~347.\cite{01500}%
}}
%
\edtext{Secundum casum determinare se ait ex principio quodam Domini \protect\index{Namensregister}{\textso{La Hire}, Philippe de 1640\textendash1718}la Hire,
cui se hanc gloriam relinquere ait, etsi ipse duabus aliis viis eodem perveniat.}{%
\lemma{Secundum \lbrack...\rbrack\ perveniat}%
\Cfootnote{%
a.a.O., S.~349f.\cite{01500}%
}}
 \lbrack S.~256\rbrack\  
%
\edtext{Tertio casu curvam statuit esse parabolam.}{%
\lemma{Tertio \lbrack...\rbrack\ parabolam}%
\Cfootnote{%
a.a.O., S.~347.\cite{01500}}}
%
Caeterum operae pretium est addere: 
%
\edtext{problema primum in catenae\protect\index{Sachverzeichnis}{catena} figura indaganda\protect\index{Sachverzeichnis}{problema in catenae figura indaganda} propositum fuisse a \protect\index{Namensregister}{\textso{Galilei} (Galilaeus, Galileus), Galileo 1564\textendash1642}Galilaeo,}{%
\lemma{problema \lbrack...\rbrack\ Galilaeo}%
\Cfootnote{%
\protect\index{Namensregister}{\textso{Galilei} (Galilaeus, Galileus), Galileo 1564\textendash1642}\textsc{G.~Galilei}, %
\cite{00050}\title{Discorsi}, Giornata Quarta, Leiden 1638, S.~283\textendash287 %	%Leidener Ausgabe
(\cite{00048}\textit{GO} VIII, S.~309\textendash312). %
}}
%
\edtext{et primum ab Illustri \protect\index{Namensregister}{\textso{Leibniz} (Leibnitius, GGL), Gottfried Wilhelm 1646\textendash1716}Leibnitio nova ipsius Methodo\protect\index{Sachverzeichnis}{Methodus nova calculi differentialis} calculi differentialis\protect\index{Sachverzeichnis}{calculus differentialis} fuisse solutum;}{%
\lemma{et primum \lbrack...\rbrack\ solutum}%
\Cfootnote{%
\protect\index{Namensregister}{\textso{Leibniz} (Leibnitius, GGL), Gottfried Wilhelm 1646\textendash1716}%
\textsc{G.\,W.~Leibniz}, %
\cite{02033}\glqq De linea in quam flexile se pondere proprio curvat\grqq, 	%
\cite{01023}\textit{AE}, Juni 1691, %
S.~277\textendash281.}}	
%
\edtext{ingeniosissimum quoque 
%
\protect\index{Namensregister}{\textso{Bernoulli}, Johann 1667\textendash1748}Joh.~Bernoullium, 
%
intellecto Leibnitiano successu, (\protect\vphantom)etsi nondum
%
publicata solutione aut ejus artificio\protect\vphantom() ejusdem quidem calculi\protect\index{Sachverzeichnis}{calculus differentialis} ope,  
%
proprio tamen Marte eodem pervenisse, quemadmodum ex \cite{01023}\textit{Actis} nostris constat;}{%
\lemma{ingeniosissimum \lbrack...\rbrack\ constat}%								
\Cfootnote{%
\protect\index{Namensregister}{\textso{Bernoulli}, Johann 1667\textendash1748}%				
\textsc{Johann Bernoulli}, %
\cite{02034}\glqq Solutio problematis funicularii\grqq, 		
\cite{01023}\textit{AE}, Juni 1691, % 				
S.~274\textendash276.}} 
%
et Autorem nostrum eadem calculandi ratione hic uti. 
%
\edtext{Notat etiam primum casum, (\protect\vphantom)qui scilicet est lineae Catenariae\protect\index{Sachverzeichnis}{linea Catenaria} 
%
(Gallis la \protect\index{Sachverzeichnis}{Chainette}Chainette)\lbrack\protect\vphantom()\rbrack\ posse, 
%
ob maximum descensum centri gravitatis catenae,\protect\index{Sachverzeichnis}{catena} eo deduci, 
%
ut quaeratur curva, ubi $y\sqrt{(dxdx + dydy)}:2a$ sit maximum.}{%
\lemma{Notat \lbrack...\rbrack\ maximum}%
\Cfootnote{%
\textsc{Parent}, \textit{Élémens}, S.~350.\cite{01500}}}
%
\edtext{Sed haec, inquit, relinquo combinationibus summatoriis Algebristarum;}{%
\lemma{}%
\Cfootnote{%
\hspace{-2.4mm}Sed \lbrack...\rbrack\ Algebristarum: \hspace{1mm}a.a.O., S.~350.\cite{01500}}}
%
quanquam interim facile agnoscat, artificium inveniendi curvas, quibus maximum praestetur, 
%
non ex combinationibus Algebristarum, neque etiam \makebox[1.0\textwidth][s]{ex summationibus tantum, 
%
sed ex alio singulari artificio pendere, quod nuper demum}
\pend
\newpage
\pstart
\noindent prodiit, 		
%
\edtext{cum linea brevissimi descensus\protect\index{Sachverzeichnis}{linea brevissimi descensus} a Domino \protect\index{Namensregister}{\textso{Bernoulli}, Johann 1667\textendash1748}Joh.~Bernoullio proposita 
%
et ab ipso pariter ac paucis aliis soluta fuisset.}{%
\lemma{cum linea \lbrack...\rbrack\ fuisset}%
\Cfootnote{%
Siehe \protect\index{Namensregister}{\textso{Bernoulli}, Johann 1667\textendash1748}%				
\textsc{Johann Bernoulli}, %
\cite{02035}\glqq Problema novum ad cujus solutionem Mathematici invitantur\grqq, 	% 			
\cite{01023}\textit{AE}, Juni 1696, %
S.~269;
%
\protect\index{Namensregister}{\textso{Newton} (Neutonus), Isaac 1643\textendash1727}%
\textsc{I.~Newton}, %
\cite{02030}\glqq De ratione temporis\grqq, 	% 			
\cite{00158}\textit{PT} XIX (1695\textendash1697), 
Februar 1697, %
S.~424f.;
%
sowie die in den \cite{01023}\textit{Acta Eruditorum} vom Mai 1697 erschienenen Aufsätze:
%
\protect\index{Namensregister}{\textso{Leibniz} (Leibnitius, GGL), Gottfried Wilhelm 1646\textendash1716}%
\textsc{G.\,W.~Leibniz}, %
\cite{02031}\glqq Communicatio \lbrack...\rbrack\ solutionum problematis curvae celerrimi descensus\grqq, %
S.~201\textendash205;
%
\protect\index{Namensregister}{\textso{Bernoulli}, Johann 1667\textendash1748}%
\textsc{Johann Bernoulli}, %
\cite{02060}\glqq Curvatura radii in diaphanis non uniformibus, Solutioque Problematis\grqq, 	%
S.~206\textendash211;
%
\protect\index{Namensregister}{\textso{Bernoulli}, Jacob 1655\textendash1705}%
\textsc{Jacob Bernoulli}, %
\cite{02061}\glqq Solutio problematum fraternorum\grqq, 	%
S.~211\textendash214;
%
\protect\index{Namensregister}{\textso{L'Hospital} (L'H\^{o}pital, Hospitalius), Guillaume Fran\c{c}ois Antoine de 1661\textendash1704}%
\textsc{G.\,F.\,A.\ de~L'Hospital}, %
\cite{02062}\glqq Solutio problematis de linea celerrimi descensus\grqq, 	%
S.~217\textendash220.}} 
%
\edtext{Hanc autem Methodum maximi descensus centri gravitatis\protect\index{Sachverzeichnis}{Methodus maximi descensus centri gravitatis} putat Dominus \protect\index{Namensregister}{\textso{Parent}, Antoine 1666\textendash1716}Autor, ad solum casum primum, non ad duos reliquos pertinere.}{%
\lemma{Hanc \lbrack...\rbrack\ pertinere}%
\Cfootnote{%
\textsc{Parent}, \textit{Élémens}, S.~350.\cite{01500}%
}}
%
Quae omnia nos iis, qui haec profundius examinarunt, consideranda relinquimus. 
%
\edtext{Cap.~9, 10, agit de figuris in fluido motis, ubi et quaedam de gubernaculo et derivatione post Dominus \protect\index{Namensregister}{\textso{Renau d'Eli\c{c}agaray}, Bernard, 1652\textendash1719}Regnault}{%
\lemma{Cap.~9, 10 \lbrack...\rbrack\ Regnault}%
\Cfootnote{%
a.a.O., S.~369\textendash386 und 387\textendash404.\cite{01500}}}
%
\edtext{et cap.~11 de proportione ponderis columnae aeris.}{%
\lemma{et cap.~11 \lbrack...\rbrack\ aeris}%
\Cfootnote{%
a.a.O., S.~404\textendash414.\cite{01500}}}
%
\edtext{Tandem cap.~12 et 13 exponit modum haec experiundi, et speciatim modum, quo pendula ictibus suis concursuum experimenta exhibent,}{%
\lemma{Tandem \lbrack...\rbrack\ exhibent}%
\Cfootnote{%
a.a.O., S.~414\textendash424 und 425\textendash431.\cite{01500}}}
%
\edtext{quod jam praestiterat \protect\index{Namensregister}{\textso{Mariotte}, Edme, Seigneur de Chazeuil ca. 1620\textendash1684}Mariottus,}{%
\lemma{quod \lbrack...\rbrack\ Mariottus}%
\Cfootnote{%
\protect\index{Namensregister}{\textso{Mariotte}, Edme, Seigneur de Chazeuil ca. 1620\textendash1684}
\textsc{Mariotte}, \cite{00311}\textit{Traité de la percussion}.}}
%
sed a Domino \protect\index{Namensregister}{\textso{Parent}, Antoine 1666\textendash1716}Autore magis promovetur. 
%
Speramus Clarissimum \protect\index{Namensregister}{\textso{Parent}, Antoine 1666\textendash1716}Autorem in hac scientia ornanda 
%
porro perrecturum. Interea optamus, ut 
%
\edtext{\textso{Nova Dynamices Scientia},\protect\index{Sachverzeichnis}{Dynamica}\protect\index{Sachverzeichnis}{scientia nova Dynamices} cujus aliquoties in his \cite{01023}\textit{Actis} facta est mentio,}{%
\lemma{\textso{Nova} \lbrack...\rbrack\ mentio}%
\Cfootnote{%
Siehe bspw.\
%
\protect\index{Namensregister}{\textso{Leibniz} (Leibnitius, GGL), Gottfried Wilhelm 1646\textendash1716}\textsc{G.\,W.~Leibniz},
\cite{02040}\glqq De causa gravitatis\grqq, %
\cite{01023}\textit{AE}, Mai 1690, % 		
S.~228\textendash239;
%
\protect\index{Namensregister}{\textso{Leibniz} (Leibnitius, GGL), Gottfried Wilhelm 1646\textendash1716}\textsc{Ders.}, 
\cite{02064}\glqq De primae philosophiae emendatione\grqq, %			
\cite{01023}\textit{AE}, März 1694, % 		
S.~110\textendash112;
%
\protect\index{Namensregister}{\textso{Leibniz} (Leibnitius, GGL), Gottfried Wilhelm 1646\textendash1716}\textsc{Ders.}, 
\cite{02032}\glqq Specimen dynamicum\grqq, Pars I, %			
\cite{01023}\textit{AE}, April 1695, % 		
S.~145\textendash157;
%
\protect\index{Namensregister}{\textso{Leibniz} (Leibnitius, GGL), Gottfried Wilhelm 1646\textendash1716}\textsc{Ders.}, 
\cite{XX}\glqq De ipsa natura\grqq, %			
\cite{01023}\textit{AE}, September 1698, % 		
S.~427\textendash440.}}
%
ab Inventore Illustri in lucem producatur, et pulcherrimum illud naturae arcanum 
%
de eadem semper potentiae motricis absolutae\protect\index{Sachverzeichnis}{potentia motrix absoluta} (debito sensu acceptae) 
%
quantitate\protect\index{Sachverzeichnis}{quantitas potentiae motricis absolutae} servanda explicetur et stabiliatur. 
%
Hujus enim doctrinae defectus Dominum 
%
\protect\index{Namensregister}{\textso{Parent}, Antoine 1666\textendash1716}Autorem 
%
(cujus librum jam recensuimus) 
%
et alios solius potentiae relativae\protect\index{Sachverzeichnis}{potentia relativa} conservatione\protect\index{Sachverzeichnis}{conservatio potentiae relativae} contentos esse coegit.
%
\pend
\count\Bfootins=1100%
\count\Afootins=1200%
\count\Cfootins=1100