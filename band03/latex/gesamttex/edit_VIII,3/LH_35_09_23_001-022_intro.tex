%   % !TEX root = ../../VIII,3_Rahmen-TeX_8-1.tex
%
%
%   Band VIII, 3 N.~??S01.00
%   Signatur/Tex-Datei: LH_35_09_23_001-022_intro
%   RK-Nr. 41208 (i. ii. ii-ii. iii. iv. v. vi. ii-vi. vii.) + 57263 (viii.) + 57264 (ix.) + 57265 (x.) + 60235 (ii-vi. tabula)
%   \ref{dcc_00}
%   Überschrift: [De corporum concursus]
%   Modul: Mechanik / Stoß ()
%   Datierung: Januar 1678
%   WZ: (keins)
%   SZ: (keins)
%   Bilddateien (PDF): (keine)
%
%
\selectlanguage{ngerman}%
\frenchspacing%
%
\footnotesize%
\pstart%
\noindent%
Der in LH~XXXV~9, 23 Bl.~1\textendash22, LH~XXXVII~4 Bl.~59\textendash60 und LH~XXXVII~5 Bl.~86\textendash91 überlieferte Textkomplex N.~\ref{dcc_00} %??S01 
\textit{De corporum concursu} ist eine umfangreiche systematische Untersuchung über die Gesetze des direkten zentralen Stoßes zweier Körper, bei welcher durch die Erörterung einzelner, zunehmend schwierigerer Stoßfälle allgemeine Lehrsätze aufgestellt und überprüft werden sollen;
erst zum Schluss werden auch einfache Fälle des Stoßes dreier Körper ansatzweise untersucht.
Mit den zwölf fortschreitend nummerierten Unterstücken folgt N.~\ref{dcc_00} %??S01
der Einteilung in nummerierte \textit{schedae} (\textit{I} bis \textit{X} mit zusätzlich \textit{II-II} und \textit{VI-II}), die Leibniz selbst für den gesamten Textkomplex vorgenommen hat.
(Dabei entspricht N.~\ref{dcc_02-2} %??S01\textsubscript{3} 
der \textit{Scheda II-II} und N.~\ref{dcc_06-2} %??S01\textsubscript{8} 
der \textit{Scheda VI-II}.)
Mit Ausnahme einer einzigen undatierten \textit{scheda} (N.~\ref{dcc_02-2}) %??S01\textsubscript{3} 
sind die übrigen von Leibniz selbst auf Januar 1678, N.~\ref{dcc_10} %??S01\textsubscript{12} 
auf Januar und Februar 1678 datiert worden.
Hieraus ergibt sich die Gesamtdatierung des Textkomplexes, % N.~\ref{dcc_00}. %??S01, 
der in seinen nicht gestrichenen Teilen erstmals in \textsc{Fichant} 1994, S.~71\textendash171 veröffentlicht worden ist. 
\pend%
%
\pstart%
Dass Leibniz ganz zu Beginn des Jahres 1678 mit der Stoßlehre im Allgemeinen und vornehmlich mit den Gesetzen besonderer Stoßfälle befasst war, bezeugt auch sein Brief an H.~Conring vom 3.~(13.) Januar 1678:%
\protect\index{Namensregister}{\textso{Conring}, Hermann 1606\textendash1681}
Dort gibt er zu, noch keine befriedigende Antwort auf die Frage zu haben, \textit{quid fiat si duo corpora dura magnitudine inaequalia concurrant} \textendash\ eine Antwort, die man selbst bei Koryphäen der neuen Naturwissenschaft wie Descartes und Huygens vergeblich suchen würde
(\textit{LSB} II,~1 \lbrack2006\rbrack\ N.~162, S.~581.5\textendash7;\cite{02053}
siehe \textsc{Fichant} 1994, S.~192).\cite{01056}%
\protect\index{Namensregister}{\textso{Descartes} (Cartesius, des Cartes), René 1596\textendash1650}%
\protect\index{Namensregister}{\textso{Huygens} (Hugenius, Ugenius, Hugens, Huguens), Christiaan 1629\textendash1695}
Ob Leibniz hiermit besonders auf N.~\ref{dcc_00} %??S01 
oder aber auf frühere Untersuchungen zur Stoßlehre anspielt, lässt sich nicht mit Sicherheit entscheiden.
Dennoch ist bemerkenswert, dass ein beachtlicher Teil der \textit{schedae} \textendash\ von N.~\ref{dcc_01} %??S01\textsubscript{1} 
bis zu N.~\ref{dcc_06-2} \textendash\ %??S01\textsubscript{2} 
vornehmlich mit dem Stoß ungleicher Körper hadert
(vgl. dazu den ebenfalls auf Januar 1678 zurückgehenden Entwurf N.~\ref{RK57279}). % De regula concursus corporum inaequalium 
Der Brief an Conring%
\protect\index{Namensregister}{\textso{Conring}, Hermann 1606\textendash1681}
rechtfertigt somit die Frage, ob der umfangreiche Textkomplex \textit{De corporum concursu} am Stück verfasst wurde oder vielmehr in verschiedenen Etappen, die sich über die ersten zwei Monate des Jahres 1678 hinweg erstreckt haben.
Der Brief lässt sogar die Vermutung zu, dass die Anfertigung von N.~\ref{dcc_00} %??S01 
bereits vor Ende 1677 begonnen haben könnte.
\pend%
%
\pstart%
Für eine Abfassung des Textkomplexes in verschiedenen Etappen sprechen auch weitere Gründe.
Zunächst gilt es,
auf die besondere Verteilung der Wasserzeichen in den Trägern der einzelnen \textit{schedae} hinzuweisen.
Von N.~\ref{dcc_01} %??S01\textsubscript{1} 
abgesehen, lassen sich in dieser Hinsicht zwei verschiedene, homogene Gruppen feststellen:
Sämtliche Bogen zu N.~\ref{dcc_02-1} %??S01\textsubscript{2} 
und N.~\ref{dcc_03} %??S01\textsubscript{4} 
bis \ref{dcc_05} %??S01\textsubscript{6} 
weisen ein und dasselbe Wasserzeichen auf, welches sich von dem einzigen unterscheidet, das in den Trägern der übrigen \textit{schedae} N.~\ref{dcc_06-1} %??S01\textsubscript{7} 
bis \ref{dcc_10} %??S01\textsubscript{10} 
vorkommt.
Diese Feststellung wird umso aussagekräftiger, als sie einigermaßen mit den inhaltlichen Zäsuren übereinstimmt, die sich im Textkomplex ermitteln lassen.
%\pend%
%%
%\pstart%
Vom inhaltlichen Standpunkt aus bilden die einzelnen zwölf \textit{schedae} nämlich keine geschlossenen Einheiten.
Der Text lässt sich diesbezüglich vielmehr in drei fortlaufende Abschnitte unterteilen:
\pend%
%
\pstart%
(1)~von der \textit{Scheda I} (N.~\ref{dcc_01}) %??S01\textsubscript{1} 
bis zur Mitte der \textit{Scheda V} (N.~\ref{dcc_05}, %??S01\textsubscript{6} 
S.~\refpassage{LH_35_09_23_011v_jbcagk}{LH_35_09_23_011-012_11v2}), wo sich durch die Einführung des Elastizitätsgedankens als zusätzlichen Gegenstand der Untersuchung ein abrupter und markanter Einschnitt ergibt;
\pend%
%
\pstart%
(2)~von dorther bis zur \textit{Scheda VI-II} (N.~\ref{dcc_06-2}), %??S01\textsubscript{8}
mit der die Untersuchung Zwischenergebnisse erreicht, die nachträglich geprüft und erweitert wurden;
\pend%
%
\pstart%

(3)~von der beim Axiom der Äquipollenz von Ursache und Wirkung neu ansetzenden \textit{Scheda VII} (N.~\ref{dcc_07}) %??S01\textsubscript{9}
bis zur \textit{Scheda X} (N.~\ref{dcc_10}), %??S01\textsubscript{12}
die eher unvermittelt die Untersuchung abschließt (siehe zum Aufbau des gesamten Textkomplexes
\textsc{Fichant} 1994, S.~175\,f.).\cite{01056}
\pend%
\newpage
\pstart%
Diese drei Abschnitte stellen möglichwerweise auch drei verschiedene Etappen der Entstehung von N.~\ref{dcc_00} %??S01 
dar, welche sich in der Verteilung der Wasserzeichen wenigstens insofern abbilden, als diese letztere besonders die Zäsur in der Mitte von N.~\ref{dcc_05} %??S01\textsubscript{6} 
zum Vorschein bringt.
\pend%
% \newpage%
%
\pstart%
Die%
\edlabel{dcc_Vorbemerkung_reform-1}
durch die Nummerierung der \textit{schedae} festgelegte Textfolge entspricht ohnehin nicht durchgängig der Entstehung und chronologischen Ordnung der Textschichten, da Leibniz nachträglich, am häufigsten mit Randbemerkungen, zuweilen aber auch durch umfangreichere Textzusätze, die älteren \textit{schedae} berichtigt oder erweitert hat.
Anlass zu den meisten nachträglichen Erweiterungen gab die \glqq Entdeckung\grqq, dass beim Phänomen des zentralen Stoßes nicht, wie früher gewöhnlich angenommen, die Summe der Bewegungsgröße (\textit{quantitas motus}) \textit{mv} erhalten bleibe, sondern die Summe der kinetischen \glqq Kraft\grqq\ (\textit{vis}) $mv^2.$
Die entsprechende Formel hatte Leibniz bereits im Mai oder Juni 1677 algebraisch hergeleitet, ohne allerdings ihre physikalische Interpretation als Erhaltungssatz der \glqq Kraft\grqq\ zu thematisieren (siehe N.~\ref{RK57273} und N.~\ref{RK60344_2}). %MS
Die neue Einsicht \textendash\ Geburtsstunde von Leibnizens \textit{réforme de la dynamique} (\textsc{Fichant} 1994\cite{01056}) \textendash\ wird am Anfang der \textit{Scheda VIII} (N.~\ref{dcc_08}, %??S01\textsubscript{10}
S.~\refpassage{LH_37_05_086r_reformatio_idzg-1}{LH_37_05_086r_reformatio_idzg-2}) angekündigt. 
Zahlreiche Texterweiterungen in den früheren \textit{schedae} sind mithin erst % danach 
\textit{post reformationem} entstanden (den Ausdruck hat Leibniz selbst eingeführt;
vgl. die Randbemerkungen zu
N.~\ref{dcc_03},
S.~\refpassage{LH_35_09_23_007v_zkuo-1}{LH_35_09_23_007v_zkuo-2};
N.~\ref{dcc_04},
S.~\refpassage{LH_35_09_23_009r_hdsw-1}{LH_35_09_23_009r_hdsw-2},
\refpassage{LH_35_09_23_010r_jdzwrkm-1}{LH_35_09_23_010r_ndadfk-2};
N.~\ref{dcc_06-1},
S.~\refpassage{LH_35_09_23_013r_ersteMarg-1}{LH_35_09_23_013r_ersteMarg-2}).%
\edlabel{dcc_Vorbemerkung_reform-2}
\pend%
%
\pstart%
Einen besonderen Fall stellen in dieser Hinsicht die \textit{Scheda II-II} (N.~\ref{dcc_02-2}) %??S01\textsubscript{3}
und die \textit{Scheda VI-II} (N.~\ref{dcc_06-2}) %??S01\textsubscript{8}
dar, welche beide, wie bereits ihre Nummerierung zeigt, von Leibniz nachträglich vervollständigt und in den Textkomplex \textit{De corporum concursu} eingefügt wurden.
\pend%
%
\pstart%
In%
\edlabel{dcc_intro_II-II_gdr-1}
der \textit{Scheda II-II} (LH XXXV 9, 23 Bl.~4\textendash5) lassen sich insgesamt mindestens zwei Textschichten unterscheiden.
In der ältesten überprüft Leibniz rechnerisch ein Ergebnis aus der \textit{Scheda II}:
Unter Annahme des Bewegungsquantums \textit{mv} als Erhaltungsgröße erweise sich im Fall des Stoßes eines kleineren auf einen ruhenden Körper als unmöglich, dass ihre relative Stoßgeschwindigkeit sowie die Richtung und die Geschwindigkeit ihres gemeinsamen Schwerpunkts erhalten bleiben würden (N.~\ref{dcc_02-1}, %??S01\textsubscript{2}
S.~\refpassage{LH_35_09_23_006r_viacentrigrav_mesj-1}{LH_35_09_23_006r_viacentrigrav_mesj-2}).
Dieses Ergebnis wird in der \textit{Scheda II-II} zunächst auf Bl.~5~r\textsuperscript{o} durch einen erneuten Beweisgang überprüft (N.~\ref{dcc_02-2}, %??S01\textsubscript{3}
S.~\refpassage{LH_35_09_23_004-005_Blatt5r-1}{LH_35_09_23_004-005_Blatt5r-2})
und dann auf dem gegenüberliegenden Bl.~4~v\textsuperscript{o} in sauberer Form nochmals bestätigt (N.~\ref{dcc_02-2}, %??S01\textsubscript{3}
S.~\refpassage{LH_35_09_23_004-005_Blatt4v-1}{LH_35_09_23_004-005_Blatt4v-2}).
Da der Beweisgang bei dieser zweifachen Prüfung den gleichen Ansatz wie in N.~\ref{dcc_02-1} %??S01\textsubscript{2}
aufweist, dürfte diese früheste Textschicht von N.~\ref{dcc_02-2} %??S01\textsubscript{3}
insgesamt in unmittelbarem Anschluss an N.~\ref{dcc_02-1} %??S01\textsubscript{2}
oder kurz danach entstanden sein.
Später \textendash\ in der \textit{Scheda VIII} (N.~\ref{dcc_08}) %??S01\textsubscript{10}
und der \textit{Scheda IX} (N.~\ref{dcc_09}) %??S01\textsubscript{11}
\textendash\ stellt Leibniz doch fest, dass unter Annahme der \glqq Kraft\grqq\ $mv^2$ als Erhaltungsgröße sowohl die relative Stoßgeschwindigkeit wie auch die Richtung und Geschwindigkeit des gemeinsamen Schwerpunkts erhalten bleiben würden.
Im Nachhinein kehrt er somit zur \textit{Scheda II-II} zurück und verfasst die aus diesem neuen Ergebnis hervorgehenden Ausführungen auf Bl.~5~v\textsuperscript{o} (N.~\ref{dcc_02-2}, %??S01\textsubscript{3}
S.~\refpassage{LH_35_09_23_004-005_Blatt5v-1}{LH_35_09_23_004-005_Blatt5v-2}).
Diese zweite Textschicht berichtigt die früheren irrtümlichen Feststellungen über die relative Stoßgeschwindigkeit und die Bewegung des gemeinsamen Schwerpunktes.
Schließlich entstand die Bemerkung auf Bl.~4~r\textsuperscript{o} (N.~\ref{dcc_02-2}, %??S01\textsubscript{3}
S.~\refpassage{LH_35_09_23_004-005_Blatt4r-1}{LH_35_09_23_004-005_Blatt4r-2}), in der klargestellt wird, dass der ursprüngliche rechnerische Beweis in der ersten Textschicht nur unter der (jetzt zurückgewiesenen) \glqq kartesischen\grqq\ Annahme des Bewegungsquantums (\textit{mv}) als Erhaltungsgröße formal gültig sei.%
\edlabel{dcc_intro_II-II_gdr-2}
\pend%
%
\pstart%
Die%
\edlabel{dcc_intro_VI-II_fzr-1}
\textit{Scheda VI-II} (N.~\ref{dcc_06-2}), %??S01\textsubrscript{8} 
die als umfangreichste im ganzen Textkomplex \textit{De corporum concursu} die Handschriften LH~XXXV 9,~23 Bl.~15\textendash20, LH~XXXVII~4 Bl.~59\textendash60 und LH~XXXVII~5 Bl.~91 umfasst, weist ebenfalls mehrere Textschichten auf, die teilweise noch vor der \textit{Scheda VII} (N.~\ref{dcc_07}) %??S01\textsubrscript{9} 
oder zur gleichen Zeit wie diese, teilweise nach den letzten drei \textit{schedae} (oder wenigstens nach N.~\ref{dcc_08}) %??S01\textsubrscript{8} 
verfasst wurden.
In der ältesten Schicht (Bl.~15~r\textsuperscript{o}, 15~v\textsuperscript{o} und 20~r\textsuperscript{o}; N.~\ref{dcc_06-2}, %??S01\textsubrscript{8}, 
S.~\refpassage{LH_35_09_23_015r_anfang_fhg}{LH_35_09_23_020r_ende_fhg}) wiederholt Leibniz eine Ausführung aus der \textit{Scheda VI}, in der der allgemeine Fall des elastischen Stoßes gegen einen ungleich schweren, ruhenden Körper berechnet wird (N.~\ref{dcc_06-1}, %??S01\textsubrscript{7}, 
S.~\refpassage{LH_35_09_23_014r_percussioinquiescens}{LH_35_09_23_014r_percussioinquiescens}\,ff.) mit der Absicht, sie endgültig zu überprüfen.
Aus der neuen Berechnung ergeben sich zwei Gleichungen, auf denen die späteren Textschichten der Scheda \textit{VI-II} aufbauen.
Die zweite Textschicht (Bl.~59~r\textsuperscript{o}, 59~v\textsuperscript{o} und 60~r\textsuperscript{o}; N.~\ref{dcc_06-2}, %??S01\textsubrscript{8}, 
S.~\pageref{37_04_059r_Regnauld_Anfang}\textendash\pageref{37_04_060r_Regnauld_Ende}) besteht aus neun tabellarischen Darstellungen experimenteller Werte, die Leibniz, zum Teil mit Hilfe eines unbekannten Schreibers, aus F.~Regnaulds%
\protect\index{Namensregister}{\textso{Regnauld} (Regnaud; Regnaldus), Fran\c{c}ois de 1626\textendash1689}
Brief vom 21. Dezember 1655 an B.~de Monconys%
\protect\index{Namensregister}{\textso{Monconys} (Monconisius), Balthasar de 1611\textendash1665}
übernommen hat (vgl. \cite{02021}B.~\textsc{de Monconys},
\cite{00118}\title{Journal des voyages},
Teil III: \glqq Lettres escrittes à Monsieur de Monconys\grqq,
Lyon 1666, S.~52\textendash55; %, getrennte Paginierung
dieser Quelle hatte Leibniz bereits früher seine Aufmerksamkeit gewidmet, wie auch die Notiz N.~\ref{RK60070} in diesem Band zeigt).
Bei diesen Exzerpten, die Leibniz zuweilen mit eigenen Bemerkungen erörtert hat, geht es um Regnaulds%
\protect\index{Namensregister}{\textso{Regnauld} (Regnaud; Regnaldus), Fran\c{c}ois de 1626\textendash1689}
Bemessungen des Stoßes eines unterschiedlich schweren Pendels, das aus verschiedenen Höhen auf ein zweites, ruhendes Pendel fällt.
Diese tabellarischen Darstellungen dienten wiederum als Teilvorlage für die dritte Textschicht von N.~\ref{dcc_06-2}, %??S01\textsubrscript{8}, 
in der Regnaulds experimentelle Ergebnisse mit entsprechenden Werten verglichen werden, die Leibniz rein rechnerisch aus beiden in der ersten Textschicht bestimmten Gleichungen (S.~\refpassage{LH_35_09_23_015v_Gleichung_1-1}{LH_35_09_23_015v_Gleichung_2-2}) herleitet.%
\protect\index{Namensregister}{\textso{Regnauld} (Regnaud; Regnaldus), Fran\c{c}ois de 1626\textendash1689}
Der Vergleich, der offenbar Leibnizens Ausführung zum elastischen Stoß in N.~\ref{dcc_06-1} %??S01\textsubrscript{7} 
(S.~\refpassage{LH_35_09_23_014r_percussioinquiescens}{LH_35_09_23_014r_percussioinquiescens}\,ff.) und N.~\ref{dcc_06-2} %??S01\textsubrscript{8} 
(S.~\refpassage{LH_35_09_23_015r_anfang_fhg}{LH_35_09_23_020r_ende_fhg}) bestätigen sollte, erfolgt anhand weiterer tabellarischer Darstellungen in zwei Schritten: zunächst im gestrichenen Text auf Bl.~91~r\textsuperscript{o} (N.~\ref{dcc_06-2}, %??S01\textsubrscript{8}, 
S.~\pageref{LH_37_05_091r_gestrtab}\textendash\pageref{LH_37_05_091r_ende}) und dann wieder in (vorübergehend) gültiger Gestalt auf Bl.~17~v\textsuperscript{o} und 18~r\textsuperscript{o} (N.~\ref{dcc_06-2}, %??S01\textsubrscript{8}, 
S.~\pageref{LH_35_09_23_017v_tab1}\textendash\pageref{LH_35_09_23_018r_tab2}).
Auf Bl.~17~v\textsuperscript{o} bis 18~v\textsuperscript{o} sind ferner begleitende Anmerkungen anzutreffen (N.~\ref{dcc_06-2}, %??S01\textsubrscript{8}, 
S.~\refpassage{LH_35_09_23_018v_iiiSchichtKomm-1}{LH_35_09_23_018v_iiiSchichtKomm-2}), die ebenfalls zur dritten Textschicht gehören und wie diese und die zwei vorausgehenden Textschichten vor der \textit{Scheda VIII} (N.~\ref{dcc_08}) %??S01\textsubrscript{10}, 
verfasst wurden.
Anders verhält es sich mit einer weiteren, auf Bl.~18~v\textsuperscript{o} vorliegenden Bemerkung (N.~\ref{dcc_06-2}, %??S01\textsubrscript{8}, 
S.~\refpassage{LH_35_09_23_018v_ivSchicht-1}{LH_35_09_23_018v_ivSchicht-2}), die auf die \textit{reformatio} hinweist und folglich mit großer Wahrscheinlichkeit nach (mindestens) N.~\ref{dcc_08} %??S01\textsubrscript{10} 
verfasst wurde.
Diese Bemerkung ist demgemäß einer vierten und letzten Textschicht von N.~\ref{dcc_06-2} %??S01\textsubrscript{8} 
zuzuschreiben, welche ebenfalls den Vermerk auf Bl.~17~r\textsuperscript{o} (S.~\refpassage{LH_35_09_23_017r_nota_bkjm-1}{LH_35_09_23_017r_nota_bkjm-2}) sowie hauptsächlich die \textit{Tabula III} auf Bl.~19~r\textsuperscript{o} (S.~\pageref{LH_35_09_23_019r_tab3}) umfasst.
Diese letztere ergibt sich aus der Neuberechnung der früheren Tabelle auf Bl.~18~r\textsuperscript{o} (S.~\pageref{LH_35_09_23_018r_tab2}) unter Berücksichtigung des Faktors $\displaystyle mv^2$ anstelle des Faktors \textit{mv}.
Die Erläuterung auf Bl.~16~v\textsuperscript{o}, welche die experimentellen Bedingungen von Regnaulds%
\protect\index{Namensregister}{\textso{Regnauld} (Regnaud; Regnaldus), Fran\c{c}ois de 1626\textendash1689}
Bemessungen ein weiteres Mal zusammenfasst und gleichsam den Abschluss von N.~\ref{dcc_06-2} %??S01\textsubrscript{8} 
darstellt, dürfte in ihrem ersten Teil (S.~\refpassage{LH_35_09_23_016v_Teil1_tsf-1}{LH_35_09_23_016v_Teil1_tsf-2}) noch zur dritten Textschicht gehören, während ihr zweiter und letzter Teil (S.~\refpassage{LH_35_09_23_016v_Teil2_tsf-1}{LH_35_09_23_016v_Teil2_tsf-2}) offensichtlich der vierten Textschicht zuzuschreiben ist.%
\edlabel{dcc_intro_VI-II_fzr-2}
\pend%
%
\pstart%
Aus der vorausgehenden Darstellung lässt sich festhalten, dass die jüngsten Textschichten in N.~\ref{dcc_02-2} %??S01\textsubrscript{3} 
und N.~\ref{dcc_06-2} %??S01\textsubrscript{8} 
\textendash\ ebenso wie zahlreiche Texterweiterungen in den übrigen \textit{ante reformationem} entstandenen \textit{schedae} \textendash\ erst in einer späten Bearbeitungsphase des Textkomplexes \textit{De corporum concursu} verfasst wurden, d.h. möglicherweise erst im Februar 1678.
\pend%
\normalsize
%
\selectlanguage{latin}%
\frenchspacing%
%
%