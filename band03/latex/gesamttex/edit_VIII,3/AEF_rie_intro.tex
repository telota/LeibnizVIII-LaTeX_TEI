%   % !TEX root = ../../VIII,3_Rahmen-TeX_8-1.tex
%
%
% N.~32.0. ((??A45/46)) Restitutio isochrona elastri // AEF (Elastizität) 
%
%
%
\selectlanguage{ngerman}%
\frenchspacing%
%
\footnotesize%
\pstart%
\noindent%
Beide im Folgenden edierte Konzepte \textendash\ die \textit{Restitutio isochrona elastri} N.~\ref{RK60353} und der titellose Entwurf N.~\ref{RK60301}, dem editorisch die Überschrift \textit{Demonstratio de restitutionis elasticae isochronismo} zugewiesen wird \textendash\ bilden ihrer Entstehung nach sowie in inhaltlicher Hinsicht eine geschlossene Einheit und werden aus diesem Grund zusammenhängend ediert.
Beide Texte sind hauptsächlich dem Versuch gewidmet, ein für die physikalische Klanglehre grundlegendes Phänomen mathematisch nachzuweisen: den Isochronismus der Schwingungen elastischer Körper.
%Mit den Schwingungen elastischer Körperund ihrer gleichmäßigen Dauer \textendash\ d.h. ihrer konstanten Frequenz \textendash\ 
Mit diesem Phänomen \textendash\ insbesondere mit dem Isochronismus der Schwingungen gespannter Saiten \textendash\ 
hatte sich Leibniz bereits Anfang der 1680er Jahren in seinen Untersuchungen über Akustik und Elastizität befasst, ohne hierbei einheitliche Ergebnisse zu erreichen.
Theoreme über den Isochronismus der \textit{vibratio} einer gezupften Saite, die aus einer beliebigen Auslenkung ihren anfänglichen Spannungsgrad zurückgewinnt, werden in N.~\ref{41152_6} (S.~\refpassage{LH_35_09_15_016r_propositio3-1}{LH_35_09_15_016r_propositio3-2}) und N.~\ref{41156} (S.~\refpassage{LH_35_09_15_021r_propositio1-1}{LH_35_09_15_021r_propositio1-2}) formuliert, aber mit Unklarheiten und ohne Beweise.
Dass die \textit{restitutio omnimoda} einer gespannten Saite \textendash\ d.h. ihr Rückgang von einem beliebigen Spannungsgrad zu ihrem \glqq natürlichen\grqq, spannungslosen Zustand \textendash\ isochron verlaufe, meint Leibniz in N.~\ref{41152_5} (S.~\refpassage{LH_35_09_15_015v_isochron-1}{LH_35_09_15_015v_isochron-2}) und N.~\ref{41153} (S.~\refpassage{LH_35_09_15_003r_aequdiut-1}{LH_35_09_15_003v_aequdiut-2}) mathematisch nachgewiesen zu haben.
Das Phänomen des Isochronismus der Schwingungen elastischer Körper spielt ferner eine tragende Rolle bei Leibnizens Erklärung der Entstehung, Ausbreitung und Aufnahme des Schalls in seinen zwischen August 1681 und Mitte 1685 entstandenen \textit{Cogitationes novae de sono} (N.~\ref{cnds_1} bis N.~\ref{cnds_3}).
\pend%
%
\pstart%
Als maßgeblich für die absolute Datierung der Konzepte N.~\ref{RK60353} und N.~\ref{RK60301} erweist sich ihr gemeinsames Wasserzeichen.
Dieses ist im Leibniz-Nachlass nach heutigem Wissensstand nur für den Zeitraum belegt, der sich zwischen Leibnizens Rückkehr von seiner Italienreise (um die Mitte Juni 1690) und der Mitte der 1690er Jahre erstreckt.
Ein gut datierbares Vorkommnis dieser Papiersorte ist beispielsweise die tagebuchförmige Aufzeichnung LH XLI~1 Bl.~11, die spätestens im Frühjahr 1695 angefertigt wurde (vgl. \textit{LSB} I,~11 N.~338, S.~494).\cite{01382}
Aufgrund des gemeinsamen Wasserzeichens ist anzunehmen, dass beide Entwürfe N.~\ref{RK60353} und N.~\ref{RK60301} frühestens in der zweiten Hälfte 1690 und spätestens im Laufe des Jahres 1695 entstanden sind.
Eine spätere Entstehungszeit ist jedoch nicht auszuschließen.
%Das gleiche, im Träger beider Texte vorliegende Wasserzeichen ist im Leibniz-Nachlass nach heutigem Wissensstand lediglich für die erste Hälfte der 1690er Jahre belegt.
\pend%
%%
\pstart%
Die relative Chronologie ergibt sich aus dem inhaltichen Vergleich beider Texte.
Das Konzept N.~\ref{RK60353} beginnt mit einer abstrakten Formulierung der Regel, die das kinematische Verhalten eines schwingenden Körpers beschreibt (S.~\refpassage{LH_37_05_180r_Anfang-1}{LH_37_05_180r_Anfang-2}).
Der folgende, erste Versuch einer mathematischen Darstellung der Regel erweist sich jedoch als untauglich und wird aufgegeben (S.~\refpassage{LH_37_05_180_m180r4}{LH_37_05_180_rndbmerk_1-2}).
Nach einem Neu\-anfang (S.~\refpassage{LH_37_05_180_m180r6}{LH_37_05_180_m180r6}) wird ein zweiter Versuch unternommen, der in N.~\ref{RK60353} aufgrund einer fehlerhaften mathematischen Behandlung in einer Sackgasse mündet:
Leibniz gibt zu, auf diesem Weg den Isochronismus nicht beweisen zu können (S.~\refpassage{LH_37_05_180v_nonassequi-1}{LH_37_05_180v_nonassequi-2}).
Nachträglich aber vermerkt er, anderswo \textendash\ d.h. wohl in N.~\ref{RK60301} \textendash\ sei der Beweis ordentlich gelungen (vgl. die Randbemerkungen zu S.~\refpassage{LH_37_05_180r_inaliaplagula-1}{LH_37_05_180r_inaliaplagula-2}).
Tatsächlich knüpft N.~\ref{RK60301} gleich zu Beginn an den zweiten in N.~\ref{RK60353} entwickelten rechnerischen Ansatz an, der jetzt, etwas verändert, erneut ausgeführt wird.
Nach Leibnizens Einschätzung führt dieser Weg zum erwünschten Beweis, dessen Bedeutung für die Akustik unverholen angepriesen wird (S.~\refpassage{LH_37_05_046r_Isochronismusbeweis-1}{LH_37_05_046r_Isochronismusbeweis-2}).
Das Konzept N.~\ref{RK60353} muss folglich abgeschlossen gewesen oder zumindest weitgehend angefertigt worden sein, als N.~\ref{RK60301} begonnen wurde.
%
Die zwei Konzepte dürften wohl eng nacheinander oder zum Teil gar nebeneinander verfasst worden sein.
Dafür sprechen nicht nur der inhaltliche Zusammenhang oder die Verwendung der gleichen Papiersorte, sondern auch Leibnizens mögliche Verwechselung beider Texte bei einem Rückverweis gegen Ende von N.~\ref{RK60301} (vgl. die Randbemerkung zu S.~\refpassage{LH_37_05_046r_Schlussmarg-1}{LH_37_05_046r_Schlussmarg-2} und die zugehörige dritte Erläuterung).%
%Dafür, dass die zwei Konzepte eng nacheinander oder zum Teil gar nebeneinander angefertigt wurde, spricht auch ein weiterer Umstand:
%Gegen Ende von N.~\ref{RK60301} erinnert Leibniz daran, dass die Bewegung der ganzen Saite durch die Bewegung ihres Schwerpunkts dargestellt wird, und weist zur Begründung auf \textit{paulo ante dictum} (vgl. die Randbemerkung zu S.~\refpassage{}{}???).
%Am Anfang von N.~\ref{RK60301} ist jedoch nur eine knappe Wiederholung dieses Gedankens (S.~\refpassage{}{}???) zu lesen, während eine ausfürliche Erläuterung desselben vielmehr in N.~\ref{RK60353} anzutreffen ist (S.~\refpassage{}{}???).
%Leibniz dürfte somit im Vermerk am Ende von N.~\ref{RK60301} die beiden Texte aufgrund ihrer zeitlichen Nähe miteianander verwechselt haben.%
\pend%
\normalsize%
%
\selectlanguage{latin}%
\frenchspacing%
%
%