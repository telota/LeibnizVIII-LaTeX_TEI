%   % !TEX root = ../../VIII,3_Rahmen-TeX_8-1.tex
%
%
%   Band VIII, 3 N.~??S01.05
%   Signatur/Tex-Datei: LH_35_09_23_009-010
%   RK-Nr. 41208 /5
%   \ref{dcc_04}
%   Überschrift: De corporum concursus scheda quarta
%   Modul: Mechanik / Stoß ()
%   Datierung: Januar 1678
%   WZ: LEd-WZ 803003 = RK-WZ 142 (eins)
%   SZ: (keins)
%   Bilddateien (PDF): (keine)
%   Verzeichniseinträge: vollständig
%   \textls{} statt \textso{} (Ausnahme: Personenverzeichnis)
%
%
\selectlanguage{ngerman}%
\frenchspacing%
%
\begin{ledgroupsized}[r]{120mm}%
\footnotesize%
\pstart%
\noindent%
\textbf{Überlieferung:}%
\pend%
\end{ledgroupsized}%
\begin{ledgroupsized}[r]{114mm}%
\footnotesize%
\pstart %
\parindent -6mm%
\makebox[6mm][l]{\textit{L}}%
Konzept:
LH~XXXV~9,~23 Bl.~9\textendash10.
Ein Bogen 2\textsuperscript{o};
ein Wasserzeichen auf Bl.~9.
Vier voll-beschriebene Seiten,
die den Text N.~\ref{dcc_03} %??S01\textsubscript{4} 
fortsetzen und vom Text N.~\ref{dcc_05} %??S01\textsubscript{6} 
fortgesetzt werden;
ein Kustos am Ende von Bl.~10v\textsuperscript{o}\! verweist auf die \textit{Scheda quinta}.
Randbemerkungen tlw. \textit{post reformationem} verfasst (siehe die editorische Vorbemerkung, S.~\refpassage{dcc_Vorbemerkung_reform-1}{dcc_Vorbemerkung_reform-2}).
\pend%
\end{ledgroupsized}%
%
\begin{ledgroupsized}[r]{114mm}%
\footnotesize%
\pstart%
\parindent -6mm%
\makebox[6mm][l]{\textit{E}}%
\textsc{Fichant}\cite{01056} 1994, S.~100\textendash105
(mit kommentierter französischer Übersetzung, S.~223\textendash228).
\pend%
\end{ledgroupsized}%
%
\selectlanguage{latin}%
\frenchspacing%
%
%
\count\Bfootins=1200%
\count\Afootins=1200%
\count\Cfootins=1200%
\vspace{8mm}
\normalsize%
\pstart%
\noindent%
%
\lbrack9~r\textsuperscript{o}\rbrack% \ % % % %   Blatt 9r
\hspace{43mm}%
De%
\edlabel{LH_35_09_23_009r_hdsw-1}
corporum concursu%
\protect\index{Sachverzeichnis}{concursus corporum}%
\hspace{30mm}
Januar. 1678
\pend%
\pstart%
\noindent%
\centering%
Scheda quarta%
\protect\index{Sachverzeichnis}{scheda}%
\edlabel{LH_35_09_23_009r_hdsw-2}%
\edtext{}{%
{\xxref{LH_35_09_23_009r_hdsw-1}{LH_35_09_23_009r_hdsw-2}}%
{\lemma{\textit{Am Rand:}}\Afootnote{%
Haec accuratius discutienda post reformationem.%
\protect\index{Sachverzeichnis}{reformatio}\vspace{-3mm}%
% \newline%
}}}%
%
\pend%
\vspace{0.5em}%
%
\pstart%
\noindent%
\edtext{Iisdem%
\edlabel{LH_35_09_23_009r_anfang-1}
positis}{%
\lemma{Iisdem positis}\Cfootnote{%
Siehe N.~\ref{dcc_03}, %??S01\textsubscript{4},
S.~\refpassage{LH_35_09_23_008v_sfghj-1}{LH_35_09_23_008v_sfghj-2}.%
}}
%
vim quae corpori excipienti%
\protect\index{Sachverzeichnis}{corpus excipiens}
%
\edtext{ultra priorem}{%
\lemma{ultra}\Bfootnote{%
\textit{(1)}~propriam
\textit{(2)}~priorem%
~\textit{L}}}
%
et quiescentis%
\protect\index{Sachverzeichnis}{corpus quiescens}
exemplo acceptam,%
\protect\index{Sachverzeichnis}{vis corpori excipienti accepta}
tertio loco accedit,
ita
%
\edtext{primum}{%
\lemma{primum}\Bfootnote{%
\textit{erg.~L}}}
%
definiemus alternative,
ut sit
%
\edtext{vel vis repulsae%
\protect\index{Sachverzeichnis}{vis repulsae}
prior}{%
\lemma{vel}\Bfootnote{%
\textit{(1)}~tota vis differentiae
\textit{(2)}~vis repulsae,
\textbar~incurrentis, si excipiens \textit{erg. u. gestr.}~%
\textbar\ vel vis propria adhuc semel, utra scilicet ex his duabus minor est
\textit{(3)}~vis repulsae prior%
~\textit{L}}}
%
incurrentis%
\protect\index{Sachverzeichnis}{corpus incurrens}
si
%
\edtext{in excipiens ut quiescens}{%
\lemma{in}\Bfootnote{%
\textit{(1)}~quiescens
\textit{(2)}~excipiens ut quiescens%
~\textit{L}}}
%
differentia celeritatum incurrisset,%
\protect\index{Sachverzeichnis}{differentia celeritatum}
vel vis prior excipientis adhuc semel;
utra scilicet ex his duabus minor est.
Et haec quidem
%
\edtext{minor}{%
\lemma{minor}\Bfootnote{%
\textit{erg.~L}}}
%
vel
%
\edtext{ex dupla vel simpla}{%
\lemma{ex}\Bfootnote{%
\textit{(1)}~toto vel ex d
\textit{(2)}~dupla vel
\textit{(a)}~pro dimidia
\textit{(b)}~simpla%
~\textit{L}}}
%
\edtext{parte\lbrack:\rbrack\
quorum}{%
\lemma{parte,}\Bfootnote{%
\textit{(1)}~ita ut repellenti
\textit{(2)}~quod
\textit{(3)}~quorum%
~\textit{L}}}
%
utrum verum sit,
%
in sequentibus
% \edtext{}{%
% \lemma{in sequentibus}\Cfootnote{%
% S.~\refpassage{}{}.???}}
%
definiendum est.
Ideoque incurrenti%
\protect\index{Sachverzeichnis}{corpus incurrens}
restabit differentia
inter vim excipientis%
\protect\index{Sachverzeichnis}{vis corporis excipientis}
et
%
\edtext{repulsae%
\protect\index{Sachverzeichnis}{vis repulsae}
et praeterea de minore ex his duabus
vel simplum vel nihil,
quod in sequentibus}{%
\lemma{repulsae}\Bfootnote{\hspace{-0,5mm}%
\textbar~et praeterea % de minore ex his duabus vel simplum 
\lbrack...\rbrack\ vel nihil \textit{erg.}~%
\textbar~,
\textit{(1)}~eaque vel
\textit{(a)}~tota
\textit{(b)}~dupla vel dimidia,
\textit{(2)}~quod in sequentibus%
~\textit{L}}}
%
definiendum est,
ut dixi.%
\edlabel{LH_35_09_23_009r_anfang-2}
%\textbf{??? Erzwungener Umbruch wegen Instabilität. ???}
\pend%
%\newpage%
%
\pstart%
Nam%
\edlabel{LH_35_09_23_009r_conflictus_bis-1}%
\edlabel{LH_35_09_23_009rv_conflictus-1}
cum intelligitur
corpus incurrens et%
\protect\index{Sachverzeichnis}{corpus incurrens}
%
\edtext{repelli%
\protect\index{Sachverzeichnis}{corpus repulsum}
ab}{%
\lemma{repelli}\Bfootnote{\hspace{-0,5mm}%
\textbar~ab \textit{streicht Hrsg. nach~E,\cite{01056} S.~100}~%
\textbar\ ab%
~\textit{L}}}
%
excipiente,%
\protect\index{Sachverzeichnis}{corpus excipiens}
quiescentis instar sum\-to,%
\protect\index{Sachverzeichnis}{corpus quiescens}
et tamen motu communi%
\protect\index{Sachverzeichnis}{motus communis}
utrique progredi,
oritur conflictus%
\protect\index{Sachverzeichnis}{conflictus}
inter repulsam%
\protect\index{Sachverzeichnis}{repulsa}
et motum progressionis%
\protect\index{Sachverzeichnis}{motus progressionis}%
\protect\index{Sachverzeichnis}{progressio}
(\protect\vphantom)%
id est motum communem,%
\protect\index{Sachverzeichnis}{motus communis}
id est motum priorem excipientis%
\protect\index{Sachverzeichnis}{motus corporis excipientis}%
\protect\vphantom()%
\edlabel{LH_35_09_23_009r_conflictus_bis-2}%
\lbrack:\rbrack\
uter horum major est
%
\edtext{vincit;
et differentia%
\protect\index{Sachverzeichnis}{differentia motuum}
ipsi incurrenti competit,%
\protect\index{Sachverzeichnis}{corpus incurrens}
vel retrorsum vel prorsum,
prout vincit repulsa%
\protect\index{Sachverzeichnis}{repulsa}
vel progressionis vis communis.%
\protect\index{Sachverzeichnis}{vis progressionis}
Ipsa vero vis mutuo se destruens%
\protect\index{Sachverzeichnis}{vis mutuo se destruens}%
}{%
\lemma{vincit;}\Bfootnote{%
\textit{(1)}~sed
\textit{(a)}~qui
\textit{(b)}~quia is qui minor est destrui non
\textit{(2)}~et \textbar~differentia \textit{erg.}~%
\textbar\ ipsi incurrenti
\textit{(a)}~data
\textit{(b)}~competit, vel \lbrack...\rbrack\ se destruens%
~\textit{L}}}
%
debilioris
(\protect\vphantom)%
sive id sit vis repulsae,%
\protect\index{Sachverzeichnis}{vis repulsae}
sive vis
%
\edtext{progressionis%
\protect\index{Sachverzeichnis}{vis progressionis}%
\protect\vphantom()
duplicata%
\protect\index{Sachverzeichnis}{vis duplicata}%
}{%
\lemma{progressionis}\Bfootnote{%
\textit{(1)}~necessario vel tota tribuitur
\textit{(2)}~\protect\vphantom() duplicata%
~\textit{L}}}
%
(\protect\vphantom)duplicata inquam,
destruit enim aequalis aequalem,
unde destruens et destructa simul,%
\protect\index{Sachverzeichnis}{vis destruens}%
\protect\index{Sachverzeichnis}{vis destructa}
%
\edtext{se alterutram}{%
\lemma{se}\Bfootnote{%
\textit{(1)}~invicem
\textit{(2)}~alterutram%
~\textit{L}}}
%
\edtext{duplicant\protect\vphantom() aut transfertur}{%
\lemma{duplicant\protect\vphantom()}\Bfootnote{%
\textit{(1)}~sive tribuitur aut
\textit{(2)}~aut transfertur%
~\textit{L}}}
%
in excipiens tota,
aut dimidium ejus,
duplicatae nimirum,
id est
ipsa simpla%
\protect\index{Sachverzeichnis}{vis simpla}
tantum transfertur in excipiens%
\protect\index{Sachverzeichnis}{corpus excipiens}
altera dimidia data incurrenti.%
\protect\index{Sachverzeichnis}{corpus incurrens}
\pend%
%
\pstart%
Sed%
\edlabel{LH_35_09_23_009r_uvodi-1}
si ponamus
%
\edtext{\lbrack eam\rbrack}{%
\lemma{eam}\Bfootnote{%
\textit{erg. Hrsg.}}}
%
posterius dividi,
nimirum in duas partes,
et dimidiam duplicatae,
id est simplam incurrenti tribui,
tribuetur utique per modum repulsae;%
\protect\index{Sachverzeichnis}{repulsa}
ponamus autem incurrens prius vicisse,
seu majorem fuisse progressionem
quam in
%
\edtext{supra}{%
\lemma{supra}\Cfootnote{%
Siehe S.~\refpassage{LH_35_09_23_009r_conflictus_bis-1}{LH_35_09_23_009r_conflictus_bis-2}.%
}}
%
dicto conflictu;
habebimus iterum novum conflictum,%
\protect\index{Sachverzeichnis}{conflictus}
intra residuam adhuc progressionem%
\protect\index{Sachverzeichnis}{progressio residua}
et hanc%
%
\edtext{}{%
\lemma{\textit{Am Rand:}}\Afootnote{%
Non rationabiles tot replicationes certaminum%
\protect\index{Sachverzeichnis}{replicatio certaminis}%
\protect\index{Sachverzeichnis}{certamen}
in uno momento.%
\newline%
}}
%
\edtext{repulsam.%
\protect\index{Sachverzeichnis}{repulsa}
Ubi iterum}{%
\lemma{repulsam.}\Bfootnote{%
\textit{(1)}~Ex quo
\textit{(2)}~Ubi iterum%
~\textit{L}}}
%
ponamus progressionem vincere,
et repulsa atque pars progressionis ei aequalis
ne se mutuo destruant ac vis aliqua pereat,%
\protect\index{Sachverzeichnis}{vis periens}
rursus ponemus ipsius
%
\edtext{duplicati}{%
\lemma{duplicati}\Bfootnote{%
\textit{erg.~L}}}
%
dimidium,
id est totam vim repulsae,%
\protect\index{Sachverzeichnis}{vis repulsae}
tribui excipienti,%
\protect\index{Sachverzeichnis}{corpus excipiens}
%
\edtext{atque alterum esse novam repulsam,%
\protect\index{Sachverzeichnis}{repulsa nova}
quae iterum}{%
\lemma{atque}\Bfootnote{%
\textit{(1)}~dimidium ejus iterum
\textit{(2)}~alterum esse % novam repulsam, 
\lbrack...\rbrack\ quae iterum%
~\textit{L}}}
%
vincenti parti progressionis opponatur,%
\protect\index{Sachverzeichnis}{progressio}
et ita porro quousque
ut id fieri potest\lbrack;\rbrack\
\edlabel{LH_35_09_23_009r_dedf-1}%
quod ut pateat clarius:%
%
\edtext{}{%
{\xxref{LH_35_09_23_009r_dedf-1}{LH_35_09_23_009r_dedf-2}}%
{\lemma{\textit{Am Rand, nachträglich hinzugefügt:}}\Afootnote{%
Confer infra scheda\textsuperscript{[a]}
septima pag.~2.\textsuperscript{[b]}%
\protect\index{Sachverzeichnis}{scheda}
\newline\vspace{-0.5em}%
\newline%
{\footnotesize%
\textsuperscript{[a]}~scheda
\textit{(1)}~7\textsuperscript{ma}
\textit{(2)}~septima%
~\textit{L}
\quad
\textsuperscript{[b]}~pag.~2:
N.~\ref{dcc_07}, %??S01\textsubscript{9},
S.~\refpassage{LH_35_09_23_021v_conflictus-1}{LH_35_09_23_021v_conflictus-2}.\vspace{-3mm}%
% ((Bl.21v = Fichant.147))%
% \newline%
}}}}
%
\edtext{sit prior vis}{%
\lemma{sit}\Bfootnote{%
\textit{(1)}~vis excipientis quam
\textit{(a)}~minimam
\textit{(b)}~minorem vi
\textit{(2)}~prior vis%
~\textit{L}}}
%
repulsae \textit{a}%
\protect\index{Sachverzeichnis}{vis repulsae}
%
\edtext{et vis motus primi communis%
\protect\index{Sachverzeichnis}{motus communis}
cum excipiente
seu vis}{%
\lemma{et}\Bfootnote{%
\textit{(1)}~vim
\textit{(2)}~vis motus \textbar~primi \textit{erg.}~%
\textbar\ communis cum excipiente \textbar~seu \textit{erg.}~%
\textbar\ vis%
~\textit{L}}}
%
progressionis:%
\protect\index{Sachverzeichnis}{vis progressionis}
$a + x\, \sqcap\, y,$
erit vis se destruens 2\textit{a},%
\protect\index{Sachverzeichnis}{vis se destruens}
de qua \textit{a} tribuetur excipienti,%
\protect\index{Sachverzeichnis}{corpus excipiens}
et \textit{a} alia confliget
cum reliquo \textit{x}.
Sit ergo \textit{x} aequ. $a + z,$
fiet iterum idem,
et ita
%
\edtext{porro ipsum \textit{z}}{%
\lemma{porro}\Bfootnote{%
\textit{(1)}~donec exhauriatur ex \textit{z}
\textit{(2)}~ipsum \textit{z}%
~\textit{L}}}
%
iterum resolvendo in
%
\edtext{$\displaystyle a\, +\, \ldots$
donec tandem \textit{y},
quam ponemus}{%
\lemma{$a + \ldots$}\Bfootnote{%
\textit{(1)}~unde ponendo
\textit{(2)}~donec tandem \textit{y}, quam ponemus%
~\textit{L}}}
%
esse multiplam ipsius \textit{a},
utcunque tota
repetitis hoc modo destructionibus%
\protect\index{Sachverzeichnis}{destructio}
exhauriatur;
unde orietur absurdum,%
\protect\index{Sachverzeichnis}{absurdum}
scilicet
omni vi illa in excipiens translata%
\protect\index{Sachverzeichnis}{vis translata in excipiens}
prorsus sisti
%
\edtext{incurrens,%
\protect\index{Sachverzeichnis}{corpus incurrens}
quod}{%
\lemma{incurrens,}\Bfootnote{%
\textit{(1)}~et totam vim
\textit{(2)}~quod%
~\textit{L}}}
%
est
%
\edtext{supra demonstratis}{%
\lemma{supra demonstratis}\Cfootnote{%
Möglicherweise in N.~\ref{dcc_03}, %??S01\textsubscript{4},
S.~\refpassage{LH_35_09_23_008v_sfghj-1}{LH_35_09_23_008v_sfghj-2}.%
}}
%
contrarium.%
\edlabel{LH_35_09_23_009r_dedf-2}%
\edlabel{LH_35_09_23_009r_uvodi-2}%
\edlabel{LH_35_09_23_009rv_conflictus-2}
%
Ergo ex duobus modis
%
\edtext{supra alternative probatis}{%
\lemma{supra \lbrack...\rbrack\ probatis}\Cfootnote{%
S.~\refpassage{LH_35_09_23_009r_anfang-1}{LH_35_09_23_009r_anfang-2}.}}
%
superest unus,
nempe cum incurrens%
\protect\index{Sachverzeichnis}{corpus incurrens}
differentia celeritatum%
\protect\index{Sachverzeichnis}{differentia celeritatum}
velut in quiescens%
\protect\index{Sachverzeichnis}{corpus quiescens}
%
\edtext{repellitur%
\protect\index{Sachverzeichnis}{corpus repulsum}
at}{%
\lemma{repellitur}\Bfootnote{%
\textit{(1)}~tota
\textit{(2)}~at%
~\textit{L}}}
%
\lbrack9~v\textsuperscript{o}\rbrack\ % % % %   Blatt 9v
%
simul tamen
%
\edtext{communi celeritate,%
\protect\index{Sachverzeichnis}{celeritas communis}%
}{%
\lemma{communi}\Bfootnote{%
\textit{(1)}~celeritate
\textit{(2)}~vi
\textit{(3)}~celeritate%
~\textit{L}}}
%
id est excipientis%
\protect\index{Sachverzeichnis}{celeritas corporis excipientis}
progreditur,
vim hoc modo destructam%
\protect\index{Sachverzeichnis}{vis destructa}
in incurrente
(\protect\vphantom)%
id est minorem ex repulsa
vel progressione duplicatam%
\protect\vphantom()
transferri in excipiens,%
\protect\index{Sachverzeichnis}{vis transferta in excipiens}
residuum vero incurrenti competere,%
\protect\index{Sachverzeichnis}{corpus incurrens}
retrorsum vel prorsum,
prout vicit repulsa%
\protect\index{Sachverzeichnis}{repulsa}
vel progressio.%
\protect\index{Sachverzeichnis}{progressio}
Hinc theorema%
\protect\index{Sachverzeichnis}{theorema}%
\lbrack:\rbrack\
\pend%
%
\count\Bfootins=1000%
\count\Afootins=1200%
\count\Cfootins=1000%
\pstart%
\edtext{Si corpus minus assequatur
%
\edtext{majus%
\protect\index{Sachverzeichnis}{corpus minus in majus}
tunc fingamus majus%
\protect\index{Sachverzeichnis}{corpus majus}
seu excipiens quiescere,%
\protect\index{Sachverzeichnis}{corpus excipiens}%
\protect\index{Sachverzeichnis}{corpus quiescens}
et minus seu incurrens moveri%
\protect\index{Sachverzeichnis}{corpus incurrens}
differentia verarum celeritatum,%
\protect\index{Sachverzeichnis}{differentia celeritatum}
utique in casu hujus fictionis%
\protect\index{Sachverzeichnis}{fictio}%
\protect\index{Sachverzeichnis}{casus fictionis}
repelletur incurrens%
\protect\index{Sachverzeichnis}{corpus repulsum}
per superiora.}{%
\lemma{majus}\Bfootnote{%
\textit{(1)}~tunc ponendo
\textbar~majus seu \textit{erg.}~%
\textbar\ excipiens quiescere et
\textit{(a)}~incurrens in
\textit{(b)}~minus seu incurrens incurrere differentia celeritatum,
\textit{(aa)}~vis repulsae quae hoc modo
\textit{(bb)}~tunc 
\textit{(2)}~tunc
\textit{(a)}~ponamus mi
\textit{(b)}~fingamus majus % seu excipiens quiescere, et minus seu incurrens moveri differentia verarum celeritatum, utique 
\lbrack...\rbrack\ in casu
\textit{(aa)}~vi
\textit{(bb)}~hujus fictionis % repelletur incurrens 
\lbrack...\rbrack\ per superiora.%
~\textit{L}}}%
}{\lemma{Si corpus \lbrack...\rbrack\ per superiora}\Cfootnote{%
Vgl. % N.~\ref{dcc_01} %??S01\textsubscript{1} bzw. 
N.~\ref{dcc_02-1}, %??S01\textsubscript{2}??,
S.~\refpassage{LH_35_09_23_003v_mininmajquies-1}{LH_35_09_23_003v_mininmajquies-2}.%
}}
%\pend%
%%
%\pstart%
At idem
%
\edtext{progredietur celeritate}{%
\lemma{progredietur}\Bfootnote{%
\textit{(1)}~vi
\textit{(2)}~celeritate%
~\textit{L}}}
%
communi,%
\protect\index{Sachverzeichnis}{celeritas progressionis}%
\protect\index{Sachverzeichnis}{celeritas communis}
nempe
%
\edtext{excipientis.%
\protect\index{Sachverzeichnis}{celeritas corporis excipientis}
His positis}{%
\lemma{excipientis}\Bfootnote{%
\textit{(1)}~itaque
\textit{(2)}~. His positis%
~\textit{L}}}
%
ajo
%
\edtext{incurrenti relinqui vim%
\protect\index{Sachverzeichnis}{vis relicta}
quae sit differentia inter vim repulsae,%
\protect\index{Sachverzeichnis}{vis repulsae}
et vim}{%
\lemma{incurrenti}\Bfootnote{%
\textit{(1)}~tribuendam esse differentiam inter vim repulsae, et vim
\textit{(2)}~relinqui vim % quae sit differentia inter vim repulsae,
\lbrack...\rbrack\ et vim%
~\textit{L}}}
%
excipientis%
\protect\index{Sachverzeichnis}{vis corporis excipientis}
%
\edtext{primam,
ita ut}{%
\lemma{primam,}\Bfootnote{%
\textit{(1)}~ut
\textit{(2)}~ita ut%
~\textit{L}}}
%
repellatur ea vi,
si
%
\edtext{repulsa%
\protect\index{Sachverzeichnis}{repulsa}
est}{%
\lemma{repulsa}\Bfootnote{%
\textit{(1)}~est
\textit{(2)}~fuit
\textit{(3)}~est%
~\textit{L}}}
%
major,
progrediatur
%
\edtext{vero si vis}{%
\lemma{vero}\Bfootnote{%
\hspace{-0,5mm}si
\textit{(1)}~celeritas
\textit{(2)}~vis%
~\textit{L}}}
%
prima excipientis est major.
Demonstratur ex praecedentibus.
\pend%
%
\pstart%
Verum ne fictionem%
\protect\index{Sachverzeichnis}{fictio}
ingredi
%
\edtext{theorema%
\protect\index{Sachverzeichnis}{theorema}%
}{%
\lemma{theorema}\Bfootnote{%
\textit{erg.~L}}}
%
opus sit,
ideo id
cujus causa eam adhibuimus,
in ea substituemus;
nempe
si fingamus incurrere incurrens
in excipiens
%
\edtext{majus}{%
\lemma{majus}\Bfootnote{%
\textit{erg.~L}}}
%
velut quiescens,%
\protect\index{Sachverzeichnis}{corpus quiescens}
tunc
%
\edtext{per priora}{%
\lemma{per priora}\Cfootnote{%
Vgl. N.~\ref{dcc_02-1}, %??S01\textsubscript{2}, 
S.~\refpassage{LH_35_09_23_003v_clrtsrpls_kzn-1}{LH_35_09_23_003v_clrtsrpls_kzn-2}.%
}}
%
celeritas repulsae%
\protect\index{Sachverzeichnis}{celeritas repulsae}
incurrentis%
\protect\index{Sachverzeichnis}{corpus incurrens}
est ad celeritatem incursus,%
\protect\index{Sachverzeichnis}{celeritas incursus}
hoc loco differentiam celeritatum%
\protect\index{Sachverzeichnis}{differentia celeritatum}%
\lbrack,\rbrack\
ut differentia corporum%
\protect\index{Sachverzeichnis}{differentia corporum}
est
ad corpus majus seu excipiens.%
\protect\index{Sachverzeichnis}{corpus excipiens}
Unde orietur theorema tale%
\protect\index{Sachverzeichnis}{theorema}%
\lbrack:\rbrack\
\pend%
%
\pstart%
Si corpus minus assequatur majus,%
\protect\index{Sachverzeichnis}{corpus minus in majus}
%
\edtext{tunc celeritas}{%
\lemma{tunc}\Bfootnote{%
\textit{(1)}~vis
\textit{(2)}~celeritas%
~\textit{L}}}
%
incurrentis%
\protect\index{Sachverzeichnis}{celeritas corporis incurrentis}
seu minoris
post incursum%
\protect\index{Sachverzeichnis}{celeritas post incursum}%
\protect\index{Sachverzeichnis}{incursus}
residua%
\protect\index{Sachverzeichnis}{celeritas residua}
erit differentia%
\protect\index{Sachverzeichnis}{differentia celeritatum}
%
\edtext{inter celeritatem}{%
\lemma{inter}\Bfootnote{%
\textit{(1)}~vim
\textit{(2)}~celeritatem%
~\textit{L}}}
%
excipientis%
\protect\index{Sachverzeichnis}{celeritas corporis excipientis}
ante incursum%
\protect\index{Sachverzeichnis}{celeritas ante incursum}%
\protect\index{Sachverzeichnis}{incursus}
%
\edtext{et celeritatem}{%
\lemma{et}\Bfootnote{%
\textit{(1)}~vim
\textit{(2)}~celeritatem%
~\textit{L}}}
%
quae sit
%
\edtext{ad differentiam celeritatum%
\protect\index{Sachverzeichnis}{differentia celeritatum}%
}{%
\lemma{ad}\Bfootnote{%
\textit{(1)}~vim incursus
\textit{(2)}~differentiam celeritatum%
~\textit{L}}}
%
(\protect\vphantom)%
excipientis et incurrentis%
\protect\vphantom()
ut differentia corporum%
\protect\index{Sachverzeichnis}{differentia corporum}
ad corpus majus.%
\protect\index{Sachverzeichnis}{corpus majus}
Ita ut
si celeritas excipientis%
\protect\index{Sachverzeichnis}{celeritas corporis excipientis}%
\protect\index{Sachverzeichnis}{corpus excipiens}
fuerit major,
corpus incurrens%
\protect\index{Sachverzeichnis}{corpus incurrens}
progrediatur,%
\protect\index{Sachverzeichnis}{progressio corporis incurrentis}
si vero sit minor altera illa
%
\edtext{celeritate
quam dixi,
tunc}{%
\lemma{celeritate}\Bfootnote{%
\textit{(1)}~tunc
\textit{(2)}~quam dixi, tunc%
~\textit{L}}}
%
corpus incurrens repellatur.%
\protect\index{Sachverzeichnis}{repulsa corporis incurrentis}%
\protect\index{Sachverzeichnis}{corpus repulsum}
Omnem autem reliquam vim%
\protect\index{Sachverzeichnis}{vis reliqua}
ad excipiens pertinere patet.
Hinc jam
re omnino per certam demonstrationem%
\protect\index{Sachverzeichnis}{demonstratio certa}
definita%
\protect\index{Sachverzeichnis}{res definita}
sciri potest
%
\edtext{quando neque repulsa%
\protect\index{Sachverzeichnis}{repulsa corporis incurrentis}
neque progressio,%
\protect\index{Sachverzeichnis}{progressio corporis incurrentis}
sed quies%
\protect\index{Sachverzeichnis}{quies corporis incurrentis}%
}{%
\lemma{quando}\Bfootnote{%
\textit{(1)}~quies
\textit{(2)}~neque repulsa % neque progressio, 
\lbrack...\rbrack\ sed quies%
~\textit{L}}}
%
oriatur.
Priores etiam
%
\edtext{in lemmatibus%
\protect\index{Sachverzeichnis}{lemma}%
}{%
\lemma{in lemmatibus}\Cfootnote{%
Vgl. N.~\ref{dcc_03}, %??S01\textsubscript{4},
S.~\refpassage{LH_35_09_23_007-008_exlemmatibus-1}{LH_35_09_23_007-008_exlemmatibus-2}.%
}}
%
expressae progressiones%
\protect\index{Sachverzeichnis}{progressio}
figuris%
\protect\index{Sachverzeichnis}{figura} exhiberi possunt.
Imo id liquidissime demonstratum habetur,
cum contra
%
\edtext{de generali ista regula%
\protect\index{Sachverzeichnis}{regula generalis}%
}{%
\lemma{de}\Bfootnote{%
\textit{(1)}~his
\textit{(2)}~generali ista regula%
~\textit{L}}}
%
nonnihil adhuc dubitari possit.
Dubitari inquam potest,
%
an quando incurrens%
\protect\index{Sachverzeichnis}{corpus incurrens}
repellitur,%
\protect\index{Sachverzeichnis}{repulsa corporis incurrentis}
repellatur%
\protect\index{Sachverzeichnis}{corpus repulsum}
%\edtext{}{%
%\lemma{an}\Bfootnote{%
%\textit{(1)}~corpus
%\textit{(2)}~quando incurrens repellitur repellatur%
%~\textit{L}}}
%
non tantum excessu repulsae%
\protect\index{Sachverzeichnis}{excessus repulsae}
supra progressionem,%
\protect\index{Sachverzeichnis}{progressio corporis incurrentis}
%
sed omnino
%\edtext{}{%
%\lemma{sed}\Bfootnote{%
%\textit{(1)}~an
%\textit{(2)}~omnino%
%~\textit{L}}}
%
tota ista vi
conflictu%
\protect\index{Sachverzeichnis}{conflictus}
destructa,%
\protect\index{Sachverzeichnis}{vis destructa}
seu an non vis conflictu destructa
ipsi potius quam excipienti tribuenda sit.
Idem est cum progreditur,
posset enim utique progredi tota
%
\edtext{vi%;%
\protect\index{Sachverzeichnis}{vis progressus}%
\lbrack,\rbrack\
nimirum}{%
\lemma{vi;}\Bfootnote{% \hspace{-0,5mm}%
\textit{(1)}~sed in eo casu
\textit{(2)}~nimirum%
~\textit{L}}}
%
et destructa,%
\protect\index{Sachverzeichnis}{vis destructa}
et excessu\lbrack;\rbrack\
sed in progressu%
\protect\index{Sachverzeichnis}{progressus corporis incurrentis}
id peculiariter refutatur,
nam si hinc major ei iterum tribueretur celeritas,%
\protect\index{Sachverzeichnis}{celeritas major}
quae
%
est excipientis,%
\protect\index{Sachverzeichnis}{celeritas corporis excipientis}
%\edtext{}{%
%\lemma{est}\Bfootnote{%
%\textit{(1)}~pro
%\textit{(2)}~excipientis%
%~\textit{L}}}
%
novus iterum oriretur
%
\edtext{conflictus;%
\protect\index{Sachverzeichnis}{conflictus novus}
unde eodem}{%
\lemma{conflictus;}\Bfootnote{%
\textit{(1)}~et quod eodem
\textit{(2)}~unde eodem%
~\textit{L}}}
%
quo
%
\edtext{paulo ante}{%
\lemma{paulo ante}\Cfootnote{%
S.~\refpassage{LH_35_09_23_009r_uvodi-1}{LH_35_09_23_009r_uvodi-2}.}}
%
modo oriretur absurditas.%
\protect\index{Sachverzeichnis}{absurditas}
Et hinc argumento a simili,%
\protect\index{Sachverzeichnis}{argumentum a simili}
quod
%
\lbrack10~r\textsuperscript{o}\rbrack\ % % % %   Blatt 10 r
%
in talibus
%\edtext{}{%
%\lemma{in}\Bfootnote{%
%\textit{(1)}~his
%\textit{(2)}~talibus%
%~\textit{L}}}
%
non probabile%
\protect\index{Sachverzeichnis}{argumentum probabile}
sed demonstrativum%
\protect\index{Sachverzeichnis}{argumentum demonstrativum}
%
\edtext{est,
colligetur etiam in caeteris casibus,%
\protect\index{Sachverzeichnis}{casus}%
}{%
\lemma{est,}\Bfootnote{%
\textit{(1)}~non
\textit{(2)}~colligetur etiam
\textit{(a)}~non ipsi
\textit{(b)}~in caeteris casibus,%
~\textit{L}}}
%
ubi non aeque ostendi potest absurditas%
\protect\index{Sachverzeichnis}{absurditas}%
\lbrack,\rbrack\
non esse hoc modo ratiocinandum,
sed vim destructam%
\protect\index{Sachverzeichnis}{vis destructa}
excipienti transcribendam.%
\protect\index{Sachverzeichnis}{vis excipienti transcribenda}
% \pend%
% %
% \pstart%
Verum facilior erit demonstratio
pro regula universali,%
\protect\index{Sachverzeichnis}{regula universalis}
ubi prius specialem%
\protect\index{Sachverzeichnis}{regula specialis}
de casu quietis%
\protect\index{Sachverzeichnis}{casus quietis}
demonstraverimus.
Theorema sane memorabile%
\protect\index{Sachverzeichnis}{theorema memorabile}
hoc est:
\pend%
% \newpage%
%
\pstart%
Si corpus minus assequatur majus,%
\protect\index{Sachverzeichnis}{corpus minus in majus}
et sit celeritas%
\protect\index{Sachverzeichnis}{celeritas corporis excipientis}
%
\edtext{excipientis ad celeritatum differentiam,%
\protect\index{Sachverzeichnis}{differentia celeritatum}%
}{%
\lemma{excipientis}\Bfootnote{%
\textbar~minor \textit{erg. u. gestr.}~%
\textbar\ ad
\textit{(1)}~celeritatem
\textit{(a)}~incurrentis
\textit{(b)}~majorem
\textit{(2)}~celeritatum differentiam%
~\textit{L}}}
%
ut differentia corporum%
\protect\index{Sachverzeichnis}{differentia corporum}
ad corpus
%
\edtext{excipiens,%
\protect\index{Sachverzeichnis}{corpus excipiens}
tunc}{%
\lemma{excipiens}\Bfootnote{\hspace{-0,5mm}%
\textbar~majus \textit{gestr.}~%
\textbar~, tunc%
~\textit{L}}}
%
post incursum corpus incurrens%
\protect\index{Sachverzeichnis}{corpus incurrens}
quiescet.%
\protect\index{Sachverzeichnis}{quies corporis incurrentis}%
%%%%%%%%%%%%%%%%%%%%%%%%
%   Erste Marginalie:   Anfang
%%%%%%%%%%%%%%%%%%%%%%%%
\edtext{}{%
{\xxref{LH_35_09_23_009-010_nmhbvx-1}{LH_35_09_23_009-010_nmhbvx-2}}%
{\lemma{\textit{Am Rand:}}\Afootnote{%
\textsuperscript{[a]}~%
Imo dicendum:
si sit celeritas minor ad majorem,%
\protect\index{Sachverzeichnis}{celeritas minor}%
\protect\index{Sachverzeichnis}{celeritas major}
ut corporum differentia%
\protect\index{Sachverzeichnis}{differentia corporum}
ad duplum majoris%
\lbrack,\rbrack\
incurrens%
\protect\index{Sachverzeichnis}{corpus incurrens}
quiescet%
\protect\index{Sachverzeichnis}{quietis corporis incurrentis}%
\lbrack:\rbrack\
$\displaystyle\frac{a - b}{2a\protect\vphantom{\textit{(}\mu\textit{)}}}$
aequ.
$\displaystyle\frac{\textit{(}\mu\textit{)}}{m}.$
Nota duplum majoris est
summa summae et differentiae%
\lbrack:\rbrack\
$a + b, +\, a - b$
aequ.
2\textit{a}.%
\textsuperscript{\lbrack b\rbrack}
%\lbrack\textit{Hierüber, gestr.:}\rbrack\
{\footnotesize%
\newline\vspace{-0.3em}%
\newline%
\textsuperscript{[a]}~%
\textit{(1)}~$\displaystyle\frac{a - b}{2a}$ aequ.\! $\displaystyle\frac{\textit{(}\mu\textit{)}}{\textit{(}m\textit{)}},$ $a + b, +\, a - m$ aequ. 2\textit{a}
\textit{(2)}~\textit{cm} ad \textit{dc} ut \textit{cd} ad \textit{cs}
\textit{(3)}~Imo dicendum: \lbrack...\rbrack\ aequ. 2\textit{a}.%
~\textit{L}%
\quad
%\newline%
%\newline%
%{\footnotesize%
\textsuperscript{\lbrack b\rbrack}~Siehe die Randbemerkung zu S.~\refpassage{LH_35_09_23_010r_marg_zfgd}{LH_35_09_23_010r_marg_zfgd} und die zugehörige Erläuterung.%
%}%
%\newline%
}}}}%
%%%%%%%%%%%%%%%%%%%%%%%%
%   Erste Marginalie:   Ende
%%%%%%%%%%%%%%%%%%%%%%%%
\, Seu\,
\edlabel{LH_35_09_23_009-010_nmhbvx-1}%
\textso{si sit celeritas minor}%
\protect\index{Sachverzeichnis}{celeritas minor}%
\protect\index{Sachverzeichnis}{differentia celeritatum}
%
\edtext{\textso{ad differentiam celeritatum,}}{%
\lemma{\textso{ad}}\Bfootnote{%
\textit{(1)}~\textso{majorem}
\textit{(2)}~\textso{differentiam celeritatum}%
~\textit{L}}}
%
\textso{ut corporum differentia ad corpus majus,
tunc corpus minus incurrens in majus,
quiescet.}%
\protect\index{Sachverzeichnis}{differentia corporum}%
\protect\index{Sachverzeichnis}{corpus majus praecedens}%
\protect\index{Sachverzeichnis}{corpus minus in majus}%
\protect\index{Sachverzeichnis}{corpus incurrens}%
\protect\index{Sachverzeichnis}{quies corporis incurrentis}%
\edlabel{LH_35_09_23_009-010_nmhbvx-2}
\pend%
% \newpage%
%
\pstart%
Hoc ita demonstratur%
\lbrack,\rbrack\
vide
%
\edtext{figur.~9.%
\protect\index{Sachverzeichnis}{figura}%
}{%
\lemma{figur.~9}\Cfootnote{%
Vgl. N.~\ref{dcc_03}, %??S01\textsubscript{4}, 
S.~\pageref{LH_35_09_23_008v_Fig.1},
Diagramm \lbrack\textit{Fig.~1}\rbrack.%
}}
%
Sit corpus minus incurrens \textit{A},%
\protect\index{Sachverzeichnis}{corpus minus in majus}%
\protect\index{Sachverzeichnis}{corpus incurrens}
majus praecedens \textit{B},%
\protect\index{Sachverzeichnis}{corpus majus praecedens}
%
\edtext{quae concurrent}{%
\lemma{quae}\Bfootnote{%
\textit{(1)}~se assequentur
\textit{(2)}~concurrent%
~\textit{L}}}
%
in \textit{(A)},\textit{(B)} puncto,
corpore \textit{B} praecedente%
\protect\index{Sachverzeichnis}{corpus praecedens}
celeritate \textit{B(B)},%
\protect\index{Sachverzeichnis}{celeritas corporis praecedentis}
corpore vero \textit{A} assequente%
\protect\index{Sachverzeichnis}{corpus assequens}
celeritate \textit{A(A)},%
\protect\index{Sachverzeichnis}{celeritas corporis assequentis}
patet
celeritatem ut \textit{B(B)} vel \textit{B(A)}
esse utrique communem,%
\protect\index{Sachverzeichnis}{celeritas communis}
%
\edtext{nempe celeritatem majoris}{%
\lemma{nempe}\Bfootnote{%
\hspace{-0,5mm}celeritatem
\textit{(1)}~minoris
\textit{(2)}~majoris%
~\textit{L}}}
%
sive excipientis;%
\protect\index{Sachverzeichnis}{celeritas corporis excipientis}
at celeritatem ut \textit{AB},
quae est differentia celeritatum,%
\protect\index{Sachverzeichnis}{differentia celeritatum}
esse incurrenti \textit{A}%
\protect\index{Sachverzeichnis}{celeritas corporis incurrentis}
%
%%%%%%%%% <<<< Runde Klammer zu eckiger Klammer
\edtext{propriam,
\edtext{\lbrack perinde%
\edlabel{LH_35_09_23_010r_jdzwrkm-1}}{%
\lemma{\lbrack perinde}\Cfootnote{%
Eckige Klammer von Leibniz.}}%
}{%
\lemma{propriam,}\Bfootnote{%
\textit{(1)}~(\protect\vphantom)perinde
\textit{(2)}~\lbrack perinde%
~\textit{L}}}
%%%%%%%%% >>>>
%
ac si in navi%
\protect\index{Sachverzeichnis}{navis}
ferrentur lata celeritate communi%
\protect\index{Sachverzeichnis}{celeritas communis}
$B \underset{\displaystyle \textit{(A)}}{\textit{(B)}}$
et corpore \textit{B} in navi%
\protect\index{Sachverzeichnis}{navis}
quiescente%
\protect\index{Sachverzeichnis}{corpus quiescens}
%
\edtext{\lbrack ferretur\rbrack}{%
\lemma{feretur}\Bfootnote{%
\textit{L~ändert Hrsg.}}}
%
\textit{A} in navi%
\protect\index{Sachverzeichnis}{navis}
celeritate
%
%%%%%%%%% <<<< Runde Klammer zu eckiger Klammer
\edtext{propria
\textit{AB}\edtext{.\rbrack%
\edlabel{LH_35_09_23_010r_jdzwrkm-2}}{%
\lemma{\textit{AB}.\,\rbrack}\Cfootnote{%
Eckige Klammer von Leibniz.}}%
}{%
\lemma{propria}\Bfootnote{%
\textit{(1)}~\textit{AB}.\protect\vphantom()
\textit{(2)}~\textit{AB}.\rbrack%
~\textit{L}}}
%%%%%%%%% >>>>
%
%%%%%%%%%%%%%%%%%%%%%%%%%%
% Große Marginalie: Anfang
%%%%%%%%%%%%%%%%%%%%%%%%%%
%
\edtext{}{%
{\xxref{LH_35_09_23_010r_jdzwrkm-1}{LH_35_09_23_010r_ndadfk-2}}%
{\lemma{\textit{Am Rand:}}\Afootnote{%
Patet hinc
si corpus excipiens sit%
\protect\index{Sachverzeichnis}{corpus excipiens}
duplum\textsuperscript{[a]}
incurrentis\lbrack,\rbrack\
debere incursum%
\protect\index{Sachverzeichnis}{incursus}
esse triplum praecessionis;%
\protect\index{Sachverzeichnis}{praecessio}
ut corpus incurrens quiescat.%
\protect\index{Sachverzeichnis}{corpus incurrens}
Si vero sint aequalia corpora,%
\protect\index{Sachverzeichnis}{corpora aequalia}
tunc incursus%
\protect\index{Sachverzeichnis}{incursus}
debet esse infinituplus praecessione.%
\protect\index{Sachverzeichnis}{praecessio}%
\textsuperscript{\lbrack\textasteriskcentered\rbrack}
%
\newline%
\hspace*{7,5mm}%
Si celeritas una sit alterius dupla,
tunc\textsuperscript{[b]}
minor erit aequalis differentiae celeritatum,%
\protect\index{Sachverzeichnis}{differentia celeritatum}
ergo\textsuperscript{\lbrack\textasteriskcentered\textasteriskcentered\rbrack}
ut fiat quies,
corporum differentia debet esse aequalis majori,%
\protect\index{Sachverzeichnis}{differentia corporum}
quod est impossibile.
Ergo si celeritas una alterius dupla
sit\textsuperscript{\lbrack\textasteriskcentered\textasteriskcentered\textasteriskcentered\rbrack}
impossibile est talem fingi incursum%
\protect\index{Sachverzeichnis}{incursus}
ut oriatur quies.%
\protect\index{Sachverzeichnis}{quies}%
%
\newline\vspace{-0.3em}%
\newline%
\textsuperscript{\lbrack\textasteriskcentered\rbrack}~%
\textit{Nachträglich hinzugefügt, auf den umklammerten Abschnitt} Patet \lbrack...\rbrack\ praecessione \textit{bezogen:}
Manet post reformationem.%
\protect\index{Sachverzeichnis}{reformatio}
\quad
\textsuperscript{\lbrack\textasteriskcentered\textasteriskcentered\rbrack}~%
\textit{Nachträglich hinzugefügt, auf den Abschnitt} Si celeritas \lbrack...\rbrack\ impossibile \textit{bezogen:}
Haec ratio nihil valet%
\protect\index{Sachverzeichnis}{ratio}
sed manet conclusio.%
\protect\index{Sachverzeichnis}{conclusio}
\quad
\textsuperscript{\lbrack\textasteriskcentered\textasteriskcentered\textasteriskcentered\rbrack}~%
\textit{Nach\-träg\-lich hinzugefügt, auf den um\-klam\-mer\-ten Abschnitt} Ergo \lbrack...\rbrack\ quies \textit{bezogen:}
NB Manet post reform.%
\protect\index{Sachverzeichnis}{reformatio}
quia $a - b$ semper minor quam 2\textit{a}.
%
\newline\vspace{-0.3em}%
\newline%
{\footnotesize%
\textsuperscript{[a]}~duplum
\textit{(1)}~excipientis
\textit{(2)}~incurrentis%
~\textit{L}
\quad
\textsuperscript{[b]}~tunc
\textit{(1)}~differentia cele
\textit{(2)}~minor%
~\textit{L}\vspace{-3mm}%
% \newline%
}}}}
%
%%%%%%%%%%%%%%%%%%%%%%%%
% Große Marginalie: Ende
%%%%%%%%%%%%%%%%%%%%%%%%
%
Porro patet corpus \textit{A}%
\protect\index{Sachverzeichnis}{corpus incurrens}
incurrens in \textit{B}
non agere in ipsum celeritate%
\protect\index{Sachverzeichnis}{celeritas communis}
%
\edtext{communi;
quia si}{%
\lemma{communi;}\Bfootnote{%
\textit{(1)}~itaque in ipsum ita aget, ac si,
\textit{(2)}~quia si%
~\textit{L}}}
%
aequali celeritate
$B \underset{\displaystyle \textit{(A)}}{\textit{(B)}}$
tantum ambo incederent,
in se invicem non
%
\edtext{agerent%
\lbrack;\rbrack\
\edlabel{LH_35_09_23_010r_ndadfk-1}%
ergo per incursum \textit{A} impellet \textit{B},}{%
\lemma{agerent,}\Bfootnote{%
\textit{(1)}~aget vero
\textit{(2)}~ergo per % incursum \textit{A} 
\lbrack...\rbrack\ impellet \textit{B},%
~\textit{L}}}
%
ut quiescens,%
\protect\index{Sachverzeichnis}{corpus quiescens}
differentia celeritatum \textit{AB};%,%
\protect\index{Sachverzeichnis}{differentia celeritatum}
quo facto cum sit minus quam \textit{B},
repelletur ab eo celeritate%
\protect\index{Sachverzeichnis}{corpus repulsum}
quae sit ad celeritatem \textit{AB},
ut differentia corporum,%
\protect\index{Sachverzeichnis}{differentia corporum}
seu ut $B - A,$
ad corpus
%
\edtext{majus \textit{B}.%
\edlabel{LH_35_09_23_010r_ndadfk-2}
%
%%%%%%%%%%%%%%%%%%%%%%%%%%%%%%%%
% Zweitgrößte Marginalie: Anfang
%%%%%%%%%%%%%%%%%%%%%%%%%%%%%%%%
%
\edtext{}{%
{\xxref{LH_35_09_23_010r_bnmf-1}{LH_35_09_23_010r_bnmf-2}}%
{\lemma{\textit{Am Rand:}}\Afootnote{%
Conatus repulsae%
\protect\index{Sachverzeichnis}{conatus repulsae}
$r \ \sqcap \ \displaystyle\frac{b - a}{b} \, \smallfrown \, e - i \ \sqcap \ e - i,\!,\, -\, \displaystyle\frac{a}{b}e\, +\displaystyle\frac{a}{b}i.$
\quad
Conatus progressus%
\protect\index{Sachverzeichnis}{conatus progressus}
$p\, \sqcap\, i.$
\quad
Ergo $r \ \sqcap \ 1-\displaystyle\frac{a}{b}.\ $
$p \ \sqcap \ 1.\ $
$p \ \groesser \ r.\ $
Ergo si celeritas incursus%
\protect\index{Sachverzeichnis}{celeritas incursus}
dupla sit
celeritatis\textsuperscript{[a]}
antecessus,%
\protect\index{Sachverzeichnis}{celeritas antecessus}
corpus incurrens%
\protect\index{Sachverzeichnis}{corpus incurrens}
progreditur post
incursum,\textsuperscript{[b]}
quantumcunque corpus excipiens%
\protect\index{Sachverzeichnis}{corpus excipiens}
excedat incurrens.
\newline\vspace{-0.3em}%
\newline%
{\footnotesize%
\textsuperscript{[a]}~celeritatis
\textit{(1)}~progressionis
\textit{(2)}~antecessus%
~\textit{L}
\quad
\textsuperscript{[b]}~incursum,
\textit{(1)}~quodcunque sit
\textit{(2)}~quantumcunque
\textit{(a)}~sit
\textit{(b)}~corpus excipiens excedat incurrens.%
~\textit{L}%
}}}}%
%
%%%%%%%%%%%%%%%%%%%%%%%%%%%%%%
% Zweitgrößte Marginalie: Ende
%%%%%%%%%%%%%%%%%%%%%%%%%%%%%%
%
\edlabel{LH_35_09_23_010r_bnmf-1}%
Retrorsum ergo tendet}{%
\lemma{majus}\Bfootnote{%
\hspace{-0,5mm}\textit{B}.
\textit{(1)}~Duo ergo erunt
\textit{(2)}~Retrorsum ergo tendet%
~\textit{L}}}
%
corpus \textit{A} celeritate hujusmodi,
sed idem prorsum tendit,
seu pergere adhuc conatur celeritate communi
ut $B \underset{\displaystyle \textit{(A)}}{\textit{(B)}};$%
\protect\index{Sachverzeichnis}{celeritas communis}
quod si ergo aequales sint hi duo conatus,%
\protect\index{Sachverzeichnis}{conatus aequales}
id est si celeritas communis,%
\protect\index{Sachverzeichnis}{celeritas communis}
sive celeritas excipientis,%
\protect\index{Sachverzeichnis}{celeritas excipientis}
sive
%
\edtext{celeritas minor
$B \protect\underset{\displaystyle \textit{(A)}}{\textit{(B)}}$
sit}{%
\lemma{celeritas}\Bfootnote{%
\textit{(1)}~\textit{B(B)}
\textit{(2)}~minor
\textit{(a)}~sit
\textit{(b)}~$B \protect\underset{\displaystyle \textit{(A)}}{\textit{(B)}}$ sit%
~\textit{L}}}
%
etiam
%
\edtext{ad \textit{AB} celeritatum differentiam,%
\protect\index{Sachverzeichnis}{differentia celeritatum}}{%
\lemma{ad}\Bfootnote{%
\textit{(1)}~celeritatem
\textit{(2)}~\textit{AB} celeritatum differentiam,%
~\textit{L}}}
%
\edtext{ut differentia corporum $B - A$%
\protect\index{Sachverzeichnis}{differentia corporum}
ad corpus}{%
\lemma{ut}\Bfootnote{%
\textit{(1)}~corpus \textit{A}
\textit{(2)}~differentia corporum $A - B$ ad
\textit{(a)}~corpus
\textit{(b)}~\textit{AB} celeritatum differentiam
\textit{(3)}~differentia corporum $B - A$ ad corpus%
~\textit{L}}}
%
majus \textit{B},
tunc nulla ratio intelligi potest,
cur alter horum
%
\edtext{conatuum%
\protect\index{Sachverzeichnis}{conatus repulsae}%
\protect\index{Sachverzeichnis}{conatus progressus}%
}{%
\lemma{conatuum}\Bfootnote{%
\textit{erg.~L}}}
%
praevaleat,
adeoque quiescet%
\protect\index{Sachverzeichnis}{corpus quiescens}
corpus incurrens.%
\protect\index{Sachverzeichnis}{corpus incurrens}%
\edlabel{LH_35_09_23_010r_bnmf-2}
% \pend%
% %
% \pstart%
Vis autem tota in corpus excipiens
\edlabel{LH_35_09_23_009-010_quarta1}%
transferetur.%
\protect\index{Sachverzeichnis}{vis excipienti transferta}%
\edtext{}{%
{\xxref{LH_35_09_23_009-010_quarta1}{LH_35_09_23_009-010_quarta2}}%
{\lemma{transferetur}\Bfootnote{%
\textit{(1)}~Nec video vel
\textit{(2)}~Hinc jam%
~\textit{L}}}}%
%
\pend%
%
\pstart%
Hinc jam%
\edlabel{LH_35_09_23_009-010_quarta2}
%
caetera etiam demonstrantur,
seu dubitatio%
\protect\index{Sachverzeichnis}{dubitatio objecta}
generali ratiocinationi%
\protect\index{Sachverzeichnis}{ratiocinatio generalis}
objecta tollitur.
%
\edtext{Nam quando aequatur vis repulsae%
\protect\index{Sachverzeichnis}{vis repulsae}
et vis progressus communis,%
\protect\index{Sachverzeichnis}{vis progressus communis}
tunc quies sequitur,}{%
\lemma{Nam}\Bfootnote{%
\textit{(1)}~si
\textit{(2)}~quando
\textit{(a)}~constat nullam esse repulsam, nullamque esse
\textit{(b)}~aequatur vis repulsae et vis
\textit{(aa)}~celeritatis
\textit{(bb)}~progressus communis, tunc quies sequitur,%
~\textit{L}}}
%
seu nec repulsa%
\protect\index{Sachverzeichnis}{repulsa}
nec progressus,%
\protect\index{Sachverzeichnis}{progressus}
ergo si paulum excedat conatus \mbox{progrediendi}%
\protect\index{Sachverzeichnis}{conatus progrediendi}
conatum
%
\edtext{repulsae,%
\protect\index{Sachverzeichnis}{conatus repulsae}
progredietur quidem corpus incurrens,%
\protect\index{Sachverzeichnis}{corpus incurrens}
sed}{%
\lemma{repulsae,}\Bfootnote{%
\textit{(1)}~non potest certe
\textit{(2)}~progredietur quidem corpus
\textit{(a)}~, sed
\textit{(b)}~incurrens, sed%
~\textit{L}}}
%
celeritate etiam parva,
%
\edtext{non vero celeritatibus illis destructis,%
\protect\index{Sachverzeichnis}{celeritas destructa}%
}{%
\lemma{non}\Bfootnote{%
\textit{(1)}~semper vero aggregat
\textit{(2)}~vero celeritatibus illis destructis,%
~\textit{L}}}
%
simul additis,%
\protect\index{Sachverzeichnis}{celeritas addita}
quae possunt esse maximae,
quod absurdum foret.%
\protect\index{Sachverzeichnis}{absurdum}
Idem est quando repulsa%
\protect\index{Sachverzeichnis}{repulsa}
%
\edtext{vincit.
Aggregatum ergo}{%
\lemma{vincit.}\Bfootnote{%
\textit{(1)}~Non e
\textit{(2)}~Aggregatum ergo%
~\textit{L}}}
%
destructarum virium%
\protect\index{Sachverzeichnis}{vis destructa}
transferetur in excipiens;%
\protect\index{Sachverzeichnis}{vis excipienti transferta}
excessus fortioris vero%
\protect\index{Sachverzeichnis}{excessus corporis fortioris}
incurrenti relinquetur.
\pend%
\newpage
\pstart%
Elegans%
\edlabel{LH_35_09_23_010r_elegansprop_cfjg-1}
est
%
\edtext{propositio%
\protect\index{Sachverzeichnis}{propositio elegans}
quam de casu quietis attulimus,%
\protect\index{Sachverzeichnis}{casus quietis}}{%
\lemma{propositio \lbrack...\rbrack\ attulimus}\Cfootnote{%
Vgl. S.~\refpassage{LH_35_09_23_009-010_nmhbvx-1}{LH_35_09_23_009-010_nmhbvx-2}.%
}}
nempe si sit
%
\edtext{celeritas minor seu excipientis%
\protect\index{Sachverzeichnis}{celeritas corporis excipientis}%
}{%
\lemma{celeritas}\Bfootnote{\hspace{-0,5mm}%
\textit{(1)}~\textbar~minor \textit{streicht Hrsg. nach~E,\cite{01056} S.~104}~\textbar\
\textit{(2)}~minor seu excipientis%
~\textit{L}}}
%
ad differentiam celeritatum,%
\protect\index{Sachverzeichnis}{differentia celeritatum}
ut differentia corporum%
\protect\index{Sachverzeichnis}{differentia corporum}
ad corpus majus,
quiescit%
\protect\index{Sachverzeichnis}{corpus incurrens}%
\protect\index{Sachverzeichnis}{corpus post incursum quiescens}
%
\edtext{}{%
{\xxref{LH_35_09_23_010r_bsjfz-1}{LH_35_09_23_010r_bsjfz-2}}%
{\lemma{si quiescat \lbrack...\rbrack\ majus}\Cfootnote{%
Der Satz ist nur dann gültig, wenn \textendash\ entgegen der Voraussetzung $\displaystyle b > a$ \textendash\ die Bedingung $\displaystyle b = a$ gilt.
Daraus folgt $\displaystyle 0 : e = 0 : b,$
womit das Theorem $\displaystyle i : (e-i) = (b-a) : b,$ wie Leibniz behauptet, übereinstimmt.%
}}}%
%
\edtext{incurrens post incursum;%
\protect\index{Sachverzeichnis}{incursus}
et respondet isti:
\edlabel{LH_35_09_23_010r_bsjfz-1}%
si quiescat}{%
\lemma{incurrens}\Bfootnote{%
\textit{(1)}~, et respondet isti, si quies
\textit{(2)}~post incursum; % et respondet isti: 
\lbrack...\rbrack\ si quiescat%
~\textit{L}}}
%
excipiens%
\protect\index{Sachverzeichnis}{corpus excipiens}
ante incursum,%
\protect\index{Sachverzeichnis}{corpus ante concursum quiescens}
fit celeritas minor
%
%\edtext{\lbrack minoris\rbrack}{%
%\lemma{minor}\Bfootnote{%
%\textit{L~ändert Hrsg. nach~E,\cite{01056} S.~104}}}
%
seu%
\protect\index{Sachverzeichnis}{celeritas corporis excipientis}
%
\edtext{\lbrack excipientis\rbrack}{%
\lemma{incurrentis}\Bfootnote{%
\textit{L~ändert Hrsg.}}}
%
ad differentiam celeritatum%
\protect\index{Sachverzeichnis}{differentia celeritatum}
(\protect\vphantom)%
id est celeritatem incurrentis%
\protect\index{Sachverzeichnis}{celeritas corporis incurrentis}%
\protect\vphantom()
ut
%
\edtext{\lbrack differentia\rbrack}{%
\lemma{differentiae}\Bfootnote{%
\textit{L~ändert Hrsg.}}} corporum%
\protect\index{Sachverzeichnis}{differentia corporum}
ad corpus majus.%
\protect\index{Sachverzeichnis}{corpus majus}%
\edlabel{LH_35_09_23_010r_bsjfz-2}%
\edlabel{LH_35_09_23_010r_elegansprop_cfjg-2}%
%%%%%%%%%%%%%%%%%%%%%%%%%
%    Neue Marginalie:    Anfang
%%%%%%%%%%%%%%%%%%%%%%%%%
\edtext{}{%
\lemma{\textit{Am Ende des Absatzes:}}\Afootnote{%
NB.
Quando est celeritas minor
ad\textsuperscript{\lbrack a\rbrack}
differentiam celeritatum%
\protect\index{Sachverzeichnis}{differentia celeritatum}
ut differentia corporum%
\protect\index{Sachverzeichnis}{differentia corporum}
ad corpus majus,
tunc etiam est celeritas minor ad majorem
ut corporum differentia ad corporum summam,%
\protect\index{Sachverzeichnis}{summa corporum}
ut calculus ostendit facillimus:%
\protect\index{Sachverzeichnis}{calculus facillimus}
\protect\rule[-1mm]{0mm}{7,0mm}%
$\displaystyle\frac{i}{e - i}$
\mbox{aequ.}
$\displaystyle\frac{a}{b}.$
Ergo
\textit{bi} aequ. $ae - ai,$
seu
$\overline{a + b}\,i$
aequ.
\textit{ae},
seu
$\displaystyle\frac{i}{e}$
aequ.
$\displaystyle\frac{a}{a + b}.$
% \protect\rule[-3mm]{0mm}{8mm}
Vide sched.~5.%
\protect\index{Sachverzeichnis}{scheda}%
\protect\index{Sachverzeichnis}{pagina}
pag.~3.%
\textsuperscript{\lbrack b\rbrack}
\newline%
\newline%
{\footnotesize%
\textsuperscript{\lbrack a\rbrack}~ad
\textit{(1)}~majorem,
\textit{(2)}~differentiam celeritatum%
~\textit{L}
\quad
\textsuperscript{\lbrack b\rbrack}~pag.~3:
Vgl. N.~\ref{dcc_05}, %??S01\textsubscript{6},
S.~\refpassage{LH_35_09_23_012r_aliuscasus_rubz-1}{LH_35_09_23_012r_aliuscasus_rubz-2}.
Es gilt aber zu bemerken, dass die Proportionen $\displaystyle i : (e-i) = (b-a) : b$ und $\displaystyle i : (e-i) = a : b$ nur dann gleich sind, wenn $b = 2a$.
Leibniz rechnet hier \textendash\ ebenso wie a.a.O. \textendash\ wohl versehentlich mit $\displaystyle a :b$ statt $\displaystyle(b-a) : b$.%
\newline%
}}}%
\edlabel{LH_35_09_23_010r_marg_zfgd}
%%%%%%%%%%%%%%%%%%%%%%%%%
%    Neue Marginalie:    Ende
%%%%%%%%%%%%%%%%%%%%%%%%%
%
\lbrack10~v\textsuperscript{o}\rbrack\ % % % %   Blatt 10v
%
\pend%
%
\pstart%
Porro ex his etiam nova demonstrandi,
vel saltem inveniendi Methodus duci potest.%
\protect\index{Sachverzeichnis}{methodus demonstrandi}%
\protect\index{Sachverzeichnis}{methodus inveniendi}
Nempe hoc theorema%
\protect\index{Sachverzeichnis}{theorema generale}
%
\edtext{generale}{%
\lemma{generale}\Bfootnote{%
\textit{erg.~L}}}
%
inventurus
%
\edtext{}{%
{\xxref{LH_35_09_23_009-010_sjhkl-1}{LH_35_09_23_009-010_sjhkl-2}}%
{\lemma{\textit{Am Rand:}}\Afootnote{%
Reformanda ut dixi\textsuperscript{[a]} haec propositio.%
\protect\index{Sachverzeichnis}{propositio reformanda}%
\newline\vspace{-0.3em}%
\newline%
{\footnotesize%
\textsuperscript{[a]}~ut dixi:
Vgl. die Randbemerkung zu
S.~\refpassage{LH_35_09_23_009-010_nmhbvx-1}{LH_35_09_23_009-010_nmhbvx-2}.\vspace{-3mm}%
%\newline%
}}}}%
%
\edlabel{LH_35_09_23_009-010_sjhkl-1}%
\edtext{\lbrack si}{%
\lemma{\lbrack si}\Cfootnote{%
Eckige Klammer von Leibniz.}}
%
sit celeritas minor%
\protect\index{Sachverzeichnis}{celeritas minor}
ad differentiam celeritatum,%
\protect\index{Sachverzeichnis}{differentia celeritatum}
ut corporum differentia%
\protect\index{Sachverzeichnis}{differentia corporum}
est ad corpus majus,%
\protect\index{Sachverzeichnis}{corpus majus}
quiescet incurrens%
\protect\index{Sachverzeichnis}{corpus incurrens}
post%
\protect\index{Sachverzeichnis}{corpus post incursum quiescens}
%
\edtext{incursum\rbrack}{%
\lemma{incursum\rbrack}\Cfootnote{%
Eckige Klammer von Leibniz.}}%
\edlabel{LH_35_09_23_009-010_sjhkl-2}
%
idque adhuc ignorans
quaeret casus quosdam jam notos,%
\protect\index{Sachverzeichnis}{casus notus}
in quibus contingit quies post
%
\edtext{incursum.%
\protect\index{Sachverzeichnis}{quies post incursum}
Qualis est casus}{%
\lemma{incursum.}\Bfootnote{%
\textit{(1)}~Exempli gratia
\textit{(2)}~Qualis est casus%
~\textit{L}}}
%
aequalitatis corporum,%
\protect\index{Sachverzeichnis}{casus aequalitatis corporum}
nempe%
\lbrack:\rbrack\
quando
%
\edtext{corpus incurrit}{%
\lemma{corpus}\Bfootnote{%
\textbar~aequale \textit{gestr.}~%
\textbar\ incurrit%
~\textit{L}}}
%
in aliud aequale quiescens,%
\protect\index{Sachverzeichnis}{corpus aequale quiescens}
tunc
%
\edtext{ipsum incurrens quiescit.%
\protect\index{Sachverzeichnis}{corpus post incursum quiescens}%
}{%
\lemma{ipsum}\Bfootnote{%
\textit{(1)}~quiescens incurrit
\textit{(2)}~incurrens quiescit.%
~\textit{L}}}
%\pend%
%%
%\pstart%
Hinc jam considerandum
quam variis modis celeritas incurrentis%
\protect\index{Sachverzeichnis}{celeritas corporis incurrentis}
in hoc casu exprimi possit;
est celeritas major,%
\protect\index{Sachverzeichnis}{celeritas major}
est aggregatum celeritatum,%
\protect\index{Sachverzeichnis}{aggregatum celeritatum}
est differentia celeritatum,%
\protect\index{Sachverzeichnis}{differentia celeritatum}
est factum ex
%
\edtext{ductu celeritatis majoris}{%
\lemma{ductu}\Bfootnote{%
\textit{(1)}~in celeritatem majorem
\textit{(2)}~celeritatis majoris%
~\textit{L}}}
%
in corpus incurrens,%
\protect\index{Sachverzeichnis}{corpus incurrens}
divisum per corpus excipiens;%
\protect\index{Sachverzeichnis}{corpus excipiens}
etc.
% \pend%
% %
% \pstart%
Haec omnia enim
%
\edtext{coincidunt.
Quies vero}{%
\lemma{coincidunt.}\Bfootnote{%
\textit{(1)}~Celeritas vero
\textit{(2)}~Quies vero%
~\textit{L}}}
%
excipientis hoc loco%
\protect\index{Sachverzeichnis}{quies corporis excipientis}
est celeritas
%
\edtext{minor.%
\protect\index{Sachverzeichnis}{celeritas minor}
Pro variis}{%
\lemma{minor.}\Bfootnote{%
\textit{(1)}~Has varias
\textit{(2)}~Hinc vari
\textit{(3)}~Pro variis%
~\textit{L}}}
%
expressionibus%
\protect\index{Sachverzeichnis}{expressio varia}
variae in hoc casu fieri
%
\edtext{possunt observationes,%
\protect\index{Sachverzeichnis}{observatio varia}%
}{%
\lemma{possunt}\Bfootnote{%
\textit{(1)}~regulae
\textit{(2)}~propositiones vel
\textit{(3)}~observationes,%
~\textit{L}}}
%
exempli gratia in hoc casu
%
\edtext{aggregatum celeritatum%
\protect\index{Sachverzeichnis}{aggregatum celeritatum}
ductum}{%
\lemma{aggregatum}\Bfootnote{%
\textit{(1)}~et differen
\textit{(2)}~ductum
\textit{(3)}~celeritatum ductum%
~\textit{L}}}
%
in
%
\edtext{corpus incurrens,%
\protect\index{Sachverzeichnis}{corpus incurrens}%
}{%
\lemma{corpus}\Bfootnote{%
\textit{(1)}~excipiens
\textit{(2)}~incurrens%
~\textit{L}}}
%
divisum per corpus excipiens%
\protect\index{Sachverzeichnis}{corpus excipiens}
dat differentiam celeritatum.%
\protect\index{Sachverzeichnis}{differentia celeritatum}
Item celeritas minor%
\protect\index{Sachverzeichnis}{celeritas minor}
ducta in corpus majus%
\protect\index{Sachverzeichnis}{corpus majus}
et divisa per corporum differentiam,%
\protect\index{Sachverzeichnis}{differentia corporum}
dat differentiam
%
\edtext{celeritatum%
\protect\index{Sachverzeichnis}{differentia celeritatum}
(\protect\vphantom)%
\edtext{qui coincidit cum nostro}{%
\lemma{qui \lbrack...\rbrack\ nostro}\Cfootnote{%
Siehe hierüber \textsc{Fichant} 1994, S.~227\,f.%
\cite{01056}%
}}%
\protect\vphantom().
Inprimis}{%
\lemma{celeritatum}\Bfootnote{%
\textit{(1)}~; viden
\textit{(2)}~cum ergo hae
\textit{(3)}~(\protect\vphantom)qui coincidit cum nostro\protect\vphantom(). Inprimis%
~\textit{L}}}
%
autem eae observationes considerandae sunt,
quae non sunt cum hoc casu reciprocae
sed quae forte latius patere possint.
Quodsi omnes enumeremus ejusmodi observationes
%
\edtext{seu notabiles hujus casus;}{%
\lemma{seu}\Bfootnote{%
\textit{(1)}~proprietates
\textit{(2)}~notabiles hujus casus;%
~\textit{L}}}
%
necesse est
ut inter caeteras contineatur etiam
%
\edtext{illa observatio}{%
\lemma{illa}\Bfootnote{%
\textit{(1)}~proprietas
\textit{(2)}~observatio%
~\textit{L}}}
%
quae ipsi communis%
\protect\index{Sachverzeichnis}{observatio communis}
est cum aliis omnibus casibus
in quibus fit quies%
\protect\index{Sachverzeichnis}{quies post incursum}
incursu%
\protect\index{Sachverzeichnis}{incursus}
incurrentis in excipiens majus%
\protect\index{Sachverzeichnis}{corpus minus in majus}
(\protect\vphantom)%
nam omne aequale
majus fingi potest
excessu infinite parvo%
\protect\index{Sachverzeichnis}{excessus infinite parvus}%
\protect\vphantom(),
%
\edtext{ideo singulas examinando}{%
\lemma{ideo}\Bfootnote{%
\textit{(1)}~caeteris rejectis
\textit{(2)}~singulas examinando%
~\textit{L}}}
%
incidemus tandem in veram,%
\protect\index{Sachverzeichnis}{observatio vera}
quam plerumque prae caeteris discernere facile est,
et licet non enumeremus observationes omnes,
tamen
%
\edtext{potissimas habere non adeo erit}{%
\lemma{potissimas}\Bfootnote{%
\textit{(1)}~observare non erit
\textit{(2)}~habere non adeo erit%
~\textit{L}}}
%
difficile,
praesertim si calculo agatur:%
\protect\index{Sachverzeichnis}{calculus}
et ex his facilius discernentur eae
quae generaliores sunt,%
\protect\index{Sachverzeichnis}{observatio generalis}
observando an demonstrari potuerint aliter
quam per reciprocas hujus casus proprietates.
Quanquam id saepe sit obscurum in his
praesertim de motu casibus,%
\protect\index{Sachverzeichnis}{casus de motu}
ubi non aeque facile est
omnia calculo subjicere.%
\protect\index{Sachverzeichnis}{calculus}
\pend%
%
\pstart%
Praeterea utile erit
plures casus notos%
\protect\index{Sachverzeichnis}{casus notus}
inter se combinare,%
\protect\index{Sachverzeichnis}{casus combinatus}
et quaerere proprietates ambobus communes,%
\protect\index{Sachverzeichnis}{proprietates communes}
nam quae non sunt tales nec inservire
%
\edtext{possunt,
item jungere inter se diversa,
et quaerere inter ea harmoniam,%
\protect\index{Sachverzeichnis}{harmonia diversorum}
ex. gr. casum
cum incurritur in corpus minus,%
\protect\index{Sachverzeichnis}{corpus minus}
et cum incurritur in corpus majus.%
\protect\index{Sachverzeichnis}{corpus majus}%
}{%
\lemma{possunt,}\Bfootnote{%
\textit{(1)}~ex. gr. alius casus hic notus est, cum corpus aliquod incurrit in aliud corpus immobile, quiescens id est tardius antecedens celeritate minima, tunc enim reflectitur etiam
\textit{(2)}~item jungere \lbrack...\rbrack\ corpus majus.%
~\textit{L}}}
%
\pend%
\count\Bfootins=1200%
\count\Afootins=1200%
\count\Cfootins=1200%
%
% \pstart%
% Scheda quinta
% \pend%
%
%
% % % %    Ende des Textes auf Bl. 10v
%
%