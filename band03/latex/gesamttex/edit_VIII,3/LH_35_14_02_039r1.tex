%
%
%   Band VIII, 3 N.~??A24
%   Signatur/Tex-Datei: LH_35_14_02_039r1
%   RK-Nr. 58233
%   Überschrift: Aus Galileo Galilei, Discorsi [früher: La corda della cetera]
%   Datierung: [??August 1682 – April 1683 ?? wie RK 58234]
%   WZ: (keins)
%   SZ: (keins)
%   Bilddateien (keine)
%
%
\begin{ledgroupsized}[r]{120mm}
\footnotesize
\pstart
\noindent\textbf{Überlieferung:}
\pend
\end{ledgroupsized}
\begin{ledgroupsized}[r]{114mm}
\footnotesize
\pstart \parindent -6mm
\makebox[6mm][l]{\textit{L}}%
Notiz aus \cite{00050}G.~\textsc{Galilei}, \textit{Discorsi}:
LH~XXXV~14,~2 Bl.~39.
Ein Blatt~2\textsuperscript{o}.
Drei Zeilen auf Bl.~39~r\textsuperscript{o};
dort ist ebenfalls N.~14\textsubscript{4} überliefert;
Bl.~39~v\textsuperscript{o} ist leer.
\pend
\end{ledgroupsized}
%
\vspace*{5mm}
\begin{ledgroup}
\footnotesize
\pstart\noindent
\noindent\textbf{Datierungsgründe:}
Die vorliegende Notiz ist auf demselben Blatt überliefert wie die titellose Aufzeich\-nung N.~14\textsubscript{4}, welche editorisch \textit{Solidum ubique aequiresistens} benannt und auf die Zeitspanne von Ende Januar bis März/April 1683 datiert worden ist (siehe zur Begründung die editorische Vorbemerkung zum Textkomplex N.~14, S.~\refpassage{LH_35_14_02_039r2_Datierung-1}{LH_35_14_02_039r2_Datierung-2}).
Anhand der gemeinsamen Überlieferung ist anzunehmen, dass N.~15 zur gleichen Zeit wie N.~14\textsubscript{4} entstand.
\pend%
\pstart%
Die Notiz N.~15 besteht aus zwei kurzen Zitaten über akustische Themen aus Galileis \textit{Discorsi}. 
Sie zeigt somit, dass Leibniz zu dem Zeitpunkt, als er im Rahmen seiner weiteren Untersuchung zur Festigkeit der Balken N.~14\textsubscript{4} verfasste, ein Exemplar der \textit{Discorsi} (vermutlich aus G.~\textsc{Galilei}, \textit{Opere}, Bd.~II, Bologna 1656\cite{01084}) bei sich gehabt haben muss.
Noch Ende Juli/Anfang August 1682 hatte er sich bei E.~Mariotte beklagt, ihm sei infolge laufender Baumaßnahmen in der Hannoveraner Hofbibliothek kein Exemplar der \textit{Discorsi} zugänglich (\cite{01263}\textit{LSB} III,~3 N.~380, S.~679.7\textendash10).%
\protect\index{Namensregister}{\textso{Galilei} (Galilaeus, Galileus), Galileo 1564\textendash1642}%
\protect\index{Namensregister}{\textso{Mariotte}, Edme, Seigneur de Chazeuil ca. 1620\textendash1684}%
\protect\index{Ortsregister}{Hannover}%
\pend%
\pstart%
Spätestens zu der Zeit, als N.~15 entstand, hatte Leibniz zudem das Konzept N.~12\textsubscript{3},~\textit{L\textsuperscript{1}} angefertigt (siehe die editorische Vorbemerkung zum Textkomplex N.~12, S.~\refpassage{explicatiosoni_difuvg-1}{explicatiosoni_difuvg-2}).
Dort hatte er im Rahmen seiner \textit{Explicatio soni et auditus} akustische Themen wie die, die in N.~15 berührt werden, systematisch und ausführlich behandelt.
\pend
\end{ledgroup}
%
\vspace*{8mm}
\pstart
\noindent
\normalsize
%
\lbrack39~r\textsuperscript{o}\rbrack\ % % % %  Bl. 39r
%
% \pend
%
% \pstart
% \noindent
% \centering
Galil. dial. I.
\pend
%\vspace*{0.5em}
%
\pstart
%\noindent
\edtext{La corda della Cetera\protect\index{Sachverzeichnis}{cetera}\protect\index{Sachverzeichnis}{corda della cetera}
movet et facit sonare non tantum chordam\protect\index{Sachverzeichnis}{chorda unisona} unisonam
sed et consonantem\protect\index{Sachverzeichnis}{chorda consonans}
ex octava.\protect\index{Sachverzeichnis}{octava}
Imo et ex quinta.\protect\index{Sachverzeichnis}{quinta}}{%
\lemma{La corda \lbrack...\rbrack\ ex quinta}\Cfootnote{%
\cite{00050}G.~\textsc{Galilei}, \textit{Discorsi}, Leiden 1638, S.~98\cite{00050} (\cite{00048}\textit{GO} VIII, S.~141.35\textendash142.2\cite{00048}).}}
\pend
%
\pstart
\edtext{Chorda\protect\index{Sachverzeichnis}{chorda tensa}
quadruplo pondere\protect\index{Sachverzeichnis}{pondus tendens quadruplum}
ad octavam\protect\index{Sachverzeichnis}{octava} tenditur.}{%
\lemma{Chorda \lbrack...\rbrack\ tenditur}\Cfootnote{%
\cite{00050}a.a.O., S.~100 (\cite{00048}\textit{GO} VIII, S.~143.26\textendash30).}}
\pend
%
%%%%    Ende des Stücks auf Bl. 39r