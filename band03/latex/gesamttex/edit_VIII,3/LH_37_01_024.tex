%   ##
%
%   Band VIII, 3 N.~??X.4 (ex: N.~??A06 / ??X.6 / ??X.5)
%   Signatur/Tex-Datei: LH_37_01_024
%   RK-Nr. 38539
%   Überschrift: Aus und zu Joseph-Guichard Duvernay, Tractatus de organo auditus
%   Datierung: [April 1684 -- erste Hälfte 1685]
%   WZ: Bl. 24: RK-Wz 358 (insgesamt eins, fragmentarisch)
%.  SZ: (keins)
%.  Bilddateien (PDF): (keine)
%
%
\begin{ledgroupsized}[r]{120mm}
\footnotesize
\pstart
\noindent\textbf{Überlieferung:}
\pend
\end{ledgroupsized}
\begin{ledgroupsized}[r]{114mm}
\footnotesize
\pstart \parindent -6mm
\makebox[6mm][l]{\textit{L}}%
Auszüge mit Bemerkungen aus
\cite{01203}\textsc{J.-G. Duverney}, \textit{Tractatus de organo auditus, continens structuram, usum et morbos omnium auris partium, e Gallico Latine versus}, Nürnberg 1684:
LH XXXVII~1 Bl.~24.
Ein senkrecht halbiertes Blatt 2\textsuperscript{o}; % (11~x~33~cm).
Fragment eines Wasserzeichens:
Papier aus dem Harz;
geringfügiger Textverlust am oberen und unteren Rand.
Zwei voll beschriebene Seiten, % 
die inhaltlich mit einem Textabschnitt im Schlussteil von N.~12\textsubscript{3}
(S.~\refpassage{LH_37_01_008v_g-1}{LH_37_01_008v_g-2}, Textschicht \textit{Lil}) sowie mit dem ihm zugrundeliegenden Konzept N.~12\textsubscript{3}, \textit{L\textsuperscript{2}} zusammenhängen.
\pend
\end{ledgroupsized}
%
%
\vspace*{4mm}
\pstart%
\normalsize%
\noindent%
\lbrack24~r\textsuperscript{o}\rbrack\
\pend%
% Überschrift
\pstart%
\centering%
\edtext{% ((Titelblatt))
\textso{De organo auditus}}{%
\lemma{\textso{De organo auditus\,}}%
\Cfootnote{\cite{01203}\textsc{J.-G. Duverney}, \textit{Tractatus de organo auditus}, Nürnberg 1684:
lateinische Übersetzung von \textsc{Ders.},
\cite{01202}\textit{Traité de l'organe de l'ouie%, contenant la structure, les usages et les maladies de toutes les parties de l'oreille
}, Paris 1683.}}
\pend%
\vspace*{0.5em}%
% \newpage
%
\count\Bfootins=1000
\count\Afootins=1000
\count\Cfootins=1000
\pstart%
\noindent%
\edtext{% ((S.~1))
Pars Organi
\edtext{Auditus\protect\index{Sachverzeichnis}{organon auditus} aperta vel caeca.\textso{ Aperta }}{%
\lemma{Auditus}\Bfootnote{%
\textit{(1)}~interna vel externa. 
\textit{(a)}~Interna
\textit{(b)}~Externa
\textit{(2)}~aperta vel caeca. \textso{Aperta}%
~\textit{L}}}%
est quae sine sectione\protect\index{Sachverzeichnis}{sectio} videri potest, estque duplex\lbrack:\rbrack
\textso{ Auris }\protect\index{Sachverzeichnis}{auris}%
\edtext{ipsa
(\phantom)\hspace*{-1.2mm}%
quae eminet extra caput%
\phantom(\hspace*{-1.2mm})
et\textso{ meatus auditorius.}\protect\index{Sachverzeichnis}{meatus auditorius}}{%
\lemma{ipsa}%
\Bfootnote{%
\textit{(1)}~et meatus auditorius. Seu Auris
\textit{(2)}~(\phantom)\hspace*{-1.2mm}quae eminet \lbrack...\rbrack\ \textso{meatus auditorius.}%
~\textit{L}}}%
\protect\index{Sachverzeichnis}{auris}\textso{ Auris }constat cartilagine,\protect\index{Sachverzeichnis}{cartilago}
quam tegit involucrum nervosum,
quam vestit pellis tenuis et subtilis.
\edtext{Figura}{%
\lemma{Figura}%
\Cfootnote{Siehe a.a.O., Anhang: Tabula I, Fig. I.}}
est quod cartilago\protect\index{Sachverzeichnis}{cartilago} habet%
\edtext{\textso{ plicas }\protect\index{Sachverzeichnis}{plica}%
et\textso{ concham }\protect\index{Sachverzeichnis}{concha}in quam plicae desinunt,}{%
\lemma{\textso{plicas}}%
\Bfootnote{%
\textit{(1)}~quae desinunt in {concham}
\textit{(2)}~et \textso{concham} \lbrack...\rbrack\ plicae desinunt,% in quam
~\textit{L}}}
habet quoque duos musculos,\protect\index{Sachverzeichnis}{musculus}
arterias\protect\index{Sachverzeichnis}{arteria}\lbrack,\rbrack\
venas\protect\index{Sachverzeichnis}{vena}\lbrack,\rbrack\ nervos.\protect\index{Sachverzeichnis}{nervus}}{%
\lemma{Pars \lbrack...\rbrack\ nervos}%
\Cfootnote{\cite{01203}\textsc{Duverney}, \textit{Tractatus de organo auditus}, S.~1.}}%
\pend%
%
%
\pstart%
\edtext{% ((S.~2)) mit Auslassungen §
\textso{Meatus Auditorius }\protect\index{Sachverzeichnis}{meatus auditorius}habet
vestibulum\protect\index{Sachverzeichnis}{vestibulum} seu introitum Concham,
fundum habet tympanum;\protect\index{Sachverzeichnis}{tympanum}
\edtext{inter haec partes duas}{%
\lemma{inter}\Bfootnote{%
\textit{(1)}~duos
\textit{(2)}~haec partes duas%
~\textit{L}}}%
\lbrack,\rbrack\
cartilagineam et osseam.%
\textit{\textso{ Cartilaginea }ex coarctatione conchae\protect\index{Sachverzeichnis}{concha} formatur;}
interrupta est
\edtext{\textit{quasi per incisuras}}{%
\lemma{\textit{quasi}}\Bfootnote{%
\hspace{-0,5mm}\textit{per incisuras}
\textit{erg.~L}}}
pluribus
\edtext{locis \textit{quae non}}{%
\lemma{locis}\Bfootnote{%
\textit{(1)}~nec
\textit{(2)}~\textit{quae non}%
~\textit{L}}}
\textit{cohaerent nisi per pellem
quae partem interiorem meatus\protect\index{Sachverzeichnis}{meatus auditorius} tegit.}}{%
\lemma{\textso{Meatus} \lbrack...\rbrack\ \textit{tegit}}%
\Cfootnote{\cite{01203}a.a.O., S.~2. Zitate mit Auslassungen.}}
\edtext{%
% ((S.~2-3)) mit Auslassungen §
Pellis\protect\index{Sachverzeichnis}{pellis} est continuatio superior,
\textit{sparsa est infinitis}
\edtext{\textit{glandulis\protect\index{Sachverzeichnis}{glandula}}\lbrack,\rbrack\
\textit{quaevis tubulum}\protect\index{Sachverzeichnis}{tubulus}}{%
\lemma{\textit{glandulis}}\Bfootnote{%
\textit{(1)}~quae t
\textit{(2)}~\textit{quaevis tubulum}%
~\textit{L}}}
\textit{habet in cavitate meatus\protect\index{Sachverzeichnis}{meatus auditorius} apertum}
unde prodit cerumen.\protect\index{Sachverzeichnis}{cerumen}
Pars cartilaginea et ossea connectuntur
partim \textit{protuberantiis inaequalibus ad ostium canalis ossei},
\edtext{versus faciem;}{%
\lemma{versus}\Bfootnote{%
\textit{(1)}~facies;
\textit{(2)}~faciem;%
~\textit{L}}}
partim validissimo
\edtext{ligamento
(\phantom)\hspace*{-1.2mm}qua}{%
\lemma{ligamento}\Bfootnote{%
\textit{(1)}~versus
\textit{(2)}~(\phantom)\hspace*{-1.2mm}qua%
~\textit{L}}}
occiput\protect\index{Sachverzeichnis}{occiput} spectant\phantom(\hspace*{-1.2mm})
\textit{quod ossi temporis est annexum.}% }{%
% \lemma{Pellis \lbrack...\rbrack\ \textit{annexum}}%
% \Cfootnote{\cite{01203}a.a.O., S.~2f. Zitat mit Auslassungen.}}
% \edtext{%
% ((S.~3)) OHNE Auslassungen §
\textso{ Ossea pars }meatus \textit{quasi annexa videtur ossi temporum.}
Meatus\protect\index{Sachverzeichnis}{meatus auditorius} totus initio quidem ascendit,
versus mediam vero partem flectitur et redescendit usque ad tympanum.\protect\index{Sachverzeichnis}{tympanum}}{%
\lemma{Pellis % \textso{Ossea} 
\lbrack...\rbrack\ tympanum}%
\Cfootnote{\cite{01203}a.a.O., S.~2f. Zitate mit Auslassungen.}}
\pend%
%
%
\pstart%
\edtext{% ((3)) mit Auslassungen §
\textit{Pars externa} seu aperta,
\textit{%
\edtext{ab interna}{%
\lemma{ab}\Bfootnote{%
\textit{(1)}~externa
\textit{(2)}~\textit{interna}%
~\textit{L}}}}
seu tecta \textit{separatur per\textso{ membranam tympani }\protect\index{Sachverzeichnis}{membrana tympani}}%
quae est in fundo meatus.\protect\index{Sachverzeichnis}{meatus auditorius}
Haec \textit{membrana prope rotunda, sicca, tenuis}\lbrack,\rbrack\ \textit{firma}\lbrack,\rbrack\
\textit{pellucida, infixa striae quae in extrema meatus}
\edtext{\lbrack\textit{ossei}\rbrack}{%
\lemma{ossi}\Bfootnote{%
\textit{L~ändert Hrsg. nach Vorlage}}}
{circumferentia excavata est.}
Extensa est membrana,\protect\index{Sachverzeichnis}{membrana tympani} introrsum tamen gibba\lbrack,\rbrack\
\textit{quippe a manubrio mallei\protect\index{Sachverzeichnis}{manubrium mallei} eo protracta.}}{%
\lemma{\textit{Pars} \lbrack...\rbrack\ \textit{protracta}}%
\Cfootnote{\cite{01203}a.a.O., S.~3. Zitat mit Auslassungen.}}
\pend%
%
%
\pstart%
\edtext{% ((3-4))
\textso{Interna }pars Organi\protect\index{Sachverzeichnis}{organon auditus} sequitur,
huius initium est\textso{ Tympanum,}\protect\index{Sachverzeichnis}{tympanum}
cavitas\protect\index{Sachverzeichnis}{cavitas auris} scil. post membranam,
introrsum clauditur membrana,\protect\index{Sachverzeichnis}{membrana tympani}
\textit{%
\edtext{retrorsum ossis petrosi\protect\index{Sachverzeichnis}{os petrosum} superficie}{%
\lemma{retrorsum}%
\Bfootnote{%
\textit{(1)}~osse petroso
\textit{(2)}~\textit{ossis petrosi superficie}:%
~\textit{L}}}%
}:
\textit{ab utroque
\edtext{latere meatibus}{%
\lemma{latere}\Bfootnote{%
\textit{(1)}~duobus
\textit{(2)}~\textit{meatibus}%
~\textit{L}}}
clauditur quorum anterior}%
\lbrack,\rbrack\
\textit{qui et\textso{ aquaeductus }\protect\index{Sachverzeichnis}{aquaeductus}dicitur}%
\lbrack,\rbrack\
\textit{in palato\protect\index{Sachverzeichnis}{palatum} ostium habet,
alter qui e regione est,
et in superiore cavitatis\protect\index{Sachverzeichnis}{cavitas auris} parte}%
\lbrack,\rbrack\
\textit{in inflexis apophyseos mastoeidis\protect\index{Sachverzeichnis}{apophysis mastoeides} sinubus.
In summo tympani\textso{ loculus }est,
in quo capita ossiculorum,\protect\index{Sachverzeichnis}{ossicula auris}
de quibus postea dicemus}\lbrack,\rbrack\
\textit{conduntur.\textso{ %
Cavitas }tympani\protect\index{Sachverzeichnis}{tympanum} admodum inaequalis est, scabra,
et inducta membrana},\protect\index{Sachverzeichnis}{membrana tympani}
in qua filamenta\protect\index{Sachverzeichnis}{filamentum} vasorum.\protect\index{Sachverzeichnis}{vas}
\textit{In\textso{ corpore }tympani}\protect\index{Sachverzeichnis}{tympanum} sunt
\textit{duo\textso{ meatus,}\protect\index{Sachverzeichnis}{fenestra rotunda}
duae\textso{ fenestrae,}\protect\index{Sachverzeichnis}{fenestra ovalis}
quatuor\textso{ ossicula,}\protect\index{Sachverzeichnis}{ossicula auris}
tres\protect\index{Sachverzeichnis}{musculus}\textso{ musculi;
ramus nervi.}\protect\index{Sachverzeichnis}{nervus}}}{%
\lemma{\hspace{0.2mm}\textso{Interna}\hspace{0.2mm} \lbrack...\rbrack\ \hspace{0.2mm}\textit{\textso{nervi}}\hspace{0.2mm}}%
\Cfootnote{\hspace{0.2mm}\cite{01203}a.a.O.,\hspace{0.2mm} S.~3f.\hspace{0.2mm} Zitate mit Auslassungen.\hspace{0.2mm}}}
\pend%
%%%%%%%%%%%%%%%
%
% \newpage
\count\Bfootins=1100
\count\Afootins=1000
\count\Cfootins=1100
\pstart%
\edtext{% ((4))
\textso{Aquaeductus }\protect\index{Sachverzeichnis}{aquaeductus}ita inseritur,
\edtext{ut \lbrack \protect\index{Sachverzeichnis}{aer}aerem\rbrack\
\textit{qui per nares\protect\index{Sachverzeichnis}{naris} in os intrat}
in transitu excipiat.}{%
\lemma{ut}\Bfootnote{%
\textit{(1)}~excipiens
\textit{(2)}~\textbar~aer \textit{streicht Hrsg.}~\textbar\
\textit{(3)}~\textbar~aererem \textit{ändert Hrsg.}~\textbar\
\textit{qui per} \lbrack...\rbrack\ transitu excipiat.%
~\textit{L}}}%
}{\lemma{\hspace{0.2mm}\textso{Aquaeductus} \hspace{0.2mm} \lbrack...\rbrack\ \hspace{0.2mm} excipiat}%
\Cfootnote{\cite{01203}\hspace{0.2mm}a.a.O., \hspace{0.2mm}S.~4.}}%
%%%%%%%%%%%
\textso{ }%
\edtext{% ((5))
\textso{Fenestrae}\protect\index{Sachverzeichnis}{fenestra ovalis}\protect\index{Sachverzeichnis}{fenestra rotunda}%
\edtext{\textso{ }%
vel foramina\protect\index{Sachverzeichnis}{foramen ovale}\protect\index{Sachverzeichnis}{foramen rotundum}
\textit{sunt}}{%
\lemma{\textso{Fenestrae}}\Bfootnote{%
\textit{(1)}~\textit{sunt}
\textit{(2)}~vel foramina \textit{sunt}%
~\textit{L}}}
\textit{in superficie ossis petrosi,\protect\index{Sachverzeichnis}{os petrosum}
os petrosum} iis \textit{pervium habet lineae\protect\index{Sachverzeichnis}{linea} crassitiem}
(\phantom)\hspace*{-1.2mm}%
$\nicefrac{1}{12}$ pollicis\protect\index{Sachverzeichnis}{pollex}%
\phantom(\hspace*{-1.2mm}).%
\textso{ Ovalis }fenestra\protect\index{Sachverzeichnis}{fenestra ovalis} paulo altior est,
fundus ejus \textit{in oram quasi foliatam\protect\index{Sachverzeichnis}{ora foliata} retorquetur,
cui basis ossiculi
quod stapes\protect\index{Sachverzeichnis}{stapes} dicitur
innititur.
Fenestra\textso{ rotunda }}\protect\index{Sachverzeichnis}{fenestra rotunda}%
(\phantom)\hspace*{-1.2mm}%
quae tamen revera etiam ovalis%
\phantom(\hspace*{-1.2mm})
\textit{incisam in medio meatu striam habet,
in quam\textso{ membrana }\protect\index{Sachverzeichnis}{membrana foraminis rotundi}tenuis, sicca, pellucida,
pene membranam tympani\protect\index{Sachverzeichnis}{membrana tympani} referens}\lbrack,\rbrack\
\textit{infixa est.}
Sequuntur%
\textso{ }\protect\index{Sachverzeichnis}{ossicula auris}%
\edtext{\textso{ossicula. Mallei\protect\index{Sachverzeichnis}{malleus}}}{%
\lemma{\textso{ossicula.}}%
\Bfootnote{%
\textit{(1)}~\textso{Malleus,} ejus
\textit{(2)}~\textso{Mallei}%
~\textit{L}}}%
\textso{ }%
caput seu pars crassior
\edtext{nidulatur \lbrack in\rbrack\ loculo}{%
\lemma{nidulatur}\Bfootnote{%
\hspace{-0,5mm}\textbar~seu \textit{ändert Hrsg. nach Vorlage}~%
\textbar\ loculo%
~\textit{L}}}
supradicto in summo tympani;\protect\index{Sachverzeichnis}{tympanum}
\textit{pars lateralis et paulo posterior capitis huius duas habet protuberantias et cavitatem,
ut articulatione jungi possit ossiculo\protect\index{Sachverzeichnis}{ossicula auris} alteri,
quod incus\protect\index{Sachverzeichnis}{incus} dicitur,
pars gracilior in longitudinem magis protensa} seu
\textit{manubrium\protect\index{Sachverzeichnis}{manubrium mallei} duabus apophysibus augetur
quarum major exterius sita, membranae tympani\protect\index{Sachverzeichnis}{membrana tympani}
\edtext{agglutinatur, altera}{%
\lemma{agglutinatur,}\Bfootnote{%
\textit{(1)}~minor
\textit{(2)}~\textit{altera}%
~\textit{L}}}
ad latus sita versus aquaeductum\protect\index{Sachverzeichnis}{aquaeductus} est gracilior
et musculi\protect\index{Sachverzeichnis}{musculus} alterius tendinem\protect\index{Sachverzeichnis}{tendo} recipit.
Manubrium\protect\index{Sachverzeichnis}{manubrium mallei} applicatur
et agglutinatur ad membranam tympani,\protect\index{Sachverzeichnis}{membrana tympani}
sed ubi extrema sui parte planius fit, firmius annectitur.}}{%
\lemma{\textso{Fenestrae} \lbrack...\rbrack\ \textit{annectitur}}%
\Cfootnote{\cite{01203}a.a.O., S.~5. Zitate mit Auslassungen.}}%
%
\edtext{% ((6)) mit Auslassungen §
\textso{ Incus }\protect\index{Sachverzeichnis}{incus}habet partem solidam et duo crura.
\textit{Solida anterius duas habet cavitates et unam protuberantiam,
ut duabus protuberantiis et cavitati uni capitis mallei\protect\index{Sachverzeichnis}{malleus} respondeat,
et ei jungitur illa articulationis specie,
\edtext{quae ginglymus}{%
\lemma{quae}\Bfootnote{%
\textit{(1)}~gynglymus
\textit{(2)}~\textit{ginglymus}%
~\textit{L}}}
et opificibus cardo dicitur.}
Haec pars solida pene tota in loculo proprie dicto superioris tympani\protect\index{Sachverzeichnis}{tympanum} partis absconditur.
Duo quasi crura sive rami seu apophyses sunt,
\textit{brevior ad ostium meatus situs est} et 
\textit{in apophysim mastoeidem\protect\index{Sachverzeichnis}{apophysis mastoeides} tendit,
alter longior perpendiculariter in tympanum descendit
intusque ad partem tympani membranae\protect\index{Sachverzeichnis}{membrana tympani} oppositam recurvata,
rostellum efformat,
quod cum stapede jungitur per quartum ossiculum.\protect\index{Sachverzeichnis}{ossicula auris}%
\textso{ Stapes }\protect\index{Sachverzeichnis}{stapes}%
\edtext{exacte refert stapedem}{%
\lemma{exacte}\Bfootnote{%
\textit{(1)}~stape
\textit{(2)}~\textit{refert}
\textit{(a)}~illam partem st
\textit{(b)}~\textit{stapedem}%
~\textit{L}}}}
equitis,
\textit{duo habens crura imposita basi planis et ovalis figurae,
superius autem nodulum habet seu caput illi stapedis parti respondens
cui lorum innectitur.
\edtext{Situs stapedis}{%
\lemma{Situs}\Bfootnote{%
\textit{(1)}~capitis
\textit{(2)}~\textit{stapedis}%
~\textit{L}}}
talis est ut si directe caput aspiciatur,
basis quasi tota eo tegatur.}
\textit{Pars tota interior crurum et basis stapedis\protect\index{Sachverzeichnis}{stapes} declivis et excavata est,
ossiculum\protect\index{Sachverzeichnis}{ossicula auris} hoc in cavitatem suam pene horizontaliter positum est.
Duo crura et basis quasi septum efficiunt,
cujus imae parti agglutinata} quadam \textit{membrana\protect\index{Sachverzeichnis}{membrana foraminis ovalis}
tenuis et vasculis sparsa.
Basis stapedis in fenestram ovalem\protect\index{Sachverzeichnis}{fenestra ovalis} infixa
quam exacte claudit orae foliatae}\protect\index{Sachverzeichnis}{ora foliata} ejus \textit{supradictae
membrana innata\protect\index{Sachverzeichnis}{membrana foraminis ovalis} adhaerens firmiter.}%
\textso{ Quartum ossiculum }\protect\index{Sachverzeichnis}{ossicula auris}est exiguum, \textit{convexum} 
\edtext{\lbrack\textit{ea parte}\rbrack}{%
\lemma{\textit{ea}}\Bfootnote{%
\hspace{-0,5mm}\textit{parte}
\textit{erg. Hrsg. nach Vorlage}%
~\textit{L}}}
\textit{qua caput stapedis\protect\index{Sachverzeichnis}{stapes}}
\edtext{(\phantom)\hspace*{-1.2mm}%
nonnihil concavum%
\phantom(\hspace*{-1.2mm})}{%
\lemma{(\phantom)\hspace*{-1.2mm}nonnihil}\Bfootnote{%
\textit{(1)}~in spatium
\textit{(2)}~concavum\phantom(\hspace*{-1.2mm})%
~\textit{L}}}
\textit{respicit}\lbrack,\rbrack\
\textit{concavum qua jungitur rostro incudis.\protect\index{Sachverzeichnis}{incus}
Ossicula\protect\index{Sachverzeichnis}{ossicula auris} haec} non habent periostium,\protect\index{Sachverzeichnis}{periostium}
nec \textit{ad articulationes eorum cartilago\protect\index{Sachverzeichnis}{cartilago} adnata est,
sed ligamentis solum ex eorundem extremitatibus prodeuntibus fortiter constringuuntur.}}{%
\lemma{% \textso{} 
Incus \lbrack...\rbrack\ \textit{constringuuntur}}%
\Cfootnote{\cite{01203}a.a.O., S.~6. Zitate m. Auslassungen.\hspace{-2mm}}}
%
\edtext{% ((6-7)) mit Auslassungen §
\textit{Malleus\protect\index{Sachverzeichnis}{malleus} et incus\protect\index{Sachverzeichnis}{incus} solidissima,
tantum quibusdam vasculis} nutritioni\protect\index{Sachverzeichnis}{nutritio} necessariis \textit{pervia,
contra stapes\protect\index{Sachverzeichnis}{stapes} ex substantia\protect\index{Sachverzeichnis}{substantia} levissima et porosa.
Ex\textso{ tribus musculis }\protect\index{Sachverzeichnis}{musculus}duo pertinent ad malleum};
primus \textit{in apophysin % apo\pgrk{f}ysin 
gracilem mallei\protect\index{Sachverzeichnis}{apophysis mallei} inseritur},
secundus \textit{situs est in dimidio canali osseo in os petrosum\protect\index{Sachverzeichnis}{os petrosum} excavato},
et tendine\protect\index{Sachverzeichnis}{tendo} suo super partem ossis petrosi
\textit{velut super trochleam\protect\index{Sachverzeichnis}{trochlea}
transiens in posticam manubrii\protect\index{Sachverzeichnis}{manubrium mallei} partem
\edtext{paulo infra}{%
\lemma{paulo}\Bfootnote{%
\textit{(1)}~supra
\textit{(2)}~\textit{infra}%
~\textit{L}}}
eum locum ubi musculus\protect\index{Sachverzeichnis}{musculus}}
\edtext{\lbrack\textit{externus}\rbrack}{%
\lemma{internus}%
\Bfootnote{\textit{L~ändert Hrsg nach Vorlage.}}}
\textit{inseritur, intrat,
ut manubrium\protect\index{Sachverzeichnis}{manubrium mallei}
versus os petrosum\protect\index{Sachverzeichnis}{os petrosum} attrahere possit.}
Involucro nervoso quasi vagina dimidio canali firmiter annectitur.
\textit{Musculus stapedis\protect\index{Sachverzeichnis}{stapes} inclusus in tubum osseum,
qui in os petrosum\protect\index{Sachverzeichnis}{os petrosum} pene in imo tympani excavatus est.
Venter grandis et carnosus subito in tendinem\protect\index{Sachverzeichnis}{tendo} exilem definit,
qui capiti stapedis\protect\index{Sachverzeichnis}{stapes} inseritur.}%
\textso{ Ramus nervi }\protect\index{Sachverzeichnis}{nervus}est ramus quinti paris.}{%
\lemma{\textit{Malleus} \lbrack...\rbrack\ paris}%
\Cfootnote{\cite{01203}a.a.O., S.~6f. Zitate m. Auslassungen.}}%
\pend
\count\Bfootins=1200
\count\Afootins=1000
\count\Cfootins=1200%
%
%
\pstart%
\edtext{% ((8)) mit Auslassungen
\textit{Duae fenestrae\protect\index{Sachverzeichnis}{fenestra ovalis}\protect\index{Sachverzeichnis}{fenestra rotunda} spectant}%
\textso{\textit{ cavitatem} auris}\protect\index{Sachverzeichnis}{cavitas auris}\textso{ internam }%
\textit{elaboratam in osse petroso,\protect\index{Sachverzeichnis}{os petrosum}
quae dicitur}\textso{ \textit{Labyrinthus},}
\protect\index{Sachverzeichnis}{labyrinthus ossis petrosi}qui constat vestibulo,\protect\index{Sachverzeichnis}{vestibulum}
\textit{tribus canalibus\protect\index{Sachverzeichnis}{canales semicirculares}
\edtext{rotundis in semicirculum}{%
\lemma{rotundis}\Bfootnote{%
\hspace{-0,5mm}\textbar~qui sunt ad latus vesti \textit{erg.~u. gestr.}~%
\textbar\ \textit{in semicirculum}%
~\textit{L}}}
inflexis}, et cochlea.\protect\index{Sachverzeichnis}{cochlea}
\edtext{Canales\protect\index{Sachverzeichnis}{canales semicirculares}
sunt in latere vestibuli versus occiput,\protect\index{Sachverzeichnis}{occiput}
cochlea in oppositum versus faciem.}{%
\lemma{Canales}\Bfootnote{% \hspace{-0,5mm}
sunt \lbrack...\rbrack\ versus faciem. \textit{erg.~L}}}%
\textso{ }\!%
\textit{\textso{Vestibulum}\protect\index{Sachverzeichnis}{vestibulum}%
\textso{ }%
est cavitas pene rotunda
in osse petroso\protect\index{Sachverzeichnis}{os petrosum}}\lbrack,\rbrack\
diametri lin.\protect\index{Sachverzeichnis}{linea} $1\nicefrac{1}{2}$\lbrack,\rbrack\
\textit{situm pene fenestram ovalem\protect\index{Sachverzeichnis}{fenestra ovalis}},
(\phantom)\hspace*{-1.2mm}%
\textit{per quam e tympano\protect\index{Sachverzeichnis}{tympanum} in}
\edtext{\lbrack\textit{vestibulum}\rbrack}{%
\lemma{vestigium}%
\Bfootnote{\textit{L~ändert Hrsg. nach Vorlage}}}
\textit{via patet}%
\phantom(\hspace*{-1.2mm})
\textit{vestitum membrana\protect\index{Sachverzeichnis}{membrana foraminis ovalis} multis vasculis plena.}
Novem in eo ostia, nempe foramen ovale,\protect\index{Sachverzeichnis}{foramen ovale}
\textit{reliqua in cavitate vestibuli,\protect\index{Sachverzeichnis}{vestibulum}
primum in superiorem cochleae\protect\index{Sachverzeichnis}{cochlea}
\edtext{spiram, quinque}{%
\lemma{spiram,}\Bfootnote{%
\textit{(1)}~reliqua
\textit{(2)}~quinque%
~\textit{L}}}
in tres canales semicirculares\protect\index{Sachverzeichnis}{canales semicirculares}}
\edtext{\lbrack\textit{ducunt}\rbrack,}{%
\lemma{\textit{ducunt}}\Bfootnote{%
\textit{erg. Hrsg. nach Vorlage}}}
\textit{duo transitum} dant \textit{gemino ramo portionis mollis nervi auditorii.\protect\index{Sachverzeichnis}{nervus auditorius}}
Ex\textso{ canalibus semicircularibus }%
\edtext{(\phantom)\hspace*{-1.2mm}%
qui sunt ad latus vestibuli\protect\index{Sachverzeichnis}{vestibulum} versus occiput\protect\index{Sachverzeichnis}{occiput}%
\phantom(\hspace*{-1.2mm})}{%
\lemma{(\phantom)\hspace*{-1.2mm}qui}\Bfootnote{%
\hspace{-0,5mm}sunt \lbrack...\rbrack\ versus occiput\phantom(\hspace*{-1.2mm})
\textit{erg.~L}}}
\textit{primum}
\edtext{vocabo \textit{superiorem,}}{%
\lemma{vocabo}\Bfootnote{%
\textit{(1)}~inferior
\textit{(2)}~\textit{superiorem,}%
~\textit{L}}}
\textit{quod laquear vestibuli arcuatum circumdet,
secundum inferiorem, quod imas eiusdem partes cingat,
tertium} medium, quod \textit{longius prodit et inter} eosdem \textit{situs est.}%
\textit{\textso{ Superior }e vestibulo tendit} primum \textit{retrorsum,
mox} paulum \textit{incurvatus antrorsum ad medium
usque posticae ossis petrosi\protect\index{Sachverzeichnis}{os petrosum} partis,
et ubi paulo plus quam dimidium circulum confecit, inferiori} jungitur.%
\textit{\textso{ Inferior }ab ima vestibuli parte prodit,
et confectus itidem paulo}
\edtext{\lbrack\textit{majori}\rbrack}{%
\lemma{major}%
\Bfootnote{\textit{L~ändert Hrsg. nach Vorlage}}}
\textit{quam dimidii circuli}%
\lbrack,\rbrack\
\textit{spatio superiori jungitur.
Juncti in unum plane coalescunt qui oblique protenditur
donec in medio vestibulo\protect\index{Sachverzeichnis}{vestibulum} ostium conficiat.}
Medius \textit{duas habet portas separatas nec plus quam semicirculum itinere suo describit.
Canales\protect\index{Sachverzeichnis}{canales semicirculares} hi aliquando rotundi, aliquando ovales interius,
versus ostia autem in tubae formam expanduntur.}}{%
\lemma{\textit{Duae} \lbrack...\rbrack\ \textit{expanduntur}}%
\Cfootnote{\cite{01203}a.a.O., S.~8. Zitate mit Auslassungen.}}
%
\edtext{% /((8-9)) OHNE Auslassungen
Isti tres canales\protect\index{Sachverzeichnis}{canales semicirculares} in vestibulo pro sex ostiis habent quinque,
quia unum duobus commune\lbrack:\rbrack\
%
\lbrack24~v\textsuperscript{o}\rbrack\ % Blatt 24v
%
duo ostia patent in summa, duo in ima,
\edtext{\lbrack unum\rbrack}{%
\lemma{duo}%
\Bfootnote{\textit{L~ändert Hrsg. nach Vorlage}}}
in media vestibuli\protect\index{Sachverzeichnis}{vestibulum} par$\langle$te$\rangle$,
primum et supremum porta est canalis superioris,
secundum
\edtext{altera porta}{%
\lemma{altera}\Bfootnote{%
\hspace{-0,5mm}porta \textit{erg.~L}}}
medii\lbrack,\rbrack\
\textit{duae istae portae prope vestibulum solo osse tenuissimo dividuntur
quod plane evanescit in ipso vestibuli introitu.
Ex duobus ostiis in imo vestibuli infimum canalis inferioris, alterum altera medii porta est.
Ostium in medio vestibuli\protect\index{Sachverzeichnis}{vestibulum} omnium maximum,
superiori et inferiori canali\protect\index{Sachverzeichnis}{canales semicirculares} est commune.}}{%
\lemma{Isti \lbrack...\rbrack\ \textit{commune}}%
\Cfootnote{\cite{01203}a.a.O., S.~8f.}}%
%
\edtext{% ((9)) mit Auslassungen §
\textso{ Cochlea }\protect\index{Sachverzeichnis}{cochlea}est tertia pars labyrinthi,\protect\index{Sachverzeichnis}{labyrinthus ossis petrosi}
\textit{ad\textso{ latus vestibuli tribus canalibus semicircularibus oppositum,}%
\protect\index{Sachverzeichnis}{vestibulum}\protect\index{Sachverzeichnis}{canales semicirculares}
versus faciem}.
Constat duabus partibus,
\textit{canali semiovali spirali et lamina\protect\index{Sachverzeichnis}{lamina cochleae}
quae in spiram ascendentem convolvitur,
canalemque sua via sequitur et in duas partes dividit}
\lbrack(\phantom)\hspace*{-1.2mm}\rbrack%
lamina scilicet ista sua spira canalem facit spiralem%
\phantom(\hspace*{-1.2mm}).
\textit{Canalis}\protect\index{Sachverzeichnis}{canalis cochleae}
facit \textit{duos circulos et dimidium circa centrum}
(+\phantom)\hspace*{-1.2mm}~%
seu potius axem~%
\phantom(\hspace*{-1.2mm}+)
\textit{minorque fit et
\edtext{angustior quanto}{%
\lemma{angustior}\Bfootnote{%
\textit{(1)}~(+\phantom)\hspace*{-1.2mm}~circulus
\textit{(2)}~\textit{quanto}%
~\textit{L}}}
magis elevatur}
extremitatesque centro
(+\phantom)\hspace*{-1.2mm}~%
axi~%
\phantom(\hspace*{-1.2mm}+)
appropinquant, ubi tam tenues fiunt quam ipsa lamina;}{%
\lemma{\textso{Cochlea} \lbrack...\rbrack\ lamina}%
\Cfootnote{\cite{01203}a.a.O., S.~9. Zitate mit Auslassungen.}}%
%
\edtext{% ((9-10)) mit Auslassungen §
\textso{ lamina }basi sua adhaeret axi, extremo superficiei
(+\phantom)\hspace*{-1.2mm}~%
seu continaculo~%
\phantom(\hspace*{-1.2mm}+)
spirae, ubi annectitur ossi petroso\protect\index{Sachverzeichnis}{os petrosum}
\textit{membrana tenui multo graciliori
quam} est \textit{lamina.\protect\index{Sachverzeichnis}{lamina cochleae}
Membrana haec evoluta totam canalis\protect\index{Sachverzeichnis}{canalis cochleae}}
\edtext{\textit{superficiem} \lbrack\textit{vestit.}\rbrack\
\textit{Lamina}\protect\index{Sachverzeichnis}{lamina cochleae}}{%
\lemma{\textit{superficiem}}\Bfootnote{%
\hspace{-0,5mm}\textbar~(\phantom)\hspace*{-1.2mm}versus os petrosum\phantom(\hspace*{-1.2mm})
\textit{vestit.} \textit{gestr.}~\textbar\
\textit{vestit.} \textit{erg.~Hrsg. nach Vorlage}~\textbar\
\textit{Lamina}%
~\textit{L}}}
\textit{est dura, friabilis,
prope basin perinde ut ipsa basis multis foraminibus pervia,
altera extremitas tenuissima, firma, intensa.}
Lamina\protect\index{Sachverzeichnis}{lamina cochleae}
ergo \textit{duos}
% \edtext{ergo \textit{duos}}{%
% \lemma{ergo}\Bfootnote{%
% \textit{(1)}~du\textlangle\textendash\textrangle\
% \textit{(2)}~\textit{duos}%
% ~\textit{L}}}
\textit{quasi ascensus scalae cochleatae\protect\index{Sachverzeichnis}{scalae cochleatae}}
facit non communicantes,
\textit{duo saltem ostia habent separata,
quorum alterum viam praebet e vestibulo\protect\index{Sachverzeichnis}{vestibulum} in scalam superiorem,
alterum quod est\textso{ ipsissima fenestra }rotunda,\protect\index{Sachverzeichnis}{fenestra rotunda}
e tympano\protect\index{Sachverzeichnis}{tympanum} immediate
in scalam inferiorem\protect\index{Sachverzeichnis}{scalae cochleatae}
deducit.}}{%
\lemma{\textso{lamina} \lbrack...\rbrack\ \textit{deducit}}%
\Cfootnote{\cite{01203}a.a.O., S.~9f. Zitate mit Auslassungen.}}
%
\edtext{% ((10)) mit Auslassungen §
\textit{Ostium est in inferiori parte ossis petrosi,\protect\index{Sachverzeichnis}{os petrosum}
infra illud per quod nervus auditorius\protect\index{Sachverzeichnis}{nervus auditorius} transit,
quod pervium est arteriae\protect\index{Sachverzeichnis}{arteria} et venae\protect\index{Sachverzeichnis}{vena}
quae rami sunt carotidis\protect\index{Sachverzeichnis}{carotides arteria}
et jugularis\protect\index{Sachverzeichnis}{jugulares vena} internae,
et hoc ipsum ostium initium est canalis}
(\phantom)\hspace*{-1.2mm}%
potius
% \edtext{\lbrack potius\rbrack}{%
% \lemma{protius}\Bfootnote{\textit{L~ändert Hrsg.}}}
vasis\protect\index{Sachverzeichnis}{vas}%
\phantom(\hspace*{-1.2mm})
\textit{qui postquam in longitudinem lineae\protect\index{Sachverzeichnis}{linea} $1\nicefrac{1}{2}$ procurrit
in inferiori cochleae\protect\index{Sachverzeichnis}{cochlea} meatu patet
prope fenestram rotundam.\protect\index{Sachverzeichnis}{fenestra rotunda}
Vasa haec ubi eo pervenere in complures ramusculos sparguntur,
qui in laminam spiralem\protect\index{Sachverzeichnis}{lamina cochleae} et membranam
quae interiora canalis spiralis\protect\index{Sachverzeichnis}{canalis cochleae} vestit distribuuntur.
Arteria\protect\index{Sachverzeichnis}{arteria} illa quae in cochleam\protect\index{Sachverzeichnis}{cochlea} intrat
ramum notabilem communicat vestibulo,
qui} rursus ibi in duos ramusculos dividitur
quorum unus \textit{per portam vestibuli duobus canalibus communem intrat,
alter per portam canalis medii superiorem\protect\index{Sachverzeichnis}{canales semicirculares} intrat
et per alteram ejusdem portam in vestibulum\protect\index{Sachverzeichnis}{vestibulum} revertitur.
Ramusculorum horum in multis vestibuli interioris locis fit anastomosis.\protect\index{Sachverzeichnis}{anastomosis}
Venae\protect\index{Sachverzeichnis}{vena} pari modo distribuuntur.
Cum duae\textso{ fenestrae }\protect\index{Sachverzeichnis}{fenestra ovalis}\protect\index{Sachverzeichnis}{fenestra rotunda}%
quae in cavitate labyrinthi\protect\index{Sachverzeichnis}{labyrinthus ossis petrosi} patent}
arctissime \textit{sint\textso{ clausae,} altera basi stapedis\protect\index{Sachverzeichnis}{stapes}
altera membrana,\protect\index{Sachverzeichnis}{membrana foraminis ovalis}
patet aerem} inclusum\protect\index{Sachverzeichnis}{aer inclusus}
\textit{cum aere tympani,\protect\index{Sachverzeichnis}{aer tympani}}
adeoque cum aere externo\protect\index{Sachverzeichnis}{aer externus} non communicare.
\edlabel{LH_37_01_024v_aerimplantatus_iuuwv-1}Hinc
\edtext{\textit{Anatomici}\protect\index{Sachverzeichnis}{anatomicus}}{%
\lemma{\textit{Anatomici}}\Cfootnote{% Quelle nicht
%Nicht nachgewiesen.???
%Vgl. aber \textsc{Aristoteles}, \textit{De anima} II~8, 420a4\cite{01217} (\protect{\pgrk{sumfu`hs >a'hr}}).
Siehe etwa C.~\textsc{Bauhin}, \textit{Theatrum anatomicum}, l.~III, cap.~61 (Frankfurt a.M. 1605, S.~848\textendash854).\cite{01123}
Der Begriff \textit{aer implantatus} geht vermutlich auf \textsc{Aristoteles}, \textit{De anima} II~8 (420a4\textendash15)\cite{01217} zurück.
}}%
\textit{\textso{ aerem implantatum }\protect\index{Sachverzeichnis}{aer implantatus}%
dixere.}\edlabel{LH_37_01_024v_aerimplantatus_iuuwv-2}}{%
\lemma{\textit{Ostium} \lbrack...\rbrack\ dixere}%
\Cfootnote{\cite{01203}a.a.O., S.~10. Zitate mit Auslassungen.}}%
%
\edtext{% ((10-11)) OHNE Auslassungen §
\textit{\textso{ Meatus per quem nervus auditorius }%
\protect\index{Sachverzeichnis}{meatus}\protect\index{Sachverzeichnis}{nervus auditorius}%
%\edtext{transit amplus admodum est,}{%
%\lemma{transit}\Bfootnote{%
%\hspace{-0,5mm}\textbar~est \textit{streicht Hrsg. \mbox{nach} Vorlage}~\textbar\
%\textit{amplus admodum est,}%
%~\textit{L}}}
\edtext{transit amplus admodum est,}{%
\lemma{transit}\Bfootnote{%
\textit{(1)}~\textbar~est \textit{streicht Hrsg. nach Vorlage}~\textbar\
\textit{(2)}~\textit{amplus admodum est,}%
~\textit{L}}}
excavatus in mediam partem posticam ossis petrosi,\protect\index{Sachverzeichnis}{os petrosum}
qua cerebrum\protect\index{Sachverzeichnis}{cerebrum} respicit,
et oblique retrorsum in duarum circiter linearum\protect\index{Sachverzeichnis}{linea} longitudinem prodiens
saccum quasi efformat,
cujus fundus partim basis est cochleae\protect\index{Sachverzeichnis}{cochlea}
partim portio laquearis vestibuli, in imo istius sacci ossiculum\protect\index{Sachverzeichnis}{ossicula auris} est
quod basin cochleae\protect\index{Sachverzeichnis}{cochlea} a foramine separat
per quod portio dura nervi auditorii\protect\index{Sachverzeichnis}{nervus auditorius} transit.}}{%
\lemma{\textit{\textso{Meatus}} \lbrack...\rbrack\ \textit{transit}}%
\Cfootnote{\cite{01203}\textsc{Duverney}, \textit{Tractatus de organo auditus}, S.~10f.}}
%
\edtext{% ((11)) mit Auslassungen §
}{{\xxref{LH_37_01_024v_S.11-1}{LH_37_01_024v_S.11-2}%
}{\lemma{Nervus \lbrack...\rbrack\ sequitur}% Acusticus ... eam
\Cfootnote{\cite{01203}a.a.O., S.~11. Zitate mit Auslassungen.}}}%
\edlabel{LH_37_01_024v_S.11-1}%
Nervus Acusticus\protect\index{Sachverzeichnis}{nervus acusticus}
oritur a \textit{posteriori parte protuberantiae} annularis,
\textit{lineae\protect\index{Sachverzeichnis}{linea} spatio
a lobulo parvo cerebelli.\protect\index{Sachverzeichnis}{lobus parvus cerebelli}}
\edlabel{LH_37_01_024v_nervusacusticus_egdtr-1}Pars nervi\protect\index{Sachverzeichnis}{nervus}
\edtext{superior et major}{%
\lemma{superior}\Bfootnote{%
\textit{(1)}~et exterior
\textit{(2)}~et major%
~\textit{L}}}
\edtext{mollis, inferior}{%
\lemma{mollis,}\Bfootnote{%
\textit{(1)}~interior
\textit{(2)}~inferior%
~\textit{L}}}
et minor dura.
Mollior est portio mollis
\textit{quam omnes nervi medullae oblongatae\protect\index{Sachverzeichnis}{medulla oblongata}
exceptis olfactoriis.\protect\index{Sachverzeichnis}{nervus olfactorius}
Portio dura prodit extra cranium,}\protect\index{Sachverzeichnis}{cranium}
mollis \textit{desinit in organo auditus.\protect\index{Sachverzeichnis}{organon auditus}
Duo hi rami}
(\phantom)\hspace*{-1.2mm}%
mollis et durus%
\phantom(\hspace*{-1.2mm})
\textit{paralleli incedunt ad foramen usque ossis petrosi\protect\index{Sachverzeichnis}{os petrosum}}
ubi \textit{portio dura supra alteram ascendit.}
Pars mollis uno ramo maxime notabili in cochleam,\protect\index{Sachverzeichnis}{cochlea}
duobus aliis in vestibulum\protect\index{Sachverzeichnis}{vestibulum} tendit,
ex his duobus notabilior in quemlibet
ex canalibus semicircularibus\protect\index{Sachverzeichnis}{canales semicirculares} filamentum mittit,
\textit{quod arteriae\protect\index{Sachverzeichnis}{arteria} ibi distributae}
\edtext{\textit{jungitur et} eam}{%
\lemma{\textit{jungitur}}\Bfootnote{%
\hspace{-0,5mm}\textit{et}
\textbar~et \textit{streicht Hrsg.}~%
\textbar\ eam~%
\textit{L}}}
\edtext{sequitur.\edlabel{LH_37_01_024v_nervusacusticus_egdtr-2} % Seitenende
\edlabel{LH_37_01_024v_S.11-2}%
%
\edtext{% ((12)) mit Auslassungen §
}{{\xxref{LH_37_01_024v_S.12-1}{LH_37_01_024v_S.12-2}%
}{\lemma{Durior \lbrack...\rbrack\ \textit{jungitur}}%
\Cfootnote{\cite{01203}a.a.O., S.~12. Zitate mit Auslassungen.}}}%
\edlabel{LH_37_01_024v_S.12-1}%
Durior ramus}{%
\lemma{sequitur.}\Bfootnote{%
\textit{(1)}~Durus ner
\textit{(2)}~Durior ramus%
~\textit{L}}}
in osse petroso\protect\index{Sachverzeichnis}{os petrosum} excavato procedens
\textit{denique per foramen exit inter apophyses
mastoeiden\protect\index{Sachverzeichnis}{apophysis mastoeides} et styloeiden,\protect\index{Sachverzeichnis}{apophysis styloeides}}
antequam tamen e foramine exeat
alium recipit ex\textso{ quinto pari.}
E foramine egressus ramum emittit versus partes auris\protect\index{Sachverzeichnis}{auris} externas,
et alios ramos versus partes alias.
Nervus\protect\index{Sachverzeichnis}{nervus} a quinto
\edtext{\lbrack pari\rbrack}{%
\lemma{pari}%
\Bfootnote{\textit{erg. Hrsg. nach Vorlage}}}
prope musculos\protect\index{Sachverzeichnis}{musculus} mallei\protect\index{Sachverzeichnis}{malleus}
superque membranam tympani\protect\index{Sachverzeichnis}{membrana tympani} provectus
e tympano\protect\index{Sachverzeichnis}{tympanum}\lbrack,\rbrack\
in os petrosum\protect\index{Sachverzeichnis}{os petrosum} immissus
\textit{ubi trunco portionis durae jungitur.}%
\edlabel{LH_37_01_024v_S.12-2}
%
\edtext{% ((12-13)) OHNE Auslassungen §
Hoc filum
\edtext{nervi\protect\index{Sachverzeichnis}{nervus}}{%
\lemma{nervi}%
\Bfootnote{\textit{erg.~L}}}
\edtext{Anatomici\protect\index{Sachverzeichnis}{anatomicus}
quasi chordam\protect\index{Sachverzeichnis}{chorda}
membranae tympani\protect\index{Sachverzeichnis}{membrana tympani}
credidere,}{%
\lemma{Anatomici \lbrack...\rbrack\ credidere}\Cfootnote{%
Siehe hierüber \textsc{Bauhin}, \textit{Theatrum anatomicum}, l.~III, cap.~51 (S.~826).\cite{01123}}}
sed nervus\protect\index{Sachverzeichnis}{nervus} est,
nec alium habent musculi\protect\index{Sachverzeichnis}{musculus} ossiculorum.\protect\index{Sachverzeichnis}{ossicula auris}}{%
\lemma{Hoc filum \lbrack...\rbrack\ ossiculorum}%
\Cfootnote{\cite{01203}a.a.O., S.~12f.}}
%
\edtext{% ((13)) mit Auslassungen §
Secundum quoque par vertebrale ramum dat auri,
qui \textit{spargitur in posteriora auris}
\edtext{\lbrack\textit{et}\rbrack}{%
\lemma{\textit{et}}\Bfootnote{%
\textit{erg. Hrsg. nach Vorlage}%
~\textit{L}}}
\textit{auriculam}
et meatum\protect\index{Sachverzeichnis}{meatus} cartilagineum.}{%
\lemma{Secundum \lbrack...\rbrack\ cartilagineum}% quoque
\Cfootnote{\cite{01203}\textsc{Duverney}, \textit{Tractatus de organo auditus}, S.~13. Zitat mit Auslassun\-gen.}}%
\pend%
%\newpage
\vspace{1.0em}%
%
%
\pstart%
\noindent\centering%
Breviter \edtext{de usu\protect\index{Sachverzeichnis}{usus} ad audiendum.%
}{\lemma{de usu ad audiendum}\Cfootnote{%
Der zweite Teil von Duverneys Abhandlung handelt vom \glqq Gebrauch der Teile des Organs des Gehörs\grqq. % \textit{usus partium organi auditus}.
Siehe \cite{01203}a.a.O., S.~17.}}
\pend%
% \vspace*{0.5em}%
%
%
\pstart%
\noindent%
\edtext{% ((17)) OHNE Auslassungen §
Externa pars auris\protect\index{Sachverzeichnis}{auris} facit
officium\protect\index{Sachverzeichnis}{officium} corniculi\protect\index{Sachverzeichnis}{corniculum}
quo utuntur surdi.\protect\index{Sachverzeichnis}{surdus}}{%
\lemma{Externa \lbrack...\rbrack\ surdi}%
\Cfootnote{\cite{01203}a.a.O., S.~17.}}
\edtext{% ((19-20)) OHNE Auslassungen §
Tympani membrana\protect\index{Sachverzeichnis}{membrana tympani} non est absolute necessaria,
nam \textit{surdi\protect\index{Sachverzeichnis}{surdus} nonnulli tenentes}
\edtext{\lbrack \textit{dentibus}\rbrack}{%
\lemma{auribus\protect\index{Sachverzeichnis}{auris}}\Bfootnote{\textit{L~ändert Hrsg. nach Vorlage}}}
%
\textit{manubrium instrumenti}\protect\index{Sachverzeichnis}{manubrium instrumenti}
audiunt,
si tamen perforetur tympanum\protect\index{Sachverzeichnis}{tympanum}
auditus\protect\index{Sachverzeichnis}{auditus} in animali\protect\index{Sachverzeichnis}{animal} aliquandiu quidem conservabitur,
tandem tamen peribit
(+\phantom)\hspace*{-1.2mm}~%
cur ita?%
~\phantom(\hspace*{-1.2mm}+).
\textit{Intenditur et remittitur per musculos\protect\index{Sachverzeichnis}{musculus} mallei\protect\index{Sachverzeichnis}{malleus}}
ad diversos sonos\protect\index{Sachverzeichnis}{sonus} corporum sonantium,\protect\index{Sachverzeichnis}{corpus sonans}
quomodo autem id fiat difficile intelligere.}{%
\lemma{Tympani \lbrack...\rbrack\ intelligere.}%
\Cfootnote{\cite{01203}a.a.O., S.~19f. Zitat mit Auslassungen.% Falsch! Nur wegen Bündigkeit!
}}
%
\edtext{% ((20)) OHNE Auslassungen §
Membrana haec cum sit sicca tenuis pellucida
valde est huic usui\protect\index{Sachverzeichnis}{usus} apta.
Trans membranam\protect\index{Sachverzeichnis}{membrana tympani}
aer in tympano\protect\index{Sachverzeichnis}{aer tympani}
motus a membrana contribuet aliquod sed non erit
\edtext{par \lbrack movendis\rbrack\ partibus labyrinthi,\protect\index{Sachverzeichnis}{labyrinthus ossis petrosi}}{%
\lemma{par}\Bfootnote{%
\hspace{-0,5mm}\textbar~movendo \textit{ändert Hrsg.}~\textbar\
\textit{(1)}~ossi petroso
\textit{(2)}~partibus labyrinthi,%
~\textit{L}}}
potius credendum tremores\protect\index{Sachverzeichnis}{tremor} membranae\protect\index{Sachverzeichnis}{membrana tympani}
communicari malleo,\protect\index{Sachverzeichnis}{malleus}
a malleo incudi,\protect\index{Sachverzeichnis}{incus} ab hoc stapedi,\protect\index{Sachverzeichnis}{stapes}
\textit{cuius tandem tremor os petrosum\protect\index{Sachverzeichnis}{os petrosum} commovet},
ut chordae barbiti\protect\index{Sachverzeichnis}{chorda barbiti} debent esse in eadem mensa.\protect\index{Sachverzeichnis}{mensa}%
}{\lemma{Membrana \lbrack...\rbrack\ mensa}%
\Cfootnote{\cite{01203}a.a.O., S.~20.}}
%
\edtext{% ((21)) mit Auslassungen §
Nam ossicula\protect\index{Sachverzeichnis}{ossicula auris} sicca et dura%
\lbrack,\rbrack\
tenuia, ideoque facile mobilia,
\textit{manubrium mallei\protect\index{Sachverzeichnis}{manubrium mallei} tota sua longitudine,
membranae tympani\protect\index{Sachverzeichnis}{membrana tympani} agglutinatum},
junguntur ossicula\protect\index{Sachverzeichnis}{ossicula auris} sine cartilagine,\protect\index{Sachverzeichnis}{cartilago}
unde
\edtext{facilior communicatio}{%
\lemma{facilior}\Bfootnote{%
\textit{(1)}~transitus
\textit{(2)}~communicatio%
~\textit{L}}}
vibrationis;\protect\index{Sachverzeichnis}{communicatio vibrationis}
videtur musculus\protect\index{Sachverzeichnis}{musculus} stapedis\protect\index{Sachverzeichnis}{stapes}
\textit{extrorsum paulum trahere basin stapedis immediate applicatam fenestrae ovali\protect\index{Sachverzeichnis}{fenestra ovalis}}
et \textit{intendere} hoc modo \textit{membranam\protect\index{Sachverzeichnis}{membrana foraminis ovalis} illam parvam
quae superiorem basis hujus partem vestit}
(\phantom)\hspace*{-1.2mm}+~%
quae orae foliatae\protect\index{Sachverzeichnis}{ora foliata} jungitur%
~+\phantom(\hspace*{-1.2mm})
et reddere eam magis aptam
\textit{ad
\edtext{recipiendos membranae tympani\protect\index{Sachverzeichnis}{membrana tympani}}{%
\lemma{recipiendos}\Bfootnote{%
\textit{(1)}~tympani
\textit{(2)}~\textit{membranae tympani}%
~\textit{L}}}
tremores\protect\index{Sachverzeichnis}{tremor}}
ut communicet labyrintho.\protect\index{Sachverzeichnis}{labyrinthus ossis petrosi}
\textit{Dici} etiam \textit{potest trahendo stapedem\protect\index{Sachverzeichnis}{stapes} alioqui satis flexibilem,
tendere} eum et reddere firmiorem \textit{ad melius} exprimendos
\textit{mallei\protect\index{Sachverzeichnis}{malleus} et incudis\protect\index{Sachverzeichnis}{incus} tremores\protect\index{Sachverzeichnis}{tremor}.}
Aer\protect\index{Sachverzeichnis}{aer tympani} per duos meatus\protect\index{Sachverzeichnis}{meatus} laterales
ex tympano\protect\index{Sachverzeichnis}{tympanum} exire potest,
ut cedat membranae\protect\index{Sachverzeichnis}{membrana tympani} vibratione\protect\index{Sachverzeichnis}{vibratio tympani} extrorsum pulsae.
Tubus\protect\index{Sachverzeichnis}{tubus} ad palatum\protect\index{Sachverzeichnis}{palatum}\lbrack,\rbrack\
novum cum opus aerem\protect\index{Sachverzeichnis}{aer}
suppeditat ex naribus.\protect\index{Sachverzeichnis}{naris}%
}{\lemma{Nam \lbrack...\rbrack\ naribus}%
\Cfootnote{\cite{01203}a.a.O., S.~21. Zitate m. Aus\-las\-sun\-gen.}}
%
\edtext{% ((22)) mit Auslassungen §
Surdi\protect\index{Sachverzeichnis}{surdus} non audient
nisi manubrium\protect\index{Sachverzeichnis}{manubrium instrumenti} dentibus\protect\index{Sachverzeichnis}{dens} applicent,
unde \textit{tremor\protect\index{Sachverzeichnis}{tremor} communicatur
ossi mandibulae,\protect\index{Sachverzeichnis}{mandibula}
ossibus temporum\protect\index{Sachverzeichnis}{tempora} et ossiculis\protect\index{Sachverzeichnis}{ossicula auris}},
imo qui non surdi
\textit{melius sonum instrumenti\protect\index{Sachverzeichnis}{sonus instrumenti}
percipiunt dentibus\protect\index{Sachverzeichnis}{dens}},
licet aures\protect\index{Sachverzeichnis}{auris} obturent.
Membrana foraminis\protect\index{Sachverzeichnis}{membrana foraminis rotundi}
\edtext{rotundi\lbrack,\rbrack\ similis}{%
\lemma{rotundi}\Bfootnote{%
\textit{(1)}~recipit
\textit{(2)}~similis%
~\textit{L}}}
ei quae in tympano\protect\index{Sachverzeichnis}{membrana tympani} est\lbrack,\rbrack\
etiam tremores\protect\index{Sachverzeichnis}{tremor} aeris tympani\protect\index{Sachverzeichnis}{aer tympani}
communicabit aeri
\edtext{}{%
{\xxref{LH_37_01_024v_lamina_asiejbnof-1}{LH_37_01_024v_lamina_asiejbnof-2}}%
{\lemma{implantato}\Bfootnote{%
\textit{(1)}~lamina
\textit{(a)}~in lab
\textit{(b)}~\textbar~qui organo \textit{\mbox{streicht} Hrsg.}~\textbar\
\textit{(2)}~seu labyrintho \lbrack...\rbrack\ cochleam lamina% in quo est. Quoad
~\textit{L}}}}%
\edlabel{LH_37_01_024v_lamina_asiejbnof-1}implantato\protect\index{Sachverzeichnis}{aer implantatus}
seu labyrintho\protect\index{Sachverzeichnis}{labyrinthus ossis petrosi} in quo est.%
}{\lemma{Surdi \lbrack...\rbrack\ est}%
\Cfootnote{\cite{01203}a.a.O., S.~22. Zitate m. Auslassungen.\hspace{-1mm}}}
%
\edtext{% ((23-24)) mit Aauslassungen §
Quoad cochleam\protect\index{Sachverzeichnis}{cochlea}
lamina\edlabel{LH_37_01_024v_lamina_asiejbnof-2} spiralis\protect\index{Sachverzeichnis}{lamina cochleae} dura, sicca, tenuis, ergo tremula,
\textit{non jacet in canali semiovali,\protect\index{Sachverzeichnis}{canales semicirculares}
sed tensa est mediante pellicula subtili} qua \textit{superficiei canalis adhaeret}.
Dividit cochleam in duas quasi scalas cochleatas,\protect\index{Sachverzeichnis}{scalae cochleatae}
inter se non communicantes.
Fenestra rotunda\protect\index{Sachverzeichnis}{fenestra rotunda} prospicit in inferiorem,
cujus nulla communicatio nec cum superiore, nec cum vestibulo,
sed ab aere
\edtext{tympani \lbrack recipit\rbrack\ vibrationes\protect\index{Sachverzeichnis}{vibratio aeris}}{%
\lemma{tympani}\Bfootnote{%
\hspace{-0,5mm}\textbar~recepit \textit{ändert Hrsg.}~%
\textbar\ vibrationes%
~\textit{L}}}
per foraminis rotundi membranam,\protect\index{Sachverzeichnis}{membrana foraminis rotundi}
at aer\protect\index{Sachverzeichnis}{aer}
\edtext{superioris scalae\protect\index{Sachverzeichnis}{scalae cochleatae}}{%
\lemma{superioris}\Bfootnote{%
\textit{(1)}~canalis
\textit{(2)}~scalae%
~\textit{L}}}
recipit tremores\protect\index{Sachverzeichnis}{tremor} ab aere vestibuli.\protect\index{Sachverzeichnis}{vestibulum}
Lamina\protect\index{Sachverzeichnis}{lamina cochleae} pulsatur inter utrumque.
Cum lamina\protect\index{Sachverzeichnis}{lamina cochleae} faciat circulos $2\,\nicefrac{1}{2},$
eo pluribus partibus tremores\protect\index{Sachverzeichnis}{tremor}
\edtext{recipit. Initio amplior est}{%
\lemma{recipit}\Bfootnote{%
\textit{(1)}~ubi arctior est
\textit{(2)}~. Initio
\textit{(a)}~arctior
\textit{(b)}~amplior est%
~\textit{L}}}%
\lbrack,\rbrack\
in fine arctior,
ut scilicet partes aliae tremere possint aliis immotis%
}{\lemma{Quoad \lbrack...\rbrack\ immotis}%
\Cfootnote{\cite{01203}a.a.O., S.~23f. Zitat m. Auslassungen.}}
(\phantom)\hspace*{-1.2mm}+~%
ut in \edtext{anisocyclis\protect\index{Sachverzeichnis}{anisocyclus} Vitruvii.}{%
\lemma{anisocyclis Vitruvii}\Cfootnote{%
Siehe \textsc{Vitruv}, \textit{De architectura} X~1,~3.\cite{01225}}}~%
\lbrack+\phantom(\hspace*{-1.2mm})\rbrack\
%
\edtext{% ((23)) mit Auslassungen §
\edlabel{LH_37_01_024v_nervusacusticus_egdtr-3}%
\textit{Ramus notabilis portionis mollis nervi auditorii\protect\index{Sachverzeichnis}{nervus auditorius}
ubi ad basin cochleae\protect\index{Sachverzeichnis}{cochlea} pervenit}
pluribus filamentis\protect\index{Sachverzeichnis}{filamentum}
\edtext{basin perforat,}{%
\lemma{basin}\Bfootnote{%
\textit{(1)}~perforans,
\textit{(2)}~perforat,%
~\textit{L}}}
\edtext{\lbrack \textit{quae}\rbrack}{%
\lemma{qui}\Bfootnote{\textit{L~ändert Hrsg. nach Vorlage}}}
\textit{in gyris laminae\protect\index{Sachverzeichnis}{lamina cochleae} absorbentur}.%
\edlabel{LH_37_01_024v_nervusacusticus_egdtr-4}%
}{\lemma{\textit{Ramus} \lbrack...\rbrack\ \textit{absorbentur}}%
\Cfootnote{\cite{01203}\textsc{Duverney}, \textit{Tractatus de organo auditus}, S.~23. Zitat mit Auslassungen.}}%
%
\edtext{% ((24)) mit Auslassungen §
}{{\xxref{LH_37_01_024v_S.24-1}{LH_37_01_024v_S.24-2}}{%
\lemma{In avibus \lbrack...\rbrack\ \textit{vestiunt}}%
\Cfootnote{\cite{01203}a.a.O., S.~24. Zitate mit Auslassungen.}}}
\edlabel{LH_37_01_024v_S.24-1}%
In avibus\protect\index{Sachverzeichnis}{avis} et piscibus\protect\index{Sachverzeichnis}{piscis}
abest cochlea,\protect\index{Sachverzeichnis}{cochlea}
et praeter tres meatus\protect\index{Sachverzeichnis}{meatus} semicirculares quartum habent rectum\lbrack,\rbrack\
\textit{altera extremitate clausum, altera vero cum caeteris in cavitatem patentem
quae praestat vicem vestibuli.\protect\index{Sachverzeichnis}{vestibulum}}
\textit{Duo rami portionis mollis} nervi\protect\index{Sachverzeichnis}{nervus}
ibi \textit{in filamenta\protect\index{Sachverzeichnis}{filamentum} et membranas\protect\index{Sachverzeichnis}{membrana} extenduntur}
\edtext{\lbrack\textit{quae}\rbrack}{%
\lemma{qui}\Bfootnote{%
\textit{L~ändert Hrsg. nach Vorlage}}}
\textit{canales\protect\index{Sachverzeichnis}{canales semicirculares} intus}
\edtext{\textit{vestiunt.}%
\edlabel{LH_37_01_024v_S.24-2}
%
\edtext{% ((25)) mit Auslassungen §
}{{\xxref{LH_37_01_024v_S.25-1}{LH_37_01_024v_S.25-2}}{%
\lemma{Canalium \lbrack...\rbrack\ mobilis}%
\Cfootnote{\cite{01203}a.a.O., S.~25. Zitat mit Auslassungen.}}}%
\edlabel{LH_37_01_024v_S.25-1}%
Canalium\protect\index{Sachverzeichnis}{canales semicirculares}}{%
\lemma{\textit{vestiunt.}}\Bfootnote{%
\textit{(1)}~Canales
\textit{(2)}~Canalium%
~\textit{L}}}
horum \textit{quivis figuram habet duarum tubarum\protect\index{Sachverzeichnis}{tuba}
quarum utraque orificio angustiori in alteram infixa},
et tremunt partes modo ampliores modo arctiores.
Materia\protect\index{Sachverzeichnis}{materia}
\edtext{\lbrack ossis\rbrack}{%
\lemma{ossi}\Bfootnote{%
\textit{L~ändert Hrsg.}}}
petrosi\protect\index{Sachverzeichnis}{os petrosum} circa canales est spongiosa,
adeoque facile mobilis.%
\edlabel{LH_37_01_024v_S.25-2}
%
\edtext{% ((26)) mit Auslassungen §
\textit{Per communicationem\protect\index{Sachverzeichnis}{communicatio} portionis durae
nervi auditorii\protect\index{Sachverzeichnis}{nervus auditorius} cum ramis 5ti paris
qui in partes} voci\protect\index{Sachverzeichnis}{vox} destinatas \textit{sparguntur}\lbrack,\rbrack\
\textit{explicatur auditus\protect\index{Sachverzeichnis}{auditus} et vocis\protect\index{Sachverzeichnis}{vox}} connexio,
et cur auditus\protect\index{Sachverzeichnis}{auditus} ad $\langle$vocem$\rangle$ imitand$\langle$am$\rangle$ provocet.%
}{\lemma{\textit{Per} \lbrack...\rbrack\ provocet}%
\Cfootnote{\cite{01203}a.a.O., S.~26. Zitat mit Auslassungen.}}
\pend%
\count\Bfootins=1200
\count\Afootins=1200
\count\Cfootins=1200
%
% ENDE DES STÜCKES auf Blatt 24v
%
%
\newpage%
%