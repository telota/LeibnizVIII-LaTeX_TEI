%   % !TEX root = ../../VIII,3_Rahmen-TeX_8-1.tex
%
%
%   Band VIII, 3 N.~??19
%   Signatur/Tex-Datei: AE_1684_319-325_intro
%   RK-Nr. 60202 + 49434; 61051
%   Überschrift: [Demonstrationes novae de resistentia solidorum]
%   Datierung: [März/April 1683 bis Juli 1684; 1693]
%   WZ: (–)
%.  SZ: (—)
%.  Bilddateien (PDF): (–) 
%
%
\selectlanguage{ngerman}%
\frenchspacing%
%
\footnotesize%
\pstart%
\noindent%
\label{AE_1684_319-325_intro_jecg}%
\edlabel{AE_1684_319-325_intro_MariotteKritik-1}%
In seinem Brief an Leibniz vom 28. April 1678 kündigte E.~Mariotte\protect\index{Namensregister}{\textso{Mariotte}, Edme, Seigneur de Chazeuil ca. 1620-1684} an, das Verhältnis von Bruchfestigkeit und Zugfestigkeit der Balken, wie Galilei\protect\index{Namensregister}{\textso{Galilei} (Galilaeus, Galileus), Galileo 1564\textendash1642} es % in der zweiten Unterredung der \textit{Discorsi} 
bestimmt habe, entspreche nicht den Ergebnissen empirischer Messungen:
Im Vergleich zur Bruchfestigkeit sei die Zugfestigkeit zweimal so groß wie vom italienischen Naturforscher behauptet.
Grund für Galileis Fehleinschätzung sei, so Mariotte, die unhaltbare Annahme gewesen, dass ein Balken sich wie ein vollkommen starrer Körper verhalte, der beim Zug sowie beim Bruch auf einmal reiße, wohingegen er sich in beiden Fällen zunächst dehnen bzw. biegen müsse (\textit{LSB} III,~2 N.~163, S.~405.11\textendash406.8\cite{01232}; vgl. G.~\textsc{Galilei}, \cite{00050}\textit{Discorsi}, Leiden 1638, S.~114\,f.;\cite{00050} \textit{GO} VIII, S.~156\,f.\cite{00048}).
In seiner Ausführung lieferte Mariotte allerdings ein missverständliches numerisches Beispiel, welches vielmehr Galileis Einschätzung wiedergab, dass die Zugfestigkeit eines Balkens sich zur Bruchfestigkeit so verhalte wie die Länge zur \textit{Hälfte} der Dicke.
\pend%
\pstart%
Obgleich Leibniz nichts auf Mariottes Mitteilung erwidert zu haben scheint (seine Antwort ist aber verschollen), dürfte er in den folgenden Monaten eigene Überlegungen zu dieser Thematik angestellt haben.
Das kann man seiner Aufzeichnung LH XXXV~9,~5 Bl.~2 entnehmen, die sich mit Galileis Ansichten über die Festigkeit der Balken befasst und anhand ihres Trägers auf den Sommer 1678 datierbar ist.
(Es ist nicht zwingend, diese Aufzeichnung als unmittelbares Konzept des Briefes an Mariotte von Ende Juli/Anfang August 1682 anzusehen, wie die Editoren von \textit{LSB} III,~3 N.~380\cite{01263} vorschlagen.)
Mit dieser Thematik hatte sich Leibniz zudem bereits in Paris auseinandergesetzt (siehe \textit{LSB} VIII,~2 N.~19\cite{01249}\cite{01250}\cite{01251}\cite{01252} bis N.~26\cite{01253}\cite{01254}\cite{01255}\cite{01256} und VI,~3 N.~11\cite{00260}), weshalb Mariottes kritische Bemerkungen ihn nicht unvorbereitet trafen.
\pend%
\pstart%
Mariotte wiederholte seine Kritik an Galilei in einem weiteren Schreiben an Leibniz vom 20. Juli 1682.
Dort hielt er ausdrücklich fest, dass bei Balken die Zugfestigkeit sich so zur Bruchfestigkeit verhalte wie die Länge zu einem \textit{Viertel} der Dicke (\textit{LSB} III,~3 N.~376, S.~670.14\textendash20\cite{01233}).
In seiner Antwort von Ende Juli/Anfang August 1682 räumte Leibniz freilich ein, weder Mariottes Argument richtig verstanden zu haben noch derzeit (infolge laufender Bauarbeiten in der Hannoveraner\protect\index{Ortsregister}{Hannover} Hofbibliothek) imstande zu sein, ein Exemplar der \textit{Discorsi} zu Rate zu ziehen; unter der Annahme undehnbarer bzw. unbiegsamer Balken, wie Mariotte sie Galilei zuschreibe, lasse sich jedoch feststellen, dass Zugfestigkeit und Bruchfestigkeit sich so zueinander verhielten wie die Länge und die \textit{Hälfte} der Dicke.
Leibniz bemerkte überdies, diese \textit{hypothèse de la rupture uniforme} treffe wohl auf den Fall zu, dass zwei harte und glatte, aneinander haftende Platten voneinander abgetrennt werden sollten; ob sich aus Mariottes \textit{hypothèse des fibres extensibles} ein anderes Verhältnis zwischen Zugfestigkeit und Bruchfestigkeit ergebe, habe er noch nicht zu prüfen vermocht (\textit{LSB} III,~3 N.~380, S.~678.25\textendash680.9; vgl. zudem S.~681.6\textendash20)\cite{01263}.
\pend%
\pstart%
Diese von Verständigungsschwierigkeiten behaftete Kontroverse setzte sich im weiteren Briefwechsel der beiden Gelehrten fort.
Am 31. August 1682 erwiderte Mariotte, Leibnizens Überlegung bestätige doch nicht Galileis Ansicht, sondern seine eigene, dass die Balken sich wie dehnbare Körper verhielten (\textit{LSB} III,~3 N.~394, S.~705.11\textendash19\cite{01340}).
Leibniz antwortete Mitte September aus dem Harz mit der Sendung einer \textit{Demonstratio regulae meae de resistentia solidorum ex Hypothesi rupturae uniformis}, welche zeigen sollte, dass Zugfestigkeit und Bruchfestigkeit eines prismatischen Balkens sich zueinander so verhielten wie die Länge und die Höhe des Schwerpunktes (\textit{LSB} III,~3 N.~400, bes. S.~713.20\textendash24).\cite{01264}
Dieses allgemeine Theorem bestätigte offenbar Galileis Ansicht.
Mariottes Entgegnung am 25. Januar 1683 lag eine kleine Abhandlung bei: seine \textit{Dissertation sur la resistance des solides pour faire voir que Galilée n'a pas bien expliqué la resistance des solides fichés perpendiculairement dans un mur quand on les tire de travers}.
Dort bekräftigte er die These, dass Zugfestigkeit und Bruchfestigkeit eines Balkens sich zueinander wie
\pend
%%%%künstlicher Seitenumbruch mitten im AbsatzKZEITZ
\newpage
\pstart\noindent die Länge zu einem \textit{Viertel} der Dicke verhielten.
Ferner ging Mariotte in seiner Abhandlung auf das seiner Berechnung zugrundeliegende physikalische Modell der Festigkeit und auf dessen Grundannahmen ein: etwa, dass die Dehnung elastischer Körper in direktem Verhältnis zur angewandten Spannkraft stehe (\textit{LSB} III,~3 N.~437, bes. S.~772.18 und 774.15\textendash17).\edlabel{AE_1684_319-325_intro_MariotteKritik-2}%
\pend%
\pstart%
Mariottes%
\edlabel{AE_1684_319-325_intro_LeibizAnMariotte-1}
\textit{Dissertation} dürfte Leibniz dazu veranlasst haben, eine eigene umfassende Untersuchung zum Verhältnis von Zugfestigkeit und Bruchfestigkeit anzustellen, deren Ergebnisse sich schließlich sowohl von denen Galileis\protect\index{Namensregister}{\textso{Galilei} (Galilaeus, Galileus), Galileo 1564\textendash1642} wie auch von denjenigen Mariottes\protect\index{Namensregister}{\textso{Mariotte}, Edme, Seigneur de Chazeuil ca. 1620-1684} unterscheiden sollten.
Zwischen Ende Januar, % oder Anfang Februar, 
als er Mariottes Brief frühestens empfang, und Ende April 1683, als er seine Antwort an Mariotte spätestens anfertigte, verfasste er einige Entwürfe, die wichtige Beiträge zu seiner Festigkeits- und Elastizitätslehre darstellen.
\pend%
\pstart%
Als erster entstand wahrscheinlich der Entwurf N.~14\textsubscript{1}, der in seiner französischen Überschrift \textit{De la resistence des solides} noch Spuren einer Auseinandersetzung mit Mariottes \textit{Dissertation} aufweisen dürfte (der Text ist sonst lateinisch verfasst).
Leibniz trägt dort beiden aus der Kontroverse mit Mariotte hervorgegangenen \glqq Hypothesen\grqq\ Rechnung:
\textit{vel enim corpus trabis solidae consideramus tanquam rigidum, vel tanquam tensile} (N.14\textsubscript{1}, S.~\refpassage{LH_37_03_073r_zweiHypothesen_ldifug-1}{LH_37_03_073r_zweiHypothesen_ldifug-2}).
Im Hinblick auf die erste Hypothese bestätigt er zunächst das Ergebnis, das er bereits Mitte September 1682 an Mariotte mitgeteilt hatte:
Im Fall eines starren Balkens verhielten sich Zugfestigkeit und Bruchfestigkeit zueinander wie die Länge zur Höhe des Schwerpunktes (N.~14\textsubscript{1}, S.~\refpassage{LH_37_03_073r-v_ergebnisse_vubhed-1}{LH_37_03_073r-v_ergebnisse_vubhed-2}).
Zum Nachweis bezieht er sich auf das Modell der aneinander haftenden harten Platten, über das er bereits Ende Juli/Anfang August 1682 an Mariotte geschrieben hatte (N.~14\textsubscript{1}, S.~\refpassage{LH_37_03_073r_zweiplatten_ndybf-1}{LH_37_03_073r_zweiplatten_ndybf-2}).
Im zweiten Teil des Entwurfes % (N.~??Y\textsubscript{1}, S.~\refpassage{LH_37_03_074r_zweiterTeil_lbhuno-1}{LH_37_03_074r_zweiterTeil_lbhuno-2}) 
geht Leibniz auf die alternative \glqq Hypothese\grqq\ eines biegsamen Balkens ein und kommt zu dem Ergebnis, dass in diesem Fall die Zugfestigkeit sich zur Bruchfestigkeit so verhalte wie die Länge zu einem \textit{Drittel} der Dicke (d.h., bei einem Balken mit gleicher Höhe wie Länge sei die Bruchfestigkeit ein Drittel der Zugfestigkeit; N.~14\textsubscript{1}, S.~\refpassage{LH_37_03_074r-v_EinDrittel_bvxycjw-1}{LH_37_03_074r-v_EinDrittel_bvxycjw-2}).
Schließlich untersucht Leibniz unter Berücksichtigung beider \glqq Hypothesen\grqq\ die von Galilei aufgeworfene und von Mariotte vernachlässigte Frage nach der Gestalt eines einseitig gestützten Balkens, der in jedem Punkt seiner Länge den gleichen Bruchwiderstand aufweist.
Aus dieser besonderen Untersuchung, der sich Leibniz bereits in Paris gewidmet hatte (\textit{LSB} VIII,~2 N.~22\cite{01252}), ergibt sich vorerst, dass im Fall eines starren Balkens die gesuchte Gestalt (Längsschnitt), wie von Galilei behauptet, ein parabolisches Dreieck sei, im Fall eines biegsamen Balkens hingegen ein gewöhnliches rechtwinkliges Dreieck (N.~14\textsubscript{1}, S.~\refpassage{LH_37_03_073v_TrabsParabolica_hegte-1}{LH_37_03_073v_TrabsParabolica_hegte-2}; S.~\refpassage{LH_37_03_074v_TrabsTriangularis_tucbv-1}{LH_37_03_074v_TrabsTriangularis_tucbv-2}; siehe G.~\textsc{Galilei}, \cite{00050}\textit{Discorsi}, S.~138\textendash141;\cite{00050} \textit{GO} VIII, S.~178\textendash181\cite{00048}).
\pend%
\pstart%
Offenbar\edlabel{LH_37_03_073r_ExcerptaMeliora_luzgw-1} zur gleichen Zeit verfasste Leibniz wohl auch den Entwurf N.~14\textsubscript{2} \textit{De firmitate corporum}, welcher wohl einer ausführlichen und systematischen Darstellung der in N.~14\textsubscript{1} entwurfartig ausgeführten Untersuchung dienen sollte.
Der neue Entwurf behandelt allerdings nur die erste \glqq Hypothese\grqq\ \textendash\ die Annahme eines unbiegsamen Balkens \textendash\ und bricht abrupt beim Übergang zur zweiten ab, \textit{qua ponimus corpus antequam frangatur aut rumpatur flecti aut tendi} (N.~14\textsubscript{2},
S.~\refpassage{LH_35_09_16_001v_zweiteHypothese_mrhze-1}{LH_35_09_16_001v_zweiteHypothese_mrhze-2}).
Dass N.~14\textsubscript{2} von N.~14\textsubscript{1} abhängt, ist vornehmlich an der Beantwortung der bereits erwähnten Frage nach der Gestalt des gleichmäßig widerstandsfähigen Balkens erkennbar:
In N.~14\textsubscript{2} (S.~\refpassage{LH_35_09_16_001r_GleichesErgebnis-lziu-1}{LH_35_09_16_001r_GleichesErgebnis-lziu-2} und gestr. Variante zu S.~\refpassage{LH_35_09_16_001r_firmioremsed-1}{LH_35_09_16_001v_firmioremsed-2}) wird unmittelbar die Lösung vorgeschlagen, die im früheren Entwurf erst nach etlichen gescheiterten Versuchen mit unterschiedlichen Ansätzen ermittelt worden ist (vgl. N.~14\textsubscript{1}, S.~\refpassage{LH_37_03_073v_TrabsParabolica_hegte-1}{LH_37_03_073v_TrabsParabolica_hegte-2}; \refpassage{LH_37_03_074v_TrabsTriangularis_tucbv-1}{LH_37_03_074v_TrabsTriangularis_tucbv-2}).
Dass Leibniz jedoch, zumindest streckenweise, an beiden Texten parallel gearbeitet hat, zeigt ein Querverweis, bei dem N.~14\textsubscript{1} (S.~\refpassage{LH_37_03_073v_gekreuzterQuerverweis_rhcg-1}{LH_37_03_073v_gekreuzterQuerverweis_rhcg-2}) anscheinend auf N.~14\textsubscript{2} (S.~\refpassage{LH_37_03_069v_ratioquadratorum-1}{LH_37_03_069v_ratioquadratorum-2}) Bezug nimmt. (Leibniz könnte dort aber auch auf einen nicht überlieferten oder unbekannten älteren Text verweisen.)
Für eine gleichzeitige Entstehung beider Entwürfe spricht auch, dass sämtliche Träger von N.~14\textsubscript{1} und N.~14\textsubscript{2} das gleiche Wasserzeichen einer Papiermühle aus dem Harz\protect\index{Ortsregister}{Harz} aufweisen.%
\edlabel{LH_37_03_073r_ExcerptaMeliora_luzgw-2}
\pend%
\newpage% % !!!!!!!!!!!!!!!!!!!!!!!!!!!!!!!!!!!!!!!!!!!!!!!!!!!!!!!!!!!!!!!!!!!!!!!!!
\pstart%
Im Anschluss an N.~14\textsubscript{2} wurde aller Wahrscheinlichkeit nach der Entwurf N.~14\textsubscript{3} \textit{Explicatio mechanica elastri} verfasst,
welcher dem abbrechenden und gestrichenen Schlussteil von N.~14\textsubscript{2} auf ein und demselben Träger unmittelbar folgt.
In dieser weiteren reichhaltigen Abhandlung skizziert Leibniz im Zusammenhang mit seiner Untersuchung über die Festigkeit und mit seinen früheren naturphilosophischen Überlegungen ein physikalisches Modell, das eine Erklärung des elastischen Verhaltens der Körper ermöglichen soll. 
\pend%
% 
\pstart%
In\edlabel{LH_35_14_02_039r2_Datierung-1}
die Entstehungszeit von N.~14\textsubscript{1} und N.~14\textsubscript{2} ist noch die skizzenhafte und titellose Aufzeichnung N.~14\textsubscript{4} zu verorten, der editorisch die Überschrift \textit{Solidum ubique aequiresistens} zugewiesen wird.
Dort behandelt Leibniz erneut die von Galilei stammende Frage nach der Gestalt des gleichmäßig widerstandsfähigen Balkens, diesmal aber vereinzelt und ohne Rücksicht auf die Unterscheidung der genannten zwei \glqq Hypothesen\grqq.
Galileis Ansicht, dass die gesuchte Gestalt ein parabolisches Dreieck sei, wird in der Aufzeichnung bestätigt, indem nach analytischer Methode die Gleichung bestimmt wird, aus der sich für jeden Punkt der Balkenlänge ein konstantes Verhältnis zwischen Bruchwiderstand und Bruchkraft (Moment des ungestützten Balkenteils) ergibt.
Als Ausdruck der Bruchkraft wird in N.~14\textsubscript{4} (S.~\refpassage{LH_35_14_02_039r_integralfaktor-1}{LH_35_14_02_039r_integralfaktor-2}), ebenso wie in N.~14\textsubscript{1} (S.~\refpassage{LH_37_03_073v_integralfaktor_ewj-1}{LH_37_03_073v_integralfaktor_ewj-2}) und N.~14\textsubscript{2} (S.~\refpassage{LH_35_09_16_001r_integralfaktor_fdhv-1}{LH_35_09_16_001r_integralfaktor_fdhv-2}), jeweils der gleiche Faktor % $\displaystyle\!\!\int\!\!\overline{xy\,d\overline{x}}$ 
$\displaystyle\!\!\int\!\!\!xy\,dx$ angegeben, was wohl als Zeichen für eine gemeinsame Entstehung zu betrachten ist.
Die skizzenhaften Diagramme \lbrack\textit{Fig.~4}\rbrack\ bis \lbrack\textit{Fig.~7}\rbrack\ am Ende von N.~14\textsubscript{4} lassen sich ferner als Entwürfe zu ähnlichen Diagrammen in N.~14\textsubscript{1} und vornehmlich N.~14\textsubscript{2} deuten (siehe die Erläuterungen auf S.~\pageref{LH_35_14_02_039r2_Fig.6}). % für die genauen Angaben
Die Aufzeichnung N.~14\textsubscript{4} entstand jedoch nicht vor den Entwürfen N.~14\textsubscript{1} und 14\textsubscript{2}, sondern in der Mitte zwischen beiden, wie dies vornehmlich an einer besonderen Entwicklung der Untersuchung deutlich wird:
Während in \textit{De la resistence des solides} der gleichmäßig widerstandsfähige Balken noch, wie bei Galilei, als \textit{konvexes} parabolisches Dreieck dargestellt wird (vgl. N.~14\textsubscript{1}, S.~\pageref{LH_37_03_073v_Fig.5}, \lbrack\textit{Fig.~5}\rbrack), zeigt er sich in \textit{De firmitate corporum} als \textit{konkaves} parabolisches Dreieck (vgl. N.~14\textsubscript{2}, S.~\pageref{LH_35_09_16_001r_Fig.13}, \lbrack\textit{Fig.~13}\rbrack), d.h. mit derselben Gestalt, die er noch im späteren Aufsatz N.~14\textsubscript{6},~\textit{E\textsuperscript{1}} (S.~\pageref{LH_37_03_072v+AE_1684_323_Fig.5e}, \lbrack\textit{Fig.~5e}\rbrack) behalten wird.
Die \glqq Umwandlung\grqq\ von der einen zu der anderen Gestalt vollzog sich aber anscheinend zu der Zeit, als Leibniz die Aufzeichnung N.~14\textsubscript{4} verfasste, wie die Entwicklung des Diagramms \lbrack\textit{Fig.~1}\rbrack\ dort zeigt (siehe die Erläuterung hierzu, S.~\pageref{LH_35_14_02_039r2_Fig.1}).
Schließlich gilt es zu bemerken, dass Leibniz wieder Zugang zu einem Exemplar von Galileis \textit{Discorsi} gehabt haben muss, als er N.~14\textsubscript{4} verfasste, wie die auf demselben Träger überlieferte Notiz N.~15 belegt.\edlabel{LH_35_14_02_039r2_Datierung-2}\edlabel{AE_1684_319-325_intro_LeibizAnMariotte-2}%.%.%.%.%.%.%.
\pend%
\pstart%
Die Ergebnisse \edlabel{AE_1684_319-325_intro_Brief456_pwyu-1}seiner umfangreichen Untersuchung zur Festigkeit der Balken teilte Leibniz an Mariotte zwischen März und April 1683 in einem wahrscheinlich aus Zellerfeld\protect\index{Ortsregister}{Zellerfeld} abgesendeten Brief mit (\textit{LSB} III,~3 N.~456\cite{01262}).
Vom Sommer 1682 bis zum Sommer 1684 hielt sich Leibniz nämlich zumeist im Harzgebiet\protect\index{Ortsregister}{Harz} auf, vorwiegend in Osterode,\protect\index{Ortsregister}{Osterode} Clausthal\protect\index{Ortsregister}{Clausthal} oder Zellerfeld\protect\index{Ortsregister}{Zellerfeld} (vgl. \textit{Chronik}, S.~68\textendash73\cite{01236}).
In seinem Schreiben an Mariotte berichtet er an erster Stelle über seine Berechnung des Verhältnisses zwischen Zugfestigkeit und Bruchfestigkeit: Im Fall eines biegsamen Balkens gleiche dieses Verhältnis dem der Länge zu einem \textit{Drittel} der Dicke (\textit{LSB} III,~3 N.~456, S.~794.2\textendash795.7\cite{01262}).
Anders als in den Entwürfen N.~14\textsubscript{1} und 14\textsubscript{2} widmet Leibniz jetzt der \glqq Hypothese\grqq\ des unbiegsamen Balkens keine Aufmerksamkeit mehr.
Dies kann als Zeichen dafür gedeutet werden, dass das Schreiben an Mariotte nach N.~14\textsubscript{1} und 14\textsubscript{2} entstand, da auch in der späteren Abhandlung N.~14\textsubscript{6} die beiden \glqq Hypothesen\grqq\ nicht mehr als gleichwertig behandelt werden und nur Leibnizens (und Mariottes) Grundannahme in den Mittelpunkt rückt.
Als Bestätigung kommt hinzu, dass im Brief an Mariotte die Frage nach der Gestalt des gleichmäßig widerstandsfähigen \textit{biegsamen} Balkens ausführlich besprochen und genauso wie in N.~14\textsubscript{1} beantwortet wird, während Galileis Fragestellung nur am Rande Erwähnung findet (\textit{LSB} III,~3 N.~456, S.~796.9\textendash32\cite{01262}; vgl. N.~14\textsubscript{1}, S.~\refpassage{LH_37_03_074v_TrabsTriangularis_tucbv-1}{LH_37_03_074v_TrabsTriangularis_tucbv-2}).
Mit dem Entwurf N.~14\textsubscript{3} % \textit{Explicatio mechanica elastri} 
weist das Schreiben an Mariotte eine möglicherweise \makebox[1.0\textwidth][s]{engere Verwandtschaft auf, die als Beleg für eine gemeinsame Entstehungszeit gedeutet werden könnte:}
\pend%
%%%%%%%%%%%zur Stabilisierung künstlicher Seitenumbruch mitten im Absatz KZeitz
\newpage
\pstart\noindent
Die am Ende von N.~14\textsubscript{3} (S.~\refpassage{LH_35_09_16_020v_kolbenmodell-1}{LH_35_09_16_020v_kolbenmodell-2}) anzutreffende Ausführung über das Verhältnis zwischen Dehnung eines elastischen Körpers und angewandter Spannkraft wiederholt sich in nahezu gleicher Form im Brief von März/April 1683 (\textit{LSB} III,~3 N.~456, S.~795.21\textendash796.3).%
\edlabel{AE_1684_319-325_intro_Brief456_pwyu-2}
% \textbf{??? erzwungener Seitenumbruch (sonst Instabilität) ???}
\pend%
%% \newpage%
%%
\pstart%
Eine noch engere Verbindung mit diesem Brief % an Mariotte 
weist die sonst von N.~14\textsubscript{2} unmittelbar abhängige Aufzeichnung N.~14\textsubscript{5} \textit{De duabus tabulis planis divellendis} auf,
in der Leibniz gesondert auf das schon erwähnte Modell der aneinander haftenden harten Platten eingeht und ein Gedankenexperiment zur Messung der für die Abtrennung der Platten notwendigen Kraft entwirft.
Die Aufzeichnung N.~14\textsubscript{5} ist nämlich auf ein und demselben Träger verfasst wie das Teilkonzept \textit{L\textsuperscript{1}} des Briefes an Mariotte (vgl. \textit{LSB} III,~3 N.~456, S.~793.16\textendash20),
weshalb es anzunehmen ist, dass auch N.~14\textsubscript{5} zwischen März und April 1683 abgefasst wurde.
Ihre unmittelbare Abhängigkeit von \textit{De firmitate corporum} ist indes daran zu erkennen, dass der Anfangsteil von N.~14\textsubscript{5} (S.~\refpassage{LH037_03_118r_wiedergabe-1}{LH037_03_118r_wiedergabe-2}) die Abschrift einer Passage von N.~14\textsubscript{2} (S.~\refpassage{LH_37_03_069r_duaetabulae-1}{LH_37_03_069r_duaetabulae-2}) darstellt.%
\pend%
% \newpage% %
\pstart%
Entweder\edlabel{DNDRS_Ueberarbeitung_scjvtx-1} noch zu der Zeit, als der Brief an Mariotte angefertigt wurde, oder erst in den folgenden Monaten begann Leibniz offenbar, an einer Veröffentlichung seiner Untersuchung über die Festigkeit der Balken zu arbeiten.
Daraus ist zunächst das Konzept N.~14\textsubscript{6},~\textit{L\textsuperscript{1}} entstanden, das die Überschrift \textit{De resistentia solidorum} trägt.
Diesen Text, der die Hintergründe der Untersuchung \textendash\ vornehmlich die Auseinandersetzung mit Galilei und Mariotte % sowie mit weiterer zeitgenössischer Literatur 
\textendash\ erörterte, hat Leibniz zunächst aufgegeben und zu einem späteren Zeitpunkt überarbeitet.
In seiner ursprünglichen Fassung wies er noch enge Verwandtschaft mit dem Brief an Mariotte von März/April 1683 und mit den zugehörigen Vorarbeiten N.~14\textsubscript{1}, 14\textsubscript{2} und 14\textsubscript{4} auf.
Dies zeigt sich insbesondere in der Beantwortung der Frage nach der Gestalt des gleichmäßig widerstandsfähigen Balkens:
Ebenso wie in dem Brief und den früheren Entwürfen hält Leibniz im Konzept N.~14\textsubscript{6},~\textit{L\textsuperscript{1}} noch fest, dass unter Annahme eines starren Balkens die gesuchte Gestalt ein parabolisches Dreieck sei, ein gewöhnliches rechtwinkliges Dreieck hingegen unter Annahme eines biegsamen Balkens (vgl. den textkritischen Apparat zu S.~\refpassage{LH_37_03_071v+AE_1684_320_Uebrgng-1}{LH_37_03_071v+AE_1684_320_Uebrgng-2} und \refpassage{LH_37_03_071v+AE_1684_321_GaliQuod-1}{LH_37_03_071v+AE_1684_321_GaliQuod-2}).
Die ursprüngliche Fassung des Teilkonzepts N.~14\textsubscript{6},~\textit{L\textsuperscript{1}} wurde demnach frühestens im März/April 1683 angefertigt und lag spätestens zu der Zeit vor, als das weitere Teilkonzept N.~14\textsubscript{6},~\textit{L\textsuperscript{2}}, das auf \textit{L\textsuperscript{1}} beruht, verfasst wurde (siehe unten).
\pend%
%\newpage% % !!!!!!!!!!!!!!!!!!!!!!!!!!!!!!!!!!!!!!!!!!!!!!!!!!!!!!!!!!!!!!!!!!!!!!!!!
\pstart%
Von der Entstehungszeit des Konzepts N.~14\textsubscript{6},~\textit{L\textsuperscript{1}} hängt diejenige des Entwurfes N.~14\textsubscript{7} ab, welcher ursprünglich den Text von \textit{L\textsuperscript{1}} auf ein und demselben Träger unmittelbar fortsetzte.
Der neue Text war allerdings abweichenden Themen gewidmet: im ersten Teil dem Nachweis, dass die Dehnung eines gespannten elastischen Körpers (etwa einer Luftmasse in einem verschlossenen Behälter) und die Größe der angewandten Spannkraft in direktem Verhältnis zueinander stünden (N.~14\textsubscript{7}, S.~\refpassage{LH_35_09_16_002_Beweis-1}{LH_35_09_16_002_Beweis-2}); im übrigen Teil dem Entwurf eines physikalischen Modells zur Erklärung der Festigkeit und Elastizität der Körper (ebd., S.~\refpassage{LH_35_09_16_002v_Festigkeitsmodell_redg-1}{LH_35_09_16_002v_Festigkeitsmodell_redg-2}).
Aufgrund dieser thematischen Abwandlung muss sich Leibniz \textendash\ wohl spätestens bei seiner Überarbeitung der Konzepte N.~14\textsubscript{6},~\textit{L\textsuperscript{1}} und \textit{L\textsuperscript{2}} (siehe unten) \textendash\ dafür entschieden haben, den Text N.~14\textsubscript{7} von N.~14\textsubscript{6},~\textit{L\textsuperscript{1}} abzulösen und mit einer selbständigen Überschrift zu versehen, die in ihrer Kurzfassung lautet: \textit{Demonstratio quod extensiones elasticorum sint viribus tendentibus proportionales}.
\pend%
% \newpage%
\pstart%
Zugleich verfolgte Leibniz weiterhin das Projekt, seine Untersuchung über die Festigkeit der Balken zu veröffentlichen.
Aus Teilen von N.~14\textsubscript{6},~\textit{L\textsuperscript{1}} (S.~\refpassage{LH_37_03_071v_Anfang-1}{LH_37_03_071v_Anfang-2}) entstand somit das weitere Konzept N.~14\textsubscript{6},~\textit{L\textsuperscript{2}}, dem Leibniz die neue Überschrift zuwies: \textit{Demonstrationes novae de resistentia solidorum}.
Obwohl \textit{L\textsuperscript{2}} deutlich umfangreicher und vollständiger ist als \textit{L\textsuperscript{1}}, erweisen sich beide Konzepte in ihrer jeweiligen ursprünglichen Fassung als inhaltlich homogen.
Dies lässt sich erneut an der Ausführung über die Gestalt des gleichmäßig widerstandsfähigen prismatischen Balkens feststellen:
In N.~14\textsubscript{6},~\textit{L\textsuperscript{2}} beteuert
Leibniz noch einmal, dass im Fall eines biegsamen Balkens der gesuchte Längsschnitt ein rechtwinkliges Dreieck darstelle (S.~\refpassage{LH_37_03_072v_L2-trabstriangularis}{LH_37_03_072v_erstzng-2}).
Demgemäß dürfte sich die Entstehungszeit des Konzepts \textit{L\textsuperscript{2}} nicht wesentlich von der des Konzepts \textit{L\textsuperscript{1}} unterscheiden.
Gewiss lagen beide vor, als Leibniz sie vor dem Sommer 1684 überarbeitete.
\pend%
% \newpage%
%
\pstart%
Bei diesem weiteren Schritt ging es wesentlich wieder um die Frage nach der Gestalt des gleichmäßig widerstandsfähigen Balkens.
Zu diesem Zeitpunkt hatte Leibniz offenbar die bislang vertretene These verworfen, dass im Fall eines biegsamen Balkens die gleichmäßig widerstandsfähige Gestalt ein gewöhnliches rechtwinkliges Dreieck sei, und strich die entsprechenden Passagen % in seinem Konzept des Briefes an Mariotte von März/April 1683 (\textit{LSB} III,~3 N.~465~\textit{L\textsuperscript{2}}, S.~796.9\textendash32) sowie 
in den Konzepten N.~14\textsubscript{6},~\textit{L\textsuperscript{1}} und~\textit{L\textsuperscript{2}} (S.~\refpassage{LH_37_03_071v+AE_1684_321_GaliQuod-1}{LH_37_03_071v+AE_1684_321_GaliQuod-2}; \refpassage{LH_37_03_072v_erstzng-1}{LH_37_03_072v_erstzng-2}).
Den aus \textit{L\textsuperscript{2}} getilgten Textabschnitt ersetzte er durch eine neu bearbeitete Passage aus~\textit{L\textsuperscript{1}}, in der die (von Galilei\protect\index{Namensregister}{\textso{Galilei} (Galilaeus, Galileus), Galileo 1564\textendash1642} stammende) These übernommen wird, dass der Längsschnitt des gleichmäßig widerstandsfähigen Balkens ein parabolisches Dreieck sei, wobei jetzt darunter ein \textit{konkaves} gemeint ist (S.~\refpassage{AE_1684_323-324_erstzng-1}{AE_1684_323-324_erstzng-2}; vgl. den Variantenapparat zu S.~\refpassage{LH_37_03_071v+AE_1684_320_Uebrgng-1}{LH_37_03_071v+AE_1684_320_Uebrgng-2}).
Bei der Überarbeitung der Konzepte \textit{L\textsuperscript{1}} und \textit{L\textsuperscript{2}} gab Leibniz somit \textendash\ hinsichtlich der Frage nach der Gestalt der \textit{trabs ubique aequiresistens} \textendash\ die Unterscheidung zwischen starrem und biegsamem Balken auf und verallgemeinerte in dieser Weise die Ansicht, die er früher nur für den Fall des starren Balkens als zutreffend erachtet hatte.
Schließlich fügte er aber eine neue Erkenntnis hinzu: Ein gleichmäßig widerstandsfähiger Balken, der neben seinem Eigengewicht eine gleichmäßig verteilte zusätzliche Last tragen soll, könne doch wohl ein gewöhnliches rechtwinkliges Dreieck als Längsschnitt haben (S.~\refpassage{LH_35_09_16_002r_trabstriangularis_ftve-1}{LH_35_09_16_002r_trabstriangularis_ftve-2}).
\pend%
\pstart%
Ausschlaggebend für diese Entwicklung könnte vorwiegend die Aufzeichnung N.~14\textsubscript{8} \textit{De figura trabis ubique aequaliter resistentis} gewesen sein.
Dieser Text ähnelt zum Teil noch der Aufzeichnung N.~14\textsubscript{4}, geht zum Teil aber auch deutlich über diese hinaus.
In N.~14\textsubscript{8} wird nämlich, nach gleicher analytischer Methode wie in N.~14\textsubscript{4}, eine Gleichung gesucht, die für jeden Punkt der Balkenlänge ein konstantes Verhältnis von Bruchwiderstand und Bruchkraft darstellt; und auch hier wird wie in N.~14\textsubscript{4} von der Unterscheidung zwischen starrem und biegsamem Balken abgesehen.
Auch N.~14\textsubscript{8} kommt somit zum Ergebnis% gleichen allgemeinen Ergebnis wie N.~??Y\textsubscript{4}: Als gleichmäßig widerstandsfähig erweise sich 
, dass ein Balken mit einem (konkaven) parabolischen Dreieck als Längsschnitt sich im Allgemeinen als gleichmäßig widerstandsfähig erweise (S.~\refpassage{LH_35_09_14_001-002_ersteUntersuchung-1}{LH_35_09_14_001-002_ersteUntersuchung-2}).
Damit spiegelt N.~14\textsubscript{8} die neue Entwicklung wider, die sich bei der Überarbeitung von N.~14\textsubscript{6},~\textit{L\textsuperscript{1}} und~\textit{L\textsuperscript{2}} abzeichnet.
Von N.~14\textsubscript{4} hebt sich die Aufzeichung N.~14\textsubscript{8} % \textit{De figura trabis ubique aequaliter resistentis} 
aber insofern ab, als sie \textendash\ vermutlich zum ersten Mal \textendash\ auch den weiteren, in N.~14\textsubscript{4} noch nicht berücksichtigten Fall untersucht, dass der Balken zusätzlich eine gleichmäßige Last zu tragen hat.
Auch in dieser Hinsicht ähnelt N.~14\textsubscript{8} auffällig den überarbeiteten Fassungen von N.~14\textsubscript{6},~\textit{L\textsuperscript{1}} und~\textit{L\textsuperscript{2}}.
Anders als in beiden überarbeiteten Konzepten führt die Untersuchung in N.~14\textsubscript{8} jedoch zum Ergebnis, dass der Längsschnitt des gleichmäßig widerstandsfähigen Balkens auch im Fall einer zusätzlichen Belastung ein (nach innen gewölbtes) parabolisches Dreieck sei (ebd., S.~\refpassage{LH_35_09_14_001-002_zweiteUntersuchung-1}{LH_35_09_14_001-002_zweiteUntersuchung-2}).
Daher ist anzunehmen, dass die Aufzeichnung N.~14\textsubscript{8} nach der Abfassung von N.~14\textsubscript{4} und vor der \textendash\ möglicherweise auch durch N.~14\textsubscript{8} selbst veranlassten \textendash\ Überarbeitung der Teilkonzepte N.~14\textsubscript{6},~\textit{L\textsuperscript{1}} und~\textit{L\textsuperscript{2}} verfasst wurde.
Bei der Überarbeitung von~\textit{L\textsuperscript{1}} und~\textit{L\textsuperscript{2}} gab Leibniz das Ergebnis über die \textit{trabs onerata}, zu dem er in N.~14\textsubscript{8} gekommen war, offenbar auf und ersetzte es mit der neuen Ansicht, die in beiden Konzepten und schließlich auch in der gedruckten Fassung \textit{E\textsuperscript{1}} vertreten wird.
Fest steht jedenfalls, dass N.~14\textsubscript{8} vor der Aufzeichnung N.~14\textsubscript{9} (siehe unten) angefertigt wurde, da diese letztere an einer Stelle (S.~\refpassage{LH_35_09_14_003r_summasummarum_rglj-1}{LH_35_09_14_003r_summasummarum_rglj-2}) wohl auf N.~14\textsubscript{8} anspielt.
% Aller Wahrscheinlichkeit nach ist \textit{De figura tebis ubique aequaliter resistentis} daher zwischen März/April 1683 und dem Frühsommer 1684 entstanden.
Gegen eine spätere Entstehungszeit der Aufzeichnung N.~14\textsubscript{8} spricht auch, dass das in deren Träger vorliegende, von einer Papiermühle aus Osterode\protect\index{Ortsregister}{Osterode} stammende Wasserzeichen im Leibniz-Nachlass lediglich für die Jahre 1683 bis 1686 (am häufigsten aber 1683/1684) belegt ist.
\pend%
\pstart%
Auch die Aufzeichnung N.~14\textsubscript{9} \textit{Invenire conoeides solidum aequalis ubique resistentiae} lässt sich \makebox[1.0\textwidth][s]{ihrem Ursprung nach auf den Text N.~14\textsubscript{4} zurückführen.
N.~14\textsubscript{9} knüpft nämlich an eine gestrichene}
Passage in N.~14\textsubscript{4} an, in der die Frage nach der Gestalt eines einseitig gestützten, gleichmäßig widerstandsfähigen Rotationskörpers gestellt wird (S.~\refpassage{LH_35_14_02_039r2_Rotationskoerper_kdfgh-1}{LH_35_14_02_039r2_Rotationskoerper_kdfgh-2}).
Diese Frage wird in N.~14\textsubscript{9} wieder aufgenommen und mit der Feststellung beantwortet, dass ein (nach innen gewölbtes) Paraboloid die erwünschte Eigenschaft aufweist.
% Daher ist anzunehmen, dass N.~??Y\textsubscript{9} nach N.~??Y\textsubscript{4} enstand.
Der Terminus post quem ergibt sich jedoch vielmehr aus der soeben erwähnten Abhängigkeit, die N.~14\textsubscript{9} gegenüber N.~14\textsubscript{8} aufweist.
Ausschlaggebend für die Bestimmung des Terminus ante quem ist indes, dass von N.~14\textsubscript{9} das Teilkonzept N.~14\textsubscript{6},~\textit{L\textsuperscript{3}} herrührt, welches seinerseits als Vorlage für die Anfertigung der \textit{Additio} diente, die den Erstdruck N.~14\textsubscript{6},~\textit{E\textsuperscript{1}} abschließt (S.~\refpassage{AE_1684_325_additio-1}{AE_1684_325_additio-2}).
Demzufolge müssen die Aufzeichnung 
N.~14\textsubscript{9} und das von ihr abhängige Teilkonzept \textit{L\textsuperscript{3}} vorgelegen haben, als Leibniz vor dem Sommer 1684 darum Sorge trug, dass eine Druckvorlage für N.~14\textsubscript{6},~\textit{E\textsuperscript{1}} angefertigt wurde.
\pend%
%%%% >>>>
%%%%
\pstart%
Anhand der überarbeiteten bzw. neu angefertigten Konzepte N.~14\textsubscript{6},~\textit{L\textsuperscript{1}},~\textit{L\textsuperscript{2}} und~\textit{L\textsuperscript{3}} wurde noch vor dem Sommer 1684 zumindest eine Reinschrift abgefasst, die als Vorlage für den Erstdruck N.~14\textsubscript{6},~\textit{E\textsuperscript{1}} diente.
Da keine solche Reinschrift nach heutigem Wissensstand überliefert ist, ist nicht zu ermitteln, ob (nur) Leibniz an deren Fertigstellung beteiligt war oder (auch) ein Schreiber.
Dass es aber (mindestens) eine solche Reinschrift gegeben haben muss, lässt sich daraus schließen, dass keines der drei Konzepte den gesamten Text von \textit{E\textsuperscript{1}}~überliefert:
\textit{L\textsuperscript{1}}~umfasst, wie bereits oben bemerkt, nur einige Teile davon;
\textit{L\textsuperscript{2}}~fehlen insbesondere die Ersatzpassage zum gleichmäßig widerstandsfähigen Balken % \textit{Est enim} \lbrack...\rbrack\ \textit{dare possum} 
(S.~\refpassage{AE_1684_323-324_erstzng-1}{AE_1684_323-324_erstzng-2}), die samt einigen Diagrammen nur in \textit{L\textsuperscript{1}}~überliefert ist;
beiden Teilkonzepten \textit{L\textsuperscript{1}}~und \textit{L\textsuperscript{2}}~fehlen das sonst nur in \textit{E\textsuperscript{1}}~überlieferte Diagramm \lbrack\textit{Fig.~3c}\rbrack\ (S.~\pageref{AE_1684_321_Fig_3c}) sowie die abschließende \textit{Additio} (S.~\refpassage{AE_1684_325_additio-1}{AE_1684_325_additio-2}) samt zugehörigem Diagramm, die beide nur im Teilkonzept \textit{L\textsuperscript{3}}~überliefert sind (siehe die Darstellung der Zeugen N.~14\textsubscript{6},~\textit{L\textsuperscript{1}} bis~~\textit{L\textsuperscript{3}}, S.~\refpassage{dnr_AE_1684_319-325_Ueberlieferung_csjhfj-1}{dnr_AE_1684_319-325_Ueberlieferung_csjhfj-2}).
Auch Textvarianten, die lediglich in \textit{E\textsuperscript{1}}~überliefert sind, zeigen, dass es zwischen den Teilkonzepten \textit{L\textsuperscript{1}},~\textit{L\textsuperscript{2}} und \textit{L\textsuperscript{3}}~einerseits und \textit{E\textsuperscript{1}}~andererseits noch (mindestens) eine weitere Textschicht gegeben haben muss.
\pend%.%.%.%.%.%.%.
\pstart%
Offenbar vor Anfang Juli 1684 wurde an die Herausgeber der \textit{Acta eruditorum}\cite{01023} in Leipzig die (nicht überlieferte) Reinschrift übersendet, auf deren Grundlage im Heft vom Juli 1684 der Druck N.~14\textsubscript{6},~\textit{E\textsuperscript{1}} mit der Überschrift \textit{Demonstrationes novae de resistentia solidorium} veröffentlicht wurde.
% Falls Leibniz in seinem Schreiben vom Februar 1684 an den Mitherausgeber C.~Pfautz\protect\index{Namensregister}{\textso{Pfautz} (Pfauzius), Christoph 1645\textendash1711} auf N.~??Y\textsubscript{6} anspielt (\textit{LSB} III,~4 N. 48,\cite{01341} S.~108.15), dann war die Druckvorlage zu jenem Zeitpunkt noch nicht nach Leipzig verschickt worden; die Sendung wird aber als baldig angekündigt.
O.~Mencke,\protect\index{Namensregister}{\textso{Mencke} (Menken, Menkenius, Menque), Otto 1644\textendash1707} Hauptherausgeber der Zeitschrift, stellt die Veröffentlichung % hingegen 
als erfolgt in seinem Brief an Leibniz vom 16. (26.) Juli 1684 dar (\cite{01258}\textit{LSB} I,~4 N.~391, S.~475.31\textendash32); 
frühere ausdrückliche Hinweise auf \textit{E\textsuperscript{1}} sind im Briefwechsel mit Mencke oder mit dem weiteren Herausgeber C.~Pfautz nicht auffindig zu machen.%
\protect\index{Namensregister}{\textso{Mencke} (Menken, Menkenius, Menque), Otto 1644\textendash1707}% 
\protect\index{Namensregister}{\textso{Pfautz} (Pfauzius), Christoph 1645\textendash1711}
Dass Mariottes\protect\index{Namensregister}{\textso{Mariotte}, Edme, Seigneur de Chazeuil ca. 1620-1684} Tod (12. Mai 1684) in der Gesamtüberlieferung von N.~14 unerwähnt bleibt, ist nicht datierungsrelevant, da Leibniz wohl erst Mitte Juni 1684, als~\textit{E\textsuperscript{1}} % der Aufsatz 
ohnehin im Druck gewesen sein muss, davon erfuhr (siehe die editorische Vorbemerkung zu N.~12, S.~\refpassage{MariottesTod_dafjh-1}{MariottesTod_dafjh-2}).%
\edlabel{DNDRS_Ueberarbeitung_scjvtx-2}
\pend%
\pstart%
Der Gesamtkomplex der ihrem Inhalt und ihrer Entstehung nach eng miteinander verbundenen Texte N.~14\textsubscript{1} bis 14\textsubscript{9} entstand somit in einer Zeitspanne, die sich von Ende Januar 1683 bis zur ersten Hälfte Juli 1684 erstreckte.
Die Texte N.~14\textsubscript{1} bis 14\textsubscript{5} entstanden bis einschließlich März/April 1683, N.~14\textsubscript{6} bis 14\textsubscript{9} zwischen März/April 1683 und dem Frühsommer 1684.
Diesem Textkomplex würde man allerdings unvollständig Rechnung tragen, wenn folgender späterer Nachtrag unberücksichtigt bliebe.%
% \textbf{??? erzwungener Seitenumbruch (sonst Instabilität) ???}
\pend%
% \newpage%
%%%
\pstart%
In den \cite{01247}\title{Indices generales auctorum et rerum primi Actorum eruditorum quae Lipsiae publicantur decennii} (Leipzig 1693, Bd.~I, S.~*Qq2~r\textsuperscript{o}) finden sich drei Berichtigungen zu N.~14\textsubscript{6} \textit{Demonstrationes novae de resistentia solidorum} samt einer anonymen Anmerkung, die ebenfalls auf N.~14\textsubscript{6} (insbesondere % S.~\refpassage{AE_1684_319-325_a1}{AE_1684_319-325_a2} sowie 
S.~\refpassage{AE_1684_319-325_a3}{AE_1684_319-325_a4}) Bezug nimmt.
Da das Vorwort zu den \textit{Indices generales} auf Juni 1693 datiert ist, lässt sich die Entstehungszeit der Einträge wohl auf die erste Hälfte desselben Jahres eingrenzen.
Trotz fehlender handschriftlicher Zeugen ist anzunehmen, dass sowohl die drei Berichtigungen als auch die Anmerkung auf Leibniz selbst zurückgehen, weshalb sie sämtlich als N.~14\textsubscript{10} ediert werden.
\pend%
%\newpage
\pstart%
Bereits Mencke\protect\index{Namensregister}{\textso{Mencke} (Menken, Menkenius, Menque), Otto 1644\textendash1707} hatte in dem oben erwähnten Brief vom Juli 1684 auf mögliche Fehler in der Druckfassung von N.~14\textsubscript{6} % der \textit{Demonstrationes novae} 
hingewiesen und Leibniz aufgefordert, eventuelle Verbesserungswünsche mitzuteilen (\textit{LSB} I,~4 N.~391, S.~475.32\textendash476.3).\cite{01258}
In einem verworfenen Konzept seines wohl in der ersten Oktoberhälfte 1684 verfassten Briefes für die Herausgeber der \textit{Acta eruditorum} trug Leibniz dann tatsächlich Korrigenda zu N.~14\textsubscript{6}, \textit{E\textsuperscript{1}} % den \textit{Demonstrationes novae} 
zusammen, welche sämtliche in N.~14\textsubscript{10} aufgelisteten Fälle umfassten (\cite{01257}\textit{LSB} III,~4 N.~72, S.~181.26\textendash33; zwei dieser Verbesserungen hatten bereits die Herausgeber der Zeitschrift vorweggenommen, vgl. \textit{AE}, September 1684, S.~438; eine zusätzliche Verbesserung wird in N.~14\textsubscript{10} nicht berücksichtigt).
Wann und wie Leibniz seine Liste an die Herausgeber der \textit{Acta eruditorum} zukommen ließ, ist nicht bekannt.
Seine in \cite{01257}\textit{LSB} III,~4 N.~72 aufgelisteten Verbesserungswünsche zeigen aber, dass die in N.~14\textsubscript{10} edierten Berichtigungen Leibnizens Absicht durchaus entsprechen.
(Es gilt auch zu bemerken, dass die dort ausgewiesenen Druckfehler \textit{nicht in allen} erhaltenen Heften der \textit{Acta eruditorum} vom Juli 1684 anzutreffen sind.)
\pend%
\pstart%
Auch\edlabel{DNRS_intro_abstandnahme-1} für die in den \textit{Indices generales} veröffentlichte anonyme Anmerkung zu den \textit{Demonstrationes novae de resistentia solidorum}
(N.~14\textsubscript{10}, S.~\refpassage{AE_1693_Indices_Qq2r_annotandum-1}{AE_1693_Indices_Qq2r_annotandum-2}) lassen sich Belege ausfindig machen, die Leibnizens Urheberschaft nachweisen.
Jacob Bernoulli\protect\index{Namensregister}{\textso{Bernoulli}, Jacob 1655\textendash1705} hatte in seinem Brief an Leibniz vom 15. (25.) Dezember 1687 die in N.~14\textsubscript{6}, \textit{E\textsuperscript{1}} % den \textit{Demonstrationes novae} 
zugrundegelegte Hypothese angezweifelt, dass die Dehnung eines biegsamen Balkens sich proportional zur angewandten Spannkraft verhalte;
zudem hatte er empirische Messwerte angeführt, die der Annahme einer direkten Proportionalität zuwiderliefen (\cite{01259}\textit{LSB} III,~4 N.~200, S.~366.1\textendash20). 
In seiner Erwiderung vom 24. September (4. Oktober) 1690 beteuert Leibniz, dass der in N.~14\textsubscript{6}, \textit{E\textsuperscript{1}} % den \textit{Demonstrationes novae} 
angeführte Beweis über die Gestalt des gleichmäßig widerstandsfähigen Balkens seine Richtigkeit unabhängig von der angezweifelten Proportionalitätshypothese behalte
(\cite{01260}\textit{LSB} III,~4 N.~279, S.~574.14\textendash575.18).
Dies ist auch der Inhalt der in N.~14\textsubscript{10} edierten % den \textit{Indices generales} veröffentlichten 
anonymen Anmerkung.
Dieselbe Ansicht äußert Leibniz ferner, als er in seinem Brief vom 26. Oktober (5. November) 1690 an R.\,C. von Bodenhausen\protect\index{Namensregister}{\textso{Bodenhausen} (Bodenus, Bodenausen) Rudolf Christian, Freiherr v. 1698} über Bernoullis Kritik berichtet;\protect\index{Namensregister}{\textso{Bernoulli}, Jacob 1655\textendash1705}
dort stellt er erneut fest, dass, wenngleich man daran zweifeln könne, \textit{ob die tensiones chordarum oder fibrarum seyen ut vires tendentes, (welche Hypothesis nicht allerdings gewiss)} \lbrack...\rbrack\ \textit{meine demonstrationes de figuris aequiresistentibus doch wahr bleiben} (\cite{01261}\textit{LSB} III,~4 N.~285, S.~628.1\textendash6).%
\edlabel{DNRS_intro_abstandnahme-2}
\pend%
\pstart%
Diese spätere Entwicklung lässt die Vermutung zu, dass die nachträglich hinzugefügte, selbstkritische Randbemerkung zum Entwurf N.~14\textsubscript{7}, S.~\refpassage{LH_35_09_16_002_erstzng-1}{LH_35_09_16_002_erstzng-2}, erst nach dem Austausch mit Bernoulli im Oktober 1690 ergänzt wurde.
\pend%
% 
% 
% \newpage%      Rein vorläufig !!!!!
\normalsize
%
\selectlanguage{latin}%
\frenchspacing%
%
%
% ENDE DER EINLEITUNG 