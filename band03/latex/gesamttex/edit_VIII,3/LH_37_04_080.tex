%   % !TEX root = ../../VIII,3_Rahmen-TeX_8-1.tex
%
%
%   Band VIII, 3 N.~??A36
%   Signatur/Tex-Datei: LH_37_04_080
%   RK-Nr. 60241
%   Überschrift: [Utrum in animalibus omnia possint fieri beneficio elateriorum mechanicorum]
%   Modul: Mechanik / Elastizität
%   Datierung: [1677 bis Januar 1680 (?)]
%   WZ: (keins)
%   SZ: (keins)
%   Bilddateien (PDF): (keins)
%
%
\selectlanguage{ngerman}%
\frenchspacing%
\begin{ledgroupsized}[r]{120mm}
\footnotesize
\pstart
\noindent\textbf{Überlieferung:}
\pend
\end{ledgroupsized}
\begin{ledgroupsized}[r]{114mm}
\footnotesize
\pstart \parindent -6mm
\makebox[6mm][l]{\textit{L}}%
Notiz:
LH~XXXVII~4 Bl.~80.
Ein Zettel (8,5 x 4 cm).
Eine Seite auf Bl.~80~r\textsuperscript{o}.
Auf Bl.~80~v\textsuperscript{o} eine Rechnung von Leibnizens Hand aus unbekanntem Zusammenhang und ohne erkennbare Verbindung mit dem Text auf der Vorderseite:
\newline%
%
% \makebox[6mm][l]{}
1 \qquad 2 \qquad 3 \qquad 4 \quad 5 \quad 6 \quad 7 \quad 8 \quad 9 \quad 10\phantom{$\dfrac{1}{1}$}
\newline%
%
% \makebox[6mm][l]{}
$\dfrac{\overline{2.3 + 2.(6)}-\overline{4+(6)+2+(6)+(7)}}{3+(6)-4+(6)}$ seu $\dfrac{3+(6)-3+(6)+(7)}{0-1}$
\pend
\end{ledgroupsized}
%
%\normalsize
\vspace{5mm}%
\begin{ledgroup}%
\footnotesize%
\pstart%
\noindent%
\textbf{Datierungsgründe:}
Die vorliegende Notiz N.~5 besteht vorwiegend aus einem kurzen, aber sorgfältigen Zitat aus den \glqq Sechsten Einwänden\grqq\ gegen Descartes'%
\protect\index{Namensregister}{\textso{Descartes} (Cartesius, des Cartes), René 1596\textendash1650}
\textit{Meditationes de prima philosophia}\cite{01999} (Paris 1641), welches sich auf die vom französischen Philosophen vertretene Auffassung der Lebewesen als mechanische Automaten bezieht. %
Obwohl Leibniz bereits 1671 in Mainz\protect\index{Ortsregister}{Mainz} die zweite Ausgabe von Descartes' \textit{Opera philosophica}\cite{01300} (Amsterdam 1650) erworben hatte (vgl. \textit{LSB} VI,~3 N.~15, S.~213.13\textendash18\cite{01302}), widmete er sich anscheinend erst in den frühen Hannoveraner\protect\index{Ortsregister}{Hannover} Jahren einer gründlichen Lektüre der \textit{Meditationes}\cite{01999}, als er sich mit Themen der kartesischen Metaphysik und Naturphilosophie kritisch auseinandersetzte.
Seine Anstreichungen und Randbemerkungen zu den \textit{Meditationes}\cite{01999} in seinem Handexemplar von Descartes' \textit{Opera philosophica}\cite{01300} entstanden wahrscheinlich zwischen 1677 und 1687 (\textit{LSB} VI,~4 N.~335, S.~1699\textendash1703\cite{01301}), während seine kommentierten Auszüge aus Descartes' Werken, die zum Teil auch die \textit{Meditationes}\cite{01999} betreffen, auf die Zeitspanne vom Sommer 1678 bis zum Winter 1680/81 zu datieren sind (\textit{LSB} VI,~4 N.~341, S.~1785\textendash1788\cite{01303}).
Auch in Aufzeichnungen und Entwürfen, die insgesamt aus den Jahren 1678 bis 1684/85 stammen und auf Thesen der kartesischen Metaphysik (vornehmlich den ontologischen Gottesbeweis und die voluntaristische Gottesauffassung) eingehen, verdichten sich Bezüge und Anspielungen auf die \textit{Meditationes}\cite{01999} (vgl. etwa \textit{LSB} VI,~4 N.~110\cite{01304}; N.~264\cite{01305}; N.~272\cite{01308}; N.~283\cite{01306}; N.~288;\cite{01317} N.~289\cite{01307}; die zu demselben Zusammenhang gehörige Notiz N.~287\cite{01318} befasst sich ausgesprochen mit Descartes' mechanistischer Auffassung der Tiere).
Ebenso in Briefen aus den späten Siebziger Jahren \textendash\ unter anderen an H.~Fabri, A.~Eckhard, die Pfalzgräfin Elisabeth und N.~Malebranche \textendash\ knüpft Leibniz mehrfach an Themen und Texte aus Descartes' \textit{Meditationes}\cite{01999} an, zumeist in Verbindung mit den soeben genannten metaphysischen Thesen (siehe etwa \textit{LSB} II,~1 \lbrack2006\rbrack\ N.~133, S.~462\textendash466;\cite{01309} N.~138;\cite{01310} N.~219;\cite{01311} N.~143;\cite{01312} N.~148;\cite{01313} N.~187b;\cite{01314} N.~207, S.~721\textendash723\cite{01315}).
Besonders erwähnenswert ist in dieser Hinsicht der Brief an C.~Philipp vom Ende Januar 1680,%
\protect\index{Namensregister}{\textso{Philipp} (Philippi), Christian 1639\textendash1682}
in dem Leibniz bei seiner Zurückweisung von Descartes' voluntaristischer Gottesauffassung ausführlich und zum Teil wörtlich aus den \glqq Sechsten Erwiderungen\grqq\ zitiert (siehe \textit{LSB} II,~1 \lbrack2006\rbrack\ N.~222, S.~787.24\textendash788.23\cite{01316}). % vgl. R.~\textsc{Descartes}, \textit{Meditationes}, responsiones VI, §~6 u.~8; 
Spätestens zu diesem Zeitpunkt dürfte er Gelegenheit gehabt haben, die Passage aus den \glqq Sechsten Einwänden\grqq\ zu lesen, von der die Notiz N.~5 unmittelbar herrührt.
Diese sollte demnach zwischen 1677 und Ende Januar 1680 verfasst worden sein.
Eine spätere Datierung (bis etwa 1687) ist jedoch nicht auszuschließen.%
\protect\index{Namensregister}{\textso{Descartes} (Cartesius, des Cartes), René 1596\textendash1650}
\pend
\end{ledgroup}
\selectlanguage{latin}%
\frenchspacing%
%
% \vspace{8mm}
\newpage%
%
\count\Bfootins=1200
\count\Afootins=1200
\count\Cfootins=1200
%
%
\pstart
% \vspace*{0.5em}% PR: nur provisorisch
\noindent
\lbrack80~r\textsuperscript{o}\rbrack%
\textso{ Elaterii }\protect\index{Sachverzeichnis}{elaterium}%
vox pro eo
quod Germanis\protect\index{Sachverzeichnis}{Germani} vocatur Feder,\protect\index{Sachverzeichnis}{Feder}
Italis\protect\index{Sachverzeichnis}{Itali} Molla,\protect\index{Sachverzeichnis}{molla}
Gallis\protect\index{Sachverzeichnis}{Galli} ressort,\protect\index{Sachverzeichnis}{ressort}
%
\edtext{}{%
{\xxref{LH37_04_080r_1}{LH37_04_080r_2}}%
{\lemma{extat \lbrack...\rbrack\ \textit{Mechanicorum}}\Cfootnote{%
Siehe \textsc{R.~Descartes}, \textit{Meditationes de prima philosophia}, Objectiones sextae, Paris 1641, S.~555\cite{01999}
(\textit{DO}~VII, S.~414.18\textendash19).\cite{00120}
\selectlanguage{german}{Die \glqq Sechsten Einwände\grqq\ wurden anonym % von unter\-schied\-li\-chen Autoren 
verfasst, von M.~Mersenne\protect\index{Namensregister}{\textso{Mersenne} (Mersennus), Marin 1588\textendash1648} gesammelt und an Descartes\protect\index{Namensregister}{\textso{Descartes} (Cartesius, des Cartes), René 1596\textendash1650} weitergeleitet.}}}}%
%
\edlabel{LH37_04_080r_1}extat apud autorem sextarum
in Cartesii\protect\index{Namensregister}{\textso{Descartes} (Cartesius, des Cartes), René 1596\textendash1650}
\textit{Meditationes} objectionum sub finem scrupuli\protect\index{Sachverzeichnis}{scrupulus} tertii,
utrum scilicet in animalibus\protect\index{Sachverzeichnis}{animal}
\textit{omnia possint fieri beneficio Elateriorum Mechanicorum.}\edlabel{LH37_04_080r_2}%
\protect\index{Sachverzeichnis}{elaterium mechanicum}
% [Meditationes, Ausgabe 1685, S. 150, anonym (gesammelt von Mersenne)]
%
\pend
% Ende des Stückes