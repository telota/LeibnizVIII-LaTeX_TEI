%
%
%   Band VIII, 3 N.~??A10 (??A10.1 bis ??A10.6)
%   Signatur/Tex-Datei: LH_35_09_15_006-017_intro
%   Einleitung zu RK-Nr. 41152 [Teile 1–6]
%   Überschrift: Tentaminum de chordarum tensione schedae
%   Datierung: [Dezember 1681]
%   WZ: ()
%   SZ: ()
%   Bilddateien (PDF): ()
%
%
\footnotesize 
\pstart
\noindent                
Das vorliegende Gesamtstück besteht aus den Einzeltexten N.~8\textsubscript{1} bis 8\textsubscript{6}.
Auf ihren inhaltlichen und chronologischen Zusammenhang weist bereits ihre Überschrift hin:
Sämtliche Texte sind \textit{Tentaminum de chordarum tensione schedae} benannt und von eins bis sechs nummeriert,
wobei jede \textit{scheda} einen Bogen umfasst.
Die Textträger sind zudem von Leibniz selbst auf den 10. (20.) Dezember 1680 datiert;
nur auf dem Bogen von N.~8\textsubscript{4} hat er \glqq Dezember 1680\grqq\ als Datum vermerkt.
Als Gesamtdatierung von N.~8 wird demgemäß Dezember 1680 angegeben.
\pend%
\pstart%
Die sechs \textit{Schedae}, in denen Leibniz mechanische Gesetze der Schwingungen gespannter elastischer Saiten zu bestimmen sucht,
weisen enge Beziehungen zu weiteren Stücken der vorliegenden Rubrik auf, insbesondere N.~8 und  N.~10. 
Berührungspunkte bestehen zudem mit Theoremen aus
\cite{00044}H.~\textsc{Fabri}, \textit{Physica},
tract. I, lib. II: \textit{De com\-pres\-so et tenso} (Bd.~I, Lyon 1669) % , S.~42\textendash189 
und tract. III, lib. II: \textit{De sonis} (Bd.~II, Lyon 1670). % , S.~131\textendash293
Bemerkenswert ist in dieser Hinsicht die Anknüpfung an Fabris Abhandlung % tract.~III, lib.~II, prop.~223 und 224 (Bd.~II, S.~215b; 216a; 216b) 
in
N.~8\textsubscript{1}, S.~\refpassage{LH_35_09_15_007v_erzt-1}{LH_35_09_15_007v_erzt-2};
N.~8\textsubscript{2}, S.~\refpassage{LH_35_09_15_009v_crassitiessuperflua-1}{LH_35_09_15_009v_crassitiessuperflua-2};
N~8\textsubscript{4}, S.~\refpassage{LH_35_09_15_012v_crassitiessuperflua-1}{LH_35_09_15_012v_crassitiessuperflua-2};
N.~8\textsubscript{6}, S.~\refpassage{LH_35_09_15_016r_crassitiessuperflua-1}{LH_35_09_15_016r_crassitiessuperflua-2}. 
\pend
\newpage % vorläufig
\normalsize