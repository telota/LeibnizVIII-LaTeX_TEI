%   ##
%
%   Band VIII, 3 N.~?? X.3 (ex: N.~??A05 / ??X.5 / ??X.4)
%   Signatur/Tex-Datei: LH_37_01_014
%   RK-Nr. 38533 
%   Überschrift: Aus Günther Christoph Schelhammer, De auditu
%   Datierung: [Frühjahr 1684 -- erste Hälfte 1685]
%   WZ: Bl. 14: RK-Wz 304 (insgesamt eins, fragmentarisch)
%   SZ: (keins)
%   Bilddateien (PDF): (keine)
%
%
\begin{ledgroupsized}[r]{120mm}
\footnotesize
\pstart
\noindent\textbf{Überlieferung:}
\pend
\end{ledgroupsized}
\begin{ledgroupsized}[r]{114mm}
\footnotesize
\pstart \parindent -6mm
\makebox[6mm][l]{\textit{L}}%
Auszüge aus \cite{01204}\textsc{G.\,C. Schelhammer}, \textit{De auditu}, Leiden 1684, S.~124\textendash129:
LH~XXXVII~1 Bl.~14.
Ein Blatt 4\textsuperscript{o};
Fragment eines Wasserzeichens.
Zwei voll beschriebene Seiten.
Leibniz verweist in N.~12\textsubscript{3} (S.~\refpassage{LH_37_01_007v_y1}{LH_37_01_007v_y2}; \refpassage{LH_37_01_007v_w1}{LH_37_01_007v_w2}; \refpassage{LH_37_01_008v_novissime-1}{LH_37_01_008v_novissime-2}) auf Schelhammers Abhandlung und nimmt insbesondere auf die in N.~12\textsubscript{4} exzerpierten Stellen Bezug.
\pend
\end{ledgroupsized}
%
%
\vspace*{4mm}%
\count\Bfootins=1100
\count\Afootins=1000
\count\Cfootins=1000
\pstart%
\normalsize%
\noindent%
%
\lbrack14~r\textsuperscript{o}\rbrack\ % Blatt 14r
%
\pend%
%
% Überschrift
%
\pstart%
\centering%
Guntherus Christophorus Schelhammerus%
\protect\index{Namensregister}{\textso{Schelhammer} (Schelhammerus), Günther Christoph 1649\textendash1716} \textit{de auditu},\\
edito Lugd. Batavorum 1684 8\textsuperscript{o}.\cite{01204}\protect\index{Sachverzeichnis}{auditus}
\pend%
\vspace{0.5em}% 
%
\pstart%
\noindent%
\edtext{Is
\edtext{axiomate\protect\index{Sachverzeichnis}{axioma} 13}{%
\lemma{axiomate 13\,}\Cfootnote{%
\glqq Sonus\protect\index{Sachverzeichnis}{sonus} quousque permeat,
omnem ambientem\protect\index{Sachverzeichnis}{aer ambiens} aerem occupat.\grqq\
\cite{01204}\textsc{G.\,C. Schelhammer}, \textit{De auditu}, Leiden 1684, S.~124.
Der exzerpierte Abschnitt folgt unmittelbar der Formulierung des Axioms.}}
pag. 124}{%
\lemma{Is}\Bfootnote{%
\textit{(1)}~pag. 124
\textit{(2)}~axiomate 13 pag. 124%
~\textit{L}}}
ita habet\lbrack:\rbrack%
\pend%
% \vspace*{1.0em}%%
\pstart%
% \noindent%
\edtext{\textit{%
Problema:\protect\index{Sachverzeichnis}{problema}
quomodo sonus per aerem propagetur explicare.\protect\index{Sachverzeichnis}{propagatio soni}}
% \pend%
%
% \pstart%
\textit{%
Ex praemissis hactenus colligere licet,
dum corpora duo vel actu\protect\index{Sachverzeichnis}{actus} vel potentia\protect\index{Sachverzeichnis}{potentia}
dura\protect\index{Sachverzeichnis}{corpus durum} ac solida\protect\index{Sachverzeichnis}{corpus solidum} colliduntur,
quae productio est soni,\protect\index{Sachverzeichnis}{productio soni}
tunc aerem ex necessitate commoveri}
et \textit{%
concuti, ob eam rationem qua fluida\protect\index{Sachverzeichnis}{fluidum} omnia id pati ostensum est,
\edtext{Axiomate 4.}{%
\lemma{Axiomate 4\,}\Cfootnote{%
\glqq Corpus fluidum\protect\index{Sachverzeichnis}{corpus fluidum} compressum
solidi trementis\protect\index{Sachverzeichnis}{corpus tremens} rationem\protect\index{Sachverzeichnis}{ratio} habet.\grqq\
\cite{01204}a.a.O.% \textsc{Schelhammer}, \textit{De auditu}, Leiden 1684
, S.~110.}}
Cum vero aer\protect\index{Sachverzeichnis}{aer} sit corpus fluidum compressum\protect\index{Sachverzeichnis}{corpus compressum}
indeque elastro\protect\index{Sachverzeichnis}{elastrum} praeditum, ac in tremorem\protect\index{Sachverzeichnis}{tremor} agi aptum,
patet non alio modo id fieri,
quam quia impressam\protect\index{Sachverzeichnis}{species impressa} semel speciem soni\protect\index{Sachverzeichnis}{species soni}
una particula\protect\index{Sachverzeichnis}{particula aeris} in aliam decurrente totoque aere contremente
per se ipsum idem propaget,
ut solida corpora idem facere,
\edtext{axiomate\protect\index{Sachverzeichnis}{axioma} 3.}{%
\lemma{axiomate 3\,}\Cfootnote{%
\glqq Corpus solidum\protect\index{Sachverzeichnis}{corpus solidum} in tremorem\protect\index{Sachverzeichnis}{tremor} agi aptum,
sonum impressum\protect\index{Sachverzeichnis}{sonus impressus} per se ipsum totum transmittit.\grqq\
\cite{01204}a.a.O.% \textsc{Schelhammer}, \textit{De auditu}, Leiden 1684
, S.~110.}}
docuimus.
Non secus enim}}{%
\lemma{\textit{Problema} \lbrack...\rbrack\ \textit{enim}}\Cfootnote{%
\cite{01204}a.a.O.% \textsc{Schelhammer}, \textit{De auditu}, Leiden 1684
, S.~124.}}
\edtext{}{{\xxref{KZeitz157}{KZeitz158}}%
{
\lemma{\textit{ac videmus} \lbrack...\rbrack\ \textit{addensatum}}\Cfootnote{%
\cite{01204}a.a.O.% \textsc{Schelhammer}, \textit{De auditu}, Leiden 1684
, S.~125. Zitat mit Auslassung.}}}%
\edlabel{KZeitz157}\textit{%
ac videmus chordam\protect\index{Sachverzeichnis}{chorda} commoveri
ac reciprocis ictibus\protect\index{Sachverzeichnis}{ictus} huc illuc decurrere}%
\lbrack,\rbrack\
\textit{%
aer quoque commoveri totus videtur,
hinc una aeris particula\protect\index{Sachverzeichnis}{particula aeris} alteri
impressam speciem\protect\index{Sachverzeichnis}{species impressa} perpetuo communicat,
donec ob reactionem particularum vis\protect\index{Sachverzeichnis}{vis} tandem ac motus omnis elanguescat,
et sonus propagari desinat.\protect\index{Sachverzeichnis}{propagatio soni}
\edlabel{LH_37_01_014r_Leibnizerwaehnt-1}%
Hoc autem inventum non mihi soli deberi fateor
sed magna ex parte viro ingeniosissimo ac varia doctrina atque eruditione instructissimo\textso{ J. Leibnizio}%
\protect\index{Namensregister}{\textso{Leibniz} (Leibnitius), Gottfried Wilhelm 1646\textendash1716}\textso{ }%
Seren. Ducis Luneb. et Brunsv. Hannoverani\protect\index{Namensregister}{\textso{Braunschweig-L{\"u}neburg}, Ernst August von, Herzog und Kurfürst von Hannover, 1680\textendash1698}
hoc tempore} \makebox[1.0\textwidth][s]{\textit{Consiliario
qui pro suo in nos affectu Lutetiae Parisiorum\protect\index{Ortsregister}{Paris} dudum nato}%
\lbrack,\rbrack\
\edtext{}{{\xxref{KZeitz159}{KZeitz160}}%
{%
\lemma{\textit{cum audiret} \lbrack...\rbrack\ \textit{esse}}\Cfootnote{%
Siehe G.\,C. \textsc{Schelhammer}, Briefe an G.\,W. Leibniz vom 8. (18.) November und vom 31. Dezember 1680 (10. Januar 1681; \textit{LSB} III,~3 N.~124;\cite{01279} 153\cite{01280}).}}}%
\edlabel{KZeitz159}\textit{cum audiret de}}
\pend
\newpage
\pstart
\noindent
\textit{auditu mihi dissertationem sub manibus esse,}\edlabel{KZeitz160}
%
\edtext{\textit{per literas mecum communicare non dubitavit,
et de corporum tremore\protect\index{Sachverzeichnis}{tremor} quaedam egregia monere.}}{%
\lemma{\textit{per literas} \lbrack...\rbrack\ \textit{monere}}\Cfootnote{%
Siehe G.\,W. \textsc{Leibniz}, Briefe an G.\,C. Schelhammer, 6. (16.) Dezember 1680; Februar/März 1681; 13. (23.) Januar 1682 (\textit{LSB} III,~3 N.~139;\cite{01275} 182;\cite{01194} 311\cite{01195}).}}\edlabel{LH_37_01_014r_Leibnizerwaehnt-2}
%
\textit{Subolfactum id autem jam olim fuit eruditis quibusdam et emunctae naris viris,
quamvis non ita distinctim, sed per transennam, quod dicitur}%
\lbrack,\rbrack\
\textit{rem perceperint.
Prae caeteris vero summo Philosopho Medico atque poetae Hieronymo Fracastorio,%
\protect\index{Namensregister}{\textso{Fracastoro} (Fracastorius), Girolamo 1478\textendash1553}
qui elegantissimis verbis id exposuit Lib. de sympathia et antipathia cap.~4.}
\edtext{}{\xxref{LH_37_01_014_a1}{LH_37_01_014_a2}{%
\lemma{\textit{Species} \lbrack...\rbrack\ \textit{densantibus}}\Cfootnote{%
Wörtlich zitiert nach G.\,\textsc{Fracastoro}, \textit{De sympathia}, cap.~4 % et antipathia rerum
(Venedig 1546, S.~3~r\textsuperscript{o}/v\textsuperscript{o};
\textit{Opera}~I, Lyon 1591, S.~9\,f.).\cite{01215}\cite{01216}}}}%
\edlabel{LH_37_01_014_a1}%
\textit{Species soni,\protect\index{Sachverzeichnis}{species soni} inquit,
si movere sensum\protect\index{Sachverzeichnis}{sensus} debet,
medium poscit continenter densum,\protect\index{Sachverzeichnis}{medium densum}
non per admistionem terrae\protect\index{Sachverzeichnis}{terra} sed vi addensatum,}\edlabel{KZeitz158}
\edtext{%
\textit{quod in aere\protect\index{Sachverzeichnis}{aer} accidit facto ictu.\protect\index{Sachverzeichnis}{ictus}
Inde enim facta prius distractione\protect\index{Sachverzeichnis}{distractio} et rarefactione\protect\index{Sachverzeichnis}{rarefactio}}%
\lbrack,\rbrack\
\textit{tum subita fit addensatio partis post partem more undarum,\protect\index{Sachverzeichnis}{unda}
unde circulationes conflantur.
Quod non aliud est,
quam successiva quaedam aeris addensatio\protect\index{Sachverzeichnis}{addensatio aeris}
in orbem\protect\index{Sachverzeichnis}{orbis} facta
per quam delata species\protect\index{Sachverzeichnis}{species propagata}}\,
%
\lbrack14~v\textsuperscript{o}\rbrack\ % Blatt 14v
%
\textit{a primo profecta sensum\protect\index{Sachverzeichnis}{sensus} dimovere potest.
Fit autem successiva illa addensatio in aere\protect\index{Sachverzeichnis}{addensatio aeris}}%
\lbrack,\rbrack\
\textit{prioribus quidem partibus subito ac vi\protect\index{Sachverzeichnis}{vis} densatis}%
\lbrack,\rbrack\
\textit{subito etiam}
se % \edtext{}{%
% \lemma{se}\Cfootnote{%
% In der Vorlage: \textit{sese}.}}
\textit{rarefacientibus ac alias successive densantibus.}%
\edlabel{LH_37_01_014_a2}%
% }{%
% \lemma{\textit{quod in} \lbrack...\rbrack\ \textit{densantibus}}\Cfootnote{%
% \cite{01204}\textsc{Schelhammer}, \textit{De auditu}% , Leiden 1684
% , S.~126.}}
\newline% \pend%
%
\indent% \pstart%
% \edtext{%
\textit{%
\protect\index{Sachverzeichnis}{paradoxon}\textso{Paradoxon.
Sonus per aerem propagatus a principio ad finem undique fertur pari velocitate sed viribus remittit. }%
\protect\index{Sachverzeichnis}{propagatio soni}\protect\index{Sachverzeichnis}{velocitas}\protect\index{Sachverzeichnis}{vis}%
\edlabel{LH_37_01_014v_z1}%
Nullus dubito hoc assertum tam a vero absonum plerisque videri, ut vix fidem habeant.
Cum actione\protect\index{Sachverzeichnis}{actio} et reactione\protect\index{Sachverzeichnis}{reactio}}
\edtext{partium}{%
\lemma{partium}\Cfootnote{%
\textit{particularum} in der Vorlage.}}
\textit{%
aeris\protect\index{Sachverzeichnis}{particula aeris} trementium,\protect\index{Sachverzeichnis}{aer tremens}
videatur omnis earum vis\protect\index{Sachverzeichnis}{vis} frangi debere
nec tantum imminui sonus,\protect\index{Sachverzeichnis}{sonus} sed tardius etiam}}{%
\lemma{\textit{quod in} \lbrack...\rbrack\ \textit{tardius etiam}}\Cfootnote{%
\cite{01204}\textsc{Schelhammer}, \textit{De auditu}, S.~126. Zitat mit Auslassung.}} % Leiden 1684, 
\edtext{}{{\xxref{KZeitz161}{KZeitz162}}%
{\lemma{\textit{ferri} \lbrack...\rbrack\ \textit{proportione}}\Cfootnote{%
\cite{01204}a.a.O.% \textsc{Schelhammer}, \textit{De auditu}, Leiden 1684
, S.~127. Zitat mit Auslassung.}}}%%
\edlabel{KZeitz161}\textit{%
ferri debere.
Et vero nos ipsi etiam in eo fuimus errore,
antequam a
\edtext{Mathematico consummatissimo et rerum naturalium apprime curioso,}{%
\lemma{Mathematico}\Bfootnote{%
\textit{(1)}~curiosissimo
\textit{(2)}~\textit{consummatissimo et} \lbrack...\rbrack\ \textit{apprime curioso,}% rerum naturalium
~\textit{L}}}
Paulo Heigelio\protect\index{Namensregister}{\textso{Heigel} (Heigelius), Paul 1640\textendash1690}
Collega Honoratissimo
dubium ea de re nobis moveretur.
Tunc enim experientia\protect\index{Sachverzeichnis}{experientia} consuluimus,
et eodem praesente comperti sumus
post mille passus\protect\index{Sachverzeichnis}{passus}
eadem celeritate\protect\index{Sachverzeichnis}{celeritas} procedere ulterius sonus,\protect\index{Sachverzeichnis}{sonus}
atque fecerat in ipso principio.
Experimentum\protect\index{Sachverzeichnis}{experimentum} tale fuit:
stationem\protect\index{Sachverzeichnis}{statio} eligimus in loco
\edtext{paulo eminentiore,}{%
\lemma{paulo}\Bfootnote{%
\hspace{-0,5mm}\textbar~paulo
\textit{streicht Hrsg. nach Vorlage}~%
\textbar\ \textit{eminentiore}~\textit{L}}}
ibique collocata bombarda\protect\index{Sachverzeichnis}{bombarda} majore
et relictis sociis\protect\index{Sachverzeichnis}{socius} explodendae bombardae
mensuravimus ab illo loco perticas 10.\protect\index{Sachverzeichnis}{pertica}
Hic pedem fiximus, et perpendiculo brevissimo,\protect\index{Sachverzeichnis}{perpendiculum}
ut citissime ac saepius posset recurrere}%
\lbrack,\rbrack\
\textit{instructi}\lbrack,\rbrack\
\textit{signum\protect\index{Sachverzeichnis}{signum} dedimus minore sclopeto,\protect\index{Sachverzeichnis}{sclopetum}
% [...]
et ut primum ignem\protect\index{Sachverzeichnis}{ignis} vidimus deflagrantem,
laxato perpendiculo computavimus seorsim quilibet
quot vicibus reciprocam decursionem perpendiculum\protect\index{Sachverzeichnis}{perpendiculum} absolveret
antequam} \makebox[1.0\textwidth][s]{\textit{ad aures\protect\index{Sachverzeichnis}{auris} allaberetur sonus.\protect\index{Sachverzeichnis}{sonus}
Hoc diligenter notato
rursus 10 perticis\protect\index{Sachverzeichnis}{pertica} emensis
signum dedi-}}
\pend
\newpage
\pstart
\noindent
\textit{mus et ut ante instituimus computationem,\protect\index{Sachverzeichnis}{computatio}
idque aliquoties fecimus,
donec internoscere signum,\protect\index{Sachverzeichnis}{signum}
vel ipsum etiam ignem\protect\index{Sachverzeichnis}{ignis} non amplius}
potuimus % \edtext{}{
% \lemma{potuimus}\Cfootnote{%
% In der Vorlage: \textit{potuerimus}.}}
\textit{idque constanter deprehendimus,
\edtext{non Geometrica sed Arithmetica proportione%
\protect\index{Sachverzeichnis}{proportio geometrica}\protect\index{Sachverzeichnis}{proportio arithmetica}
}{%
\lemma{non}\Bfootnote{%
\textit{(1)}~Arithmetica
\textit{(2)}~\textit{Geometrica sed Arithmetica}
\textit{(a)}~ratione
\textit{(b)}~\textit{proportione}%
~\textit{L}}}%
}\edlabel{KZeitz162}
\edtext{%
\textit{%
crescere numerum reciprocationis}\protect\index{Sachverzeichnis}{reciprocatio}
\edtext{penduli.\protect\index{Sachverzeichnis}{pendulum}}{%
\lemma{penduli}\Cfootnote{%
\textit{perpendiculi} in der Vorlage.}}
\textit{%
Interim sonus\protect\index{Sachverzeichnis}{sonus} ita imminuebatur,
ut percipi}
vix
\textit{%
tandem posset.\edlabel{LH_37_01_014v_z2}
% [...]
Ego vero inexpectato eventu turbatus}%
\lbrack,\rbrack\
\textit{%
de causa\protect\index{Sachverzeichnis}{causa} rei tam mirae cogitare sedulo cepi.
Et parum abfuit}
\edtext{\textit{quin}
subtilibus}{%
\lemma{\textit{quin}}\Bfootnote{%
\textit{(1)}~\textit{subtilius}
\textit{(2)}~subtilibus%
~\textit{L}}}
\textit{%
aere\protect\index{Sachverzeichnis}{aer} elementum\protect\index{Sachverzeichnis}{elementum}\protect\index{Sachverzeichnis}{aether}
% [interstitiis ejus ac]
poris\protect\index{Sachverzeichnis}{porus}}
ejus
\textit{%
haerere 
% [cum Cartesio]
crederem}.
% \edtext{\lbrack \textit{et per illud sonum statuerem propagari.}\protect\index{Sachverzeichnis}{propagatio soni}\rbrack}{\lemma{\textit{et per} \lbrack...\rbrack\ \textit{statuerem propagari.}}\Bfootnote{\textit{erg. Hrsg. nach Vorlage}}}
Sed quod
\textit{%
haec sententia perquam improbabilis semper mihi plurimis de causis antea visa}
esset % \edtext{}{
% \lemma{esset}\Cfootnote{%
% In der Vorlage: \textit{fuerit}.}}
\textit{%
% [nondum pedibus in eam ut irem, potui hoc unum persuaderi, sed altius expendere eadem induxi animum. Itaque]
ad stagnum\protect\index{Sachverzeichnis}{stagnum} spatiosum non admodum profundum me contuli,
ibique statutis signis\protect\index{Sachverzeichnis}{signum} eadem distantia ab invicem dissidentibus}%
\lbrack,\rbrack\
\textit{baculis et lapillis injectis,\protect\index{Sachverzeichnis}{lapillus injectus}
coepi ut antea momenta temporis\protect\index{Sachverzeichnis}{momentum temporis}
% [quo gyri ad extrema decurrunt,]
solicite connumerare.}
% [ibi vero nova observato se menti atque oculis obtulit.]
Et
\textit{%
vidi manifesto circulos\protect\index{Sachverzeichnis}{circulus aqueus} istos initio velocius}%
\lbrack,\rbrack\
\textit{%
remissius postremo loco decurrere per superficiem,
verum
\edtext{nihilominus}{%
\lemma{nihilominus}\Bfootnote{\textit{erg.~L}}}
non esse diversum tempus,\protect\index{Sachverzeichnis}{tempus}
quo ab uno spatio notato ad alterum devenirent illorum extremi.
Hoc autem fieri ideo quod
qui primus excitatur,
non}
ideo
\textit{%
primus permanet,
sed ubi parum processit,}}{%
\lemma{\textit{crescere} \lbrack...\rbrack\ \textit{processit}}\Cfootnote{%
\cite{01204}a.a.O.% \textsc{Schelhammer}, \textit{De auditu}, Leiden 1684
, S.~128. Zitat mit Auslassungen.}}
\edtext{%
\textit{%
alios atque alios justo spatio excitat}%
\lbrack,\rbrack\
\textit{%
hi rursus alios qui}
\edtext{\lbrack\textit{omnes}\rbrack}{\lemma{rursus}\Bfootnote{\textit{L~ändert Hrsg. nach Vorlage}}}
\textit{%
illum qui primo loco factus erat, praecurrunt longe.
Sicque hujus tarditatem\protect\index{Sachverzeichnis}{tarditas}
sua frequentia\protect\index{Sachverzeichnis}{frequentia} et multiplicatione pensatum eunt.
Ut pateat pati hic aquam\protect\index{Sachverzeichnis}{aqua} trementem
a reactione\protect\index{Sachverzeichnis}{reactio} proxime adstantis
et tamen eodem modo quo aer\protect\index{Sachverzeichnis}{aer}
aequali temporis\protect\index{Sachverzeichnis}{tempus} spatio
hos circulos\protect\index{Sachverzeichnis}{circulus aqueus} ad finem decurrere.%
% [..]
}}{%
\lemma{\textit{alios} \lbrack...\rbrack\ \textit{decurrere}}\Cfootnote{%
\cite{01204}a.a.O.% \textsc{Schelhammer}, \textit{De auditu}, Leiden 1684
, S.~129.}}%
\pend%
\count\Bfootins=1200
\count\Afootins=1200
\count\Cfootins=1200
%
%
% ENDE DES STÜCKES auf Blatt 14v
%
\newpage%