%   % !TEX root = ../../VIII,3_Rahmen-TeX_9-0.tex
%  
%   Band VIII, 3 N.~?? 	[STO??.?]			(Unter)rubrik			??
%   Signatur/Tex-Datei:	LH_37_05_143
%   RK-Nr. 		60343
%   Überschrift: 	(keine)
%   Datierung:		Wie 57267_2: [Mai 1677] (a. St.)
%   Textfolge: 			143r. Rückseite leer.
%   WZ: 				keins
%   edlabels:			6
%
%
%
%
\selectlanguage{ngerman}
\frenchspacing
%
\begin{ledgroupsized}[r]{120mm}
\footnotesize
\pstart
\noindent\textbf{Überlieferung:}
\pend
\end{ledgroupsized}
%
\begin{ledgroupsized}[r]{114mm}
\footnotesize
\pstart \parindent -6mm
\makebox[6mm][l]{\textit{L}}%
Reinschrift:
LH~XXXVII~5~Bl.~143. %ggf. (unsere Druckvorlage).
Ein Zettel (ca.~8,5~x~4,5~cm);
Papiererhaltungsmaßnahmen.
Eine Seite auf Bl.~143~r\textsuperscript{o}; Rückseite leer.
\pend
\end{ledgroupsized}
%
%
\selectlanguage{latin}
\frenchspacing
% \newpage%
\vspace{8mm}
\count\Bfootins=1300%
\count\Afootins=1200%
\count\Cfootins=1300
\pstart%
\normalsize%
\noindent%
%
\edlabel{37_05_143_1a}%
\edtext{}{% NEUER ABSATZ UND VARIANTEN – Anfang
{\xxref%
{37_05_143_1a}{37_05_143_1b}}%
\lemma{\lbrack143~r\textsuperscript{o}\rbrack}%
\Bfootnote{%
\textit{(1)} Si eadem servetur %
\textit{(2)} Ut eadem servaretur %
\textit{L}%
}}%
\lbrack143~r\textsuperscript{o}\rbrack\
\edlabel{37_05_143_4a}%
\edtext{}{% C-Footnote – "Reinschrift 1"
{\xxref%
{37_05_143_4a}{37_05_143_4b}}%
\lemma{Ut \lbrack...\rbrack\ in \textit{AC}}%
\Cfootnote{%
Vorlage für diese Passage ist N.~\ref{57267_2}, S.~\refpassage{37_05_144-145_25a}{37_05_144-145_25b}.%
}}%
%
Ut eadem servaretur%
\edlabel{37_05_143_1b}
%
quantitas motus\protect\index{Sachverzeichnis}{quantitas motus} deberet esse \textit{AE} in $BC +BE$ in \textit{AC}  
%
aequ.\ \textit{AD} in $BC+BD$ in \textit{AC}.%
\edlabel{37_05_143_4b}
%
%
\edlabel{37_05_143_5a}%
\edtext{}{% C-Footnote – "Reinschrift 2"
{\xxref%
{37_05_143_5a}{37_05_143_5b}}%
\lemma{Seu \lbrack...\rbrack\ $\protect\displaystyle\frac{\text{corp.}\ B}{\text{corp.}\ A}$}%
\Cfootnote{%
a.a.O., S.~\refpassage{37_05_144-145_26a}{37_05_144-145_26b}.%
}}%
%
Seu $ \underset{\displaystyle -AE}{+AD} \smallfrown BC \sqcap \underset{\displaystyle -BD}{+BE} \smallfrown AC$. 
\rule[0cm]{0mm}{12pt}Sive 
%
\pend
%
\rule[0cm]{40mm}{0cm}
\pstart
$\displaystyle\frac{AC}{BC} \sqcap \displaystyle\frac{DA-AE}{EB-BD} \sqcap \displaystyle\frac{\textup{corp.}\ B}{\textup{corp.}\ A}$.%
\edlabel{37_05_143_5b}
\pend
%
%
\pstart
\raggedleft
\begin{edtabularl}		
%
\edlabel{37_05_143_6a}%
\edtext{}{% C-Footnote – "Reinschrift 3"
{\xxref%
{37_05_143_6a}{37_05_143_6b}}%
\lemma{1\textsuperscript{mo} \lbrack...\rbrack\ falsum}%
\Cfootnote{%
a.a.O., S.~\refpassage{37_05_144-145_27a}{37_05_144-145_27b}. Leibnizens Bewertung von Huygens' Ergebnissen als falsch oder ungewiss geht, hier wie in der Vorlage, auf seine fehlerhafte Handhabung der Vorzeichen in der Vektoraddition zurück.%
}}%
%
\hfill
1\textsuperscript{mo} casu
&\edtext{$\displaystyle\frac{\lbrack AC\rbrack}{BC}$}{%
\lemma{}%
\Bfootnote{%
\textit{AB} %
\textit{L ändert Hrsg.}}}
&$\sqcap$
&$\displaystyle\frac{-DE}{-DE}$
&falsum %
\vphantom{\vrule height 20pt depth 0pt}
\hfill\hfill\hfill\hfill\hfill\hfill\\  
%
\hfill
2\textsuperscript{do} casu 
&\dots
&$\sqcap$
&$\displaystyle\frac{+DE}{-DE}$ 
&falsum %
\vphantom{\vrule height 20pt depth 0pt}
\hfill\hfill\hfill\hfill\hfill\hfill\\
%
\hfill
3\textsuperscript{tio}  
&\dots  
&$ \sqcap $
&$\displaystyle\frac{DA -AE}{DE}$ 
&\edlabel{37_05_143_3a}%	% C-Fn zu Destilletur
\edtext{}{%
{\xxref%
{37_05_143_3a}{37_05_143_3b}}%
\lemma{\Denarius}%
\Cfootnote{%
In N.~\ref{57267_2} lautet die Bewertung dieser drei Fälle: \glqq dubium\grqq.%
}}%
\Denarius%
\vphantom{\vrule height 20pt depth 0pt}
\hfill\hfill\hfill\hfill\hfill\hfill\\
%
\hfill
4\textsuperscript{to}  
&\dots  
&$\sqcap$
&$\displaystyle\frac{DA-AE}{EB-BD}$ 
&\Denarius %
\vphantom{\vrule height 20pt depth 0pt}
\hfill\hfill\hfill\hfill\hfill\hfill\\
%
\hfill
5\textsuperscript{to}
&\dots  
&$\sqcap$
&$\displaystyle\frac{D E}{EB-BD}$
&\Denarius%
\edlabel{37_05_143_3b}%
\vphantom{\vrule height 20pt depth 0pt}
\hfill\hfill\hfill\hfill\hfill\hfill\\
%
\hfill
6\protect\vphantom()
&\dots  
&$ \sqcap $
&$\displaystyle\frac{DA \phantom{\sqcap D A}}{ED \sqcap D A}$ 
&falsum %
\vphantom{\vrule height 20pt depth 0pt}
\hfill\hfill\hfill\hfill\hfill\hfill\\
%
\hfill
7\textsuperscript{mo}
&\dots  
&$\sqcap$
&\edlabel{37_05_143_2a}%
\edtext{}{% B-Footnote innerhalb der Tabelle
{\xxref%
{37_05_143_2a}{37_05_143_2b}}%
\lemma{$\displaystyle\frac{D A}{EB-BD}$}%
\Bfootnote{\textit{(1)}\ \Denarius\ %
\textit{(2)}\ falsum %
\textit{(3)}\ falsum %
\textit{L}\ }}%
$\displaystyle\frac{D A}{EB-BD}$
&falsum\edlabel{37_05_143_2b} %
\vphantom{\vrule height 20pt depth 0pt}
\hfill\hfill\hfill\hfill\hfill\hfill\\
%
\hfill
8\protect\vphantom()
&\dots  
&$\sqcap$
&$\displaystyle\frac{-DE \phantom{\sqcap -DE}}{-BD \sqcap -DE}$ 
&fals.%
\vphantom{\vrule height 20pt depth 0pt}
\hfill\hfill\hfill\hfill\hfill\hfill\\
%
\hfill
9\protect\vphantom()
&\dots  
&$\sqcap$
&$\displaystyle\frac{DE \phantom{\sqcap DE}}{EB \sqcap DE}$
&falsum. %
\vphantom{\vrule height 20pt depth 0pt}
\hfill\hfill\hfill\hfill\hfill\hfill\\
%
\hfill
10\protect\vphantom()
&\dots  
&$\sqcap$
&$\displaystyle\frac{DA - AE}{EB}$  
&falsum.%
\vphantom{\vrule height 20pt depth 0pt}
\hfill\hfill\hfill\hfill\hfill\hfill%
\edlabel{37_05_143_6b}
%
\end{edtabularl}
%
\pend
\count\Bfootins=1200%
\count\Afootins=1200%
\count\Cfootins=1200