%   % !TEX root = ../../VIII,3_Rahmen-TeX_8-1.tex
%
%
%   Band VIII, 3 N.~??A17 (= N.~??A17.1 + N.~??A17.2)
%   Signatur/Tex-Datei: LH_35_14_02_048 + LH_35_10_17_005-008 / Boccabadati_intro
%   RK-Nr. 58423 (= N.~??A17.1) + 60140 (= N.~??A17.2, L2) + 60141 (= N.~??A17.2, L1)
%   Überschrift: [Appendix de vi absoluta]
%   Datierung: [Januar bis April 1690]
%   WZ: LEd8-WZ 803012 & LEd8-WZ 803039 (insgesamt zwei)
%.  SZ: ()
%.  Bilddateien (PDF): ()
%
%
\selectlanguage{ngerman}%
\frenchspacing%
%
\footnotesize%
\pstart%
\noindent%
Während \label{LH_35_10_17_005-008_intro}%
seiner Italienreise\protect\index{Ortsregister}{Italien}
hielt sich Leibniz von Ende Dezember 1689 bis zum 2. Februar 1690 in Modena\protect\index{Ortsregister}{Modena} auf
(\cite{01236}\textit{Chronik}, S.~99\textendash101).
Bei dieser Gelegenheit lernte er unter anderen auch den Mathematiker G.\,B. Boccabadati % (1635\textendash1696)%
\protect\index{Namensregister}{\textso{Boccabadati} (Boccabadatus), Giovan Battista 1635\textendash1696} kennen,
welcher damals an einer
(nie veröffentlichten und heute verschollenen)
Abhandlung \cite{01237}\textit{De conatu mechanico} arbeitete.
Obwohl Boccabadati nie zu einem festen Korrespondenten Leibnizens wurde,
diskutierten beide Gelehrte Anfang 1690 über Fragen der Elastizität,
wie Leibniz selbst in einem Brief an B.~Ramazzini vom 25. Februar 1690 bezeugt:
\textit{In literis ad Dn. Boccabadatum quaedam attigi de Motus Tonici aestimatione}
(\cite{01238}\textit{LSB} III,~4 N.~239, S.~467.11\textendash12;
die von Leibniz erwähnten Briefe an Boccabadati sind nicht ermittelt).
Auch in einem verschollenen Brief an C.~Marchesini%
\protect\index{Namensregister}{\textso{Marchesini}, Camillo, gest. vor 17. Juni 1706}
vom Frühjahr 1691 soll Leibniz auf einen im vorherigen Jahre erfolgten Gedankenaustausch mit Boccabadati hingewiesen haben,
bei dem es sich um die Prinzipien der Mechanik
und insbesondere um Boccabadatis Absicht,
\textit{il tutto alla tensione e alla pressione} zurückzuführen,
gehandelt habe
(siehe Brief von Ramazzini%
\protect\index{Namensregister}{\textso{Ramazzini} (Ramazzinus), Bernardino 1633\textendash1714}
an A.~Magliabechi\protect\index{Namensregister}{\textso{Magliabechi} (Magliabecchi, Magliabecki, Magliabekius, Magliabequius), Antonio 1633\textendash1714} vom 3. Mai 1691,
in \cite{01239}B.~\textsc{Ramazzini}, \textit{Epistolario}, Modena 1964, S.~98).
\newline%
\indent%
Die Annahme liegt nahe,
dass die von der Spannungskraft elastischer Körper handelnde Notiz N.~31\textsubscript{1}
im Rahmen des Gedankenaustausches mit Boccabadati entstand,
und zwar vermutlich noch zu der Zeit, als Leibniz in Modena verweilte.
Der italienische Mathematiker wird im Text nämlich gleich zu Beginn erwähnt;
zudem liegt im Texträger von N.~31 dasselbe Wasserzeichen wie in Briefen vor,
die Leibniz während seines Aufenthaltes in Modena versandte:
etwa am 31. Dezember 1689 an Magliabechi,
am 1. Januar 1690 an Herzog Franz II. von Modena%
\protect\index{Namensregister}{\textso{Modena}, Franz II., Herzog von 1662\textendash1694}
und im Januar 1690 an Marchesini (\textit{LSB} I,~5 N.~275 bis N.~277).
\newline%
\indent%
In unmittelbarem Anschluss hieran entstand höchstwahrscheinlich auch das Konzept N.~31\textsubscript{2},~\textit{L\textsuperscript{1}}.
Denn der dort überlieferte Text ist offenbar eine Entfaltung desselben Gedankens,
der zunächst in der Notiz N.~31\textsubscript{1} aufgezeichnet worden war.
Ferner ist im Textträger von N.~31\textsubscript{2},~\textit{L\textsuperscript{1}} das gleiche Wasserzeichen anzutreffen
wie im Textträger von N.~31\textsubscript{1}.
In einer gestrichenen Variante von N.~31\textsubscript{2},~\textit{L\textsuperscript{1}} wird überdies auf einen der oben erwähnten,
verschollenen Briefe an Boccabadati verwiesen
(siehe den textkritischen Apparat zu S.~\refpassage{LH_35_10_17_005r_variant-1}{LH_35_10_17_005r_variant-2}).
Eine Datierung des Konzeptes N.~31\textsubscript{2}, \textit{L\textsuperscript{1}} auf Leibnizens Aufenthalt in Modena schlägt \textendash\ mit Hinweis auf den Austausch mit Boccabadati \textendash\ auch \cite{01240}A.~\textsc{Robinet}, \textit{Iter Italicum}, Florenz 1988, S.~346\,f. und S.~458, vor.
\newline%
\indent%
Die Reinschrift N.~31\textsubscript{2}, \textit{L}\textsuperscript{2}, die auf dem Konzept N.~31\textsubscript{2},~\textit{L\textsuperscript{1}} beruht, ist auf einem Papierbogen abgefasst, dessen Wasserzeichen das Wappen der Abtei zu Thierhaupten\protect\index{Ortsregister}{Thierhaupten} (bei Augsburg) % und ihres 1677 bis 1700 regierenden Abtes Sartorius\protect\index{Namensregister}{\textso{Sartorius} Abt von Thierhaupten, Regierungszeit 1633\textendash1714}
aufweist.
Leibniz hielt sich in Augsburg\protect\index{Ortsregister}{Augsburg} und Regensburg\protect\index{Ortsregister}{Regensburg} im April 1690 auf dem Weg von Venedig\protect\index{Ortsregister}{Venedig} nach Wien\protect\index{Ortsregister}{Wien} auf (\textit{Chronik}, S.~102\,f.).\cite{01236}
Die Reinschrift N.~31\textsubscript{2},~\textit{L}\textsuperscript{2} entstand aller Wahrscheinlichkeit nach zu dieser Zeit.
\pend
%
\newpage % vorläufig
\normalsize
\selectlanguage{latin}%
\frenchspacing%
%