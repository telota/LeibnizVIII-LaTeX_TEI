%   % !TEX root = ../../VIII,3_Rahmen-TeX_9-0.tex
%  
%   Band VIII, 3 N.~?? 	[XXX??.?]			Gerader Stoß 
%   Signatur/Tex-Datei:	LH_37_05_152v
%   RK-Nr. 	60344-2
%   Überschrift: 	(keine)
%   Titel: 			Rechnungen zur via centri gravitatis
%   Datierung:		???? bis ???? (a. St.?), eigh. (?)				??
%   Textfolge: 		Zusammenhang mit anderen Stücken klären??				??
%   WZ: 	Nr. 803004
%   edlabels:			13
%
%   Erstaufnahme:			(wer?)
%   Bearbeitung MS ab: 		Juni 2020
%
%   NB: 		Zusammenhang mit anderen Stücken klären	 ??
%
%
%
\selectlanguage{ngerman}
\frenchspacing
%
\begin{ledgroupsized}[r]{120mm}
\footnotesize
\pstart
\noindent\textbf{Überlieferung:}
\pend
\end{ledgroupsized}
%
\begin{ledgroupsized}[r]{114mm}
\footnotesize
\pstart \parindent -6mm
\makebox[6mm][l]{\textit{L}}%
Konzept:
LH~XXXVII~5~Bl.~152. 
Ein Blatt~2\textsuperscript{o},
nachträglich in 4\textsuperscript{o} gebrochen;
Wasserzeichen in der Blattmitte;
Papiererhaltungsmaßnahmen.
Eineinhalb Spalten auf Bl.~152~v\textsuperscript{o}, quer zweispaltig beschrieben;
Bl.~152~r\textsuperscript{o} überliefert N.~\ref{RK60344_1}.
\pend
\end{ledgroupsized}
%
\vspace{5mm}
\begin{ledgroup}
\footnotesize
\pstart
\noindent%
\textbf{Datierungsgründe:} %
Leibniz greift in N.~\ref{RK60344_2} eine in N.~\ref{RK57273} (Mai bis Mitte Juni 1677) hergeleitete Gleichung 
%
über die Quadrate der Geschwindigkeiten zweier Körper, in der anhand von \textit{signa ambigua} 
%
mehrere mögliche Stoßfälle berücksichtigt werden, wieder auf.
%
Außerdem nimmt er auf den dort erbrachten Nachweis, dass die Vorzeichen sich vereinfachen lassen
%
und aus allen drei Fällen eine und dieselbe Formel hervorgeht (siehe S.~\refpassage{37_05_146-147_22a}{37_05_146-147_22b}), ausdrücklich Bezug.
%
Leibniz kommentiert die Bedeutung der Gleichung nicht weiter;
%
nach heutigem Kenntnisstand wird erst in der
%
\textit{Scheda octava} \textit{De corporum concursu} von Januar 1678 (N.~\ref{dcc_08}, S.~\refpassage{LH_37_05_086r_aequatioinfall-1}{LH_37_05_086r_aequatioinfall-1})
%
dieser \glqq aequatio infallibilis\grqq\ eine Schlüsselrolle zukommen, denn Leibniz wird dort
%
ihre physikalische Interpretation als Erhaltungssatz der Größe $mv^2$ beim Stoß
%
im Rahmen seiner Umdeutung der \textit{vis} als quadratische Größe hervorheben.
%
\pend
%
\pstart
%
Das Stück ist auf demselben Folioblatt wie N.~\ref{RK60344_1} (ebenfalls Mai bis Mitte Juni 1677) überliefert.
%
Die materiellen Verhältnisse deuten darauf hin, dass Leibniz zuerst für die Abfassung von N.~\ref{RK60344_1} 
%
die Recto-Seite des Folioblatts verwendet und es erst dann in Quart gebrochen, um auf dem Verso
%
zweispaltig, bzw.\ auf zwei Quartseiten, N.~\ref{RK60344_2} zu verfassen.
\pend
%
\pstart
Da Leibniz in N.~\ref{RK60344_2} auf die Ergebnisse von N.~\ref{RK57273} Bezug nimmt,
%
muss ersteres Stück nach letzterem entstanden sein;
%
unter der zusätzlichen Annahme einer zeitnahen Abfassung beider Stücke, die durch 
%
das enge Überlieferungsverhältnis zu N.~\ref{RK60344_1} bekräftigt wird, 
%
ergibt sich für N.~\ref{RK60344_2} die Zeitspanne Mai bis ca.\ Juni 1677.
%
\pend
%
\end{ledgroup}
%
%
\selectlanguage{latin}
\frenchspacing
% \newpage%
\vspace{8mm}
\count\Bfootins=1100%
\count\Afootins=1100%
\count\Cfootins=1100
\pstart%
\normalsize%
\noindent%
\lbrack152~v\textsuperscript{o}\rbrack\
%
\rule[-4mm]{0mm}{10mm}$\displaystyle\frac{a}{b}\sqcap \displaystyle\frac{m-f}{e-i}\sqcap \displaystyle\frac{(+)m^2-f^2}{\protect\pmD\; e^2-i^2}$. Si $m+f\sqcap e+i$, fiet: 
%
\edlabel{37_05_152v_18a}%
\edtext{}{{\xxref{37_05_152v_18a}{37_05_152v_18b}}\lemma{$\displaystyle\frac{+m^2-f^2}{+e^2-i^2}\sqcap \displaystyle\frac{(+)m^2-f^2}{\protect\pmD\; e^2-i^2}$.}\Bfootnote{\textit{(1)}~\textbar\ Unde \textit{str.\ Hrsg.}\ \textbar\ %
 dabuntur tres aequationes \textit{(2)}~Quod~\textit{L}}}%
\rule[-4mm]{0mm}{10mm}$\displaystyle\frac{+m^2-f^2}{+e^2-i^2}\sqcap \displaystyle\frac{(+)m^2-f^2}{\protect\pmD\; e^2-i^2}$. 
%
Quod\edlabel{37_05_152v_18b}
%
%
si jam $(+)\sqcap\; +$ et \ppmD etiam $\sqcap\; +$, tunc habebimus duas tantum aequationes 
%
$\displaystyle\frac{m-f}{e-i}\sqcap \displaystyle\frac{a}{b}$ et $m+f\sqcap e+i$, ex quibus fit tertia illa \rule[-4mm]{0mm}{10mm}$\displaystyle\frac{m^2-f^2}{e^2-i^2}\sqcap\displaystyle\frac{a}{b}$. 
%
Sed 
% 
\edtext{si tertia}{%
\lemma{si}%
\Bfootnote{%
\textit{(1)}~est %
\textit{(2)}~tertia%
~\textit{L}%
}}
alia esset
%
\edtext{v.\,g.}{%
\lemma{}%
\Bfootnote{%
v.\,g.\ %
\textit{erg.~L}}}
%
ut \rule[-4mm]{0mm}{10mm}$\displaystyle\frac{-m^2-f^2}{+e^2-i^2}$, haberetur \makebox[1.0\textwidth][s]{nimia determinatio seu impossibilitas\lbrack,\rbrack\ ergo id fieri non potest. 
%
\edtext{}{{\xxref{KZeitz201}{KZeitz202}}%
{%
\lemma{Ergo si \lbrack...\rbrack\ ostendimus}%
\Cfootnote{%
Siehe S.~\refpassage{37_05_146-147_22a}{37_05_146-147_22b} von N.~\ref{RK57273}.}}}%
\edlabel{KZeitz201} Ergo si $e+i\,\sqcap\, m+f$ signa}
\pend
\newpage
\pstart
\noindent
 sunt eadem. 
Item si $m\;\groesser f$, seu $e\;\groesser i$, signa
 sunt eadem ut ostendimus.
Item si $f+m\;\groesser e+i$, ut etiam ostendimus.\edlabel{KZeitz202}
%%
\pend
%
\pstart
Superest ut ostendamus si $h\overline{f+m}\sqcap e+i$. Et $\displaystyle\frac{m-f}{e-i}\sqcap \displaystyle\frac{(+)m^2-f^2}{\protect\pmD\; e^2-i^2}$. 
%
Ergo \rule[-4mm]{0mm}{10mm}$\sqcap\displaystyle\frac{h\overline{m^2-f^2}}{+e^2-i^2}$. 
%
Sit jam \ppmD$\sqcap\, +$, fiet %
\edtext{$(+)m^2-f^2\sqcap hm^2-hf^2$, fiet 1.}{\lemma{$(+)m^2-f^2\sqcap hm^2-hf^2$}\Bfootnote{\textit{(1)}~. Est autem si jam $(+)$ est $-$ tunc negativum \textit{(2)}~, fiet 1.~\textit{L}}} %
%
 Ergo $(+)m^2-f^2\groesser hm^2-f^2$. Ergo $(+)m^2\groesser hm^2$. Ergo necessario $(+)\sqcap +$. et $h\sqcap 1$. 
%
Ergo necessario si \ppmD $\sqcap\; +$, et \textit{h} est non minor unitate, seu si non est $f+m\,\groesser e+i$, tunc \textit{h} est aequalis unitati. Ergo si \ppmD est $+$, etiam $(+)$ est $+$ et non potest esse $e+i\,\groesser +f+m$. 
\pend
%
\pstart 
Superest unus casus si $\ppmD \sqcap\; -$, et $e+i\,\groesser f+m$. 
\pend
%
\pstart 
Erit $b\,\overline{f+m}\, \groesser e+i$ (posita $h \,\groesser 1$) et $\displaystyle\frac{h\overline{m^2-f^2}}{e^2-i^2}\sqcap \displaystyle\frac{(+)m^2-f^2}{\pmD\; e^2-i^2}$. 
%
Si jam sit $\ppmD \sqcap\; -$, et $(+)\sqcap +$, fiet: 
%
\rule[-4mm]{0mm}{10mm}$\displaystyle\frac{hm^2-hf^2}{e^2-i^2}\sqcap \displaystyle\frac{+m^2-f^2}{-e^2-i^2}$. Ergo $e^2-i^2\sqcap -he^2-hi^2$, quod est absurdum. Ergo semper 
%
signa sunt 
%
\edlabel{37_05_152v_7a}%
eadem.%
\edtext{}{{\xxref{37_05_152v_7a}{37_05_152v_7b}}\lemma{eadem.}\Bfootnote{\textit{(1)}~Brevior calculus: \textbar\ $m-f$ \textit{streicht Hrsg.} \textbar\ \textit{(2)}~Generaliter:~\textit{L}}} %
% 
\pend
%
\pstart %
Generaliter:\edlabel{37_05_152v_7b}
%
\rule[-4mm]{0mm}{10mm}%
$\displaystyle\frac{m-f}{e-i}\sqcap %
\edtext{\displaystyle\frac{(+)m^2-f^2}{\pmD\; e^2-i^2}$. Sit}{\lemma{$\displaystyle\frac{(+)m^2-f^2}{\pmD\; e^2-i^2}$.}\Bfootnote{\textit{(1)}~Ergo \textit{(2)}~Sit~\textit{L}}} %
%
 $h\; \overline{f+m}\sqcap e+i$ (\protect\vphantom)posito \textit{h} esse vel $\sqcap \,1$, vel %
\edtext{esse $h\,\groesser i$,}{\lemma{esse}\Bfootnote{\textit{(1)}~$b\,\sqcap $ fractioni \textit{a} \textit{(2)}~\textbar\ esse \textit{streicht Hrsg.} \textbar\ $h\groesser i$,~\textit{L}}} %
%
vel $b\; \kleiner 1$\protect\vphantom(). 
%
Fiet \rule[-4mm]{0mm}{10mm}$\displaystyle\frac{hm^2-hf^2}{e^2-i^2}\sqcap 
%
\edlabel{37_05_152v_8a}\displaystyle\frac{(+)m^2-f^2}{\pmD\; e^2-i^2}$. Sit $\ppmD \sqcap\; +$, fiet $hm^2-hf^2 \sqcap (+)m^2-f^2$. Ergo\edlabel{37_05_152v_8b}%
\edtext{}{{\xxref{37_05_152v_8a}{37_05_152v_8b}}\lemma{$\displaystyle\frac{(+)m^2-f^2}{\pmD\; e^2-i^2}$}\Bfootnote{%
\textit{(1)}~positoque $\ppmD\;\sqcap \,+$ fiet $hm^2-hf^2\sqcap (+) m^2-f^2$. Ergo sit $h\,\groesser 1$, et $\ppmD \sqcap +$ fiet $hm^2-hf^2 \kleiner $ %
\textit{(2)}~. Sit $\ppmD \sqcap +$, fiet $hm^2-hf^2 \sqcap (+)m^2-f^2$. %
\textit{(a)}~Ergo %
\textit{(aa)}~$h \sqcap $%
\textit{(bb)}~\textbar\ si $\protect\begin{array}[t]{c}h \,\groesser 1\\\text{vel}\,\sqcap 1\protect\end{array}$\textit{streicht Hrsg.} \textbar\ erit $h \,\protect\ovalbox{$m^2$}\,\kleiner (+)\protect\ovalbox{$m^2$} $ \textbar\ seu %
\protect\begin{tabular}[t]{c}$h\,\kleiner\; 1.$\\vel $\sqcap 1$\protect\end{tabular} \textit{erg.} \textbar\ %
 Ergo erit $hm^2 \sqcap 1$, $h\sqcap 1$, et $(+)\;\sqcap +$, posito \ppmD esse $+$ et \textit{h} non esse \kleiner\,1. Sin $h\,\kleiner 1$ fiet $hm^2 \groesser (+)m^2$, seu $h\, \groesser\, 1$, contra hypothesin. %
\textit{(b)}~. Ergo %
\textit{L}}} %
%
si $h\; \groesser 1$, erit $hm^2 \kleiner (+)m^2$. Seu $h\, \groesser (+)1$.  
%
%
\edlabel{37_05_152v_19a}%
\edtext{}{% C-Footnote
{\xxref%
{37_05_152v_19a}{37_05_152v_19b}}%
\lemma{Ergo \lbrack/\rbrack\ $\displaystyle\frac{(+)m^2-f^2}{\pmD\; e^2-i^2}$}%
\Cfootnote{%
Die Absätze sind durch eine waagerechte Linie getrennt.}}%
%
Ergo
\lbrack\textit{Text bricht ab}.\rbrack 
\pend
\newpage
\pstart 
$\displaystyle\frac{(+)m^2-f^2}{\pmD\; e^2-i^2}%
\edlabel{37_05_152v_19b}%
\sqcap \displaystyle\frac{h\; \overline{m^2-f^2}}{e^2-i^2}$ si $(+)\sqcap +$. 
%
Fiet $\ppmD he^2-hi^2\sqcap e^2-i^2$. Ponamus \ppmD esse $-$ fiet: 
%
$-he^2-hi^2\sqcap e^2-i^2$ seu $+\overline{1+h}e^2\sqcap +\overline{1-h}i^2$. 
%
Seu $\displaystyle\frac{1+h}{1-h}\sqcap \displaystyle\frac{i^2}{e^2}$, quod est absurdum, cum $e\,\groesser i$\lbrack,\rbrack\ item cum $h\,\groesser 1$. Ergo illis casibus signa \edlabel{37_05_152v_9a}eadem.%
\edtext{}{{\xxref{37_05_152v_9a}{37_05_152v_9b}}\lemma{eadem.}\Bfootnote{\textit{(1)}~Ponamus \textit{(2)}~Si~\textit{L}}} %
% 
\pend
%
\pstart 
 Si\edlabel{37_05_152v_9b} $(+)\sqcap -$ 
%
\edtext{et $\ppmD \sqcap +$}{%
\lemma{}%
\Bfootnote{%
et $\ppmD \sqcap +$ %
\textit{erg.~L}%
}}
%
fiet $\displaystyle\frac{-f^2-m^2}{+hm^2-hf^2}\sqcap \displaystyle\frac{+e^2-i^2}{+e^2-i^2}$\ fiet: $-f^2-m^2 \sqcap \,hm^2-hf^2$ seu $\overline{h+1}m^2 \sqcap \overline{h-1}f^2$, seu $\displaystyle\frac{1+h}{h-1}\sqcap \rule[-4mm]{0mm}{10mm}\displaystyle\frac{f^2}{m^2}$ quod est absurdum cum $h\,\kleiner 1$, item cum \edlabel{37_05_152v_10a}$m\,\groesser f$.%
\edtext{}{{\xxref{37_05_152v_10a}{37_05_152v_10b}}\lemma{$m\,\groesser f$.}\Bfootnote{\textit{(1)}~Hinc conclud \textit{(2)}~Conclusiones: 1\protect\vphantom() Cum $m\,\groesser f$ vel $e\,\groesser i$ signa sunt eadem. 2\protect\vphantom() Cum \textit{(3)}~Si~\textit{L}}} %
% 
\pend
%
\pstart 
Si\edlabel{37_05_152v_10b}
%
$\ppmD \sqcap +$ fiet: $(+)m^2-f^2 \sqcap hm^2-hf^2$. Ergo $\overline{h(-)1}m^2 \sqcap \overline{h-1}f^2$. 
%
Seu $\displaystyle\frac{f^2}{m^2}\sqcap \displaystyle\frac{h(-)1}{h-1}$. Ergo vel erit $f\sqcap m$, vel erit $(+) \sqcap -$. 
\pend 
\vspace{0.5em}%
%
\pstart \noindent
\edtext{}{\lemma{}\Afootnote{%
\textit{Am oberen Rand, über der rechten Textspalte}: $\displaystyle\frac{a}{b}\sqcap\displaystyle\frac{m-f}{e-i}$\textsuperscript{\lbrack a\rbrack}\newline\newline{\footnotesize \textsuperscript{\lbrack a\rbrack} \textbar\ $\displaystyle\frac{m-f}{e-i}\sqcap \displaystyle\frac{(+)m^2-f^2}{\pmD\;e^2-i^2}$. Si \textit{gestr.} \textbar\ $\displaystyle\frac{a}{b}\sqcap\displaystyle\frac{m-f}{e-i}$~\textit{L}}\newline}} 
%
$c^2\sqcap +bf^2 \ppmD ae^2\sqcap (+)bm^2+ai^2$. Ergo $(+)bm^2-bf^2\sqcap\; \ppmD\; ae^2-ai^2$. %
\pend
%
\pstart\noindent
Ergo $\displaystyle\frac{(+)m^2-f^2}{\pmD\; e^2-i^2}\sqcap \displaystyle\frac{a}{b}\sqcap \displaystyle\frac{m-f}{e-i}$. Ergo $ae-ai \sqcap bm-bf$ et $i\sqcap \displaystyle\frac{ae+bf-bm}{a}$, \edlabel{37_05_152v_16a}seu $i\sqcap e+\displaystyle\frac{a}{b}\; \overline{f-m}$. %
\edtext{Ergo}{\lemma{}\Afootnote{\textit{Am Rand, gestrichen}: $\displaystyle\frac{(-)m^2a}{b}+\displaystyle\frac{f^2a}{b}$\vspace{-3mm}}}
%
$i^2 \sqcap e^2+\displaystyle\frac{2ae}{b}f-\displaystyle\frac{2ae}{b}m+\displaystyle\frac{a^2}{b^2}f^2-
%
\edtext{\displaystyle\frac{2a^2}{b^2}\lbrack f\rbrack m}{%
\lemma{}%
\Bfootnote{%
$f$ %
\textit{erg.~Hrsg.}}}
%
+\displaystyle\frac{a^2}{b^2}m^2$\rule[-4mm]{0mm}{10mm} \edlabel{37_05_152v_11a}at \rule[-4mm]{0mm}{10mm}$\displaystyle\frac{(+)m^2b}{a}%
\edlabel{37_05_152v_11b}%
\edtext{}{{\xxref{37_05_152v_11a}{37_05_152v_11b}}\lemma{at}\Bfootnote{\textit{(1)}~idem $i^2 \sqcap $ \textit{(2)}~$\displaystyle\frac{(+)m^2b}{a}$~\textit{L}}} %
%
-\displaystyle\frac{f^2b}{a}\sqcap\; \ppmD\; e^2-i^2$. 
%
Ergo $i^2\sqcap\; \ppmD\; e^2(-)\displaystyle\frac{m^2b}{a}+f^2\displaystyle\frac{b}{a}$. %
%
\edlabel{37_05_152v_12a}%
Ergo opus esse si $b \sqcap a$%
\edlabel{37_05_152v_12b}%
\edtext{}{{\xxref{37_05_152v_12a}{37_05_152v_12b}}\lemma{Ergo}\Bfootnote{\textit{(1)}~$\displaystyle\frac{e^2a}{b}+2a$ \textit{(2)}~opus esse \textit{(a)}~$f^2+a$ \textit{(b)}~si $b\sqcap a$~\textit{L}}} %
%
\edlabel{37_05_152v_16b} \lbrack\textit{Text bricht ab.}\rbrack%
\edtext{}{{\xxref{37_05_152v_16a}{37_05_152v_16b}}\lemma{$i\sqcap e+\displaystyle\frac{a}{b}\; \overline{f-m}$}\Cfootnote{Die Gleichung lautet richtig: $i\sqcap e+\displaystyle\frac{b}{a}\; \overline{f-m}$. Der Fehler beeinträchtigt die weitere Rechnung und führt zu dem Abbruch.}} 
\pend 
\newpage
\pstart
%
\edtext{$c^2\sqcap \displaystyle\frac{(+)bn^2l+ao^2l}{ao+bn}$. Est}{\lemma{$c^2\sqcap \displaystyle\frac{(+)bn^2l+ao^2l}{ao+bn}$.}\Bfootnote{\textit{(1)}~Est autem $\displaystyle\frac{o}{n}\sqcap \displaystyle\frac{ai}{bm}$ et $o\sqcap \displaystyle\frac{ai}{bm}n$ ergo $c^2\sqcap \displaystyle\frac{(+)bn^{\xout{2}}l+\displaystyle\frac{aa^2i^2n^{\xout{2}}}{b^2m^2}}{\displaystyle\frac{aai}{bm}\xout{n}+b\xout{n}}$ \textit{(2)}~Est~\textit{L}}} %
%
autem \rule[-4mm]{0mm}{10mm}$\displaystyle\frac{o}{n}\sqcap \displaystyle\frac{ai}{bm}$ et $ai+bm\sqcap ae+bf$. Ergo $ai \sqcap ae+bf-bm$. 
%
Ergo \rule[-4mm]{0mm}{10mm}$\displaystyle\frac{o}{n}\sqcap\displaystyle\frac{ae+bf-bm}{bm}$, seu $\displaystyle\frac{ae}{bm}+\displaystyle\frac{f}{m}-1$ \edlabel{37_05_152v_17a}et \rule[-4mm]{0mm}{10mm}$\displaystyle\frac{ao}{bn}\edlabel{37_05_152v_17b}$%
\edtext{}{{\xxref{37_05_152v_17a}{37_05_152v_17b}}\lemma{et}\Bfootnote{\textit{(1)}~$m\,\sqcap $ \textit{(2)}~$\displaystyle\frac{ao}{an}\sqcap \displaystyle\frac{a^2i}{abm}$ \textit{(3)}~$\displaystyle\frac{ao}{bn}$~\textit{L}}} %
%
$\sqcap \displaystyle\frac{a^2i}{b^2m}$. %
\pend
%
\pstart\noindent
\rule[-4mm]{0mm}{10mm}$\displaystyle\frac{o}{n}+1\sqcap \displaystyle\frac{ai+bm}{bm}\sqcap \displaystyle\frac{ae+bf}{bm}\sqcap \displaystyle\frac{o+n}{n}$ \edlabel{37_05_152v_13a}et \rule[-4mm]{0mm}{10mm}$\displaystyle\frac{bm+ai}{ai}\edlabel{37_05_152v_13b}$ %
\edtext{}{{\xxref{37_05_152v_13a}{37_05_152v_13b}}\lemma{et}\Bfootnote{\textit{(1)}~$\displaystyle\frac{n}{o}+1\,\sqcap$ \textit{(2)}~$\displaystyle\frac{bm+ai}{ai}$~\textit{L}}} %
%
$\sqcap \displaystyle\frac{ae+bf}{ai}$. 
%
\pend
%
\pstart\noindent
Ergo $\displaystyle\frac{ao+bn}{bn}\sqcap 1+\displaystyle\frac{a^2i}{b^2m}\sqcap 
%
\edlabel{37_05_152v_14a}%
\edtext{}{{\xxref{37_05_152v_14a}{37_05_152v_14b}}\lemma{$\displaystyle\frac{b^2m+a^2i}{b^2m}$.}\Bfootnote{\textit{(1)}~Ergo 
\textit{(2)}~\textbar~Potius sic faciamus, ut \textit{erg.}~\textbar\ %
video~\textit{L}}}%
\displaystyle\frac{b^2m+a^2i}{b^2m}$.\rule[-4mm]{0mm}{10mm}
%
\pend
%
\pstart
Potius sic faciamus, ut video%
\edlabel{37_05_152v_14b}
%
jam\lbrack:\rbrack\ via centri\protect\index{Sachverzeichnis}{via centri} 
%
est summa\protect\index{Sachverzeichnis}{summa potentiarum applicata ad summam corporum}
%
vel differentia\protect\index{Sachverzeichnis}{differentia potentiarum applicata ad summam corporum} %
\edtext{potentiarum applicata}{\lemma{potentiarum}\Bfootnote{\textit{(1)}~ducta in \textit{(2)}~applicata~\textit{L}}} %
%
\rule[0cm]{0mm}{20pt}ad summam corporum
seu $\displaystyle\frac{\pmG\; ae\;\pmH\; bf}{a+b}$ seu $\displaystyle\frac{(\pmG)ai(\pmH)bm}{a+b}$. Est autem $\displaystyle\frac{a}{b}\sqcap \displaystyle\frac{m-f}{e-i}$ seu 
%
$ae+bf\sqcap %
\edlabel{37_05_152v_15a}%
ai+bm$. 
%
Ergo\edlabel{37_05_152v_15b}%
\edtext{}{{\xxref{37_05_152v_15a}{37_05_152v_15b}}\lemma{$ai+bm$.}\Bfootnote{\textit{(1)}~Ergo $\displaystyle\frac{a+b}{b}\sqcap \displaystyle\frac{m-f+e-i}{e-i}$ \textit{(2)}~Ergo~\textit{L}}} %
%
fiet posterior via centri: \rule{0pt}{20pt}$\displaystyle\frac{(\pmG)ae(\pmG)bf(\pmA)bm(\pmH)bm}{a+b}$. 
\pend 
\count\Bfootins=1200%
\count\Afootins=1200%
\count\Cfootins=1200