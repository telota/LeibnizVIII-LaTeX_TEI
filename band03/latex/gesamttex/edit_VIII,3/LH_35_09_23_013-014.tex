%   % !TEX root = ../../VIII,3_Rahmen-TeX_8-1.tex
%
%
%   Band VIII, 3 N.~??S01.07 (\ref{dcc_06-1})
%   Signatur/Tex-Datei: LH_35_09_23_013-014
%   RK-Nr. 41208 /7
%   Überschrift: De corporum concursus scheda sexta
%   Modul: Mechanik / Stoß ( )
%   Datierung: Januar 1678
%   WZ: (Bl. 14) LEd-WZ 803017 = RK-WZ 1264 (eins)
%   SZ: \pleibdashv \pleibvdash bzw. \leibdashv \leibvdash (insgesamt zwei in zweifacher Fassung)
%   Bilddateien (PDF): (keine)
%   Verzeichniseinträge: vollständig
%   \textls{} statt \textso{} (Ausnahme: Personenverzeichnis)
%
%
\selectlanguage{ngerman}%
\frenchspacing%
%
\begin{ledgroupsized}[r]{120mm}%
\footnotesize%
\pstart%
\noindent\textbf{Überlieferung:}
\pend%
\end{ledgroupsized}%
\begin{ledgroupsized}[r]{114mm}%
\footnotesize%
\pstart \parindent -6mm%
\makebox[6mm][l]{\textit{L}}%
Konzept: LH XXXV~9,~23 Bl.~13\textendash14.
Ein Bogen. 2\textsuperscript{o};
ein Wasserzeichen auf Bl.~14.
Vier vollbeschriebene Seiten,
die den Text N.~\ref{dcc_05} %??S01\textsubscript{6} 
fortsetzen
und vom Text N.~\ref{dcc_06-2} %??S01\textsubscript{8}
fortgesetzt werden;
ein Kustos am Ende von Bl.~14~v\textsuperscript{o} verweist auf die \textit{Scheda secundo-sexta} (wobei \textit{secundo-sexta} aus \textit{septima} verbessert ist).
Randbemerkungen zum Teil \textit{post reformationem} verfasst (siehe die editorische Vorbemerkung, S.~\refpassage{dcc_Vorbemerkung_reform-1}{dcc_Vorbemerkung_reform-2}).
\pend%
\end{ledgroupsized}%
%
\begin{ledgroupsized}[r]{114mm}%
\footnotesize%
\pstart \parindent -6mm%
\makebox[6mm][l]{\textit{E}}%
\textsc{Fichant} 1994, S.~116\textendash124\cite{01056}
(mit kommentierter französischer Übersetzung, S.~236\textendash255).
\pend%
\end{ledgroupsized}%
%
\selectlanguage{latin}%
\frenchspacing%
\count\Bfootins=1000%
\count\Afootins=1200%
\count\Cfootins=1000
%
%
\vspace{8mm}
\pstart%
\normalsize%
\noindent%
%
\lbrack13~r\textsuperscript{o}\rbrack% \ %%%% Blatt 13r
\hspace{47mm}
Scheda sexta%
\protect\index{Sachverzeichnis}{scheda}%
\hspace{37mm}
Januar. 1678
\pend%
\pstart%
\noindent%
\centering%
De concursu Corporum%
\protect\index{Sachverzeichnis}{concursus corporum}
\pend%
\vspace{0.5em}%
%
\pstart%
\noindent
Hactenus de corporis in aliud quiescens%
\protect\index{Sachverzeichnis}{corpus quiescens}
vel antecedens%
\protect\index{Sachverzeichnis}{corpus antecedens}
incursu%
\protect\index{Sachverzeichnis}{incursus in corpus quiescens}%
\protect\index{Sachverzeichnis}{incursus in corpus antecedens}
tractavi sine consideratione percussionis.%
\protect\index{Sachverzeichnis}{percussio}
Sed nunc ea quoque adhibenda videtur.
Dico igitur:
Abreptionem esse%
\protect\index{Sachverzeichnis}{abreptio}
%
\edtext{cum corpus aliquod
ideo in eam plagam%
\protect\index{Sachverzeichnis}{plaga}
tendit fortius quam ante,
quia aliud nunc ipsum comitatur.}{%
\lemma{cum}\Bfootnote{%
\textit{(1)}~corpus alteri celeriori
\textit{(2)}~duo corpora
\textit{(3)}~corpus aliquod 
\textbar~ideo \textit{erg.}~%
\textbar\ in eam plagam tendit
\textit{(a)}~in qua aliud
\textit{(b)}~fortius quam
\lbrack...\rbrack\ ipsum comitatur.%
~\textit{L}}}
%
Fit autem abreptio vel sine percussione,%
\protect\index{Sachverzeichnis}{abreptio sine percussione}
vel cum percussione.%
\protect\index{Sachverzeichnis}{abreptio cum percussione}
Sine percussione,%
\protect\index{Sachverzeichnis}{percussio}
ut si corpori in motu exsistenti
aliquod aliud imponi
intelligatur vel appendi\lbrack,\rbrack\
tunc enim procedent
%
\edtext{simul,
theorema autem progressus%
\protect\index{Sachverzeichnis}{theorema progressus}%
\protect\index{Sachverzeichnis}{progressus}
hoc erit:}{%
\lemma{simul}\Bfootnote{%
\textit{(1)}~celeritate quae sit ad priorem
\textit{(2)}~, theorema autem progressus hoc erit:%
~\textit{L}}}
% ((\textbf{???? wegen Instabilität gezwungener Seitenumbruch ????}))
\pend%
%\newpage%
%
\pstart%
\edtext{}{%
{\xxref{LH_35_09_23_013r_ersteMarg-1}{LH_35_09_23_013r_ersteMarg-2}}%
{\lemma{\textit{Am Rand:}}\Afootnote{%
\textsuperscript{[a]}%
Post reformationem%
\protect\index{Sachverzeichnis}{reformatio}%
\lbrack:\rbrack\
$ae^2$ vis%
\protect\index{Sachverzeichnis}{vis post reformationem}
quae divisa per $a+b$ dat
$\displaystyle\frac{a}{a+b}e^2,$%
\textsuperscript{[b]}
momentum imo totius molis motae,%
\protect\index{Sachverzeichnis}{momentum molis motae}
cujus radix est $e\,\sqrt{\displaystyle\frac{a}{a+b}}$
quae foret celeritas.%
\protect\rule[-5mm]{0mm}{0mm}
Itaque res hoc modo non procedit.
\newline\vspace{-0.4em}%
\newline%
{\footnotesize%
\textsuperscript{[a]}~%
\textit{(1)}~Imo sic 
\textit{(2)}~Post reformationem%
~\textit{L}
\quad
\textsuperscript{[b]}~$\displaystyle\frac{a}{a+b}e^2,$
\textit{(1)}~celeritate corporis
\textit{(2)}~momentum imo totius molis motae,%
~\textit{L}%
%\newline%
}}}}%
%
Si%
\edlabel{LH_35_09_23_013r_ersteMarg-1}%
\edlabel{LH_35_09_23_013r_alku_jwr-1}
corpus unum abripiat aliud%
\protect\index{Sachverzeichnis}{corpus abripiens}%
\protect\index{Sachverzeichnis}{corpus abreptum}
%
\edtext{quiescens%
\protect\index{Sachverzeichnis}{corpus quiescens}%
}{%
\lemma{quiescens}\Bfootnote{%
\textit{erg.~L}}}
%
sine percussione;%
\protect\index{Sachverzeichnis}{abreptio sine percussione}
procedent ambo celeritate%
\protect\index{Sachverzeichnis}{celeritas communis}
quae sit ad celeritatem incursus,%
\protect\index{Sachverzeichnis}{celeritas incursus}
%
\edtext{ut corpus incurrens%
\protect\index{Sachverzeichnis}{corpus incurrens}
ad}{%
\lemma{ut}\Bfootnote{%
\textit{(1)}~summa corporum ad
\textit{(2)}~corpus incurrens ad
\textit{L}}}
%
corporum
%
\edtext{summam.%
\protect\index{Sachverzeichnis}{summa corporum}
Alioqui%
\edlabel{LH_35_09_23_013r_ersteMarg-2}%
}{%
\lemma{summam.}\Bfootnote{%
\textit{(1)}~Ut s
\textit{(2)}~Alioqui
\textit{L}}}
%
enim non servabitur eadem potentia.%
\protect\index{Sachverzeichnis}{potentia servanda}
Idem aliter ostendi potest,
quia utique tardius progredi necesse
%
\edtext{est,
et eo magis quidem
quo majus est corpus additum,%
\protect\index{Sachverzeichnis}{corpus additum}
unde ob progressionis simplicitatem,%
\protect\index{Sachverzeichnis}{simplicitas progressionis}%
}{%
\lemma{est,}\Bfootnote{%
\textit{(1)}~unde necesse est pro
\textit{(2)}~et eo
\lbrack...\rbrack\ progressionis simplicitatem,
\textit{L}}}
%
ostendendo saltem lineam
de qua agitur
esse debere rectam,%
\protect\index{Sachverzeichnis}{linea recta}
habetur
\edlabel{LH_35_09_23_013-014_13r1}%
quaesitum.%
\protect\index{Sachverzeichnis}{quaesitum}%
\edtext{}{%
{\xxref{LH_35_09_23_013-014_13r1}{LH_35_09_23_013-014_13r2}}%
{\lemma{quaesitum.}\Bfootnote{%
\textit{(1)}~Hinc
\textit{(2)}~Si corpus
\textit{(3)}~Idem est si corpus%
~\textit{L}}}}
%
\pend%
%
\pstart%
Idem est si corpus%
\edlabel{LH_35_09_23_013-014_13r2}
abripiat%
\protect\index{Sachverzeichnis}{corpus abripiens}
aliud%
\protect\index{Sachverzeichnis}{corpus abreptum}
antecedens,%
\protect\index{Sachverzeichnis}{corpus antecedens}
quod probatur
tum independenter a praecedenti,
ex eo quod vis eadem,
tum
%
\edtext{etiam posita compositione}{%
\lemma{etiam}\Bfootnote{%
\textit{(1)}~ex eo quod
\textit{(2)}~posita compositione
\textit{L}}}
%
motus%
\protect\index{Sachverzeichnis}{compositio motuum}
ex communi%
\protect\index{Sachverzeichnis}{motus communis}
ambobus
%
\edtext{et eo
qui}{%
\lemma{et eo}\Bfootnote{%
\textit{(1)}~quod
\textit{(2)}~qui%
~\textit{L}}}
%
incurrenti est proprius.%
\protect\index{Sachverzeichnis}{motus corporis incurrentis}
Qui consensus%
\protect\index{Sachverzeichnis}{consensus}
tum ratiocinationem nostram%
\protect\index{Sachverzeichnis}{ratiocinatio probabilis}
per compositionem%
\protect\index{Sachverzeichnis}{compositio motuum}
et per vim conservatam,%
\protect\index{Sachverzeichnis}{vis conservata}
magis probabilem reddit,
si opus ea haberet probabilitate.%
\protect\index{Sachverzeichnis}{probabilitas}
\pend%
%
\pstart%
Corpora duo concurrentia%
\protect\index{Sachverzeichnis}{corpora concurrentia}
cum percussione,%
\protect\index{Sachverzeichnis}{concursus cum percussione}
post percussionem%
\protect\index{Sachverzeichnis}{percussio}
a se invicem recedunt.%
\protect\index{Sachverzeichnis}{corpora recedentia}%
\edlabel{LH_35_09_23_013r_alku_jwr-2}
Haec propositio est experimentum%
\protect\index{Sachverzeichnis}{experimentum}%
\lbrack,\rbrack\
%
\edtext{vel si dura%
\protect\index{Sachverzeichnis}{corpus durum}
aut elastica%
\protect\index{Sachverzeichnis}{corpus elasticum}
dicimus concurrere cum percussione%
\protect\index{Sachverzeichnis}{concursus cum percussione}
erit definitio.%
\protect\index{Sachverzeichnis}{definitio}%
}{%
\lemma{vel}\Bfootnote{%
\hspace{-0,5mm}si \lbrack...\rbrack\ erit definitio
\textit{erg.~L}}}
%
\pend%
%
\pstart%
%
\edtext{}{%
{\xxref{LH_35_09_23_013r_abhistribus-1}{LH_35_09_23_013r_abhistribus-2}}%
{\lemma{\textit{Am Rand:}}\Afootnote{%
Videtur nimirum percussio%
\protect\index{Sachverzeichnis}{percussio}
ab his tribus pendere.%
\newline%
}}}%
%
\edtext{Si%
\edlabel{LH_35_09_23_013r_abhistribus-1}
augeatur corpus incurrens,%
\protect\index{Sachverzeichnis}{corpus incurrens}
etiam percussio%
\protect\index{Sachverzeichnis}{percussio aucta}
augetur,}{%
\lemma{Si}\Bfootnote{%
\textit{(1)}~majus sit
\textit{(2)}~augeatur corpus incurrens, etiam percussio
\textit{(a)}~fit major,
\textit{(b)}~augetur,%
~\textit{L}}}
%
manentibus caeteris.
\pend%
%
\pstart%
%
\edtext{Si augeatur corpus excipiens,%
\protect\index{Sachverzeichnis}{corpus excipiens}
manentibus caeteris,
etiam percussio%
\protect\index{Sachverzeichnis}{percussio aucta}
augebitur.}{%
\lemma{Si}\Bfootnote{%
\textit{(1)}~minu
\textit{(2)}~majus sit
\textit{(3)}~augeatur corpus excipiens,
\textbar~manentibus caeteris, \textit{erg.}~%
\textbar\ etiam percussio
\textit{(a)}~fit major.
\textit{(b)}~augebitur.%
~\textit{L}}}
%
\pend%
%
\pstart%
%
Si%
\edlabel{LH_35_09-23_013r_supradixi-1}
augeatur celeritas corporis incurrentis,%
\protect\index{Sachverzeichnis}{celeritas corporis incurrentis}
etiam percussio,%
\protect\index{Sachverzeichnis}{percussio aucta}
id est conatus corporum a se invicem recedendi,%
\protect\index{Sachverzeichnis}{conatus recedendi}
augetur.%
\protect\index{Sachverzeichnis}{conatus auctus}%
\edlabel{LH_35_09_23_013r_abhistribus-2}%
\edlabel{LH_35_09-23_013r_supradixi-2}
\pend%
%
\pstart%
\edtext{}{%
{\xxref{LH_35_09_23_hlgh-1}{LH_35_09_23_hlgh-2}}%
{\lemma{\textit{Am Rand:}}\Afootnote{%
Haec verissima.\vspace{-4mm}}}}%
Si%
\edlabel{LH_35_09_23_hlgh-1}
duo corpora sibi
%
\edtext{occurrant,%
\protect\index{Sachverzeichnis}{corpora occurrentia}
et propius}{%
\lemma{occurrant,}\Bfootnote{%
\textit{(1)}~eo major est
\textit{(2)}~et propius%
~\textit{L}}}
%
ad aequalitatem accedat eorum potentia%
\protect\index{Sachverzeichnis}{potentia corporum occurrentium}
utrinque,
etiam major
%
\edtext{est percussio.%
\protect\index{Sachverzeichnis}{percussio major}
Si duo corpora sibi%
\protect\index{Sachverzeichnis}{corpora occurrentia}
aequali vi occurrant,%
\protect\index{Sachverzeichnis}{vis corporum occurrentium}
percussio tanta est
quanta est tota vis.}{%
\lemma{est percussio.}\Bfootnote{%
\textit{(1)}~Maxima est percussio
\textit{(2)}~Percussio semper minor est
\textit{(3)}~Si duo corpora
\textbar~duo \textit{gestr.}~%
\textbar~sibi aequali vi occurrant,
\textit{(a)}~aequalis est
\textit{(b)}~percussio tanta % est quanta est 
\lbrack...\rbrack\ tota vis.%
~\textit{L}}}
%
%\pend%
%%
%\pstart%
In aliis casibus omnibus%
\protect\index{Sachverzeichnis}{casus occursus}
percussio minor%
\protect\index{Sachverzeichnis}{percussio minor}
est tota vi.%
\protect\index{Sachverzeichnis}{vis corporum occurrentium}%
\edlabel{LH_35_09_23_hlgh-2}%
\pend%
%
\pstart%
Si duo corpora sibi occurrant,%
\protect\index{Sachverzeichnis}{corpora occurrentia}
vis percussionis%
\protect\index{Sachverzeichnis}{vis percussionis}
non potest esse minor
vi communi duplicata.%
\protect\index{Sachverzeichnis}{vis corporum occurrentium}
Sive de tota vi residuum%
\protect\index{Sachverzeichnis}{residuum de vi}
%
\edtext{pro simplici abreptione%
\protect\index{Sachverzeichnis}{abreptio simplicis}%
}{%
\lemma{pro}\Bfootnote{%
\textit{(1)}~non simplicis abreptionis
\textit{(2)}~simplici abreptione%
~\textit{L}}}
%
non potest esse majus differentia virium.%
\protect\index{Sachverzeichnis}{differentia virium}
Quod ita demonstratur.
Ponamus corpus potentius esse \textit{A},
debilius \textit{B},
auferatur ab \textit{A} virium excessus,%
\protect\index{Sachverzeichnis}{excessus virium}
constat
%
\edtext{tunc percussionem%
\protect\index{Sachverzeichnis}{percussio}%
}{%
\lemma{tunc}\Bfootnote{%
\textit{(1)}~excessum
\textit{(2)}~percussionem%
~\textit{L}}}
%
fore aequalem vi communi%
\protect\index{Sachverzeichnis}{vis communis}
duplicatae,
id est vi toti;%
\protect\index{Sachverzeichnis}{vis corporum occurrentium}
ergo nunc
si quid vi ipsius \textit{A} adjiciatur,
non ideo minuetur percussio.%
\protect\index{Sachverzeichnis}{percussio}
Imo potius
%
\edtext{augebitur,%
\protect\index{Sachverzeichnis}{percussio aucta}
et ideo}{%
\lemma{augebitur}\Bfootnote{%
\hspace{-0,5mm}%
\textbar~quia supra dixi aucta
\textit{erg. u. gestr.}~%
\textbar~, et ideo%
~\textit{L}}}
%
affirmo\lbrack:\rbrack\
\pend%
\count\Bfootins=1200%
\count\Afootins=1200%
\count\Cfootins=1200
\pstart%
Si duo corpora sibi occurrant,%
\protect\index{Sachverzeichnis}{corpora occurrentia}
vis percussionis%
\protect\index{Sachverzeichnis}{vis percussionis}
major est
vi communi%
\protect\index{Sachverzeichnis}{vis communis}
duplicata,
quia
%
\edtext{supra}{%
\lemma{supra}\Cfootnote{%
S.~\refpassage{LH_35_09-23_013r_supradixi-1}{LH_35_09-23_013r_supradixi-2}.}}
%
dixi aucta tota incurrentis
(\protect\vphantom)%
hoc
%
\edtext{loco alterutrius}{%
\lemma{loco}\Bfootnote{%
\textit{(1)}~alterius
\textit{(2)}~alterutrius%
~\textit{L}}}
%
occurrentis%
\protect\vphantom()
celeritate,%
\protect\index{Sachverzeichnis}{celeritas corporis incurrentis}%
\protect\index{Sachverzeichnis}{celeritas corporum occurrentium}
augeri percussionem.%
\protect\index{Sachverzeichnis}{percussio aucta}
\pend%
%
\pstart%
Si duorum corporum
unum incurrat%
\protect\index{Sachverzeichnis}{corpus incurrens}
in aliud
cum percussione,%
\protect\index{Sachverzeichnis}{incursus cum percussione}
minore celeritate progreditur,%
\protect\index{Sachverzeichnis}{progressus corporis incurrentis}%
\protect\index{Sachverzeichnis}{celeritas corporis incurrentis}
quam si incurreret sine percussione%
\protect\index{Sachverzeichnis}{incursus sine percussione}%
\lbrack:\rbrack\
patet
%
\edtext{ex demonstratis}{%
\lemma{ex demonstratis}\Cfootnote{%
Vgl. N.~\ref{dcc_05}, %??S01\textsubscript{6},
S.~\refpassage{LH_35_09_23_012v_subfine-1}{LH_35_09_23_012v_subfine-2}.%
}}
%
sub finem praecedentis schedae,%
\protect\index{Sachverzeichnis}{scheda}
et locum habet tam in incursu quam in occursu.%
\protect\index{Sachverzeichnis}{occursus}
\pend%
%
\pstart%
Si corpus incurrat in quiescens,%
\protect\index{Sachverzeichnis}{corpus quiescens}%
\protect\index{Sachverzeichnis}{incursus in corpus quiescens}
minor est percussio%
\protect\index{Sachverzeichnis}{percussio minor}
quam si
%
\edtext{occurrat in occurrens.%
\protect\index{Sachverzeichnis}{corpora occurrentia}%
\protect\index{Sachverzeichnis}{occursus in corpus occurrens}%
}{%
\lemma{occurrat}\Bfootnote{%
\hspace{-0,5mm}in
\textit{(1)}~quiescens.
\textit{(2)}~occurrens.%
~\textit{L}}}
%
Aucta est enim celeritas alterius incurrentis.%
\protect\index{Sachverzeichnis}{celeritas corporis incurrentis}
\pend%
%
\pstart%
Potest fieri ut minor sit percussio%
\protect\index{Sachverzeichnis}{percussio minor}
duorum corporum sibi occurrentium,%
\protect\index{Sachverzeichnis}{corpora occurrentia}%
\protect\index{Sachverzeichnis}{percussio corporum occurrentium}
quam si alterutrum quiesceret.
Modo enim concedatur
aliquam esse percussionem.%
\protect\index{Sachverzeichnis}{percussio aliqua}
Patet enim
vim occursus minorem quavis data%
\protect\index{Sachverzeichnis}{vis occursus minor quavis data}
fieri posse.
Si autem nulla esset
%
\edtext{percussio
cum in quiescens%
\protect\index{Sachverzeichnis}{corpus quiescens}%
}{%
\lemma{percussio}\Bfootnote{%
\textit{(1)}~nam inquiescitur
\textit{(2)}~cum in quiescens%
~\textit{L}}}
%
\lbrack13~v\textsuperscript{o}\rbrack\    %  %  %  %    Blatt 13v
%
impingitur,%
\protect\index{Sachverzeichnis}{corpus impingens in quiescens}
sequeretur
nec corpus parvum
in maximum corpus
impingens%
\protect\index{Sachverzeichnis}{corpus parvum impingens in maximum}
repelli nonnihil.%
\protect\index{Sachverzeichnis}{corpus repulsum}
\pend%
%
\pstart%
Quaeritur
an eadem sit percussio
si corpora eadem
eadem celeritate
%
\edtext{sibi appropinquent,}{%
\lemma{sibi}\Bfootnote{%
\textit{(1)}~appropinquant
\textit{(2)}~appropinquent,%
~\textit{L}}}
%
quaecunque sit
%
\edtext{denique}{%
\lemma{denique}\Bfootnote{%
\textit{erg.~L}}}
%
propria cujusque celeritas absoluta.%
\protect\index{Sachverzeichnis}{celeritas absoluta}
Sint corpora \textit{A},\,%
\textit{B} concurrentia%
\protect\index{Sachverzeichnis}{corpora concurrentia}
celeritatibus \textit{e}, \textit{i},%
\protect\index{Sachverzeichnis}{celeritas corporum concurrentium}
et sit \textit{ae} aequ. \textit{bi},
erit
%
\edtext{vis percussionis%
\protect\index{Sachverzeichnis}{vis percussionis}%
}{%
\lemma{vis}\Bfootnote{%
\textit{(1)}~concursus
\textit{(2)}~percussionis%
~\textit{L}}}
%
aequ. 2\textit{ae}
aequ. 2\textit{bi},
seu $ae+bi.$
Ponamus jam quiescere \textit{b},
et celeritatem ipsius \textit{a} esse \textit{H},
esse autem \textit{H} aequ.
%
\edtext{$e+i,$
tunc erit}{%
\lemma{$e+i,$}\Bfootnote{%
\textit{(1)}~\textbar~erit \textit{streicht Hrsg.}~\textbar\
\textit{(2)}~tunc erit%
~\textit{L}}}
%
eadem celeritas appropinquationis%
\protect\index{Sachverzeichnis}{celeritas appropinquationis}
quae ante,
percussio vero erit $ae+bi,$
et vis tota erit $ae+ai.$%
\protect\index{Sachverzeichnis}{vis tota}
Ergo vis residua%
\protect\index{Sachverzeichnis}{vis residua}
erit
$\raisebox{-2.25mm}{{\def\firstcircle{(0,0) circle (0.3cm)}\begin{tikzpicture}\draw \firstcircle node {\textit{ae}};\end{tikzpicture}}} + ai \ \raisebox{-2.75mm}{{\def\firstcircle{(0,0) circle (0.35cm)}\begin{tikzpicture}\draw \firstcircle node {$-$\,\textit{ae}};\end{tikzpicture}}} - bi,$
seu
%
\edtext{$ai-bi$
quae divisa per $a+b,$
dat celeritatem $\displaystyle\frac{a-b}{a+b}i$%
\protect\rule[-3mm]{0pt}{8mm}
pro residuo abreptionis,%
\protect\index{Sachverzeichnis}{residuum abreptionis}
cui si addatur
\protect\rule[-3mm]{0pt}{8mm}%
$\displaystyle\frac{ae+bi}{2b},$}{%
\lemma{$ai-bi$}\Bfootnote{%
\textit{(1)}~cui si addatur $ae+$
\textit{(2)}~quae divisa % per $a+b,$ dat celeritatem 
\lbrack...\rbrack\ $\displaystyle\frac{a-b}{a+b}i$ pro
\textit{(a)}~celeritate
\textit{(b)}~residuo abreptionis, % cui si
\lbrack...\rbrack\ addatur $\displaystyle\frac{ae+bi}{2b},$
~\textit{L}}}
%
fit
\rule[-3mm]{0pt}{8mm}%
$\cornersize{1}\displaystyle\frac{a^2e\;\ovalbox{$+b^2i$}+abe \;\ovalbox{$+abi$}+ \underset{3}{ \ovalbox{2}} \;abi\;\ovalbox{$-\ovalbox{2}\; b^2i$}}{2ab+2b^2}$
pro celeritate progressus incurrentis%
\protect\index{Sachverzeichnis}{celeritas progressus}%
\protect\index{Sachverzeichnis}{progressus corporis incurrentis}
%
\edtext{quando incurrens \lbrack progreditur\rbrack,%
\protect\index{Sachverzeichnis}{corpus incurrens}%
\protect\index{Sachverzeichnis}{corpus repulsum}%
}{%
\lemma{quando}\Bfootnote{%
\hspace{-0,5mm}incurrens
\textbar~repellitur \textit{ändert Hrsg.}~\textbar\
\textit{erg.~L}}}
%
et
\rule[-3mm]{0pt}{8mm}%
%
\edtext{si ad $\displaystyle\frac{a-b}{a+b}i$ addatur}{%
\lemma{si}\Bfootnote{%
\textit{(1)}~a
\textit{(2)}~ad $\displaystyle\frac{a-b}{a+b}i$
\textit{(a)}~detrahatur
\textit{(b)}~addatur%
~\textit{L}}}%
%
$\displaystyle\frac{ae+bi}{2a},$
quando scil. repellitur,%
\rule[-3mm]{0pt}{8mm}
%
tunc etiam patet
quid scribi
%
\edtext{debeat,
quando scil.}{%
\lemma{debeat,}\Bfootnote{%
\textit{(1)}~sed quando
\textit{(2)}~quando scil.%
~\textit{L}}}
%
majus est
\rule[-3mm]{0pt}{8mm}%
$\displaystyle\frac{ae+bi}{2a}$
quam
\rule[-3mm]{0pt}{8mm}%
$\displaystyle\frac{a-b}{a+b}i,$
quando vero minus,
tunc debet ad
\rule[-4mm]{0pt}{9mm}%
$\displaystyle\frac{a-b}{a+b}i$
addi
%\rule[-3mm]{0pt}{8mm}%
%
\edtext{$\displaystyle\frac{ae+bi}{b},$}{%
\lemma{$\displaystyle\frac{ae+bi}{b}$}\Cfootnote{%
Wohl eher $\displaystyle\frac{ae+bi}{2b}.$}}
%
et
%\rule[-3mm]{0pt}{8mm}%
$\displaystyle\frac{ae+bi}{2a}$
detrahi a
\rule[-4mm]{0pt}{9mm}%
$\displaystyle\frac{a-b}{a+b}i.$%
\edlabel{LH_35_09_23_013v_margmitrand-1}%
%\pend%
%%
%\pstart%
\edtext{}{%
{\xxref{LH_35_09_23_013v_margmitrand-1}{LH_35_09_23_013v_margmitrand-2}}%
{\lemma{\textit{Am Rand:}}\Afootnote{%
Error in calculo%
\protect\index{Sachverzeichnis}{error in calculo}%
\protect\index{Sachverzeichnis}{calculus}
et ratiocinatione%
\protect\index{Sachverzeichnis}{error in ratiocinatione}%
\protect\index{Sachverzeichnis}{ratiocinatio}
ex falsa hypothesi%
\protect\index{Sachverzeichnis}{hypothesis falsa}
servandae\textsuperscript{[a]}
quantitatis motus.%
\protect\index{Sachverzeichnis}{hypothesis servandae quantitatis motus}%
\protect\index{Sachverzeichnis}{quantitas motus}
\newline\vspace{-0.4em}%
\newline%
{\footnotesize%
\textsuperscript{[a]}~servandae
\textit{(1)}~virium
\textit{(2)}~quantitatis motus%
~\textit{L}%
\newline%
}}}}
%
Video tamen ex his
regulam illam%
\protect\index{Sachverzeichnis}{regula de concursu corporum}
procedere non posse.
Igitur concludo
\edlabel{LH_35_09_23_013-014_13v1}%
sic:%
%
\edtext{}{%
{\xxref{LH_35_09_23_013-014_13v1}{LH_35_09_23_013-014_13v2}}%
{\lemma{sic:}\Bfootnote{%
\textit{(1)}~Nam f
\textit{(2)}~Falsa est%
~\textit{L}}}}
%
\pend%
\count\Bfootins=1000%
\count\Afootins=1200%
\count\Cfootins=1000
\pstart%
Falsa est%
\edlabel{LH_35_09_23_013-014_13v2}
propositio%
\protect\index{Sachverzeichnis}{propositio falsa}
quod eadem sit percussio,%
\protect\index{Sachverzeichnis}{percussio eadem}
si corpora sibi eadem celeritate propinquent.%
\protect\index{Sachverzeichnis}{celeritas appropinquationis}
Quod probo per instantiam%
\protect\index{Sachverzeichnis}{probatio per instantiam}%
\protect\index{Sachverzeichnis}{instantia}
hoc modo.%
\edlabel{LH_35_09_23_013v_margmitrand-2}
%\pend%
%%
%\pstart%
Sint duo corpora \textit{a}, \textit{b} concurrentia%
\protect\index{Sachverzeichnis}{corpora concurrentia}
celeritatibus \textit{e}, \textit{i},%
\protect\index{Sachverzeichnis}{celeritas corporum concurrentium}
ita ut sint in reciproca corporum ratione,%
\protect\index{Sachverzeichnis}{celeritates in ratione reciproca corporum}
%
\edtext{seu ut sit}{%
\lemma{seu}\Bfootnote{%
\hspace{-0,5mm}ut
\textit{(1)}~sint
\textit{(2)}~sit%
~\textit{L}}}
%
\textit{ae} aequ. \textit{bi}.
Ponatur autem
esse corpus \textit{b} majus
corpore \textit{a}.
Porro patet celeritatem
qua sibi appropinquant corpora%
\protect\index{Sachverzeichnis}{celeritas appropinquationis}
esse
%
\edtext{$e + i.$
Ponatur}{%
\lemma{$e + i.$}\Bfootnote{\hspace{-0,5mm}%
\textit{(1)}~Fingatur
\textit{(2)}~Ponatur%
~\textit{L}}}
%
\edtext{jam corpus}{%
\lemma{jam}\Bfootnote{%
\textit{(1)}~eadem
\textit{(2)}~corpus%
~\textit{L}}}
%
\textit{b} quiescere,%
\protect\index{Sachverzeichnis}{corpus quiescens}
et corpus \textit{a} in ipsum incurrere%
\protect\index{Sachverzeichnis}{corpus incurrens}
celeritate $e+i.$%
\protect\index{Sachverzeichnis}{celeritas incursus}
Erit percussionis vis etiam $ae+bi,$%
\protect\index{Sachverzeichnis}{vis percussionis}
vis autem tota est $ae+ai,$%
\protect\index{Sachverzeichnis}{vis tota}
quae est minor quam $ae+bi,$
quia \textit{a} minor quam \textit{b},
erit ergo vis
%
\edtext{percussionis%
\protect\index{Sachverzeichnis}{vis percussionis}
major quam vis tota,%
\protect\index{Sachverzeichnis}{vis tota}%
}{%
\lemma{percussionis}\Bfootnote{%
\textit{(1)}~minor quam vis
\textit{(2)}~major quam
\textit{(a)}~vis
\textit{(b)}~vis tota,%
~\textit{L}}}
%
quod est absurdum.%
\protect\index{Sachverzeichnis}{absurdum}
%
\edtext{Ergo in hoc casu%
\protect\index{Sachverzeichnis}{casus incursus}
(\protect\vphantom)%
cum corpus scil. incurrens est minus excipiente%
\protect\index{Sachverzeichnis}{corpus incurrens minus}%
\protect\index{Sachverzeichnis}{corpus minus in majus}%
\protect\vphantom()
falsa est propositio,%
\protect\index{Sachverzeichnis}{propositio falsa}%
}{%
\lemma{Ergo}\Bfootnote{%
\textit{(1)}~falsa est prop
\textit{(2)}~in hoc \lbrack...\rbrack\ est propositio,%
~\textit{L}}}
%
quod eadem sit percussio%
\protect\index{Sachverzeichnis}{percussio eadem}
si corpora sibi eadem celeritate propinquent.%
\protect\index{Sachverzeichnis}{celeritas appropinquationis}
\pend%
% \newpage%
%
\pstart%
%
\edtext{}{%
{\xxref{LH_35_09_23_013v_objectio-1}{LH_35_09_23_013v_objectio-2}}%
{\lemma{\textit{Am Rand:}}\Afootnote{%
Haec objectio diligenter examinanda,%
\protect\index{Sachverzeichnis}{objectio examinanda}
nam et potest et meretur solvi.%
\protect\index{Sachverzeichnis}{objectio solvenda}\vspace{-4mm}%
%\newline%
}}}
%
Si\edlabel{LH_35_09_23_013v_objectio-1}
eadem esset percussio%
\protect\index{Sachverzeichnis}{percussio eadem}
nulla ratione motuum propriorum habita,%
\protect\index{Sachverzeichnis}{motus proprius}
sequeretur
%
\edtext{in corporibus mollibus%
\protect\index{Sachverzeichnis}{corpus molle}
seu percussione carentibus%
\protect\index{Sachverzeichnis}{corpus percussione carens}%
}{%
\lemma{in}\Bfootnote{%
corporibus % mollibus seu 
\lbrack...\rbrack\ percussione carentibus
\textit{erg.~L}}}
%
\edtext{perdi vim,%
\protect\index{Sachverzeichnis}{vis perdita}%
}{%
\lemma{perdi}\Bfootnote{%
\textit{(1)}~motum
\textit{(2)}~vim,%
~\textit{L}}}
%
et
%
\edtext{quidem omnem perdi
quae}{%
\lemma{quidem}\Bfootnote{%
\textit{(1)}~illam quae
\textit{(2)}~omnem perdi quae%
~\textit{L}}}
%
alioqui,
si elastica essent,%
\protect\index{Sachverzeichnis}{corpus elasticum}
conatum discedendi%
\protect\index{Sachverzeichnis}{conatus discedendi}
produci
%
\edtext{ponitur.%
\edlabel{LH_35_09_23_013v_objectio-2}
Quod}{%
\lemma{ponitur.}\Bfootnote{%
\textit{(1)}~Quia
\textit{(2)}~Quod%
~\textit{L}}}
%
ipsorum
%
\edtext{qui eandem percussionem%
\protect\index{Sachverzeichnis}{percussio eadem}
defendunt}{%
\lemma{qui \lbrack...\rbrack\ defendunt}\Cfootnote{%
Anspielung auf E.~\textsc{Mariotte}, \textit{De la percussion}, partie~I, prop.~3 (Paris 1673, S.~25\,f.);\cite{00311}
siehe hierzu \textit{LSB} VIII,~2 N.~50, S.~423.22\textendash424.3,\cite{01292}
sowie N.~\ref{RK57267-3} in diesem Band, S.~\refpassage{LH_37_05_145r_prop.I.3_Mario-1}{LH_37_05_145r_prop.I.3_Mario-2}. %
\protect\index{Namensregister}{\textso{Mariotte}, Edme, Seigneur de Chazeuil ca. 1620\textendash1684}
Leibniz berichtigt allerdings seine Aussage N.~\ref{dcc_08}, %??S01\textsubscript{10},
S.~\refpassage{LH_37_05_086r_percussioeadem-1}{LH_37_05_086r_percussioeadem-2}.%
}}
%
sententiae%
\protect\index{Sachverzeichnis}{sententia}
contradicit.
Ratio connexionis%
\protect\index{Sachverzeichnis}{ratio connexionis}%
\protect\index{Sachverzeichnis}{connexio}
est,
quod Elastica%
\protect\index{Sachverzeichnis}{corpus elasticum}
ab aliis mollibus%
\protect\index{Sachverzeichnis}{corpus molle}
non nisi in eo differunt,
quod resiliunt;
ambo ergo recipiunt in se vim percussionis,%
\protect\index{Sachverzeichnis}{vis percussionis}
mollia absorbent,%
\protect\index{Sachverzeichnis}{corpus molle}%
\protect\index{Sachverzeichnis}{corpus absorbens}
Elastica reddunt.%
\protect\index{Sachverzeichnis}{corpus elasticum}%
\protect\index{Sachverzeichnis}{corpus reddens}
\pend%
%
\pstart%
\edtext{}{%
{\xxref{LH_35_09_23_013v_hvgf-1}{LH_35_09_23_013v_hvgf-2}}%
{\lemma{\textit{Am Rand:}}\Afootnote{%
Inquirendum an hoc verum.\vspace{-3mm}%
% \newline%
}}}%
%
Si\edlabel{LH_35_09_23_013v_hvgf-1}
%
\edtext{data sint}{%
\lemma{data}\Bfootnote{%
\textit{(1)}~sunt
\textit{(2)}~sint%
~\textit{L}}}
%
duo corpora
et datum sit,
%
\edtext{quid
altero in alterum
velut quiescens incurrente%
\protect\index{Sachverzeichnis}{corpus incurrens in quiescens}%
\protect\index{Sachverzeichnis}{corpus quiescens}
eventurum sit,}{%
\lemma{quid}\Bfootnote{%
\textit{(1)}~eveniat uno
\textit{(2)}~altero in \lbrack...\rbrack\ eventurum sit,%
~\textit{L}}}
%
dabitur etiam
quid in quocunque alio casu%
\protect\index{Sachverzeichnis}{casus incursus}
sit eventurum.%
\edlabel{LH_35_09_23_013v_hvgf-2}
%
Nam primo dabitur
%
\edtext{ex praecedentibus,}{%
\lemma{ex praecedentibus}\Cfootnote{%
Vgl. S.~\refpassage{LH_35_09_23_013r_alku_jwr-1}{LH_35_09_23_013r_alku_jwr-2}.%
}}
%
quid futurum sit,
si unum praecedat,%
\protect\index{Sachverzeichnis}{corpus praecedens}
alterum assequatur,%
\protect\index{Sachverzeichnis}{corpus assequens}
tantum enim addendus est motus communis.%
\protect\index{Sachverzeichnis}{motus communis}
Nunc
%
\edtext{ostendam tantum,
quod hinc etiam duci possit}{%
\lemma{ostendam}\Bfootnote{%
\hspace{-0,5mm}tantum,
\textit{(1)}~quod futurum sit
\textit{(2)}~quod hinc etiam duci possit%
~\textit{L}}}
%
casus occursuum.%
\protect\index{Sachverzeichnis}{casus occursus}
Quando corpora duo occurrunt sibi,%
\protect\index{Sachverzeichnis}{corpora occurrentia}
tunc vel est aequalis potentia eorum,%
\protect\index{Sachverzeichnis}{potentia corporum occurrentium}
vel inaequalis,
si aequalis
constat
%
\edtext{\lbrack ea\rbrack}{%
\lemma{ea}\Bfootnote{\textit{erg. Hrsg.}}}
%
eadem celeritate repelli,%
\protect\index{Sachverzeichnis}{celeritas repulsae}
sin inaequalis,
tunc pars potentiae majoris aequalis
%
\edtext{\lbrack minori\rbrack}{%
\lemma{minoris}\Bfootnote{%
\textit{L~ändert Hrsg.}}}
%
potentiae,%
\protect\index{Sachverzeichnis}{potentia corporum occurrentium}
eique juncta,
id est potentia
%
\edtext{\lbrack minor\rbrack}{%
\lemma{minoris}\Bfootnote{%
\textit{L~ändert Hrsg.}}}
%
\edtext{duplicata dabit partem conatus}{%
\lemma{duplicata}\Bfootnote{%
\textit{(1)}~erit jam conatus
\textit{(2)}~dabit partem conatus%
~\textit{L}}}
%
recedendi,%
\protect\index{Sachverzeichnis}{conatus recedendi}
%
\edtext{et corpus}{%
\lemma{et}\Bfootnote{% \hspace{-0,5mm}%
\textit{(1)}~reliqua
\textit{(2)}~corpus%
~\textit{L}}}
%
fortius differentia potentiarum%
\protect\index{Sachverzeichnis}{differentia potentiarum}
seu excessu suae potentiae%
\protect\index{Sachverzeichnis}{excessus potentiae}
praeterea in debilius
velut quiescens%
\protect\index{Sachverzeichnis}{corpus quiescens}
agere intelligetur,
unde rursus separatim
partim percussio%
\protect\index{Sachverzeichnis}{percussio}
partim conatus abreptionis,%
\protect\index{Sachverzeichnis}{conatus abreptionis}
quibus inter se junctis,
regulae etiam occursuum habentur.%
\protect\index{Sachverzeichnis}{regulae occursuum}
Ergo%
\protect\index{Sachverzeichnis}{regulae incursuum}%
\textls{ datis regulis incursuum in quiescens
dantur regulae concursuum omnes.}%
\protect\index{Sachverzeichnis}{regulae concursuum}
%
\lbrack14~r\textsuperscript{o}\rbrack\  %  %  %  %  Blatt 14r
%
\pend%
%\newpage%
%
\pstart%
\edtext{}{%
{\xxref{LH_35_09_23_014r_elastictus-1}{LH_35_09_23_014r_elastictus-2}}%
\lemma{Quando \lbrack...\rbrack\ divisa}\Cfootnote{%
Ähnliche Ausführung in dem von Leibniz eigenhändig auf den 11. Juni 1677 datierten Text N.~\ref{RK57271} \textit{De vi ictus}, S.~\refpassage{37_05_159-160_8a}{37_05_159-160_8b}.%??M??
}}%
\edtext{%
\edlabel{LH_35_09_23_014r_percussioinquiescens}%
\edlabel{LH_35_09_23_014r_elastictus-1}%
Quando corpus incurrit in aliud quiescens%
\protect\index{Sachverzeichnis}{corpus incurrens in quiescens}%
\protect\index{Sachverzeichnis}{corpus quiescens}
tunc}{%
\lemma{Quando}\Bfootnote{%
\hspace{-0,5mm}corpus \lbrack....\rbrack\ quiescens tunc
\textit{erg.~L}}}%
%
\textls{ Percussio semper aequalis est vi
quam corpus incurrens transferret in excipiens,
si percussio abesset.}%
\protect\index{Sachverzeichnis}{corpus excipiens}%
\protect\index{Sachverzeichnis}{percussio}
\pend%
%\newpage%
%
\pstart%
Nam corpus%
\protect\index{Sachverzeichnis}{corpus impingens}
%
\edtext{impingens in aliud
vim suam%
\protect\index{Sachverzeichnis}{vis corporis impingentis}
in duo exercere potest,}{%
\lemma{impingens}\Bfootnote{%
\hspace{-0,5mm}in
\textit{(1)}~minus duo facere potest
\textit{(2)}~aliud vim % suam in duo 
\lbrack...\rbrack\ exercere potest,%
~\textit{L}}}
%
vel in corporum elaterium%
\protect\index{Sachverzeichnis}{elaterium corporum concurrentium}
sive flexibilitatem,%
\protect\index{Sachverzeichnis}{flexibilitas corporum concurrentium}
vel in tota.
Id est movere potest vel totum vel partes.
Facilius autem
%
\edtext{partes movere patet,}{%
\lemma{partes}\Bfootnote{%
\textit{(1)}~moveret
\textit{(2)}~movere patet,%
~\textit{L}}}
%
quia etsi cohaereant,%
\protect\index{Sachverzeichnis}{partes cohaerentes elaterio}%
\protect\index{Sachverzeichnis}{cohaerentia partium}
elaterio%
\protect\index{Sachverzeichnis}{elaterium corporum concurrentium}
tamen cohaerent,%
\protect\index{Sachverzeichnis}{partes cohaerentes elaterio}
quod minimo impulsui%
\protect\index{Sachverzeichnis}{impulsus minimus}
aliquantum cedet.
Facilius autem est flectere,
ideo flectetur seu tendetur ictu%
\protect\index{Sachverzeichnis}{ictus corporum concurrentium}
elaterium corporum concurrentium%
\protect\index{Sachverzeichnis}{elaterium corporum concurrentium}
eousque
%
\edtext{quousque facilius est}{%
\lemma{quousque}\Bfootnote{%
\textit{(1)}~non est diffic
\textit{(2)}~facilius est%
~\textit{L}}}
%
flectere quam movere excipiens;%
\protect\index{Sachverzeichnis}{corpus excipiens}
id est tota vis
quae transferenda est in impingens,%
\protect\index{Sachverzeichnis}{corpus impingens}
in Elaterium%
\protect\index{Sachverzeichnis}{elaterium corporis impingentis}
potius
%
\edtext{transfertur;
interea}{%
\lemma{transfertur;}\Bfootnote{%
\textit{(1)}~non quasi
\textit{(2)}~interea%
~\textit{L}}}
%
corpus ipsum eodem momento
quo conatum%
\protect\index{Sachverzeichnis}{conatus impressus}
quem corpori debebat
elaterio impressit,%
\protect\index{Sachverzeichnis}{elaterium corporis impacti}
cum amplius flectere non%
\protect\index{Sachverzeichnis}{flexio corporis impacti}
%
\edtext{possit,
pergit cum excipiente vi residua%
\protect\index{Sachverzeichnis}{vis residua}%
}{%
\lemma{possit,}\Bfootnote{%
\textit{(1)}~pergit vi residua
\textit{(2)}~pergit cum excipiente vi residua%
~\textit{L}}}
%
per summam corporum%
\protect\index{Sachverzeichnis}{summa corporum}
\edlabel{LH_35_09_23_014r_dfghs-1}%
divisa.%
\edlabel{LH_35_09_23_014r_elastictus-2}%
%
\edtext{}{%
{\xxref{LH_35_09_23_014r_dfghs-1}{LH_35_09_23_014r_dfghs-2}}%
{\lemma{divisa}\Bfootnote{%
\textit{(1)}~seu residua celeritate cui dimidium percussionis subtractum reliquit vim
\textit{(2)}~cujus vis
\textit{(3)}~. $\displaystyle\protect\underset{\displaystyle \epsilon,\ y}{\protect\underset{\displaystyle e,\ i\,}{a,\ b}}$ % $v^2$ aequ. \textit{ae} aequ. $ae+bi,$ seu \textit{i} 
\lbrack...\rbrack\ aequ. 0.
\textit{(a)}~\textit{p} seu vis percussionis $\displaystyle\frac{a}{a+b}.$
\textit{(b)}~Celeritas%
~\textit{L}}}}%
%
\edtext{}{%
\lemma{\textit{Am Ende des Absatzes:}}\Afootnote{%
Hoc verum repertum et in reformatione,%
\protect\index{Sachverzeichnis}{reformatio}
sed nescio an hinc bene concludatur
quia mutanda ratiocinatio de vi.%
\protect\index{Sachverzeichnis}{ratiocinatio de vi}\vspace{-3mm}
% \newline%
}}%
%
\pend%
%
\pstart%
$\displaystyle\underset{\displaystyle \epsilon,\ y}{\underset{\displaystyle e,\ i\,}{a,\ b}}$
\quad
$v^2$ aequ. \textit{ae} aequ. $ae+bi,$ seu \textit{i} aequ. 0.
Celeritas%
\edlabel{LH_35_09_23_014r_dfghs-2}%
\rule[-2mm]{0pt}{0mm}
qua,%
\protect\index{Sachverzeichnis}{celeritas post concursum}
si abesset percussio,%
\protect\index{Sachverzeichnis}{percussio omissa}
corpora simul ferrentur
\rule[-2mm]{0pt}{6mm}%
$\displaystyle\frac{a}{a+b}e.$
Unde hoc modo vis%
\protect\index{Sachverzeichnis}{vis impressa}%
\protect\index{Sachverzeichnis}{vis corporis excipientis}
corpori excipienti \textit{b}%
\protect\index{Sachverzeichnis}{corpus excipiens}
impressa esset
\rule[-2mm]{0pt}{6mm}%
$\displaystyle\frac{ab}{a+b}e$ et vis corporis incurrentis seu residua
\rule[-2mm]{0pt}{6mm}%
%
\edtext{$\displaystyle\frac{a^2}{a+b}[e]$.}{%
\lemma{\textit{e\,~erg.}}\Bfootnote{\hspace{-0,5mm}%
\textit{Hrsg. \mbox{nach}~E,\cite{01056} S.~119}}}
%
Quia vero vis percussionis%
\protect\index{Sachverzeichnis}{vis percussionis}
est eadem cum vi
quae corpori excipienti
\rule[-2mm]{0pt}{6mm}%
si percussio abesset
imprimeretur,
hinc erit vis percussionis%
\protect\index{Sachverzeichnis}{vis percussionis}
$p^2\;\sqcap \;\displaystyle\frac{ab}{a+b}e$%
\rule[-2mm]{0pt}{7mm}
quae,
quod notabile est,
eodem modo refertur ad utrumque corpus,
et post
%
\edtext{percussionem
residua vis erit $v^2-p^2$
\lbrack seu\rbrack\
$ae-\displaystyle\frac{ab}{a+b}e,$%
\protect\rule[-2mm]{0pt}{6mm}
seu
\protect\rule[-2mm]{0pt}{6mm}%
$\displaystyle\frac{a^2e+abe-abe}{a+b}$ aequ.
\protect\rule[-2mm]{0pt}{6mm}%
$\displaystyle\frac{a^2}{a+b}e\,
\sqcap\, v^2-p^2,$
quae}{%
\lemma{percussionem}\Bfootnote{%
\textit{(1)}~totum procedet
\textit{(2)}~residua vis erit
\textbar~$v^2-p^2$ \textit{erg.}
\textbar~seu \textit{erg. Hrsg.}~%
\textbar\ $ae-\displaystyle\frac{ab}{a+b}e,$ seu $\displaystyle\frac{a^2e+abe-abe}{a+b}$ aequ. $\displaystyle\frac{a^2}{a+b}e$
\textbar~$\sqcap\ \,v^2-p^2$ \textit{erg.}~%
\textbar~, quae%
~\textit{L}}}
%
divisa per $a+b$
dabit celeritatem
qua, omissa percussione,%
\protect\index{Sachverzeichnis}{percussio omissa}
pergerent,%
\protect\index{Sachverzeichnis}{celeritas post concursum}
nempe
\rule[-2mm]{0pt}{6mm}%
$\displaystyle\frac{a^2}{a^2+2ab+b^2}e$
%
\edtext{aequ. $\pi,$ et vis in \textit{a} erit}{%
\lemma{aequ. $\pi,$}\Bfootnote{%
\textit{(1)}~ab hac vi
\textit{(2)}~et vis in \textit{a} erit%
~\textit{L}}}
%
\rule[-2mm]{0pt}{6mm}%
$\displaystyle\frac{a}{a+b}\fbox{$2$}\, ,\!, \,ae$ aequ. $\pi a,$
vis in \textit{b} erit
\rule[-2mm]{0pt}{6mm}%
%
\edlabel{LH_35_09_23_014r_tzgd-1}%
$\displaystyle\frac{a}{a+b}\fbox{$2$}\, ,\!,be$ aequ. $\pi b.$%
%
\edtext{}{%
{\xxref{LH_35_09_23_014r_tzgd-1}{LH_35_09_23_014r_tzgd-2}}%
{\lemma{aequ. $\pi b$}\Bfootnote{%
\textit{(1)}~si ab illa subtrahas $\displaystyle\frac{ab}{2,a+b}e$ et ab hac tantundem fiet illic $\displaystyle\frac{2a^3-a^2b-ab^2}{2,a^2+2ab+b^2}$ pro
\textit{(2)}~. Sive differentia
\textit{(3)}~. Jam corpus \textit{a} duos habet conatus
\textit{(a)}~, unum
\textit{(b)}~contrarios, unum
\textbar~pergendi \textit{erg.}~%
\textbar\ celeritate $\pi,$ alterum
\textbar~regrediendi \textit{erg.}~%
\textbar\ celeritate
\textit{(aa)}~$\displaystyle\protect\frac{ab}{\phantom{2}},$
\textit{(bb)}~$\displaystyle\frac{p^2}{2a},$%
~\textit{L}}}}
%
\pend%
%
\pstart%
Jam corpus \textit{a} duos habet conatus contrarios,%
\protect\index{Sachverzeichnis}{conatus contrarius}%
\rule[-2mm]{0pt}{6mm}
unum pergendi%
\protect\index{Sachverzeichnis}{conatus pergendi}
celeritate $\pi,$
alterum regrediendi%
\protect\index{Sachverzeichnis}{conatus regrediendi}
celeritate $\displaystyle\frac{p^2}{2a},$%
\edlabel{LH_35_09_23_014r_tzgd-2}%
\rule[-2mm]{0pt}{6mm}
seu dimidia vi percussionis.%
\protect\index{Sachverzeichnis}{vis percussionis}
Ergo pergit vel regreditur eorum differentia%
\protect\index{Sachverzeichnis}{differentia conatuum}
%
\edtext{prout alteruter}{%
\lemma{prout}\Bfootnote{%
\textit{(1)}~alterutrum
\textit{(2)}~alteruter%
~\textit{L}}}
%
vincit;%
\protect\index{Sachverzeichnis}{conatus vincens}
eritque
%
$\epsilon$ aequ.
$\pleibdashv\!\protect\underset{\displaystyle\overbrace{\displaystyle\frac{v^2 - p^2}{a + b}}}{\pi}$\!
%
\edtext{$\pleibvdash \displaystyle\frac{p^2}{2a},$
seu
$\displaystyle\frac{\leibdashv\, 2av^2\,\leibvdash\, 2ap^2\,\leibvdash\, ap^2 \,\leibvdash\, bp^2}{[2a,]\, a+b},$
seu
$\displaystyle\frac{\leibdashv\, 2av^2\,\leibvdash\, 3ap^2\,\leibvdash \, bp^2}{[2a,]\, a+b},$
seu explicando}{%
\lemma{$\pleibvdash\displaystyle\frac{p^2}{2a},$}\Bfootnote{% \hspace{-0,5mm}%
\textit{(1)}~\textbar~seu \textit{streicht Hrsg.}~%
\textbar\ quae
\textit{(2)}~seu 
\textbar~$\displaystyle\frac{\leibdashv\, 2av^2\,\leibvdash\, 2ap^2\,\leibvdash\, ap^2\,\leibvdash\, bp^2}{a+b}$ \textit{ändert Hrsg.}~%
\textbar~, seu
\textbar~$\displaystyle\frac{\leibdashv\, 2av^2\,\leibvdash\, 3ap^2\,\leibvdash\, bp^2}{a+b}$ \textit{ändert Hrsg.}~%
\textbar~, seu explicando%
~\textit{L}}}
%
$\leibdashv\,\displaystyle\frac{a}{a+b}\fbox{$2$}\, e\,\pleibvdash\, \displaystyle\frac{b}{2,\; a+b}e$
\rule[-2mm]{0pt}{6mm}%
\edtext{sive
$\displaystyle%
\frac{\leibdashv\,2a^2\; \leibvdash\,ab\; \leibvdash\,b^2}{2,\,\overline{a+b}^2}e$
aequ.
$\epsilon.$
\protect\rule[-2mm]{0pt}{6mm}%
Ubi patet corpus \textit{a}
incurrens%
\protect\index{Sachverzeichnis}{corpus incurrens}
continuare progredi
si $2a^2$ major}{%
\lemma{sive}\Bfootnote{%
\textit{(1)}~$\displaystyle\frac{\leibdashv\,2a^2\; \leibvdash\,ab\; \leibvdash\,b^2}{a+b}$
\textit{(2)}~$\displaystyle\frac{\leibdashv\,2a^2\; \leibvdash\,ab\; \leibvdash b^2}{2,\,\overline{a+b}^2}e$
\textit{(a)}~; ubi patet cor
\textit{(b)}~aequ. $\epsilon.$ % Ubi patet corpus 
\lbrack...\rbrack\ \textit{a} incurrens
\textit{(aa)}~repelli si $2a^2$ majo
\textit{(bb)}~continuare
\textbar~progredi \textit{erg.}~%
\textbar\ si $2a^2$ major%
~\textit{L}}}
%
quam
%
\edtext{$ab+b^2,$
quod fit
quando corpus \textit{a} majus}{%
\lemma{$ab+b^2,$}\Bfootnote{%
\textit{(1)}~seu si
\textit{(2)}~quod fit quando
\textit{(a)}~$a\,+$
\textit{(b)}~corpus \textit{a} majus%
~\textit{L}}}
%
quam \textit{b}.
Sit enim \textit{a} majus quam \textit{b},
erit utique $a^2$ majus quam $b^2,$ auferendo
%
\edtext{ergo illinc $a^2,$
\lbrack hinc\rbrack\ $b^2$}{%
\lemma{ergo}\Bfootnote{%
\textit{(1)}~hinc $a^2$
\textit{(2)}~illinc $a^2,$
\textbar~hic \textit{ändert Hrsg.}~%
\textbar\ $b^2$%
~\textit{L}}}
%
manebit illic $a^2$ hic \textit{ab};
est autem $a^2$
majus quam \textit{ab},
ergo et erit $a^2\!+a^2$
majus quam
\edlabel{LH_35_09_23_013-014_14r1}%
$ab+b^2.$%
\edtext{}{%
{\xxref{LH_35_09_23_013-014_14r1}{LH_35_09_23_013-014_14r2}}%
{\lemma{$ab+b^2.$}\Bfootnote{%
\textit{(1)}~Contra videamus quid sit futurum
\textit{(2)}~Ergo quandocunque \lbrack...\rbrack\ futurum sit%
~\textit{L}}}}
\pend%
%
\pstart%
Ergo quandocunque corpus \textit{a}
est majus quam \textit{b},%
\protect\index{Sachverzeichnis}{corpus majus in minus}
progreditur post concursum.
Contra videamus quid futurum
sit%
\edlabel{LH_35_09_23_013-014_14r2}
%
si sit \textit{b} majus quam \textit{a},%
\protect\index{Sachverzeichnis}{corpus minus in majus}
an tunc quoque futurum sit
$ab+b^2$ majus quam
%
\edtext{$2a^2,$
\lbrack seu\rbrack\
an}{%
\lemma{$2a^2$}\Bfootnote{%
\textit{(1)}~. Utique si
\textit{(2)}~, seu quod
\textit{(3)}~, \textbar~seu \textit{erg. Hrsg.}~\textbar\ an%
~\textit{L}}}
%
$ab+b^2$ majus quam $a^2\!+a^2.$
Ita ajo,
nam \textit{ab} est majus quam unum $a^2,$
et $b^2$ majus quam alterum $a^2,$
%
\edtext{itaque tunc corpus \textit{a} incurrens repelletur.}{%
\lemma{itaque}\Bfootnote{%
\hspace{-0,5mm}%
tunc % corpus \textit{a} 
\lbrack...\rbrack\ incurrens repelletur.
\textit{erg.~L}}}
%
Hinc etiam statim colligi potest sine nova%
\textls{ ratiocinatione }%
\protect\index{Sachverzeichnis}{ratiocinatio}%
(\protect\vphantom)%
singulari tamen%
\textls{ concludendi }%
%
\edtext{genere%
\protect\index{Sachverzeichnis}{genus concludendi}\protect%
\vphantom()
quando}{%
\lemma{genere\protect\vphantom()}\Bfootnote{%
\textit{(1)}~est
\textit{(2)}~quando%
~\textit{L}}}
%
\textit{a} et \textit{b} aequales
tunc corpus incurrens%
\protect\index{Sachverzeichnis}{corpus incurrens}
%
\edtext{non progredi nec repelli,
sed quiescere,%
\protect\index{Sachverzeichnis}{corpus quiescens}%
}{%
\lemma{non}\Bfootnote{%
\textit{(1)}~denique quando \textit{a} et \textit{b} aequales, tunc etiam
\textit{(2)}~progredi nec repelli, sed quiescere,%
~\textit{L}}}
%
idem ex
%
\edtext{calculo%
\protect\index{Sachverzeichnis}{calculus}
reapse facto}{%
\lemma{calculo}\Bfootnote{%
\textit{(1)}~separato
\textit{(2)}~reapse facto%
~\textit{L}}}
%
constat,
nam fiet
$\pleibdashv 2a^2\; \pleibvdash a^2\; \pleibvdash a^2$
id est 0.
\pend%
%
\pstart%
%
\edtext{%
Est ergo vis continuationis%
\protect\index{Sachverzeichnis}{vis continuationis}
\protect\rule[-2mm]{0pt}{0mm}%
$\displaystyle\frac{2a^2-ab-b^2}{2,\overline{a+b}^2}e$
aequ. $\epsilon,$
et vis repulsae%
\protect\index{Sachverzeichnis}{vis repulsae}
\protect\rule[-2mm]{0pt}{0mm}%
$\displaystyle\frac{-2a^2+ab+b^2}{2,\overline{a+b}^2}e$
aequ. $\epsilon.$%
%
}{%
\lemma{\textit{Am Rand:}}\Afootnote{%
Haec accuratius examinanda,
nam subest error.%
\protect\index{Sachverzeichnis}{error}\vspace{-3mm}%
%\newline%
}}
%
Porro hoc conflictu duorum conatuum,%
\protect\index{Sachverzeichnis}{conflictus conatuum}
\rule[-4mm]{0pt}{6mm}%
pergendi%
\protect\index{Sachverzeichnis}{conatus pergendi}
et resiliendi,%
\protect\index{Sachverzeichnis}{conatus resiliendi}
seu $\pi$ et $\displaystyle\frac{p^2}{2a},$
destruitur minor ex ipsis,%
\protect\index{Sachverzeichnis}{conatus destructus}
et quidem bis;
ideoque transfertur
in alterum%
\protect\index{Sachverzeichnis}{conatus translatus}
\edlabel{LH_35_09_23_013-014_14r3}%
corpus.%
%
\edtext{}{%
{\xxref{LH_35_09_23_013-014_14r3}{LH_35_09_23_013-014_14r4}}%
{\lemma{corpus.}\Bfootnote{%
\textit{(1)}~Porro differentia corporum addita minori facit corpus
\textit{(2)}~Hinc
\textit{(a)}~totum,
\textit{(b)}~majus,
\textit{(3)}~Hinc si a tota vi
\textit{(4)}~Quod ut generali formula complectamur sic agemus, dicemus,
\textit{(a)}~in corpus excip
\textit{(b)}~vim destructam in corpus excipiens transferendam esse.
\textit{(5)}~Est ergo vis illa
\textit{(a)}~vel $2\pi a$ vel $p^2$
\textit{(b)}~vel $\pi \smallfrown 2a$ % vel $\displaystyle\frac{p^2}{2a}\smallfrown 2a,$ seu $2\pi a,$ 
\lbrack...\rbrack\ vel $p^2.$
\textit{(aa)}~Scribatur $2a$
\textit{(bb)}~Sint duo, \textit{y} et \textit{x} et volumus generali formula exprimere alterutrum eorum esse sumendum, hoc ita puto fieri poterit: $\pleibdashv\,by\; \pleibdashv\,cx\; \pleibvdash\,ey \;\pleibvdash\,fx$
\textit{(cc)}~Sit $z\; \pleibdashv\,x$
\textit{(dd)}~Sit $\pleibdashv$ aequ. $+$
\textbar~et $\pleibvdash$\,\lbrack\textit{!}\rbrack\ \textit{erg.}~%
\textbar\ et formula debet esse aequ. \textit{y} fiet $+\, by + cx - ey - fx$ aequ. \textit{y}.
Ergo $\pleibdashv x\; \pleibvdash y$
\textit{(ee)}~Si $+\,x-y$ aequ. $\, \pleibdashv\,x\; \pleibvdash\,y$ erit $z\sqcap x.$%
~\textit{L}}}}
% \pend%
% %
% \pstart%
Est ergo vis illa vel $\pi \smallfrown 2a,$
vel $\displaystyle\frac{p^2}{2a}\smallfrown 2a,$
\rule[-4mm]{0pt}{6mm}%
seu $2\pi a,$ vel $p^2.$
\pend%
%
\pstart%
Si $+\,x-y$ aequ. $\pleibdashv x\; \pleibvdash y,$
erit $z \,\sqcap\, x.$%
\edlabel{LH_35_09_23_013-014_14r4}
% \pend%
% %
% \pstart%
%
Si $-\,x+y$ aequ.
$\pleibdashv x\; \pleibvdash y$
erit
\textit{z} aequ. \textit{y}.
Multipli%
%
\lbrack14~v\textsuperscript{o}\rbrack    %  %  %  %    Blatt 14 v
%
centur in se invicem haec duae aequationes%
\protect\index{Sachverzeichnis}{aequatio}
$\displaystyle\efrac{\leibdashv\, x\; \leibvdash\, y\ \ \text{aequ.}\, + x - y}{\leibdashv\, x\; \leibvdash\, y\ \ \text{aequ.}\, - x + y};$
\newline%
seu brevius ponatur
\rule[-3mm]{0mm}{8mm}%
$\pleibdashv x\; \pleibvdash y$ aequ. $\omega,$
fiet
\pend%
\vspace{-1.95em}%
%
\pstart%
\noindent%
\hspace*{42mm}%
%
$\begin{array}{lllllr}
&&+\omega&-x&+y &\text{ aequ. }0\\
&&+\omega&+x&-y &\text{ aequ. }0\\
\cline{3-5}\rule{0pt}{10pt}
&&-y\omega & +xy & +y^2&\\
& +x\omega && +xy & -x^2&\\
+\omega^2 & -x\omega & +y\omega &&&
\end{array}$
%
\pend%
% \vspace{-0.45em}%
%
\pstart%
\noindent%
seu fiet $\omega^2$ aequ. $y^2 + x^2-2xy.$%
\rule[-3mm]{0mm}{8mm}
% \pend%
% %
% \pstart%
Porro \textit{z} aequ. \textit{x},
si $\pleibdashv x\; \pleibvdash y$ aequ. $+\,x-y.$
Ergo tunc $+\,x\; \pleibvdash x$ aequ. $+\,y\; \pleibvdash y$
seu \textit{x} aequ. $\displaystyle\frac{1+\leibvdash}{1+\leibvdash}\, y$
%
\edtext{(aequ. \textit{z})}{%
\lemma{(aequ. \textit{z})}\Bfootnote{%
\textit{erg.~L}}}
%
unde fit vel \textit{x} aequ. \textit{y} vel $1\, +\, \pleibvdash$ aequ. 0.
\pend%
%\newpage%
%
\pstart%
% \noindent%
\edtext{%
%
Contra \textit{z} aequ. \textit{y} si $\pleibvdash\,x\; \pleibvdash\,y$ aequ. $-x+y.$
Ergo tunc \textit{y} aequ. $\displaystyle\frac{1+\,\leibdashv}{1+\,\leibdashv}\, x$
%
\edtext{(aequ. \textit{z})}{%
\lemma{(aequ. \textit{z})}\Bfootnote{%
\textit{erg.~L}}}
%
id est
vel \textit{y} aequ. \textit{x}
vel $1+\,\pleibdashv$ aequ. 0.%
%
}{%
\lemma{\textit{Am Rand:}}\Afootnote{%
NB.
Non semper $\displaystyle\frac{a}{a}$
facit\textsuperscript{[a]} 1,
cum scilicet \textit{a} aequ. 0.
Paradoxum mirabile.%
\protect\index{Sachverzeichnis}{paradoxon mirabile}
\newline\vspace{-0.4em}%
\newline%
{\footnotesize%
\textsuperscript{[a]}~facit 1
\textit{(1)}~sed
\textit{(2)}~cum%
~\textit{L}%
%\newline%
}}}
%
\pend%
%
\pstart%
\noindent%
Ergo \textit{z} aequ. $\displaystyle\frac{1+\,\leibvdash}{1+\,\leibvdash}\, y$
%
\quad
vel \textit{z} aequ.
%\edtext{}{%
%\lemma{\textit{y}}\Bfootnote{%
%\textit{(1)}~aequ.
%\textit{(2)}~vel \textit{z} aequ.%
%~\textit{L}}}
%
$\displaystyle\frac{1+\,\leibdashv}{1+\,\leibdashv}\, y$
\quad
seu\,%
$\begin{array}{rl} 
\!z - \displaystyle\frac{1+\,\leibvdash}{1+\,\leibvdash}\, y & \!\text{aequ. }0. \\
\!z - \displaystyle\frac{1+\,\leibdashv}{1+\,\leibdashv}\, x & \!\text{aequ. }0. 
\end{array}$
%
\newline%
\hspace*{-1,66mm}%
\raisebox{-3,8mm}{%
$\begin{array}{rll}
z^2 & -\ \displaystyle\frac{1 + \leibvdash}{1+\leibvdash}\, y z & \displaystyle\frac{+1+ \,\leibvdash+\,\leibdashv-1}{+1+\,\leibvdash + \,\leibdashv-1}\, yx \\
& -\ \displaystyle\frac{1+\leibdashv}{1+\leibdashv}\, x\; \cdot &
\end{array}$
}
%\newline%
\quad
sed videtur hic \textit{z} destrui.
\rule[-14mm]{0mm}{20mm}%
%\edtext{}{%
%\lemma{\textit{Am Rand:}}\Afootnote{%
%$\pleibdashv\ \text{aequ.}\; -\; \pleibvdash$}}
%% \raisebox{-0.6ex}{$\leibdashv$} aequ. $-$\raisebox{-0.6ex}{$\leibvdash$}
%%%%
\newline%
%\quad
$z^2
\protect\begin{array}[t]{c}
-\; \displaystyle\frac{m}{\vphantom{Ig}m}\,yz \\
-\; \displaystyle\frac{\vphantom{I}n}{n}\,x\;\cdot
\protect\end{array}\!\!
+\; \displaystyle\frac{mn}{mn}\, xy
\;\sqcap\; 0.$
\quad
$\protect\begin{array}[t]{ccc}
m& \!\!\!\!\sqcap\!\!\!\! &1+\,\raisebox{-1.0ex}{\leibvdash}\\
n& \!\!\!\!\sqcap\!\!\!\! &1+\,\raisebox{-1.0ex}{\leibdashv}\\
\protect\end{array}$
\quad
$\pleibdashv\ \text{aequ.}\; -\; \pleibvdash$
\rule[-10mm]{0mm}{16mm}%
%%%%
\newline%
% \indent%
Ergo
\edlabel{LH_35_09_23_013-014_14v5}%
%
\edtext{}{%
{\xxref{LH_35_09_23_013-014_14v5}{LH_35_09_23_013-014_14v6}}%
{\lemma{\textit{z} aequ. \lbrack...\rbrack\ alter \textit{y}}\Cfootnote{Die Bildung sowie die Lösung der quadratischen Gleichung in \textit{z} sind fehlerhaft.}}}%
%
\textit{z} aequ. $\pm \sqrt{ \left\{ \begin{matrix} \displaystyle\frac{\displaystyle \frac{2\; \leibvdash \;2}{2\; \leibvdash \;2}\,y^2 + \displaystyle\frac{2\; \leibdashv \;2}{2\; \leibdashv \;2}\,x^2 + \displaystyle\frac{+1 + \leibvdash + \leibdashv -1}{+1 +  \leibvdash + \leibdashv -1}\,2xy}{\displaystyle 4} \\
\rule{0pt}{22pt}
-\displaystyle\frac{+\,1 + \leibvdash + \leibdashv - 1}{+\,1 + \leibvdash + \leibdashv - 1} xy \end{matrix}\right.}
+\displaystyle\frac{1 + \leibvdash}{1 + \leibvdash}\,y\; + \displaystyle\frac{1 + \leibdashv}{1 + \leibdashv}\,x.$%
%
\edtext{}{%
\lemma{\textit{Nebenrechnung:}}\Afootnote{%
(\protect\vphantom)%
$z^2+rz+s$ fiet $z^2-rz+\displaystyle\frac{r^2}{4}\sqcap\displaystyle\frac{r^2}{4}-s$ et \textit{z} aequ. $\sqrt{\displaystyle\frac{r^2}{4}-s}+\displaystyle\frac{r}{2}$%
\lbrack\protect\vphantom()\rbrack\
%\newline%
}}
\pend%
%
\pstart%
\noindent%
Itaque \textit{z} habet duos valores%
\protect\index{Sachverzeichnis}{valor aequationis}%
\edlabel{LH_35_09_23_013-014_14v6}
\rule[-4mm]{0mm}{10mm}%
quorum unus debet facere \textit{x} alter \textit{y}.
%\newline%
%$z^2
%\protect\begin{array}[t]{c}
%-\; \displaystyle\frac{m}{\vphantom{Ig}m}\,yz \\
%-\; \displaystyle\frac{\vphantom{I}n}{n}\,x\;\cdot
%\protect\end{array}\!\!
%+\; \displaystyle\frac{mn}{mn}\, xy
%\;\sqcap\; 0.$
%\quad
%$\protect\begin{array}[t]{ccc}
%m& \!\!\!\!\sqcap\!\!\!\! &1+\,\raisebox{-1.0ex}{\leibvdash}\\
%n& \!\!\!\!\sqcap\!\!\!\! &1+\,\raisebox{-1.0ex}{\leibdashv}\\
%\protect\end{array}$
\pend%
\count\Bfootins=1200%
\count\Afootins=1200%
\count\Cfootins=1200
\pstart%
%\noindent%
Si
\edlabel{LH_35_09_23_013-014_14v1}%
%\edtext{}{%
%{\xxref{LH_35_09_23_013-014_14v1}{LH_35_09_23_013-014_14v2}}%
%{\lemma{aequationem}\Bfootnote{%
%\textit{(1)}~multiplicemus per
%\textit{(2)}~$z^2\protect\begin{array}[t]{c} -\displaystyle\frac{m}{\phantom{I}m}yz \\ -\displaystyle\frac{\phantom{I}n}{n}x\;\cdot \protect\end{array} +\displaystyle\frac{mn}{mn}\, xy\;\sqcap\; 0$ multiplicemus per \textit{mn}%
%~\textit{L}}}}%
%
aequationem%
\protect\index{Sachverzeichnis}{aequatio}
$z^2
\protect\begin{array}[t]{c}
-\; \displaystyle\frac{m}{\vphantom{Ig}m}\,yz \\
-\; \displaystyle\frac{\vphantom{I}n}{n}\,xz
\protect\end{array}\!\!
+\; \displaystyle\frac{mn}{mn}\, xy
\;\sqcap\; 0$
multiplicemus per \textit{mn},%
\edlabel{LH_35_09_23_013-014_14v2}
%
%\newline%
fiet
$mn\, z^2-mn\, yz-mn\, xz+mn\, xy\; \sqcap\; 0.$
%
\edtext{}{%
\lemma{\textit{Am Rand:}}\Afootnote{%
\textit{z} aequ.
$\sqrt{\displaystyle\frac{y^2+x^2+2xy}{4}-xy}+\displaystyle\frac{y+x}{2}.$}}%
%
\rule[0mm]{0mm}{4mm}%
Ubi pro variis procedendi modis%
\protect\index{Sachverzeichnis}{modus procedendi}
varia prodibunt,
ut si omnia dividas per \textit{mn},
restabit:
$z^2\protect\begin{array}[t]{c}-yz\\-x\protect\end{array}+xy\; \sqcap\; 0,$%
\rule[-0mm]{0pt}{4mm}
quod ostendit
aequationis hujus%
\protect\index{Sachverzeichnis}{aequatio}
duos esse
%
\edtext{valores,%
\protect\index{Sachverzeichnis}{valor aequationis}
seu \textit{z} esse aequ. \textit{x} vel \textit{y},}{%
\lemma{valores,}\Bfootnote{%
\textit{(1)}~\textbar~seu \textit{z}
\textit{(a)}~esse \textit{x}
\textit{(b)}~vel \textit{y}~%
\textit{streicht Hrsg.}
\textit{(2)}~seu \textit{z} % esse aequ. \textit{x} 
\lbrack...\rbrack\ vel \textit{y},%
~\textit{L}}}
%
est autem
\textit{x} aequ.
$\displaystyle\frac{a}{a+b}\,\framebox{2}\, e\,$
et \textit{y} aequ.
$\displaystyle\frac{b}{2,\overline{a+b}}e,$%
\rule[-0mm]{0pt}{3mm}
fiet
$z^2\begin{array}[t]{c}-\displaystyle\frac{a}{a+b}\fbox{2}\,ez\\-\displaystyle\frac{b}{2,\overline{a+b}}\;\cdot\cdot\end{array}+\displaystyle\frac{a^2b}{2, \overline{a+b}\,\fbox{3}}\; e^2,$
%
\edtext{quae erit minor harum duarum, % \advanceline{1} 
\textit{x} vel \textit{y},
%
\lbrack ea\rbrack\
%
duplicata tribuenda est excipienti,%
\protect\index{Sachverzeichnis}{corpus excipiens}%
\protect\rule[-0mm]{0pt}{4mm}
quia destruitur in incurrente.%
\protect\index{Sachverzeichnis}{corpus incurrens}}{%
\lemma{quae}\Bfootnote{%
\hspace{-0,5mm}erit
\lbrack...\rbrack\ vel \textit{y},
\textbar~quae \textit{ändert Hrsg.}~%
\textbar\ duplicata tribuenda
\lbrack...\rbrack\ in incurrente
\textit{erg.~L}}}
%
\pend%
% \vspace{0.5em}%
\count\Bfootins=1200%
\count\Afootins=1200%
\count\Cfootins=1200
\pstart%
\textit{z} aequ.
$\displaystyle\frac{\sqrt{y^2+x^2-2xy}}{2}+\displaystyle\frac{y+x}{2},$%
\rule[-3mm]{0pt}{0mm}% \rule[-3mm]{0pt}{10mm}
seu
$2z$ aequ.
$\pleibdashv\, y\; \pleibvdash\, x + x + y,$
seu
$z\,\sqcap\,\displaystyle\frac{\leibdashv\, y\; \leibvdash\, x + x + y}{2},$
% \rule[-3mm]{0pt}{5mm}
id est \textit{z} aequ. \textit{x} vel \textit{y},
quod est pulcherrimum.
Et ratio cur initio statim non invenerim est,
quod posui tantum arbitrarias
cum ambiguis signis,%
\protect\index{Sachverzeichnis}{signum ambiguum}
non vero cum alteris,
v.g. debuissem dicere:
\textit{z} aequ. $\pleibdashv\, \alpha y\; \pleibvdash\, \beta y+\gamma y-\delta y\; \pleibdashv\, \eta x\; \pleibvdash\, \theta x+\kappa x-\lambda x,$
et postea
explicando literas arbitrarias%
\protect\index{Sachverzeichnis}{litera arbitraria}
hoc modo
ut posito \!$\pleibdashv$\! esse $+$ et \!$\pleibvdash$\! esse $-$
fiat \textit{z} aequ. \textit{x},
et posito contrario fiat \textit{y}.
Hoc patet effectu facile esse,
imo plus quam facile
ob arbitrariarum superabundantiam,
fieri enim potest infinitis modis,
ex quibus simplicissimus haud dubie iste est%
\lbrack;\rbrack\
ut autem aequationibus altioribus,%
\protect\index{Sachverzeichnis}{aequatio altior}
ubi plures sunt
quam tres radices%
\protect\index{Sachverzeichnis}{radix aequationis}%
\lbrack,\rbrack\
res procedat\lbrack,\rbrack\
efficere non licebit
nisi pluribus novis adhibitis signis,
quae sane inquirere operae pretium est.
Habemus ergo interim hoc
%
\edtext{}{%
{\xxref{LH_35_09_23_014v_notabiletheorema-1}{LH_35_09_23_014v_notabiletheorema-2}}%
{\lemma{theorema \lbrack...\rbrack\ minor}\Cfootnote{%
Der Satz ist falsch.
Vielmehr gilt:
Wenn $a > b$,
dann $\displaystyle \frac{(a + b) + (a - b)}{2} = a;$
wenn $a < b$,
dann $\displaystyle \frac{(a + b) - (a - b)}{2} = b.$
Leibniz selbst berichtigt sich in N.~\ref{dcc_06-2}, %??S01\textsubscript{8},
S.~\refpassage{LH_35_09_23_015r_notabiletheorema-1}{LH_35_09_23_015r_notabiletheorema-2};
siehe hierzu \textsc{Fichant} 1994, S.~123.\cite{01056}%
}}}%
%
\edtext{notabile
\edlabel{LH_35_09_23_014v_notabiletheorema-1}%
theorema:%
\protect\index{Sachverzeichnis}{theorema notabile}
Si differentia}{%
\lemma{notabile}\Bfootnote{%
\textit{(1)}~, si differen
\textit{(2)}~theorema: Si differentia%
~\textit{L}}}
%
addatur summae,
aggregati dimidium
%
\edtext{erit quantitas minor.%
\edlabel{LH_35_09_23_014v_notabiletheorema-2}}{%
\lemma{erit}\Bfootnote{%
\textit{(1)}~alterutra
\textit{(2)}~quantitas minor.%
~\textit{L}}}
%
Ejusmodi theoremata maximi sunt usus
ad faciendas propositiones universales.%
\protect\index{Sachverzeichnis}{propositio universalis}
\makebox[1.0\textwidth][s]{Itaque hoc loco
\edlabel{LH_35_09_23_013-014_14v3}%
$\underset{\displaystyle\textup{differentia}}{\epsilon}+\underset{\displaystyle\overbrace{\displaystyle\frac{v^2 - p^2}{a + b}}}{\pi}+\displaystyle\frac{p^2}{2a}$
aequ.
$2a,\overline{a+b}\, \epsilon + \underset{\displaystyle2a^3e}{2av^2} - \ovalbox{2}\,ap^2 \;\ovalbox{$+\,ap^2$}\, +\, bp^2,$}
\pend
\newpage
\pstart
\noindent
fiet%
\edlabel{LH_35_09_23_013-014_14v4}
% \edtext{}{{\xxref{14v3}{14v4}}\lemma{$+bp^2$}\Bfootnote{%
% \textit{(1)}~vel
% \textit{(2)}~fiet%
% ~\textit{L}}}
%
minor quantitas % \advanceline{2}
\rule[-5,5mm]{0pt}{12mm}%
%
\edtext{$\displaystyle\frac{\leibdashv\; 2a^2 + a^2 \;\leibvdash\; ab + ab \;\leibvdash\; b^2 + b^2}{2,\overline{a+b}^2},[e],$
%
quae multiplicata per \textit{a}
dat vim destructam%
\protect\index{Sachverzeichnis}{vis destructa}%
\lbrack,\rbrack\
in excipiens transferendam,%
\protect\index{Sachverzeichnis}{vis in excipiens transferenda}
seu divisa per \textit{b}
celeritatem excipientis%
\protect\index{Sachverzeichnis}{celeritas excipientis}
hinc acceptam.}{%
{\lemma{$\displaystyle\frac{\leibdashv \,2a^2+a^2\,\leibvdash\, ab+ab\,\leibvdash\, b^2+b^2}{2,\overline{a+b}^2},$}\Bfootnote{% \hspace{-0,5mm}%
\textbar~$\displaystyle\frac{ea}{b},$
\textit{erg., ändert Hrsg.}~\textbar\
\textit{(1)}~qu\textlangle ae\textrangle\ duplicata
\textit{(2)}~quae
\textit{(a)}~addenda ad
\textit{(b)}~multiplicata per \textit{a}
\textit{(3)}~quae
\textbar\textbar~quae \textit{streicht Hrsg.}~%
\textbar\ multiplicata per \textit{a}
\textbar~divisa per \textit{b} \textit{gestr.}~%
\textbar\ \textit{erg.}~%
\textbar\ dat vim destructam in excipiens 
\textit{(a)}~transferendam, 
\textit{(aa)}~\textlangle\textendash\textrangle\ autem
\textit{(bb)}~seu celeritatem excipientis, excipiens autem dudum habet vim
\textit{(b)}~transferendam, seu % divisa per \textit{b} celeritatem excipientis 
\lbrack...\rbrack\ hinc acceptam.%
~\textit{L}}}%
{\lemma{$\displaystyle\frac{\leibdashv \,2a^2+a^2\,\leibvdash\, ab+ab\,\leibvdash\, b^2+b^2}{2,\overline{a+b}^2},$ \lbrack...\rbrack\ acceptam}\Cfootnote{%%
Leibniz ergänzt den Faktor $\displaystyle\frac{ea}{b},$ beschreibt aber anschließend die Multiplikation mit \textit{a} und die Division durch \textit{b}.
Tatsächlich berücksichtigt er im Folgenden nur den Faktor \textit{e}.
Siehe hierzu \textsc{Fichant} 1994, S.~123.\cite{01056}%
}}}
%
Porro idem excipiens%
\protect\index{Sachverzeichnis}{corpus excipiens}
duas alias habet celeritates,%
\protect\index{Sachverzeichnis}{celeritas corporis excipientis}
unam $\pi,$
quae erat
\rule[-3mm]{0pt}{8mm}%
$\displaystyle\frac{a}{a+b}\,\fbox{2}\,,e,$
alteram
\rule[-3mm]{0pt}{8mm}%
$\displaystyle\frac{p^2}{2a},$
seu ponendo $\pi\; \sqcap\; x$ et
\rule[-3mm]{0pt}{8mm}%
$\displaystyle\frac{p^2}{2a}$
aequ. \textit{y},
habebit $x+\displaystyle\frac{a}{b}y.$%
\rule[-3mm]{0pt}{7mm}
Ergo habebit in summa celeritates
$2z+x+\displaystyle\frac{a}{b}y.$%
\rule[-3mm]{0pt}{7mm}
Est autem $2z$ aequ.
$\pleibdashv y\; \pleibvdash x + x + y$
cui addatur
$x+\displaystyle\frac{a}{b}y,$%
\rule[-3mm]{0pt}{7mm}
fiet:
$\pleibdashv y\; \pleibvdash x + 2x + y + \displaystyle\frac{a}{b}y$%
\rule[-3mm]{0pt}{7mm}
celeritas%
\protect\index{Sachverzeichnis}{celeritas corporis excipientis}
%
\edlabel{LH_35_09_23_013-014_14v7}%
\edtext{}{%
{\xxref{LH_35_09_23_013-014_14v7}{LH_35_09_23_013-014_14v8}}%
{\lemma{excipientis}\Bfootnote{%
\textit{(1)}~posito
\textit{(2)}~est autem $y+\displaystyle\frac{a}{b}y$
\textit{(3)}~posito \textit{x} % aequ. $\displaystyle\frac{a^2}{\overline{a+b}^2}e$ et \textit{y} 
\lbrack...\rbrack\ aequ. $\displaystyle\frac{b}{2,a+b}e$%
~\textit{L}}}}%
%
excipientis.%
\protect\index{Sachverzeichnis}{corpus excipiens}
Posito \textit{x} aequ.
\rule[-3mm]{0pt}{7mm}%
$\displaystyle\frac{a^2}{\overline{a+b}^2}e$
et
\textit{y} aequ.
\rule[-3mm]{0pt}{7mm}%
$\displaystyle\frac{b}{2,a+b}e,$%
\edlabel{LH_35_09_23_013-014_14v8}
%
fiet
%
\edtext{in summa\:}{%
\lemma{in summa\:}\Cfootnote{%
Das richtige Ergebnis ist:
$\displaystyle \frac{\pm\, b^2 \pm ab \mp 2a^2 + b^2 + 2ab + 5a^2}{2(a+b)^2}e.$
Vgl. N.~\ref{dcc_06-2}, %??S01\textsubscript{8} 
S.~\refpassage{LH_35_09_23_015r_Ergebnis_gje-1}{LH_35_09_23_015r_Ergebnis_gje-2}.%
}}
%
$\raisebox{-1.38ex}{$\begin{array}[b]{rrrl}
\leibdashv\, 1 & \smallfrown +\, ab & +\, 1\smallfrown +\, ab & \leibvdash\, 2 \smallfrown a^2\\
+\, 1 & +\, b^2 & + \,a^2 & + \,4\\
\cline{1-4}\\
\end{array}%
\hspace{-37,5mm}%
\protect\vphantom{a^{\frac{l}{l}}}%
\raisebox{-0.5ex}{$2, a^2 + 2ab + \,b^2$}%
$%
\hspace{15,5mm}%
\raisebox{1.5ex}{\textit{e}.}%
}$%
\edlabel{LH_35_09_23_014v_ende}
\pend%
\count\Bfootins=1200%
\count\Afootins=1200%
\count\Cfootins=1200
%
%
%  %  %  %    Ende des Textes auf Blatt 14v.
%
%