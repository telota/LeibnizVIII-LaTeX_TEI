%   % !TEX root = ../../VIII,3_Rahmen-TeX_9.tex
%  
%   Band VIII, 3		Rubrik STOSS
%
%   Signatur/Tex-Datei:	LH_35_14_02_007
%
%   RK-Nr. 	58221
%
%
%
%
\selectlanguage{ngerman}
\frenchspacing
%
\begin{ledgroupsized}[r]{120mm}
\footnotesize
\pstart
\noindent\textbf{Überlieferung:}
\pend
\end{ledgroupsized}
%
\begin{ledgroupsized}[r]{114mm}
\footnotesize
\pstart \parindent -6mm
\makebox[6mm][l]{\textit{L}}%
Konzept:
LH~XXXV~14, 2~Bl.~7. 
Ein Zettel (ca.~14,5~x~4~cm.);
Wasserzeichenfragment am Blattrand (italienisches Papier);
alle Ränder beschnitten.
Zwei Seiten.
\pend
\end{ledgroupsized}
%
%
\vspace{5mm}
\begin{ledgroup}
\footnotesize
\pstart
\noindent%
\textbf{Datierungsgründe:}
Wegen der Verwendung italienischen Papiers kann von einer Abfassung des Konzepts während des Italienaufenthaltes (März 1689 bis März 1690) ausgegangen werden.
%
\pend 
\end{ledgroup}
%
%
\selectlanguage{latin}
\frenchspacing
% \newpage%
\vspace{8mm}
\pstart%
\normalsize%
\noindent%
\lbrack7~r\textsuperscript{o}\rbrack\
%
$a\left\{%
\begin{array}{c|c}v&y\\%
x&z\end{array}%
\right\}b$.%
\quad 
%\pend
%
%\pstart\noindent
$z=\bigoplus\displaystyle\frac{av}{b}$. 
%
$x=\phccps\displaystyle\frac{by}{a}$.\quad
%
\edtext{Permutatio impetuum\lbrack;\rbrack\protect\index{Sachverzeichnis}{permutatio impetuum}}{\lemma{Permutatio}\Bfootnote{\textit{(1)}~impetum \textit{(2)}~impetuum~\textit{L}}}
%
videamus quid ex \rule[0cm]{0mm}{10pt}hoc sequatur. %
\pend
%
\pstart
%
$avv+byy$ an $=axx+\edtext{bzz$, potentiae\protect\index{Sachverzeichnis}{potentia}}{%
\lemma{}\Bfootnote{\textit{bzz}, \textbar\ ita sane \textit{gestr.} \textbar\ potentiae~\textit{L}}} 
%
servatio\protect\index{Sachverzeichnis}{servatio potentiae} succedit\lbrack?\rbrack\ Ita sane nam fit $avv+byy=byy+avv$. %
\pend
%
\pstart
%
$v\;\pleibdashv\; y=a+b$. $z=\;\pleibvdash\; y+\text{bis}\;a$.
%
$x=v-\text{bis}\;b$.
%
$z-x=a+b$.
%
$z-x=\text{bis}\;\overline{a+b}\;\pleibvdash\; y-v=a+b$. %
%
$\bigoplus av:b\phccms by:a = \bigoplus aav\phccms bby, :ab= \overline{a+b}$.%
\pend
%
\pstart
%
\lbrack7~v\textsuperscript{o}\rbrack\ 
Ergo $\bigoplus aav \phccms bby \stackrel{(1)}{=} ab\;\overline{a+b}$.\enskip Sed $a+b=v\;\pleibdashv\;y$.
%
Ergo fiet $\bigoplus aav \phccms bby = 
\edtext{ab\;\overline{v\;\leibdashv\;y}$.
Rursus $v\;\pleibdashv\;y \stackrel{(2)}{=} a+b$. Ope horum duorum}{\lemma{$\overline{v\;\leibdashv\;y}$.}%
\Bfootnote{\textit{(1)}~Sed hoc locum non habet, debuisset enim prodire aequatio anter \textit{(2)}~Rursus  \lbrack...\rbrack\ duorum~\textit{L}}}
%
debet prodire aequatio identica: $\pleibdashv\;y=a+b-v$.
%
Et in aequ.~1.\ fiet 
%
$\bigoplus aav \phccms \,\pleibdashv abb \phccms \,\pleibdashv b^3 \phccps \,\pleibdashv bbv = a^2b+ab^2$\lbrack,\rbrack\ 
%
quod succedere non potest nisi cum \textit{a} et \textit{b} aequales,
%
non ergo aliter locum habet impetus permutatio.\protect\index{Sachverzeichnis}{permutatio impetuum}\pend 
\count\Afootins=1200%
\count\Bfootins=1200%
\count\Cfootins=1200