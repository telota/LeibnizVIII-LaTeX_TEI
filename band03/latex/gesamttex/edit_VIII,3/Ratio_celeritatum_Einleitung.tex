%   % !TEX root = ../../VIII,3_Rahmen-TeX_9-0.tex
%  
%   Band VIII, 3 N.~?? 	Stoß
%   Gemeinsame Einleitung 
%   RK-Nr. 	57266_3 + 57268
%			
%   Titel: 			De ratione celeritatum ante et post concursum
%   Datierung:		[März bis Mai 1677]
%
%
%
\selectlanguage{ngerman}
\frenchspacing
%
%\vspace{5mm}
\begin{ledgroup}
\footnotesize
%
%
\pstart\noindent
Die Konzepte N.~\ref{57266_3} und N.~\ref{57268} hängen ihrem Inhalt wie ihrer Entstehung nach eng miteinander zusammen. 
%
Das Stück N.~\ref{57266_3} beinhaltet zwei Anläufe einer Untersuchung des Stoßes zweier Körper,
%
von denen der erste zu einem absurden Ergebnis führt und der zweite abbricht.
%
Es hat als (wahrscheinlich unmittelbare) Vorstufe für die Abfassung von  N.~\ref{57268} (eigh.\ auf Mai 1677 datiert)  gedient,
%
worin Leibniz im dritten Anlauf zu einem \glqq theorema memoria tenendum\grqq\ gelangt, das er zu diesem Zeitpunkt  für gültig erachtet.
%
Innerhalb von N.~\ref{57266_3} ist formal wie inhaltlich eine graduelle Entwicklung feststellbar: 
%
der erste Anlauf (S.~\refpassage{37_05_162v_11a}{37_05_162v_11b}) wurde nach der Niederschrift revidiert und um Elemente ergänzt,
%
die wiederum die Basis für den zweiten Anlauf (S.~\refpassage{37_05_162v_12a}{37_05_162v_12b}) bildeten.
%
Dies umfasst die Streichung oder Umarbeitung der Figuren, 
%
die Umbenennung einzelner Punkte (in den Diagrammen wie auch im Text) sowie 
%
eine formale Verschiebung in deren Bezeichnung von der Klammer- zur Indicesnotation
%
(von \textit{A}(\textit{A}) zu \textit{{\scriptsize 1}A{\scriptsize 2}A} u.ä.).
%
Einige Aspekte dieser Entwicklung reichen bis in den dritten Anlauf (N.~\ref{57268}) hinein:
%
So formuliert Leibniz innerhalb von N.~\ref{57266_3} die Annahmen immer klarer,
%
bis sie in N.~\ref{57268} unter dem Namen \glqq duae regulae\grqq\ als Prämissen für das gesamte Stück an dessen Anfang stehen;
%
die Diagramme von N.~\ref{57266_3} (\protect\vphantom)\lbrack\textit{Fig.~1}\rbrack\ und \lbrack\textit{Fig.~3}\rbrack\ 
%
im ersten Anlauf und \lbrack\textit{Fig.~4}\rbrack\ im zweiten\protect\vphantom() 
%
werden entwickelt und in die \lbrack\textit{Fig.~1}\rbrack\ und \lbrack\textit{Fig.~2}\rbrack\ N.~\ref{57268}  überführt.
%%
\pend
%
\pstart
Der festgestellte Zusammenhang zwischen den Stücken gibt Aufschluss über ihre Datierung.
%
Das Stück N.~\ref{57268} mit der Überschrift \glqq Specimina artis condendi theoremata\grqq, 
%
eine stark bearbeitete Handschrift, die neben N.~\ref{57267_1} und N.~\ref{57267_2} auf dem Bogen LH~XXXVII~5 Bl.~144\textendash145 überliefert ist, 
%
wurde von Leibniz auf Mai 1677 (a.\ St.)\ datiert (siehe die Randanmerkung auf S.~\refpassage{37_05_144r-145v_8a}{37_05_144r-145v_8b}). 
%
Das Stück N.~\ref{57266_3} mit dem editorisch zugewiesenen Titel \glqq De ratione celeritatum ante et post concursu\grqq\ 
%
wird neben N.~\ref{57266_1} und N.~\ref{57266_2} auf dem Bogen LH~XXXVII~5 Bl.~161\textendash162 überliefert. 
%
N.~\ref{57266_1}, eigenhändig auf März 1677 (a.\ St.)\ datiert, ist auf den ersten drei Seiten des Bogens überliefert (Bl.~161~r\textsuperscript{o} bis 162~r\textsuperscript{o}), N.~\ref{57266_3} auf der vierten (Bl.~162~v\textsuperscript{o}).
%
Unter Annahme einer durchgehenden Beschreibung des Bogens und damit einer Entstehung von N.~\ref{57266_3} nach N.~\ref{57266_1} und N.~\ref{57266_2} ergibt sich der Terminus post quem:  März 1677.
%
Aus dem festgestellten Zusammenhang mit dem eigh.\ auf Mai 1677 datierten N.~\ref{57268} ergibt sich insgesamt die Datierungsspanne März bis Mai 1677.
\pend
%
\pstart
In beiden Stücken nimmt  Leibniz sich eine allgemeine Analyse des geradlinigen zentralen Stoßes beliebiger 
%
Körper vor. Nach den ersten zwei Anläufen in N.~\ref{57266_3} kommt er in N.~\ref{57268} 
%
zu einem bestimmten Ergebnis, das er als \glqq conclusio pulcherrima\grqq\ und \glqq theorema memoria tenendum\grqq\ feiert (siehe S.~\refpassage{37_05_144r-145v_11a}{37_05_144r-145v_11b}):
%
Die Geschwindigkeit des ersten Körpers \textit{nach} dem Stoß verhält sich zur Geschwindigkeit des zweiten \textit{vor} dem Stoß wie die Masse des zweiten zur Masse des ersten.
%
Hiermit wäre eine allgemeine Antwort auf die Ausgangsfrage gegeben, sowie Formeln, die eine Berechnung
%
der Geschwindigkeiten beider Körper nach dem Stoß als Funktion ihrer Massen und der Geschwindigkeiten vor dem Stoß ermöglichen.
%
Bereits der erste Anlauf von N.~\ref{57266_3} lieferte einige Teilergebnisse, auf denen Leibniz in N.~\ref{57268} aufbaut:
%
einerseits den Grundsatz \glqq Mutationes celeritatum sunt ut corpora reciproce\grqq, der die Gleichung (11) 
%
verbalisiert (S.~\refpassage{37_05_162v_10a}{37_05_162v_10b}) und aus dem Satz der Erhaltung der gesamten Bewegungsgröße fließt (der späteren \glqq regula (2)\grqq\   von N.~\ref{57268});
%
andererseits das Theorem über die Addition bzw.\ Subtraktion der Geschwindigkeiten von S.~\refpassage{37_05_162v_9a}{37_05_162v_9b}.
%
Beide Teilergebnisse werden in N.~\ref{57268} übernommen:
%
Dem Grundsatz entspricht die \makebox[1.0\textwidth][s]{Passage auf S.~\refpassage{37_05_144r-145v_9a}{37_05_144r-145v_9b} (eine weitere Parallelstelle findet sich im zeitgenössischen Stück N.~\ref{57267_2});
%
das} Theorem wird auf S.~\refpassage{37_05_144r-145v_10a}{37_05_144r-145v_10b}  wieder aufgegriffen.
%
%
Das \glqq theorema memoria tenendum\grqq\ ist hingegen laut der Randanmerkung auf S.~\refpassage{37_05_144r-145v_8a}{37_05_144r-145v_8b} eine durchaus neue Errungenschaft von N.~\ref{57268}. %
%
\pend
\newpage
\pstart
Allerdings ist Leibnizens angebliches \glqq Theorem\grqq\  nicht uneingeschränkt gültig.
%
Er folgt nicht aus den \textit{duae regulae} allein, sondern es kann nur aus der zusätzlichen Annahme 
%
hergeleitet werden \textendash\ und gilt nur unter der Bedingung \textendash\ dass jeder Körper dem 
%
anderen seine ganze Bewegungsgröße abgibt, bzw.\ dass die Körper ihre Größen austauschen 
%
(\glqq permutare\grqq). Diese Behauptung macht Leibniz tatsächlich im Laufe des Beweises 
%
(S.~\refpassage{37_05_144r-145v_7a}{37_05_144r-145v_7b} und die Randanmerkung). 
%
Die Bedingung ist genau dann erfüllt, wenn die Körper gleiche Massen haben.
%
Tatsächlich erwägt Leibniz in späteren Stücken (z.B.\ N.~\ref{RK57269}, N.~\ref{RK57270} und N.~\ref{RK57271} von Juni 1677) 
%
die Aufgabe dieser Lösung der kinematischen Frage:
%
\glqq ergo haec regula falsa est, ex qua sequeretur semper permutari potentias\grqq\ (S.~\refpassage{37_05_159-160_13a}{37_05_159-160_13b} von N.~\ref{RK57271}).
%
An ihre Stelle soll eine neue Lösung treten, die im Grunde mit \protect\index{Namensregister}{\textso{Huygens} (Hugenius, Ugenius, Hugens, Huguens), Christiaan 1629\textendash1695}Huygens' Stoßregel 
%
übereinstimmt, auf dem Relativitätsprinzip fußt und dementsprechend mithilfe der Huygens'schen Schiffsanalogie hergeleitet wird.
%
In den letzten Abschnitten von \textit{De corporum concursu} (\ref{dcc_08}\textendash\ref{dcc_10}) 
%
von Januar und Februar 1678
%
wird die Erhaltung der respektiven Geschwindigkeit zum wesentlichen Bestandteil der Leibniz'schen Stoßlehre.
\pend
%
\pstart
%
Leibniz hat in N.~\ref{57268} zwei längere Anmerkungen zu den eingangs genannten \glqq duae regulae\grqq\  nachgetragen, 
%
wovon die erste große inhaltliche Nähe zu den Ausführungen in \textit{De corporum concursu}, \textit{Scheda nona} 
%
(\ref{dcc_09} von Januar 1678) aufweist. 
%
Dies lässt die Vermutung zu, dass beide Anmerkungen etwa im Vorfeld der Abfassung der \textit{Scheda nona} entstanden sein könnten \textendash\ 
% 
jedoch mit Sicherheit vor dieser, und überhaupt vor den drei letzten \textit{schedae} \textit{De corporum concursu} (\ref{dcc_08}\textendash\ref{dcc_10}).
%
Denn die Sätze über die Kinematik des Stoßes, die Leibniz in den letzteren Texten erstmals beweist,
%
widersprechen dem zentralen Ergebnis von N.~\ref{57268} (dem \glqq theorema memoria tenendum\grqq),
%
so dass sie nicht zu einer Ergänzung des Stücks um zwei Anmerkungen, 
%
als vielmehr zu einer Revision oder gar zu seiner Verwerfung hätten führen müssen.
\pend
%
\end{ledgroup}