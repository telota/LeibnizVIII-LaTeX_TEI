%   ##
%
%   Band VIII, 3 N.~??A10.5
%   Signatur/Tex-Datei: LH_35_09_15_014-015
%   RK-Nr. 41152 [Teil 5]
%   Überschrift: Tentaminum de chordarum tensione scheda quinta
%   Datierung: 1680.12.10
%   WZ: (keins)
%.  SZ: (keins)
%.  Bilddateien (PDF): (keine)
%
%
\count\Bfootins=1000
\count\Afootins=1000
\count\Cfootins=1000
\begin{ledgroupsized}[r]{120mm}
\footnotesize
\pstart
\noindent\textbf{Überlieferung:}
\pend
\end{ledgroupsized}
\begin{ledgroupsized}[r]{114mm}
\footnotesize
\pstart \parindent -6mm
\makebox[6mm][l]{\textit{L}}%
Aufzeichnung: LH XXXV~9,~15 Bl.~14\textendash15.
Ein Bogen 4\textsuperscript{o}.
Vier einspaltig beschriebene Seiten.
% Zahlreiche Streichungen, Ergänzungen und Ersetzungen.
N.~8\textsubscript{5} setzt den Text von 8\textsubscript{4} fort.
% Kein Wasserzeichen.
\pend
\end{ledgroupsized}
%
%
\vspace*{8mm}
\pstart%
\normalsize%
\noindent%
%
\lbrack14~r\textsuperscript{o}\rbrack\ % Blatt 14r
%
\pend%
% Überschrift
\pstart%
\centering%
Tentaminum\protect\index{Sachverzeichnis}{tentamen}
de Chordarum Tensione\protect\index{Sachverzeichnis}{tensio chordae}
scheda\protect\index{Sachverzeichnis}{scheda} 5\textsuperscript{ta} 10 Xb. 1680
\pend%
\vspace{0.5em}%
%
\pstart%
\noindent%
Ex
\edtext{schemate\protect\index{Sachverzeichnis}{schema} praecedenti}{%
\lemma{schemate praecedenti}\Cfootnote{%
Siehe N.~8\textsubscript{4},
S.~\pageref{LH_35_09_15_013v_Fig.3},
% das Diagramm 
\lbrack\textit{Fig.~3a}\rbrack.}}
patet comparatio duarum chordarum
sola tensione\protect\index{Sachverzeichnis}{tensio chordae} differentium
\edtext{ratione temporum\protect\index{Sachverzeichnis}{tempus restitutionis}}{%
\lemma{ratione}\Bfootnote{%
\textit{(1)}~temporis
\textit{(2)}~temporum%
~\textit{L}}}
quibus differunt,
spatiorum quae percurrunt\protect\index{Sachverzeichnis}{spatium percursum}
\edtext{et celeritatum quas}{%
\lemma{et}\Bfootnote{%
\textit{(1)}~impetuum quos
\textit{(2)}~celeritatum quas%
~\textit{L}}}
acquirunt\protect\index{Sachverzeichnis}{celeritas acquisita}
quovis temporis momento;\protect\index{Sachverzeichnis}{momentum temporis}
sed illud superest
ut jam investigemus data temporum ratione,
quae sit ratio tensionum vel contra.
\edtext{}{%
{\xxref{LH_35_09_15_014r_gleicheransatz-1}{LH_35_09_15_014r_gleicheransatz-2}}%
{\lemma{Tensiones \lbrack...\rbrack\ complementi}\Cfootnote{%
Ähnlicher Ansatz wie in N.~9, S.~\refpassage{LH_35_09_15_002r_gleicheransatz-1}{LH_35_09_15_002r_gleicheransatz-2}.}}}%
\edlabel{LH_35_09_15_014r_gleicheransatz-1}%
Tensiones\protect\index{Sachverzeichnis}{tensio chordae}
autem variant impetus\protect\index{Sachverzeichnis}{impetus impressus}
singulis momentis impressos;
sunt autem impetus novi,\protect\index{Sachverzeichnis}{impetus novus}
qui singulis momentis\protect\index{Sachverzeichnis}{momentum temporis} accedunt,
ut sinus
\edtext{complementi.\protect\index{Sachverzeichnis}{sinus complementi}%
\edlabel{LH_35_09_15_014r_gleicheransatz-2}
Nempe momentis ut $\phi,$
tempora insumta\protect\index{Sachverzeichnis}{tempus insumtum}
sunt proportionalia ipsis arcubus\protect\index{Sachverzeichnis}{arcus circuli}
ut $\alpha\phi,$}{%
\lemma{complementi.}\Bfootnote{%
\textit{(1)}~Nempe momentis $\phi,$
\textit{(a)}~spatia
\textit{(b)}~tempus insumtum utrobique est $\alpha\phi,$ imp
\textit{(2)}~Nempe momentis ut $\phi,$
\textit{(a)}~et
\textit{(b)}~tempora insumta
\textit{(aa)}~sunt
\textit{(bb)}~respondent
\textit{(cc)}~sunt proportionalia ipsis arcubus ut $\alpha\phi,$%
~\textit{L}}}
spatia percursa\protect\index{Sachverzeichnis}{spatium percursum}
ipsis figuris sinuum\protect\index{Sachverzeichnis}{figura sinuum}
$\alpha\phi\mu\alpha$ et
\edtext{$\alpha\phi\rho\alpha.$
Habentur autem harum figurarum quadraturae,\protect\index{Sachverzeichnis}{quadratura}
est enim rectang. $\upsilon\alpha\delta$\protect\index{Sachverzeichnis}{rectangulum}
aequ. figurae sin. $\alpha\phi\mu\alpha$\protect\index{Sachverzeichnis}{figura sinuum} et
(\phantom)\hspace*{-1.2mm}%
posito $\alpha\phi\psi y$ quadrato\protect\index{Sachverzeichnis}{quadratum}%
\phantom(\hspace*{-1.2mm})
erit $y\alpha w$ aequ. $\alpha\phi\omega\alpha,$
unde et datur $\alpha\phi\rho\alpha.$}{%
\lemma{$\alpha\phi\rho\alpha.$}\Bfootnote{%
\textit{(1)}~Sunt autem
\textit{(2)}~Habentur autem harum figurarum quadraturae,
\textit{(a)}~sunt enim
\textit{(b)}~est enim rectang. $\upsilon\alpha\delta$ aequ. 
\textit{(aa)}~figuris
\textit{(bb)}~figurae sin. $\alpha\phi\mu\alpha$ \lbrack...\rbrack\ quadrato\phantom(\hspace*{-1.2mm}) erit
\textit{(aaa)}~rectang. $y\alpha$
\textit{(bbb)}~$y\alpha w$ aequ. $\alpha\phi\omega\alpha,$ unde et datur $\alpha\phi\rho\alpha.$%
~\textit{L}}}
Spatiorum\protect\index{Sachverzeichnis}{incrementum spatii}
\edtext{autem}{%
\lemma{autem}\Bfootnote{\textit{erg.~L}}}
incrementa seu spatia quovis momento\protect\index{Sachverzeichnis}{momentum temporis}
\edtext{percursa\protect\index{Sachverzeichnis}{spatium percursum}
\lbrack sunt proportionalia\rbrack\
sinubus\protect\index{Sachverzeichnis}{sinus}
vel quasi sinubus $\phi\mu$\protect\index{Sachverzeichnis}{quasi-sinus}}{%
\lemma{percursa}\Bfootnote{%
\hspace{-0,5mm}\textbar~sunt proportionalia \textit{erg. Hrsg.}~\textbar\ sinubus
\textit{(1)}~$\phi\omega$
\textit{(2)}~\textbar~vel quasi sinubus \textit{erg.}~\textbar\ $\phi\mu$%
% (\phantom)\hspace*{-1.2mm}vel $\delta\xi$\phantom(\hspace*{-1.2mm})%
~\textit{L}}}
%
(\phantom)\hspace*{-1.2mm}%
vel $\delta\xi$%
\phantom(\hspace*{-1.2mm})
et $\phi\rho$
(:~\phantom)\hspace*{-1.2mm}%
loco $\phi\omega$ vel $w\nu$
\edtext{ob productionem figurae\lbrack,\rbrack\
sunt enim $\phi\rho$ ut $\phi\omega$
seu ut sinus,\protect\index{Sachverzeichnis}{sinus}
unde ipsas $\phi\rho$ appello quasi sinus%
\phantom(\hspace*{-1.2mm}~:).
Eadem spatiorum incrementa\protect\index{Sachverzeichnis}{incrementum spatii} sunt}{%
\lemma{ob}\Bfootnote{%
\textit{(1)}~multiplicationem
\textit{(2)}~productionem figurae
\textit{(a)}~\phantom(\hspace*{-1.2mm})
\textit{(b)}~est
\textit{(c)}~sunt enim \lbrack...\rbrack\ sinus, unde
\textit{(aa)}~$\phi\rho$ appello
\textit{(bb)}~ipsas $\phi\rho$ appello quasi sinus\phantom(\hspace*{-1.2mm}~:).
\textit{(aaa)}~Impe
\textit{(bbb)}~Eadem
\textit{(aaaa)}~sunt
\textit{(bbb)}~spatiorum incrementa sunt%
~\textit{L}}}
etiam
\edtext{proportionalia celeritatibus quovis}{%
\lemma{proportionalia}\Bfootnote{%
\textit{(1)}~incrementis
\textit{(2)}~spatiorum
\textit{(3)}~celeritatum
\textit{(4)}~celeritatibus
\textit{(a)}~quo
\textit{(b)}~quovis%
~\textit{L}}}
momento acquisitis.\protect\index{Sachverzeichnis}{celeritas acquisita}
\edtext{Itaque momentis\protect\index{Sachverzeichnis}{momentum temporis} $\phi$ aut $\lambda,$}{%
\lemma{Itaque}\Bfootnote{%
\hspace*{-0,5mm}\textbar~Itaque \textit{streicht Hrsg.}~\textbar\
\textit{(1)}~tempor
\textit{(2)}~momentis $\phi$
\textit{(a)}~et
\textit{(b)}~aut $\lambda,$%
~\textit{L}}}
celeritates quaesitae\protect\index{Sachverzeichnis}{celeritas quaesita}
a chorda tardiore\protect\index{Sachverzeichnis}{chorda tardior}
\edtext{erunt $\phi\mu$ aut $\lambda\varPi,$}{%
\lemma{erunt}\Bfootnote{%
\hspace*{-0,5mm}$\phi\mu$
\textit{(1)}~vel
\textit{(2)}~aut $\lambda\varPi,$%
~\textit{L}}}
quaesitae a celeriore\protect\index{Sachverzeichnis}{chorda celerior}
seu magis tensa\protect\index{Sachverzeichnis}{chorda tensa} erunt $\phi\rho$ aut $\lambda\varOmega.$
Impetus ergo novi\protect\index{Sachverzeichnis}{impetus novus}
acquisiti\protect\index{Sachverzeichnis}{impetus acquisitus}
quovis momento\protect\index{Sachverzeichnis}{momentum temporis}%
\lbrack,\rbrack\
\edtext{seu incrementa}{%
\lemma{seu}\Bfootnote{%
\textit{(1)}~differen
\textit{(2)}~incrementa%
~\textit{L}}}
celeritatum,\protect\index{Sachverzeichnis}{incrementum celeritatis}
erunt in chorda\protect\index{Sachverzeichnis}{chorda tardior}
\edtext{tardiore ut differentiolae\protect\index{Sachverzeichnis}{differentiola}}{%
\lemma{tardiore}\Bfootnote{%
\hspace*{-0,5mm}ut
\textit{(1)}~differentiae
\textit{(2)}~differentiolae%
~\textit{L}}}
ipsarum $\phi\mu$
\edtext{seu sinuum\protect\index{Sachverzeichnis}{sinus}}{%
\lemma{seu}\Bfootnote{%
\hspace*{-0,5mm}sinuum \textit{erg.~L}}}%
\lbrack,\rbrack\
et in celeriore\protect\index{Sachverzeichnis}{chorda celerior}
ut differentiolae\protect\index{Sachverzeichnis}{differentiola} ipsarum $\phi\rho$
\edtext{seu quasi sinuum.\protect\index{Sachverzeichnis}{quasi-sinus}}{%
\lemma{seu}\Bfootnote{%
\hspace*{-0,5mm}quasi sinuum
\textit{erg.~L}}}
Scimus
\edtext{autem differentias sinuum
si arcus sint aequales\lbrack,\rbrack\
seu sinuum $\phi\mu$\protect\index{Sachverzeichnis}{sinus}
(\phantom)\hspace*{-1.2mm}%
seu $\delta\xi$%
\phantom(\hspace*{-1.2mm})
arcubus $\alpha\phi$\protect\index{Sachverzeichnis}{arcus circuli}
(\phantom)\hspace*{-1.2mm}%
vel $\alpha\xi$%
\phantom(\hspace*{-1.2mm})
applicatorum\lbrack,\rbrack\
esse ut sinus complementi $\beta\delta.$\protect\index{Sachverzeichnis}{sinus complementi}
Eodem modo}{%
\lemma{autem}\Bfootnote{%
\textit{(1)}~esse ut sinus
\textit{(a)}~versus
\textit{(b)}~complementi
\textit{(2)}~differentias sinuum, [...] aequales
\textit{(a)}~esse ut sinus complementi
\textit{(b)}~seu sinuum
\textit{(aa)}~$\phi\omega$
\textit{(bb)}~$\phi\mu$
(\phantom)\hspace*{-1.2mm}seu $\delta\xi$\phantom(\hspace*{-1.2mm}) arcubus
\textbar~$\alpha\phi$ (\phantom)\hspace*{-1.2mm}vel $\alpha\xi$\phantom(\hspace*{-1.2mm}) \textit{erg.}~%
\textbar\ applicatorum esse ut sinus complementi
\textit{(aaa)}~$\beta\delta.$
\textit{(bbb)}~$\beta\gamma.$
\textit{(ccc)}~$\beta\delta.$
\textit{(aaaa)}~Ergo qua
\textit{(bbbb)}~Eodem modo%
~\textit{L}}}
differentiae sinuum ut $\phi\omega$\protect\index{Sachverzeichnis}{sinus}
(\phantom)\hspace*{-1.2mm}%
seu $w\nu$%
\phantom(\hspace*{-1.2mm})
arcubus $\alpha\phi$\protect\index{Sachverzeichnis}{arcus circuli}
(\phantom)\hspace*{-1.2mm}%
seu $\alpha\nu$%
\phantom(\hspace*{-1.2mm})
applicatorum erunt ut sinus
\edtext{complementi $\phi w.$\protect\index{Sachverzeichnis}{sinus complementi}
Cumque quasi sinus $\phi\rho$ sint ad}{%
\lemma{complementi}\Bfootnote{% \hspace*{-0,5mm}
$\phi w.$
\textit{(1)}~Et qua
\textit{(2)}~Cumque quasi sinus
\textit{(a)}~$\phi\rho$ sint ad
\textit{(b)}~$\phi\rho$ sint ad%
~\textit{L}}}
sinus $\phi\omega$
in ratione $\lambda\varOmega$ ad $\lambda z,$
etiam
\edtext{differentiae quasi sinuum\protect\index{Sachverzeichnis}{quasi-sinus} erunt}{%
\lemma{differentiae}\Bfootnote{%
\textit{(1)}~erunt
\textit{(2)}~quasi sinuum erunt%
~\textit{L}}}
\edtext{ad
%
\lbrack14~v\textsuperscript{o}\rbrack\ % Blatt 14v
%
differentias sinuum\protect\index{Sachverzeichnis}{sinus}
(\phantom)\hspace*{-1.2mm}%
quibus sinus complementi\protect\index{Sachverzeichnis}{sinus complementi}
$\phi w$ respondent%
\lbrack\phantom(\hspace*{-1.2mm})\rbrack, ut}{%
\lemma{ad}\Bfootnote{%
\hspace*{-0,5mm}\lbrack14~v\textsuperscript{o}\rbrack\
\textit{(1)}~differentiis sinuum respondentes
\textit{(2)}~differentias si\-nu\-um
\textit{(a)}~ut
\textit{(b)}~(\phantom)\hspace*{-1.2mm}quibus sinus complementi
\textbar~$\phi w$ \textit{erg.}~\textbar\ respondent, ut%
~\textit{L}}}
$\lambda\varOmega$ ad $\lambda z.$
Itaque
\edtext{sumatur
$\displaystyle \phi\text{\hebr{g}} : \phi w \,\squaredots\, \lambda\varOmega : \lambda z,$
erunt quasi sinus complementi\protect\index{Sachverzeichnis}{quasi-sinus complementi} $\phi\text{\hebr{g}}$
ut incrementa quasi sinuum,\protect\index{Sachverzeichnis}{quasi-sinus}
vel ut incrementa celeritatum,\protect\index{Sachverzeichnis}{incrementum celeritatis}
seu ut impetus novi.\protect\index{Sachverzeichnis}{impetus novus}}{%
\lemma{sumatur}\Bfootnote{%
\textit{(1)}~$\displaystyle \alpha\text{\hebr{g}} : \alpha\delta$
\textit{(2)}~$\displaystyle \phi\text{\hebr{g}} : \phi w \,\squaredots\, \lambda\varOmega : \lambda z,$
\textit{(a)}~et similiter $\displaystyle \alpha\text{\hebr{l}} : \alpha\phi \,\squaredots\, \lambda\varOmega : \lambda z$
\textit{(b)}~erunt
\textit{(aa)}~$\alpha\text{\hebr{g}},$ $\alpha\lambda,$ et intermedii,~%
\textbar\ ut quasi sinuum $\phi\rho,$ \textit{streicht Hrsg.}~%
\textbar\ $\lambda\varOmega$ et intermediorum, incrementa, seu ut incrementa celeritatum vel impetus novi.
\textit{(bb)}~quasi sinus complementi
\textit{(cc)}~quasi sinus \lbrack...\rbrack\ sinuum, vel
\textbar~ut incrementa \textit{erg.}~%
\textbar\ celeritatum, seu ut impetus novi.%
~\textit{L}}}
Et ut est $\beta\delta$ ad $\phi\text{\hebr{g}}$\,
\edtext{ita erit impetus novus\protect\index{Sachverzeichnis}{impetus novus}}{%
\lemma{ita}\Bfootnote{%
\textit{(1)}~erunt
\textit{(2)}~erit impetus
\textit{(a)}~novi
\textit{(b)}~novus%
~\textit{L}}}
chordae
\edtext{tardiori\protect\index{Sachverzeichnis}{chorda tardior}
in momento $\phi$\protect\index{Sachverzeichnis}{momentum temporis}
impressus,\protect\index{Sachverzeichnis}{impetus impressus}
ad impetum novum\protect\index{Sachverzeichnis}{impetus novus}}{%
\lemma{tardiori}\Bfootnote{%
\textit{(1)}~absoluto tempore ut $\alpha\phi$ quaesitus, ad impetum novum
\textit{(2)}~in momento \lbrack...\rbrack\ impetum novum%
~\textit{L}}}
chordae celeriori\protect\index{Sachverzeichnis}{chorda celerior}
eodem momento $\phi$\protect\index{Sachverzeichnis}{momentum temporis}
impressum.\protect\index{Sachverzeichnis}{impetus impressus}
\pend%
% \newpage%
\vspace{0.5em}%
%
\pstart%
\noindent%
\lbrack\textit{Nachfolgend kleingedruckter Text in L gestrichen:}\rbrack\
\pend%
\vspace{0.5em}%
%
\footnotesize%
\pstart%
\noindent%
Quoniam autem impetus\protect\index{Sachverzeichnis}{impetus impressus}
qui chordae\protect\index{Sachverzeichnis}{chorda tensa}
\edtext{imprimitur in progressu motus\protect\index{Sachverzeichnis}{progressus motus}
momento $\phi,$\protect\index{Sachverzeichnis}{momentum temporis}
est ad impetum}{%
\lemma{imprimitur}\Bfootnote{%
\textit{(1)}~initi
\textit{(2)}~in progressu motus
\textit{(a)}~est ad impetum
\textit{(b)}~tempore
\textit{(c)}~momento $\phi,$ est ad impetum%
~\textit{L}}}
impressum\protect\index{Sachverzeichnis}{impetus impressus} ab initio
\edtext{seu tensionem,\protect\index{Sachverzeichnis}{tensio chordae}}{%
\lemma{seu}\Bfootnote{%
\hspace*{-0,5mm}tensionem,
\textit{erg.~L}}}
ut spatium totum percurrendum est ad spatium percurrendum residuum.%
\protect\index{Sachverzeichnis}{spatium percurrendum}
Manifestum
\edtext{est diversis chordis\protect\index{Sachverzeichnis}{chorda tensa} inter se allatis
rationes impetuum $\phi\delta : \phi\text{\hebr{g}}$
aliquo momento ut $\phi$\protect\index{Sachverzeichnis}{momentum temporis}
impressorum\protect\index{Sachverzeichnis}{impetus impressus} componi}{%
\lemma{est}\Bfootnote{%
\textit{(1)}~componi ratio
\textit{(2)}~diversis chordis \lbrack...\rbrack\ rationes impetuum
\textit{(a)}~aliq
\textit{(b)}~$\phi\delta : \phi\text{\hebr{g}}$ aliquo \lbrack...\rbrack\ impressorum componi%
~\textit{L}}}
tum ex rationibus spatiorum percurrendorum\protect\index{Sachverzeichnis}{spatium percurrendum}
\edtext{$\mu\phi\epsilon\theta\mu:$
$\rho\phi\lambda\varOmega,$
tum ex rationibus tensionum,\protect\index{Sachverzeichnis}{tensio chordae}}{%
\lemma{$\mu\phi\epsilon\theta\mu:$}\Bfootnote{%
\textit{(1)}~tum ex rationibus tensionum $\squaredots$\,
\textit{(2)}~$\rho\phi\lambda\varOmega,$ tum ex rationibus tensionum,%
~\textit{L}}}
hinc fiet:
tens. min. : tens. maj.
$\squaredots$ $\displaystyle \beta\delta : \phi\text{\hebr{g}}. \smallfrown \rho\phi\lambda\varOmega : \mu\phi\epsilon\theta\mu.$
\newline
\indent%
Est autem
$\displaystyle \beta\delta : \phi\text{\hebr{g}} \,\squaredots\, \beta\delta : \phi\omega. \smallfrown
\hspace*{-10,5mm}%
\efrac{}{\efrac{\displaystyle\underbrace{\displaystyle\lambda z : \lambda\varOmega}}{\displaystyle\alpha\phi\ \mbox{quadr.} : \alpha\beta\ \mbox{quadr.}}}$
\hspace*{-10,5mm}%
et $\mu\phi\epsilon\theta\mu$ aequ. rectang. $\text{\hebr{g}}\beta\delta$ aequ. $\alpha\beta$ in
\edtext{$\beta\delta$ et $\rho\phi\lambda\varOmega$ aequ.}{%
\lemma{$\beta\delta$}\Bfootnote{%
\hspace*{-0,5mm}et
\textit{(1)}~$\rho\phi\lambda\omega$ aequ.
\textit{(2)}~$\rho\phi\lambda\varOmega$ aequ.%
~\textit{L}}}
$\psi\phi\text{\hebr{g}}$ seu aequ. $\alpha\phi$ in $\phi\text{\hebr{g}}.$
\rule[0pt]{0mm}{10pt}%
Ergo omnibus analogiis\protect\index{Sachverzeichnis}{analogia}
in aequationem\protect\index{Sachverzeichnis}{aequatio} collectis,
fiet:%
\pend%
\vspace{0.5em}
%
\pstart%
\centering%
$\displaystyle\frac{\mbox{tens. min.}}{\mbox{tens. maj.}}$
aequ.
$\displaystyle\frac{\underset{\displaystyle{,}}{\smash[b]{\ovalbox{$\displaystyle\beta\delta$}}}}{\underset{\displaystyle{,\!,}}{\smash[b]{\ovalbox{$\displaystyle\phi\text{\hebr{g}}$}}}\,\big(\mbox{\footnotesize{aequ.}} \frac{\displaystyle\alpha\beta\ \mbox{\footnotesize{quadr.}}}{\displaystyle\alpha\phi\ \mbox{\footnotesize{quadr.}}}\phi w\big)}
% $\displaystyle\frac{\large\textcircled{$\scriptstyle\beta\delta$}}{\large\textcircled{$\scriptstyle\phi\text{\hebr{g}}$} \big(\mbox{\footnotesize{aequ.}} \frac{\displaystyle\alpha\beta\ \mbox{\footnotesize{quadr.}}}{\displaystyle\alpha\phi\ \mbox{\footnotesize{quadr.}}}\phi w\big)}
\smallfrown
\frac{\alpha\phi \smallfrown \underset{\displaystyle{,\!,}}{\smash[b]{\ovalbox{$\displaystyle\phi\text{\hebr{g}}$}}}}{
\vphantom{\frac{\displaystyle\alpha\beta\ \mbox{\footnotesize{qd}}}{\displaystyle\alpha\phi\ \mbox{\footnotesize{qd}}}{\beta\delta}}
\underset{\displaystyle{,}}{\smash[b]{\ovalbox{$\displaystyle\beta\delta$}}} \smallfrown \alpha\beta}.\ $%
\protect\index{Sachverzeichnis}{tensio chordae}
%
\normalsize{\lbrack15~r\textsuperscript{o}\rbrack\ }% Blatt 15r
%
\pend%
\vspace{1.0em}
%
\normalsize%
\pstart%
Haec ut clarius appareant
describamus lineas sinuum\protect\index{Sachverzeichnis}{linea sinuum complementi}
\edtext{complementi et}{%
\lemma{complementi}\Bfootnote{%
\textit{(1)}~aut
\textit{(2)}~et%
~\textit{L}}}
quasi sinu\-um complementi,\protect\index{Sachverzeichnis}{quasi-sinus}
arcubus\protect\index{Sachverzeichnis}{arcus circuli}
seu temporibus\protect\index{Sachverzeichnis}{tempus restitutionis}
\edtext{applicatorum.
Nempe quadrantis $\alpha\beta\gamma\alpha$\protect\index{Sachverzeichnis}{quadrans circuli}
figura sinu\-um complementi\protect\index{Sachverzeichnis}{figura sinuum complementi}}{%
\lemma{applicatorum.}\Bfootnote{%
\textit{(1)}~Linea
\textit{(2)}~Nempe quadrantis $\alpha\beta\gamma\alpha$
\textit{(a)}~linea
\textit{(b)}~sinuum com
\textit{(c)}~figura sinuum complementi%
~\textit{L}}}
erit $\text{\hebr{'}}\alpha\epsilon\text{\hebr{t}}\text{\hebr{'}}$
cui producta $\rho\phi$ occurret
\edtext{in \hebr{t},
eritque $\phi\text{\hebr{t}}$ ad $\alpha\text{\hebr{'}}$
ut sinus complementi $\beta\delta$\protect\index{Sachverzeichnis}{sinus complementi}
arcus\protect\index{Sachverzeichnis}{arcus circuli} $\alpha\phi$ vel $\alpha\xi$
ad sinum\protect\index{Sachverzeichnis}{sinus} totum $\beta\alpha.$}{%
\lemma{in}\Bfootnote{% \hspace*{-0,5mm}
\hebr{t},
\textit{(1)}~sit
\textit{(2)}~eritque $\phi\text{\hebr{t}}$
\textit{(a)}~aequ. sinui complementi $\beta\delta$ arcus $\alpha\phi$ vel $\alpha\xi.$
\textit{(aa)}~Sed linea quasi sinuum complementi esto
\textit{(bb)}~Sed linea
\textit{(b)}~ad
\textit{(c)}~ad
\textit{(aa)}~$\alpha$
\textit{(bb)}~$\beta\alpha$
\textit{(cc)}~$\beta\text{\hebr{'}}$
\textit{(dd)}~$\alpha\text{\hebr{'}}$ ut sinus \lbrack...\rbrack\ sinum totum
\textit{(aaa)}~$\beta\alpha.$
\textit{(bbb)}~$\beta\alpha.$%
~\textit{L}}}
Imo fieri\edlabel{LH_35_09_15_015r_phitav-1}
\edtext{poterit $\phi\text{\hebr{t}}$ aequ. $\beta\delta$
et $\alpha\text{\hebr{'}}$ aequ. $\beta\alpha.$%
\edlabel{LH_35_09_15_015r_phitav-2}
Sed quadrantis\protect\index{Sachverzeichnis}{quadrans circuli} $\alpha\phi\psi\alpha$
figura quasi-sinuum}{%
\lemma{poterit}\Bfootnote{%
\hspace*{-0,5mm}$\phi\text{\hebr{t}}$
\textbar~aequ. $\beta\text{\hebr{'}}$ et \textit{gestr.}~\textbar\ aequ. $\beta\delta$ et
\textit{(1)}~$\beta\alpha$ aequ. $\beta$
\textit{(2)}~$\alpha\text{\hebr{'}}$ aequ. $\beta\alpha.$
\textit{(a)}~Linea vero
\textit{(aa)}~sinu
\textit{(bb)}~quasi sin
\textit{(b)}~Sed quadrantis $\alpha\phi\psi\alpha$ figura quasi-sinuum%
~\textit{L}}}
\edtext{complementi\protect\index{Sachverzeichnis}{quasi-sinuum complementi} sit}{%
\lemma{complementi}\Bfootnote{%
\textit{(1)}~erit
\textit{(2)}~sit%
~\textit{L}}}
$\text{\hebr{S}}\alpha\lambda\text{\hebr{b}}\text{\hebr{S}}$
sitque $\phi\text{\hebr{b}}$ ad $\alpha\text{\hebr{S}}$ ut $\phi w$
(\phantom)\hspace*{-1.2mm}%
sinus complementi\protect\index{Sachverzeichnis}{sinus complementi}
arcus\protect\index{Sachverzeichnis}{arcus circuli} $\alpha\phi$%
\phantom(\hspace*{-1.2mm})
ad $\phi\alpha$ sinum ejus quadrantis totum.
Debet autem esse $\alpha\text{\hebr{S}}$
\edtext{ad $\alpha\phi$
(\phantom)\hspace*{-1.2mm}%
vel $\phi\alpha$%
\phantom(\hspace*{-1.2mm})
seu quasi sinus complementi\protect\index{Sachverzeichnis}{quasi-sinus complementi}
debent esse ad veros sinus complementi\protect\index{Sachverzeichnis}{sinus complementi}
tales,
ut quemadmodum rectangulum\protect\index{Sachverzeichnis}{rectangulum} $\alpha\beta\gamma$
(\phantom)\hspace*{-1.2mm}%
seu quadr.\protect\index{Sachverzeichnis}{quadratum} $\alpha\beta$%
\phantom(\hspace*{-1.2mm})
aequatur figurae sinuum complementi\protect\index{Sachverzeichnis}{figura sinuum complementi}
(\phantom)\hspace*{-1.2mm}%
nam}{%
\lemma{ad}\Bfootnote{%
\hspace*{-0,5mm}$\alpha\phi$
\textit{(1)}~ita ut
\textit{(2)}~(\phantom)\hspace*{-1.2mm}vel $\phi\alpha$\phantom(\hspace*{-1.2mm})
\textit{(a)}~it
\textit{(b)}~seu quasi \lbrack...\rbrack\ complementi tales,
\textit{(aa)}~ut fiat figura sinuum complementi
\textit{(bb)}~ut quemadmodum rectangulum
\textbar~$\alpha\beta\gamma$ (\phantom)\hspace*{-1.2mm}seu quadr. $\alpha\beta$\phantom(\hspace*{-1.2mm}) \textit{erg.}~%
\textbar\ $\alpha\beta\gamma$ \textit{streicht Hrsg.}~%
\textbar\ aequatur figurae sinuum complementi (\phantom)\hspace*{-1.2mm}%
\textit{(aaa)}~cum
\textit{(bbb)}~nam%
~\textit{L}}}
figura sinuum\protect\index{Sachverzeichnis}{figura sinuum}
et sinuum complementi\protect\index{Sachverzeichnis}{figura sinuum complementi}
reapse non
\edtext{differunt,
est}{%
\lemma{differunt,}\Bfootnote{%
\textit{(1)}~sunt
\textit{(2)}~est%
~\textit{L}}}
enim altera alterius inversa\protect\index{Sachverzeichnis}{figura inversa}%
\phantom(\hspace*{-1.2mm})
ita rectang.
(\phantom)\hspace*{-1.2mm}%
non quidem $\alpha\phi\omega$
sed ejus multiplicatum in ratione $\lambda\varOmega$ ad $\lambda z$ vel
\edtext{\lbrack$\phi\rho$ ad\rbrack}{%
\lemma{$\phi\rho$}\Bfootnote{%
\hspace*{-0,5mm}ad
\textit{erg. Hrsg.}}}
$\phi\omega$%
\phantom(\hspace*{-1.2mm})
nempe $\alpha\phi$ in $\lambda\varOmega$
sit aequal. figurae quasi sinuum complementi\protect\index{Sachverzeichnis}{quasi-sinus complementi}
$\text{\hebr{S}}\alpha\lambda\text{\hebr{b}}\text{\hebr{S}}.$
Cumque futurum esset rectang.\protect\index{Sachverzeichnis}{rectangulum} $\alpha\phi\omega$ aequ.
figurae sinuum complementi\protect\index{Sachverzeichnis}{figura sinuum complementi} verorum
quadrantis $\alpha\phi\psi\alpha$
sitque $\alpha\phi\omega$ ad $\alpha\phi$ in $\lambda\varOmega$
ut $\phi\omega$ ad 
\edtext{\lbrack$\phi\rho$\rbrack}{%
\lemma{$\lambda\varOmega$}\Bfootnote{\textit{L~ändert Hrsg.}}}
seu ut $\lambda z$ ad $\lambda\varOmega.$
Hinc et debent applicatae quasi figurae sinuum\protect\index{Sachverzeichnis}{quasi-sinus}
esse ad applicatas figurae sinuum\protect\index{Sachverzeichnis}{figura sinuum} verorum
ejusdem quadrantis,\protect\index{Sachverzeichnis}{quadrans circuli}
seu $\alpha\text{\hebr{S}}$
\edtext{ad $\phi\alpha$
(\phantom)\hspace*{-1.2mm}%
vel $\text{\hebr{b}}\phi$}{%
\lemma{ad}\Bfootnote{% \hspace*{-0,5mm}
$\phi\alpha$
\textbar~vel \textit{streicht Hrsg.}~\textbar\
(\phantom)\hspace*{-1.2mm}vel $\text{\hebr{b}}\phi$%
~\textit{L}}}
ad $\phi w$%
\phantom(\hspace*{-1.2mm})
ut $\lambda\varOmega$ ad $\lambda z.$
Jam $\lambda\varOmega$ ad $\lambda z$ ut
quadr.\,$\alpha\beta$ ad quadr.\,$\alpha\phi.$
Ergo $\alpha\text{\hebr{S}}$ \lbrack:\rbrack\ $\phi\alpha$
$\squaredots$
$\alpha\beta$\,quadr. \lbrack:\rbrack\ $\phi\alpha$\,quadr.
\edtext{Ergo $\alpha\beta$ vel $\alpha\text{\hebr{'}}$ med. prop. inter}{%
\lemma{Ergo}\Bfootnote{%
\hspace{-0,5mm}$\alpha\beta$
\textit{(1)}~med. prop. inter
\textit{(2)}~vel $\alpha\text{\hebr{'}}$ med. prop. inter%
~\textit{L}}}
$\alpha\phi$ et $\alpha\text{\hebr{S}}.$
%
\lbrack15~v\textsuperscript{o}\rbrack\ % Blatt 15v
%
Adeoque ut $\alpha\phi$ ad $\alpha\beta$ seu ut
\edtext{tempus\protect\index{Sachverzeichnis}{tempus restitutionis} minus
nempe restitutio\protect\index{Sachverzeichnis}{restitutio chordae}
chordae magis tensae\protect\index{Sachverzeichnis}{chorda tensa}
ad tempus\protect\index{Sachverzeichnis}{tempus restitutionis} majus
seu restitutionem\protect\index{Sachverzeichnis}{restitutio chordae}
chordae minus tensae,\protect\index{Sachverzeichnis}{chorda tensa}
\lbrack ita\rbrack\ $\alpha\text{\hebr{'}}$ ad $\alpha\text{\hebr{S}}$
seu tensio minor ad majorem.\protect\index{Sachverzeichnis}{tensio chordae}}{%
\lemma{tempus}\Bfootnote{%
\textit{(1)}~minus
\textit{(2)}~minus nempe
\textit{(a)}~vibratio
\textit{(b)}~restitutio chordae \lbrack...\rbrack\ majus seu
\textit{(aa)}~vibrationem
\textit{(bb)}~restitutionem chordae minus tensae,
\textbar~ut \textit{ändert Hrsg.}~%
\textbar\ $\alpha\text{\hebr{'}}$ ad $\alpha\text{\hebr{S}}$ seu
\textbar~ut \textit{streicht Hrsg.}~%
\textbar\ tensio
\textbar~chordae \textit{gestr.}~%
\textbar\ minor ad majorem.%
~\textit{L}}}
%
\edtext{Quia applicatae harum figurarum}{%
\lemma{Quia}\Bfootnote{%
\textit{(1)}~sinus
\textit{(2)}~applicatae harum figurarum%
~\textit{L}}}
(\phantom)\hspace*{-1.2mm}%
$\text{\hebr{'}}\alpha\epsilon\text{\hebr{t}}\text{\hebr{'}}$
et
$\text{\hebr{S}}\alpha\lambda\text{\hebr{b}}\text{\hebr{S}}$%
\phantom(\hspace*{-1.2mm})
repraesentant impetus novos,\protect\index{Sachverzeichnis}{impetus novus}
et primae applicatae
$\alpha\text{\hebr{'}}$ ad $\alpha\text{\hebr{S}}$
repraesentant primos impetus,\protect\index{Sachverzeichnis}{impetus primus}
seu ipsas tensiones,\protect\index{Sachverzeichnis}{tensio chordae}
\edlabel{LH_35_09_15_015v_recprtens-1}%
Ergo demonstratum est%
\textso{ tempora }%
\protect\index{Sachverzeichnis}{restitutio chordae}%
\protect\index{Sachverzeichnis}{tempus restitutionis}%
\protect\index{Sachverzeichnis}{restitutio omnimoda}%
\protect\index{Sachverzeichnis}{tensio chordae}%
\protect\index{Sachverzeichnis}{chorda tensa}%
\edtext{\textso{restitutionum omnimodarum in chordis sola tensione differentibus esse reciproce ut}}{%
\lemma{\textso{restitutionum}}\Bfootnote{%
\textit{(1)}~esse reciproce ut
\textit{(2)}~\textso{omnimodarum in} \lbrack...\rbrack\ \textso{reciproce ut}%
~\textit{L}}}%
\textso{ tensiones. }%
\edlabel{LH_35_09_15_015v_recprtens-2}%
\pend%
% \newpage% !!!! Rein vorläufig !!!!
%
\pstart%
Nullo modo autem hic sermo\protect\index{Sachverzeichnis}{sermo}
est de vibrationibus,\protect\index{Sachverzeichnis}{vibratio chordae}
sed de restitutionibus.\protect\index{Sachverzeichnis}{restitutio chordae}
Nam
\edtext{in vibrationibus\protect\index{Sachverzeichnis}{vibratio chordae}
multa sunt discrimina\protect\index{Sachverzeichnis}{discrimen} notanda.}{%
\lemma{in}\Bfootnote{%
\hspace*{-0,5mm}vibrationibus
\textit{(1)}~vis restitu
\textit{(2)}~vires restituentes\protect\index{Sachverzeichnis}{vis restituens} post secundam tensionem, longe aliae sunt quam vires primo tendentes\protect\index{Sachverzeichnis}{vis tendens}
\textit{(3)}~multa sunt discrimina notanda.%
~\textit{L}}}
\pend%
%
\pstart%
Hinc%
\edlabel{LH_35_09_15_015v_isochron-1}
demonstratur%
\protect\index{Sachverzeichnis}{restitutio chordae}%
\protect\index{Sachverzeichnis}{restitutio omnimoda}%
\protect\index{Sachverzeichnis}{restitutio isochrona}
porro%
\textso{ Restitutiones ejusdem chordae }%
\edtext{\textso{omnimodas}}{%
\lemma{\textso{omnimodas}}\Bfootnote{\textit{erg.~L}}}%
\textso{ esse isochronas.}
Quod theorema\protect\index{Sachverzeichnis}{theorema} nemo hactenus accurate
\edtext{ostendit.
Hoc autem ita conficiemus:}{%
\lemma{ostendit.}\Bfootnote{%
\textit{(1)}~Jam
\textit{(2)}~Nam
\textit{(3)}~Hoc autem
\textit{(a)}~ita
\textit{(b)}~ita conficiemus:%
~\textit{L}}}
supra demonstratum
\edtext{est
\edtext{(\phantom)\hspace*{-1.2mm}%
schedae\protect\index{Sachverzeichnis}{scheda} tertiae 10 Xb. paginae 2dae%
\phantom(\hspace*{-1.2mm})}{%
\lemma{(\phantom)\hspace*{-1.2mm}schedae \lbrack...\rbrack\ 2dae\phantom(\hspace*{-1.2mm})}\Cfootnote{%
N.~8\textsubscript{3}, S.~\refpassage{LH_35_09_15_010v_slbsthinw-1}{LH_35_09_15_010v_slbsthinw-2}.}}
chordae alicujus diversimode tensae\protect\index{Sachverzeichnis}{chorda diversimode tensa}
restitutiones\protect\index{Sachverzeichnis}{restitutio chordae}}{%
\lemma{est}\Bfootnote{%
\textit{(1)}~chordae alicujus diversimode tensa restitutiones sed
\textit{(2)}~(\phantom)\hspace*{-1.2mm}schedae tertiae \lbrack...\rbrack\ tensae restitutiones%
~\textit{L}}}
esse inter se ex composita restitutionum\protect\index{Sachverzeichnis}{restitutio chordae}
quas haberent si his solis tensionibus differrent,
\edtext{et tensionum.\protect\index{Sachverzeichnis}{tensio chordae}
At si solis tensionibus}{%
\lemma{et}\Bfootnote{%
\textit{(1)}~longitudinum. At hoc loco si solis te
\textit{(2)}~tensionum. At si solis tensionibus%
~\textit{L}}}
\edtext{differrent, essent}{%
\lemma{differrent,}\Bfootnote{%
\textit{(1)}~haberent
\textit{(2)}~essent%
~\textit{L}}}
restitutiones reciproce ut longitudines.\protect\index{Sachverzeichnis}{longitudo chordae}
Ergo Restitutiones\protect\index{Sachverzeichnis}{restitutio chordae}
ejusdem chordae diversimode tensae\protect\index{Sachverzeichnis}{chorda diversimode tensa}
sunt inter
\edtext{se in}{%
\lemma{se}\Bfootnote{%
\textit{(1)}~et
\textit{(2)}~in%
~\textit{L}}}
ratione composita ex tensionum\protect\index{Sachverzeichnis}{tensio chordae}
rationibus directa et reciproca.
Id est sunt aequales inter se.
Q.E.D.\edlabel{LH_35_09_15_015v_isochron-2}
\pend%
%
\pstart%
Ex his patet
\edtext{etiam restitutionum}{%
\lemma{etiam}\Bfootnote{%
\hspace{-0,5mm}\textbar~chordarum \textit{gestr.}~\textbar\ restitutionum%
~\textit{L}}}
isochronismum,\protect\index{Sachverzeichnis}{isochronismus restitutionis}
et sectionem\protect\index{Sachverzeichnis}{sectio monochordi}
\edtext{monochordi pro restitutionibus omnimodis\protect\index{Sachverzeichnis}{restitutio omnimoda}
esse per se invicem}{%
\lemma{monochordi}\Bfootnote{%
\hspace{-0,5mm}\textbar~pro restitutionibus omnimodis \textit{erg.}~\textbar\
\textit{(1)}~ex se invicem
\textit{(2)}~esse
\textit{(a)}~aequales inter
\textit{(b)}~per se invicem%
~\textit{L}}}
\edtext{demonstrabiles.}{%
\lemma{demonstrabiles}\Cfootnote{%
Einen solchen Beweis versucht Leibniz vornehmlich in späteren Entwürfen aus den Jahren 1690\textendash1695 zu erbringen;
vgl. N.~\ref{RK60353} %??A46 
und
N.~\ref{RK60301}, %??A45 
bes. S.~\refpassage{LH_37_05_046r_Isochronismusbeweis-1}{LH_37_05_046r_Isochronismusbeweis-2}.%
}}
\pend%
\newpage
\count\Bfootins=1200
\count\Afootins=1200
\count\Cfootins=1200% Rein vorläufig
 %
% ENDE DES STÜCKES auf Blatt 15 v