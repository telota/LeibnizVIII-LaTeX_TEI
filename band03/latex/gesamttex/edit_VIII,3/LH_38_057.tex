%   % !TEX root = ../../VIII,3_Rahmen-TeX_8-1.tex
%
%
%   Band VIII, 3 N.~??A38
%   Signatur/Tex-Datei: LH_38_057
%   RK-Nr. 55749
%   Überschrift: [Ob lange Seile eher reißen als kurze?]
%   Modul: Mechanik / Elastizität
%   Datierung: 29. April (9. Mai) 1685 (eigenhändig)
%   WZ: (keins)
%   SZ: (keins)
%   Bilddateien (PDF): (keine)
%
%
\selectlanguage{ngerman}%
\frenchspacing%
%
\begin{ledgroupsized}[r]{120mm}%
\footnotesize%
\pstart%
\noindent\textbf{Überlieferung:}%
\pend%
\end{ledgroupsized}%
\begin{ledgroupsized}[r]{114mm}%
\footnotesize%
\pstart%
\parindent -6mm%
\makebox[6mm][l]{\textit{L}}%
Notiz: LH XXXVIII Bl.~57.
Ein Zettel (9,4 x 12,5~cm).
Eine Seite auf Bl.~57~r\textsuperscript{o};
Bl.~57~v\textsuperscript{o} ist leer.
\pend
\end{ledgroupsized}
\begin{ledgroupsized}[r]{114mm}%
\footnotesize%
\pstart%
\parindent -6mm%
\makebox[6mm][l]{\textit{E}}%
\textsc{Gerland} 1906, S.~175 (Nr.~94).
\pend
\end{ledgroupsized}
%
\selectlanguage{ngerman}%
\frenchspacing%
%
\vspace*{8mm}
\pstart%
\noindent%
\normalsize%
\lbrack57~r\textsuperscript{o}\rbrack\ 29 April. 1685
% \edtext{}{\lemma{\textit{Am oberen Rand, mittig:}}\Afootnote{%
% 29 April. 1685}}
\pend%
\vspace{0.5em}%
%
\pstart%
\noindent%
Es solte scheinen ein seil\protect\index{Sachverzeichnis}{seil}\lbrack,\rbrack\
welches lang\lbrack,\rbrack\
reiße mit gleichen gewicht\protect\index{Sachverzeichnis}{gewicht} nicht so leicht,
als ein anders so kurz und eben so dick und starck,
%
\edtext{dieweilen
\edlabel{LH_38_057r_gleichSpann_kfjg-1}%
die tensio\protect\index{Sachverzeichnis}{tensio}
oder spannung\protect\index{Sachverzeichnis}{spannung} in mehr}{%
\lemma{dieweilen}\Bfootnote{%
\textit{(1)}~mehr
\textit{(2)}~die tensio oder spannung in mehr%
~\textit{L}}}
%
partes vertheilet wird in einem langen seil,
und
%
\edtext{also iedes theil eines langen}{%
\lemma{also}\Bfootnote{%
\textit{(1)}~ein lange
\textit{(2)}~iedes theil eines langen%
~\textit{L}}}
%
seiles bey weiten nicht
%
\edtext{mit gleichen gewicht\protect\index{Sachverzeichnis}{gewicht}}{%
\lemma{mit}\Bfootnote{%
\hspace*{-0,5mm}gleichen gewicht
\textit{erg.~L}}}
%
so viel gespannet als iedes theil eines kurzen,
daher auch
%
\edtext{das lange}{%
\lemma{das}\Bfootnote{%
\textit{(1)}~kurz \textit{(2)}~lange%
~\textit{L}}}
%
nicht so sehr
%
\edtext{nothleidet.%
\edlabel{LH_38_057r_gleichSpann_kfjg-2}
Denn wenn man ein langes seil\protect\index{Sachverzeichnis}{seil}
einem kurzen gleich}{%
\lemma{nothleidet.}\Bfootnote{%
\textit{(1)}~Und kurz ist
\textit{(2)}~Denn wenn \lbrack...\rbrack\ kurzen gleich%
~\textit{L}}}
%
spannen will\lbrack,\rbrack\
daß es eben den thon oder laut bekommet,
muß man umb soviel mehr gewichte geben.
Dieses nun ist theoretice ganz gewiß,
und ohnfehlbar;
%
\edtext{wenn das lange}{%
\lemma{wenn}\Bfootnote{%
\textit{(1)}~ein langes
\textit{(2)}~das lange%
~\textit{L}}}
%
seil\protect\index{Sachverzeichnis}{seil} uberall gleich starck ist. 
\pend
\pstart
Alleine\lbrack,\rbrack\
wenn man sezet,
%
\edtext{daß ein}{%
\lemma{daß}\Bfootnote{%
\textit{(1)}~wie
\textit{(2)}~ein%
~\textit{L}}}
%
theil schwächer als das andere
(\protect\vphantom)%
wie denn solches in praxi\protect\index{Sachverzeichnis}{praxis} nicht zu vermeiden,%
\protect\vphantom()
so komt es auff eins hinauß,
das seil\protect\index{Sachverzeichnis}{seil} sey lang oder kurz,
wenn ein gewichte\protect\index{Sachverzeichnis}{gewicht} daran henget;
denn nicht nur das gewicht,
sondern auch die feder\protect\index{Sachverzeichnis}{feder} oder Spannung der andern theile arbeitet gegen das
%
\edtext{schwächste, dahehr}{%
\lemma{schwächste,}\Bfootnote{%
\textit{(1)}~und
\textit{(a)}~gegen
\textit{(b)}~muß also
\textit{(2)}~dahehr%
~\textit{L}}}
%
ob schon das gewicht die
%
\edtext{krafft\protect\index{Sachverzeichnis}{krafft} nicht ganz auff iedes theil wenden kan,}{%
\lemma{krafft}\Bfootnote{%
\textit{(1)}~so es auff die gespanten theile wendet, ie
\textit{(2)}~nicht ganz \lbrack...\rbrack\ wenden kan,% auff iedes theil
~\textit{L}}}
%
so macht doch der gespanten theile wiederstand per suam vim Elasticam\protect\index{Sachverzeichnis}{vis elastica},
%
\edtext{daß iedes theil}{%
\lemma{daß}\Bfootnote{%
\textit{(1)}~sie
\textit{(2)}~iedes theil%
~\textit{L}}}%
\lbrack,\rbrack\
%
in sonderheit von der gantzen krafft\protect\index{Sachverzeichnis}{krafft} gleichsam, alternative angegriffen wird, und also das schwächste überwunden wird.
\pend%
%
\pstart%
Weil nun ie länger das seil ie großer der unterschied der theile,
%
\edtext{und ie ehe ein}{%
\lemma{und}\Bfootnote{%
\hspace*{-0,5mm}ie
\textit{(1)}~mehr \textlangle die\textrangle\
\textit{(2)}~ehe ein%
~\textit{L}}}
%
aller schwächstes darunter,
so pflegen auch lange seile\protect\index{Sachverzeichnis}{seil} ehe zu reißen als kurze.
\pend%
%%
%%  Ende des Stücks auf Bl. 57r
%%