%   % !TEX root = ../../VIII,3_Rahmen-TeX_8-1.tex
%
%
%   Band VIII, 3 N.~??A44
%   Signatur/Tex-Datei: LH_37_05_123
%   RK-Nr. 60338
%   \ref{60338}
%   Überschrift: Corpora impulsa agunt a se ipsis [ex: Defendi potest]
%   Modul: Mechanik / AEF (Elastizität)
%   Datierung: (?) Frühjar 1677 bis Anfang 1678
%   WZ: (keins)
%   SZ: (keins)
%   Bilddateien (PDF): (keine)
%
%
\selectlanguage{ngerman}%
\frenchspacing%
%
\begin{ledgroupsized}[r]{120mm}%
\footnotesize%
\pstart%
\noindent\textbf{Überlieferung:}%
\pend%
\end{ledgroupsized}%
\begin{ledgroupsized}[r]{114mm}%
\footnotesize%
\pstart%
\parindent -6mm%
\makebox[6mm][l]{\textit{L}}%
Aufzeichnung:
LH XXXVII~5 Bl.~123.
Ein Blatt 4\textsuperscript{o}, schräg beschnitten (etwa 16,5 x 18 cm);
Papiererhaltungsmaßnahmen.
Eineinhalb Seiten.
Auf Bl.~123~r\textsuperscript{o} in der oberen linken Ecke, vermutlich von fremder Hand vermerkt: \textit{ad 41}.
\pend%
\end{ledgroupsized}%
%
\vspace*{5mm}%
\begin{ledgroup}%
\footnotesize%
\pstart%
\noindent%
\textbf{Datierungsgründe:}
In der vorliegenden Aufzeichnung N.~\ref{60338} %??A44 
hält Leibniz fest, dass die Veränderung des kinetischen Zustands eines gestoßenen Körpers ursächlich auf die innere elastische Kraft dieses Körpers selbst zurückgehe und nur scheinbar vom stoßenden Körper herrühre, welcher demnach als bloße Gelegenheitsursache der Veränderung zu betrachten sei.
Hieraus zieht Leibniz schließlich die Folgerung, dass in jedem Körper ursprünglich eine unendliche Kraft (\textit{vis}) und somit eine unendliche \textit{quantitas motus} vorliegen müsse (S.~\refpassage{LH_37_05_123v_iqciqmdfm-1}{LH_37_05_123v_iqciqmdfm-2}).
Diese Gleichsetzung von (kinetischer) Kraft und Bewegungsquantum lässt sich als Zeichen dafür deuten, dass die Aufzeichnung N.~\ref{60338} %??A44 
noch vor der eigenhändig auf Januar 1678 datierten \textit{Scheda VIII de corporum concursu} (N.~\ref{dcc_08}) entstand.
Denn dort verkündet Leibniz zum ersten Mal die Entdeckung
\textendash\ Geburtsstunde seiner \textit{réforme de la dynamique} (\textsc{Fichant} 1994\cite{01056})~\textendash,
dass die \textit{vis} eines sich bewegenden Körpers dem Produkt der Masse und des Quadrats der Geschwindigkeit $mv^2$ entspreche und daher grundsätzlich anders als dessen Bewegungsquantum \textit{mv} sei (N.~\ref{dcc_08}, S.~\refpassage{LH_37_05_086r_reformatio_idzg-1}{LH_37_05_086r_reformatio_idzg-2}).
\pend%
%
\pstart%
Die in der vorliegenden Aufzeichnung geäußerte kausaltheoretische These vertritt Leibniz freilich auch in späteren Texten: etwa in den auf Sommer 1678 bis Winter 1680/81~(?) datierbaren \textit{Definitiones cogitationesque metaphysicae} (\textit{LSB} VI,~4 N.~267, S.~1401.1\textendash5\cite{01339}) sowie noch in der Stoßlehre der \textit{Dynamica} (pars II, sectio III, prop.~6\cite{01354}), wo selbst das in N.~\ref{60338} angeführte Gleichnis des vom Ufer zurückgestoßenen Schiffes wieder vorkommt (\textit{LMG} VI, S.~409;\cite{01043} vgl. N.~\ref{60338}, S.~\refpassage{LH_37_05_123r_navisexripa-1}{LH_37_05_123r_navisexripa-2}).
An keiner dieser Parallelstellen aber wird die innere elastische Kraft der Körper als Bewegungsgröße beschrieben.
Eine Entstehung von N.~\ref{60338} nach Januar 1678 erweist sich somit als unwahrscheinlich, sie kann jedoch nicht ausgeschlossen werden.
\pend%
\pstart%
Der Terminus post quem der Datierung ist ebenfalls mit Unsicherheit behaftet.
Mit den Fragen der Stoßlehre, die der Aufzeichnung N.~\ref{60338} %??A44 
zugrundeliegen, hatte sich Leibniz bekanntlich seit dem Sommer 1669 befasst.
Die Art, wie elastischer und plastischer Stoß in N.~\ref{60338} %??A44 
beschrieben und unterschieden werden (S.~\refpassage{LH_37_05_123r_ostensumst_hjjd-1}{LH_37_05_123r_ostensumst_hjjd-2}; \refpassage{LH_37_05_123v_plastischerstoss-1}{LH_37_05_123v_plastischerstoss-2}), setzt aber die Bekanntschaft mit J.~Wallis'%
\protect\index{Namensregister}{\textso{Wallis} (Wallisius), John 1616\textendash1703}
und E.~Mariottes%
\protect\index{Namensregister}{\textso{Mariotte}, Edme, Seigneur de Chazeuil ca. 1620\textendash1684}
späteren Überlegungen über den Stoß voraus, mit denen sich Leibniz besonders zwischen den letzten Monaten 1674 und dem Sommer 1675 auseinandergesetzt hatte (vgl. \textit{LSB} VIII,~2 N.~8,\cite{01343} S.~89\textendash93; N.~50\cite{01292}).
Die Aufzeichnung N.~\ref{60338} %??A44 
entstand daher höchstwahrscheinlich nachher. 
Die im Text vertretene kausaltheoretische Ansicht dürfte aber vornehmlich an die seit März 1677 entwickelten Untersuchungen zum Stoßgesetz anknüpfen, bei denen Leibniz das Phänomen des elastischen Stoßes in ähnlicher Weise wie Wallis und Mariotte erörtert
(vgl. die Erläuterung zu S.~\refpassage{LH_37_05_123r_ostensumst_hjjd-1}{LH_37_05_123r_ostensumst_hjjd-2}).
Hieraus ergibt sich die vorgeschlagene Datierung.
Eine frühere Entstehungszeit der Aufzeichnung N.~\ref{60338} %??A44 
(jedenfalls nach 1674) ist dennoch nicht auszuschließen.
\pend%
%
\pstart%
Bemerkenswert ist schließlich, dass in der eigenhändig auf Januar 1678 datierten \textit{Scheda VI-II de corporum concursu} stoßtheoretische Überlegungen ebenfalls eine Digression über die kausale Selbständigkeit mechanisch interagierender Körper veranlassen (N.~\ref{dcc_06-2}, %??S01\textsubscript{8}
S.~\refpassage{LH_35_09_23_018v_occasionalismus_jyr-1}{LH_35_09_23_018v_occasionalismus_jyr-2}).
Anders als in N.~\ref{60338} %??A44 der Aufzeichnung 
aber vertritt Leibniz dort eine okkasionalistische Betrachtungsweise, die den Status einer echten Wirkursache nicht den einzelnen Körpern bzw. deren ursprünglicher elastischer Kraft zuweist, sondern nur Gott.
\pend%
\end{ledgroup}%
%
\selectlanguage{latin}%
\frenchspacing%
%
%
\newpage%
\count\Bfootins=1100
\count\Afootins=1200
\count\Cfootins=1100
%\vspace*{8mm}%
\pstart%
\normalsize%
\noindent%
%
\lbrack123~r\textsuperscript{o}\rbrack\ %    %    %    %    Blatt 123r
%
Defendi potest corpus non impelli%
\protect\index{Sachverzeichnis}{corpus impulsum immediate ab alio}%
\protect\index{Sachverzeichnis}{impulsus immediatus}
%
\edtext{immediate}{%
\lemma{immediate}\Bfootnote{%%
\textit{erg.~L}}}
%
ab alio corpore,%
\protect\index{Sachverzeichnis}{corpus impellens}
sed occasione alterius a se%
\protect\index{Sachverzeichnis}{occasio agendi}%
\protect\index{Sachverzeichnis}{corpus impulsum occasione alterius}
%
\edtext{ipso neque}{%
\lemma{ipso}\Bfootnote{%%
\textit{(1)}~. Si enim
\textit{(2)}~neque \textit{L}}}
%
adeo nisi propria vi cieri.%
\protect\index{Sachverzeichnis}{vis corporis propria}
Nam
%
\edlabel{LH_37_05_123r_ostensumst_hjjd-1}%
ostensum est%
\edtext{}{%
{\xxref{LH_37_05_123r_ostensumst_hjjd-1}{LH_37_05_123r_ostensumst_hjjd-2}}%
{\lemma{ostensum \lbrack...\rbrack\ recedere}\Cfootnote{%
Leibniz erörtert in dieser Weise den elastischen Stoß etwa in dem eigenhändig auf März 1677 datierten Entwurf N.~\ref{RK57266-1} % \textit{Elastrum est causa imperfectionis in corporum concursu}
und verweist dabei (S.~\refpassage{LH_37_05_161v_mariottewallis-1}{LH_37_05_161v_mariottewallis-2}) auf \textit{Mariotti ac Wallisii rationem explicandi};%
\protect\index{Namensregister}{\textso{Wallis} (Wallisius), John 1616-1703}%
\protect\index{Namensregister}{\textso{Mariotte}, Edme, Seigneur de Chazeuil ca. 1620-1684}
siehe hierzu
\textsc{J.~Wallis}, \cite{00301}\textit{Mechanica}, pars~III, cap.~XI u.~XIII (London 1670\textendash1671, Bd.~II, S.~660\textendash682 u.~686\textendash707; \cite{01008}\textit{WO}~I, S.~1002\textendash1015 u.~1018\textendash1031)
sowie
\textsc{E.~Mariotte}, \cite{00311}\textit{De la percussion}, partie I, prop.~XIII u.~XIX (Paris 1673, S.~68\textendash72 u.~115\textendash119).
Ähnlich fasst Leibniz den elastischen Stoß auch später in dem eigenhändig auf Januar und Februar 1678 datierten Textkomplex \textit{De corporum concursu} (N.~\ref{dcc_05} u.~\ref{dcc_06-1}) auf.%
}}}
%
corpora concursu%
\protect\index{Sachverzeichnis}{concursus corporum}
%
\edtext{comprimi,%
\protect\index{Sachverzeichnis}{corpus compressum}
prius quam propelli,%
\protect\index{Sachverzeichnis}{corpus propulsum}
et restituente sese Elasmate%
\protect\index{Sachverzeichnis}{elasma se restituens}
a se invicem proprio nisu%
\protect\index{Sachverzeichnis}{nisus corporis recedentis}
recedere,%
\protect\index{Sachverzeichnis}{corpus proprio nisu recedens}%
\edlabel{LH_37_05_123r_ostensumst_hjjd-2}
\edlabel{LH_37_05_123r_navisexripa-1}%
quemadmodum nos ex navi%
\protect\index{Sachverzeichnis}{navis}
conto%
\protect\index{Sachverzeichnis}{contus}
ripam%
\protect\index{Sachverzeichnis}{ripa}
aut fundum%
\protect\index{Sachverzeichnis}{fundus}
impellendo
una cum navi%
\protect\index{Sachverzeichnis}{navis}
inde recedimus.%
\edlabel{LH_37_05_123r_navisexripa-2}
Haec}{%
\lemma{comprimi,}\Bfootnote{%
\textit{(1)}~nec prius propell
\textit{(2)}~prius quam propelli,
\textit{(a)}~et antequam
\textit{(b)}~et restituente % sese Elasmate 
\lbrack...\rbrack\ a se
\textit{(aa)}~mutuo \textlangle vi\textrangle\
\textit{(bb)}~invicem proprio nisu recedere
\textit{(aaa)}~%
\textbar~mutuo \textit{erg.}~%
\textbar\ conari,
\textit{(bbb)}~, quemadmodum
\textbar~%
\textit{(1)}~is qui
\textit{(2)}~ex navi
\textit{(3)}~ex
\textit{(4)}~si quis
\textit{(5)}~nos ex navi \textit{erg.}~%
\textbar\ conto
\textit{(aaaa)}~aut
\textit{(bbbb)}~ripam aut fundum impellendo
\textit{(aaaaa)}~ex navi nos
\textit{(bbbbb)}~una cum navi inde recedimus
\textbar~atque abspellimur \textit{gestr.}~%
\textbar~. Haec%
~\textit{L}}}
%
partim experimentis,%
\protect\index{Sachverzeichnis}{experimentum}
partim etiam firmis%
\protect\index{Sachverzeichnis}{ratio firma}
%
\edtext{rationibus ostendi possunt,
quoniam nihil impetum momento accipit%
\protect\index{Sachverzeichnis}{impetus acceptus momento}
sed per gradus%
\protect\index{Sachverzeichnis}{impetus acceptus per gradus}
intermedios%
\protect\index{Sachverzeichnis}{gradus intermedius}
a quiete,%
\protect\index{Sachverzeichnis}{quies}
quod}{%
\lemma{rationibus}\Bfootnote{%
\textit{(1)}~ostendimus
\textit{(2)}~ostendi possunt quoniam
\textit{(a)}~ostensum est
\textit{(b)}~certum demonstr
\textit{(3)}~nihil impetum momento 
\textit{(a)}~accipere
\textit{(b)}~accipit sed % per gradus intermedios 
\lbrack...\rbrack\ a quiete
\textit{(aa)}~omnes transire
\textit{(bb)}~, quod%
~\textit{L}}}
%
fit restitutione Elastri%
\protect\index{Sachverzeichnis}{elastrum se restituens}%
\protect\index{Sachverzeichnis}{restitutio elastri}%
\lbrack;\rbrack\
%
\edtext{a solo igitur Elastro%
\protect\index{Sachverzeichnis}{elastrum}
oritur motuum translatio%
\protect\index{Sachverzeichnis}{translatio motus}
atque communicatio:%
\protect\index{Sachverzeichnis}{communicatio motus}%
}{\lemma{a}\Bfootnote{%
\hspace{-0,5mm}solo \lbrack...\rbrack\ atque communicatio
\textit{erg.~L}}}
%
porro impellens,%
\protect\index{Sachverzeichnis}{corpus impellens}
quod Elastrum intendit%
\protect\index{Sachverzeichnis}{elastrum intensum}%
\lbrack,\rbrack\
non vim ipsi tribuit,%
\protect\index{Sachverzeichnis}{vis tributa}
%
\edtext{sed determinat%
\protect\index{Sachverzeichnis}{vis determinata}
praestita occasione agendi.%
\protect\index{Sachverzeichnis}{occasio agendi}%
}{%
\lemma{sed}\Bfootnote{%
\textit{(1)}~occasionem praestat
\textit{(2)}~determinat praestita occasione agendi.%
~\textit{L}}}
%
Intus enim perfluit velocissima materia,%
\protect\index{Sachverzeichnis}{materia intus perfluens}%
\protect\index{Sachverzeichnis}{materia velocissima}
quae nihil insoliti inveniens,
neque sentitur,%
\protect\index{Sachverzeichnis}{materia insensibilis}
sed obstaculo objecto%
\protect\index{Sachverzeichnis}{obstaculum objectum}
ostendit quid possit;
quemadmodum flumina si coerceantur.%
\protect\index{Sachverzeichnis}{flumen coercitum}
Vis igitur qua corpus cietur%
\protect\index{Sachverzeichnis}{vis ciens}
intra ipsum est.%
\protect\index{Sachverzeichnis}{vis intra corpum}
\pend%
%
\pstart%
At, inquies, saltem partes in compressione%
\protect\index{Sachverzeichnis}{compressio partium}
impelluntur et aliunde vim accipiunt.%
\protect\index{Sachverzeichnis}{vis accepta aliunde}
%
\edtext{Respondeo,
quod de toto diximus
etiam de parte dicendum esse,}{%
\lemma{Respondeo,}\Bfootnote{%
\textit{(1)}~partes ips
\textit{(2)}~quod de % toto diximus etiam de parte
\lbrack...\rbrack\ dicendum esse,%
~\textit{L}}}
%
ut vicissim illae
non nisi suis prius partibus compressis%
\protect\index{Sachverzeichnis}{pars corporis compressa}
ac sese restituentibus%
\protect\index{Sachverzeichnis}{pars corporis se restituens}
impellantur.%
\protect\index{Sachverzeichnis}{pars corporis impulsa}
Quae quidem in omnibus partium partibus continuata%
\protect\index{Sachverzeichnis}{partes partium}
utcunque subdivisione vera sunt,%
\protect\index{Sachverzeichnis}{subdivisio partium}
et nihil movetur,
quin praecedat
%
infinitorum aliorum
%\edtext{\lbrack infinitarum aliarum\rbrack}{%
%\lemma{infinitorum aliorum}\Bfootnote{%
%\textit{L~ändert Hrsg.}}}
%
motus%
\protect\index{Sachverzeichnis}{motus infinitorum aliorum}
per respondentis temporis partes%
\protect\index{Sachverzeichnis}{pars temporis}%
\lbrack,\rbrack\
etiam proportione exiguas distributas.
Habent enim omnia quendam flexilitatis gradum,%
\protect\index{Sachverzeichnis}{flexilitas}%
\protect\index{Sachverzeichnis}{gradus flexilitatis}
nihilque infinitae rigiditatis est,%
\protect\index{Sachverzeichnis}{rigiditas infinita}
hinc semper prius movetur pars,%
\protect\index{Sachverzeichnis}{pars mota prius quam totum}
quam
\edlabel{LH_37_05_123r_12301}%
totum.%
\edtext{}{%
{\xxref{LH_37_05_123r_12301}{LH_37_05_123r_12302}}%
{\lemma{totum.}\Bfootnote{%
\textit{(1)}~Caeterum
\textit{(2)}~Dici
\textit{(a)}~porro
\textit{(b)}~etiam potest%
~\textit{L}}}}
\pend%
%\newpage%
%
\pstart%
Dici etiam potest%
\edlabel{LH_37_05_123r_12302}
non tantum corpus omne moveri a se ipso,%
\protect\index{Sachverzeichnis}{corpus motum a se ipso}
vel eo quod in
%
\edtext{ipso est,}{%
\lemma{ipso}\Bfootnote{\hspace{-0,5mm}%
\textbar~conclusum \textit{gestr.}~%
\textbar\ est%
~\textit{L}}}
%
sed etiam ex ipsius statu praecedenti%
\protect\index{Sachverzeichnis}{status corporis praecedens}
consequi praesentem motum,%
\protect\index{Sachverzeichnis}{motus consequens ex statu praecedenti}%
\protect\index{Sachverzeichnis}{motus praesens}
ita ut agat sponte,%
\protect\index{Sachverzeichnis}{corpus agens sponte}
ac nihil absolute loquendo in
%
\edtext{natura sit}{%
\lemma{natura}\Bfootnote{%
\textit{(1)}~esse
\textit{(2)}~sit%
~\textit{L}}}
%
violentum.%
\protect\index{Sachverzeichnis}{natura}%
\protect\index{Sachverzeichnis}{violentum}
Nam etsi videatur compressio%
\protect\index{Sachverzeichnis}{compressio ab externo}
saltem ab externo fieri,
\edtext{attamen cum}{\lemma{attamen}\Bfootnote{%
\textit{(1)}~re vera
\textit{(2)}~comm
\textit{(3)}~cum%
~\textit{L}}}
%
omnis compressio motum partium%
\protect\index{Sachverzeichnis}{motus partium in compresssione}
%
\edtext{contineat,
omnis autem res moveatur}{\lemma{contineat,}\Bfootnote{%
\textit{(1)}~motus autem sit fact
\textit{(2)}~omnis autem res moveatur%
~\textit{L}}}
%
a se ipsa,%
\protect\index{Sachverzeichnis}{res mota a se ipsa}
%
\edtext{ut ostensum est,}{%
\lemma{ut ostensum est}\Cfootnote{%
Soeben in der vorliegenden Aufzeichnung N.~\ref{60338}.%??A44
}}
%
nec compressio ab externo
%
\edtext{\lbrack fieri\rbrack}{%
\lemma{facta}\Bfootnote{%
\textit{L~ändert Hrsg.}}}
%
poterit,%
\protect\index{Sachverzeichnis}{compressio ab externo}
nihil aliud
%
\edtext{ergo est}{\lemma{ergo}\Bfootnote{%
\textit{(1)}~fuerit
\textit{(2)}~est%
~\textit{L}}}
%
corpus externum,%
\protect\index{Sachverzeichnis}{corpus externum}
quam ut sit comitans atque connexum,%
\protect\index{Sachverzeichnis}{corpus comitans}%
\protect\index{Sachverzeichnis}{corpus connexum}
quod nobis
rerum interiora ignorantibus%
\protect\index{Sachverzeichnis}{interiora rerum}
causa videtur.%
\protect\index{Sachverzeichnis}{causa apparens}
%
\edtext{Interim notio}{%
\lemma{interim}\Bfootnote{%
\textit{(1)}~si
\textit{(2)}~certa not
\textit{(3)}~sit
\textit{(4)}~notio%
~\textit{L}}}
%
causae%
\protect\index{Sachverzeichnis}{notio causae}
assignari poterit%
\lbrack,\rbrack\
qua adhibita recte
%
\edtext{dicetur motum corporis}{%
\lemma{dicetur}\Bfootnote{%
\textit{(1)}~corporis
\textit{(2)}~motum corporis%
~\textit{L}}}
%
unius esse causam motus%
\protect\index{Sachverzeichnis}{causa motus mediata}
corporis alterius,
etsi non sit causa immediata.%
\protect\index{Sachverzeichnis}{causa motus immediata}
%
\lbrack123~v\textsuperscript{o}\rbrack%    %    %    %    Blatt 123v
%
\pend%
%
\pstart%
\edtext{Quando%
\edlabel{LH_37_05_123v_plastischerstoss-1}
corpus unum incurrit in aliud quiescens%
\protect\index{Sachverzeichnis}{corpus incurrens in quiescens}%
}{%
\lemma{Quando}\Bfootnote{%
\textit{(1)}~duo corpora concurrunt
\textit{(2)}~corpus unum % incurrit in 
\lbrack...\rbrack\ aliud quiescens%
~\textit{L}}}
%
et post ictum simul%
\protect\index{Sachverzeichnis}{ictus consumtus}
%
\edtext{procedunt dubites an}{%
\lemma{procedunt}\Bfootnote{%
\textit{(1)}~non videtur
\textit{(2)}~dubites an%
~\textit{L}}}
%
commode defendi possit
%
\edtext{corpus quod quieverat%
\protect\index{Sachverzeichnis}{corpus quiescens}
impulsum esse a se ipso.%
\protect\index{Sachverzeichnis}{corpus impulsum a se ipso}%
}{%
\lemma{corpus}\Bfootnote{%
\textit{(1)}~impelli a se ipso
\textit{(2)}~quod quieverat %
impulsum esse a 
\lbrack...\rbrack\ se ipso.%
~\textit{L}}}
%
Nam ictus%
\protect\index{Sachverzeichnis}{ictus consumtus}
omnis in partibus mollibus%
\protect\index{Sachverzeichnis}{pars mollis}
consumtus est.%
\edlabel{LH_37_05_123v_plastischerstoss-2}
Verum cum certum videatur
nullum corpus momento motum notabilem accipere,%
\protect\index{Sachverzeichnis}{motus acceptus}
cogitandum,
quomodo res explicari possit.
\pend
%
\pstart
Videtur%
\edlabel{LH_37_05_123v_iqciqmdfm-1}
omne corpus jam tum
in se eam omnem vim habere%
\protect\index{Sachverzeichnis}{vis in corpore}
quam unquam acquirere potest%
\lbrack,\rbrack\
adeoque infinitam,%
\protect\index{Sachverzeichnis}{vis infinita}
sunt enim infinitorum corporum motus%
\protect\index{Sachverzeichnis}{motus infinitorum corporum}
magnae celeritatis%
\protect\index{Sachverzeichnis}{celeritas magna}%
\protect\index{Sachverzeichnis}{motus magnae celeritatis}
in eo inclusi,%
\protect\index{Sachverzeichnis}{motus in corpore inclusus}
ita ut in quolibet corpore insit quantitas motus%
\protect\index{Sachverzeichnis}{quantitas motus in corpore insita}
data finita major.%
\protect\index{Sachverzeichnis}{quantitas motus major finita}%
\edlabel{LH_37_05_123v_iqciqmdfm-2}
%
Corpus omne videtur
non nisi a seipso impelli,%
\protect\index{Sachverzeichnis}{corpus impulsum a se ipso}
sive motu suo,%
\protect\index{Sachverzeichnis}{corpus impulsum a motu suo}
sive elaterio%
\protect\index{Sachverzeichnis}{elaterium impellens}
quod a motu intestino proficiscitur.%
\protect\index{Sachverzeichnis}{elaterium a motu intestino profectum}
Verum inquies\lbrack,\rbrack\
cum corpus in quiescens incurrit,%
\protect\index{Sachverzeichnis}{corpus incurrens in quiescens}
prius ejus partes impellit ac comprimit%
\protect\index{Sachverzeichnis}{pars corporis impulsa}%
\protect\index{Sachverzeichnis}{pars corporis compressa}
quam ullum elastrum existere possit;
respondeo
in omni corpore
quod totum quiescit%
\protect\index{Sachverzeichnis}{corpus quiescens}
partes esse in celerrimo motu,%
\protect\index{Sachverzeichnis}{motus partium celerrimus}
eoque magis
quo est firmius,
et corpus incurrens%
\protect\index{Sachverzeichnis}{corpus incurrens}
atque impellens%
\protect\index{Sachverzeichnis}{corpus impellens}
in
%
\edtext{hunc motum partium extimarum%
\protect\index{Sachverzeichnis}{motus partium extimarum}%
}{%
\lemma{hunc}\Bfootnote{%
\textit{(1)}~intestinum
\textit{(2)}~motum partium extimarum \textit{L}}}
%
incidere,
atque hinc
%
reflectere%
\protect\index{Sachverzeichnis}{pars corporis reflexa}
%\edtext{\lbrack???reflecti\rbrack}{%
%\lemma{reflectere}\Bfootnote{%
%\textit{L~ändert Hrsg.}}}
%
illas partes. 
\pend
%
%
%
%    %    %    %   Ende des Textes auf Blatt 123v