%   % !TEX root = ../../VIII,3_Rahmen-TeX_9-0.tex
%  
%   Signatur/Tex-Datei:	CompositioniNonFidendum_intro
%
%   Einleitung zu:
%   RK-Nr. 	41169+41167+60318
%							
%   Datierung:		1686 (?) bis Oktober 1687
%
%
\selectlanguage{ngerman}
\frenchspacing
%
\vspace{5mm}
\begin{ledgroup}
\footnotesize%
\pstart
\noindent%
Die vorliegenden drei Stücke hängen ihrer Entstehung nach eng miteinander zusammen. Es handelt sich um die Aufzeichnungen N.~\ref{RK41169} und N.~\ref{RK41167} und um das aus diesen hervorgehende Konzept N.~\ref{RK60318}.
\pend
%
\pstart
%
Zur Datierung von N.~\ref{41169} kann zunächst festgestellt werden, dass die Aufzeichnung als Vorlage für N.~\ref{60318} gedient hat.
%
Der Text von N.~\ref{41169} wurde zum Abschnitt auf S.~\refpassage{37_05_094-095_9a}{37_05_094-095_9b}  von N.~\ref{60318} ausgebaut.
%
Noch deutlicher ist das Abhängigkeitsvherhältnis im Fall des Diagramms von N.~\ref{41169} (\lbrack\textit{Fig.~1}\rbrack):
%
Bei der Abfassung des erwähnten Abschnitts von N.~\ref{60318} muss Leibniz sich an der Zeichnung von N.~\ref{41169}
%
orientiert haben, nicht an dem entsprechenden, leicht vereinfachten
%
Diagramm auf dem Träger von N.~\ref{60318} selbst (siehe dort \lbrack\textit{Fig.~1a}\rbrack). 
%
Denn die Punktebezeichnungen in der Passage auf S.~\refpassage{37_05_094-095_9a}{37_05_094-095_9b}  von N.~\ref{60318} 
%
stimmen mit \lbrack\textit{Fig.~1}\rbrack\ von N.~\ref{41169} (dort als \lbrack\textit{Fig.~1b}\rbrack\ wiedergegeben) genau überein, 
%
während sie von \lbrack\textit{Fig.~1a}\rbrack\ von N.~\ref{60318} abweichen 
%
(\protect\vphantom)die überdies erst an einer späteren Stelle des Stücks gezeichnet wurde, als sie erwähnt wird\protect\vphantom().
%
Damit steht fest, dass N.~\ref{41169} eine unmittelbare Vorlage für die Abfassung des Konzepts N.~\ref{60318} bildete; demnach
%
dürfte es kurz vor ihm entstanden sein.
%
\pend
%
\pstart
%
Bei N.~\ref{41167} liegen folgende Anhaltspunkte für die Datierung vor.
%
Die Aufzeichnung ist auf dem Umschlag eines 
%
an Leibniz in Hannover\protect\index{Ortsregister}{Hannover} adressierten
%
Briefs eines unbekannten Korrespondenten verfasst.
%
Dieser Umstand schließt zunächst eine Entstehung im Zeitraum Ende Oktober 1687 bis Juni 1690 aus,
%
da Leibniz sich auf Reisen in Süddeutschland, Österreich und Italien befand.
%
Zudem zieht Leibniz in der Aufzeichnung eine Anwendung der \glqq regula alternorum\grqq\ oder \glqq alternativorum\grqq\ 
%
auf das Problem der anteiligen Übertragung der Kraft auf mehrere Körper beim schiefen Stoß in Betracht.
%
Er hatte diese Regel im Aufsatz \cite{01098}\glqq Demonstratio geometrica regulae apud Staticos receptae\grqq\
%
(\cite{01023}\textit{Acta Eruditorum} vom November 1685, S.~501\textendash505) besprochen, 
%
den er an zwei Stellen von N.~\ref{41167} ausdrücklich erwähnt, und dessen Veröffentlichung daher einen Terminus post quem
%
für das Stück abgibt. 
%
Darüber hinaus gibt Leibniz in der ersten Erwähnung des Aufsatzes ein falsches Datum an: \glqq1686\grqq\ statt 1685. 
%
Ein solcher Fehler dürfte ihm im Jahr 1685 kaum unterlaufen sein; deshalb kann von einer Abfassung ab 1686 ausgegangen werden.
%
Mit dem Konzept N.~\ref{60318} ist zugleich ein Terminus ante quem für die Entstehung der Aufzeichnung gegeben:
%
N.~\ref{41167}, wie bereits N.~\ref{41169}, hat als Vorlage für einen Teil von N.~\ref{60318} gedient, 
%
nämlich für den Abschnitt über den rechtwinkigen schiefen Stoß dreier Körper, 
%
zu dessen Beginn Leibniz sich ausdrücklich auf die Ergebnisse einer \glqq separata scheda\grqq\
%
über die Anwendung der \glqq regula alternorum\grqq\ auf den schiefen Mehrkörperstoß beruft (\protect\vphantom)siehe S.~\refpassage{37_05_094-095_10a}{37_05_094-095_10b}\protect\vphantom().
% 
Mit der \glqq scheda\grqq\ ist die Aufzeichnung N.~\ref{41167} gemeint, die daher vor N.~\ref{60318} entstanden sein muss.
%
Der für N.~\ref{41167} ermittelte Terminus post quem (1686) gilt wiederum auch für N.~\ref{60318}.
%
\pend
%
\pstart
Schließlich bieten folgende Überlegungen  einen unmittelbaren Terminus ante quem für N.~\ref{RK60318},
%
der zugleich ein mittelbarer für N.~\ref{RK41169} und N.~\ref{RK41167} ist.
%
Das Wasserzeichen in Bl.~98 (Papier aus dem Harz) ist nach heutigem Kenntnisstand ausschließlich für die Mitte der 1680er Jahre belegt.
%
Das Zeichen in Bl.~96 kommt im Leibniz-Nachlass im Zeitraum 1683\textendash1687 häufig vor, unter anderem im Konzept N.~\ref{60320}. 
%
Es handelt sich dabei um Papier von der Papiermühle in Sedemünder bei Hannover, 
%
dessen Wasserzeichen aus dem gekrönten Monogramm \glqq EA\grqq\ 
%
(\protect\vphantom)für \protect\index{Namensregister}{\textso{Braunschweig-L{\"u}neburg}, Ernst August von, Herzog und Kurf{\"u}rst von Hannover, 1679\textendash1698}Herzog Ernst August\protect\vphantom()
%
und der Jahreszahl \glqq 1680\grqq\ besteht.
%
Dieses Papier wurde nur ca.\ zwischen 1680 und 1690 fabriziert; danach wurde die Jahreszahl durch \glqq 1690\grqq\ abgelöst 
%
(das entsprechende Wasserzeichen ist bei Leibniz für die frühen 1690er Jahre belegt).
%
Da Leibniz von Ende Oktober 1687 bis Juni 1690 sich auf Reisen durch Süddeutschland, Österreich und Italien befand, 
%
hat er N.~\ref{60318} aller Wahrscheinlichkeit nach vor Antritt seiner Reise verfasst.
%
Dies wird durch den Vergleich mit dem Konzept N.~\ref{RK60320} bestätigt, welches
%
die in N.~\ref{RK60318} gewonnenen Erkenntnisse in eine Systematik der verschiedener Stoßarten integriert 
%
und ihre Herleitung aus allgemeinen Grundsätzen anstrebt.
%
N.~\ref{RK60320} ist mit Sicherheit vor Leibnizens Reise entstanden (siehe die Datierungsgründe),
%
was deshalb auch auf N.~\ref{RK60318} zutrifft.
%
\pend
%
\pstart
Daraus ergeben sich die Datierungsspannen
%
1686 (?) bis Oktober 1687 für die Aufzeichnung N.~\ref{41169}
%
und 1686 bis Oktober 1687 für N.~\ref{41167} und N.~\ref{60318}.
%
\pend
\end{ledgroup}
%