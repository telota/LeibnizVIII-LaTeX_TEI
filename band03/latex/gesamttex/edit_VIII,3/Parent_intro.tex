%   % !TEX root = ../../VIII,3_Rahmen-TeX_9-0.tex
%  
%   Band VIII, 3		Rubrik STOSS
%
%   Signatur/Tex-Datei:	Parent-intro
%
%   RK-Nr. 	Gemeinsame Einleitung und Datierung zu 55822+61042
%   
%   Unterrubrik:			Auszüge
%
%
%
\selectlanguage{ngerman}
\frenchspacing
%
\vspace{5mm}
\begin{ledgroup}
\footnotesize
\pstart
\noindent%
\textbf{Datierungsgründe:}
Die Entstehung der nachfolgenden zwei Stücke, Leibnizens kommentierter Auszüge aus 
\protect\index{Namensregister}{\textso{Parent}, Antoine 1666\textendash1726}Antoine Parents 
\cite{01500}\textit{Élémens de méchanique et de physique} (N.~\ref{RK55822}) und seiner Rezension davon (N.~\ref{RK61042}), lässt sich anhand des Briefwechsels beleuchten.%
\pend
%
\pstart
Während seines kurzen Aufenthalts in \protect\index{Ortsregister}{Helmstedt (Helmaestadium)}Helmstedt 
auf der Rückreise aus Wien\protect\index{Ortsregister}{Wien}
und \protect\index{Ortsregister}{Prag}Prag
traf Leibniz am 30.\ Dezember 1700
\protect\index{Namensregister}{\textso{Schmidt} (Schmidius), Johann Andreas 1652\textendash1726}Johannes Andreas Schmidt.
Dieser bot ihm einige Bücher zur Lektüre an, die Leibniz allerdings mitzunehmen vergaß.
%
Am Folgetag, den 31.\ Dezember, schrieb Leibniz aus \protect\index{Ortsregister}{Hannover}Hannover
einen Brief\cite{02056}(\textit{LSB} I, 19 N.~150), in dem er 
\protect\index{Namensregister}{\textso{Schmidt} (Schmidius), Johann Andreas 1652\textendash1726}Schmidt um die Zusendung der Bücher bat, 
wobei er \glqq Parentii Galli librum de re Mechanica\grqq\ namentlich erwähnte,
die um 1700 erschienenen
\cite{01500}\textit{Élémens de méchanique et de physique}.
%
\protect\index{Namensregister}{\textso{Schmidt} (Schmidius), Johann Andreas 1652\textendash1726}Schmidt übersandte die Bücher mit seinem Brief vom 7.\ Januar 1701 
(\cite{02057}\textit{LSB} I, 19 N.~164).
%
Er wünschte sich von Leibniz eine Buchbesprechung der \cite{01500}\textit{Élémens} für die
\cite{01023}\textit{Acta Eruditorum}.
%
Leibniz kam dieser Bitte nach.
%
In seinem Brief vom 15.\ Februar informierte er 
%
\protect\index{Namensregister}{\textso{Schmidt} (Schmidius), Johann Andreas 1652\textendash1726}Schmidt über den Fortschritt (und die Hindernisse) seiner Parent-Lektüre:
%
\glqq Recensionem ejus libri absolvam quam primum. Sic satis progressus eram, sed interrupere alia urgentia...\grqq\ 
(\textit{LSB} I, 19 N.~206, hier S.~423).
%
Diese Unterbrechung lässt sich tatsächlich am Schriftbild der Auszüge (N.~\ref{RK55822}) gut ausmachen (oben auf Bl.~215~r\textsuperscript{o}) und wird durch Leibnizens eigene Bemerkung unterstrichen: \glqq Je n'ay pas le loisir d'achever tous ces chapitres comme j'ay commencé\grqq\ (S.~\refpassage{38_212-215_6a}{38_212-215_6b}). 
%
Die großzügigen Exzerpte der Vorrede und die Kommentierung der ersten Teile der \cite{01500}\textit{Élémens} bis Teil~III, Kap.~9 nehmen etwas mehr als sechs der sieben Folioseiten ein (Bl.~212~r\textsuperscript{o} bis Anfang von Bl.~215~r\textsuperscript{o}),
%
während auf den Rest des Buchs (S.~183\textendash449) nur eine knappe, vergleichsweise flüchtig geschriebene Folioseite (Bl.~215~r\textsuperscript{o}) entfällt. %
\pend
%
\pstart
Die Auszüge müssen spätestens Mitte April fertig vorgelegen haben, denn Leibniz teilte
\protect\index{Namensregister}{\textso{Schmidt} (Schmidius), Johann Andreas 1652\textendash1726}Schmidt 
%
am 29.\ April mit, dass die daraus hervorgegangene Rezension (N.~\ref{RK61042}) bereits an
\protect\index{Namensregister}{\textso{Mencke} (Menken, Menkenius, Menque), Otto 1644\textendash1707}O.~Mencke 
%
abgegangen war (\cite{02058}\textit{LSB} I, 19 N.~336, hier S.~633).
%
Sie erschien anonym in den \cite{01023}\textit{Acta Eruditorum} vom Juni 1701.
\pend
%
%
\pstart
%
Leibnizens Urteil über 
%
\protect\index{Namensregister}{\textso{Parent}, Antoine 1666\textendash1726}Parents 
%
Leistung für die Stoßlehre ist nicht nur in N.~\ref{RK61042}, sondern auch in späteren Briefen an 
%
\protect\index{Namensregister}{\textso{Bernoulli}, Johann 1667\textendash1748}Johann Bernoulli festgehalten.
%
\protect\index{Namensregister}{\textso{Parent}, Antoine 1666\textendash1726}Parent
%
sah sein Hauptergebnis darin, jede Art von Stoß
%
\glqq auf das Gleichgewicht reduziert\grqq\ zu haben, d.\,h.\ auf einen geraden zentralen Stoß, 
%
in dem die Geschwindigkeiten sich reziprok zu den Massen verhalten 
%
(\cite{01500}\textit{Élémens}, Preface, S.~\lbrack1\rbrack).
%
Der Beweis 
%
(\cite{01500}Pars I, chap.~XIII) beruht auf dem Prinzip, dass der gerade zentrale Stoß zweier Körper 
%
nur von ihrer relativen Geschwindigkeit abhängt und dass die Stoßphänomene, abzüglich der gemeinsamen Bewegung des Systems, gleich bleiben. 
%
Dies verdeutlicht er durch die Analogie des Zusammenstoßes zweier Kugeln auf einem fahrenden Schiff, der vom Ufer aus betrachtet wird (S.~\lbrack5\rbrack).
%
Für Leibniz, der in N.~\ref{RK55822} alle diese Stellen exzerpiert hat, ist dieser Ansatz, wie auch die Ergebnisse und die Schiffsanalogie,
%
vor allem mit \protect\index{Namensregister}{\textso{Huygens} (Hugenius, Ugenius, Hugens, Huguens), Christiaan 1629\textendash1695}Huygens fest assoziiert, 
%
obwohl dessen einschlägige Schriften zu seinen Lebzeiten unveröffentlicht blieben
%
und die Abhandlung \cite{00530}\textit{De motu corporum ex percussione} erst 1703, 
%
nach \protect\index{Namensregister}{\textso{Parent}, Antoine 1666\textendash1726}Parents \textit{Élémens}, erschien 
%
(\protect\index{Namensregister}{\textso{Huygens} (Hugenius, Ugenius, Hugens, Huguens), Christiaan 1629\textendash1695}\textsc{C.~Huygens}, \cite{02076}\textit{Opuscula postuma}, 1703, S.~367\textendash398, bes.\ S.~370);
%
siehe dazu die editorische Vorbemerkung zu N~\ref{RK57269}.
%
%
\protect\index{Namensregister}{\textso{Parent}, Antoine 1666\textendash1726}Parent behauptet in der Vorrede (S.~\lbrack3f.\rbrack),
%
er habe seine Ergebnisse selbständig erreicht und
%
erst nach Vollendung der Abhandlung von den Arbeiten 
%
\protect\index{Namensregister}{\textso{Wallis} (Wallisius), John 1616\textendash1703}Wallis', 
\protect\index{Namensregister}{\textso{Mariotte}, Edme, Seigneur de Chazeuil ca. 1620\textendash1684}Mariottes und 
\protect\index{Namensregister}{\textso{Huygens} (Hugenius, Ugenius, Hugens, Huguens), Christiaan 1629\textendash1695}Huygens'  
%
erfahren.
%
In der Rezension beurteilt Leibniz
%
\protect\index{Namensregister}{\textso{Parent}, Antoine 1666\textendash1726}Parents Unabhängigkeitsanspruch 
%
nicht, sondern stellt lediglich fest, dass er in seinen Thesen über den geraden Stoß 
%
\glqq non dissentit quoad conclusiones ab iis, quae \protect\index{Namensregister}{\textso{Mariotte}, Edme, Seigneur de Chazeuil ca. 1620\textendash1684}Mariottus aliique dedere\grqq\ 
%
(siehe \ref{RK61042}, S.~\refpassage{AE_1701_252-256_2a}{AE_1701_252-256_2b}).
%
Allerdings scheint Leibniz um 1704 an die  Abhängigkeit 
%
\protect\index{Namensregister}{\textso{Parent}, Antoine 1666\textendash1726}Parents von \protect\index{Namensregister}{\textso{Huygens} (Hugenius, Ugenius, Hugens, Huguens), Christiaan 1629\textendash1695}Huygens' Methode zu glauben, 
%
wenn er \protect\index{Namensregister}{\textso{Bernoulli}, Johann 1667\textendash1748}Johann Bernoulli
%
die \cite{01500}\textit{Élémens} empfiehlt:
%
\glqq librum Dn.\ Parent Galli quem inscripsit \textit{Elemens de Mecanique et de physique}, sed in quo tantum Leges motus ex Hugeniano principio (navis scilicet) deduxit\grqq\
%
(Brief vom 25.\ März 1704, Basel \textit{Universitätsbibl}.\ L~Ia 19 Bl.~221\textendash222).
%
In seinem Schreiben vom 2.\ Mai 1704 vermutet er bei Parent sogar ein Plagiat der
%
\protect\index{Namensregister}{\textso{Huygens} (Hugenius, Ugenius, Hugens, Huguens), Christiaan 1629\textendash1695}Huygens'schen Ergebnisse, die zwar um 1700 noch nicht veröffentlicht vorlagen, 
%
über die aber der Autor in der \textit{Académie des Sciences} vorgetragen hatte:
%
\glqq...librum Dni Parent Academiae Regiae Gallicae Socii, inscriptum \textit{Elemens de Mathematique et de Physique}, Paris 1700 editum ubi tamen non nisi de motus Legibus agit, ex Hugeniano plane principio, quod suspicor ipsi innotuisse ex iis quae Hugenius forte Parisiis olim communicavit Academiae, et Parentius reperire potuit in Academiae schedis aut ex relatu habere\grqq\ 
%
(Basel \textit{Universitätsbibl}.\ L~Ia 19 Bl.~223\textendash224; beide Briefe erscheinen in \textit{LSB} III, 9).
% 
% 
%
\pend
%
\end{ledgroup}