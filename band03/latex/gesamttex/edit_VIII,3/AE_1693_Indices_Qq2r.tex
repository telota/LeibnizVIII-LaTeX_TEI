%   % !TEX root = ../../VIII,3_Rahmen-TeX_8-1.tex
%
%
%   Band VIII, 3 N.~??A19.2/Y.10
%   Signatur/Tex-Datei: AE_1693_Indices_Qq2r
%   RK-Nr. 61051
%   Überschrift: [Corrigenda et annotandum]
%   Datierung: [1693]
%   WZ: (keins)
%.  SZ: (keins)
%.  Bilddateien (PDF): (keine)
%
%
\selectlanguage{ngerman}%
\frenchspacing%
%
\begin{ledgroupsized}[r]{120mm}
\footnotesize
\pstart
\noindent
\textbf{Überlieferung:}
\pend
\end{ledgroupsized}
%
\begin{ledgroupsized}[r]{114mm}
\footnotesize%
\pstart%
\parindent -6mm
\makebox[6mm][l]{\textit{E}}%
Druck:
\cite{01247}\title{Indices generales auctorum et rerum primi Actorum eruditorum quae Lipsiae publicantur decennii},
Leipzig 1693, Bd.~I, S.~*Qq2~r\textsuperscript{o}. % \lbrack= S.~584\rbrack.
Drei anonyme Berichtigungen und eine anonyme Anmerkung, die sich auf N~14\textsubscript{6},~\textit{E\textsuperscript{1}} beziehen;
sämtliche Einträge gehen höchstwahrscheinlich auf Leibniz selbst zurück.
Das Heft der \textit{Acta eruditorum} vom Juli 1684 liegt in verschiedenen Druckfassungen vor, von denen einige nicht alle in N.~14\textsubscript{10} verbesserten Druckfehler aufweisen.
\pend%
\end{ledgroupsized}
%
\selectlanguage{latin}%
\frenchspacing%
%
%
\count\Bfootins=1000
\count\Afootins=1000
\count\Cfootins=1000
%
%
\vspace*{4mm}
\pstart%
\normalsize%
\noindent%
\lbrack S.~*Qq2~r\textsuperscript{o}\rbrack\
\pend%
% Überschrift
\pstart%
\centering%
Corrigenda\protect\index{Sachverzeichnis}{corrigendum}
in \protect\index{Sachverzeichnis}{schediasma}Schediasmatibus\textso{ Leibnitianis,}\\
quae \cite{01023}\textit{Actis Eruditorum Lipsiensibus} sunt inserta
\pend%
% \newpage%
\vspace*{0.5em}%
%\textit{}
\pstart%
\noindent%
% \lbrack...\rbrack\
Anno 84,
% \edtext{}{\lemma{Anno 84}\Cfootnote{%
% Voraus gehen \textit{Corrigenda} zu \cite{01023}\textit{AE}, Februar 1682; Oktober 1683.}}
% \pend%
% \vspace*{0.5em}%
%
% \pstart%
% \noindent%
mense Julio,\edlabel{AE_1693_Indices_Qq2r_corr-1}
\edtext{p.~322,\edlabel{AE_1684_322,1_corrige-3} lin.~1,
pro\textso{ fibris,} \protect\index{Sachverzeichnis}{fibra}ponatur\textso{ fibra,}\edlabel{AE_1684_322,1_corrige-4}}{%
\lemma{p.~322 \lbrack...\rbrack\ \textso{fibra}\,}\Cfootnote{%
Siehe N.~14\textsubscript{6}, S.~\refpassage{AE_1684_322,1_corrige-1}{AE_1684_322,1_corrige-2} (\textit{E\textsuperscript{1}}).
Hierauf beziehen sich auch Leibnizens Verbesserungen in einem ver\-worfenen Konzept seines
wohl in der ersten Oktoberhälfte 1684 verfassten Briefes für die Herausgeber der \textit{AE}; % \textit{Acta eruditorum};
vgl. \textit{LSB} III,~4 N.~72, S.~181.26\textendash27\cite{01257}.}}
%
et
\edtext{p.~325,\edlabel{AE_1684_325,19_corrige-3} lin.~19,
pro\textso{ liberationis,} \protect\index{Sachverzeichnis}{libratio}ponatur\textso{ librationis,}\edlabel{AE_1684_325,19_corrige-4}}{%
\lemma{p.~325 \lbrack...\rbrack\ \textso{librationis}\,}\Cfootnote{%
Siehe N.~14\textsubscript{6}, S.~\refpassage{AE_1684_325,19_corrige-1}{AE_1684_325,19_corrige-2} (\textit{E\textsuperscript{1}}).
% Der Text von \textit{E}\textsuperscript{1} ist dort nicht fehlerhaft.
Hierauf beziehen sich auch Berichtigungen der Herausgeber in \textit{AE}, September 1684, S.~438\cite{01023}
sowie Leibnizens Verbesserungen in \textit{LSB} III,~4 N.~72, S.~181.28\textendash32\cite{01257}.}}
%
\edtext{lin.~22, pro\textso{ conveides,}\protect\index{Sachverzeichnis}{conoeides} ponatur\textso{ Conoeides.}}{%
\lemma{lin.~22 \lbrack...\rbrack\ \textso{Conoeides}\,}\Cfootnote{%
(Es geht eigentlich um Z.~23.)
Siehe N.~14\textsubscript{6}, S.~\refpassage{AE_1684_325,22_corrige-1}{AE_1684_325,22_corrige-2} (\textit{E\textsuperscript{1}}).
% Der Text von \textit{E}\textsuperscript{1} ist dort  nicht fehlerhaft.
Hierauf beziehen sich auch Berichtigungen der Herausgeber in \textit{AE}, September 1684, S.~438\cite{01023}
sowie Leibnizens Verbesserungen in \textit{LSB} III,~4 N.~72, S.~181.32\textendash33\cite{01257}.}}%
\edlabel{AE_1693_Indices_Qq2r_corr-2}
%
\pend%
%
\pstart%
Annotandum\edlabel{AE_1693_Indices_Qq2r_annotandum-1}%
\edtext{}{%
{\xxref{AE_1693_Indices_Qq2r_annotandum-1}{AE_1693_Indices_Qq2r_annotandum-2}}%
{\lemma{Annotandum \lbrack...\rbrack\ evellente}\Cfootnote{%
Ähnlich äußert sich Leibniz in seinen Briefen
an Jacob Bernoulli vom 24. September (4. Oktober) 1690 und
an R.\,C. von Bodenhausen vom 26. Oktober (5. November) 1690;
\textit{LSB} III,~4 N.~279, S.~574.14\textendash575.18;\cite{01260}
N.~285, S.~628.1\textendash6.\cite{01261}}}}
%
est autem illic,
licet dubitaretur de
\edtext{hypothesi,\protect\index{Sachverzeichnis}{hypothesis}
quod extensiones\protect\index{Sachverzeichnis}{extensio}
sint viribus tendentibus\protect\index{Sachverzeichnis}{vis tendens} proportionales,}{%
\lemma{hypothesi \lbrack...\rbrack\ proportionales}\Cfootnote{%
Siehe N.~14\textsubscript{6},
S.~\refpassage{AE_1684_319-325_a1}{AE_1684_319-325_a2} (\textit{E\textsuperscript{1}}).}}
% sowie N.~??Y\textsubscript{7}.
%
manere tamen verum quod diximus
\edtext{fig.~5}{%
\lemma{fig.~5}\Cfootnote{%
Das Diagramm \lbrack\textit{Fig.~5e}\rbrack\ in N.~14\textsubscript{6}, S.~\pageref{LH_37_03_072v+AE_1684_323_Fig.5e} (\textit{E\textsuperscript{1}}).}}
%
\edtext{resistentiam\protect\index{Sachverzeichnis}{resistentia trabis} in \textit{FG} esse ad resistentiam in \textit{BA},
ut quadratum \textit{FG} ad quadratum \textit{BA};\protect\index{Sachverzeichnis}{quadratum}}{%
\lemma{resistentiam \lbrack...\rbrack\ quadratum \textit{BA}}\Cfootnote{%
Siehe N.~14\textsubscript{6}, S.~\refpassage{AE_1684_319-325_a3}{AE_1684_319-325_a4} (\textit{E\textsuperscript{1}}).}}
%
quia quaecunque sint figurae \textit{BAEB}, et \textit{FGHF}\protect\index{Sachverzeichnis}{figura}
(\protect\vphantom)%
quae ex dicta hypothesi\protect\index{Sachverzeichnis}{hypothesis}
trilineae parabolicae\protect\index{Sachverzeichnis}{trilinea parabolica} fiunt%
\protect\vphantom()
quia tamen sunt similes,
utique sunt ut quadrata circumscripta,\protect\index{Sachverzeichnis}{quadratum circumscriptum}
seu ab \textit{AB}, \textit{FG}.
Unde etsi mutaretur hypothesis,\protect\index{Sachverzeichnis}{hypothesis}
nihil tamen esset mutandum in dictis,
nisi circa comparationem potentiae transverse abrumpentis,\protect\index{Sachverzeichnis}{potentia abrumpens}
cum directe evellente.%
\edlabel{AE_1693_Indices_Qq2r_annotandum-2}%
\protect\index{Sachverzeichnis}{potentia evellens}
% \lbrack S.~*Qq2~v\textsuperscript{o}\rbrack\
% \pend%
% \vspace*{0.5em}%
%
% \pstart%
% \noindent%
% \edtext{\lbrack...\rbrack}{\lemma{evellente}\Cfootnote{%
% Es folgen \textit{Corrigenda} zu \cite{01023}\textit{AE}, Oktober 1684.}}
\pend%
\count\Bfootins=1200
\count\Afootins=1200
\count\Cfootins=1200
%
% ENDE DES STÜCKES