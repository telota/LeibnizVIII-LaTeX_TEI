\thispagestyle{empty}
\vspace{3.0ex}
\begin{center}\uppercase{\normalsize Konkordanzen}\end{center}
\footnotesize
%
\vspace{3.0ex}
\textsc{Editionen}\\
\vspace{-3mm}
\setlength{\columnseprule}{0.4pt}
\renewcommand*{\chapter}{\OrigChapter}
\setlength\LTleft{\fill} \setlength\LTright{\fill}
\begin{longtable}{ll}
\footnotesize
%\textit{n.n.}
%\textit{Journal général de l'Instruction publique et des cultes} XXVI (1857), Nr. 32, S. 235f. & N. 81\\%% = ex N. 93 = RK ?????.
\textsc{Fichant} 1994, S.~71\textendash79\cite{01056} & N.~\ref{dcc_01}\\
\textsc{Fichant} 1994, S.~80\textendash88\cite{01056} & N.~\ref{dcc_02-1}\\
\textsc{Fichant} 1994, S.~89\textendash92\cite{01056} & N.~\ref{dcc_02-2}\\
\textsc{Fichant} 1994, S.~93\textendash99\cite{01056} & N.~\ref{dcc_03}\\
\textsc{Fichant} 1994, S.~100\textendash105\cite{01056} & N.~\ref{dcc_04}\\
\textsc{Fichant} 1994, S.~106\textendash115\cite{01056} & N.~\ref{dcc_05}\\
\textsc{Fichant} 1994, S.~116\textendash124\cite{01056} & N.~\ref{dcc_06-1}\\
\textsc{Fichant} 1994, S.~125\textendash144\cite{01056} & N.~\ref{dcc_06-2}\\
\textsc{Fichant} 1994, S.~145\textendash151\cite{01056} & N.~\ref{dcc_07}\\
\textsc{Fichant} 1994, S.~152\textendash158\cite{01056} & N.~\ref{dcc_08}\\
\textsc{Fichant} 1994, S.~159\textendash165\cite{01056} & N.~\ref{dcc_09}\\
\textsc{Fichant} 1994, S.~166\textendash171\cite{01056} & N.~\ref{dcc_10}\\
\textsc{Fichant} 1994, S.~346\textendash352\cite{01056} & N.~\ref{RK57266-1}\\
\textsc{Fichant} 1994, S.~352\cite{01056} & N.~\ref{RK57266-2}\\
\textsc{Fichant} 1994, S.~356f.\cite{01056} & N.~\ref{RK57267-1}\\
\textsc{Fichant} 1994, S.~357\textendash361\cite{01056} & N.~\ref{RK57267-2}\\
\textsc{Fichant} 1994, S.~361\textendash364\cite{01056} & N.~\ref{RK57267-3}\\
\textsc{Fichant} 1994, S.~365\textendash367\cite{01056} & N.~\ref{RK57268}\\
\textsc{Fichant} 1994, S.~375\textendash378\cite{01056} & N.~\ref{RK57269}\\
\textsc{Fichant} 1994, S.~379\textendash383\cite{01056} & N.~\ref{RK57270}\\
\textsc{Fichant} 1994, S.~384\textendash387\cite{01056} & N.~\ref{RK57271}\\
\textsc{Fichant} 1994, S.~387f.\cite{01056} & N.~\ref{RK57272}\\
\textsc{Fichant} 1994, S.~390f.\cite{01056} & N.~\ref{RK57273}\\
\textsc{Fichant} 1994, S.~391\textendash393\cite{01056} & N.~\ref{RK57274}\\
\textsc{Fichant} 1994, S.~394\cite{01056} & N.~\ref{RK57275}\\
\textsc{Fichant} 1994, S.~397f.\cite{01056} & N.~\ref{RK57276}\\
\textsc{Fichant} 1994, S.~399\textendash402\cite{01056} & N.~\ref{RK57277}\\
\textsc{Fichant} 1994, S.~403\textendash405\cite{01056} & N.~\ref{RK52278}\\
\textsc{Fichant} 1994, S.~406\textendash408\cite{01056} & N.~\ref{RK57279}\\
%
\textsc{Gerland} 1906, S.~11\textendash15\cite{00197} & N.~\ref{cnds_1}\\
\textsc{Gerland} 1906, S.~16\textendash27\cite{00197} & N.~\ref{cnds_3}\\
\textsc{Gerland} 1906, S.~27\textendash31\cite{00197} & N.~\ref{cnds_2}\\
\textsc{Gerland} 1906, S.~31\textendash35\cite{00197} & N.~\ref{38538}\\
\textsc{Gerland} 1906, S.~35\cite{00197} & N.~\ref{38540}\\
\textsc{Gerland} 1906, S.~175\cite{00197} & N.~\ref{55749}\\
%
\end{longtable}
\vspace{3.0ex}
\noindent
%\vspace{3.0ex}
\newpage
\noindent
\textsc{Nachdrucke}\\
\vspace{-3mm}
\setlength{\columnseprule}{0.4pt}
\renewcommand*{\chapter}{\OrigChapter}
\setlength\LTleft{\fill} \setlength\LTright{\fill}
\begin{longtable}{ll}
\footnotesize
%\textit{n.n.}
%\textit{Journal général de l'Instruction publique et des cultes} XXVI (1857), Nr. 32, S. 235f. & N. 81\\%% = ex N. 93 = RK ?????.
\textsc{Jacob Bernoulli}, \title{Der Briefwechsel}, hrsg. von D.~\textsc{Speiser} und A.~\textsc{Weil}, & \\
Basel 1993, S.~51\textendash57\cite{01044} & N.~\ref{ddrs_06}\\
%
\textsc{Lamarra/Palaia} 2005, S.~59\textendash66\cite{01289} & N.~\ref{ddrs_06}\\
%
\textit{LOD} III, S.~161\textendash166, Tab.~V (Fig. 21\textendash28)\cite{00150} & N.~\ref{ddrs_06}\\
%
\textit{LMG} VI, S.~106\textendash112, Faltblatt (Fig.~1\textendash8)\cite{01043} & N.~\ref{ddrs_06}\\
%
\end{longtable}
\vspace{3.0ex}
\noindent
\vspace{3.0ex}
\textsc{Übersetzungen ins Französische}\\
\vspace{-3mm}
\setlength{\columnseprule}{0.4pt}
\renewcommand*{\chapter}{\OrigChapter}
\setlength\LTleft{\fill} \setlength\LTright{\fill}
\begin{longtable}{ll}
\footnotesize
%\textit{n.n.}
%\textit{Journal général de l'Instruction publique et des cultes} XXVI (1857), Nr. 32, S. 235f. & N. 81\\%% = ex N. 93 = RK ?????.
\textsc{Fichant} 1994, S.~185\textendash200\cite{01056} & N.~\ref{dcc_01}\\
\textsc{Fichant} 1994, S.~200\textendash207\cite{01056} & N.~\ref{dcc_02-1}\\
\textsc{Fichant} 1994, S.~208\textendash212\cite{01056} & N.~\ref{dcc_02-2}\\
\textsc{Fichant} 1994, S.~213\textendash223\cite{01056} & N.~\ref{dcc_03}\\
\textsc{Fichant} 1994, S.~223\textendash228\cite{01056} & N.~\ref{dcc_04}\\
\textsc{Fichant} 1994, S.~229\textendash236\cite{01056} & N.~\ref{dcc_05}\\
\textsc{Fichant} 1994, S.~236\textendash255\cite{01056} & N.~\ref{dcc_06-1}\\
\textsc{Fichant} 1994, S.~257\textendash277\cite{01056} & N.~\ref{dcc_06-2}\\
\textsc{Fichant} 1994, S.~278\textendash302\cite{01056} & N.~\ref{dcc_07}\\
\textsc{Fichant} 1994, S.~308\textendash316\cite{01056} & N.~\ref{dcc_08}\\
\textsc{Fichant} 1994, S.~317\textendash330\cite{01056} & N.~\ref{dcc_09}\\
\textsc{Fichant} 1994, S.~331\textendash337\cite{01056} & N.~\ref{dcc_10}\\
%
G.\,W. \textsc{Leibniz}, \textit{{\OE}uvre concernant la physique}, hrsg. von P.~\textsc{Peyroux}, Paris 1985, S.~15\textendash20\cite{01248} & N.~\ref{ddrs_06}\\
%
\end{longtable}
%%%%
%%%% Konkordanzen
%%%%
%% \begin{center} \uppercase{Konkordanzen}\end{center}
%In den folgenden zwei Registern sind die im Band edierten St\"{u}cke verzeichnet
%gemäß ihrer Nummerierung im \textit{Kritischen Katalog 1} (KK~1) bzw. im \textit{Catalogue critique~2} (Cc~2),
%falls diese besteht.
%Zu jeder ver\-zeich\-ne\-ten KK~1- bzw. Cc~2-Nummer wird angegeben,
%welchen Stücknummern sie im Band entspricht.
%Die zusätzliche Angabe \glqq tlw.\grqq\ weist gegebenenfalls darauf hin,
%dass unter der betreffenden Katalognummer mehr als nur ein St\"{u}ck erfasst ist.
%Welchen Kata\-log\-nummern (nach KK~1 bzw. Cc~2) die im Band edierten Stücke gegebenenfalls entsprechen,
%ist jeweils im Kopf der einzelnen Stücke vermerkt.
%\newline\indent%
%Die weder im KK~1 noch im Cc~2 erfassten Stücke aus dem Band
%sind unten in einem dritten Register verzeichnet.
%Zu dieser Gruppe gehören sämtliche Anstreichungen und Anmerkungen in Hand\-exemplaren.
%%
%\\[4.0ex]
%%[1.0ex]
%%
%%%%%%%%%%%%%%%%%% Konkordanz KK1 - N.
%%
%\begin{center}\footnotesize{\uppercase{KK 1-Konkordanz}}\end{center}
%
%\vspace{2.0ex}
%\noindent
%\begin{tabular}{p{2.9cm}|p{2.9cm}|p{2.9cm}|p{2.9cm}}
%\begin{tabular}{ll}
%185 & N. 85\\%%
%194 A & N. 85\\%%
%194 B & N. 84%%
%\end{tabular}
%&
%\begin{tabular}{ll}
%194 C & N. 83\\%%
%194 D & N. 87\\%%
%194 E & N. 86%%
%\end{tabular}
%&
%\begin{tabular}{ll}
%975 & N. 70\\%%
%976 & N. 69\\%%
%977 & N. 71%%
%\end{tabular}
%&
%\begin{tabular}{ll}
%% 978 & N. ***\\%% N. *78.
%979 & N. 68\\
%\hphantom{xxxx} & \hphantom{N. xx}\\%%
%\hphantom{xxxx} & \hphantom{N. xx}%% = ex N. 83.
%\end{tabular}
%\end{tabular}
%% \vspace{2.0ex}
%%\clearpage% Rein provisorisch.
%%\noindent%
%%\\
%%%%%%%%
%%
%%%%%%%%%%%%%%%%%% Konkordanz Cc2 - N.
%%
%% \vspace{1.0ex}% PR: Rein provisorisch!
%%
%%%%%%%%%%%%%%%%%%%%%%%%% Konkordanz Cc2 - N.
%%
%\vspace{6.0ex}
%\begin{center}\footnotesize{\uppercase{Cc 2-Konkordanz}}\end{center}
%
%\vspace{2.0ex}
%\noindent
%\begin{tabular}{p{2.9cm}|p{2.9cm}|p{2.9cm}|p{2.9cm}}
%\begin{tabular}{ll}
%% 340 & N. ***\\%%
%423 & N. 7\\%%
%430 & N. 73\\%%
%480 A\textendash B & N. 48\\%%
%482 & N. 49\\%%
%485 & N. 4\\%%
%502 & N. 1\\%%
%508 & N. 78\\%%
%509 & N. 53\\%%
%529 & N. 61\\%%
%541 & N. 41\\%%
%543 tlw. & N. 16\\%%
%835 & N. 10\\%%
%836 & N. 92\\%%
%837 & N. 93\\%%
%838 & N. 94\\%%
%869 & N. 72\\%%
%897 tlw. & N. 95\\%%
%914 & N. 89\\%%
%921 & N. 2\\%%
%939 & N. 11\\%%
%941 A & N. 8\\%%
%941 B & N. 9%%
%\end{tabular}
%&
%\begin{tabular}{ll}
%942 A-B & N. 50\\%%
%943 & N. 51\\%%
%944 & N. 31\textsubscript{1}\\%%
%945 A & N. 30\\%%
%945 B & N. 31\textsubscript{2}\\%%
%945 C & N. 32\\%%
%945 E & N. 33\\%%
%946 & N. 31\textsubscript{3}\\%%
%947 & N. 35\\%%
%948 & N. 38\\%%
%964 B & N. 52\\%%
%965 A\textendash C & N. 34\textsubscript{1}\\%%
%965 D & N. 34\textsubscript{2}\\%%
%965 E & N. 34\textsubscript{3}\\%%
%965 F & N. 34\textsubscript{4}\\%%
%965 G\textendash H & N. 34\textsubscript{1}\\%%
%965 J & N. 34\textsubscript{3}\\%%
%965 K & N. 34\textsubscript{2}\\%%
%965 L & N. 37\\%%
%967 A & N. 26\\%%
%967 B & N. 21\\%%
%968 A & N. 25%%
%\end{tabular}
%&
%\begin{tabular}{ll}
%968 B & N. 23\\%%
%968 C & N. 24\\%%
%968 D & N. 22\\%%
%969 & N. 15\\%%
%971 A & N. 19\\%%
%971 B & N. 20\\%%
%972 & N. 42\\%%
%973 tlw. & N. 43\\%%
%973 tlw. & N. 88\\%%
%974 & N. 18\\%%
%975 A & N. 17\textsubscript{1}\\%%
%975 B & N. 17\textsubscript{2}\\%%
%976 & N. 14\\%%
%1054 tlw. & N. 63\\%%
%1054 tlw. & N. 64\\%%
%1133 A & N. 97\textsubscript{1}\\%%
%1133 B & N. 97\textsubscript{3}\\%%
%1141 & N. 97\textsubscript{4}\\%%
%1142 & N. 97\textsubscript{2}\\%%
%1179 & N. 65\\%%
%1187 & N. 98\\%%
%1189 A & N. 36\textsubscript{1}%%
%\end{tabular}
%&
%\begin{tabular}{ll}
%1189 B & N. 36\textsubscript{2}\\%%
%1189 C & N. 36\textsubscript{1}\\%%
%1189 D\textendash G & N. 36\textsubscript{2}\\%%
%1190 A & N. 28\textsubscript{4}\\%%
%1190 B & N. 28\textsubscript{3}\\%%
%1190 C & N. 28\textsubscript{2}\\%%
%1190 D & N. 28\textsubscript{1}\\%%
%1191 tlw. & N. 5\\%%
%1191 tlw. & N. 28\textsubscript{5}\\%%
%1192 A\textendash B & N. 28\textsubscript{6}\\%%
%1192 C & N. 28\textsubscript{7}\\%%
%1213 A & N. 45\textsubscript{3}\\%%
%1213 B & N. 45\textsubscript{1}\\%%
%1213 C & N. 45\textsubscript{4}\\%%
%1213 D & N. 45\textsubscript{2}\\%%
%1271 & N. 75\\%%
%1322 A\textendash C & N. 58\\%%
%1322 D tlw. & N. 6\\%%
%1322 D tlw. & N. 58\\%%
%1322 E & N. 58\\%%
%1323 A tlw. & N. 76\\%%
%1323 A tlw. & N. 82%%
%\end{tabular}
%\end{tabular}
%%
%%
%%
%\clearpage
%\noindent
%\begin{tabular}{p{2.9cm}|p{2.9cm}|p{2.9cm}|p{2.9cm}}
%\begin{tabular}{ll}
%1323 B & N. 76\\%%
%1324 & N. 54\\%%
%1366 A & N. 59%% = Extraits de lettres de Mons. Boccone
%\end{tabular}
%&
%\begin{tabular}{ll}
%1366 B & N. 60\\%% = Notizen zur Botanik
%1367 & N. 56\\%%
%1503 & N. 12%%
%\end{tabular}
%&
%\begin{tabular}{ll}
%1504 & N. 29\\%%
%1516 & N. 99\\%
%% 1562 & N. ***\\%% = ex N. 77.
%1563 & N. 74
%\end{tabular}
%\end{tabular}
%\noindent%
%\\
%%%%%%%%
%%
%%%%%%%%%%%%%%%%%% WEDER KK 1 NOCH CC 2
%%
%
%\vspace{4.0ex}% PR: Rein provisorisch!
%\begin{center}\footnotesize{\uppercase{Weder im KK 1 noch im Cc 2 erfasste Stücke}}\end{center}
%
%\vspace{2.0ex}
%\noindent\begin{tabular}{p{2.9cm}|p{2.9cm}|p{2.9cm}|p{2.9cm}}
%\begin{tabular}{ll}
%N. & 3\hphantom{xxxxxxxxxi}\\%%
%N. & 13\\%%
%N. & 27\\%%
%N. & 34\textsubscript{5}\\%%
%N. & 39\\%%
%N. & 40%%
%\end{tabular}
%&
%\begin{tabular}{ll}
%N. & 44\hphantom{xxxxxxxxxi}\\%%
%N. & 46\\%%
%N. & 47\\%%
%N. & 55\\%%
%N. & 57%%
%\end{tabular}
%&
%\begin{tabular}{ll}
%N. & 62\hphantom{xxxxxxxxxi}\\%%
%N. & 66\\%%
%N. & 67\\%%
%N. & 77\\%%
%N. & 79%%
%\end{tabular}
%&
%\begin{tabular}{ll}
%N. & 80\hphantom{xxxxxxxxxi}\\%%
%N. & 81\\%%
%N. & 90\\%%
%N. & 91\\%%
%N. & 96%%
%\end{tabular}
%\end{tabular}
%
%\vspace{1.0ex}% PR: Rein provisorisch!
%\noindent
%Davon sind N. 3, 13, 44, 46, 47 und 67 Anstreichungen und Anmerkungen in Handexemplaren.
%%%
%%%
%%%
%\clearpage
%%%
%%% 
%%% Hier enden die Konkordanzen.