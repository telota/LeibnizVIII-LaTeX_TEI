\thispagestyle{empty}
\vspace{3.0ex}
\begin{center}\uppercase{\normalsize Berichtigungen und Ergänzungen}\end{center}
\vspace{4.0ex}%%
%%
%% BERICHTIGUNGEN
%%
%%
% \vspace{4.0em}% PR: Rein provisorisch!
\renewcommand*{\chapter}{\OrigChapter}
%%
%%
\noindent\footnotesize
%%
\vspace{3.0ex}% PR: Rein provisorisch!
%%
\noindent%
Zu Band VIII, 1:
\vspace{-3.0ex}%\\% [1.0ex]
%%
\setlength\LTleft{0pt}\setlength\LTright{0pt}
\begin{longtable}{p{36mm}p{92mm}}
\footnotesize
S.~229, Anmerkung %zu \lbrack\textit{Fig.~1}\rbrack 
& \textit{statt} ergo dupl.\ \textit{CP} dupl.\ \textit{CS}. \textit{CT}. \textit{SO} \textit{lies} erunt dupl.\ \textit{CO} dupl.\ \textit{CS}. \textit{CT}. \lbrack\textit{ST}\rbrack\ \textit{u.\ ergänze dazu im Apparat} \textit{SO} \textit{L ändert Hrsg.}\\
%
S.~230, Anmerkung %zu \lbrack\textit{Fig.~1}\rbrack 
& \textit{statt} $c:M + x \squaredots y:x$ ergo \textit{y} aeq.~$cx:M + x$. Si ponatur \textit{c} et \textit{M} \textit{lies} $c:m + x \squaredots y:x$ ergo \textit{y} aeq.~$cx:m + x$. Si ponantur \textit{c} et \textit{m}\\%
%
S.~265, Z.~5 & \textit{statt} Jam \textit{lies} Tam\\%
S.~273, Z.~2 & \textit{statt} Petronis \textit{lies} Petronii\\%
S.~341, Z.~17 & \textit{statt} festinandam \textit{lies} festinandum\\%
\end{longtable}
%%
\vspace{2.0ex}% PR: Rein provisorisch!
%%
%\newpage
\noindent%
Zu Band VIII, 2:
%\\% [1.0ex]
%%
\setlength\LTleft{0pt}\setlength\LTright{0pt}
%\begin{longtable}{lp{70mm}}
\begin{longtable}{p{36mm}p{92mm}}
\footnotesize
S.~134, Erläuterung zu Z.~1 & \textit{statt} \textit{De aequiponderantibus.} \textit{lies} \textit{De aequiponderantibus}, \textit{De corporibus fluitantibus.}\\%
S.~135, Z.~18 & \textit{statt} sumtorum ad rem \textit{lies} sumtorum. Ad rem\\%
S.~136, Z.~18 & \textit{statt} eijus \textit{lies} ejus\\%
S.~185, Z.~19 & \textit{statt} \textit{de} \textit{lies} \lbrack\textit{ce}\rbrack\ \textit{u.\ ergänze dazu im Apparat} \textit{de} \textit{L ändert Hrsg.}\\%
S.~189, Z.~19 & \textit{statt} hunc \textit{lies} \lbrack hanc\rbrack\ \textit{u.\ ergänze dazu im Apparat} hunc \textit{L ändert Hrsg.}\\%
S.~275, Z.~25 & \textit{statt} infinita \textit{lies} insumta\\%
S.~291, Z.~12 & \textit{statt} \lbrack il est\rbrack\ \textit{lies} ce \lbrack qui est]\\%
S.~296, Z.~4 & \textit{streiche} reciproce\\%
S.~296, Z.~8 & \textit{statt} proportionis \textit{lies jeweils} progressionis\\%
S.~298, Z.~7 & \textit{statt} $\displaystyle\frac{d}{a}gtc$ \textit{lies} $\displaystyle\frac{d}{a^2}gtc$\\%
\hangindent=9,5mm S.~313, Erläuterung zu Z.~1 \textit{GEA} & \textit{statt} \textit{M} \textit{lies jeweils} \textit{B}\\%
S.~342, Z.~27 & \textit{statt} 285 \textit{lies} 258\\%
\hangindent=10mm S.~355, Erläuterung zu Z.~8 \textso{I. fig.} & \textit{statt} \lbrack\textit{Fig.~2}\rbrack\ \textit{lies} \lbrack\textit{Fig.~1}\rbrack\ in N. 36\textsubscript{1}, S.~349\\%
\hangindent=9,5mm S.~355, Erläuterung zu Z.~13 Triangle \textit{GEA} & \textit{statt} \lbrack\textit{Fig.~2}\rbrack\ \textit{lies} \lbrack\textit{Fig.~1}\rbrack\ in N. 36\textsubscript{1}, S.~349\\%
S.~366, Datierung von N.~38 & \textit{statt} Ende 1675 \textit{lies} April bis Mai 1675\\%
S.~366, Z.~5\textendash9 & \textit{statt} Im \lbrack...\rbrack\ sein. \textit{lies} Der Inhalt dieses Résumé N.~38 spiegelt den Stand der Untersuchung nach N.~31 und N.~32 wider. Somit ist die Angabe in Z.~12 \glqq j'y ay travaillé depuis quelques jours\grqq\ vermutlich auf die genannten Texte zu beziehen. Die sich daraus ergebende Datierung ist April bis Mai 1675.\\%
S.~423, Z.~15 & \textit{statt} elles \textit{lies} ils\\%
S.~424, Z.~10 & \textit{statt} on arrestera \textit{lies} on \lbrack n'\rbrack arrestera \textit{u.\ ergänze dazu im Apparat} n' \textit{erg. Hrsg.}\\%
S.~430, Z.~22 & \textit{statt} remis \textit{lies} \lbrack remissus\rbrack\ \textit{u.\ ergänze dazu im Apparat} remis \textit{L ändert Hrsg.}\\%
S.~442, Z.~9 & \textit{statt} resistentiis \textit{lies} resistentis\\%
S.~447, Z.~8 & \textit{statt} \textit{F.} \textit{lies} \textit{A.}\\%
S.~448, Z.~1 & \textit{statt} \textit{AF,} \textit{lies} \textit{A,}\\%
S.~470, Z.~3 & \textit{statt} duorum \textit{lies} durorum\\%
S.~471, Z.~4 & \textit{statt} praetera \textit{lies} praeterea\\%
S.~479, Z.~8 & \textit{statt} ferat \textit{lies} ferat.\\%
S.~482, Z.~6 & \textit{statt} vacuo \textit{lies} vacuo,\\%
S.~494, Z.~4 & \textit{statt} hoc \textit{lies} hoc,\\%
S.~686, Z.~23\textendash25 & \textit{ergänze Erl.\ zu} Boylius habet \lbrack...\rbrack\ in Iezzo: Möglicherweise handelt es sich um eine Bemerkung zu \textsc{D.~Rembrantsz van Nierop}: \textit{Tweede deel van enige oefeningen in de geografie}, Amsterdam 1674. Eine ausführliche Rezension erschien in den \textit{Philosophical Transactions}: 
\glqq A narrative of some observations made upon several voyages \lbrack...\rbrack\ Together with instructions given by the Dutch East-India Company for the discovery of the famous land of \textit{Jesso} near \textit{Japan}\grqq, \textit{PT}, Bd~9 N.~109, 14.\ (24.)~Dezember 1674, S.~197\textendash208.\\
S.~687, Z.~27 & \textit{statt} de Verret \textit{lies} du Verney \textit{u.\ ergänze Erl.\ dazu} \textsc{C.~Perrault}, \textit{Essais de physique}, Bd~3, Paris 1680, S.~5 (Duverney) und S.~226 (\glqq fibre du coeur \lbrack...\rbrack\ en spirale\grqq); sowie \textsc{A.~Guerrini}, \textit{The Courtier’s Anatomists}, Chicago/London, 2015, S.~12 und S.~171.\\% 
S.~687, Z.~29 & \textit{ergänze Erl.\ zu} un Anglois: \textsc{W.~Cole} \glqq A discourse concerning the \textit{spiral}, instead of the supposed \textit{annular}, structures of the fibres of the intestins\grqq, \textit{PT}, Bd~11 N.~125, 22.~Mai (1.~Juni) 1676, S.~603\textendash609.\\%
S.~699, Z.~7 & \textit{ergänze} Weitere Drucke nach \textit{E\textsuperscript}:\enskip 
\textsc{M.~Petzet}, \textit{Claude Perrault und die Architektur des Sonnenkönigs}, München 2000, S.~568f.;
\textsc{H.~Bredekamp}, \textit{Die Fenster der Monade}, Berlin 2004, S.~210\textendash214.\\%
\end{longtable}
%
%
%