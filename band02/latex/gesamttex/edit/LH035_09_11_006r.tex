\count\Bfootins=1000
\edtext{Mais [il est]}{\lemma{Mais}\Bfootnote{\textit{(1)}\ il est \textit{(2)}\ ce \textit{L ändert Hrsg.}}}
pouss\'{e} en \edtext{consequence comme l'eau est m\^{u}e par un corps qui se meut dans elle; et ce mouuement communiqu\'{e} au medium se doit estimer}{\lemma{consequence}\Bfootnote{\textit{(1)}\ ; et il faut estimer \textit{(2)}\ comme [...] elle; et  \textit{(a)}\ cet  \textit{(b)}\ ce mouuement [...] estimer \textit{ L}}}
par la quantit\'{e} de la surface de celuy qui y est m\^{u}. Or un \edtext{corps plus solide}{\lemma{corps}\Bfootnote{\textbar\ plus \textit{erg.} \textbar\ solide \textit{L}}} \edtext{par dedans}{\lemma{par dedans}\Bfootnote{\textit{erg.} \textit{L}}}
a plus de surface, comme un bois plus \'{e}pais a plus d'\'{e}corce des arbres qu'il y a.
\pend
\pstart
D'o\`{u} vient que le mouuement se diminue par la quantit\'{e} de matiere\protect\index{Sachverzeichnis}{quantit\'{e} de mati\`{e}re} augment\'{e}e.
C'est \`{a} cause \edtext{de l\`{a}}{\lemma{de}\Bfootnote{\textit{(1)}\ la resti \textit{(2)}\ l\`{a} \textit{L}}}
que plus de matiere liquide est divis\'{e}e, et comme elle resiste \`{a} cette
\edtext{division; une m\^{e}me force doit faire autant d'effect, s\c{c}avoir de division}{\lemma{division;}\Bfootnote{%
\textit{(1)}\ elle %
\textit{(2)}\ une m\^{e}me force %
\textit{(a)}\ ne se %
\textit{(b)}\ auroit %
\textit{(c)}\ doit faire autant %
\textit{(aa)}\ de division %
\textit{(bb)}\ d'effect [...] division \textit{L}}}
qu'auparavant, et comme c'est plus de matiere qui est divis\'{e}e, il faut que le mouuement soit plus doux.
\pend
\pstart
\edtext{Quand un corps dur rencontre un corps mol le mouuement}{\lemma{Quand un corps}\Bfootnote{%
\textit{(1)}\ mol %
\textit{(2)}\ dur rencontre un corps mol %
\textit{(a)}\ l'effort se p %
\textit{(b)}\ le mouuement \textit{L}}}
ne se perd pas, mais il est dispers\'{e} par le nombre innumerable des petites parties qui \edtext{sont m\^{e}ues}{\lemma{sont}\Bfootnote{\textit{(1)}\ m\^{e}us \textit{(2)}\ m\^{e}ues \textit{L}}},
et d'o\`{u} vient qu'on croit qu'il se [perd]\edtext{}{\Bfootnote{pert\textit{\ L \"{a}ndert Hrsg.}}}.
\pend
\pstart
Un \edtext{grand poids}{\lemma{grand}\Bfootnote{\textit{(1)}\ corps \textit{(2)}\ poids \textit{L}}}
commence avec la m\^{e}me vitesse\protect\index{Sachverzeichnis}{vitesse} qu'un petit, car
[quoiqu']\edtext{}{\Bfootnote{quoqu'\textit{\ L \"{a}ndert Hrsg.}}}
il fasse plus de division; il y a aussi plus de
\edtext{force\protect\index{Sachverzeichnis}{force}.\\
\hspace*{7,5mm}Le trouble du mouuement general des corps liquides invisibles qui environnent les corps visibles, est la cause du retardement des mouuemens particuliers.}{\lemma{}\Bfootnote{force. \textit{(1)}\ Un grand corps estant \textit{(2)}\ Les corps estant m\^{u}s comme les balanciers\protect\index{Sachverzeichnis}{balancier} des monstres\protect\index{Sachverzeichnis}{montre}, ou les volans\protect\index{Sachverzeichnis}{volant}, ne  \textit{(a)}\ causent point d'autre trouble que celuy  \textit{(b)}\ sont retardez qu'\`{a} cause du trouble de la division de la matiere liquide dans laquelle ils sont m\^{e}us. \textit{(3)}\  Le trouble [...] particuliers. \textit{L}}}
\pend
%\newpage
\pstart
Les mouvemens particuliers visibles troublent bien moins que les mouuemens particuliers invisibles ou interieurs dans les corps par ce qu'il y a moins de matiere et moins de \edtext{vitesse.\\
\hspace*{7,5mm}Quand un balancier equilibr\'{e}}{\lemma{vitesse.}\Bfootnote{\textit{(1)}\ Quand une pendule\protect\index{Sachverzeichnis}{pendule} est agit\'{e}e, ou quand un balancier \textit{(2)}\ Quand [...] equilibr\'{e} \textit{L}}}
ou un \edtext{volant tournent ou vont}{\lemma{volant}\Bfootnote{\textit{(1)}\ vont \textit{(2)}\ tournent ou vont \textit{L}}}
et viennent, ce n'est que le mouuement visible qui trouble et qui est retard\'{e} par la
\edtext{division de la matiere liquide}{\lemma{division}\Bfootnote{\textit{(1)}\ -- le retardement vient de ce que les corps  \textit{(a)}\ meus  \textit{(b)}\ mols \textit{(2)}\ de la matiere liquide. \textit{L}}}\edtext{.\\\hspace*{7,5mm}La pesanteur}{\lemma{liquide.}\Bfootnote{\textit{(1)}\ Quand un corps agit \textit{(2)}\ La pesanteur \textit{L}}}
est l'effort de la matiere \edtext{commune, de chasser}{\lemma{commune,}\Bfootnote{\textit{(1)}\ pour \textit{(2)}\ de chasser \textit{L}}}
les corps dont les mouuements particuliers interieurs troublent le mouuement general.
\pend
\count\Bfootins=1000
\pstart
Un corps pesant qui \edtext{en commen\c{c}ant}{\lemma{en}\Bfootnote{\textit{(1)}\ descendent \textit{(2)}\ re \textit{(3)}\ commen\c{c}ant \textit{L}}}
\`{a} descendre, commence \`{a} lever un autre, va avec la m\^{e}me vitesse, dont il iroit sans cela, ne considerant
\edtext{pas le simple mouuement}{\lemma{pas}\Bfootnote{\textit{(1)}\ la vitesse \`{a} qui \textit{(2)}\ le simple mouuement, \textit{L}}},
qui sans la pesanteur, trouue quelque resistence, comme un balancier ou volant plus pesant. Mais mettant cela en ligne de
\edtext{conte, c'est}{\lemma{conte,}\Bfootnote{\textit{(1)}\ ce \textit{(2)}\ c'est \textit{L}}}
comme un balancier equilibr\'{e}, et par consequent le mouuement commence plus doucement: et la force de la difference des deux corps fait le m\^{e}me effect, comme si elle faisoit jouer avec son mouuement 
\edtext{particulier un balancier de la pesanteur de la somme}{\lemma{particulier}\Bfootnote{\textit{(1)}\ la somme \textit{(2)}\ un balancier [...] somme \textit{L}}}
des deux corps, et par \edtext{consequent si ce n'est pas dans le commencement mais dans la cheute}{\lemma{consequent}\Bfootnote{\textit{(1)}\ dans la cheute \textit{(2)}\ si ce [...] cheute \textit{L}}}
qu'il rencontre un autre corps, c'est comme si elle poussoit alors un tel balancier, et au commencement la force de la pesanteur en ce cas d'equilibre est bien comme celle de la grandeur de la difference des corps; mais elle est diminu\'{e}e par le balancier \`{a}
\edtext{traisner. Voyons pourtant: le mouuement general voulant}{\lemma{traisner.}\Bfootnote{\textit{(1)}\ \`{A} cause que le corps voulant \textit{(2)}\ Voyons pourtant:  \textit{(a)}\ le corp \textit{(b)}\ la r \textit{(c)}\ le mouuement general voulant \textit{L}}}
prevenir une division ou
\edtext{trouble, [en] produit}{\lemma{trouble,}\Bfootnote{\textit{(1)}\ en \textit{(2)}\ tro \textit{(3)}\   \textbar\ en \textit{erg.} \textit{Hrsg. } \textbar\ produit \textit{L}}}
un autre. Car en chassant les corps pesants il est oblig\'{e} de les remuer[,] par consequent il faut oster les divisions qu'il cause de ceux qu'il evite, et ce sera la
\count\Bfootins=1200%
% \pend