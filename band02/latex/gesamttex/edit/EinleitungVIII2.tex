%\thispagestyle{empty}
\selectlanguage{german}
%{\vrule height 0mm depth 30mm width 0mm}
%\newpage
%\noindent
\thispagestyle{empty}
{\vrule height 0mm depth 30mm width 0mm}
%\par
%\noindent
\vspace*{2em}
\par\noindent Der vorliegende zweite Band der naturwissenschaftlichen, medizinischen und technischen Schriften vereint 99 Stücke aus allen drei Teilbereichen der 2001 gegründeten Reihe VIII der Leibniz-Edition. Fast alle in diesem Band edierten Texte sind in den Jahren, die Leibniz in Paris verbrachte (1672 bis 1676), entstanden. Keine der hier herausgegebenen Schriften war zu seinen Lebzeiten erschienen; der Text von insgesamt 14 Stücken wurde teilweise oder ganz im Zeitraum von 1849 bis 2001 abgedruckt (N. 6, N. 11, N. 12, N.~$36_2$, N. 50, N. 54, N. 55, N. 58, N. 69, N. 70, N. 76, N. 81, N. 82, N. 98). Dagegen werden 85 Stücke in diesem zweiten Band erstmals veröffentlicht und können damit erst jetzt eine Leserschaft finden. Bis auf wenige Ausnahmen stehen alle Handschriften, die Gegenstand der Reihe VIII sind, in hochauflösenden Digitalisaten online zur Verfügung (http://ritter.bbaw.de). Dies gilt auch für die Stücke dieses Bandes bis auf diejenigen, die Marginalienexemplare (N. 3, N. 13, N. 44, N. 46, N. 47), einen Druck (N. 81) oder erst jüngst entdeckte Handschriften aus Hannover (N. 51) oder Göttingen (N. 66) zur Vorlage haben.
\par\vspace{6.0ex}
\noindent
\noindent\uppercase{Themen des Bandes}
\par
\vspace{1.0ex}
\noindent
Der zweite Band bildet eine chronologische Einheit mit dem im Jahre 2009 erschienenen ersten Band der Reihe: Beide Bände decken die Pariser Jahre ab, in denen sich Leibniz hoch produktiv und vielfältig mit unterschiedlichen Themen auf verschiedenen Gebieten beschäftigte, neue Felder für sich entdeckte, sich mit Zeitgenossen und dem Forschungsstand seiner Zeit auseinandersetzte. Wie im ersten Band der Pariser Jahre folgt die thematische Einteilung seiner Schriften der Klassifikation, die Leibniz selbst im März 1673 in seinen \textit{Observata Philosophica} (VIII,1 N. 1) gegeben hat. Daher entspricht die Gliederung des Bandes nicht unserem heutigen Verständnis von Fachgebieten und -grenzen.\par 
\newpage
Der erste Band der Schriften aus der Pariser Zeit enthält Stücke zur Nautik (Nautica), Optik (Optica), Pneumatik (Pneumatica) und Technik (Technica). Der zweite Band verschafft der Rubrik Technica einen Zuwachs von 17 Stücken (N. 83 - N. 99) in Form von Nachträgen zum ersten Band. Rubriken, die in VIII,2 neu hinzukommen und sich auf Leibnizens Einteilung von 1673 stützen, sind Astronomica (N. 1, N. 2), Magnetica (N. 3 - N. 6), Mechanica (N. 7 - N. 52), Meteorologica (N. 53, N. 54), Physica (N. 55 - N. 57), Anatomica (N. 58), Botanica (N. 59, N. 60), Chymica (N. 61 - N. 65), Medica (N. 66 - N. 77) und Miscellanea (N. 78 - N. 81).\par
Schriften zur Mechanik haben den weitaus größten Anteil im Band, sowohl was die Anzahl der Stücke (46) als auch den Seitenumfang (374 Seiten) angeht. Leibniz beschäftigt sich hier mit verschiedenen Teilgebieten, die er in seiner Klassifikation von 1673 nicht eigens berücksichtigt. Sie bilden in dem vorliegenden Band folgende Unterrubriken zur Mechanik: Allgemeines, Bewegung, Festigkeit, Kraft, Reibung, spezielle Probleme und Stoß. Innerhalb der Mechanik sind es die Stücke zur Reibung (123 Seiten), die mit Abstand den größten Umfang haben, gefolgt von Stücken zu allgemeinen Problemen (80 Seiten); dagegen nehmen die weiteren Unterrubriken zur Mechanik jeweils deutlich weniger Raum ein, variieren aber in ihrem Umfang: Stoß (42 Seiten), Kraft (38 Seiten), Festigkeit (35 Seiten), spezielle Probleme (33 Seiten), Bewegung (23 Seiten). \par
Mit den naturwissenschaftlichen (Astronomica, Magnetica, Mechanica, Physica, Chymica), medizinischen (Anatomica, Botanica, Medica) und technischen (Technica) Schriften, die hiermit herausgegeben werden, vereint der aktuelle Band erstmals alle drei Teilbereiche der Reihe VIII. In dieser Hinsicht wird VIII,2 ein Unikum bleiben, da geplant ist, in allen weiteren Bänden der Reihe nur Schriften jeweils eines Teilbereichs zu edieren. 
\par\vspace{6.0ex}
\noindent
\noindent\uppercase{Stücke, Sprachen, Textarten}
\par\vspace{1.0ex}
\noindent
Von den gezählten 99 Stücken des Bandes zerfallen sieben in insgesamt 27 Unterstücke. Diese 119 Stücke und Unterstücke unterscheiden sich in Textart und Sprache. Auf Lateinisch sind 78 Stücke und Unterstücke geschrieben, auf Französisch 36, auf Deutsch vier (ein weiteres Stück besteht nur aus Zeichnungen); in 18 Fällen verwendet Leibniz zusätzlich noch weitere Sprachen (neben Lateinisch, Französisch und Deutsch sind dies Italienisch und Englisch). Eigentümlichkeiten in der Orthographie der von Leibniz benutzten französischen Sprache werden im Rahmen der Editionsrichtlinien beibehalten (z.B. \textit{servic} für \textit{service}, \textit{resistence} für \textit{résistance}).\par
\newpage
Das Gros der Stücke besteht aus 42 Aufzeichnungen (N. 4, N. 5, N. 11, N. 14, N.~$17_1$, N. $17_2$, N. 18, N. 20, N. 22 - N. 26, N. 33, N. $34_5$, N. 39, N. 42, N. 43, N. 49, N.~51, N. 52, N. 60, N. 63, N. 64, N. 69, N. 71, N. 72, N. 75, N. 83 - N. 87, N. 90, N.~96, N. $97_1$ - N. $97_4$, N. 99). Dabei handelt es sich um ausführlichere Notizen, die sich Leibniz von eigenen und fremden Gedanken, Erfahrungen, Beobachtungen, Berichten macht, oder um Listen, Rechnungen oder Zeichnungen, die er erstellt; darunter finden sich auch drei Gesprächsnotizen (N. 65, N. 77, N. 88). Von diesen 42 Aufzeichnungen sind 26 auf Lateinisch, 14 auf Französisch und zwei auf Deutsch verfasst; Nebensprachen sind Lateinisch in zwei französischen Aufzeichnungen (N. 77, N. 99), Deutsch (N. 84, N. 85) in zwei lateinischen und Italienisch in einer lateinischen (N. 22). Die meisten Aufzeichnungen stammen aus der Rubrik Technik (11), mit deutlichem Abstand gefolgt von zwei Unterrubriken der Mechanik, Festigkeit (6) und Bewegung (4), sowie der Medizin (5). \par
Als zweithäufigste Textart enthält der Band 31 Konzepte (N. 9, N. 10, N. 12, N. 15, N. 19, N. 21, N. 27, N. $28_5$, N. $28_6$, N. 29, N. 30, N. $31_2$, N. $31_3$, N. 32, N. $34_1$ - N. $34_4$, N.~35, N. 41, N. $45_1$ - N. $45_3$, N. 62, N. 78, N. 89, N. 91 - N. 94, N. $97_1$), d.h. längere Texte, teils mit ausformulierten Überschriften, die den Charakter von Entwürfen haben und zur weiteren Ausarbeitung, für Vorträge, zur Weitergabe oder zur Veröffentlichung bestimmt gewesen sein dürften. 26 Konzepte sind auf Lateinisch, die übrigen fünf auf Französisch geschrieben; Nebensprachen sind Französisch (N. $34_2$, N. 35, N. $45_1$, N. 78), Lateinisch (N. $34_1$) und Italienisch zusammen mit Deutsch (N. 78). Die meisten Konzepte stammen aus der Unterrubrik Reibung (9), gefolgt von der Rubrik Technik (6) und den weiteren Unterrubriken Kraft (4) und Spezielle Probleme (4).\par
Die dritthäufigste Textart im Band stellen die 20 Auszüge dar, d.h. Exzerpte oder Paraphrasen, die Leibniz aus gedruckten, handschriftlichen oder verschollenen Vorlagen erstellt und kommentiert hat (oder wie in N. $31_1$, N. 66 unkommentiert lässt). 16 Auszüge sind auf Lateinisch geschrieben, die übrigen vier auf Französisch; Nebensprachen sind Französisch (N. 7, N. 56), Lateinisch (N. 50), Deutsch (N. 68) und Englisch (N. 2). Die Themen der exzerpierten Literatur sind breit gestreut: Jeweils vier Auszüge gehören zur Mechanik (N. 7, N. 8, N. $31_1$, N. 50) und Medizin (N. 66, N. 68, N. 74, N. 76), drei zur Physik (N. 55 - N. 57), jeweils zwei zur Astronomie (N. 1, N. 2), Meteorologie (N. 53, N. 54) und Technik (N. 82, 98) sowie jeweils ein Auszug zur Anatomie (N. 58), Botanik (N. 59) und zum Magnetismus (N. 6). Zeichnungen, die ohne Stellennachweise in den edierten Auszügen wiedergegeben werden, sind von Leibniz selbst angefertigt, insofern dies anhand einer erhaltenen Vorlage überprüfbar gewesen ist. Nicht erhalten sind die Vorlagen derjenigen Auszüge, die Leibniz von Manuskripten aus dem verschollenen Nachlass René Descartes' (N. 6, N. 54, N. 58, N. 76, N. 82) und von einem verlorenen Manuskript Ole R{\o}mers (N. 98) anfertigt; die einzige Überlieferung bietet Leibniz auch für das Manuskript eines gewissen Herrn Acar (N. 74) und für ein weiteres von Claude Perrault (N. 57).\par
Eine vierte Gruppe an Texten bilden die zwölf Reinschriften, davon neun mit Verbesserungen (N. $28_2$ - N. $28_4$, N. $28_7$, N. 37, N. 38, N. $45_4$, N. 48, N. 70), zwei mit Verbesserungen und  Ergänzungen (N. $36_1$, N. $36_2$) und eine einzige von Schreiberhand (N. $28_1$). Die Reinschriften stammen bis auf eine Ausnahme (N. 70: Medica) sämtlich aus der Rubrik Mechanik, und zwar aus den Unterrubriken Kraft (5), Reibung (4), spezielle Probleme (1) und Stoß (1).\par
Eine kleinere Gruppe stellen die sieben Notizen (N. 16, N. 40, N. 61, N. 79, N. 80, N. 88, N. 95) dar, die den Aufzeichnungen als Textart nicht unähnlich sind, nur dass sie kürzer ausfallen und eher fragmentarischen Charakter haben. Drei dieser Notizen sind auf Lateinisch, eine auf Französisch, eine auf Deutsch geschrieben; eine weitere (N. 79), die größtenteils nicht von Leibniz stammt, ist mehrsprachig (Lateinisch, Italienisch, Deutsch, Französisch). Die nächstkleinere Gruppe an Stücken besteht aus sechs Anstreichungen mit Anmerkungen in Handexemplaren (N. 3, N. 13, N. 44, N. 46, N. 47, N. 67) lateinischer und französischer Bücher. Hier gibt die Edition Text und Zeichnungen des von Leibniz gelesenen Buches zu denjenigen Stellen wieder, die er markiert oder kommentiert hat. Informationen über seine Anstreichungen und den Inhalt seiner Anmerkungen liefert (wenn nicht anders vermerkt) der Marginalienapparat zur jeweiligen Seite.\par
Einen Einzelfall unter den im Band auftretenden Textarten stellt eine Abschrift (N.~73) dar, die Leibniz von einer Vorlage auf Französisch macht, die unbekannt ist. Da es dadurch keine Möglichkeit der Überprüfung gibt, könnte es sich hierbei auch um einen Auszug handeln.\par
\par\vspace{6.0ex}
\noindent
\noindent\uppercase{Datierung, inhaltliche Schwerpunkte und zeitlicher Verlauf}
\par\vspace{1.0ex}
\noindent
Die textkritische Ausgabe dokumentiert alle Änderungen und Ergänzungen und liefert die Marginalien, mit denen Leibniz den Inhalt seiner Schriften nachträglich kommentiert hat. Wann die Genese eines Textes jeweils ihren Abschluss fand und den heute erhaltenen Textbestand erreichte, lässt sich selten mit Sicherheit sagen. Nachträgliche Überarbeitungen sind nicht auszuschließen; zahlreiche Textschichten, wie sie der Variantenapparat dokumentiert, können, müssen aber kein Hinweis darauf sein, dass Leibniz in wiederholten Anläufen an einem Stück gearbeitet hat. Spätere Zusätze lassen sich selten wie in N. $36_2$ datieren: Dieses Stück zur Reibung ist laut Leibniz im Winter 1675 entstanden, während der Zusatz aus der Zeit nach Paris stammen muss. In zwei Fällen liefert uns Leibniz explizit das Datum einer späteren Überarbeitung: Den eigenhändig datierten Entwurf zum Perpetuum mobile (N. 92) aus dem Jahr 1674 nimmt er sich im Mai 1678 nochmals vor und formuliert einen Zusatz; eine spätere Ergänzung genau zu derselben Zeit erfährt ein weiteres Stück (N. 96) aus der Rubrik Technik: Leibniz korrigiert hier drei Jahre später Berechnungen, die er 1675 angestellt hat. Mit diesen drei Stücken zusammen sind im Ganzen 16 des Bandes eigenhändig von Leibniz datiert (N.~10, N. 11, N. 32, N. 34, N. $36_2$, N. 51, N. 52, N. 64, N. 65, N. 75, N. 76, N. 79, N.~92, N.~96, N. $97_1$, N. 98); im ersten Band der Reihe trifft dies nur auf insgesamt sechs Stücke zu.\par
Wie bereits der erste Band der Pariser Jahre enthält auch der zweite Schriften, die noch in der Mainzer Zeit entstanden sind (N. 39, N. 55, N. 66, N. 67, N. 69, N. 70, N.~71, N. 83). Die zwei ältesten (N. 39, N. 66) dieser insgesamt acht Stücke sind zugleich diejenigen des Bandes, die am ungenauesten datierbar sind: Beide haben ihre untere Datierungsgrenze im Jahr 1668, und ihre mögliche Entstehung erstreckt sich über eine Spanne von 47 bzw. 43 Monaten. Mit Ausnahme von vier weiteren Stücken (N. 47, N.~55, N. 73, N. 74) ist es bei allen übrigen gelungen, die jeweilige Entstehungszeit auf einen Zeitraum von maximal 13 Monaten einzugrenzen.\par 
Die 93 Stücke des Bandes, die mit dieser Genauigkeit datiert sind, können Aufschluss darüber geben, wie Leibniz sich im Verlauf seiner Pariser Zeit mit den verschiedenen Themen beschäftigte, die Gegenstand der in diesem Band veröffentlichten Schriften sind. In den acht noch in Mainz (von 1668 bis Februar 1672) entstandenen Stücken setzt sich Leibniz mit medizinischen Fragen (N. 67, N. 68, N. 69, N. 70, N. 71), mit Mechanik (N. 39) und mit der technischen Realisierung eines Perpetuum mobile auseinander (N. 83); daneben verfasst er sehr umfangreiche Exzerpte zur Physik (N. 55), die er womöglich erst in Paris abschließt.\par
Am 19. März 1672 tritt Leibniz seine Reise nach Paris an. 27 Stücke des Bandes haben ihre untere Datierungsgrenze im Monat seiner Abreise oder in den darauf folgenden Monaten des Jahres: Neben einzelnen Schriften zur Astronomie (N. 1), Chemie (N. 61), Medizin (N. 68) und zwei Stücken zum Magnetismus (N. 3, N. 4) verfasst Leibniz in diesem ersten Jahr seines Aufenthaltes überwiegend Schriften, die Technik (N. 84, N. 85, N. 86, N. 87, N. 88, N. 89) und vor allem Mechanik zum Gegenstand haben. Intensiv und erstmals überhaupt, wie er schreibt, setzt er sich hier mit Phänomenen der Bruchfestigkeit auseinander: Alle acht Stücke zur Festigkeitslehre sind in den ersten zwölf Monaten seines Aufenthalts entstanden (N. 19, N. 20, N. 21, N. 22, N. 23, N. 24, N. 25, N. 26). Des Weiteren\hfill sind\hfill es\hfill spezielle\hfill Probleme\hfill in\hfill der\hfill Mechanik\hfill (N.\hfill 40,\hfill N.\hfill 41,\hfill N.\hfill 42,\hfill N.\hfill 43),\hfill die 
\par\newpage\noindent Bewegungslehre (N. 13, N. 14, N. 15) und Stoßgesetze (N. 47, N. 48), womit er sich in der frühen Pariser Zeit beschäftigt.\par 
In das darauf folgende Jahr fällt seine Reise nach England, die von Ende Januar bis Anfang März dauert. Vergleichsweise wenige Stücke des Bandes sind 1673 entstanden: Exzerpte zu allgemeinen Fragen der Mechanik (N. 7), Anstreichungen in einem Marginalienexemplar zu speziellen Problemen der Mechanik (N. 44), Notizen zum Stoß (N.~49) sowie eine Aufzeichnung und ein Entwurf zur Technik von Uhrwerken (N. 90, N.~91). Hinzu kommt ein erstes Stück zur Chemie (N. 62), in dem ein alchemischer Ofen beschrieben wird.\par
1674 macht Leibniz in der Mechanik erstmals Kraft (N. 27, N. 28 im Umfang von 30 Editionsseiten) zu einem eigenen Thema, setzt sich weiter mit Bewegungslehre (N.~16, 17), allgemeinen (N. 8, N. 9, N. 10) und speziellen Fragestellungen (N. 45, N. 46) auseinander und macht Exzerpte zum Stoß (N. 50). Damit bildet die Mechanik für 1674 mit zehn von insgesamt 20 Stücken wieder die größte Rubrik, ohne dass jedoch ein klarer Schwerpunkt darin erkennbar wird, der allenfalls in seinen Untersuchungen zum Kraftbegriff liegen könnte. Zeitgleich fertigt Leibniz Entwürfe und Zeichnungen zur Technik (N. 92, N. 93, N. 94, N. 95) an, erstellt Exzerpte aus einem Erdbebenbericht (N. 53), macht Notizen zum Magnetismus (N. 5) und zu einem Gespräch medizinischen Inhalts (N. 72; seine Abschrift eines medizinischen Manuskripts in N. 73 kann auch zwei Jahre später entstanden sein). Unter den vermischten Schriften (N. 79, N. 80) des Jahres 1674 findet sich ein Entwurf zu militärischen Fachtermini (N. 78).\par
Die Mechanik vergrößert für das darauf folgende Jahr ihren Anteil am Band. Von den 23 im Jahr 1675 entstandenen Stücken entfallen 13 auf diese Rubrik, wobei sich hier ein deutlicher Schwerpunkt zeigt: In neun Stücken setzt sich Leibniz mit dem Phänomen der Reibung auseinander und verfasst im Laufe dieses Jahres alle Schriften, die in seiner Pariser Zeit hierzu entstanden sind. Die Reibungsstücke weisen nicht nur unter allen Unterrubriken der Mechanik den größten Seitenumfang auf, sondern übertreffen darin auch alle übrigen Rubriken des Bandes. Bis auf Bewegungslehre (N. 18 und vielleicht noch N. 16, N. 17) sowie Stoßgesetze (N. 51, N. 52) werden 1675 andere Gebiete der Mechanik nur allgemein (N. 11) behandelt. Außerhalb der Mechanik beschäftigt sich Leibniz im vorletzten Jahr seines Auslandsaufenthaltes mit Technik verschiedener Art (N.~96, N.~97, N. 98) und notiert sich chemische Verfahren (N. 63, N. 64) sowie Gespräche, die er darüber geführt hat (N. 65). Er fertigt Exzerpte aus einem Buch zur Astronomie und aus einer physikalischen Abhandlung (N. 57) an; weitere Auszüge physikalischen Inhalts (N. 56) sowie zur Medizin (N. 74) könnten 1675 oder auch später entstanden sein.\par\newpage
Die aus dem letzten Jahr seines Paris-Aufenthaltes stammenden Stücke des Bandes lassen erkennen, dass Leibniz sich 1676 nicht mehr überwiegend mit Phänomenen der Mechanik auseinandersetzt. Der Schwerpunkt seiner Beschäftigung verschiebt sich in den Bereich der Lebenswissenschaften, repräsentiert durch Schriften zur Anatomie (N. 58), Botanik (N. 59, N. 60), Medizin (N. 75, N. 76, N. 77) und Meteorologie (N. 54, ein Stück, das teils biologischen Inhalts ist). Außerhalb dieses Bereichs exzerpiert Leibniz aus Schriften, die das Phänomen des Hagels (N. 54), den Magnetismus (N. 6) und die optische Brechung (N. 82) zum Gegenstand haben, notiert sich Inhalte aus einem Gespräch über Architektur (N. 81) und verschiedene technische Einfälle (N. 99). In der Mechanik macht er einen Entwurf zu deren allgemeinen Prinzipien (N. 12) und einen weiteren zum Begriff der Kraft (N. 29). Diese Verschiebung des Schwerpunktes in die Lebenswissenschaften geht zeitlich damit einher, dass Leibniz in der Zeit von Februar bis Anfang Oktober 1676 Zugang zu dem heute verschollenen Nachlass René Descartes' hatte. Auszüge aus dessen verlorenen Manuskripten (N. 6, N. 54, N. 58, N. 76, N. 82), darunter vor allem die Anatomica (N.~58), bilden den größten Seitenumfang unter den Stücken, die in diesem Band aus dem Jahr 1676 stammen. Auch Gespräche, die er nach seiner Abreise aus Paris am 4. Oktober 1676 in London führt, haben überwiegend Medizinisches zum Gegenstand (N. 77).\par
Die in VIII,2 edierten Stücke zeigen, dass sich Leibniz in jedem Jahr seiner Pariser Zeit mit Technik und Mechanik beschäftigte. In der Mechanik konzentrierte er sich 1672 auf die Festigkeitslehre, 1674 auf den Kraftbegriff und 1675 auf Reibungsphänomene; letztere behandelt er besonders ausgiebig. Gemessen an dem Umfang seiner in VIII,2 edierten Schriften setzt sich Leibniz mit dem Stoß ähnlich intensiv auseinander wie mit Festigkeit und Kraft. Jedoch verläuft seine Beschäftigung damit auf quantitativ niedrigem Niveau und zeitlich über die Pariser Zeit gestreckt, wobei aus dem letzten Jahr seines Aufenthaltes kein Stück zum Stoß stammt. In diesem letzten Jahr überwiegen Themen, die nicht in den Bereich der Technik und Mechanik fallen, sondern den Lebenswissenschaften zuzurechnen sind.\par 
Genauer lässt sich der zeitliche Verlauf seiner Pariser Arbeiten fassen, wenn die 61 Stücke (von insgesamt 71) berücksichtigt werden, die im ersten Band der Reihe ediert und ähnlich gut datierbar sind wie diejenigen, die aus VIII,2 in die Betrachtung einfließen. Von 1669 bis zu seiner Abreise nach Paris forscht und schreibt Leibniz demnach nicht nur auf Gebieten der Medizin, Mechanik und Technik, wie aus VIII,2 zu ersehen, sondern beschäftigt sich zudem mit Nautik (VIII,1 N. 2 - N. 5) und Optik (VIII,1 N. 14 - N. 18, N. 33). Vor allem schreibt er in diesem Zeitraum aber über Technik, zu der die meisten 
\par\newpage\noindent Stücke aus der Zeit vor Paris gehören, die in beiden Bänden der Reihe VIII ediert sind (VIII,1 N. 56 - N. 62, VIII,2 N. 83).\par
Zählt man die in VIII,1 und VIII,2 edierten Stücke, die sich auf 13 Monate genau datieren lassen, nach Jahren getrennt zusammen, ergibt sich für das erste Jahr in Paris eine Zahl von 22 Stücken, die Leibniz seit März begonnen und auch 1672 abgeschlossen hat; bei 20 weiteren reicht die Datierungsspanne noch bis ins nächste Jahr, so dass diese Stücke erst 1673 in Angriff genommen oder beendet worden sein könnten. Sicher begonnen und abgeschlossen hat Leibniz 1673 seine Arbeit an 25 Stücken. Aus den zwölf Monaten von 1674 stammen 14 Stücke, 19 weitere könnte Leibniz noch in diesem Jahr oder erst im nächsten angefangen oder fertig gestellt haben. Im Laufe des Jahres 1675 sind 33 Stücke entstanden; zwei weitere vielleicht erst im Jahr darauf. In diesem letzten Jahr haben 19 Stücke ihren Ursprung und finden auch 1676 ihren Abschluss. Somit liefern diejenigen Stücke, die sich jeweils sicher auf die zwölf Monate eines Jahres datieren lassen, folgende Zahlen für die Pariser Zeit: 22 Stücke 1672 (seit März), 25 Stücke 1673, 14 Stücke 1674, 33 Stücke 1675, 19 Stücke 1676.\par 
Diejenigen Stücke, deren Entstehung nicht sicher auf ein Kalenderjahr datierbar ist, lassen sich dadurch berücksichtigen, dass die Pariser Zeit entsprechend den Datierungsspannen in zusammenhängenden Zeiträumen betrachtet wird. Daraus ergibt sich eine Zahl von 67 Stücken, die 1672 und 1673 insgesamt entstanden sind, von 66 Stücken aus den Jahren 1674 und 1675, von 19 Stücken aus dem Jahr 1676; unberücksichtigt dabei bleiben aus beiden Bänden nur zwei der auf 13 Monate genau datierbaren Stücke (VIII,1 N. 69 und VIII,2 N. 56) – sie könnten 1675 oder 1676 entstanden sein. Somit geht eine etwa gleich große Zahl an Stücken in VIII, 1 und VIII,2 auf die ersten beiden (67 Stücke 1672 und 1673) wie auf die zwei darauf folgenden Jahre (66 Stücke 1674 und 1675) zurück, wobei die Produktivität an Stücken für den ersten Zeitraum höher anzusiedeln wäre, da Leibniz erst im März des Jahres in Paris zu arbeiten beginnt. Im letzten Jahr seiner Zeit im Ausland, die er mit einer Reise nach England und in die Niederlande beschließt, entstehen 19 Stücke, die verglichen mit dem Jahresschnitt (33) an Stücken aus den vorangegangenen beiden Zeiträumen eine kleinere Zahl darstellen.\par
Der Band VIII,1 liefert für jedes Jahr (bis auf 1674) Stücke zur Technik (VIII,1 N. 62 - N. 65, N. 68, N. 69, N. 71) und bestätigt damit, was sich in VIII,2 gezeigt hat, dass Technik während der gesamten Pariser Zeit ein konstantes Arbeitsgebiet für Leibniz gewesen ist. Problemen der Nautik (1672: N. 6 - N. 8; 1673: N. 9. - N. 11; 1676: N. 12, N. 13) und der Optik (1672: N. 19; 1673: N. 20 - N. 26; 1676: N. 33 - N.~35) geht Leibniz in den ersten beiden Jahren und im letzten Jahr seines Aufenthalts nach. Nicht nur die zeitliche Abfolge ihrer Entstehung haben die Schriften zu Nautik und Optik gemein, sondern auch der Seitenumfang, den die hierzu edierten (und datierbaren) Stücke in VIII,1 einnehmen, ist annähernd gleich groß (63 bzw. 62 Seiten). Damit liefern Nautik und Optik Themen, zu denen Leibniz 1672 zusätzlich zur Astronomie, Chemie, Medizin und Mechanik gearbeitet hat. Hinzu kommt ganz besonders noch das Gebiet der Pneumatik. Diese Rubrik nimmt überhaupt den größten Teil der in VIII,1 edierten Seiten ein. Allein diejenigen Stücke zur Pneumatik, die aus dem Jahr 1672 stammen (VIII,1 N.~36 - N. 46), sind (mit 169 Seiten) deutlich umfangreicher als seine Schriften zur Festigkeit (35 Seiten), die sämtlich in demselben Jahr entstanden sind. Die Pneumatik dürfte 1672 daher im Zentrum seines Interesses gestanden haben. \par
Auch 1673, im zweiten Jahr seines Aufenthaltes, findet Pneumatisches (VIII,1 N.~47 - N. 51 im Umfang von 86 Editionsseiten) den größten Niederschlag in seinen naturwissenschaftlichen, medizinischen und technischen Schriften. In dieses zweite Jahr seines Aufenthaltes fällt zugleich der Höhepunkt seiner Pariser Produktion zur Optik (VIII,1 N. 20 - N. 26 im Umfang von 43 Editionsseiten). Daneben beschäftigt er sich 1673 noch mit Chemie (VIII,2 N. 62), Mechanik (VIII,2 N. 7, N. 44, N. 49) und Technik (VIII,2 N.~90, N. 91), die Gegenstand des zweiten Bandes sind, aber in weit geringerem Umfang, so dass Optik und noch viel mehr Pneumatik Schwerpunkte seines zweiten Jahres in Paris bilden.\par
Für das dritte Jahr seiner Pariser Zeit liefert der Band VIII,1 im Ganzen nur drei Stücke, deren Entstehung sich zeitlich auf 1674 eingrenzen lässt und die alle drei zur Pneumatik (VIII,1 N. 52 - N. 54) gehören. Damit ist für dieses dritte Jahr seines Aufenthaltes zwar ein weiteres Feld zu berücksichtigen, auf dem Leibniz tätig gewesen ist. Es ergibt sich daraus aber für 1674 kein neuer Schwerpunkt, der sich wie oben festgestellt allenfalls für seine Untersuchungen zum Kraftbegriff beanspruchen ließe. Ebenso wenig ändert sich das Bild für 1675: Zwei auf dieses Jahr datierbare Stücke finden sich im ersten Band der Pariser Zeit, das eine zur Pneumatik (VIII,1 N. 55), das andere zur Technik (VIII,1 N. 68); zwei weitere Stücke, die entweder in diesem oder erst im darauf folgenden Jahr entstanden sein könnten, gehören beide wiederum zur Technik (VIII,1 N. 69, N.~70). Schwerpunkt für Leibniz bleibt 1675 damit die Reibung. Dass VIII,1 für dieses Jahr fast nur Technik-Stücke liefert, lässt sich mit der Reibung als Hauptinteresse in Einklang bringen, da es Phänomenen gilt, deren genauere Kenntnis von technisch praktischem Nutzen ist.\par 
Für das letzte Jahr, das Leibniz in Paris verbringt, finden sich in VIII,1 im Ganzen sechs Stücke, darunter keines zur Pneumatik, drei zur Optik (VIII,1 N. 33 - N. 35), zwei zur Nautik (VIII,1 N. 12, N. 13) und eines zur Technik (VIII,1 N. 71). Damit scheint sich die aus VIII,2 gewonnene Einschätzung durchaus zu bestätigen, dass Mechanisches im weitesten Sinne (Mechanik mit ihren verschiedenen Unterrubriken sowie Technik und Pneumatik) für Leibniz im letzten Jahr seines Aufenthaltes von geringerem Interesse gewesen ist als in der Pariser Zeit zuvor und weniger Niederschlag in seinen Schriften findet als Themen aus dem Bereich der Lebenswissenschaften.\par
Fasst man die in VIII,1 und VIII,2 edierten Stücke schwerpunktartig und nach einzelnen Jahren zusammen, ergibt sich abschließend daraus folgender Verlauf für die Pariser Arbeiten: 1672 Pneumatik, daneben Festigkeitslehre; 1673 Pneumatik, daneben Optik; 1674 Kraftbegriff; 1675 Reibungslehre; 1676 Medizin und Biologie. Die Datierung der hierbei berücksichtigten Stücke erlaubt teils zwar eine Spanne von bis zu 13 Monaten. Über die Arbeitsschwerpunkte selbst und deren zeitliche Abfolge dürften sie damit aber dennoch, wenn auch nicht aufs Jahr genau, Aufschluss geben können.\par
\par\vspace{6.0ex}
\noindent
\noindent\uppercase{Inhalt der Stücke im Überblick}
\par\vspace{3.0ex}
\noindent
I. Astronomica (N.~1, N.~2)\par\vspace{1.0ex}
\noindent
Seine Beschäftigung mit Astronomie zeigt sich an Exzerpten, die Leibniz am Anfang und gegen Ende seiner Pariser Zeit anfertigt. Bei seiner Lektüre von Pierre Gassendis\protect\index{Namensregister}{\textso{Gassendi} (Gassendus), Pierre 1592-1655}
 1658 posthum gedruckten \title{Opera omnia} (N.~1) hält er ganz überwiegend Stellen zu astronomischen Phänomenen (Mondillusion\protect\index{Sachverzeichnis}{Mondillusion}, Nebensonnen\protect\index{Sachverzeichnis}{Nebensonnen}, Kometen\protect\index{Komet}{pendule}, Fixsterne\protect\index{Sachverzeichnis}{Fixsterne}, Merkurdurchgang\protect\index{Sachverzeichnis}{Merkurdurchgang}) und zu kosmologisch relevanten Fragen fest (Galileis Gezeitentheorie\protect\index{Sachverzeichnis}{Gezeiten}, Ursachen von Ebbe und Flut). Weitaus umfangreichere Auszüge, überwiegend in lateinischer Übersetzung, fertigt Leibniz von einem 1674 erschienenen Buch Robert Hookes\protect\index{Namensregister}{\textso{Hooke}, Robert 1635-1703}
 an (N.~2). Aus den gegen den Danziger Astronomen Johannes Hevelius\protect\index{Namensregister}{\textso{Hevelius}, Johannes 1611-1687} gerichteten \title{Animadversions} referiert Leibniz Argumente, mit denen Hooke eine teleskopgestützte Vermessung der Sternörter\protect\index{Sachverzeichnis}{Sternörter} propagiert und die Praxis des Hevelius sowie dessen Einwände dagegen als hinfällig darzulegen sucht; besondere Aufmerksamkeit schenkt Leibniz hierbei den material- und konstruktionstechnischen Überlegungen Hookes sowie dem Auflösungsvermögen des Auges, dessen Begrenztheit Hooke auch experimentell nachweist. Mehr Raum bietet Leibniz aber \mbox{Hookes} Ausführungen zu dessen eigener Beobachtungspraxis, hierbei insbesondere zur atmosphärischen Brechung, zu Instrumenten, Entwürfen und Zielen einer teleskopgenauen Bestimmung der Sternpositionen.\par\vspace{3.0ex}
 \newpage
\noindent
II. Magnetica (N.~3\ -\ N.~6)\par\vspace{1.0ex}
\noindent
Wiederum zu Beginn und Anfang seines Aufenthaltes in Paris finden sich Lesespuren für eine Beschäftigung mit dem Magnetismus\protect\index{Sachverzeichnis}{Magnetismus}. Seine Anstreichungen in Vincent Léotauds\cite{01061} \title{Magnetologia} (N.~3) beziehen sich auf grundlegende Erkenntnisse über die magnetische Missweisung\protect\index{Sachverzeichnis}{Missweisung} und Eigenschaften der Magnetkraft\protect\index{Sachverzeichnis}{Magnetkraft}, die durch William Gilbert\protect\index{Namensregister}{\textso{Gilbert}, William 1544-1603}, Niccolò Cabeo\protect\index{Namensregister}{\textso{Cabeo}, Niccolò 1586-1650} und Athanasius Kircher\protect\index{Namensregister}{\textso{Kircher}, Athanasius 1602-1680} als gesichert galten. In seinen Auszügen aus Kirchers \title{Magnes} (N.~6) zeigt sich Leibniz eher an praktischen Anwendungen der Magnetkraft interessiert. Eigene Beobachtungen macht Leibniz an Magnetnadeln verschiedenen Gewichts, unterschiedlicher Grö{\ss}e, Lage, Form und Temperatur (N.~4). Er entwirft Versuchsanordnungen, um die Kraft des Magnetismus dahingehend zu bestimmen, ob sie in einer bestimmten Entfernung gar nicht mehr oder blo{\ss} schwächer wirkt (N.~5).
\par\vspace{3.0ex}
\noindent
III. Mechanica (N.~7\ -\ N.~52)\par\vspace{2.0ex}
\noindent
\textit{III.A. Allgemein (N.~7 - N.~12)}
\par\noindent
Schriften zur Mechanik bilden die grö{\ss}te Gruppe an Stücken in diesem Band. Die erste Unterrubrik hierzu versammelt Stücke, die sich nicht ausschlie{\ss}lich einem Bereich zuordnen lassen. 
Verschiedene Themen, mit denen sich Leibniz in Paris beschäftigt, werden auch in \title{La Statique}\cite{00296} behandelt, ein Werk des Jesuiten Ignace Gaston Pardies\protect\index{Namensregister}{\textso{Pardies}} von 1673, das Leibniz kurz nach dem Erscheinen zu lesen begann. Leibniz interessiert sich hier neben dem Gleichgewicht von Körpern vor allem für den freien Fall, Pendelbewegung und die Bestimmung des Schwerpunkts (N. ~7). 
Bedeutend intensiver verläuft seine Auseinandersetzung mit John Wallis\protect\index{Namensregister}{\textso{Wallis} (Wallisius), John 1616-1703}. Umfangreiche und kritisch kommentierte Auszüge aus dessen \title{Mechanica} beziehen sich auf Pendelbewegung, Hebelkraft, Reibung und Rollwiderstand, Sto{\ss} und Elastizität, Hydrostatik, auf Versuche mit Quecksilbersäulen sowie auf Überlegungen und Anschauungsbeispiele zum Schwerpunkt von Körpern (N.~8).
In Anlehnung an dieses Werk definiert Leibniz grundlegende Begriffe der Mechanik, kritisiert Wallis'\protect\index{Namensregister}{\textso{Wallis} (Wallisius), John 1616-1703} Bewegungslehre, legt seine eigenen Vorstellungen über Geschwindigkeit und Kräfte beim freien Fall und entlang schiefer Ebenen dar und erörtert die Natur der Schwerkraft (N.~9). Zu verschiedenen Themen (Statik, Hebelkräfte, Festigkeit u.a.) fasst Leibniz seine eigene Kritik sowie die von Zeitgenossen zusammen (N.~10). 
Ähnlich stichpunktartig kommt er auch auf Errungenschaften seiner Zeit zu sprechen bzw. auf das, was eine \textit{schönere Geometrie} zu leisten vermag, die erklären soll, was Natur und Kunst hervorbringen (N.~11). 
Gegen Ende seines Aufenthalts erkennt Leibniz die Notwendigkeit, die bislang entdeckten Bewegungsgesetze auf ein allgemeines Prinzip zurückzuführen; das aus dem Verhältnis von Ursache und Wirkung zu gewinnen sei; dies würde erlauben, Wirkungen vorherzusagen und zu berechnen und somit die Mechanik zu vervollkommnen (N.~12).
\par\vspace{2.0ex}
\noindent
\textit{III.B. Bewegung (N.~13 - N.~18)}
\par\noindent
Fall- und Pendelbewegung sind Gegenstand der Stücke dieser Unterrubrik. In zwei Stücken geht Leibniz der Frage nach, wie die Länge von Pendeln und die Schwingungszahl im Verhältnis stehen (N. 16, N. 17). Leibnizens Anstreichungen in einem Exemplar der \title{Physica demonstratio} von Pierre Le Cazre\protect\index{Namensregister}{\textso{Le Cazre} (Cazreus), Pierre 1589-1664} beziehen sich auf Versuche zum freien Fall und dessen Gesetzmä{\ss}igkeit (N.~13). Mit der Fallbewegung setzt er sich auch in zwei weiteren Stücken auseinander (N.~14, N.~15).
Freier Fall, Pendelbewegung und Sto{\ss} lassen Leibniz im Phänomen der Bewegung mehr als bloße Ortsveränderung erkennen und eine Unterscheidung nach verschiedenen Kräften treffen (\textit{vires mortuae}, \textit{vires vivae}) (N.~18).
\par\vspace{2.0ex}
\noindent
\textit{III.C. Festigkeit (N. 19 - N. 26)}
\par\noindent
Galileis \textit{Discorsi} bilden für Leibniz den Ausgangspunkt, sich mit dem Widerstand und der Bruchfestigkeit von Balken zu beschäftigen. Obwohl oder gerade weil Leibniz, wie er selbst bemerkt, völlig neu auf diesem Gebiet war, spart er nicht mit umfassender Kritik an Galileis Festigkeitslehre (N. 19, N. 21); eine längere Passage aus den \textit{Discorsi} zitiert Leibniz in eigener Übersetzung und kommentiert sie eingehend (N. 21); Leibniz weist Ergebnisse und Methoden Galileis zurück, beklagt, dass echte bzw. mathematische Beweise fehlten. Von dieser Kritik ausgenommen ist die Erkenntnis Galileis, dass ein Balken an jeder Stelle in gleichem Ma{\ss}e dem Verbiegen stand hält, wenn er dem Längsschnitt nach von der Aufhängung zum Ende hin parabolisch geformt ist, was Leibniz mathematisch darlegt (N. 22). Des Weiteren zeigt er, unter welchen Bedingungen der Biegungswiderstand eines Balkens an verschiedenen Stellen gleich gro{\ss} ist (N. 21). Er geht der Frage nach, wie genau sich der Punkt bestimmen lässt, an dem ein auf unterschiedliche Weise eingespannter Balken bricht (N. 23) und will den Biegungswiderstand in Relation zur Dicke und zum Gewicht, das ihn belastet, bestimmen (N. 26). Leibniz stellt Überlegungen zur Zugfestigkeit eines aus Platten bestehenden Gegenstands an, die von der Kraft eines angehängten Gewichts zusammengehalten und von zwei weiteren Kräften auseinander gezogen werden (N. 24). Diese Überlegungen verfolgt er weiter, insbesondere an fadenförmigen Gegenständen, die teils über verschiedene Knotenpunkte und Verbindungen verfügen, teils von unterschiedlich festem Material sind und von gleich oder verschieden gro{\ss}en Kräften auseinander gezogen werde (N. 25). 
\par
\par\vspace{2.0ex}
\newpage
\noindent
\textit{III.D. Kraft (N. 27 - N. 29)}
\par\noindent
Die drei Stücke dieser Unterrubrik hat Leibniz in der zweiten Hälfte seines Paris-Aufenthalts verfasst. Er liefert darin Ansätze, ein Ma{\ss} für die Kraft als Bewegungsgröße zu bestimmen. Im frühesten dieser Stücke, das aus dem Jahr 1674 stammt, geht Leibniz der Frage nach, wie gro{\ss} die Kraft sein kann, die ein fallender Körper auf einen ruhenden ausübt und wovon seine Wirkung abhängt. Seine Überlegungen stützt er auf ein (Gedanken)Experiment, wonach ein Körper aus unterschiedlicher Höhe fällt und vermittels Schnüren und Winden andere Körper, die verschieden oder gleich schwer oder gro{\ss} sind, in die Höhe befördert. Leibniz sieht hierin eine Möglichkeit feststellen zu können, wie sich Fall- und Steighöhen bei gleichen und unterschiedlichen Körpern quantitativ zueinander verhalten und ob die Reibung des Körpers sowie die der Luft bei grö{\ss}erer Kraft zunimmt (N. 27). In mehreren Entwürfen (N. $28_2$, N. $28_5$, N. $28_6$, N. $28_7$) und Reinschriften (N.~$28_1$, N. $28_3$, N. $28_4$) bestimmt Leibniz geometrisch die Kraft einer Maschine, die aus einem drehbar aufgehängten Rad besteht, an dem au{\ss}en wie innen jeweils Gewichte angebracht sind; Leibniz will dies für eine Anfangsdrehung ohne Beschleunigung (N. $28_2$), für jede beliebige Drehung (N. $28_1$, N. $28_3$) sowie für eine beschleunigte Bewegung (N.~$28_7$) geleistet haben. Einen allgemeinen Grundsatz (\textit{axioma}), um eine Kraft zu messen, findet Leibniz darin, dass die Wirkungen (\textit{effectus}) ähnlicher Wirkursachen (\textit{potentiae}, \textit{causae}) gleichmächtig seien. So lie{\ss}en sich die Wirkursachen an ihren Wirkungen messen. Ein gemeinsames Ma{\ss} für die Wirkursachen könne daraus gewonnen werden, in welche Höhe ein Körper dadurch angehoben werde; abschlie{\ss}end kommt Leibniz kritisch auf seine bisherigen Überlegungen zum Sto{\ss} zu sprechen, die auf der Annahme beruhen, dass der \textit{conatus} nicht verloren gehe (N. 29). \par
\par\vspace{2.0ex}
\noindent
\textit{III.E. Reibung (N. 30 - N. 38)}
\par\noindent
Bei seinen Untersuchungen zum Sto{\ss} (N. 48 - N. 52) nimmt Leibniz das Phänomen der Reibung (N. 50) in den Blick, mit dem er sich im Anschluss daran eingehend beschäftigt, um es physikalisch und mathematisch zu beschreiben. In seinen frühesten Überlegungen, die er selbst noch als verworren (\textit{confusanea}) bezeichnet, führt Leibniz eine ganze Reihe von Faktoren an, denen er einen Anteil bei der Reibung (\textit{detrimentum motus}) zuschreibt: Grö{\ss}e, Gewicht, Geschwindigkeit, Oberflächenbeschaffenheit und Kontaktflächengrö{\ss}e des bewegten Körpers, die Eigenschaften der Ebene (Glattheit, Rauheit), auf der sich ein Körper bewegt, bzw. die Art des Mediums (Luft, Äther, Wasser), in dem sich die Bewegung vollzieht (N. 30). Nach seiner Lektüre von John Wallis, der sich in dem zweibändigen Werk \textit{Mechanica} (1670-1671) mit dem Phänomen eines \textit{motus retardatus} bzw. einer \textit{vis impeditiva} auseinandersetzt (N. $31_1$), konzentriert sich Leibniz darauf, den Widerstand zu untersuchen, den das Medium einem Körper entgegensetzt, der sich gleichförmig bewegt; dabei kommt er zu dem Schluss, dass sich die infolge der Reibung verringerten Kräfte bzw. Geschwindigkeiten so zu den durchlaufenen Strecken verhalten wie Zahlen zu ihren Logarithmen, und stellt hierzu infinitesimale Berechnungen an (N. $31_2$). Schon früh beabsichtigt Leibniz offenbar zu veröffentlichen (N. $31_3$), was er für seine gro{\ss}e Entdeckung hält: Wie bedeutend diese mathematische Bestimmung der Reibung im Rahmen der Mechanik sowie für technische Anwendungen sei, legt Leibniz eigens dar; überdies will er beweisen, dass sich bei einer gleichförmigen Bewegung wie auch bei einer gleichmä{\ss}ig beschleunigten Bewegung die zurückgelegten Strecken wie die Logarithmen der durchlaufenen Zeiten verhalten; seine logarithmische Beschreibung der Reibung sieht er auch bei der Pendelschwingung und der Wurfparabel bestätigt; er verfolgt wieder einen infinitesimalen Ansatz, um das Phänomen zu berechnen (N. 32, N. 33). In einer Gruppe von fünf Unterstücken (N. $34_1$ bis N. $34_5$) differenziert Leibniz Reibung begrifflich, versucht das Phänomen durch Vorgänge auf mikroskopischer Ebene mechanisch zu erklären und stützt seine geometrische Beschreibung durch fünf bzw. sechs Theoreme, bei denen er sich teils auf das \textit{Opus geometricum} von Gr\'{e}goire de Saint-Vincent und auf Nicolaus Mercators \textit{Logarithmotechnia} beruft. Ausgehend von Sto{\ss} und Pendelschwingung betrachtet er, wie das Medium und ein sich darin bewegender Körper aufeinander wirken, und kommt zu dem Schluss, dass die Reibung mit der Geschwindigkeit der Bewegung zunehme (N.~$34_1$). Den reibungsbedingten Verlust an Bewegung führt er auf die Oberflächenstruktur der Materie zurück: Sie bestehe aus vielen kleinen Stiften, Zähnen oder Spitzen, die wippen oder elastisch biegsam wie Federn seien und beim Kontakt zwischen Körpern umgebogen und überwunden werden müssten, was auch hörbar geschehe (N. $34_2$, N. $34_3$). Leibniz unterscheidet zwischen einer relativen und einer absoluten Form der Reibung: Die \textit{Resistence respective} nehme (N. $34_1$, N. $34_2$) proportional mit der Geschwindigkeit des bewegten Körpers zu, während die \textit{Resistence absolue} davon unabhängig sei (N. $34_4$, N. $34_5$). Leibniz bestimmt den durch absoluten Widerstand bedingten Verlust an Geschwindigkeit ($\sqrt{2ax}$), berechnet und erklärt hiervon ausgehend in mehreren Ansätzen, wie ein bewegter Körper durch den respektiven und den absoluten Widerstand verlangsamt wird; Ursache hierfür seien einerseits elastische Unebenheiten der Kontaktfläche, andererseits eine Abfolge korpuskularer Stö{\ss}e im Medium (N. 35). Leibniz fasst seine Ergebnisse offenbar wieder mit Blick auf eine geplante Veröffentlichung zusammen (N. 36): Er präzisiert nochmals, wie man sich die mechanischen Ursachen der Reibung vorstellen kann (wovon N. 37 eine Abschrift liefert), konzentriert sich im Ganzen aber auf die \textit{Resistence absolue}, die er in sechs Theoremen darlegt (N. $36_1$); eine spätere Ergänzung, die erst in der Zeit nach Paris entstanden sein kann, lässt seine Ergebnisse mit Hinweis darauf haltlos werden, dass sich das richtige Kraftma{\ss} der Bewegung nicht aus $mv$ (wie bisher angenommen), sondern aus $mv^{2}$ ergebe (N. $36_2$). Im letzten Stück zur Reibung stellt Leibniz seine wichtigsten Untersuchungsergebnisse in einer Art Thesenpapier zusammen (N. 38). 
\par
\par\vspace{2.0ex}
\noindent
\textit{III.F. Spezielle Probleme (N. 39 - N. 46)}
\par\noindent
Den grö{\ss}ten Raum in dieser Unterrubrik nehmen Stücke ein, in denen sich Leibniz mit dem Schwerpunkt bzw. Massemittelpunkt (\textit{centrum gravitatis}) beschäftigt; weitere Themen sind zusammengesetzte Hebel und die Oberflächenspannung an Wassertropfen. Seine Beschäftigung mit der Schwerpunktsbestimmung belegen Anstreichungen in Büchern von Ignace Gaston Pardies und Jean de Beaugrand (N. 44 u. N. 46). Im Fall des Sto{\ss}es hofft Leibniz, den Schwerpunkt mittels mathematischer Folgen zu finden, in denen sich die Geschwindigkeiten der bewegten Körper ausdrücken (N. 40). An seine ersten Versuche, den Schwerpunkt geometrisch zu bestimmen (N. 41), knüpfen sich Überlegungen zur Statik (N. 42) und zu Hebelkräften (N. 45). Das Verhältnis dieser Kräfte untersucht er bei zusammengesetzten Hebeln und formuliert ein Theorem, das für alle derartigen Fälle gültig sei und alle Probleme in diesem Zusammenhang löse (N. $45_4$). Daneben beschäftigt sich Leibniz mit der Form und dem Verhalten von Tropfen: An einer schiefen oder horizontalen Ebene könne man bei Wasser- oder Wachstropfen die besondere Eigenschaft beobachten, dass sie sich halten (\textit{tenacitas}), und womöglich die Stärke ihres Festhaltens (\textit{gradus tenacitatis}) beurteilen (N. 39). Die besondere Tropfenform, die Leibniz hier als erster erklärt haben will, führt er auf den inneren Zusammenhalt der Tropfenteile (\textit{cohaesio}, \textit{connexio partium}) zurück (N. 43). 
\par
\par\vspace{2.0ex}
\noindent
\textit{III.G. Sto{\ss} (N. 47 - N. 52)}
\par\noindent
Seine ersten Pariser Stücke zum Sto{\ss} zweier Körper knüpfen an Überlegungen an, die Leibniz im Jahr zuvor in der \textit{Theoria motus abstracti} und der \textit{Hypothesis physica nova} angestellt hat: Geschwindigkeit ist der entscheidende Faktor beim Sto{\ss}; der schnellere Körper rei{\ss}t den schwächeren mit sich und verliert dabei an Geschwindigkeit; zwei gleich schnelle, aber unterschiedlich lange Körper kommen zur Ruhe, wenn sie geradlinig und zentral zusammensto{\ss}en (N. 48). Eine gemeinsame Bewegungsrichtung nimmt Leibniz auch für den Fall an, dass zwei Körper nicht zentral zusammensto{\ss}en, und fragt, wie die daraus resultierende Richtung zu ermitteln sei (N. 49). Wie die Kraft des Sto{\ss}es zu bestimmen sei, versucht er anhand eines Gedankenexperiments zu klären, bei dem die Bewegung eines Pendels ein Gewicht hebt (N. 52). Mit dem flüssigen Medium (Äther), in dem sich alle Körper bewegen, erklärt Leibniz, dass die absolute Geschwindigkeit beim elastischen Sto{\ss} entscheidend sei, und führt das Phänomen des Sto{\ss}es auf die Kohäsion der Körper zurück (N. 51). An seine Exzerpte aus Edme Mariottes \textit{Traité de la percussion} (1673) schlie{\ss}en sich weitergehende Überlegungen zum Medium an; dieses sei Ursache dafür, dass sich jeder Körper im Verhältnis zu seiner Grö{\ss}e einer Bewegung widersetze. Beim Sto{\ss} müsse diese eigene Kraft oder Bewegung des Mediums überwunden werden; den Widerstand des Mediums bezeichnet Leibniz als \textit{frottement} oder \textit{detrimentum motus} (N. 50). Ausgehend hiervon wird sich Leibniz dem Phänomen der Reibung eigens zuwenden und intensiv untersuchen (N. 30 - N. 38). 
\par
\vspace{3.0ex}
\noindent
IV. Meteorologica (N. 53 - N. 54)
\vspace{1.0ex}
\par\noindent
Nur zwei Stücke haben geologische und atmosphärische Phänomene zum Inhalt, Themen also, die traditionell unter die Bezeichnung der aristotelischen Meteorologie fielen. Leibniz exzerpiert aus Francesco Travaginis Augenzeugenbericht über das Erdbeben, das 1667 Ragusa (das heutige Dubrovnik) erschütterte; die kosmologischen Schlussfolgerungen, die Travagini in seinem 1673 erschienenen Werk zieht, bleiben dabei aber unberücksichtigt (N. 53). Aus einem heute verlorenen Manuskript von René Descartes fertigt Leibniz Auszüge zu sehr verschiedenen Themen an; neben botanischen, chemischen, medizinischen und physikalischen Inhalten hält er insbesondere Beobachtungen und Überlegungen fest, die Descartes zur Entstehung von Hagel und Schneekristallen anstellt (N. 54). 
\par
\vspace{3.0ex}
\noindent
V. Physica (N. 55 - N. 57)
\vspace{1.0ex}
\par\noindent
Ein breites Spektrum naturwissenschaftlicher Themen bieten Auszüge, die Leibniz zu drei Autoren erstellt. Das längste Stück in VIII,2 bilden die umfangreichen Exzerpte, die Leibniz aus dem ersten Band der \textit{Physica} erstellt, den der Jesuit Honoré Fabri 1669 veröffentlichte (N. 55). Auch wenn sich die Auszüge nur auf einen kleinen Teil des Werkes beziehen, schenkt Leibniz den grundlegenden Prinzipien und Annahmen, die Fabri einleitend vorwegschickt, gro{\ss}e Aufmerksamkeit. Von den darauf folgenden zehn Traktaten, in die sich Fabris \textit{Physica} gliedert, exzerpiert Leibniz nur die ersten beiden, in denen es breit gefächert um Körper und ihre Beschaffenheit geht. Seine Aufmerksamkeit finden Themen der Bewegungslehre, der Optik und Bruchfestigkeit sowie physikalische Eigenschaften und Zustände, deren Phänomene, Ursachen, Wirkungen und Veränderungen Fabri zu erklären sucht. Daneben hält Leibniz einzelne Experimente und Beobachtungen (zur Pneumatik, zu Glas mit Einschlüssen, zur Quecksilbersäule) ausführlicher fest. Wiederum in Form von Exzerpten und eigenen Kommentaren setzt sich Leibniz mit den ebenfalls 1669 erschienenen \textit{Elementa physica} des Freiherrn Franz Wilhelm von Nylandt auseinander (N. 56). Auf sein Interesse sto{\ss}en zum einen Nylandts Überlegungen zur Rotationsbewegung und Huygens' Kritik daran, zum anderen die Aussagen zur Materie, die mit Begriffen wie Dichte und Vakuum, Undurchdringlichkeit und Elastizität sowie der unendlichen Teilbarkeit von Körpern verbunden sind. Des Weiteren notiert Leibniz sich Nylandts Sto{\ss}gesetze sowie verschiedene Faktoren (Form, Schwerpunkt, Auftreffwinkel, Art des Mediums), die beim Sto{\ss} von Einfluss seien. Auszüge zu sehr verschiedenen Themen erstellt Leibniz aus einem heute offenbar verlorenen Manuskript (\textit{Discours des causes de la pesanteur des corps et du ressort, et de leur dureté}), bei dem es sich um die frühere Fassung einer Abhandlung handeln dürfte, die Claude Perrault 1680 unter leicht geändertem Titel im ersten Band seiner \title{Essais de physique} veröffentlichte (N. 57). Leibniz exzerpiert daraus, wie Perrault auf korpuskularer Ebene Zusammensetzung und Eigenschaften von Körpern erklärt und deren Veränderung auf physikalisch-chemische Prozesse zurückführt, die bei der Metallverarbeitung und bei anderen Stoffen zu beobachten seien. Daneben interessiert sich Leibniz dafür, wie Perrault das Problem der Reibung bei Waagen gelöst haben will und wie er die Wirbelbewegung des Äthers darlegt; Perrault habe demnach angenommen, dass der Äther nicht anders als Dinge der täglichen Erfahrung einen Widerstand (\textit{répugnance}) gegen Bewegung zeige.  
\par
\vspace{3.0ex}
\noindent
VI. Anatomica (N. 58)
\vspace{1.0ex}
\par\noindent
Leibniz hatte spätestens seit dem 24. Februar 1676 Zugang zu dem heute verlorenen Nachlass René Descartes' und erstellte daraus umfangreiche Auszüge (N. 6, N. 54, N.~58, N. 76, N. 82), die er auch selbst als \textit{excerpta} bezeichnete. Entgegen dem Titel des Stücks stammen seine Auszüge zur Anatomie allerdings nicht aus einer einzelnen Handschrift, sondern beruhen auf einer Auswahl, die Leibniz aus Aufzeichnungen unterschiedlicher Inhalte und Entstehungszeiten (1631, 1637, 1648) trifft; dies bestätigt sich auch anhand einer Parallelüberlieferung, die eine andere Textfolge und einen anderen Textzusammenhang aufweist. Den jeweils ausgewählten Text schreibt Leibniz diplomatisch ab und kommentiert seine Vorlage sowohl formal als auch inhaltlich. Leibniz gibt Beobachtungen wieder, die Descartes beim Sezieren junger Kälber, von Schafen, Hühnern und Fischen macht und die sich auf die wichtigsten Organe, Blutgefä{\ss}e, Körperteile, Nervenbahnen und Sinnesorgane beziehen. Er hält fest, wie Descartes bei Rind und Huhn die embryonale Entwicklung in verschiedenen Stadien beschreibt. Daneben kopiert er Aufzeichnungen zur Physiologie, zu Wachstum und Ernährung von Tieren, Pflanzen und Menschen. Den anatomischen Zeichnungen, die im Editionstext nur schematisch wiedergegeben werden, sind Faksimiles zur Seite gestellt, so dass alle Details der handschriftlichen Zeichnung für eine Interpretation herangezogen werden können. 
\par
\vspace{3.0ex}
\noindent
VII. Botanica (N. 59 - N. 60)
\vspace{1.0ex}
\par\noindent
Sein Interesse an der belebten Natur schlägt sich in Exzerpten zu Paolo Boccones \mbox{\textit{Recherches et observations naturelles}} von 1674 (N. 59) sowie in weiteren Notizen und Auszügen nieder (N. 60). Leibniz hält vor allem Stellen bzw. Briefauszüge aus Boccones Werk fest, an denen von Korallen, ihren verschiedenen Arten und Eigenschaften berichtet wird; daneben ist von Schwämmen, Knochenfunden, Versteinerungen und Bezoarsteinen zu lesen; auf Pflanzen nach heutigem Verständnis beziehen sich seine Exzerpte zur Wurzelbildung und besonderen (geometrischen) Formen (N. 59). Welche Quellen Leibniz bei seinen weiteren Auszügen und Aufzeichnungen zur Botanik verwendet, ist nicht ganz klar (N. 60): Leibniz vermerkt Autorennamen und Naturalienkabinette; neben Notizen zu Mineralien geht es überwiegend um Pflanzen und Pilze, von deren Eigenschaften, Wirkungen und Verwendung berichtet wird. 
\par
\vspace{3.0ex}
\noindent
VIII. Chymica (N. 61 - N. 65)
\vspace{1.0ex}
\par\noindent
Einen gro{\ss}en Gewinn für die \textit{Chemia} verspricht sich Leibniz von einem alchemischen Ofen, der seine Temperatur selbst regulieren bzw. konstant halten kann; die Konstruktion, die er hier vorstellt, könnte auf Cornelius Drebbel zurückgehen (N. 62). Leibniz hält fest, wie sich Edelsteine unterschiedlicher Art herstellen lassen (N. 63) und Quecksilber eine goldene Farbe annimmt (N. 64). Leibniz notiert Buchtitel (N. 61) und Inhalte eines Gesprächs, das er mit Artus Gouffier de Roannez und anderen führt: Gegenstand sind die Entsalzung von Meerwasser und verschiedene Mittel, Flüssigkeiten zu filtrieren (N.~65). 
\par
\vspace{3.0ex}
\noindent
IX. Medica (N. 66 - N. 77) 
\vspace{1.0ex}
\par\noindent
Die Stücke dieser Rubrik sind zur Hälfte noch in Mainz entstanden. Bereits in dieser Zeit äu{\ss}ert Leibniz seine Begeisterung für Medizin und hält gro{\ss}e Fortschritte auf diesem Gebiet für möglich (N. 69, N. 70); weitere Themen sind Behandlungsmittel und -methoden (N. 66, N. 67, N. 71, N. 72, N. 73, N. 74, N. 75, N. 76, N. 77), anatomische und physiologische Erkenntnisse (N. 68, N. 76) sowie Berichte von Krankheitsfällen (N. 71, N. 75, N.~77). Leibniz ruft dazu auf, Anstrengungen zu unternehmen, die Medizin in Theorie und Praxis voranzubringen, hebt ihren gro{\ss}en Nutzen unter den Wissenschaften und den zu erwartenden Fortschritt hervor (N. 69). Im Anschluss hieran macht Leibniz eine Fülle von Vorschlägen, wie sich Medizin und Gesundheitswesen konkret verbessern lie{\ss}en. Diese Art Wunschliste formuliert er als Richtlinien (\title{Directiones}, N. 70). Sie betreffen die Organisation, Ausbildung und Ausstattung der Ärzteschaft sowie die Versorgung der Kranken und Gesunden. Leibniz wünscht sich neue Untersuchungsmethoden (Thermometer, Mikroskop), Behandlungsformen (Operation, Anästhesie, Injektionen, Bluttransfusion) und Diagnoseverfahren (anhand von Blut, Speichel, Schwei{\ss}, Atem, Geruch). Seine Forderungen umfassen eine allgemeine Sterbestatistik und eine verbindliche Obduktion aller Verstorbenen. Des Weiteren sollen Ärzte zur Dokumentation ihrer Diagnose und Behandlung sowie Kranke zu einer Art medizinischen Beichte verpflichtet werden. Er weist auf körperliche und seelische Faktoren für die Gesundheit hin und hebt die Bedeutung der Ernährungsweise hervor. Einen Fortschritt verspricht er sich von Experimenten an Tieren, um Krankheitsverläufe, Behandlungsformen und die Wirkung von Medikamenten untersuchen zu können. Aus Athanasius Kirchers \title{Magneticum naturae regnum} notiert sich Leibniz Heilmittel gegen Pest, Schlangenbiss und Epilepsie (N. 66). In seinem Exemplar der \textit{Idea praxeos medicae} von Franciscus De Le Boë merkt Leibniz Stellen an, die sich auf Ursache und Heilung verschiedener Krankheiten und gesundheitlicher Störungen beziehen (N. 67). In einem gemeinsamen Gespräch erzählt ihm Edme Mariotte von einem Mittel zum Gurgeln, mit dem er wiederholt Halsentzündungen kuriert habe, und liefert Leibniz die Rezeptur dafür (N. 72). Über mehrere Ecken erfährt Leibniz von einer blutstillenden Lösung zur inneren wie äu{\ss}eren Anwendung, deren Rezeptur er für unterschiedlich starke Dosen notiert (N. 73). Eine ganze Sammlung an Rezepturen gegen verschiedene Krankheiten exzerpiert Leibniz aus dem Manuskript eines unbekannten Buches (N. 74). Von Robert Boyle erfährt Leibniz über die Person und die Erfolge des Heilers Valentine Greatrakes (Greatrick), der durch Handauflegen Schmerzen genommen oder gelindert haben soll; neben weiteren Notizen zur Medizin (Verletzungen, Medikamente, Anatomisches betreffend) überliefert Leibniz aus einem Gespräch mit Boyle Nachrichten vermischten Inhalts (Personen, ferne Länder, Politisches, Kuriositäten) (N.~77). Aus anderen Gesprächen hält Leibniz Berichte fest, wie sich mittels Schröpfköpfen und chi\-rurgischer Eingriffe die Gicht behandeln oder gegen sie vorbeugen lasse; weiter berichtet er über Kreislaufschwächen und Synkopen, die in Paris vor allem unter Frauen verbreitet seien und womöglich aus der Verstopfung von Gefä{\ss}en herrührten, wie das Ergebnis einer Obduktion vermuten lasse (N. 75). Leibniz schreibt von Fällen, in denen Messer verschluckt wurden, und verteidigt diejenigen, die sich bei Fieber über ein ärztliches Trinkverbot hinweggesetzt hätten (N. 71). Wieder auf De Le Boë (vgl. N. 67) geht ein Teil derjenigen Schriften direkt oder indirekt zurück, aus denen Leibniz Kenntnisse über chemische Prozesse im Körper und verschiedene Körperflüssigkeiten zusammenfasst (N.~68). Für physiologische Vorgänge interessiert er sich auch bei weiteren Auszügen, die er aus dem Nachlass Descartes' erstellt: Hier nimmt er Überlegungen und Beobachtungen auf, die Descartes zur Verdauung anstellt, notiert, wie bestimmte Stoffe und Nahrungsmittel  \par\newpage\noindent 
 verdauungsfördernd oder -hemmend wirkten und mit welchen Mitteln sich Darmverstopfung lösen lasse (N. 76). 
\par
\vspace{3.0ex}
\noindent
X. Miscellanea (N. 78 - N. 81)
\vspace{1.0ex}
\par\noindent
Unter den vermischten Themen finden sich so kuriose Stücke wie die Schriftprobe einer Frau ohne Hände (N. 78) und eine Einkaufsliste für Lebensmittel mit Nebenrechnungen zusammen mit einem Essensplan für die Woche (N. 80). Zum Bereich der Militaria gehört eine alphabetische Liste, die Leibniz aus verschiedenen Autoren erstellt, und die moderne militärische Fachausdrücke in drei Sprachen (Italienisch, Französisch, Deutsch) aufführt und jeweils die Bedeutung auf Lateinisch (teils auf Deutsch) angibt (N. 79). Aus einem Gespräch berichtet Leibniz, was ihm Claude Perrault (s.a. N. 57) von den Planungen erzählt, die der Neugestaltung des Louvre vorangingen, und wie Perrault selbst durch Vermittlung seines Bruders zu einem der Architekten des Umbaus wurde (N. 81); die Handschrift dieses Stückes ist verloren; als Vorlage dient ein Druck aus dem 19. Jh., der bis auf eine Ergänzung und ohne die sonst übliche Normalisierung des Französischen wiedergegeben wird. 
\par
\vspace{3.0ex}
\noindent
XI. Nachträge zu Optica (N. 82) und Technica (N. 83 - N. 99) 
\vspace{1.0ex}
\par\noindent
Die Nachträge zum ersten Band der Reihe VIII betreffen fast ausschlie{\ss}lich die Rubrik Technik. Ausnahme hiervon ist ein Stück zur Optik. Darin exzerpiert Leibniz wieder aus Descartes' Nachlass (s.a. N. 6, N. 54, N. 58, N. 76). In dem heute verlorenen Manuskript macht Descartes quantitative und qualitative Angaben zur Lichtbrechung in verschiedenen Medien und gibt eine Berechnungstabelle aus Witelos \title{Perspectiva} wieder, die Leibniz mit Korrekturen übernimmt (N. 82). 
\par
Die Nachtragsstücke zur Technik lassen sich in vier Gruppen gliedern: (1) Perpetuum mobile, (2) Mechanismen für Uhrwerke, (3) Strömungsmechanik und Wasserbau, (4) Verschiedenes.
\par
(1) Bereits 1671 (s. VIII,1 N. 59) wollte Leibniz ein Perpetuum mobile bauen. Auch in seiner Pariser Zeit hielt er dies anfangs noch für möglich und liefert 1674 drei weitere Pläne dafür (N. 92, N. 93, N. 94): Ein Entwurf beruht darauf, Röhren mit beweglichen Gewichten, die teils innen laufen, teils au{\ss}en schwingen können, drehbar aufzuhängen (N. 92). Ein weiterer Entwurf sieht vor, dass auf einer drehbaren Scheibe rechtwinklig zueinander vier versiegelte Röhren gefüllt mit Flüssigkeit und Stahlkügelchen angebracht sind, an deren Ende jeweils ein drehbar aufgehängter Magnet sitzt und ein weiterer in der Achse der Scheibe; Leibniz weist aber zugleich auf Schwierigkeiten bei der technischen Umsetzung hin, die u.a. auf das Phänomen der Reibung zurückzuführen seien (N. 93). Bei der dritten Ausführung handelt es sich nur mehr dem Titel nach um ein Perpetuum mobile. Das hier beschriebene Windrad kommt nicht ohne wiederholte Energiezufuhr aus: Bei Wind hebt ein horizontales Windrad mittels zahlreicher Zahnradübersetzungen ein Gewicht, das bei Windstille für eine möglichst lange fortgesetzte Bewegung sorgt und bei erneutem Wind Bewegungsenergie weiter speichert, indem es wieder angehoben wird (N. 94). 
\par
(2) Leibniz beschreibt verschiedene Mechanismen zur Erzeugung einer gleichförmigen Bewegung, die sich für sein Perpetuum mobile, wie er es 1671 entworfen hat (LSB VIII,1 N. 59), oder für eine unentwegt laufende Uhr nutzen lie{\ss}e (N. 83). Wie ein fortgesetzt laufendes und genau gehendes Uhrwerk zu konstruieren sei, beschäftigt Leibniz weiter und er hofft, hierfür die Kraft der Elastizität nutzen zu können: Sie würde sich technisch aus einer Vielzahl dehnbarer und miteinander kommunizierender Ringe erzielen lassen, auf deren Grundlage Leibniz auch wiederum einen \textit{motus perpetuus} für möglich hält (N.~85). Einer anderen Überlegung zufolge lie{\ss}e sich eine gleichmä{\ss}ige Bewegung aus periodisch aufeinanderfolgenden Zuständen der Ver- und Rückformung elastischer Körper gewinnen, für die Leibniz bestimmte Eigenschaften fordert; eine zentrale Funktion bei dieser elastischen Bewegung kommt den Auslösermechanismen (\textit{detentacula}, \textit{detentes}) zu, die nicht näher geklärt werden (N. 89). Um eine genaue Zeitmessung ganz anders zu erzeugen, beruft sich Leibniz auf eine Idee von Isaac Vossius: Ein Pendel soll durch kontinuierlich fallende Tropfen oder Körner in Schwingung versetzt werden, wobei die Tropfen oder Körner in einen Messbehälter fallen, an dem sich die Zahl der Schwingungen ablesen lasse (N. 84). In Auseinandersetzung mit Christiaan Huygens fragt Leibniz, unter welchen Bedingungen sich eine isochrone Pendelschwingung, wie sie für Uhren nötig ist, aufrechterhalten lässt (N. 91). Daneben beschäftigt sich Leibniz in N. 85 und eigens in einem fragmentarischen Stück (N. 86), wie ein Mechanismus bestehend aus verschieden schnell laufenden Rädern in all seinen Teilen eine konstante Geschwindigkeit erzeugt, so dass sich für das Gesamtsystem eine gleichförmige Bewegung ergibt, die sich für einen genauen Gang von Uhren nutzen lie{\ss}e. Leibniz hält es auch für möglich, gleichförmig laufende Uhrwerke zu realisieren, indem Magneten anstelle von Pendeln oder elastischen Teilen zum Einsatz kommen (N. 87) oder mit letzteren zusammen eine Unruhe verwendet wird (N. 90).
\par
(3) Ausgangspunkt für seine Überlegungen zu Strömungsmechanik und Wasserbau sind Gespräche mit Artus Gouffier de Roannez (N. 97), der auch in N. 65 Gesprächspartner ist. In kritischer Auseinandersetzung mit Roannez will Leibniz klären, inwieweit die Flie{\ss}geschwindigkeit eines Flusses vom Gefälle und von der Tiefe des Wassers abhängt (N. $97_1$), wobei er auch das Phänomen der Reibung (\textit{frottement}) in seine Überlegungen mit einbezieht. Roannez berichtet ihm von wasserbaulichen Schutzanlagen entlang von Flüssen, bei denen Leibniz interessiert, wie dadurch technisch Überflutungen verhindert werden oder deren Zerstörungskraft verringert wird (N. $97_2$, N. $97_3$). Ebenfalls durch Roannez erfährt Leibniz von einem Instrument, mit dem sich die Flie{\ss}geschwindigkeit von Gewässern messen lasse, und überlegt, wie sich vergleichbare Messungen auf dem Meer anstellen lie{\ss}en, um die Geschwindigkeit des Windes, der Strömung und des Schiffes selbst daraus zu ermitteln (N. $97_4$).
\par
(4) Neben grö{\ss}eren Themen sind Stücke zu verschiedenen Gegenständen der Technik überliefert, darunter auch blo{\ss}e Zeichnungen (N. 95). Eigens notiert Leibniz, wie sich Wind durch künstlich hergestellte Temperatur- und Feuchtigkeitsgefälle in der Luft erzeugen lasse (N. 88). Daneben hält er verschiedene Einfälle zu technischen Verbesserungen und Lösungen (z.B. zum Heizen, Kochen, Wasserpumpen) fest (N. 99). In einem Nachtragsstück zur Pneumatik (s. VIII,1 N. 36 - N. 55) berechnet Leibniz die Luftmenge, die aus einem Behälter herausgepumpt wird; seine diesbezügliche Berechnung von 1675 korrigiert er 1678 in einem eigenhändig datierten Zusatz (N. 96). Nicht ohne kritische Anmerkungen exzerpiert Leibniz aus einem heute verlorenen Manuskript Ole R{\o}mers dessen Überlegungen zu Zahnrädern, insbesondere wie das Übersetzungsverhältnis einzelner Paarungen zu bestimmen sei (N. 99).
\par
\vspace*{2em}
\hspace{95mm}Harald Siebert
