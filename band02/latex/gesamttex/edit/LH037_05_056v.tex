\begin{ledgroupsized}[r]{120mm}
\footnotesize 
\pstart 
\noindent\textbf{\"{U}berlieferung:}
\pend
\end{ledgroupsized}

\begin{ledgroupsized}[r]{114mm}
\footnotesize 
\pstart \parindent -6mm
\makebox[6mm][l]{\textit{L}}Aufzeichnung:
LH XXXVII 5 Bl. 56.
1 Bl. 4\textsuperscript{o}.
\nicefrac{4}{5} S. auf Bl.~56~v\textsuperscript{o}.
Bl. 56~r\textsuperscript{o} und das unterste Fünftel von Bl. 56~v\textsuperscript{o} überliefern N.~17\textsubscript{1}.
% = LH037_05_056v (Sur les petites 1)
Ein Wasserzeichen, beschnitten.
Papier durch Erhaltungsma{\ss}nahmen stabilisiert.%
\newline%
Cc 2, Nr. 975 B%
%%Konzept: LH XXXVII 5 Bl. 56. 1 Bl. 4\textsuperscript{o}. Papier durch Erhaltungsma{\ss}nahmen stabilisiert. \nicefrac{4}{5} S. auf Bl. 56~v\textsuperscript{o}. Auf Bl. 56~r\textsuperscript{o} und \nicefrac{1}{5} S. auf Bl. 56~v\textsuperscript{o} das St\"{u}ck N. ??
%% = LH037_05_056v (Sur les petites 1)
%\\Cc 2, Nr. 975B
\pend
\end{ledgroupsized}
%%\normalsize
%\vspace*{5mm}
%\begin{ledgroup}
%\footnotesize
%\pstart
%\noindent\footnotesize{\textbf{Datierungsgr\"{u}nde}: Siehe oben die Einleitung zu N. ??Intro??.}
%\pend
%\end{ledgroup}

\count\Bfootins=1200
	\count\Afootins=1200
\vspace*{8mm}
\pstart\normalsize\noindent%
[56~v\textsuperscript{o}]
% \selectlanguage{french}
Si vous demandez la longueur d'un pendule,\protect\index{Sachverzeichnis}{pendule}
qui fasse un certain nombre de batte\-ments dans un certain temps,
\edtext{par exemple dans un quart d'heure;}{\lemma{par [...] d'heure}\Bfootnote{\textit{erg. L}}}
vous la pourrez trouver ainsi:%
\pend%
\pstart %
Prenez
\edtext{une pendule,}{\lemma{une}\Bfootnote{\textit{(1)}\ autre pendule, dont \textit{(2)}\ pendule, \textit{L}}}
\`{a} discretion, mesurez sa longueur; et contez combien de batte\-ments elle fait dans le m\^{e}me temps susdit, par exemple dans un quart d'heure.
\pend% 
\pstart%
A present pour s'expliquer plus aisement, appellons le nombre des battements de  la pendule, prise \`{a} discretion, \textit{$(A)$} et le nombre des battements  demand\'{e}, de la pendule dont nous cherchons la longueur, \textit{$(B)$}  et la longueur de la pendule prise \`{a} discretion, \textit{$(C)$}  et enfin la longueur de la pendule demand\'{e}e, \textit{$(D)$}.
Cela estant pos\'{e}, l'operation sera telle.
\pend%
\vspace*{2.0em}%
\pstart\noindent%
\lbrack%
\textit{Nachfolgend klein gedruckter Text gestrichen:}%
\rbrack %
\pend%
\vspace*{0.5em}
\pstart\noindent\footnotesize%
Des deux nombres, $(A)$ et $(B)$ divisez le plus grand par le moindre;
et multipliez le quotient par luy m\^{e}me, ou (ce qui est  la m\^{e}me chose) prenez le quarr\'{e} du dit quotient:
appellons le dit quarr\'{e}, $(E)$.%
\pend%
\pstart\footnotesize%
Enfin faites l'operation suivante de la regle des
\edtext{trois;%
\newline%
Si le nombre quarr\'{e} $(E)$,}{\lemma{trois;}\Bfootnote{%
\textit{(1)}\ Comme le nombre quarr\'{e} $F$, %
\textit{(2)}\ Si le nombre quarr\'{e} $(E)$ \textit{L}}}
donne l'Unit\'{e}; combien%
\pend%
\vspace*{1mm}%
\pstart\footnotesize%
$\displaystyle\frac{A}{B}$\hspace*{2mm}$\displaystyle\frac{A^{2}}{\underline{\underline{{B}^{2}}}}$\hspace*{2mm}$\displaystyle\frac{C}{D}$
\pend%
\pstart%
\vspace{0,8mm}\hspace*{6mm}$r$
\pend%
\vspace*{1.5em}%
\pstart\normalsize\noindent%
Multipliez le nombre $A$ par luy m\^{e}me, ou (:~ce qui est la m\^{e}me chose~:) prenez son quarr\'{e};
de m\^{e}me multipliez le nombre $B$ par luy m\^{e}me, ou prenez son quarr\'{e};
et enfin faites une telle operation de la regle des trois:
\pend%
\newpage%
\pstart%
\edtext{Si le quarr\'{e} du nombre $A$ des battements}{\lemma{Si le}\Bfootnote{%
\textit{(1)}\ nombre %
\textit{(2)}\ quarr\'{e} du nombre %
\textit{(a)} des battements $A$ %
\textit{(b)}\ $A$ des battements \textit{L}}}
de la pendule prise \`{a} discretion, donne la longueur $C$, de sa pendule.
\pend%
\pstart%
Combien donnera le quarr\'{e} du Nombre
\edtext{donn\'{e}}{\lemma{donn\'{e}}\Bfootnote{\textit{erg. L}}}
$B$ des battements de la pendule demand\'{e}e,
pour la longueur $D$, de la dite pendule.
\pend%
\pstart%
Et le produit de cette operation, vous donnera la dite longueur $D$, que vous aviez demand\'{e}e.
\selectlanguage{latin}
\pend
\count\Bfootins=1500
	\count\Afootins=1500

