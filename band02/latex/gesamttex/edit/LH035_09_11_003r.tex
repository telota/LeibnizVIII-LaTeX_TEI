\begin{ledgroupsized}[r]{120mm}
\footnotesize
\pstart
\noindent\textbf{\"{U}berlieferung:}
\pend
\end{ledgroupsized}
%
\begin{ledgroupsized}[r]{114mm}
\footnotesize
\pstart \parindent -6mm
\makebox[6mm][l]{\textit{L}}Reinschrift mit Verbesserungen und Ergänzungen:
LH XXXV 9, 11 Bl. 3-4.
1 Bog. 2\textsuperscript{o}.
3~S. Bl. 4~v\textsuperscript{o} leer.
Zeichnungen [\textit{Fig. 1}] und [\textit{Fig. 2}] stark \"{u}berarbeitet.
Ein jeweils verschiedenes Wasserzeichen auf Bl.~3 und Bl.~4.
Der Text wird editorisch in zwei Teile unterteilt, die als verschiedene Redaktionsstufen zu deuten sind.%
 \\Cc 2, Nr. 1189 B, D-G
\pend
\pstart \parindent -6mm
\makebox[6mm][l]{\textit{E}}(tlw.) \cite{00188}\textsc{H.-J. Hess}, \glqq Die unver\"{o}ffentlichten naturwissenschaftlichen und technischen Arbeiten von G.W. Leibniz aus der Zeit seines Parisaufenthaltes. Eine Kurzcharakteristik\grqq, \textit{Studia Leibnitiana. Supplementa} XVII (1978), S. 183-217: S. 206-210.
\pend
\end{ledgroupsized}
%%
\vspace*{8mm}
\pstart
\normalsize
\noindent
[3~r\textsuperscript{o}]
Hyeme 1675
\pend
\pstart
\begin{center}
DU FROTTEMENT.\\
Essais Geometriques en fait de Mechanique.
\end{center}	
\pend
\vspace*{1em}
\pstart
\centering
[\textit{Teil 1}]
\pend
\pstart
\noindent
Les Geometres n'ont pas encor donn\'{e} des regles sur cette matiere,
et ceux qui ont fait des traitez de Mechanique n'en parlent qu'en passant,
et pour la renvoyer \`{a} l'experience des ouuriers.
Il est constant toutesfois que souuent des projects bien conceus ont avort\'{e} \`{a} cause de la perte de la force mouuante,
dont une grande partie avoit est\'{e} employ\'{e}e \`{a} surmonter le
frottement\protect\index{Sachverzeichnis}{frottement}
des pieces de la
machine\protect\index{Sachverzeichnis}{machine}.
On s\c{c}ait que les machines qui servent \`{a} lever des grands fardeaux,
les pompes\protect\index{Sachverzeichnis}{pompe},
les chariots et autres voitures y sont interess\'{e}es,
et on a cherch\'{e} et trouu\'{e} depuis peu quelques inventions propres \`{a} eviter ou diminuer cette perte:
Mons. \edtext{Perrault\protect\index{Namensregister}{\textso{Perrault} (Perraltus), Claude 1613-1688}}{\lemma{Perrault}\Cfootnote{\cite{01014}\textsc{Vitruvius}, \textit{Les dix livres d'architecture}, hrsg. von C. \textsc{Perrault}, Paris 1673, l. X, ch. V, S. 280f. und 324f. Keine der dort beschriebenen Maschinen wird allerdings \textit{barulcus} genannt. F\"{u}r diesen auf Heron von Alexandria zur\"{u}ckgehenden Begriff siehe vielmehr \cite{01015}\textsc{Pappus}, \textit{Mathematica collectio},
l. VIII, probl. VI, prop. X.}}
a publi\'{e} dans son
Vitruue\protect\index{Namensregister}{\textso{Vitruvius Pollio}, Marcus ca. 70-10 v. Chr.}
une espece de Machine \`{a} lever des fardeaux ou\textso{ Barulcum,}
o\`{u} il n'y a quasi point de frottement.
\edtext{On a present\'{e}}{\lemma{On a present\'{e}}\Cfootnote{Quelle nicht nachgewiesen.}}
\`{a} l'Academie Royale une\textso{ pompe }tres ingenieuse,
o\`{u} le principe de
Torricelli\protect\index{Namensregister}{\textso{Torricelli} (Torricellius), Evangelista 1608-1647}
est appliqu\'{e} \`{a} la m\^{e}me fin.
La pens\'{e}e de \edtext{celuy}{\lemma{celuy}\Cfootnote{Quelle nicht nachgewiesen.}}
qui a fait faire des\textso{ chariots }qui se fournissent eux m\^{e}mes des planches pour marcher l\`{a} dessus doucement,
n'a pas est\'{e} mauvaise:
Et je croy qu'on trouuera avec le temps de semblables remedes pour quelques autres mouuements.
Cependant l'estime de la perte faite par le frottement ne laisse pas d'estre utile,
et sans \edtext{parler de la figure des vaisseaux}{\lemma{parler}\Bfootnote{\textit{(1)}\ des vai \textit{(2)}\ de la figure des vaisseaux \textit{L}}}
qui marchent dans de l'eau avec quelque difficult\'{e},
il est constant que les corps jettez\protect\index{Sachverzeichnis}{corps jet\'{e}}
sont retardez notablement par la resistence de l'air\protect\index{Sachverzeichnis}{r\'{e}sistance de l'air}:
et comme il y a de l'apparence que les hommes trouueront un jour des regles assez justes pour
\edtext{[la]}{\lemma{}\Bfootnote{la \textit{erg. Hrsg.}}}
donner dans un point propos\'{e},
il est ais\'{e} de juger,
que ce ne sera qu'apr\`{e}s que le frottement sera reduit en regles:
quoyque cependant un long usage des personnes qui s'y sont exerc\'{e}es d\`{e}s leur jeunesse,
puisse suppl\'{e}er \`{a} ce defaut.
\pend
\vspace*{3mm}% PR: Abstand provisorisch. Hier fängt ein neuer Abschnitt an.
\begin{Geometrico}
% PR: Erste Zeile bitte hängend (more geometrico).
Le\textso{ Frottement }est la resistence\protect\index{Sachverzeichnis}{r\'{e}sistance}
du lieu\protect\index{Sachverzeichnis}{resistance du lieu} par o\`{u} le mobile\protect\index{Sachverzeichnis}{mobile}
passe.\\% PR: Folgenden Absatz bitte links ganz einrücken. 
J'entends par le\textso{ Lieu }la surface du corps ambient ou environnant,
(:~entierement ou en partie~:)
comme \edtext{Aristote\protect\index{Namensregister}{\textso{Aristoteles}, 384-322 v. Chr.}}{\lemma{Aristote}\Cfootnote{\cite{00235}\textit{Phys.} IV 4, 212a2-30.}} l'a defini.
\end{Geometrico}
\pstart
\noindent% PR: Diesen Absatz bitte gar nicht einrücken.
Cette\textso{ Resistence }se fait par la complication de deux causes,
et c'est pourquoy elle est aussi de deux especes,\textso{ absolue,} ou\textso{ respective.}
Je veux traiter icy de la resistence absolue\protect\index{Sachverzeichnis}{r\'{e}sistance absolue},
et je me reserve de parler de la respective\protect\index{Sachverzeichnis}{r\'{e}sistance respective}
dans un
\edtext{autre cahier,}{\lemma{autre cahier}\Cfootnote{Vermutlich N. 35.}}
o\`{u} j'expliqueray la difference qu'il y a entre ces deux resistences,
et leurs origines.
\pend
%\newpage
\pstart
\vspace{1.5em} 
\begin{center} 
Premiere section\\
De la Resistence absolue, qui se trouue\\dans le frottement et qui est tousjours la m\^{e}me quelque vitesse que le mobile\\puisse avoir
\end{center} \pend
\begin{Geometrico}
% PR: Erste Zeile bitte hängend (more geometrico).
\textso{%
Acceleration\protect\index{Sachverzeichnis}{acc\'{e}l\'{e}ration}}\textso{
ou Retardation\protect\index{Sachverzeichnis}{retardation}}\textso{
\'{e}gale selon les lieux}\edtext{\textso{
[temps]}}{\lemma{\textso{[temps]}\! }\Cfootnote{Die eckigen Klammern stammen von Leibniz.}}\textso{
}est une addition ou soubstraction continuelle,
d'un \edtext{m\^{e}me degr\'{e}}{\lemma{m\^{e}me}\Bfootnote{\textit{(1)}\ degrez \textit{(2)}\ degr\'{e} \textit{L}}}
de \edtext{vitesse\protect\index{Sachverzeichnis}{degr\'{e} de vitesse}
en}{\lemma{vitesse}\Bfootnote{\textit{(1)}\ \`{a} \textit{(2)}\ en \textit{L}}}
chaque point du lieu
\edtext{\lbrack\`{a} chaque moment du temps\rbrack}{\lemma{\lbrack\`{a} chaque moment du temps\rbrack}\Cfootnote{Die eckigen Klammern stammen von Leibniz.}}.
\end{Geometrico}
\count\Bfootins=1200
\count\Cfootins=1200
\pstart
\noindent% PR: Diesen Absatz bitte gar nicht einrücken.
Celle qui est selon les temps a
\edtext{est\'{e} employ\'{e}e}{\lemma{est\'{e}}\Bfootnote{\textit{(1)}\ appliqu\'{e}e \textit{(2)}\ employ\'{e}e \textit{L}}}
par \edtext{Galilei\protect\index{Namensregister}{\textso{Galilei} (Galilaeus, Galileus), Galileo 1564-1642}}{\lemma{Galilei}\Cfootnote{\cite{00050}\textit{Discorsi}, Leiden 1638, S. 157f. und 163-165 (\cite{00048}\textit{GO} VIII, S. 197f. und 202-204).}}
\`{a} l'explication de la descente des corps pesans:
Mais celle qui se fait selon les lieux n'a pas encor est\'{e} reduite au calcul,
\`{a} ce que j'en ay p\^{u} apprendre:
Quoyque \edtext{plusieurs}{\lemma{plusieurs}\Cfootnote{Vermutliche Anspielung auf \cite{01022}\textsc{P. Le Cazre}, \textit{Physica demonstratio}, Paris 1645. Leibniz' eigenhändige Randbemerkungen befinden sich in seinem Handexemplar von Le Cazres\protect\index{Namensregister}{\textso{Le Cazre} (Cazreus), Pierre 1589-1664} Abhandlung; siehe N. 13.}}
l'ayent cr\^{u} preferable \`{a} celle de Galilei\protect\index{Namensregister}{\textso{Galilei} (Galilaeus, Galileus), Galileo 1564-1642}
pour expliquer m\^{e}me la dite descente.
Je ne suis pas de leur opinion,
et il me suffit, de la pouuoir appliquer au frottement.
\pend
\pstart% PR: Normal einrücken, bitte.
Theoreme I.
\pend
\pstart
\sloppy
\noindent% PR: Diesen Absatz bitte gar nicht einrücken.
\textso{%
Un corps dont le mouuement est uniforme en soy m\^{e}me estant retard\'{e} \'{e}galement \`{a} chaque endroit du lieu o\`{u} il passe,
les vistesses\protect\index{Sachverzeichnis}{vitesse}}\textso{
residues sont entre elles, comme les espaces qui restent \`{a} parcourir.}\edtext{}{\lemma{}\Afootnote{\textit{\"{U}ber der Zeile:} Haec melius enuntianda.}}
\pend
\pstart
\noindent% PR: Diesen Absatz bitte gar nicht einrücken.
Dans la\edtext{\textso{ I. fig. }}{\lemma{\textso{I. fig.}\! }\Cfootnote{Siehe [\textit{Fig. 2}].}}soit un mobile $\displaystyle M$ qui parcoureroit la ligne $\displaystyle EA$ avec la vistesse uniforme represent\'{e}e par $\displaystyle EG$,
et par consequent avec un mouuement,
qui seroit represent\'{e} tout entier par $\displaystyle EG$
appliqu\'{e}e \`{a} tous les points $\displaystyle B.$ $\displaystyle (B)$ etc.
de la dite ligne $\displaystyle EA$ ou par le rectangle $\displaystyle GEA$,
si chaque point $\displaystyle B.$ ou $\displaystyle (B)$ etc.
ne diminuoit \'{e}galement la vitesse du mobile.
Donc les vistesses d\'{e}croissant \'{e}galement jusqu'au repos dans $\displaystyle A$;
celles qui resteront en chaque point $\displaystyle B.$ $\displaystyle (B)$ etc.
seront represent\'{e}es par les appliqu\'{e}es du
\edtext{Triangle $\displaystyle GEA$,}{\lemma{Triangle $\displaystyle GEA$}\Cfootnote{Bei der gleichf\"{o}rmigen Bewegung von $\displaystyle M$ bezeichnet $\displaystyle GEA$ ein in [\textit{Fig. 2}] nicht gezeichnetes Viereck; bei der gleichf\"{o}rmig verz\"{o}gerten Bewegung von $\displaystyle M$ bezeichnet $\displaystyle GEA$ das gezeichnete gleichnamige Dreieck.}}
s\c{c}avoir par $\displaystyle CB$ ou $\displaystyle (C)(B)$ etc.
paralleles \`{a} la base $\displaystyle EG$.
Or $\displaystyle CB.$ $\displaystyle (C)(B)$ sont
comme $\displaystyle AB.$ $\displaystyle A(B)$ espaces qui restent \`{a} parcourir.
Donc les vistesses residues sont comme les espaces qui restent \`{a} parcourir.
\pend
\pstart% PR: Normal einrücken, bitte.
Theoreme II.
\pend
\pstart
\sloppy
\noindent% PR: Diesen Absatz bitte gar nicht einrücken.
\textso{Les m\^{e}mes conditions estant pos\'{e}es,
le temps employ\'{e} croist \`{a} chaque endroit de l'espace
en raison reciproque des espaces qui restent \`{a} parcourir.}
\pend
\count\Bfootins=1200
\count\Cfootins=1200
\pstart
\noindent% PR: Diesen Absatz bitte gar nicht einrücken.
Car generalement les accroissemens du temps en chaque endroit du lieu
sont en raison reciproque des vitesses que le mobile y a,\textso{
par le lemme suivant,}
or icy ces vitesses sont en raison des espaces qui restent \`{a} parcourir,\textso{
par le th. I.}
Donc les dits accroissemens du temps seront en raison reciproque des dits espaces.
[3~v\textsuperscript{o}]
\pend 