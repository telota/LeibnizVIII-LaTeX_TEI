\footnotesize 
\pstart 
\noindent               
Bei den folgenden St\"{u}cken N.~36\textsubscript{1} und 36\textsubscript{2} handelt es sich um zwei verschiedene Bearbeitungen eines Textes,
welcher die Resultate der vorhergehenden Untersuchungen \"{u}ber das mechanische Ph\"{a}nomen der Reibung darstellen soll.
Der Zusammenhang der zwei St\"{u}cke ist eindeutig:
Der erste Teil von N.~36\textsubscript{2} ist eine leicht ver\"{a}nderte Abschrift des ersten Teils von N.~36\textsubscript{1}.
Die einleitenden \"{U}berlegungen zur praktischen Bedeutung einer Berechnung der Reibung kn\"{u}pfen dort vornehmlich an N.~32 an;
die Aufstellung der Theoreme \"{u}ber die gleichm\"{a}{\ss}ige Verz\"{o}gerung eines sich in einem widerstehenden Medium bewegenden K\"{o}rpers geht indessen unmittelbar auf N.~34 zur\"{u}ck.
Der zweite sowie der dritte Teil von N.~36\textsubscript{1} entfallen in N.~36\textsubscript{2}
(der zweite Teil von N.~36\textsubscript{1} ist allerdings die unmittelbare Vorlage von N.~37).
Das St\"{u}ck N.~36\textsubscript{2} tr\"{a}gt Leibniz' eigenh\"{a}ndigen Vermerk \textit{Hyeme 1675} und weist dasselbe Wasserzeichen wie N.~36\textsubscript{1} auf.
Daher wird N.~36 insgesamt auf das Ende 1675 datiert.
\newline%
\indent%
Der zweite Teil von N.~36\textsubscript{2} muss allerdings nach dem Pariser Aufenthalt entstanden sein.
Leibniz bemerkt dort nämlich, dass die bewegende Kraft eines Körpers nicht nach dessen bloßer Geschwin\-dig\-keit einzuschätzen sei,
sondern nach dem Quadrat seiner Geschwindigkeit (S.~\refpassage{35,09,11_004r_vis-1}{35,09,11_004r_vis-2}).
Eine solche Bemerkung kann erst nach Januar 1678 verfasst worden sein
(siehe \cite{01056}\textsc{G.W. Leibniz}, \textit{La réforme de la dynamique: De corporum concursu (1678) et autres textes inédits}, hrsg. von \textsc{M.~Fichant}, Paris 1994).
\newline%
\indent%
In seinem 1689 in den \cite{01023}\textit{Acta eruditorum} (Januarheft, S.~38-47) veröffentlichten
\cite{01024}\glqq Schediasma de resistentia medii et motu projectorum gravium in medio resistente\grqq~%
erwähnt Leibniz,
er habe während seines Aufenthaltes in Paris\protect\index{Ortsregister}{Paris} der königlichen Akademie der Wissenschaften ähnliche Überlegungen mitgeteilt
(\cite{01025}\textit{LMG} VI, S.~136).
Und in einer handschriftlichen Vorlage zum \glqq Schediasma\grqq,
welche die Überschrift \cite{01026}\textit{De resistentia medii absoluta} trägt,
vermerkt er:
\textit{Haec jam pleraque habentur in scheda mea Parisiis\protect\index{Ortsregister}{Paris} scripta 1675 et Academiae Regiae communicata.}
(LH XXXV 9, 5 Bl.~26~r\textsuperscript{o})
Die Vermutung liegt nahe, dass es sich bei den St\"{u}cken N.~36\textsubscript{1} und 36\textsubscript{2} um die von Leibniz erwähnte \textit{scheda} handelt oder um Vorlagen dazu.
Es gilt allerdings zu bemerken,
dass sich ein ausdrücklicher Hinweis auf Leibniz' Mitteilung
weder in \cite{00295}\textsc{B. de Fontenelle}, \textit{Histoire de l'Académie Royale des Sciences}, Paris 1733,
noch in \cite{01027}\textsc{J.B. du Hamel}, \textit{Regiae Scientiarum Academiae Historia}, 2. Ausgabe, Paris 1721,
finden lässt.
Siehe hierzu \cite{01028}\textsc{E.J. Aiton}, \glqq Leibniz on Motion in a Resisting Medium\grqq, \textit{Archive for History of Exact Sciences} 9 (1972), S.~285, Anm.~9.
\pend
\newpage
\normalsize