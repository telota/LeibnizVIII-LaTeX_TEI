% [7~v\textsuperscript{o}]
\pstart%
Manifeste observavi plexus choroides non adhaerere ventriculis, sed instar
\edtext{[tapetorum]}{\lemma{tapetiorum}\Bfootnote{\textit{L \"{a}ndert Hrsg.}}}
esse ibi appensos, et quidem circa glandulam pinealem, ex qua
\edtext{[conopei]}{\lemma{onopei}\Bfootnote{\textit{L \"{a}ndert Hrsg.}}}
instar pendent et tegunt foramen cerebri, quod infundibulum excipit, adeo ut spiritus ascendentes per hoc infundibulum ex glandula quam pituitariam vocant, ad pinealem inde perveniunt, modo sint satis fortes. Sin minus reflectuntur 1\textsuperscript{mo} versus ventriculum 4\textsuperscript{um} per canalem qui est infra nates, deinde versus foramen quod est post nervorum opticorum occursum; unde elabuntur ex cerebro.
Easdem etiam vias sequuntur partes eorum superfluae, cum sunt satis fortes; et praeterea ex ventriculis versus nates purgantur. Quippe notavi accurate unam glandulam alteri superponi, infundibulum plane esse ejusdem substantiae atque arterias carotides quae ipsi insident.
\pend%
\newpage
\pstart %
Cum venae omnes
\edtext{in vitulo cujus caput ita percusserant, mactando,
ut ossa ab invicem in sutura [lambdoides] essent disjuncta}{%
\lemma{in vitulo [...] percusserant,}\Bfootnote{%
\textit{(1)} cum %
\textit{(2)} mactando, [...] sutura %
\textbar\ lamboides \textit{ändert Hrsg.} \textbar\ %
essent disjuncta \textit{erg. L}}}
et \edtext{nares}{\lemma{nares}\Cfootnote{%
Die Lesung der Handschrift ist eindeutig.
Dem Sinn nach könnte eher \textit{nates} gemeint sein.
Vgl. \cite{01192}\textsc{A. Bitbol-Hespériès}, a.a.O., S.~165.}} et spatium inter piam matrem et cerebrum et plexus choroides multo sanguine concreto implerentur;
nullus fuit in carotidibus nec in isto infundibulo, nullusque in ventriculis praeterquam circa glandulam pinealem ubi plexus choroides.
Post concursum nervorum opticorum adhuc patebat via, per quam spiritus ex ventriculis egredi possent; licet ibi etiam circumcirca mistus esset sanguis; canalis etiam sub natibus patebat, et membranula qua tegitur sursum erat evecta.
\pend%
\vspace{1.0em}% PR: Diesen leeren Zeilenabstand bitte behalten !!!
\pstart%
\noindent% PR: Neuer Abschnitt.
% \edtext{}{\lemma{}\Afootnote{\textit{Am Rand:} Novemb. 1637.}}%
Novemb. 1637. Vitulus e matrice excisus 5 vel 6 hebdomadis post conceptionem ut suspicor, erat indicis mei longitudine, a summo capite ad podicem plane formatus uteri cornua erant versus anteriorem partem reflexa. Vituli caput erat versus dextrum cornu%
\edtext{}{\lemma{}\Afootnote{\textit{Am Rand:} \Denarius\vspace{-4mm}}}
dorsum versus fundum matricis et umbilicus versus orificium; in quo umbilico quatuor vasa facile distinxi, quorum duo scilicet rubebant et alia duo magis nigrescebant, ita ut 2\textsuperscript{as} venas et 2\textsuperscript{as} arterias esse appareret, reliqua autem erant dia$\phi$ana.
Hujus longitudo mediam ipsius vituli longitudinem superabat.
Non autem erat ullo modo intortus, nisi forte aliquantulum videretur coepisse torqueri tanquam si caput foetus fuisset initio versus umbilicum venae, et inde versus dextrum latus se convertisset: postquam autem umbilici intestinum a foetu ad membranas
\edtext{illum}{\lemma{illum}\Bfootnote{\textit{erg. L}}}
investientes pervenerat in duas insignes partes dividebatur, in quarum unaquaque erat una vena, et una arteria, quae in plures ramos dividebantur, et unae versus dextram, aliae versus sinistram uteri partem se spargebant.
\pend%
\pstart%
Immisso deinde stylo satis crasso, nempe magnae aciculae caput in foramen nempe quod inter istas duas intestinuli partes apparebat, inveni ibi esse patentissimum meatum (urachum videlicet) qui tamen versus foetus umbilicum, angustior evadebat. Humor in uracho intestinuli contentus, magis lentus ac glutinosus videbatur, quam inter membranas erat.
\pend%
\newpage
\pstart%
Podex vituli nondum videbatur perforatus,
\edtext{sed apparebat}{\lemma{sed}\Bfootnote{\textit{(1)}\ erat \textit{(2)}\ apparebat \textit{L}}}
tamen puncti instar foraminis locus, ut in oculis palpebrarum fissurae rudimenta. Sed infra podicem apparebat tuberculum, quod initio pro scroto
\edtext{sumebamus admota}{\lemma{sumebamus}\Bfootnote{\textit{(1)}\ sed \textit{(2)}\ admota \textit{L}}} autem acicula vidi esse
\edtext{carunculam versus}{\lemma{}\Bfootnote{carunculam\ \textbar\ vidi esse carunculam \textit{gestr.}\ \textbar\ versus \textit{L}}}
caudam reflexam, ut $abc$ et intra istam flexuram esse rimam
\edtext{partem}{\lemma{}\Afootnote{\hspace*{-2mm}\textit{Oberhalb der Silbe} tem \textit{des Wortes} partem: \Denarius\ vam 
\newline%
\hspace{5,5mm}
\textit{Am Rand dazu:} \Denarius \vspace{-8mm}}}
quae caput minutae aciculae admittebat, et quam pro vulva foemellae accepi; erant etiam 4 mamillae formatae, ut in mare, quem \edtext{alias}{\lemma{alias}\Cfootnote{Siehe oben, S.~\pageref{mammae}.}} vidi.
Et suspicor in embryone scrotum semper humore aliquo distendi, qui humor si foris versus umbilicum reflectatur format membra masculi; si versus caudam format femellam; si utrinque herma$\phi$roditum.
Totus foetus nigricanti sanguine plenus erat unde judico magnum esse calorem sanguinis a quo formatur, nempe qui est tantum purissimus, qui per arterias matris accedat.
Oris anterior pars erat aperta, nondum autem posterior; item etiam nares nondum manifeste patebant, sed carunculae ex illis videbantur protuberare.
Adeo ut a materia intus contenta et egredi volente, debere aperiri appareret.
Humeri collum et caput paulo magis albebant quam crura, venter autem omnium maxime nigrescebat:
caput clunibus crassius erat, ventris autem regio erat crassissima:
aures videbantur esse aliquantulum perforatae, sed ab humore etiam egrediente ista autem foramina tegebantur extremitate auris triangularis figurae, quae a reliqua cute erat excisa.
\pend%
%\count\Bfootins=1500
%\count\Cfootins=1500
%\count\Afootins=1500
\pstart%
In hoc vitulo intestinum rectum ad finem usque videbatur esse perforatum, nam erat multo crassius jejuno, ut neque
\edtext{colon, nec}{\lemma{colon,}\Bfootnote{\textit{(1)}\ necque \textit{(2)}\ nec \textit{L}}}
caecum etiam crassius notavi.
Ventriculi autem tumebant, erantque aliquo humore glutinoso repleti, caro hepatis non erat firma, sed instar sanguinis concreti lienem non inveni, sed notavi aliquid ipsi simile valde exiguum a tergo ventriculi, quod prius pro hepatis parte sumebam non enim erat alius coloris.
Renes firmiter adhaerebant spinae erantque valde crassi et vicini vesicae, nec ullos ureteres notavi. Unde conjicio illos postea a faecibus in recto intestino et colo collectis, sursum propelli.
Vesica et urachus intra% Hier endet Bl. 7v.