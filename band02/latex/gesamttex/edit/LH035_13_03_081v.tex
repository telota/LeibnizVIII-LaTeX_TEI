\hspace*{-1.4mm}[81~v\textsuperscript{o}] comparata sunt, ut \edtext{ex posita causa plena, necessario sequatur}{\lemma{ut}\Bfootnote{\textit{(1)} alterum ex altero necessario sequatur \textit{(2)} ex \textit{(a)} posito effectu \textit{(b)} posita causa plena, necessario sequatur \textit{L}}} effectus integer. Est ergo causa plena, status omnium ad rem pertinentium simul sumtorum ad rem pertinentia voco, quae scilicet \edtext{agendo ad effectum}{\lemma{scilicet}\Bfootnote{\textit{(1)} agunt in aliquid \textit{(2)} agendo ad effectum \textit{L}}} contribuunt. Effectus autem integer, est status omnium ad rem pertinentium in aliquo tempore assignato posteriore; qui scilicet status ex \edtext{priore}{\lemma{ex}\Bfootnote{\textit{(1)} posteriore statu \textit{(2)} priore \textit{L}}} est consecutus; tametsi autem infinitae semper causae in natura ad eundem semper concurrant effectum, possumus tamen abstrahere animum a nonnullis praesertim minus sensibilibus separatasque separatarum rerum consequentias considerare; ita cum corpus grave descendit ab aeris resistentia\protect\index{Sachverzeichnis}{resistentia aeris} possumus animum abstrahere; aliasque irregularitates negligere, ut ipsius per se descensus consequentias aestimemus. 
\pend 
%\newpage
\count\Bfootins=1000
\count\Cfootins=1200
\pstart 
Quoniam ergo causa et effectus \edtext{hoc loco}{\lemma{}\Bfootnote{hoc loco \textit{erg. L}}} sunt ut prius et posterius, necessario inter se connexa; hinc necesse utique est hanc connexionem posse demonstrari, omnis enim propositio necessaria demonstrabilis est, ab eo saltem qui eam intelligit. Omnis autem demonstratio\protect\index{Sachverzeichnis}{demonstratio} fit per definitiones resolutione in propositiones identicas; necesse est ergo causam et effectum perfecte resoluta in idem denique desinere; cumque ex effectu rursus alius sequatur, necesse est, perpetuo identitatem illam servari, porro identitas illa non nisi in eo \edtext{consistere potest}{\lemma{consistere}\Bfootnote{\textit{(1)} debet \textit{(2)} potest \textit{L}}}, \edtext{in quo}{\lemma{potest,}\Bfootnote{\textit{(1)} quod in omni \textit{(2)} in quo \textit{L}}} conveniunt; conveniunt autem, in eo quod tam causa quam effectus habet potentiam quandam, id est capacitatem producendi alium effectum, differunt tantum in varia applicatione et situ, quemadmodum \edtext{linea eadem}{\lemma{quemadmodum}\Bfootnote{\textit{(1)} figurae in alias fo \textit{(2)} linea eadem \textit{L}}} utcunque flexa, eandem longitudinem retinet. Hinc necessarium est tantum posse causam quantum effectum et contra. Adeoque quilibet effectus \edtext{plenus}{\lemma{}\Bfootnote{plenus \textit{erg. L}}}, si \edtext{occasio se offerat}{\lemma{occasio}\Bfootnote{\textit{(1)} est \textit{(2)} se offerat \textit{L}}}, reproducere perfecte potest suam causam id est satis virium habet ad rem in eundem statum redigendam in quo prius erat, aut in aequivalentem. Ut autem aequivalentia possit aestimari; ideo utile est mensuram assumi, qualis est vis necessaria ad elevandum aliquod grave, ad aliquam \edtext{altitudinem. Et}{\lemma{altitudinem.}\Bfootnote{\textit{(1)} Nam \textit{(2)} Et \textit{L}}} dicendum est, si ponatur aliquod corpus aut compositum ex corporibus in eo statu, ut totam suam actionem libere exercendo grave aliquod datum ad datam altitudinem attollere possit, non poterit unquam \edtext{alium}{\lemma{unquam}\Bfootnote{\textit{(1)} idem \textit{(2)} alium \textit{L}}} effectum producere qui plus possit; adeoque omnes applicationes in eam rem inutiles erunt. 
\pend 
\pstart 
Hinc fit ut lapis \edtext{qui ex}{\lemma{}\Bfootnote{qui \textbar\ libere \textit{gestr.} \textbar\ ex \textit{L}}} aliqua altitudine descendit pendulo alligatus, si nihil obstet, et perfecte agat, ad eandem altitudinem resurgere possit; non vero ad altiorem, nec si nihil virium detractum sit ad inferiorem. \edtext{Et}{\lemma{inferiorem.}\Bfootnote{\textit{(1)} Nec dabi \textit{(2)} Et \textit{L}}} arcus aliquis tensus et resiliens, in alteram se partem tantundem tenderet, nisi ipsa corporis eijus moles ictum exciperet, unde fit, ut aliquando inter detendendum rumpatur: Nam ictum nihil excipit, nisi ipsemet, qui cum in ipsa \edtext{ejus massa}{\lemma{ejus}\Bfootnote{\textit{(1)} mole \textit{(2)} massa \textit{L}}} velut interim oriatur; ingentes ex displosione mutationes licet nobis invisibiles in arcus\protect\index{Sachverzeichnis}{arcus} corpore oriri necesse est. Hinc nos cum magnum ictum aeri infligimus, licet exceperint aurae vulnus, nos tamen dolorem sentimus, cum sub ipsum ictus finem sistitur manus. \pend 
\pstart 
Constituenda ergo \edtext{regula est. Causae plenae et effectus integri, eadem potentia est.}{\lemma{regula est.}\Bfootnote{\textit{(1)} Effectus tantundem potest \textit{(2)} Causae [...] potentia est. \textit{L}}} (\textso{Potentia} est \edtext{status ex}{\lemma{}\Bfootnote{status \textbar\ agentis \textit{erg. u. gestr.} \textbar\ ex \textit{L}}} \edtext{quo sequitur effectus positis circumstantiis}{\lemma{quo}\Bfootnote{\textit{(1)} sublato impedimento sequitur effectus \textit{(2)} sequitur effectus positis circumstantiis \textit{L}}} magnitudinis determinatae.) Hinc effectus \edtext{plenus}{\lemma{}\Bfootnote{plenus \textit{erg. L}}} potest reproducere \edtext{causam integram.}{\lemma{reproducere}\Bfootnote{\textit{(1)} suam causam. \textit{(2)} causam integram. \textit{L}}} Effectus potest reproducere se ipsum. Effectus non potest producere aliquod se ipso potentius. Si effectus debilior causa est, integer \edtext{non est. Si causae sint similes, etiam effectus erunt similes.}{\lemma{non est.}\Bfootnote{\textit{(1)} Si causae sint proportionales etiam effectus sunt proportionales, et contra. \textit{(2)} Si causae [...] erunt similes. \textit{L}}} Si effectus \textit{E} eodem modo producatur ex causa \textit{C} quo effectus (\textit{E}) ex \edtext{causa (\textit{C}) eadem erit relatio}{\lemma{causa (\textit{C})}\Bfootnote{\textit{(1)} et sit aequatio explicans relationem \textit{(2)} eadem erit relatio \textit{L}}} inter \textit{E} et (\textit{E}) quae inter \textit{C} et (\textit{C}) (relatio inquam non ratio) quoniam eadem est relatio inter \textit{E} et \textit{C} quae inter \edtext{(\textit{E}) et (\textit{C}). $E \overset{(1)}{\sqcap} C^r$ et $(E) \overset{(2)}{\sqcap} (C)^r.$ $(C) \overset{(3)}{\sqcap} C^a$}{\lemma{(\textit{E}) et (\textit{C}).}\Bfootnote{\textit{(1)} Relatio inter \textit{C} et (\textit{C}) sit \textit{a}, erit $(C) \sqcap C^a$, eodem modo $(E) \sqcap E^a$. Jam \textit{C} ad \textit{E} relatio sit \textit{r}, erit $E^a \sqcap C^{ar}$, erit \textit{(a)} $E \sqcap C$ \textit{(b)} $\textit{(E)} \sqcap C^{\protect\overline{\protect\underline{r}}^{a}}$. \textit{(2)} $E \protect\overset{(1)}{\sqcap} C^r$ et $(E) \protect\overset{(2)}{\sqcap}(C)^r.\ (C) \protect\overset{(3)}{\sqcap} C^a$ \textit{L}}}. \rule[-4mm]{0mm}{10mm}\edtext{Demonstrandum est esse $(E) \sqcap E^a$, erit \protect$\displaystyle\frac{E}{(E)} \sqcap \frac{C^r}{(C)^r\sqcap C^{\overline{\underline{r}}^{\underline{a}}}}$}{\lemma{$(E) \sqcap E^a$,}\Bfootnote{\textit{(1)} ex 3. erit \textit{(2)} componendo 1. et 2. fiet $E + (E) \sqcap C^r + (C)^r$ \textit{(3)} erit $\displaystyle\frac{E}{(E)} \sqcap \frac{C^r}{(C)^r \sqcap C^{\protect\overline{\protect\underline{r}}^{\protect\underline{a}}}}$ \textit{L}}}. Sed haec rectius opinor demonstrabuntur ex solis definitionibus sive substitutionibus\protect\index{Sachverzeichnis}{substitutio}. Nunc satis erit fundamenta generalium de motu ratiocinationum tradidisse. Ut Geometria\protect\index{Sachverzeichnis}{geometria} pendet ex Metaphysicis\protect\index{Sachverzeichnis}{metaphysicum} de toto et parte, ita Mechanica\protect\index{Sachverzeichnis}{mechanica} ex metaphysicis de causa et effectu. Verum a priori Mechanicae principium: Effectus aequipollet causae plenae, seu causa eadem nec plus nec minus producet, modo neque juvetur neque impediatur. Quicunque causam plenam alicujus effectus producere non potest, nec effectum integrum producere potest. Seu causa quae producere non potest \edtext{causam unde aliquis}{\lemma{potest}\Bfootnote{\textit{(1)} effectum unde al \textit{(2)} causam unde aliquis \textit{L}}} effectus produci potest, nec effectum producere potest. Ex. gr. corpus in plano horizontali positum nemo celeritate impellere potest, qui non ad eam altitudinem elevare potest, ex qua delapsum altitudinem de qua agitur haberet. 
\pend 
\count\Bfootins=1000
\count\Cfootins=1200
\pstart 
Quaedam horum theorematum\protect\index{Sachverzeichnis}{theorema} etiam ex eo demonstrantur, quod \edtext{eundem conflictum}{\lemma{quod}\Bfootnote{\textit{(1)} eandem resistentiam\protect\index{Sachverzeichnis}{resistentia} \textit{(2)} eundem conflictum \textit{L}}} esse necesse est, ex concursus celeritate, etsi nihil referat quod moveatur; fateor tamen nec id posse demonstrari nisi per experientiam\protect\index{Sachverzeichnis}{experientia}. Ultimum utique et vera horum ratio est ex primis et metaphysicis causae et effectus. Si constet \edtext{ejusdem causae}{\lemma{ejusdem}\Bfootnote{\textit{(1)} effectus \textit{(2)} causae \textit{ L}}} duos effectus necessario fore aequipollentes, constabit et effectum et causam aequipollere (vel contra) quia duorum unius causae effectuum alter alterius causa esse potest, ut ejusdem rei sint tres status in tribus temporibus \textit{A. B. C}. status in tempore \textit{A} causa status in tempore \textit{B}, et status in tempore \textit{C}. Sed status in tempore \textit{B}. etiam causa status in tempore \textit{C}. In causa et effectu nihil impediet omnia fingi \edtext{inversa, effectumque fingi causam}{\lemma{inversa,}\Bfootnote{\textit{(1)} effectumque duci ex causa \textit{(2)} effectumque fingi causam \textit{L}}}, et causam fingi effectum. Quemadmodum qui per foramen intrat, etiam exire potest. Videndum an non semper demonstrari possit, nisi causa et effectus aequipollerent dari motum perpetuum\protect\index{Sachverzeichnis}{motus perpetuus}, tum nempe etiam cum minor est effectus causa. Equidem tunc semper certum est sequi quietem perpetuam seu extinctionem. Datur motus perpetuus\protect\index{Sachverzeichnis}{motus perpetuus} sed non efficax. \pend \pstart Tot excogitari possunt motus perpetui, quot fere in Mechanica\protect\index{Sachverzeichnis}{mechanica} fieri possunt paralogismi. Corpus ex natura sua resistit velocitati non motui. Experimentum\protect\index{Sachverzeichnis}{experimentum} quod corpus majus etiam in plano horizontali difficilius movetur, non ergo gravitas causa est, sed ipsa soliditas. Nisi corpus resisteret, sequeretur motus perpetuus\protect\index{Sachverzeichnis}{motus perpetuus}. Quia resistit corpus in proportione \edtext{suae molis}{\lemma{proportione}\Bfootnote{\textit{(1)} celeritatis \textit{(2)}  \textbar\ resistit \textit{streicht Hrsg.} \textbar\ suae molis \textit{L}}}, quia nulla alia ratio determinandi. Aliter item quia nulla ratio quae impediat quo minus resurgat ad altitudinem suam. Quia per se sine impedimento extrinseco corpori impulso totum suum motum dedisset et suum retinuisset. \pend
\count\Bfootins=1500
\count\Cfootins=1500


%Probleme durch Rechnungen in Bfns: 
%platzhalter1 ist $C^{\overline{\underline{r}}^{a}}$  ODER  $C^{\frac{}{\underline{r}}{a}}$
%platzhalter2 ist $C^{\overline{\underline{r}}^{\underline{a}}}$  ODER  $C^{\frac{}{\underline{r}}\underline{a}}$
%platzhalter3 ist $E \overset{(1)}{\sqcap} C^r$\
%platzhalter4 ist $(E) \overset{(2)}{\sqcap}(C)^r.\ (C) \overset{(3)}{\sqcap} C^a$.

