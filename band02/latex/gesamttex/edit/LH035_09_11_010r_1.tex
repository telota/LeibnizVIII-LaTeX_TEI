[10~r\textsuperscript{o}]
fixo $\displaystyle P$ ductaque recta $\displaystyle PQ$ ad Hyperbolam applicata,
constat ex inventis Gregorii a
\edtext{S\edtext{. Vincentio\protect\index{Namensregister}{\textso{Saint-Vincent}, Gr\'{e}goire de (Gregorius a S. Vincentio) S.J. 1584-1667},}{\lemma{S. Vincentio}\Cfootnote{\cite{00316}\textit{Opus geometricum}, Antwerpen 1647, lib. VI, prop. 129, S. 596f.}} si rectae $\displaystyle A(B).$
 $\displaystyle AB.$ $\displaystyle AE$ sint progressionis geometricae, spatia $\displaystyle (D)(B)PQ(D).$ $\displaystyle DB(B)(D)D.$ $\displaystyle FEBDF$ fore aequalia.
Unde}{\lemma{S. Vincentio,}\Bfootnote{\textit{(1)}\ (si  \textit{(a)}\ $\displaystyle AP$ sumatur pro unitate,  \textit{(b)}\ $\displaystyle AP$  \textit{(c)}\ $\displaystyle AB.$ $\displaystyle A(B)$ etc. usque ad $\displaystyle AE$ sint ut numeri, spatia $\displaystyle QP(B)(D)Q.$ $\displaystyle QPBDQ$ etc. $\displaystyle QPEFQ$ esse ut   \textit{(aa)}\ earum  \textit{(bb)}\ Logarithmos rationum dictorum numerorum ad  \textit{(aaa)}\ unitatem  \textit{(bbb)}\ rectam constantem $\displaystyle AP.$   \textit{(aaaa)}\ rectae  \textit{(bbbb)}\ )  \textit{(cccc)}\ Ergo  \textit{(dddd)}\ )  \textit{(2)}\ si rectae  \textit{(a)}\ $\displaystyle AP.$  \textit{(b)}\ $\displaystyle A(B).$ $\displaystyle AB.$   \textit{(aa)}\ etc.  \textit{(bb)}\ $\displaystyle AC.$  \textit{(cc)}\ $\displaystyle AE$ sint [...] spatia \textit{(aaa)}\ $\displaystyle QP(B)(D)Q.$ $\displaystyle QPBDQ.$ $\displaystyle QPEFQ$  \textit{(aaaa)}\ erunt \textit{(bbbb)}\ fore progressionis Arithmeticae. \textit{(bbb)}\ $\displaystyle (D)(B)PQ(D).$ [...] Unde \textit{L}}}
si $\displaystyle EA.$ $\displaystyle BA.$ $\displaystyle (B)A$
\edtext{spatia a mobili percurrenda}{\lemma{}\Bfootnote{spatia a mobili percurrenda \textit{erg. L}}}
decrescant in progressione
\edtext{Geometrica, tempora}{\lemma{Geometrica,}\Bfootnote{\textbar\ ita \textit{gestr.} \textbar\ \textit{(1)}\ spatia \textit{(2)}\ tempora \textit{L}}}
a mobili jam insumta, quae
\edtext{repraesentantur portionibus Hyperbolicis $\displaystyle FEBDF.$ $\displaystyle FE(B)(D)F.$ $\displaystyle FEPQF$}{\lemma{repraesentantur}\Bfootnote{\textit{(1)}\ $\displaystyle FEBDF.$ \textit{(2)}\ portionibus [...] $\displaystyle FEPQF$ \textit{L}}}
crescent per incrementa aequalia,
scilicet per spatia $\displaystyle FEBDF.$ $\displaystyle DB(B)(D)D.$
\edtext{$\displaystyle (D)(B)PQ(D)$. crescent ergo progressione Arithmetica}{\lemma{$\displaystyle (D)(B)PQ(D)$.}\Bfootnote{\textit{(1)}\ erunt ergo progressionis Arithmeticae \textit{(2)}\ crescent [...] Arithmetica. \textit{L}}}.
Ergo si tempora insumta sint ut numeri, spatia residua percurrenda erunt ut Logarithmi.
\pend
\count\Bfootins=1500
\count\Cfootins=1500
\count\Afootins=1500
\newpage

   