% [108~v\textsuperscript{o}]
\count\Bfootins=1200
\count\Cfootins=1200
\count\Afootins=1200
\pstart%
Mons. \edtext{Boccone\protect\index{Namensregister}{\textso{Boccone}, Paolo 1633-1704}
\`{a} Mons. Swammerdam:\protect\index{Namensregister}{\textso{Swammerdam}, Jan 1637-1680}
\textit{j'ay rencontr\'{e} fungus undulatus,}\protect\index{Sachverzeichnis}{fungus undulatus}
% }{\lemma{\textit{j'ay} [...] \textit{undulatus}}\Cfootnote{%\textsc{P. Boccone}, \textit{Recherches}, Amsterdam 1674, 
% a.a.O., S. 146.\cite{00318}}}\edtext{
%
ou \textit{fungus maritimus Coralloeides}\protect\index{Sachverzeichnis}{fungus maritimus Coralloeides}
% }{\lemma{\textit{fungus} [...] \textit{Coralloeides}}\Cfootnote{a.a.O., S. 145f.\cite{00318}}}
% \edtext{
\textit{renferm\'{e} entre les racines de la plante maritime appell\'{e}e par Clusius\protect\index{Namensregister}{\textso{L'Ecluse} (Clusius), Charles de 1526-1609}: retiformis, ou palma marina\protect\index{Sachverzeichnis}{palma marina retiformis}, et il y a apparence,} \edtext{\textit{que cette plante estant}}{\lemma{\textit{que}}\Bfootnote{\textit{(1)}\ \textit{l'estant} \textit{(2)}\ \textit{cette plante estant} \textit{L}}} \textit{n\'{e}e sur ce fungus,\protect\index{Sachverzeichnis}{fungus}
il y soit demeur\'{e} attrap\'{e} dans le milieu des membranes de la dite plante; comme on voit une pierre enchass\'{e}e dans une bague.}
% }{\lemma{\textit{renferm\'{e}} [...] \textit{bague}}\Cfootnote{a.a.O., S. 146.\cite{00318}}}
% \edtext{
\textit{Ceux qui voudront observer cette raret\'{e}, pourront aller dans la boutique d'un jardinier, \edtext{qui vend des grains et des oignons, demeurant}{\lemma{jardinier,}\Bfootnote{\textit{(1)}\ \textit{demeurant} \textit{(2)}\ \textit{qui} [...] \textit{demeurant} \textit{L}}} \`{a} Londres\protect\index{Ortsregister}{London}, dans une grande rue, appell\'{e}e Hyde-Street\protect\index{Ortsregister}{Hyde-Street}} o\`{u} l'\textit{on trou\-uera une plante entiere de} palma marina \textit{retiformis\protect\index{Sachverzeichnis}{palma marina retiformis}, tres grande et belle et une autre sans branches;} et \textit{c'est cette derniere, qui renferme dans les membranes de sa racine ce fungus undulatus.
J'ay veu ici \`{a} Amsterdam,\protect\index{Ortsregister}{Amsterdam} chez Mons. Isaac Jean Nys\protect\index{Namensregister}{\textso{Nys}, Isaac Jean} cette production maritime ou Fungus undulatus,\protect\index{Sachverzeichnis}{fungus undulatus} produite par hazard sur une petite branche d'Antipates qui luy a est\'{e} envoy\'{e}e de} Batavia.\protect\index{Ortsregister}{Batavia}%
}{\lemma{Boccone [...] Batavia}\Cfootnote{\cite{00318}a.a.O., S. 145f. Zitat mit Auslassungen.}}
% ******************************
Petrifications qu'on voit \edtext{\textit{des dens de poissons Carcharias\protect\index{Sachverzeichnis}{Carcharodon carcharias}, Lamies, chiens de Mer et semblables; des Herissons Spatagi, ovarius, Histrix marinus d'Imperatus,\protect\index{Namensregister}{\textso{Imperato}, Ferrante 1550-1631}
Millepora\protect\index{Sachverzeichnis}{Millepora},
d'Imperatus\protect\index{Namensregister}{\textso{Imperato}, Ferrante 1550-1631}
Echinus Brissus compressus placentae similis, et vertebres petrifi\'{e}es;}%
}{\lemma{\textit{des dens} [...] \textit{petrifi\'{e}es}}\Cfootnote{a.a.O., S. 150f.\cite{00318}}}
%
et je croy que fungus maritimus\protect\index{Sachverzeichnis}{fungus maritimus} coralloides undulatus n'est autre chose que pierre:
Astroites undulatus,\protect\index{Sachverzeichnis}{astroites undulatus} major,
\`{a} cause de l'arrangement.
Un marchand de Calais\protect\index{Ortsregister}{Calais} m'a asseur\'{e} que dans le nort on trouue des os de poisson de l'epaisseur de la cuisse d'un homme,
qui ont le milieu ondoyans pareil aux plis, et aux marques que je luy fis voir dans un morceau de pierre d'Astroites undulatus.\protect\index{Sachverzeichnis}{astroites undulatus}
\pend%
%
\pstart%
Mons. Boccone\protect\index{Namensregister}{\textso{Boccone}, Paolo 1633-1704} ecrit une
\edtext{lettre \`{a} Messieurs
\textit{Tulpius,\protect\index{Namensregister}{\textso{Tulpius}, Nicolaus Petreus 1593-1674}
Fran\c{c}ois de Vicq,\protect\index{Namensregister}{\textso{Vicq}, Fran\c{c}ois de 1646-1707}
et Piso,\protect\index{Namensregister}{\textso{Pies} (Piso), Willem 1611-1678}
Medecins d'Amsterdam\protect\index{Ortsregister}{Amsterdam} touchant}
le \textit{Bezoar Mineral}\protect\index{Sachverzeichnis}{Bezoar mineral} et \textit{fossile de la Sicile.\protect\index{Ortsregister}{Sizilien}}
% }{\lemma{Boccone [...] \textit{Sicile}}\Cfootnote{a.a.O., S. 225.\cite{00318}}}\edtext{
%
C'est \textit{une pierre qui au goust et \`{a} la consistence est approchante au bole blanc d'Armenie,\protect\index{Ortsregister}{Armenien}
dans la Sicile} on l'appelle \textit{communement pierre Bezoar Mineral.\protect\index{Sachverzeichnis}{Bezoar mineral}}%
}{\lemma{lettre [...] \textit{Mineral}}\Cfootnote{a.a.O., S. 225f.\cite{00318}\hspace{20mm}}}
%
\edtext{Touchant la pierre Bezoar Mineral\protect\index{Sachverzeichnis}{Bezoar mineral} des anciens,
\textit{Serapion}\protect\index{Namensregister}{\textso{Serapion d.J.}}\edlabel{Boccone13}
%\edtext{}{\lemma{\textit{Serapion}}\Cfootnote{\cite{00331}\textsc{J. Serapion}, \textit{Liber de simplici medicina}, Leiden 1525, cap. 196.}}
\textit{de simplicibus mineralibus}
\edtext{\textit{cap. \edlabel{Boccone14}196}}{\lemma{\textit{cap}.}\Bfootnote{\textit{(1)}\ \textit{166} \textit{(2)}\ \textit{169} \textit{(3)}\ \textit{196} \textit{L}}} 
dit qu'elle \textit{est citrini coloris et pulverulenta,}}{\lemma{Touchant [...] \textit{pulverulenta}}\Cfootnote{\cite{00318}a.a.O., S. 226. Siehe \cite{00331}\textsc{Serapion d.J.}, \textit{Liber de simplici medicina}, Leiden 1525, cap. 196.}}
%
\edtext{\textit{Rasis},\protect\index{Namensregister}{\textso{ar-R\={a}z\=\i}, Muhammad ibn Zakar\=\i y\={a} ca. 864-925} dit \textit{qu'elle est citrina friabilis, nullius saporis, qu'on trouue dans la Syrie\protect\index{Ortsregister}{Syrien} dans les Indes,\protect\index{Ortsregister}{Indien}
et dans l'Arabie},\protect\index{Ortsregister}{Arabien}
de sorte que \textit{la pierre Bezoar des Arabes est une pierre fossile.
J'ay veu certains gobelets ou} tasses \textit{de pierre tendre, d'une couleur citrine, qui sont appell\'{e}es gobelets de pierre Bezoar mineral}\protect\index{Sachverzeichnis}{Bezoar mineral}
%}{\lemma{\textit{Rasis} [...] \textit{mineral}}\Cfootnote{a.a.O., S. 226f.}} \edtext{
et \textit{viennent des Indes\protect\index{Ortsregister}{Indien} ou de la Perse},\protect\index{Ortsregister}{Persien}
\`{a} ce qu'on dit \textit{on en trouue \`{a} Paris\protect\index{Ortsregister}{Paris} et ailleurs chez les curieux,
d'une couleur plus ou moins charg\'{e}e, s\c{c}avoir d'une couleur de saffran et de noix.}%
}{\lemma{\textit{Rasis} [...] \textit{noix}}\Cfootnote{\cite{00318}\textsc{P. Boccone}, \textit{Recherches}, Amsterdam 1674, S. 226f.}}
%
\edtext{Ils sont tous \textit{tendres, de la nature d'Alabastre,\protect\index{Sachverzeichnis}{Alabaster}}}{\lemma{\textit{tendres} [...] \textit{d'Alabastre}}\Cfootnote{a.a.O., S. 227.\cite{00318}}}
et je\edlabel{Boccone15}\edtext{}{{\xxref{Boccone15}{Boccone16}\lemma{et je les rangerois [...]\ mani\`{e}res}\Cfootnote{a.a.O., S. 227f.\cite{00318}}}}
les rangerois plus tost du cost\'{e} de l'Alabastre,\protect\index{Sachverzeichnis}{Alabaster}
que des pierres par ce que la duret\'{e} leur manque.
Si l'on pouuoit trouuer dans ces gobelets ce que les anciens ont attribu\'{e} \`{a} leur Bezoar mineral,\protect\index{Sachverzeichnis}{Bezoar mineral}
les Medecins n'en seroient pas fachez.
Benotti Lapidaire m'a monstr\'{e} petits morceaux d'une telle pierre.
J'ay veu de ces gobelets \`{a} Paris\protect\index{Ortsregister}{Paris} chez Mons. l'Abb\'{e} Charles, et chez Mons. Savary d'Arbagnon.\protect\index{Namensregister}{\textso{Savary d'Arbagnon}, ?}
On en trouuera chez Mr Jean Jacques Swammerdam\protect\index{Namensregister}{\textso{Swammerdam}, Jan 1637-1680} jusqu'\`{a} 15 pi\`{e}ces de diverses \edlabel{Boccone16}mani\`{e}res.
%\edtext{}{{\xxref{Boccone15}{Boccone16}\lemma{et je les rangerois [...]\ mani\`{e}res}\Cfootnote{a.a.O., S. 227f.\cite{00318}}}}
%
\edtext{\textit{L'an 1626 chez Jacobi Pignoni a est\'{e} imprim\'{e} \`{a} Florence\protect\index{Ortsregister}{Florenz} par Pietro Francesco Giraldini\protect\index{Namensregister}{\textso{Giraldini}, Pier Francesco fl. 1626} un petit ouurage in 4\textsuperscript{o} nomm\'{e} Discorsi sopra la Pietra Belzuar Minerale},\protect\index{Sachverzeichnis}{Bezoar mineral} elle est \textit{transparente blanche, se trouue}
%}{\lemma{\textit{L'an 1626} [...] \textit{se trouue}}\Cfootnote{a.a.O., S. 228f.\cite{00318}}}\edtext{
en Toscane,\protect\index{Ortsregister}{Toskana} il y a beaucoup de certificats ou t\'{e}moignages des effects merveilleux.
Elle est produite aux endroits favorisez du soleil.
Il dit que c'est un medicament universel, mais sur tout propre \`{a} guerir la pierre, pleuresie; obstruction, fieures malignes.
%}{\lemma{\textit{L'an 1626} [...] malignes}\Cfootnote{a.a.O., S. 228f.\cite{00318}}}\edtext{
\textit{Mais il en cache la description entiere, et l'endroit de la naissance de cette pierre.}}{\lemma{\textit{L'an} [...] \textit{pierre}}\Cfootnote{a.a.O., S. 228f.\cite{00318}}}
%
\edtext{\textit{Il ordonnoit ce Bezoar mineral\protect\index{Sachverzeichnis}{Bezoar mineral} en poudre la} \edtext{\textit{pesanteur de}}{\lemma{\textit{pesanteur}}\Bfootnote{\textit{(1)}\ \textit{des} \textit{(2)}\ \textit{de} \textit{L}}} \textit{deux dragmes, dans du vin, du bouillon, ou dans des eaux cordiales, le matin et le soir avant le repas, il la faisoit continuer par plusieurs jours et souuentes fois il la donnoit de 6 en 6 heures aux malades pour les faire suer.}%
% }{\lemma{\textit{Il ordonnoit} [...] \textit{suer}}\Cfootnote{a.a.O., S. 229.}}\edtext{
%
\textit{Je m'en informay} (dit Boccone\protect\index{Namensregister}{\textso{Boccone}, Paolo 1633-1704})
\textit{chez Messieurs Redi,\protect\index{Namensregister}{\textso{Redi}, Francesco 1626-1697}
et Charles Dati,\protect\index{Namensregister}{\textso{Dati}, Carlo Roberto 1619-1676}}}{\lemma{\textit{Il ordonnoit} [...] \textit{Dati}}\Cfootnote{\cite{00318}a.a.O., S. 229. Zitat mit Auslassung.}}
%
et \edlabel{Boccone17}\textit{j'appris} d'eux que \textit{Giraldini\protect\index{Namensregister}{\textso{Giraldini}, Pier Francesco fl. 1626} avoit declar\'{e}} son secret \textit{\`{a} S.A.S. Ferdinand II que cette pierre se trouuoit} \edtext{\textit{\`{a}}}{\lemma{}\Bfootnote{\textit{\`{a}} \textit{erg. L}}} \textit{2 milles de Florence\protect\index{Ortsregister}{Florence} dans un lieu appell\'{e} Mugnone\protect\index{Ortsregister}{Mugnone}, ils m'en donnerent une grosse de la pesanteur de 8 liures ou environ. La superficie de cette pierre estoit blanchastre ou bien de la couleur d'un marbre qui est sale, et ressembloit \`{a} un vilain cailloux, qu'on trouue souuent par les rues, elle est dure, unie, et extremement pesante, en la cassant elle se divise aisement, et chaque partie est luisante presque \mbox{comme} du Talc, et par cette marque elle a} est\'{e} \textit{bien d\'{e}crite par Giraldini\protect\index{Namensregister}{\textso{Giraldini}, Pier Francesco fl. 1626}. Outre cela j'ay observ\'{e} que les petits morceaux brisez le plus souuent prennent la figure romboeidale, \`{a} cause d'un particulier arrangement des parties qui composent la dite \edlabel{Boccone18}pierre.}\edtext{}{{\xxref{Boccone17}{Boccone18}}\lemma{\textit{j'appris} [...] \textit{pierre}}\Cfootnote{a.a.O., S. 229f.\cite{00318}}}
%
\textit{Outre\edlabel{Boccone19}}\edtext{}{{\xxref{Boccone19}{Boccone20}}\lemma{\textit{Outre cette} [...] \textit{purifi\'{e}e}}\Cfootnote{a.a.O., S. 230f.\cite{00318}}} \textit{cette espece de caillou} de \edtext{\textit{Toscane}\protect\index{Ortsregister}{Toskana}, on trouuera}{\lemma{\textit{Toscane},}\Bfootnote{\textit{(1)}\ vous trouuerez \textit{(2)}\ on trouuera \textit{L}}} \textit{dans plusieurs endroits du Royaume de Sicile\protect\index{Ortsregister}{Sizilien} une poudre en Mine appell\'{e}e \textso{terre de Baira}, \`{a} cause que l'on tire d'un endroit de ce nom, qui est proche de la ville de Palerme\protect\index{Ortsregister}{Palermo}, quoyque l'on} en \textit{trouue} aussi \textit{aux lieux circumvoisins} comme \textit{proche le grand monastere des peres de S. Benoist, proche la ville de Montreale\protect\index{Ortsregister}{Montreale}, et \`{a} l'entour de son ancien chasteau appell\'{e} Mont-Real\protect\index{Ortsregister}{Montreale}. Cette Terre est aussi appell\'{e}e par Hyperbole Elixir vitae\protect\index{Sachverzeichnis}{elixir vitae}, et par d'autres Bezoar mineral\protect\index{Sachverzeichnis}{Bezoar mineral}, pour la rendre plus renomm\'{e}e.} Elle fossilis, \textit{friabilis, sablonneuse, blanche, et pesante pareille \`{a} une espece de tophus\protect\index{Sachverzeichnis}{tophus}. Les peres cordeliers}
\edtext{[\textit{nomm\'{e}s}]}{\lemma{\textit{nomm\'{e}}}\Bfootnote{\textit{L \"{a}ndert Hrsg.}}}
\textit{\`{a} Palerme\protect\index{Ortsregister}{Palermo} Zuccolanti
%}\edtext{}{\lemma{\textit{cordeliers, Zuccolanti}}\Cfootnote{Bezeichnungen f\"{u}r Franziscaner.}}\textit{
donnent \`{a} tout le monde de cette terre \textso{gratis}\hfill et\hfill par\hfill charit\'{e}.\hfill Ils\hfill adjoutent\hfill de\hfill l'avoir\hfill experiment\'{e}\hfill pour\hfill tenir\hfill le\hfill ventre\hfill lache,} 
\pend
\newpage
\pstart \noindent \textit{pour arrester les fluxions de la teste, pour la gravelle, pour la viscosit\'{e} des reins\protect\index{Sachverzeichnis}{reins}, et pour beaucoup de maux. Et particulierement pour purifier la masse du sang\protect\index{Sachverzeichnis}{sang}. On a coustume de la tirer de sa mine dans le mois d'Aoust lorsque le soleil est dans les jours caniculaires.}
Car \textit{on tient} qu'en \textit{ce temps la dite terre est plus \edlabel{Boccone20}purifi\'{e}e}.
%
\edtext{}{{\xxref{Boccone21}{Boccone22}}\lemma{\textit{Dose} [...] \textit{Baira}}\Cfootnote{\cite{00318}a.a.O., S. 231f. Zitat mit Auslassungen.}}%
%\edtext{
\edlabel{Boccone21}\textit{Dose: tantost plus tantost moins 4 drachmes, mais l'ordinaire est la quantit\'{e} que peut receuuoir une petite cuilliere d'argent} dont \textit{on se sert \`{a} table. On la fait prendre \`{a} jeun, le matin mesl\'{e} dans de la conserve de rose, et avaller apr\`{e}s un verre d'eau fraiche et aussi apres le souper auparavant que de s'endormir. Quelques fois ils la prennent mesl\'{e}e dans de l'eau simple, quelques fois dans de l'eau et du vin ensemble dans du bouillon et semblables liqueurs}; \`{a} \textit{plaisir}%}{\lemma{\textit{Dose} [...] \textit{plaisir}}\Cfootnote{a.a.O., S. 231.\cite{00318}}}
%
et cela \edtext{durant plusieurs}{\lemma{durant}\Bfootnote{\textit{(1)}\ beau \textit{(2)}\ plusieurs \textit{L}}} \textit{jours il y en a qui s'en servent} quoyque \edtext{[estant]}{\lemma{}\Bfootnote{estans\textit{\ L \"{a}ndert Hrsg.}}} \textit{en bonne sant\'{e}, pour tenir la circulation dans une egalit\'{e}. Une autre terre semblable \`{a} cellecy} se \textit{trouue encor} en \textit{Sicile\protect\index{Ortsregister}{Sizilien} dans un endroit dit \textso{la montagne di Cane}\protect\index{Ortsregister}{Cane}.
Elle est plus grossiere plus} \edtext{\textit{sablonneuse et}}{\lemma{\textit{sablonneuse}}\Bfootnote{\textit{(1)}\ \textit{est} \textit{(2)}\ \textit{et} \textit{L}}} \textit{moins blanche}, mais \textit{on dit qu'elle est plus agissante.}
%
Toutes deux \textit{operent par insensible transpiration et quelques fois par urine.}
% \edtext{}{\lemma{\textit{operent par} [...] \textit{par urine.}}\Cfootnote{a.a.O., S. 232.}}\edtext{
\textit{Terre de Baira\protect\index{Sachverzeichnis}{terre de Baira} mise en poudre et jett\'{e}e sur des charbons ardens qui doiuuent estre plac\'{e}s dans un lieu obscur fait voir des etincelles pareilles \`{a} celles que produit le salpetre ou le souffre dans le feu.
Quoyqu'en petite quantit\'{e}, et par l\`{a} je croy que cette} pierre \textit{peut estre aperitive, deobstruente.}
% }{\lemma{\textit{Terre de} [...] \textit{deobstruente.}}\Cfootnote{a.a.O., S. 232.}} \edtext{
\textit{D'autres personnes de probit\'{e} m'ont avou\'{e} que proche de la Terre de Misilmeri l'on trou\-ue la m\^{e}me en tout semblable \`{a} celle de Baira.\edlabel{Boccone22}}
% }{\lemma{Toutes [...] \textit{Baira}}\Cfootnote{a.a.O.%, S. 232.\cite{00318}}}
%
\edlabel{Boccone23}\edtext{}{{\xxref{Boccone23}{Boccone24}}\lemma{\textit{Il y a} [...] \textit{breuuages}}\Cfootnote{\cite{00318}a.a.O., S. 232f. Zitat mit Auslassungen.}}% 
% \edtext{
\textit{Il y a quelques ann\'{e}es qu}'un homme \textit{Chiaramonte,\protect\index{Namensregister}{\textso{Chiaramonti}, Scipione 1565-1652}}
%}{\lemma{\textit{Il y a} [...] \textit{Chiaramonte}}\Cfootnote{\cite{00318}a.a.O.%, S. 232.}}
publia aussi un trait\'{e} de la grandeur de \textit{celuy de Giraldini\protect\index{Namensregister}{\textso{Giraldini}, Pier Francesco fl. 1626} intitul\'{e} Bezoar mineral}\protect\index{Sachverzeichnis}{Bezoar mineral} et \textit{Elixir vitae\protect\index{Sachverzeichnis}{elixir vitae}, disant de l'avoir eue de la Sicile\protect\index{Ortsregister}{Sizilien} qu'il la falloit prendre deux fois par jour, je croy que} c'est \textit{celle} de \textit{Baira. Les peres jesuites de Rome\protect\index{Ortsregister}{Rom} distribuent avec billets imprimez une certaine poudre blanche innocente pour guerir beaucoup de maux, ils} en \textit{font prendre deux dragmes et quelques grains d'avantage deux fois par jour dans le vin ou dans le bouillon}%