                  
                \begin{ledgroupsized}[r]{120mm}
                \footnotesize 
                \pstart                
                \noindent\textbf{\"{U}berlieferung:}   
                \pend
                \end{ledgroupsized}
            
              
                            
                            \begin{ledgroupsized}[r]{114mm}
                            \footnotesize 
                            \pstart \parindent -6mm
                            \makebox[6mm][l]{\textit{L}}Aufzeichnung: LH XXXV 15, 6 Bl. 60. 1 Bl. 8\textsuperscript{o}, ungleichm\"{a}{\ss}ig beschnitten. 2 S. \\KK 1, Nr. 194 D \pend
                            \end{ledgroupsized}
                %\normalsize
                \vspace*{5mm}
                \begin{ledgroup}
                \footnotesize 
                \pstart
            \noindent\footnotesize{\textbf{Datierungsgr\"{u}nde}: Im vorliegenden Stück N.~87 %?? = LH035,15,06_060 = Quomodo penduli motus magnete effici possit
sucht Leibniz nach einer L\"{o}sung f\"{u}r die Stabilisierung des Gangs von Pendeluhren,
indem er die Anwendung von Magneten in Erwägung zieht.
Mit der eingangs angedeuteten Methode Isaac Vossius' setzt er sich auch in der Aufzeichnung N.~84 %?? = LH035,15,06_058 = Chronologia. Efficere horologia accurata
auseinander.
Der Hinweis auf die L\"{a}ngengradbestimmung im zweiten Teil von N.~87 %?? = LH035,15,06_060 = Quomodo penduli motus magnete effici possit
dürfte sich auf das St\"{u}ck \cite{01072}\textit{LSB} VIII,~1 N.~6\textsubscript{2} beziehen, 
welches editorisch auf die zweite H\"{a}lfte 1672 datiert wurde.
N.~87 %?? = LH035,15,06_060 = Quomodo penduli motus magnete effici possit
könnte somit zeitnah entstanden sein.}
                \pend
                \end{ledgroup}
            
                \vspace*{8mm}
                \pstart 
                \normalsize \noindent
            [60~r\textsuperscript{o}] \selectlanguage{ngerman}Den Motum penduli zu continuiren siehe was Vossius\protect\index{Namensregister}{\textso  {Vossius}, Isaac (1618-1689)} vorschl\"{a}gt\selectlanguage{latin} apud\edtext{ Monconys}{\lemma{Monconys}\Cfootnote{\textsc{B. de Monconys, }\cite{00134}\textit{Journal des voyages}, Teil II, Paris 1666, S. 154.}} pag. 154. \textit{Voyage des pays bas}.\protect\index{Namensregister}{\textso {Monconys}, Balthasar de (1611-1665)} Hugenius\protect\index{Namensregister}{\textso  {Huygens}, Christiaan (1629-1695)} mus eine andre  manier haben, so mir noch nicht bewust. Fortasse optime per magnetem\protect\index{Sachverzeichnis}{magnes}, si nimirum penduli\protect\index{Sachverzeichnis}{pendulum} inferior  pars esset acus magneti\protect\index{Sachverzeichnis}{acus magnetica} affricta, sed contrario quam qui suppositus est polo. Haec a polo repulsa redelaberetur gravitate\protect\index{Sachverzeichnis}{gravitas naturalis} naturali, et praeterveheret impetu in alteram partem, redux  rursus impelleretur, et ita porro. Modo scilicet proportio ea sit, ut impetu labentem non possit, quiescentem aut progredientem possit vincere \edtext{magnes\protect\index{Sachverzeichnis}{magnes}.
An fortasse aptius ita magnes}{\lemma{magnes.}\Bfootnote{\textit{(1)}\ Sed idem \textit{(a)}\ si ta \textit{(b)}\ aptius si tam fortis \textit{(2)}\ An [...] magnes \textit{L}}} sine ullo pendulo\protect\index{Sachverzeichnis}{pendulum} et perpetuum simul et uniformem efficere motum\protect\index{Sachverzeichnis}{motus perpetuus}\protect\index{Sachverzeichnis}{motus uniformis } potest. \edtext{Finge}{\lemma{potest.}\Bfootnote{\textit{(1)}\ Esto \textit{(2)}\ Finge \textit{L}}} magnetem\protect\index{Sachverzeichnis}{magnes} ita fortiter repellere  acum, ut totam circumagat, atque ita denuo repellere \edtext{possit. Si unus magnes}{\lemma{possit.}\Bfootnote{\textit{(1)}\ Si hoc procederet \textit{(2)}\ Si unus magnes \textit{L}}} non satis fortis  esset applicandus esset alter similiter, traderentur sibi  plures per manus. Quilibet admitteret, quia \edtext{ab}{\lemma{quia}\Bfootnote{\textit{(1)}\ ob \textit{(2)}\ ab \textit{L}}}  obliquo et a tergo, et ab alio magnete\protect\index{Sachverzeichnis}{magnes} impulsam\protect\index{Sachverzeichnis}{impulsus}  repelleret ubi sibi e directo venit. Erunt ergo  omnes periodi uniformes. Etsi non forte partes periodorum  finge quolibet minuto secundo periodum absolvi,  applicetur logistica decimalis, ut dixi. Sed  hoc artificio, ut non sit opus magnetibus\protect\index{Sachverzeichnis}{magnes} aliquando movere  omnes omnino \edtext{gyros;}{\lemma{gyros;}\Bfootnote{\textit{(1)}\ sed ta \textit{(2)}\ quod \textit{L}}} quod alioquin fieret nonnunquam concurrentibus aliquando \edtext{revolutionibus. Verum}{\lemma{revolutionibus.}\Bfootnote{\textit{(1)}\ Sed \textit{(2)}\ Verum \textit{L}}} ut gyrus propior aperiat tantum aliquid quo alter  se sponte moveat. Haec machinula\protect\index{Sachverzeichnis}{machinula} esset pure magnetica independens a gravitate\protect\index{Sachverzeichnis}{gravitas} et Elatere\protect\index{Sachverzeichnis}{elater}, atque  ita fortasse omnium quae cogitari possunt ad horologia\protect\index{Sachverzeichnis}{horologium}  aptissima, aeri excludendo includendus magnes\protect\index{Sachverzeichnis}{magnes} vitro  sigillato, cujus fortasse lateribus includatur aqua, aut spiritu vini\protect\index{Sachverzeichnis}{spiritus vini} excludendae tanto rectius aeri. Et conservandis  viribus magnetum\protect\index{Sachverzeichnis}{magnes}. Possunt et plures acus esse in  eodem circulo, eadem quae magnetum\protect\index{Sachverzeichnis}{magnes} est distania. Res  certa est: modo hoc unum obtineatur magnetes\protect\index{Sachverzeichnis}{magnes} fortius repellere [60~v\textsuperscript{o}] et attrahere in axe continuato, quam in linea  aliqua obliqua. Quod mihi rationi consentaneum videtur, si quod maxime. 
Quaerenda:%
\newline\indent%
1) motus perp.\protect\index{Sachverzeichnis}{motus perpetuus}%
\newline\indent%
2) motus uniformis,\protect\index{Sachverzeichnis}{motus uniformis}
ope penduli,\protect\index{Sachverzeichnis}{pendulum}
Elateris,\protect\index{Sachverzeichnis}{elaterium}
magnetis;\protect\index{Sachverzeichnis}{magnes}
omnium si possibile praestare ea de quibus alibi,
ut nimis celeriter motum aut nimis tarde tanto magis excitetur aut retardetur ab applicato.%
\newline\indent%
3) motus
perpetuo respiciens datam plagam, seu quiddam absolute immobile ex dato centro,
hoc fiet meridiano\protect\index{Sachverzeichnis}{meridianus} universali
\edtext{Grandamici\protect\index{Namensregister}{\textso{Grandami}(Grandamicus), Jacques 1588-1672}%
\edtext{.}{\lemma{Grandamici}\Cfootnote{\textsc{J. Grandami}, \cite{00292}\textit{Nova demonstratio immobilitatis terrae}, La Fl\`{e}che 1645, S. 66.}}
\newline\indent%
\textso{Primi}}{\lemma{Grandamici.}\Bfootnote{\textit{(1)}\ Dum posteri \textit{(2)}\ \textso{Primi} \textit{L}}}\textso{ }%
maximus usus ad levandos hominum
\edtext{labores in terra.\protect\index{Sachverzeichnis}{terra}}{\lemma{in terra}\Bfootnote{\textit{erg. L}}}%
\newline\indent%
\textso{Postremi }ad perfecte inveniendas longitudines sine ope coeli, et dirigendos hominum
\edtext{labores in mari.}{\lemma{labores}\Bfootnote{\textit{(1)}\ in terra \textit{(2)}\ in mari. \textit{L}}}
\newline\indent%
Medius ad dirigenda hominum judicia ubique. Primum  dat modum: secundum dat tempus; tertium  dat locum. Quibus tribus omnis mechanica continetur. In machinamento postremo magnetico  potest polus amicus relinqui illaboratus, ne  quid hic agat retinendo. Si circulus  magnetum cum circulo acuum sit in eodem  plano, seu concentricus firmatus: nulla mutatio  machinae sequetur quaecunque sit navis jactatio,  quia gravitatis\protect\index{Sachverzeichnis}{gravitas} tantule variatae per momentaneam rei pendulae jactatae obliquitatem nullam  habet sensibilem proportionem ad vim repulsivam \edtext{magnetis. Sed}{\lemma{magnetis.}\Bfootnote{\textit{(1)}\ Et si \textit{(2)}\ Sed \textit{L}}} observandum tamen an  non tractu temporis fieri possit mutatio sensibilis, et  tunc adhibenda est aequatio. Tum an non crebris  reactionibus desinat inimicitia, aut mutuo destruatur  et quanto tempore possunt et nonnunquam  reparari.  \pend%
\pstart%
\edtext{Cum contactus virtualis sit insensibilis affrictus credibile est acum postremo deventuram unicam.
Quid si ergo talis applicatio ut ab inimico repulsa simul ab amico attrahatur sed qui non satis fortis ad retinendum.
Quia plures repellentes, vel etiam quia interim et alius attrahit aliam.
NB.}{\lemma{Cum [...] NB}\Cfootnote{Auf dem linken Rand von Bl.~60~v\textsuperscript{o}, quer zum übrigen Text.}}
\pend 
 


 


 


 


 


 


 

