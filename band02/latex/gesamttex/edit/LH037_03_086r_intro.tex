\footnotesize%
\pstart%
\noindent%
Die folgenden vier St\"{u}cke,
die vom mechanischen Verhalten strömender Gewässer handeln,
sind vermut\-lich alle nach Gespr\"{a}chen mit Artus de Roannez
\protect\index{Namenregister}{\textso{Roanez}, Artus Gouffier de 1627-1696}%
in Paris entstanden.
In N.~97\textsubscript{1} %?? 037,03_086
vermerkt Leibniz selbst als Gespr\"{a}chsdatum den 25. November 1675.
Auch in N.~97\textsubscript{2} %?? 037,03_087
und N.~97\textsubscript{4} %?? 037,03_088
wird auf Roannez ausdrücklich hin\-ge\-wie\-sen.
N.~97\textsubscript{4} %?? 037,03_088
bildete zudem ursprünglich ein einziges Blatt zusammen mit dem von Leibniz auf den 31. Dezember 1675 datierten Stück N.~65, %?? 037,03_089
in dem ebenfalls über ein Gespräch mit Roannez
\protect\index{Namenregister}{\textso{Roanez}, Artus Gouffier de 1627-1696}%
berichtet wird.
Aufgrund der ein\-heit\-lichen Thematik ist für sämtliche vier Stücke ein gemeinsamer Entsteh\-ungs\-zeitraum von Ende November bis Ende Dezember 1675 anzunehmen.
Die in den Textträgern von N.~97\textsubscript{1}, %?? 037,03_086
N.~97\textsubscript{2} %?? 037,03_087
und N.~97\textsubscript{4} %?? 037,03_088
vorliegenden Wasserzeichen unterstützen diese Annahme.
Der Textträger von N.~97\textsubscript{3} %?? 038_024
weist hingegen kein Wasserzeichen auf.
\pend
\normalsize
