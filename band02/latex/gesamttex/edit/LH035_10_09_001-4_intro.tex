\footnotesize%
\pstart%
\noindent%
Die folgenden sieben Unterstücke bilden inhaltlich eine Einheit.
Leibniz sucht dort nach einer mathematischen Beschreibung der Kr\"{a}fte in einem rotierenden Rad.
Aus\-gangs\-punkt der Untersuchung ist die diagrammatische Darstellung eines Rades mit angedeuteter Drehung (siehe [\textit{Fig.~1}] in N.~28\textsubscript{1}),
auf die im Gesamtstück N.~28 mehrfach Bezug genommen wird.
Die relative Chronologie der einzelnen Unterstücke ergibt sich aus folgenden Überlegungen:
\pend\pstart%
Beide frühesten Texte (N.~28\textsubscript{1} und N.~28\textsubscript{2}) sind auf demselben Träger (LH XXXV 10, 9 Bl.~3~v\textsuperscript{o}) überliefert:
N.~28\textsubscript{1} findet sich in der oberen Blatth\"{a}lfte und ist von Schreiberhand; N.~28\textsubscript{2} ist von Leibniz' Hand.
Die Anordnung beider Texte auf dem Blatt sowie die Kennzeichnung von N.~28\textsubscript{2} als \textit{mieux conceu} sind eindeutige Hinweise auf deren Entstehungsabfolge.
\pend\pstart%
Bei den Texten N.~28\textsubscript{3} und N.~28\textsubscript{4} kann man aufgrund des identischen Schreibduktus eine zeitgleiche Entstehung annehmen.
Diese Annahme wird durch die (fragmentarischen) Wasserzeichen unterst\"{u}tzt, die in beiden Textträgern (LH XXXV 10, 9 Bl.~1 und Bl.~2) anzutreffen sind.
N.~28\textsubscript{3} gibt das Ende von N.~28\textsubscript{1} wieder, um einen Ausdruck f\"{u}r am Rad befestigte Gewichte erg\"{a}nzt.
N.~28\textsubscript{4} ist eine Wiedergabe von N.~28\textsubscript{2} (ohne den letzten Absatz).
Ein Schreibfehler am Anfang von N.~28\textsubscript{4} zeigt, dass Leibniz zuerst N.~28\textsubscript{1} zu kopieren begann,
dann aber N.~28\textsubscript{2} abschrieb.
Beide Reinschriften dürften demnach zu einem Zeitpunkt entstanden sein, als sowohl N.~28\textsubscript{1} wie auch N.~28\textsubscript{2} bereits vorlagen.
\pend\pstart%
Der Text N.~28\textsubscript{5}, der auf demselben Träger (LH XXXV 10, 9 Bl.~3~r\textsuperscript{o})
überliefert ist wie N.~28\textsubscript{1} und N.~28\textsubscript{2},
beruht auf einer erweiterten Fassung der Zeichnung [\textit{Fig.~1}] aus N.~28\textsubscript{1}.
Demnach ist zu vermuten, dass N.~28\textsubscript{5} nach N.~28\textsubscript{1} und N.~28\textsubscript{2} entstanden ist.
Da in N.~28\textsubscript{5} zudem eine gedankliche Weiterentwicklung gegenüber N.~28\textsubscript{3} und N.~28\textsubscript{4} erfolgt,
dürfte N.~28\textsubscript{5} später verfasst worden sein.
\pend\pstart%
Der Text N.~28\textsubscript{6} (LH XXXV 10, 9 Bl.~4) 
knüpft ebenfalls an die Zeichnung [\textit{Fig.1}] in N.~28\textsubscript{1} an und kennzeichnet sie als \textit{figura} $\aleph$.
Die gleiche Kennzeichnung findet sich auch in N.~28\textsubscript{1}, fehlt dagegen in allen Texten dazwischen,
weshalb sie wahrscheinlich erst w\"{a}hrend der Entstehung von N.~28\textsubscript{6} und nach der Anfertigung von N.~28\textsubscript{5} eingefügt wurde.
\pend\pstart%
Im Text N.~28\textsubscript{7} (LH XXXVIII Bl.~25) 
werden Ergebnisse aus den Überlegungen dargestellt,
die in den übrigen sechs Unterstücken ihren Niederschlag gefunden haben.
Somit liegt es nahe, bei N.~28\textsubscript{7} eine gemeinsame Entstehungszeit anzunehmen wie bei den Texten N.~28\textsubscript{1} bis N.~28\textsubscript{6}.
Dabei dürfte N.~28\textsubscript{7} aber als letzter Text in der Reihe entstanden sein.
\pend\pstart%
Die absolute Datierung des Gesamtstücks N.~28 beruht auf folgenden Betrachtungen:
(1) In N.~28\textsubscript{5} und N.~28\textsubscript{6} kommen in algebraischen Ausdrücken kombinierte Vorzeichen vor,
die Leibniz nur in der zweiten H\"{a}lfte 1674 und am Anfang 1675 verwendet hat (siehe \textit{LSB} VII, 5, S.~XXXII\,f.).
Insbesondere kommen in N.~28\textsubscript{5} komplexe kombinierte Vorzeichen vor,
die Leibniz spätestens Ende Dezember 1674 aufgegeben hat (siehe die Datierungsgründe in N.~50).
Die in N.~28\textsubscript{6} vorkommenden einfachen Formen der kombinierten Vorzeichen wurden hingegen auch in den ersten Monaten 1675 verwendet.
(2) Die in den Textträgern von N.~28 anzutreffenden Wasserzeichen lassen eine gemeinsame Entstehung der Texte vermuten.
Somit dürfte N.~28 insgesamt in einem Zeitraum verfasst worden sein, welcher die zweite Hälfte 1674 und den Anfang 1675 einschließt.
\pend\pstart%
F\"{u}r den Text N.~28\textsubscript{1} ist eine -- bislang unbekannte -- Vorlage von Leibniz' Hand anzunehmen,
die auch zu einem früheren Zeitpunkt entstanden sein könnte.
\pend%
\normalsize%
%\footnotesize
%\pstart\noindent
%Die hier vorliegenden sieben Texte ergeben aufgrund inhaltlicher und formaler Gemeinsamkeiten eine Einheit. Leibniz sucht in diesen Texten nach einem arithmetischen Ausdruck zur Beschreibung der Kr\"{a}fte in einem rotierenden Rad. Ausgangspunkt dabei ist die Zeichnung eines Rades mit angedeuteter Rotation. Die Ausf\"{u}hrung des Rades sowie die zur Identifizierung ausgezeichneter Punkte verwendeten Buchstaben bleiben in allen Texten unver\"{a}ndert. Auf Bl. 3~v\textsuperscript{o} stehen zwei Texte. Der Text in der oberen Blatth\"{a}lfte ist von Schreiberhand, der in der unteren von Leibniz. Angesichts der Anordnung und der Einleitung des zweiten mit \textit{mieux conceu} ist die Abfolge der Entstehung mit dem oberen Text als fr\"{u}herem, und dem unteren als sp\"{a}terem zuverl\"{a}ssig abzuleiten.
%\pend
%\pstart 
%Die beiden Bl\"{a}tter LH XXXV 10,9 Bl. 1 und 2 k\"{o}nnen aufgrund der identischen Schreiberhand als zur gleichen Zeit entstanden angenommen werden. Das wird durch Teile desselben Wasserzeichens in beiden Bl\"{a}ttern unterst\"{u}tzt. Auf Bl. 2~r\textsuperscript{o} steht das Ende des ersten Textes von Bl. 3~v\textsuperscript{o}, erg\"{a}nzt um einen Ausdruck f\"{u}r am Rad befestigte Gewichte. Bl. 1~r\textsuperscript{o} enth\"{a}lt eine Wiedergabe des zweiten Textes auf Bl. 3~v\textsuperscript{o} ohne dessen letzten Absatz. Ein Schreibfehler am Anfang von 1~r\textsuperscript{o} zeigt, dass Leibniz zuerst den ersten Text von Bl. 3~v\textsuperscript{o} zu kopieren begann, dann jedoch den unteren abschrieb. Diese Reinschriften entstanden demnach zu einem Zeitpunkt, als beide Texte auf Bl. 3~v\textsuperscript{o} bereits vorlagen. Da Verwendung eines bereits beschriebenen Blattes f\"{u}r eine Reinschrift unwahrscheinlich ist, kann angenommen werden, dass die zwei Texte auf Bl. 3~r\textsuperscript{o} nach denen auf der v\textsuperscript{o}-Seite entstanden sind. Der erste Text (linke Spalte, obere H\"{a}lfte) bespricht magnetische Kr\"{a}fte und wird als N. ??
%%= LH035,10,9Bl.3r = Demonstratio geometrica de magnetis sphaera/De magnetis sphaera
%ediert. Der zweite beginnt mit einer erweiterten Figur $\aleph$ und entwickelt einen analytischen Ausdruck f\"{u}r Kr\"{a}fte eines Gewichtes in einem Rad. Unklar bleibt sein zeitliches Verh\"{a}ltnis zu den Abschriften auf Bl. 1 und 2. Da in diesem Text eine gedankliche Weiterentwicklung erfolgt, kann er nach den Abschriften eingeordnet werden.
%\pend
%\pstart 
%Der Text auf Bl. 4 nimmt ebenfalls Bezug auf die Zeichnung auf Bl. 3~v\textsuperscript{o} und bezeichnet sie als \textit{figura $\aleph$}. Diese Bezeichnung steht auf Bl. 3~v\textsuperscript{o}, fehlt dagegen in der Abschrift auf Bl. 1, so dass sie vermutlich w\"{a}hrend der Enstehung des Bl. 4 und nach den Reinschriften angebracht wurde. Demnach lag der obere Text auf Bl. 3~v\textsuperscript{o} bereits vor, als der Text auf Bl. 4 geschrieben wurde. Die Gleichzeitigkeit der Reinschriften auf Bl. 1 und 2 und des Textes auf Bl. 4 werden durch identische Wasserzeichen in allen drei Bl\"{a}ttern unterst\"{u}tzt. Aus diesen \"{U}berlegungen ergeben sich der Zusammenhang der Texte und ihre Reihenfolge als erster und zweiter Text auf Bl. 3~v\textsuperscript{o}, dann die Reinschriften auf Bl. 2~r\textsuperscript{o} vor 1~r\textsuperscript{o}, danach der Text auf Bl. 3~r\textsuperscript{o}, zuletzt der Text auf Bl. 4.  
%\pend
%\pstart
%Die auf Bl. 3 und Bl. 4 verwendeten kombinierten Vorzeichen wurden von Leibniz in der zweiten H\"{a}lfte 1674 benutzt (dazu VII,5 S. XXXII-XXXIII). Eine Rechnung auf Bl. 4v\textsuperscript{o} ist identisch mit einer Nebenrechnung in \textit{LSB} VII,5 N. 24 (S. 198), die auf Januar 1675 datiert wird. Die in drei Blättern identischen Wasserzeichen lassen auf eine zeitlich nahe Entstehung schließen, die vermutlich in den Zeitraum von Juli 1674 bis Januar 1675 fällt. F\"{u}r den ersten Text von Schreiberhand, 3~v\textsuperscript{o} oben, ist eine Vorlage von Leibnizens Hand anzunehmen. Diese ist bislang nicht bekannt, so dass eine fr\"{u}here Datierung nicht ausgeschlossen ist.
%\pend
%\normalsize

