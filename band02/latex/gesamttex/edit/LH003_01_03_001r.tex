\begin{ledgroupsized}[r]{120mm}%
\footnotesize %
\pstart%
\noindent\textbf{\"{U}berlieferung:}%
\pend%
\end{ledgroupsized}%
\begin{ledgroupsized}[r]{114mm}%
\footnotesize%
\pstart%
\parindent -6mm%
\makebox[6mm][l]{\textit{L}}%
Reinschrift mit Verbesserungen:
LH III 1, 3 Bl. 1-8.
4 Bog. 2\textsuperscript{o}.
Etwa 16 S. einspaltig beschrieben (Bl. 4~v\textsuperscript{o} zweispaltig).
% Die Textfolge entspricht der Blattnummerierung.
Die Bogen sind von Leibniz' Hand auf der jeweils ersten Seite \textit{(1)} bis \textit{(4)} nummeriert.
Geringf\"{u}giger Textverlust durch Papierbeschädigung am Falz der einzelnen Bogen. % auf Bl. 1~r\textsuperscript{o}, 2~r\textsuperscript{o}, 3~r\textsuperscript{o}, 4~v\textsuperscript{o}, 5~r\textsuperscript{o}, 6~r\textsuperscript{o} und 8~r\textsuperscript{o}
Gleiches Wasserzeichenpaar auf jedem Bogen.%
\newline%
KK1, Nr. 975%
\pend%
\end{ledgroupsized}%
\begin{ledgroupsized}[r]{114mm}%
\footnotesize%
\pstart%
\parindent -6mm%
\makebox[6mm][l]{\textit{E}}%
\cite{01131}\textsc{G.W. Leibniz}, \glqq Directiones ad rem medicam pertinentes\grqq, hrsg. von \textsc{F. Hartmann} und \textsc{M. Kr\"{u}ger}, \textit{Studia Leibnitiana}, 8, 1 (1976), S. 40-68: S. 50-66.
\pend%
\end{ledgroupsized}%
\begin{ledgroupsized}[r]{114mm}%
\footnotesize%
\pstart%
\parindent -6mm%
\"{U}bersetzung: \cite{01132}\textsc{J.E.H. Smith}, \textit{Divine Machines: Leibniz and the Sciences of Life}, Princeton 2011, S.~275-287.
\pend%
\end{ledgroupsized}%
%
\vspace*{5mm}%
\begin{ledgroup}%
\footnotesize%
\pstart%
\noindent%
\footnotesize{%
\textbf{Datierungsgr\"{u}nde:}
Die in sämtlichen Textträgern des vorliegenden Stücks % N.~?? 
anzutreffenden Wasserzeichen sind für den Zeitraum Mitte 1671 bis Anfang 1672 belegt.
Mangels weiterer Anhaltspunkte für eine genauere chronologische Einordnung wird der gesamte Zeitraum als Datierung von N.~70 % = vorliegendem Stück = LH003,01,03_001-008
vorgeschlagen.
}%
\pend%
\end{ledgroup}%
%
%
\vspace{6mm}% PR: Rein provisorisch !!!
\pstart
\noindent
[1~r\textsuperscript{o}]
\pend
\pstart% PR: Bitte als Überschrift gestalten. Danke.
\normalsize%
\centering%
% [1~r\textsuperscript{o}]
Directiones ad rem Medicam pertinentes
\pend%
\vspace{0.7em}% PR: Rein provisorisch !!!
\pstart%
\noindent%
Man mus Instrumenta haben Urin\protect\index{Sachverzeichnis}{Urin} und Puls\protect\index{Sachverzeichnis}{Puls} genauer zu betrachten, weil solches general zeichen seyn des Menschlichen zustandes.
\pend%
\count\Bfootins=1200
\count\Cfootins=1200
\count\Afootins=1200
\pstart%
Vor die Urin\protect\index{Sachverzeichnis}{Urin} ist nichts be{\ss}er als ein guthes Microscopium\protect\index{Sachverzeichnis}{microscopium} von einem glase, denn solches wird tausenterley dinge so sonst sich nicht finden in der Urin\protect\index{Sachverzeichnis}{Urin} entdecken machen, und wird man in kurzer zeit zu solchen Regeln kommen, so alle bishehrige \"{u}bertreffen. 
\pend%
\pstart%
Ebenm\"{a}{\ss}ig wird das zur ader gela{\ss}ene blut\protect\index{Sachverzeichnis}{Blut} k\"{o}nnen examinirt werden. Den Puls\protect\index{Sachverzeichnis}{Puls} zu fuhlen ist nicht ohne da{\ss} die h\"{a}nde der geringsten Medicorum\protect\index{Sachverzeichnis}{medicus} zu der perfection kommen
\edtext{werden alle}{\lemma{werden}\Bfootnote{\textit{(1)}\ es \textit{(2)}\ alle \textit{L}}}
differentien zu f\"{u}hlen, so Galenus\protect\index{Namensregister}{\textso{Galen} ca. 130-200}
\edtext{bemerket.}{\lemma{bemerket}\Cfootnote{Etwa \cite{01133}\textsc{Galen}, \textit{De praecognitione}, 14, 3-5%; \cite{?????}\textsc{Ders.}, \textit{De pulsuum differentiis}, ??? (GK , S.)
.}}
%
Dahehr ist n\"{u}zlich da{\ss} die Herrlichen gedancken, so der ber\"{u}hmte Marcus Marci\protect\index{Namensregister}{\textso{Marci}, Jan Marek 1595-1667} in \edtext{\textit{Sphygmica}}{\lemma{\textit{Sphygmica}}\Cfootnote{\cite{00075}\textsc{J.M. Marci}, \textit{De pro\-por\-tione motus seu Regula sphygmica}, Prag 1639.}}
zu papyr gebracht, ins werck gerichtet w\"{u}rden.
\pend%
\pstart%
Die urin\protect\index{Sachverzeichnis}{Urin} und bluth\protect\index{Sachverzeichnis}{Blut} k\"{o}nnen auch mit gewicht, distilliren\protect\index{Sachverzeichnis}{destillieren}, durchseigen, mit und ohne feuer und auf andere weise probirt werden, sonderlich wenn man im Zweifel stehet.
\pend%
\newpage
\pstart%
Ebenm\"{a}{\ss}ig sind auch mit dem speichel\protect\index{Sachverzeichnis}{Speichel} sowohl als blut\protect\index{Sachverzeichnis}{Blut} und Urin\protect\index{Sachverzeichnis}{Urin} und noch mehr als mit blut\protect\index{Sachverzeichnis}{Blut}, dieweil er ehe zu haben proben anzustellen.
\pend
\pstart%
Und ich glaube da{\ss} au{\ss} der Saliva\protect\index{Sachverzeichnis}{saliva} ein gro{\ss}es von menschlicher constitution\protect\index{Sachverzeichnis}{Konstitution} sowohl als aus der Urin\protect\index{Sachverzeichnis}{Urin} zu schlie{\ss}en, und da{\ss} aus der anatomi\protect\index{Sachverzeichnis}{Anatomie} des Speichels\protect\index{Sachverzeichnis}{Speichel} die Ursachen zu befinden, warumb ein Mensch dieses der andre jenes gern e{\ss}e. Man k\"{o}nte den speichel\protect\index{Sachverzeichnis}{Speichel} clarificiren, im claren brunnen wa{\ss}er dissolviren\protect\index{Sachverzeichnis}{dissolvieren} etc. wie auch mit urin\protect\index{Sachverzeichnis}{Urin}, man kan ihn und urin\protect\index{Sachverzeichnis}{Urin} la{\ss}en zu cristallen\protect\index{Sachverzeichnis}{Kristall} anschie{\ss}en, gewi{\ss}e solventia\protect\index{Sachverzeichnis}{solventia} oder reagentia\protect\index{Sachverzeichnis}{reagentia} dazumischen etc. werden farben herauskommen aus denen Von constitution\protect\index{Sachverzeichnis}{Konstitution} des Menschen zu judiciren.
\pend%
\pstart%
Hiernechst ist eine general inquisition $\langle$a$\rangle$uf die Menschen anzustellen, vermit$\langle$tel$\rangle$st medicinae staticae, so von Sanctorio\protect\index{Namensregister}{\textso{Santorio}, Santorio 1561-1636} $\langle$z$\rangle$u erst durch 30 jahrige experienz in%
% Hier folgt Bl. 1v.