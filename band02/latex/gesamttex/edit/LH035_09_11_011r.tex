[11~r\textsuperscript{o}]
\count\Bfootins=1200
conceuuoir que ce corps est egalement \^{a}pre par tout,
il faut s'imaginer que sa surface est parsem\'{e}e de
\edtext{telles}{\lemma{}\Bfootnote{telles \textit{erg. L}}}
pointes de distance en distance, \`{a} intervalles
\edtext{egaux. Il y a}{\lemma{egaux.}\Bfootnote{\textit{(1)}\ Voil\`{a}  \textit{(a)}\ sur quoy  \textit{(b)}\ la suppo \textit{(2)}\ Cela pos\'{e} \textit{(3)}\ Il y a \textit{L}}}
encor deux autres circomstances
\edtext{\`{a} considerer}{\lemma{}\Bfootnote{\`{a} considerer \textit{erg. L}}}
qui m\'{e}ritent un raisonnement \`{a} part;
dont la complication avec celuy-cy \'{e}puiseroit cette matiere;
mais il \edtext{suffit \`{a} present de poursuiure cette supposition en vertu de laquelle}{\lemma{suffit}\Bfootnote{\textit{(1)}\ de calculer celle cy \`{a} present, et de supposer par consequent, qu \textit{(2)}\ \`{a} present [...] laquelle \textit{L}}}
un\edtext{ corps qui frotte contre}{\lemma{corps}\Bfootnote{\textit{(1)}\ qui va le long \textit{(2)}\ qui frotte contre \textit{L}}}
un autre corps homogene,
(:~comme fait une boule
\edtext{qui roule}{\lemma{qui}\Bfootnote{\textit{(1)}\ marche \textit{(2)}\  roule \textit{L}}}
sur un plan uni,
ou comme font ceux qui glissent sur la glace avec une vistesse\protect\index{Sachverzeichnis}{vitesse} et facilit\'{e} surprenante~:),
perd autant de degrez de vistesse\protect\index{Sachverzeichnis}{degr\'{e} de vitesse} dans un
\edtext{endroit que}{\lemma{}\Bfootnote{endroit\ \textbar\ du plan \textit{gestr.}\ \textbar\ que \textit{L}}}
dans un autre; quoyque en
\edtext{vertu de}{\lemma{vertu}\Bfootnote{\textit{(1)}\ des \textit{(2)}\ de \textit{L}}}
deux autres circomstances il arrivent encor
\edtext{d'autres changemens, dont}{\lemma{d'autres}\Bfootnote{%
\textit{(1)}\ accidens %
\textit{(2)}\ changemens, %
\textbar\ mais qui sont bien moins considerables, et \textit{erg. u. gestr.} \textbar\ dont \textit{L}}}
nous faisons abstraction \`{a}
\edtext{present.}{\lemma{present}\Bfootnote{\textbar\ ; et qui ne sont pas si considerables dans les corps durs \textit{gestr.} \textbar\ . \textit{L}}}
\pend
\newpage
\count\Bfootins=1200
\begin{Geometrico}
\edtext{\textso{Definition[:]} Acceleration ou Retardation\textso{ egale }}{\lemma{\textso{Definition[:]}}\Bfootnote{%
\textit{(1)}\ Mouuemen %
\textit{(2)}\ Acceleration ou Retardation %
\textit{(a)}\ uniforme %
\textit{(b)}\ \textso{egale} \textit{L}}}%
selon
les temps\protect\index{Sachverzeichnis}{temps} (lieux\protect\index{Sachverzeichnis}{lieu})
est celle qui
\edtext{arrive egalement, \`{a} chaque intervalle du temps (lieu) incomparablement plus petit que l'on se puisse imaginer}{\lemma{arrive}\Bfootnote{\textit{(1)}\ , en chaque partie du temps quelque petite (lieu) qu'on la puisse conceuuoir \textit{(2)}\  egalement, dans \textit{(3)}\ egalement, [...] petit  \textit{(a)}\ qu'on  \textit{(b)}\ que [...] imaginer \textit{L}}}.
\end{Geometrico}
\begin{Geometrico}
\edtext{\textso{Mouuement uniforme\protect\index{Sachverzeichnis}{mouvement uniforme} }\edtext{\textso{en soy}}{\lemma{\textso{en}}\Bfootnote{\textit{(1)}\ \textso{luy} \textit{(2)}\ \textso{soy} \textit{L}}}\textso{ m\^{e}me }est
celuy qui demeureroit
\edtext{uniforme,
sans la resistence d'un autre corps, et sans la percussion d'un autre corps que de celuy dont}
{\lemma{uniforme,}\Bfootnote{\textit{(1)}\ sans la percussion ou resistance  \textit{(a)}\ d'un corps exterieur sensible  \textit{(b)}\ d'un autre corps, que celuy qui \textit{(2)}\ sans la resistence [...] celuy dont \textit{L}}}
il est men\'{e}}
{\lemma{\textso{Mouuement}}\Bfootnote{[...] men\'{e}. \textit{erg. L}}}.
\end{Geometrico}
\begin{Geometrico}
\edtext{Th. 1. Un corps dont le mouuement est uniforme en soy m\^{e}me, estant retard\'{e} egalement \`{a} chaque endroit du lieu o\`{u} il passe; les vitesses residues sont entre elles, comme les espaces qui restent \`{a} parcourir.}
{\lemma{men\'{e}.}\Bfootnote{\textit{(1)}\ Theor. I. Un corps\ \textbar\ m\^{u} uniformement en luy \textit{erg.}\ \textbar\  estant retard\'{e} uniformement par le lieu o\`{u} il passe, les vitesses residues seront comme les espaces. \textit{(2)}\ Th. 1. Un corps [...] uniforme en  \textit{(a)}\ luy  \textit{(b)}\ soy m\^{e}me,  \textit{(aa)}\ mais  \textit{(bb)}\ estant retard\'{e} egalement  \textit{(aaa)}\ par le lieu o\`{u} il  \textit{(bbb)}\ \`{a} chaque [...] \`{a} parcourir. \textit{L}}}
\end{Geometrico}
\count\Bfootins=1000
\count\Cfootins=1000
\pstart
\noindent 
Dans la \edtext{fig. I.}{\lemma{fig. I.}\Cfootnote{Siehe [\textit{Fig. 3}] auf S.~\pageref{035,09,11_009v_Fig.3}.}}
soit le corps mobile represent\'{e} par le point $\displaystyle B$
\edtext{qui [parcourroit] l'espace de la ligne $\displaystyle EA$}{\lemma{qui}\Bfootnote{%
\textit{(1)}\ doit parcourir %
\textit{(2)}\ parcoureroit %
\textit{(a)}\ l'espace $\displaystyle EA.$ %
\textit{(b)}\ l'espace de la ligne $\displaystyle EA$ \textit{L ändert Hrsg.}}}
avec la vistesse
\edtext{uniforme represent\'{e}e par $\displaystyle EG$,}{\lemma{uniforme}\Bfootnote{\textit{(1)}\ $\displaystyle EG$ \textit{(2)}\ represent\'{e}e par $\displaystyle EG$, \textit{L}}}
et par consequent avec un mouuement qui seroit represent\'{e}
\edtext{tout entier}{\lemma{tout entier}\Bfootnote{\textit{erg. L}}}
par $\displaystyle EG$ appliqu\'{e}e \`{a} tous les
\edtext{points $\displaystyle B.$ $\displaystyle (B)$ de l'espace ou de la ligne $\displaystyle EA$,}{\lemma{points $\displaystyle B.$}\Bfootnote{%
\textit{(1)}\ de l'espace \textit{(2)}\ $\displaystyle (B)$ de l'espace ou de la ligne $\displaystyle EA$, \textit{L}}}
ou par le rectangle
\edtext{$\displaystyle GEA$[,] si chaque point $\displaystyle B.$ $\displaystyle (B)$}{\lemma{$\displaystyle GEA$[,]}\Bfootnote{%
\textit{(1)}\ si \textit{(2)}\ s'il \textit{(3)}\ si \textit{(a)}\ point $\displaystyle B.$ \textit{(b)}\ chaque point $\displaystyle B.$ $\displaystyle (B)$ \textit{L}}}
\edtext{de la dite ligne,}{\lemma{de la}\Bfootnote{\textit{(1)}\ ligne \textit{(2)}\ dite ligne, \textit{L}}}
\edtext{ne diminuoit}{\lemma{ne}\Bfootnote{\textit{(1)}\ retardoit \textit{(2)}\ diminuoit \textit{L}}}
egalement sa vitesse[;]
donc les vitesses
\edtext{decroissant egalement jusqu'au repos en $\displaystyle A$,}{\lemma{decroissant}\Bfootnote{%
\textit{(1)}\ egalement \textit{(2)}\ uniformement \textit{(3)}\ egalement [...] en $\displaystyle A$, \textit{L}}}
\edtext{celles qui resteront en chaque point $\displaystyle B$, $\displaystyle (B)$}%
{\lemma{celles [...] point $\displaystyle B$, $\displaystyle (B)$}\Bfootnote{\textit{erg. L}}}
seront comme les appliqu\'{e}es du
\edtext{Triangle $\displaystyle GEA$}{\lemma{$\displaystyle GEA$}\Cfootnote{Bei der gleichm\"{a}{\ss}igen Bewegung von $\displaystyle M$ bezeichnet $\displaystyle GEA$ ein in [\textit{Fig. 3}] auf S.~\pageref{035,09,11_009v_Fig.3} nicht gezeichnetes Viereck; bei der gleichm\"{a}{\ss}ig verz\"{o}gerten Bewegung von $\displaystyle M$ bezeichnet $\displaystyle GEA$ das gezeichnete gleichnamige Dreieck.}}
(:~par ce qui a est\'{e} demonstr\'{e}
\edtext{par Galilaei\protect\index{Namensregister}{\textso{Galilei} (Galilaeus, Galileus), Galileo 1564-1642}}{\lemma{par Galilaei}\Cfootnote{Die gleiche Konstruktion wird in \cite{00050}\textsc{G. Galilei}, \textit{Discorsi}, Leiden 1638, S.~169-171 (\cite{00048}\textit{GO} VIII, S.~208f.) zum Beweis des ersten Satzes \"{u}ber die gleichm\"{a}{\ss}ig beschleunig\-te Bewegung fallender Körper verwendet. Dort bezeichnet die senkrechte Achse allerdings nicht wie bei Leibniz die durch den beweglichen K\"{o}rper durchlaufene Strecke, sondern den zeitlichen Ablauf der Bewegung.}}~:)
s\c{c}avoir comme les droites $\displaystyle CB.$
$\displaystyle (C)(B)$ paralleles \`{a} la base $\displaystyle EG$.
Or $\displaystyle CB.$ $\displaystyle (C)(B)$ sont comme $\displaystyle AB.$ $\displaystyle A(B)$
ou comme les espaces\protect\index{Sachverzeichnis}{espace} qui restent \`{a} parcourir.
Donc les vistesses residues sont comme les espaces qui restent \`{a} parcourir.
\pend
\vspace{2em}
\pstart
\noindent
[\textit{Folgender kleingedruckter Text gestrichen:}]
\pend
\vspace{0.5em}
\pstart
\footnotesize
\noindent
\edtext{Les m\^{e}mes circomstances estant pos\'{e}es les temps dont chaque endroit du lieu doit estre parcouru,}{\lemma{}\Bfootnote{\textit{(1)}\ Les augmentations continuelles des temps qu'il faudroit employer pour parcourir les m\^{e}mes lieux \textit{(2)}\  Les m\^{e}mes [...] pos\'{e}es les \textit{(a)}\ augmentations continuelles du temps en chaque point  \textit{(b)}\ temps dont [...] du lieu  \textit{(aa)}\ qui  \textit{(bb)}\ doit estre parcouru, \textit{L}}}
sont en raison reciproque des espaces qui restent \`{a} parcourir.
Soit l'espace $\displaystyle EA$ divis\'{e} en parties $\displaystyle EB.$ $\displaystyle B(B).$ $\displaystyle (B)P$ etc. infiniment petites, \'{e}gales entre elles;
je dis que les parties infiniment petites du temps, que le mobile employe \`{a} parcourir ces parties du lieu
$\displaystyle EB$, $\displaystyle B(B)$,
\edtext{$\displaystyle (B)P$, ou les points ou endroits de l'espace, $\displaystyle B.$ $\displaystyle (B).$ $\displaystyle P$
sont en raison reciproque des espaces \`{a} parcourir $\displaystyle AE.$ $\displaystyle AB.$ $\displaystyle A(B).$}%
{\lemma{$\displaystyle (B)P$,}\Bfootnote{%
\textit{(1)}\ seront %
\textit{(2)}\ ou les points [...] $\displaystyle B.$ $\displaystyle (B).$ $\displaystyle P$ %
\textbar\ (car c'est ce que j'appelle Augmentations continuelles du temps en chaque point du lieu) \textit{gestr.} \textbar\ %
sont %
\textit{(a)}\ comme les espaces $\displaystyle AE$ $\displaystyle A(B)$ %
\textit{(b)}\ en raison [...] $\displaystyle A(B).$ \textit{L}}}
Car les temps dont le mobile
\edtext{va par chaque}{\lemma{va}\Bfootnote{\textit{(1)}\ en chac \textit{(2)}\ par chaque \textit{L}}}
point \edtext{$\displaystyle B$}{\lemma{}\Bfootnote{$\displaystyle B$ \textit{erg. L}}}
ou partie infiniment petite \edtext{$\displaystyle B(B)$ sont}{\lemma{$\displaystyle B(B)$}\Bfootnote{\textit{(1)}\ est \textit{(2)}\ sont \textit{L}}}
en raison reciproque des vistesses qui luy reste [\textit{sic!}]:
parce que generalement les espaces estant les m\^{e}mes ou egaux
(:~comme icy $\displaystyle EB.$ $\displaystyle B(B).$ $\displaystyle (B)P$~:)
\count\Bfootins=1500% [11~v\textsuperscript{o}]
%\pend