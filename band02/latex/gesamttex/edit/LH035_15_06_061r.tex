\begin{ledgroupsized}[r]{120mm}
\footnotesize
\pstart
\noindent\textbf{\"{U}berlieferung:}
\pend
\end{ledgroupsized}

\begin{ledgroupsized}[r]{114mm}
\footnotesize
\pstart \parindent -6mm
\makebox[6mm][l]{\textit{L}}Aufzeichnung: LH XXXV 15, 6 Bl. 61.
1 Zettel (10 x 11 cm), unregelm\"{a}{\ss}ig beschnitten.
1~S. auf Bl.~61~r\textsuperscript{o}. Bl.~61~v\textsuperscript{o} leer.%
\\KK 1, Nr. 194 E
\pend
\end{ledgroupsized}
%\normalsize
\vspace*{5mm}

\begin{ledgroup}
\footnotesize
\pstart
\noindent\footnotesize{\textbf{Datierungsgr\"{u}nde}: In seinen \"{U}berlegungen zur Verbesserung der Ganggenauigkeit von Uhren
bezieht sich Leibniz oftmals auf flaschenzug\"{a}hnliche \"{U}bersetzungen und R\"{a}dersysteme.
% was insbesondere f\"{u}r N.~87 %?? = LH XXXV 15, 6 Bl. 60 = Quomodo motus penduli magnete effici possit % gilt.
Der im vorliegenden Stück diskutierte Ansatz entspricht insbesondere demjenigen in N.~85
%?? = LH XXXV 15, 6 Bl. 62 = De horologio elastico
und dürfte mithin zeitnah entstanden sein.}
\pend
\end{ledgroup}
            
\vspace*{8mm}
\pstart 
\normalsize \noindent
[61~r\textsuperscript{o}] Si \edtext{duorum mobilium}{\lemma{Si}\Bfootnote{\textit{(1)} mobilia \textit{(2)} duorum mobilium \textit{L}}} $a.\,b.$ Ratio
\edtext{motus sit quaecunque inaequalis; $c\!\rightpropto\!d$.}{\lemma{motus}%
\Bfootnote{\textit{(1)}\ $1 \rightpropto a;$ \textit{(2)}\ sit [...] $c \rightpropto d$. \textit{ L}}} Mobilia  connexa, conjuncta vel inserta, ut uno moto alterum moveatur, ut rotae dentibus, trochleae\protect\index{Sachverzeichnis}{trochlea} funibus connectuntur. Augeaturque  aut minuatur motus unius ut \textit{a} extrinseco aliquo accidente aut obsistente minuetur augebiturque in eadem proportione  geometrica et celeritas\protect\index{Sachverzeichnis}{celeritas} alterius, ut si $a$ moveatur ut $10.$ $b$ ut $20.$
\edtext{minuto $a$ moveatur}{\lemma{minuto $a$}\Bfootnote{\textbar\ ut \textit{gestr.}\ \textbar\ moveatur \textit{L}}} ut $9.$ movebitur \textit{b} ut $18.$ nisi quid obstet. Sed nos ponamus efficiendum esse, ut tantum praecise progressione arithmetica\protect\index{Sachverzeichnis}{progressio arithmetica} decedat uni quantum alteri, et si $a$ ex $10.$ fiat $9.$ etiam $b$ ex $20.$ \edtext{fiat $19.$%
\newline%
\hspace*{7,5mm}%
Quaeritur}{\lemma{fiat $19.$}\Bfootnote{\textit{(1)}\ \textso{Nota} \textit{(2)}\ Quaeritur \textit{L}}} ratio efficiendi Motum Uniformem\protect\index{Sachverzeichnis}{motus uniformis}. Efficiendum ergo, ut quandocunque crescit motus, in duobus mobilibus crescat  inaequaliter. Et porro, ut quando crescit inaequaliter, cesset. Cessare autem ob vim praedominantem moventis nequeat, ergo non crescat inaequaliter, ergo non crescat omnino. Idem de decremento. 
\pend 
 


 


 


 


 


 


 


 


 

