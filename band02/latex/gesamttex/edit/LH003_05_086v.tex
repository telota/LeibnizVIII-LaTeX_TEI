[86~v\textsuperscript{o}] gozier, qui luy eust coup\'{e} une partie de la gorge ou du col et que les veines et les arteres carotides fussent emport\'{e}es.
\pend%
\count\Bfootins=1200
\pstart%
Mais tant qu'il est possible d'arrester le sang par l'essence styptique foible ou mediocre, il ne faut point se servir de la forte, qui n'est ainsi faite que pour les grandes et extremes necessitez, suivant cet axiome de \pgrk{f}ilosophie, \edtext{frustra fiunt per plura, quae fieri possunt \edtext{[per]}{\lemma{per}\Bfootnote{\textit{erg. Hrsg.}}} pauciora:}{\lemma{frustra [...] pauciora}\Cfootnote{%Frustra fieri per plura quae fieri possunt per pauciora, 
Siehe etwa \cite{01176}\textsc{W. von Ockham}, \textit{Summa totius logicae}, I 12.}}
ou plustost suivant cet axiome de Medecine, qui dit qu'en matiere de remedes a mitioribus est incipiendum, et ad fortiora progrediendum. On se sert de la foible ou plustost de la mediocre, tant interieurement parce qu'on la donne \`{a} boire dans de l'eau de
\edtext{plantain, avec un peu de succre rosat \`{a} ceux}%
{\lemma{plantain,}\Bfootnote{%
\textit{(1)} sur l' \textit{(2)} avec un peu de
\textit{(a)} goutte \textit{(b)} succre rosat
\textit{(aa)} avec un \textit{(bb)} \`{a} ceux \textit{L}}}
qui crachent ou qui glissent le sang pour avoir quelque veine ou artere rompue dans le corps et on en voit un succes le plus heureux du monde. On en donne alors 2. ou 3. cuiller\'{e}es ou d\'{e}puis environ un quart d'once jusqu' \`{a} \includegraphics[width=0.014\textwidth]{images/uncia.pdf}i et on en reitere cette prise 2 ou 3 fois par jour tant que le sang soit bien arrest\'{e}, meslant \`{a} chaque prise \includegraphics[width=0.013\textwidth]{images/drachma.pdf}i ou ij de sucre rosat et environ \includegraphics[width=0.014\textwidth]{images/uncia.pdf}i ou ij d'eau de plantain ou d'eau rose, ou bien on la donne dans quelque decoction faite avec les herbes
\edtext{vulneraires, comme}{\lemma{vulneraires,}\Bfootnote{\textit{(1)} car \textit{(2)} comme \textit{L}}}
sont les mille feuilles, sanicles, grand consoulde, rubus, etc. ou bien on donne l'essence styptique toute seule, mais si le sang qu'on crache vient de quelque vaisseau rompu dans la poitrine le sucre rosat y est tres necessaire, et il n'est pas non plus inutile ailleurs. Voil\`{a} quand \`{a} la fa\c{c}on de l'essence styptique, et quant \`{a} son usage interne.
\pend%
\pstart%
Usage \textso{externe de cette essence}. On s'en sert de m\^{e}me que celle de Mons. Denys\protect\index{Namensindex}{\textso{Denys}, ???? ??-??)} c'est \`{a} dire que l'on trempe des compresses ou des \edtext{[tampons]}{\lemma{tempons}\Bfootnote{\textit{L \"{a}ndert Hrsg.}}} de linge dans cette eau, et on les applique sur la playe, puis on serre bien cela avec un bandage propre \`{a} la partie, et dans un moment le sang s'arreste en peu de temps apres la playe guerit par la seule application de la m\^{e}me foible ou mediocre selon que le mal est grand. Mais lors que ce n'est point pour arrester le sang mais pour guerir une playe ou une ulcere, il faut adjouter \includegraphics[width=0.014\textwidth]{images/uncia.pdf} d'esprit de vin sur \Pfund j d'essence, parce qu'outre la vertu qu'elle a d'arrester le sang, elle ne souffre point qu'il se face aucune corruption en la partie, ny par consequent aucun pus.
\pend%
\pstart%
Elle est donc detersive astringente, glutinatiue, incarnatiue, consolidatiue et souuerainement curatiue de playes, ulceres, loupes, chancres, et par excellence de la gangrene, en adjoustant \includegraphics[width=0.014\textwidth]{images/uncia.pdf}i d'esprit du vin et sur une once de la dite essence foible ou mediocre. Que si la gangrene est fort \edtext{grande et fort}{\lemma{}\Bfootnote{grande\ \textbar\ est fort grande \textit{streicht Hrsg.}\ \textbar\ et fort \textit{L}}} avanc\'{e}e par une grande modification de la partie, il est bon de se servir de la plus forte, elle revigore la chaleur naturelle, et n'appelle les esprits \`{a} la partie gangren\'{e}e. Pour la guerison des ulceres et des playes la foible suffit en y adjoutant \includegraphics[width=0.014\textwidth]{images/uncia.pdf}i d'esprit de vin sur \Pfund j d'essence.
\pend%
\pstart%
Outre cela elle guerit les Erysipeles, la bruslure, et presque toutes les maladies externes mieux que tous les emplastres, et tous les baumes du Monde estant elle m\^{e}me le vray baume, de nature comme l'experience infallible fera voir \`{a} ceux qui s'en serviront bien \`{a} propos. Pour la guerison \edtext{[des]}{\lemma{les}\Bfootnote{\textit{L \"{a}ndert Hrsg.}}} Eresypeles, la plus foible suffit aussi avec \includegraphics[width=0.014\textwidth]{images/uncia.pdf}ij d'esprit \edtext{[de]}{\lemma{de}\Bfootnote{\textit{erg. Hrsg.}}} vin sur \Pfund j et il la faut aussi semblable pour une petite bruslure, mais si la bruslure \edtext{[est]}{\lemma{et}\Bfootnote{\textit{L \"{a}ndert Hrsg.}}} fort grande, il faut de la mediocre.
\pend%
\pstart%
Pour l'hemorragie du nez ou des narines, on met un peu de cette essence styptique dans le creux de la main, et on la tire par le nez, puis on fait un petit bouchon de charpie ou de coton, qu'on trempe dans la dite essence, et on le met dans la narine qui saigne que \edtext{[si]}{\lemma{s'il}\Bfootnote{\textit{L \"{a}ndert Hrsg.}}} l'hemorragie se rend encore opiniastre \`{a} tout cela il faut boire quelques cuiller\'{e}es de cette essence dans de l'eau d'ortie ou de plantin. 
\pend%
\pstart%
Pour la dysenterie. Elle est encor tres excellente pour les boyaux-ulcerez ou flux de sang qu'on appelle dysenterie, estant prise par%
%% Hier folgt Bl. 87r.