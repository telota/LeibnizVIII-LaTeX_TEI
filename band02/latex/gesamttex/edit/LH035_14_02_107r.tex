\newpage
\count\Bfootins=1200
\count\Cfootins=1200
\count\Afootins=1200
\pstart%
% [107~r\textsuperscript{o}]
Il y a en Dauphin\'{e}\protect\index{Ortsregister}{Dauphin\'{e}} un arc haut d'une montagne o\`{u} il se trouue point de fonds et qui est couuert, d'une herbe particuliere entrelass\'{e}e si serr\'{e}e qu'elle porte les hommes et le bestial.
\pend%
%
\pstart%
Lettre \textso{de }\textso{fucis maritimis}\protect\index{Sachverzeichnis}{fucus maritimus}. Lettre \`{a} Messieurs Mousson et I. Rayus touchant toutes sortes de plantes observ\'{e}es en Angleterre\protect\index{Ortsregister}{England} de variis fucis et muscis maritimis\protect\index{Sachverzeichnis}{muscus maritimus}. Muscum\protect\index{Sachverzeichnis}{muscus} in terram missum propagare sui simile nemo observavit. Professeur en Hollande\protect\index{Ortsregister}{Holland} qui croit que les couleurs dans les feuilles des plantes precipitations du sel de vitriol\protect\index{Sachverzeichnis}{sel de vitriol} dans l'air, que les fleurs ne prennent leur couleur que quand elles sont dans l'air. Car pour rendre blanches
les feuilles et les tiges de fenouil\protect\index{Sachverzeichnis}{fenouil}, asperges\protect\index{Sachverzeichnis}{asperge}, artichauts\protect\index{Sachverzeichnis}{artichaut}, chicor\'{e}es\protect\index{Sachverzeichnis}{chicor\'{e}e} cardons\protect\index{Sachverzeichnis}{cardon}, scellery\protect\index{Sachverzeichnis}{scellery} on les enterre et on les empaille pour les rendre blanches, en empechant cette precipitation. J'ay remarqu\'{e} \`{a} ce sujet que la racine de plantago latifolia incana\protect\index{Sachverzeichnis}{plantago latifolia incana} \edtext{(Bauhin. \textit{pinax}),}{\lemma{(Bauhin. \textit{pinax})}\Cfootnote{%\textsc{C. Bauhin}, \textit{Pinax theatri botanici}, Basel 1671, S. 189.
\cite{00502}a.a.O., S. 189.}}
%
estant moiti\'{e} plant\'{e}e en terre moiti\'{e} hors de terre, a la partie expos\'{e}e \`{a} l'air rouge, l'autre \edlabel{Boccone11}plante [\textit{Satz bricht ab.}]
\pend%
%
\pstart%
\edtext{D'o\`{u} \edlabel{Boccone12}vient}{{\xxref{Boccone11}{Boccone12}}\lemma{plante}\Bfootnote{\textit{(1)}\ D'o\`{u} vient \textit{(2)}\ Il y  \textit{(3)}\ D'o\`{u} vient \textit{L}}}
que les sucs des plantes sont color\'{e}s sans avoir besoin de cette precipitation de l'air, ut Chelidonium\protect\index{Sachverzeichnis}{chelidonium} majus
\edtext{Matthioli,}{\lemma{Matthioli}\Cfootnote{\cite{00501}\textsc{P.A. Mattioli}, \textit{Opera omnia}, Basel 1674, S.~468f.}}
%
testes coup\'{e}es de Cnicus sylvestris\protect\index{Sachverzeichnis}{Cnicus sylvestris} spinosior
\edtext{Bauhini \textit{pinax},}{\lemma{Bauhini \textit{pinax}}\Cfootnote{\cite{00502}\textsc{C. Bauhin}, \textit{Pinax theatri botanici}, Basel 1671, S. 378f.}}
%
car ces couleurs sont renferm\'{e}es dans les tiges des plantes. D'o\`{u} vient qu'il n'y a point de fleur entierement noire.
\pend%
%
\pstart%
Le Cinabre\protect\index{Sachverzeichnis}{cinabre} artificiel est rang\'{e} par fibres, par le melange du mercure\protect\index{Sachverzeichnis}{mercure} et par son arrangement en se sublimant.
\pend%
%
\pstart%
Observations touchant Stenomarga\protect\index{Sachverzeichnis}{stenomarga} \`{a} Mons. Matthaeus Sladus\protect\index{Namensregister}{\textso{Sladus}, Matthaeus (1569-1628)} Medec. d'Amsterdam\protect\index{Ortsregister}{Amsterdam} \`{a} Rouan\protect\index{Ortsregister}{Rouen} cass\'{e} des cailloux appell\'{e}s Birets en Normandie\protect\index{Ortsregister}{Normandie}. Tousjours $\frac{1}{2}$ lieue au plus de la ville. J'ay remarqu\'{e} quelques fois de la boue dedans semblable \`{a} celle de la terre o\`{u} ils estoient. L'opinion de M. des Cartes\protect\index{Namensregister}{\textso{Descartes}, Ren\'{e} (1596-1650)} sur les boules vraysemblables en la composition de plusieurs corps solides: on en trouue dans la corne d'ammon\protect\index{Sachverzeichnis}{corne d'ammon} de Bauhin\protect\index{Namensregister}{\textso{Bauhin}, ()} et Wormius\protect\index{Namensregister}{\textso{Wormius}, Olaus (1588-1654)}, dans l'eau o\`{u} a est\'{e} dissout la stenomarga\protect\index{Sachverzeichnis}{stenomarga}, dans le sang, dans le jaune d'oeuf, dans le lait. Et selon les remarques de \edtext{Swammerdam\protect\index{Namensregister}{\textso{Swammerdam}, Jan (1637-1680)}  et de M. Lewenhoeck dans le Corail}{\lemma{Swammerdam}\Bfootnote{ \textit{(1)}\ dans le Corail\protect\index{Sachverzeichnis}{corail|textit} \textit{(2)}\ et de [...] Corail. \textit{L}}}.
\pend%
%
\pstart%
Terre de M. Mililli dont j'ay donn\'{e} \`{a} Paris\protect\index{Ortsregister}{Paris} \`{a} Mons. Emery\protect\index{Namensregister}{\textso{L'Emery}, Nicolas (1645-1715)} apothicaire en Angleterre\protect\index{Ortsregister}{England} \`{a} Mons. Charles Howard\protect\index{Namensregister}{\textso{Howard}, Charles (1536-1624)} frere du Comt. de Norfolk\protect\index{Ortsregister}{Norfolk} et M. Charles Halton\protect\index{Namensregister}{\textso{Halton}, Charles ?} esquire sous le nom de terra samia\protect\index{Sachverzeichnis}{terra samia} ou Bol de Siale.
\pend%
%
\pstart%
Je juge la terre trouu\'{e}e dans ces birets medicamenteuse, comme la [samienne]\edtext{}{\lemma{}\Bfootnote{samie\textit{\ L \"{a}ndert Hrsg.}}}, ou Agaricum minerale\protect\index{Sachverzeichnis}{agaricum minerale}.
Voyez sa Medulla\protect\index{Sachverzeichnis}{medulla} ou Stenomarga Agricolae\protect\index{Sachverzeichnis}{stenomarga Agricolae}. Trouu\'{e}e un morceau de fer cela grosseur d'une \'{e}pingle dans le milieu d'un \edtext{Biret}{\lemma{d'un}\Bfootnote{\textit{(1)}\ Bizet \textit{(2)}\ Biret \textit{L}}} (+ Biret ou Bizet~+) laver le plus subtil, et en faire des pastilles \edtext{ou tablettes}{\lemma{}\Bfootnote{ou tablettes \textit{erg.} \textit{L}}} avec gomme trayant. Je croy que c'est de m\^{e}me vertu avec les tablettes que fait Camilli\protect\index{Namensregister}{\textso{Camilli}, Annibal (1498-?)} medecin \`{a} Nocera\protect\index{Ortsregister}{Nocera} en Ombrie\protect\index{Ortsregister}{Umbrien} du clocher de Spoleto\protect\index{Ortsregister}{Spoleto}.
\pend%
%
\pstart%
Messieurs Syen\protect\index{Namensregister}{\textso{Syen}, ??}, Margrave\protect\index{Namensregister}{\textso{Margrave}, ??}, Maetz\protect\index{Namensregister}{\textso{Maetz}, ??}, Schrader\protect\index{Namensregister}{\textso{Schrader}, Friedrich (1657-1704)}, Swammerdam\protect\index{Namensregister}{\textso{Swammerdam}, Jan (1637-1680)}, Bellanger\protect\index{Namensregister}{\textso{Bellanger}, ??}, Gravesande\protect\index{Namensregister}{\textso{Gravesande}, Cornelius (??-??)}, Meyer\protect\index{Namensregister}{\textso{Meyer}, ??}, Medecins, Jean Commelin\protect\index{Namensregister}{\textso{Commelijn}, Jan (1629-1692)}, Vandenbrug\protect\index{Namensregister}{\textso{Vandenbrug}, ??}, Blayne\protect\index{Namensregister}{\textso{},}, Droguistes; Servenhuisen\protect\index{Namensregister}{\textso{Servenhuisen}, ?} Apoticaire Bleau\protect\index{Namensregister}{\textso{Bleau}, David? (?-?)}, Frisius Elzevir\protect\index{Namensregister}{\textso{ Elzevir}, Frisius ?}, et Waesbergue\protect\index{Namensregister}{\textso{Waesbergen}, Jan van (1661-1681)}, libraires.
\pend%
%
\pstart%
Monsieur Muntinus\protect\index{Namensregister}{\textso{Muntinus}, Gotthard (?-?)}, professeur de Groningue.\protect\index{Ortsregister}{Groningen}
\pend%
%
\pstart%
Il faut que le peuple cultive l'usage des plantes et autres experiences sans cela elles se perdent, comme ils cultivent encor le Cottinus ou coccigrya\protect\index{Sachverzeichnis}{cotinus coggygria} dans les collines de Rome\protect\index{Ortsregister}{Rom}, le Rhus\protect\index{Sachverzeichnis}{Rhus} ou Summac\protect\index{Sachverzeichnis}{Summac} en Cr\`{e}te\protect\index{Ortsregister}{Kreta}, le Lentiscus\protect\index{Sachverzeichnis}{Lentiscus} en Scio,\protect\index{Ortsregister}{Scio} %\edtext{}{\lemma{Scio}\Cfootnote{Insel mit gleichnamiger Stadt im Golf von Smyrna.}}
le Coton\protect\index{Sachverzeichnis}{Coton} \`{a} Malthe\protect\index{Ortsregister}{Malta} et aux environs des villes de Marsala\protect\index{Ortsregister}{Marsala} et Mazzara\protect\index{Ortsregister}{Massera} en Sicile,\protect\index{Ortsregister}{Sizilien}
\edtext{l'Isatis\protect\index{Sachverzeichnis}{Isatis}}{\lemma{Isatis}\Cfootnote{I. tinctoria, Quelle f\"{u}r den Farbstoff Indigo.}} ou Glastum\protect\index{Sachverzeichnis}{Glastum}, le \edtext{Lutum herba,\protect\index{Sachverzeichnis}{Lutum herba}}{\lemma{Lutum herba}\Cfootnote{Herba lutea oder Gelber Enzian, Heilpflanze und Quelle f\"{u}r gelben Farbstoff.}} la Rubia Major\protect\index{Sachverzeichnis}{Rubia major}, Genistella tinctorum\protect\index{Sachverzeichnis}{Genistella tinctorum}, et le Carduus fullonum\protect\index{Sachverzeichnis}{Carduus fullonum} en plusieurs provinces d'Europe\protect\index{Ortsregister}{Europa}.
On a perdu \`{a} Naples\protect\index{Ortsregister}{Neapel} et Rome\protect\index{Ortsregister}{Rom} faute de cultiver l'usage de plusieurs plantes,
comme de la Radicetta\protect\index{Sachverzeichnis}{Radicetta} ou Struthium,\protect\index{Sachverzeichnis}{Struthium} dont
\edtext{Imperatus}{\lemma{Imperatus}\Cfootnote{\cite{00500}\textsc{F. Imperato}, \textit{Historia naturale}, Venedig 1672, S. 661.}}
%
nous a donn\'{e} beaucoup de lumiere, et qui se rapporte exactement aux anciens.
Bellon\protect\index{Namensregister}{\textso{Bellonius}, Pierre (1517-1564)} rapporte qu'en Grece\protect\index{Ortsregister}{Griechenland} ils tannent leurs cuirs et \'{e}paississent les peaux des calyces\protect\index{Sachverzeichnis}{calyce} des glandes d'Esculus\protect\index{Sachverzeichnis}{glandes d'Esculus} etc.
Je s\c{c}ay par experience, qu'en Sicile\protect\index{Ortsregister}{Sizilien} on a coustume de se servir des feuilles de Myrthus\protect\index{Sachverzeichnis}{Myrthus} communis Italica
\edtext{Bauh. \textit{pin.}}{\lemma{Bauh. \textit{pin.}}\Cfootnote{\cite{00502}\textsc{C. Bauhin}, \textit{Pinax theatri botanici}, Basel 1671, S. 468ff.}}
%
\edtext{abondant}{\lemma{abondant}\Bfootnote{\textit{erg. L}}} dans le Val de Mazzera.\protect\index{Ortsregister}{Mazzara}
Les nouuelles experiences abolissent les vieilles. Le savon dur blanc\protect\index{Sachverzeichnis}{savon dur blanc} ce savon noir\protect\index{Sachverzeichnis}{savon noir} et autres sont en vogue \`{a} cause de leur senteur bonne comme ceux de Naples,\protect\index{Ortsregister}{Neapel} Bologne,\protect\index{Ortsregister}{Bologna} Italie.\protect\index{Ortsregister}{Italien}
Les paysans de Lionnois\protect\index{Ortsregister}{Lyon} et ceux de Toscane\protect\index{Ortsregister}{Toscana} se servent quelques fois de saponaria recentiorum s. major laevis
\edtext{Bauhinus \textit{pin.}}{\lemma{Bauhinus \textit{pin.}}\Cfootnote{% \textsc{C. Bauhin}, \textit{Pinax theatri botanici}, Basel 1671, S. 206.\cite{00502}
a.a.O., S. 206.}}
lorsquelle est en fleur, pour exciter l'\'{e}cume comme celle du savon.
%
Adde \edtext{Piso\protect\index{Sachverzeichnis}{Piso}}{\lemma{Piso}\Cfootnote{\cite{00503}\textsc{G. Piso}, \textit{De Indiae utriusque re naturali}, Amsterdam 1658, S. 162.}}
%
et autres de la saponaria\protect\index{Sachverzeichnis}{saponaria} du Bresil\protect\index{Ortsregister}{Brasilien}.
Les jesuites \`{a} Sacca\protect\index{Ortsregister}{Sacca} se servent du Kali floridum Neapolitanum columnae\protect\index{Sachverzeichnis}{Kali} pour oster les taches des draps noirs. 
À Malthe\protect\index{Ortsregister}{Malta} j'ay veu l'experience de les oster avec les feuilles de laegosagamus Alpini\protect\index{Sachverzeichnis}{laego-sagamus Alpini} \edtext{albus creticus\protect\index{Sachverzeichnis}{albus creticus}}{\lemma{albus creticus}\Bfootnote{\textit{erg. L}}}.
On frotte le drap avec un pacquet de fleurs, sur la tache on laisse sescher, et ensuite on lave le drap avec de l'eau chaude. En cas de besoin on le repete cenere di chebba\protect\index{Sachverzeichnis}{cenere di chebba} \edtext{ou scebba}{\lemma{ou scebba}\Bfootnote{\textit{erg. L}}} des paysans d'Agrigentum\protect\index{Ortsregister}{Agrigent} pour blanchir la toile qui sort du metier prefer\'{e}e \`{a} tout autre.
\edtext{Est ex}{\lemma{autre.}\Bfootnote{\textit{(1)}\ Tachenius \textit{(2)}\ Est ex \textit{L}}} Kali florido\protect\index{Sachverzeichnis}{Kali floridum} lignoso floribus membranaceis semine cochleato \protect\index{Sachverzeichnis}{cochleatum} on a accoustum\'{e} de faire un trou en fa\c{c}on de fourneau dans les champs, on met sur ce fourneau les branches de la plante, qu'ils appellent Liuta\protect\index{Sachverzeichnis}{Liuta}, et que j'appelle Kali floridum etc. apres l'avoir laiss\'{e} secher \`{a} l'air 24 heures ou environ en est\'{e}. Estant ainsi prepar\'{e}es on les brusle avec de la paille, et \`{a} mesure qu'elles se consument on y adjoute des \edtext{nouuelles branches}{\lemma{nouuelles}\Bfootnote{\textit{(1)}\ plantes \textit{(2)}\ branches; \textit{L}}}; dont on tire enfin une cendre spongieuse gris\^{a}tre approchante \`{a} la cendre gravell\'{e}e qu'on vend cher les chandeliers et vinaigriers mais elle est plus legere de poids. On souffle pendant l'operation beaucoup de fum\'{e}e. Les paisans la font aussi avec d'autres plantes, comme malva vulgaris\protect\index{Sachverzeichnis}{malva vulgaris}, chrysanthemum creticum\protect\index{Sachverzeichnis}{Chrysanthemum creticum} et autres.
Mais la meilleure est du Kali.\protect\index{Sachverzeichnis}{Kali}
\edtext{Tachen\protect\index{Namensregister}{\textso{Tachenius} (Tacke), Otto 1610-1680}}{\lemma{Tachen}\Cfootnote{\cite{00479}\textsc{O. Tachenius}, \textit{Hippocrates chymicus}, Braunschweig 1668, S. 113.}} in \textit{Hippoc. chym.}
%
donne une operation approchante afin de garder une grande partie du sel volatil\protect\index{Sachverzeichnis}{sel volatil} des plantes.
\pend%
\count\Bfootins=1500
\count\Cfootins=1500
\count\Afootins=1500