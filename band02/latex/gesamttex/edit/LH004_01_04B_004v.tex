% [4~v\textsuperscript{o}]
\count\Bfootins=1200
\count\Cfootins=1200
\count\Afootins=1200
\pstart
Consideravi postea figuram cordis illudque sursum sinistra per aortam et cavam ascendentes attrahens et per cavam infra tanquam ex hepate paululum dextra trahens verum, ejus situm sum contemplatus, veniebatque cava paululum a dextra et posteriore parte; vergebatque in sinistram et anteriorem, ascendebatque supra aortam ita ut vicinior esset pectori.
Jam vidi ambarum auricularum origines dextra enim cava deorsum incipiens ei ascendenti adnascebatur, ejusque extremitas erat in sinu inter aortam recta ascendentem et tubum aortae descendenti, et venae art. communem, ideoque sursum erat magis inflexa, contra vero sinistra veniebat a ramo satis insigni, qui a cava veniens per medium cordis parietem tubum aortae descendentis amplectebatur, et nescio an rursus
\edtext{cavae jungeretur}{\lemma{cavae}\Bfootnote{\textit{(1)}\ conjungeretur \textit{(2)}\ jungeretur \textit{L}}} versus caput vel seorsum
\edtext{ascenderet, vel potius}{\lemma{ascenderet, vel}\Bfootnote{\textit{(1)}\ rursus \textit{(2)}\ potius \textit{L}}}
inter pulmones absolveretur; sed auricula sinistra ei adnata non tam alte cum illo sequebatur, depressa scilicet a trunco cavae descendentis sub quem latebat, ideoque ejus extremitas deorsum flectebatur, quanquam etiam in medio sui etiam aliquantulum magis deorsum
\edtext{[descenderet],}{\lemma{dependeret}\Bfootnote{\textit{L \"{a}ndert Hrsg.}}}
non erat vero minor dextra, et utraque habebat extremitatem instar cristae galli totaeque erant corrugatae, sed sinistra duobus in locis magis rugosa, quod nempe 1\textsuperscript{o} ascendebat cum ramo cui adnascebatur, postea vero descendebat a trunco cavae pressa; quare etiam duobus in locis deorsum flectebatur.
\pend%
\pstart%
Jam circa\label{jamcirca} basin cordis undiquaque adeps erat, nulla vero versus mucronem nisi quaedam vestigia, quae per quasi venas super cor apparentes descendebat, cujusmodi erant quatuor ex triplici tantum origine. 1\textsuperscript{ma} erat infra cavam hepar versus tanquam ex origine auriculae dextrae quae desinebat versus, mucronem magis, ut puto, quam cavitas dextri ventriculi, cujus tantum lineamenta referebat; secunda ex origine auriculae sinistrae veniebat descendens quoque versus mucronem tanquam vestigium ventriculi sinistri, sed jungebatur tamen tertiae longe altius quam in fine cavitatis sinistrae;
\edtext{nec ostendebant}{\lemma{nec}\Bfootnote{\textit{(1)}\ ascendebant \textit{(2)}\ ostendebant \textit{L}}}
nisi per exiguum sinistrum sinum, simul tamen junctae usque ad finem mucronis fere descendebant. Jam tertia et quarta simul oriebantur ab extremitate auriculae sinistrae nempe in sinu a descensu aortae facto, quarum 3\textsuperscript{a} ut dixi 2\textsuperscript{ae} jungebatur ita ut ex protuberantia foris apparente posset tamen judicari cavitatem dextram non adeo esse profundam. Jam 2\textsuperscript{da} et 4\textsuperscript{ta}, aliaeque innumerae quasi venae a basi cordis versus mucronem non ad perpendiculum descendebant, sed tanquam a \edtext{spina}{\lemma{spina}\Cfootnote{%
Die Lesung der Handschrift ist eindeutig.
Dem Sinn nach dürfte eher \textit{spira} gemeint sein.
Vgl. \cite{01192}\textsc{A.~Bitbol-Hespériès},
\glqq Sur quelques \textit{errata} dans les textes biomédicaux latins de Descartes, AT XI\grqq,
\textit{Archives de Philosophie}, 78 (2015), S.~164.}} versus sinistram, deinde ad dextram flectebantur; sola 1\textsuperscript{a} videbatur esse perpendicularis, quae vero in sinistro latere erant minus flectebantur quam quae in dextro; sola tertia denique in contrarias partes
\edtext{flectebatur, ut}{\lemma{flectebatur,}\Bfootnote{\textbar\ ut scilicet, \textit{gestr.}\ \textbar\ ut \textit{L}}} scilicet 1\textsuperscript{mae}
\edtext{[jungeretur: apparebant]}{\lemma{jungerentur: apparebat}\Bfootnote{\textit{L \"{a}ndert Hrsg.}}}
vero etiam tales venulae transversae in ipsa basi una inter origines utriusque auriculae, alia sub auricula sinistra; sed erat etiam alius exiguus ramus ex dextrae auriculae extremitate versus extremitatem sinistrae reflexus, tanquam ut cum tertia vena concurreret. Notandum vero ex his quas voco quasi venas alias revera venas videri, alias tantum
\edtext{[arterias vel]}{\lemma{vel arterias}\Bfootnote{\textit{L \"{a}ndert Hrsg.}}}
nervos: Avulsis deinde quam potui accuratissime fibris tenacissimis ex pericardio quae vasa e corde egredientia circumplicabant, ipsa
\edtext{vasa consideravi,}{\lemma{vasa}\Bfootnote{\textit{(1)}\ circumplicabant \textit{(2)}\ consideravi, \textit{L}}}
quae erant duo ab origine maxime unum ex media basi nempe aorta, quae recta quidem sursum tota ascendebat, sed statim in duos ramos dividebatur, e quibus sinister deorsum in aliud majus vas ferebatur: aliud foris plane ex anteriore
\edtext{[cordis]}{\lemma{pectoris}\Bfootnote{\textit{L \"{a}ndert Hrsg.}}}
parte egrediebatur, nempe vena arteriosa, quae statim versus sinistrum deorsum versus tendebat, sed statim etiam haec in duos ramos secabatur, e quibus superior et dexterior in aortam descendentem confluebat quod vas aortae descendentis erat omnium longe maximum et decuplo
\edtext{[majus]}{\lemma{major}\Bfootnote{\textit{L \"{a}ndert Hrsg.}}} trunco cavae,
\edtext{[minus]}{\lemma{minor}\Bfootnote{\textit{L \"{a}ndert Hrsg.}}}
tantum erat venae arteriosae initio, in quo notanda erat insignis ruga in egressu e corde quae ibi cavitatem faciebat, eratque indicium illud fuisse longe majus sed jam decrescere.
\pend%
\pstart%
Alter vero venae arteriosae ramus inferior statim in duos alios insignes ramos dividebatur, qui in duobus pulmonum lobis ibant, horumque dexter rursus ex se 3\textsuperscript{tium} ramum insignem emittebat, pro superiori parte dextri pulmonis, adeo ut
\edtext{omnino tribus ramis}{\lemma{omnino}\Bfootnote{\textit{(1)}\ 3\textsuperscript{tius} ramus \textit{(2)}\  tribus ramis \textit{L}}}
asperae arteriae responderent; notandum vero hos duos ramos praecipuos supra duo foramina arteriae venosae existere et esse latiores imo tertium supra tria foramina venae arteriosae; item hos tres ramos non diu conservare duritiem membranarum suarum sed absque ulla sectione a carne pulmonum avelli potuisse, ita ut vix transversi digiti latitudinem retinerent.
\pend%
\pstart%
Notavi praeterea nervum exiguum (procul dubio ex 6\textsuperscript{to} pari) inter initia venae arteriosae et aortae ex medio cordis sursum cum aorta ascendentem. Vasa ad cor ingredientia erant truncus cavae $ae$ qui fere solus proprie cor ingredi videbatur; alia vero vel ex ipsa vel a corde esse exorta nempe ramus $edc$ per medium parietem $de$ sursum $dc$ ascendebat in parte sinistra, deinde tria orificia $i.$ $o.$ $l$ arteriae venosae tribus asperae ramis correspondentia. Erat autem $dfe$ carnea moles utramque auriculam conjungens et plane ejusdem cum illis substantiae et cum vena cava, erat autem sinus in puncto $e$ inter illam et dextram aurem, quare alias dixeram sinistram auriculam esse quasi duplicem. Infra autem istud punctum $e$, ubi 1\textsuperscript{ma} quasi vena cordis cutanea basi committitur, est exiguum foramen\label{foramen} adeo angustum, ut nondum sciam an penetret in cor longius denique inter vasa omnia ubicunque erat aliquid spatii illud adipe quadam molliori et in glandulas degeneranti replebatur, nec istarum glandularum substantia aliter a cordis adipe differebat, quam auricularum caro a cordis carne, quod nempe una motu firmiore fuerat siccata quam altera idem etiam dicendum de differentia inter venae et arteriae tunicas.
[5~r\textsuperscript{o}]
\pend
%\count\Bfootins=1500
%\count\Cfootins=1500
%\count\Afootins=1500