\count\Bfootins=1200
\count\Cfootins=1200
\count\Afootins=1200
\pstart
Je viens d'apprendre par une lettre que Mons. Chatton chirurgien tres habile de Montargis\protect\index{Ortsregister}{Montargis} a \edtext{ecrit il}{\lemma{}\Bfootnote{ecrit \textbar\ de \textit{streicht Hrsg.}\ \textbar\ il \textit{L}}} y a quelques mois \`{a} Mons. Thuillier docteur en Medecine de l'universit\'{e} d'Angers\protect\index{Ortsregister}{Angers} d'un bled
que les paisans du Gastinois\protect\index{Ortsregister}{Gastinois} et de la Sologne\protect\index{Ortsregister}{Sologne} appellent bled
cornu \`{a} cause de sa figure. Estant \`{a} Amsterdam\protect\index{Ortsregister}{Amsterdam} Mons. Justus Schrader\protect\index{Namensregister}{\textso {Schrader}, Justus} me dit qu'il vouloit \'{e}crire de veneno\protect\index{Sachverzeichnis}{venenum}, et d'enseigner les veritables remedes et antidotes\protect\index{Sachverzeichnis}{antidote} contre chacun en particulier. Il me pria de luy communiquer ce que j'en s\c{c}avois. Je \edtext{m'excusa}{\lemma{Je}\Bfootnote{\textit{(1)}\ luy repo \textit{(2)}\ m'excusa, \textit{L}}}, disant que cela me paroissoit trop dangereux \`{a} cause de l'abus. Des for\c{c}ats\protect\index{Sachverzeichnis}{for\c{c}ats} ayant mang\'{e} dans une salade apparemment des feuilles de mandragore\protect\index{Sachverzeichnis}{mandragore}, ayant cueilli cette salade au bord de la mer, l'un en mourut, les autres n'en furent que legerement incommod\'{e}s ayant mang\'{e} peu. Je croy que c'estoit Mandragore\protect\index{Sachverzeichnis}{mandragore}, car elle est commune dans les isles de la mediterran\'{e}e\protect\index{Ortsregister}{Mittelmeer}. La cigue a est\'{e} mise souuent pour du cerfeuil\protect\index{Sachverzeichnis}{cerfeuil} et persil\protect\index{Sachverzeichnis}{persil}. Un esclave \`{a} Pisa\protect\index{Ortsregister}{Pisa} l'ayant fait, on le trouua mort, ayant jett\'{e} par le nez et la bouche une matiere blanche en fa\c{c}on d'ecume ou de cervelle liqu\'{e}fi\'{e}e. Un enfant de Bruxelles\protect\index{Ortsregister}{Br\"{u}ssel}, aag\'{e} de 6 ans, en avoit mang\'{e} avec son
%
% Bu$\langle$ra$\rangle$ramme.
Bu$\langle$ter$\rangle$ramme.
%
Des paysans proche de l'isle l'ann\'{e}e que le Roy la prit, mangerent du solanum Lethale furiosum\protect\index{Sachverzeichnis}{solanum lethale}, ou Belladonna Italorum\protect\index{Sachverzeichnis}{Belladonna}, fruits qui ressemblant assez \`{a} des cerises, \edtext{les uns}{\lemma{cerises,}\Bfootnote{\textit{(1)}\ les un \textit{(2)}\ les uns \textit{L}}} en moururent, les autres devinrent fols, et furent gueris enfin après 8 jours ou plus. Voyez Tragus\protect\index{Sachverzeichnis}{tragus} des proprietez de cette plante qui s'y rapporte assez. Semence d'Hyoscyamum\protect\index{Sachverzeichnis}{hyoscyamum} dans la salade en Italie\protect\index{Ortsregister}{Italien} fit une violente fieure de 10 heures vin noir, ou rouge antidote de cicuta\protect\index{Sachverzeichnis}{cicuta}, sel saturne\protect\index{Sachverzeichnis}{sel saturne} pris trop souuent \edtext{fit}{\lemma{souuent}\Bfootnote{\textit{(1)}\ fait pe \textit{(2)}\ fit \textit{L}}} mourir ayant ost\'{e} l'appetit.
\pend
\pstart Semence de Stramonium\protect\index{Sachverzeichnis}{stramonium} rend les personnes folles et furieuses. Remede avec vinaigre ou jus de citron\protect\index{Sachverzeichnis}{citron}. Le m\^{e}me remede \edtext{[est salutaire]}{\lemma{sont salutaires}\Bfootnote{\textit{L \"{a}ndert Hrsg.}}} pour la graine de jusquiasme\protect\index{Sachverzeichnis}{jusquiasme} et pour l'opium\protect\index{Sachverzeichnis}{opium}. On a remarqu\'{e} en quelques quartiers d'Italie\protect\index{Ortsregister}{Italien} que dans \edtext{[l'ombelle\protect\index{Sachverzeichnis}{ombelle}]}{\lemma{}\Bfootnote{l'umbelle\textit{\ L \"{a}ndert Hrsg.}}} de fenouil\protect\index{Sachverzeichnis}{fenouil} et dans le coeur de raphanus\protect\index{Sachverzeichnis}{raphanus} se \textso{trouue des estranges bestes} longa radice se trouuent des insectes, qui causent estranges aardens. En Angleterre\protect\index{Ortsregister}{England} on fait grand estat de la serpentaria virginiana\protect\index{Sachverzeichnis}{serpentaria virginiana}, je ne s\c{c}ay si cette plante est la radix Snagroel\protect\index{Sachverzeichnis}{radix Snagroel} (notae Anglicae). Les curieux de Florence\protect\index{Ortsregister}{Florenz} ont fait experience devant son altesse de Toscane\protect\index{Ortsregister}{Toskana} d'une huyle qu'ils s\c{c}avent preparer dans la quelle si l'on trempe un fil et que l'on le passe dans la cuisse de quelques poules ou chapons, les animaux meurent incontinent. Ceux qui ont quelque connoissance des plantes n'auront point de peine \`{a} decouurir qu'elle est l'herbe de Balestrero\protect\index{Sachverzeichnis}{herbe de Balestrero} dont on se sert en Espagne\protect\index{Ortsregister}{Spanien} pour empoisonner les fleches au rapport de Schenckius\protect\index{Namensregister}{\textso {Schenck}, Matthias 1517-1571} en ces observations. J'ay quelque part dans mes memoires le nom d'une plante de laquelle si l'on fait du charbon, \edtext{pour}{\lemma{charbon,}\Bfootnote{\textit{(1)}\ et si \textit{(2)}\ pour \textit{L}}} en composer de la poudre \`{a} canon on aura une poudre qui fera mourir sur \edtext{[les]}{\lemma{le}\Bfootnote{\textit{L \"{a}ndert Hrsg.}}} champs, les animaux qui en auront est\'{e} blessez. 
\pend 
\pstart 
A Peruse\protect\index{Ortsregister}{Perugia} on me fit voir un trait\'{e} des venins, Ms. avec \edtext{40}{\lemma{avec}\Bfootnote{\textit{(1)}\ une \textit{(2)}\ 40 \textit{L}}} figures deja grav\'{e}es en tailledouce. Ce discours en Latin, L'auteur Annibal Camilli\protect\index{Namensregister}{\textso {Camilli}, Annibale geb. 1498} deja mort. Son ms. possed\'{e} par Charles Camilli\protect\index{Namensregister}{\textso {Camilli}, Charles} \edtext{notaire}{\lemma{}\Bfootnote{notaire \textbar\ notaire \textit{streicht Hrsg.}\ \textbar\ de \textit{L}}} de la dite ville. Ces figures de Camilli et de Mons. Grevin\protect\index{Namensregister}{\textso {Grevin}, Jacques 1538-1570} me parurent si semblables, comme si les unes estoient copi\'{e}es des autres mais cela ne peut pas estre. 
\pend
\newpage
\pstart 
Le bled cornu vient vers le bas de l'espic du seigle, il est de l'epaisseur d'un grain de froment, il est egalement gris dans toute sa longueur except\'{e} vers les extremes, o\`{u} il deminue un peu, et il est compos\'{e} de 3 morceaux[,] il est de couleur de chastaigne mais fort obscure tirant sur le noir. Il a tantost demy pouce de longueur, tantost un pouce, presque figure quarr\'{e}e, estant divis\'{e} par un sulcus qui va d'une extremit\'{e} \`{a} l'autre. J'ay dessein de le semer, pour voir s'il degenere, et quelle plante il pourroit produire. Caspar\protect\index{Namensregister}{\textso {Bauhin}, Caspar 1560-1624} et Joh. Bauhinus\protect\index{Namensregister}{\textso {Bauhin}, Johann 1541-1612} en ont fait mention sous le nom de secale luxurians\protect\index{Sachverzeichnis}{secale luxurians} vid. pag. 23.
\edtext{\textit{pinax.}}{\lemma{\textit{pinax}}\Cfootnote{\cite{00502}\textsc{C. Bauhin}, \textit{Pinax theatri botanici}, Basel 1671, S. 23.}}
%
L'an 1675 aux environs de Bourges\protect\index{Ortsregister}{Bourges} capitale du Berry\protect\index{Ortsregister}{Berry} plusieurs paisans en furent estrangement incommodes et que lors la gangrene\protect\index{Sachverzeichnis}{gangrene} s'attache elle rong\'{e} et fond la chair en pourriture, et sur les os ordinairement, il ne s'engendre point de chair, laissant la peau \edtext{et les nerfs}{\lemma{et les}\Bfootnote{\textit{(1)}\ os \textit{(2)}\ nerfs \textit{L}}} \`{a} sec. Extrait de la lettre m\^{e}me. C'est un des grains de l'epic du seigle qui se convertit ainsi, croissant et s'allongeant de beaucoup hors de l'egalit\'{e} de l'epy, il est courb\'{e} un peu, \includegraphics[width=0.2\textwidth]{images/lh0351402_106v-d1.pdf} noir \edtext{dessous}{\lemma{noir}\Bfootnote{\textit{(1)}\ dedans \textit{(2)}\ dessous, \textit{L}}}, blanc dedans. \edtext{Les}{\lemma{dedans.}\Bfootnote{\textit{(1)}\ Un \textit{(2)}\ Les \textit{L}}} auteurs n'en parlent pas, on n'en ressent point incontinent les effects, mais un certain engourdissement quelques fois \`{a} une, quelques fois \`{a} 2 jambes, après cela un peu de douleur, enflure pas inflammation, on sent de la froideur; ensuite il y paroist une lividit\'{e}, et après une gangrene\protect\index{Sachverzeichnis}{gangrene}, qui est longtemps aux parties internes; avant qu'elle paroisse \`{a} la peau, ces jours pass\'{e}s ouurant \edtext{ce}{\lemma{ouurant}\Bfootnote{\textit{(1)}\ de \textit{(2)}\ ce \textit{L}}} cuir pour s\c{c}avoir si le dessous estoit gangren\'{e}, introduisant mon doigt dans l'ouuerture, et separant les chairs qui estoient gangren\'{e}es, il en sortit des vents avec bruit si grand qu'ils eteignirent la chandelle qui estoit proche et celuy qui la tenoit, fils du chirurgien tomba \`{a} la renverse par la puanteur extraordinaire. Si on n'est pas securu par l'amputation\protect\index{Sachverzeichnis}{amputation}, la gangrene\protect\index{Sachverzeichnis}{gangrene} monte jusqu' aux epaules, avant qu'ils meurent. C'est un de mes etonnemens, mais il faut croire, que des jambes et cuisses elle monte par derriere le long de l'\'{e}pine du dos, et non pas par devant. Les jambes cependant deviennent seches, menues et d'une noirceur \'{e}pouuentable, sans tomber en pourriture, comme s'il ne restoit que des os couuerts de la peau. J'ay veu il y a 38 ans ce mal regner dans la Sologne\protect\index{Ortsregister}{Sologne} dont il mourut beaucoup de paisans qui y sont plus sujets, ne mangeant que du pain de seigle. Le pays n'en produisant point d'autre. Et ce pays cornu ne s'engendre que dans le seigle comme l'yuraye au froment. L'un et l'autre arrive dans les annees extraordinaires. Mais le bled cornu est plus rare: je ne le s\c{c}ay que pour la 3\textsuperscript{me} fois en 38 ans.
\`{A} la seconde il, n'y en avoit pas beaucoup.
[107~r\textsuperscript{o}]
\pend
\count\Bfootins=1500
\count\Cfootins=1500
\count\Afootins=1500