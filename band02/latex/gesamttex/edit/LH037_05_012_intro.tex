\footnotesize
\pstart
\noindent
Die folgenden drei St\"{u}cke bilden sowohl inhaltlich als auch hinsichtlich des Texttr\"{a}gers eine geschlossene Einheit. N. 31\textsubscript{1} ist ein Auszug aus: \cite{00301}\cite{01008}\textsc{John Wallis}, \textit{Mechanica sive De motu}, London 1670-1671; Thema des exzerpierten Textes ist die Verz\"{o}gerung der Bewegung eines K\"{o}rpers durch eine hindernde Kraft wie etwa den Widerstand des umgebenden Mediums. N. 31\textsubscript{2} kn\"{u}pft an Wallis' Ansatz an und entwickelt ihn weiter; schließlich wird zur mathematischen Beschreibung der Verz\"{o}gerung die logarithmische Funktion verwendet. N. 31\textsubscript{3} nimmt das Ergebnis von N. 31\textsubscript{2} wieder auf; der Text bleibt aber ein Bruchst\"{u}ck. S\"{a}mtliche drei St\"{u}cke sind ferner auf demselben Blatt verfasst worden. N. 31\textsubscript{2} ist durch Leibniz' eigen\-h\"{a}ndigen Vermerk auf April 1675 datiert. Es liegt demnach nahe, diese Datierung auch f\"{u}r N.~31\textsubscript{1} und N. 31\textsubscript{3} zu \"{u}bernehmen.
\pend 
\normalsize 