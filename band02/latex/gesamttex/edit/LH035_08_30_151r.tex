\begin{ledgroupsized}[r]{120mm}
\footnotesize 
\pstart 
\noindent\textbf{\"{U}berlieferung:}
\pend
\end{ledgroupsized}
\begin{ledgroupsized}[r]{114mm}
\footnotesize 
\pstart \parindent -6mm
\makebox[6mm][l]{\textit{L}}%
Aufzeichnung: LH XXXV 8, 30 Bl. 151. 1 Bl. 4\textsuperscript{o} beschnitten.
2 S. Überschrift am rechten Rand von Bl. 151~r\textsuperscript{o} mittig.\\%
Cc 2, Nr. 939
\pend
\end{ledgroupsized}
%
\begin{ledgroupsized}[r]{114mm}
\footnotesize 
\pstart \parindent -6mm
\makebox[6mm][l]{\textit{E}}%
(tlw.) \textsc{G.W. Leibniz}, \textit{Mathematische Schriften}, hrsg. von \textsc{C.I. Gerhardt}, Bd. I, Berlin 1849, S. 8f.\cite{01043}
\pend
\end{ledgroupsized}
%
\vspace*{8mm}
\count\Afootins=1200
\count\Bfootins=1200
\count\Cfootins=1200
%\pstart%
%\normalsize%
%\noindent%
% [151~r\textsuperscript{o}]
%\edlabel{amoenior3}\textso{April. 1675 \ Geometria Amoenior}\edlabel{amoenior4}\edtext{}{{{\xxref{amoenior3}{amoenior4}}\lemma{\textso{April} [...] \textso{Amoenior}}\Bfootnote{\textit{ erg. L}}%
%}{\lemma{}\Afootnote{\textit{Am Rand unter Datum und Überschrift:}\vspace{0.5mm}\\
%Subjicienda \textso{Geometriae arcanae}\\
%Wallisii\protect\index{Namensregister}{\textso{Wallis} (Wallisius), John 1616-1703}
%et Rivii
%\protect\index{Namensregister}{\textso{Ryff} (Rivius), Walther Hermann ca. 1500-1548}\textsuperscript{[a]}
%contignationes.\textsuperscript{[c]}\\
%Blondelli
%\protect\index{Namensregister}{\textso{Blondel} (Blondellus), Fran\c{c}ois 1618-1686}
%linea diminutionum\textsuperscript{[d]}
%Architectonica\textsuperscript{[b]} \\ Varenii
%\protect\index{Namensregister}{\textso{Varenius}, Bernhardus 1622-1650}
%de crepusculis Analysis\textsuperscript{[e]}\\ Beaugrandii
%\protect\index{Namensregister}{\textso{Beaugrand}, Jean de 1595-1640}
%\textit{Geostatice}\textsuperscript{[e]}.
%Ellipticus Compassus forma crucis.\\
%Scriptura coelestis, Gaffarelli\textsuperscript{[f]} et Bangii\textsuperscript{[g]}\\
%\textit{Tachygraphia Anglicana}\textsuperscript{[h]}\vspace{0.5mm}\\
%\footnotesize{\textsuperscript{[a]} et Rivii \textit{ erg. L}%
%\hspace{5mm}\textsuperscript{[b]} Architectonica \textit{ erg. L}\\%
%\textsuperscript{[c]} contignationes: Siehe \cite{00301}\textsc{J. Wallis}, \textit{Mechanica}, London 1670-1671, Prop. X, S. 589-604;
%\cite{01105}\textsc{W.H. Ryff}, \textit{Der furnembsten Künst}, Nürnberg 1547.%
%\hspace{5mm}\textsuperscript{[d]} linea diminutionum: \cite{00343}\textsc{F. Blondel}, \textit{Cours d'architecture}, Paris 1675, S. 13-16.%
%\hspace{5mm}\textsuperscript{[e]} Analysis: \cite{00109}\textsc{B. Varenius}, \textit{Geographia generalis}, Cambridge 1672, S. 245-247.%
%\hspace{5mm}\textsuperscript{[e]} \textit{Geostatice}: \cite{00334}\textsc{J. Beaugrand}, \textit{Geostatice}, Paris 1636.%
%\hspace{5mm}\textsuperscript{[f]} Gaffarelli: \cite{00339}\textsc{J. Gaffarel}, \textit{Curiositez inouyes}, Paris 1637.%
%\hspace{5mm}\textsuperscript{[g]} Bangii: \cite{00340}\textsc{T. Bang}, \textit{Caelum}, Kopenhagen 1657.%
%\hspace{5mm}\textsuperscript{[h]} \textit{Anglicana}: \cite{00338}\textsc{T. Shelton}, \textit{Tachygraphy}, London 1674.\vspace{1mm}}}}}\\
%%\pend
%%\pstart%
%%\noindent%
%%\hangindent=7,5mm%
%Geometriae est explicare figuras quas natura et ars
%\edtext{singulari quadam ratione producit:%
%}{\lemma{ars}\Bfootnote{\textit{(1)}\ nobis non cogitantibus producit, \textit{(2)}\ singulari quadam ratione producit \textit{L}}}%
%\edtext{}{\lemma{}\Afootnote{\textit{Am Rand quer:}\\
%Vieta
%\protect\index{Namensregister}{\textso{Vi\`{e}te} (Vieta), Fran\c{c}ois 1540-1603}
%fine Apollonii\cite{00344} [Galli],\textsuperscript{[a]\,[b]}\,%
%loquitur de aliquot problematis Alhazeni,
%\protect\index{Namensregister}{\textso{Ibn al-Hai\b{t}am}, al-\d{H}asan (Alhazen) ca. 965-1040}
%Rhaetici,\protect\index{Namensregister}{\textso{Rheticus}, Georg Joachim 1514-1576}
%Regio\-montani,\protect\index{Namensregister}{\textso{Regiomontanus}, Johannes 1436-1476}
%Peurbachii.\protect\index{Namensregister}{\textso{Peurbach}, Georg von 1423-1461}\\
%{\footnotesize\textsuperscript{[a]} Gallii \textit{L \"{a}ndert Hrsg.}%
%\hspace{5mm}\textsuperscript{[b]} [Galli]: \cite{00344}\textsc{F. Viete}, \textit{Apollonius Gallus}, Paris 1600, S. 8v.}\vspace{-12mm}}}
%\pend
\pstart%
\normalsize%
\noindent%
[151~r\textsuperscript{o}]
\edlabel{amoenior3}%
\edtext{}{{\xxref{amoenior3}{amoenior4}}\lemma{April [...] \textso{Amoenior}}\Bfootnote{\textit{erg. L}}}%
 April. 1675
\pend%
\pstart%
\noindent%
\centering%
\textso{Geometria Amoenior}%
\edlabel{amoenior4}%
\edtext{}{\lemma{}\Afootnote{\textit{Am Rand unter Datum und Überschrift:}\vspace{0.5mm}\newline
Subjicienda \textso{Geometriae arcanae}
\newline%
Wallisii\protect\index{Namensregister}{\textso{Wallis} (Wallisius), John 1616-1703}
et Rivii\protect\index{Namensregister}{\textso{Ryff} (Rivius), Walther Hermann ca. 1500-1548}\textsuperscript{[a]}
contignationes\textsuperscript{[b]}
\newline%
Blondelli\protect\index{Namensregister}{\textso{Blondel} (Blondellus), Fran\c{c}ois 1618-1686}
linea diminutionum Architectonica\textsuperscript{[c]}\,\textsuperscript{[d]}
\newline%
Varenii\protect\index{Namensregister}{\textso{Varenius}, Bernhardus 1622-1650}
de crepusculis Analysis\textsuperscript{[e]}
\newline%
Beaugrandii\protect\index{Namensregister}{\textso{Beaugrand}, Jean de 1595-1640}
\textit{Geostatice}\textsuperscript{[f]}
Ellipticus Compassus forma crucis
\newline%
Scriptura coelestis, Gaffarelli\textsuperscript{[g]} et Bangii\textsuperscript{[h]}
\newline%
\textit{Tachygraphia Anglicana}\textsuperscript{[i]}\vspace{0.5mm}%
\newline%
%
\footnotesize{%
%
\textsuperscript{[a]}~et Rivii
\textit{erg. L}%
%
\hspace{5mm}\textsuperscript{[b]}~contignationes:
\cite{00301}\textsc{J. Wallis}, \textit{Mechanica}, London 1670-1671, S.~589-604 (\cite{01008}\textit{WO} I, S.~953-964).
\cite{01105}\textsc{W.H.~Ryff}, \textit{Der furnembsten Künst}, Nürnberg 1547.%
%
\hspace{5mm}\textsuperscript{[c]}~Architectonica
\textit{erg. L}%
%
\hspace{5mm}\textsuperscript{[d]}~Architectonica:
\cite{00343}\textsc{F. Blondel}, \textit{Cours d'architecture}, Paris 1675, S. 13-16.%
%
\hspace{5mm}\textsuperscript{[e]}~Analysis:
\cite{00109}\textsc{B. Varenius}, \textit{Geographia generalis}, Cambridge 1672, S. 245-247.%
%
\hspace{5mm}\textsuperscript{[f]}~\textit{Geostatice}:
\cite{00334}\textsc{J. Beaugrand}, \textit{Geostatice}, Paris 1636.%
%
\hspace{5mm}\textsuperscript{[g]}~Gaffarelli:
\cite{00339}\textsc{J. Gaffarel}, \textit{Curiositez inouyes}, Paris 1637.%
%
\hspace{5mm}\textsuperscript{[h]}~Bangii:
\cite{00340}\textsc{T. Bang}, \textit{Caelum}, Kopenhagen 1657.%
%
\hspace{5mm}\textsuperscript{[i]}~\textit{Anglicana}:
\cite{00338}\textsc{T. Shelton}, \textit{Tachygraphy}, London 1674.\vspace{0.5em}
}}}
\pend%
\vspace*{0.5em}%
% \newpage%
\pstart%
\noindent%
%\hangindent=7,5mm%
Geometriae est explicare figuras quas natura et
\edtext{ars singulari quadam ratione producit:%
}{\lemma{ars}\Bfootnote{\textit{(1)}\ nobis non cogitantibus producit, \textit{(2)}\ singulari quadam ratione producit: \textit{L}}}%
\edtext{}{\lemma{}\Afootnote{\textit{Am Rand quer:}%
\newline%
Vieta\protect\index{Namensregister}{\textso{Vi\`{e}te} (Vieta), Fran\c{c}ois 1540-1603}
fine Apollonii\cite{00344} [Galli],\textsuperscript{[a]\,[b]}\,%
loquitur de aliquot problematis Alhazeni,
\protect\index{Namensregister}{\textso{ Ibn al-Hai\b{t}am}, al-\d{H}asan (Alhazen) ca. 965-1040}
Rhaetici,\protect\index{Namensregister}{\textso{Rheticus}, Georg Joachim 1514-1576}
Regio\-montani,\protect\index{Namensregister}{\textso{Regiomontanus}, Johannes 1436-1476}
Peurbachii.\protect\index{Namensregister}{\textso{Peurbach}, Georg von 1423-1461}%
\newline%
{\footnotesize\textsuperscript{[a]} Gallii \textit{L \"{a}ndert Hrsg.}%
\hspace{5mm}\textsuperscript{[b]} [Galli]: \cite{00344}\textsc{F. Viete}, \textit{Apollonius Gallus}, Paris 1600, S. 8v.}\vspace{-12mm}}}
\pend%
\count\Afootins=1200
\count\Bfootins=1200
\count\Cfootins=1200
\pstart%
\noindent%
\hangindent=7.5mm%
%\hspace{7,5mm}
Ita guttae liquorum.
Vid. \edtext{experim. florentina,}{\lemma{experim. florentina}\Cfootnote{\cite{00143}\textsc{L. Magalotti}, \textit{Saggi di naturali esperienze}, Florenz 1666, S.~XXIIIff.
Siehe hierzu \textit{LSB} VIII, 1 N. 37.\cite{01100}}}
quibus probatur non esse ab aeris pressu. Korn im abtreiben.\\
%\pend
%\pstart%
%\noi\hangindent=15mm%
\hspace*{-7,5mm}%
Orbiculi pinguedinis in aqua natantis egregie rotundi.
Si a forma rotunda dimoveas ad eam redeunt.
Ita ut rotunditate sua gravitatis,\protect\index{Sachverzeichnis}{gravitas}
restitutione Elaterii\protect\index{Sachverzeichnis}{elaterium} umbram exhibeant.
\pend
\pstart%
\noindent%
\hangindent=7.5mm%
%\hspace{7,5mm}
Bullae aeris\protect\index{Sachverzeichnis}{bulla aeris}
\edtext{rotundae, fiant}{\lemma{rotundae,}\Bfootnote{\textit{(1)}\ sumant \textit{(2)}\ fiant \textit{L}}} ex aqua saponata.\\
%\pend
%\pstart%
%\noindent%
%\hangindent=7.5mm%
\hspace*{-7,5mm}Pentagonum factum ope quadrati; et hexagonum ope pentagoni:\\
par l'\'{e}quarissoir.\\
%\pend
%\pstart%
%\noindent%
\hangindent=7.5mm%
\hspace*{-7,5mm}Figurae cristallisationum, gemmarum, lapidum,
de quibus quaedam non inelegantia in Davissonii\protect\index{Namensregister}{\textso{Davison}, William 1593-1669} 
\edtext{libro de igne.}{\lemma{libro de igne}\Cfootnote{\cite{00507}\textsc{W. Davison}, \textit{Philosophia pyrotechnica}, Paris 1640, S.~184ff., bes. S.~208f.}}\\
%\pend
%\pstart%
%\noindent%
\hangindent=7.5mm%
\hspace*{-7,5mm}Undae\protect\index{Sachverzeichnis}{unda} quae in charta undulata
\edtext{turkisch papier}{\lemma{turkisch papier}\Bfootnote{\textit{erg. L}}}
conspiciuntur, factae motu aquae pectinatae, cui liquores innatabant.
\edtext{Experimentum Hugenii\protect\index{Namensregister}{\textso{Huygens} (Hugenius, Ugenius, Hugens, Huguens), Christiaan 1629-1695}}{\lemma{Experimentum Hugenii}\Cfootnote{\cite{00062}\textsc{C. Huygens}, \textit{Extrait d'une lettre}, \textit{JS} (1672), S.~133-140 (\cite{00113}\textit{HO} VII, S. 201-206).
Siehe hierzu \cite{01069}\textit{LSB} VIII, 1 N. 39.}}
in vasis gyrati \edlabel{amoenior5}fundo.\\
%\pend
%\pstart%
%\noindent%
\hangindent=7.5mm%
\hspace*{-7,5mm}\edtext{Ascensus per \edlabel{amoenior6}descensum}{{\xxref{amoenior5}{amoenior6}}\lemma{fundo.}\Bfootnote{ \textit{(1)}\ Descensus \textit{(2)}\ Ascensus per descensum \textit{L}}} in Cochlea Archimedea\protect\index{Sachverzeichnis}{cochlea Archimedea}.
\pend
\pstart%
\noindent%
\hangindent=7,5mm%
Elegantes formae, quas singulari quodam delectu vitrarii et pavimentarii sive tessellifices sola dispositione conciliant.
\pend
\pstart%
\noindent%
%\hangindent=7,5mm%
Geometria Sartorum.\\
%\pend
%\pstart%
%\noindent%
%\hangindent=7,5mm%
De linea recta par le moyen de la filiere et per tornum.\\
De dividendis instrumentis par la \edtext{canetille.
\newline%
\edtext{Wrenni Hyperbola per Tornum.%
%
}{\lemma{Wrenni [...] Tornum}\Cfootnote{C. \textsc{Wren}, \textit{Generatio corporis cylindroidis hyperbolici}, \textit{PT} 4 (1669), S. 961f.,\cite{01059} sowie \textit{A Descritption of C. Wren's Engin, designed for grinding Hyperbolical Glasses}, \textit{PT} 4 (1669), S. 1059f.\cite{01064}}}%
%
}{\lemma{canetille.}\Bfootnote{ \textit{ (1) }\ Hyperboloeidum \textit{ (2) }\ Hyperbolae descriptione per Wrennu \textit{ (3) }\ Wrenni [...] Tornum \textit{ L}}}
\newline%
%\pend
%\pstart%
%\noindent%
% \hangindent=7,5mm%
%De dividendis instrumentis par la \edtext{canetille.
%% \pend
%% \pstart%
%% \noindent%
%% \hangindent=7,5mm%
%\newline
%Wrenni Hyperbola per Tornum}{{\lemma{canetille.}\Bfootnote{ \textit{ (1) }\ Hyperboloeidum \textit{ (2) }\ Hyperbolae descriptione per Wrennu \textit{ (3) }\ Wrenni [...] Tornum \textit{ L}}}{\lemma{Wrenni [...] Tornum}\Cfootnote{C. \textsc{Wren}, \textit{Generatio corporis cylindroidis hyperbolici}, \textit{PT} 4 (1669), S. 961f.,\cite{01059} sowie \textit{A Descritption of C. Wren's Engin, designed for grinding Hyperbolical Glasses}, \textit{PT} 4 (1669), S. 1059f.\cite{01064}}}}.\\
%\pend
%\pstart%
%\noindent%
%\hangindent=7,5mm%
Hyperbola par la fus\'{e}e.\\
%\pend
%\pstart%
%\noindent%
%\hangindent=7,5mm%
Parabola, Ellipsis Hyperbola, ope flexionis.\\
%\pend
%\pstart%
%\noindent%
%\hangindent=7,5mm%
Ellipses, des arcades et \edtext{\textit{de la coupe des pierres.}}{\lemma{\textit{coupe des pierres}}\Cfootnote{\cite{01101}\textsc{A. Bosse}, \textit{La pratique du trait à preuves pour la coupe des pierres}, Paris 1643.}}
\pend
%\newpage
\pstart%
\noindent%
%\hangindent=7,5mm%
\edtext{Descriptio lineae Logarithmicae mea.}{\lemma{\hspace{-1mm}Descriptio [...] mea}\Cfootnote{\cite{01161}\textit{LSB} VII, 3 N. $38_{11}$ S.~481 (Z.~21-23)\ \textendash\ $38_{14}$ S.~511.\hspace{-2mm}}}\\
% \edtext{}{\lemma{mea}\Cfootnote{Vgl. VII,3 S. 484-496.}} PR: Beliebig, nicht wirklich pertinent.
%\pend
%\pstart%
%\noindent%
%\hangindent=7,5mm%
\edtext{Libella per Bullam aeris Thevenotiana.\protect\index{Namensregister}{\textso{Th\'{e}venot}, Melchis\'{e}dec 1620-1692}}{\lemma{\hspace{-1mm}Libella [...] Thevenotiana}\Cfootnote{\cite{01102}\textsc{M. Thévenot}, \cite{00135}\textit{Machine nouvelle}, \textit{JS}, 15. November 1666, S.~439-443. Siehe hierzu \textit{LSB} VIII, 1 N.~11, S.~103, Z.~3, Erl.}}\\
%\pend
%\pstart%
%\noindent%
%\hangindent=7,5mm%
De circulis qui in aqua aut alio liquore injecto lapillo nascuntur.\\
%\pend
%\pstart%
%\noindent%
%\hangindent=7,5mm%
Quomodo Vitri-\edtext{fusores}{\lemma{Vitri-}\Bfootnote{\textit{(1)}\ fusoris \textit{(2)}\ fusores \textit{L}}} oris flatu forment vitra.
\pend
\count\Bfootins=1200
\count\Cfootins=1000
\pstart%
\noindent%
\hangindent=7,5mm%
\edtext{De Huddenianis\protect\index{Namensregister}{\textso{Hudde}, Johan 1628-1704)} Lentibus, physico artificio tornatis.}{\lemma{De Huddenianis [...] tornatis}\Cfootnote{\cite{00125}\textsc{J. Hudde}, \textit{Specilla circularia}, \textit{Studia Leibnitiana} 27 (1995), S. 113-121.
% Hudde produzierte 1663 kugelf\"{o}rmige Linsen, 1665 Zusammenarbeit mit Spinoza an Teleskop-Linsen.
Siehe hierzu \cite{01104}\textit{LSB} VIII, 1 N. 19.}}
\edlabel{amoenior1}%
Addatur \edtext{P.~Pardies\protect\index{Namensregister}{\textso{Pardies}, Ignace Gaston 1636-1673}
De omnis generis instrumentis Geometricis.%
}{\lemma{Pardies [...] Geometricis.}\Cfootnote{\cite{00508}\textsc{I.G. Pardies}, \textit{Elemens de geometrie}, Paris 1671. In Buch IX (\textit{Problèmes ou la géométrie pratique}, S. 96-116) werden geometrische Werk\-zeuge dargestellt wie Proportionalkompass, Alidade, Quadrant usw.}}
\edtext{Hookii\protect\index{Namensregister}{\textso{Hooke}, Robert 1635-1703} tornus \edlabel{amoenior2}dioptricus.%
}{\lemma{Hookii [...] dioptricus}\Cfootnote{\cite{00061}\textsc{R. Hooke}, \textit{Micrographia}, London 1665.
Hookes Schleifmaschine wird am Ende der (unpaginierten) Vorrede dargestellt.}}%
\edtext{}{{\xxref{amoenior1}{amoenior2}}\lemma{Addatur [...] dioptricus.}\Bfootnote{\textit{erg. L}}}
\pend
\pstart%
\noindent%
%\hangindent=7,5mm%
De Tornatoria arte, \edtext{vide Brucstorf.}{\lemma{vide Brucstorf}\Cfootnote{Hinweis nicht nachgewiesen.}}\\
%\pend
%\pstart%
%\noindent%
%\hangindent=7,5mm%
De annulis sibi inclusis, ut modus non appareat.\\
%\pend
%\pstart%
%\noindent%
%\hangindent=7,5mm%
De artificio puerorum, quo fila digitis implicata educunt.\\
%\pend
%\pstart%
%\noindent%
%\hangindent=7,5mm%
De linea quam describunt Lapilli ita jacti, ut aliquot per aquam Subsultationes exerceant.\\
%\pend
%\pstart%
%\noindent%
%\hangindent=7,5mm%
De Geometria apum, \edtext{et aranearum}{\lemma{et}\Bfootnote{\textit{(1)} araneae \textit{(2)} aranearum \textit{L}}}.
\edtext{Vid.
[Thevenotium].\edtext{}{\lemma{Thevenotius}\Bfootnote{\textit{L ändert Hrsg.}}}\protect\index{Namensregister}{\textso{Th\'{e}venot}, Melchis\'{e}dec 1620-1692}}{\lemma{Vid. [Thevenotium]}\Cfootnote{\cite{00106}\textsc{M. Thévenot}, \textit{Discours sur l'art de la navigation}, S.~24-27, in \textit{Recueil de voyages}, Paris 1681. Es ist anzunehmen, dass Teile dieses Werkes Thévenots schon früher veröffentlicht wurden.}}\\
%\pend
%\pstart%
%\noindent%
%\hangindent=7,5mm%
De Textoria arte. De omnis generis telis. Velours etc.\\
%\pend
%\pstart%
%\noindent%
%\hangindent=7,5mm%
De \edtext{l'instrument}{\lemma{De}\Bfootnote{\textit{(1)}\ instrumento \textit{(2)}\ Tel \textit{(3)}\ l'instrument \textit{L}}} des bas de soye.
\pend
\pstart%
\noindent%
\hangindent=7,5mm%
De divisione \edtext{methodo Florentinorum}{\lemma{methodo Florentinorum}\Bfootnote{\textit{erg. L}}}
instrumenti ope Cochleae cylindraceae circumductae e longinquo.\\
%\pend
%\pstart%
%\noindent%
%\hangindent=7,5mm%
\hspace*{-7,5mm}\edtext{De arte metiendi optica ex una Statione mea}{\lemma{De [...] mea}\Cfootnote{\cite{01157}\textit{LSB} VIII, 1 N. $14_2$, S.~27f.}}
et \edtext{Lanae.\protect\index{Namensregister}{\textso{Lana de Terzi}, Francesco 1631-1687}%
}{\lemma{Lanae}\Cfootnote{\cite{00069}\textsc{F. Lana}, \textit{Prodromo}, Brescia 1670.
Siehe hierzu \cite{01103}\textit{LSB} VIII, 1 N. 16.}}\\
%De arte metiendi optica
%\edtext{ex una Statione mea}{\lemma{ex [...] mea}\Cfootnote{Hinweis nicht nachgewiesen.}}
%et \edtext{Lanae.\protect\index{Namensregister}{\textso{Lana de Terzi}, Francesco 1631-1687}%
%}{\lemma{Lanae}\Cfootnote{\cite{00069}\textsc{F. Lana}, \textit{Prodromo}, Brescia 1670.
%Siehe hierzu \cite{01103}\textit{LSB} VIII, 1 N. 16.}}\\
%\pend
%\pstart%
%\noindent%
%\hangindent=7,5mm%
\hspace*{-7,5mm}De Terebra \edtext{in circulum eunte}{\lemma{in circulum eunte}\Bfootnote{\textit{ erg. L}}}
\edtext{de qua mihi locutus Helmontius,\protect\index{Namensregister}{\textso{Van Helmont}, Johan Baptista 1580-1644}%
}{{\lemma{}\Bfootnote{locutus \textbar\ dictu \textit{gestr.}\ \textbar\ Helmontius, \textit{L}}}%
{\lemma{de qua [...] Helmontius}\Cfootnote{Hier handelt es sich anscheinend um eine mündliche Überlieferung.}}}
et \edtext{quam habere ait Servierius.\protect\index{Namensregister}{\textso{Grollier de Servi\`{e}res}, Nicolas 1596-1689}}{\lemma{quam [...] Servierius}\Cfootnote{Der von Leibniz erwähnte Kreisbohrer dürfte sich in dem bekannten, u.a. von Ludwig XIV. besuchten Kabinett von de Servières befunden haben.}}
\pend
\pstart%
\noindent%
\hangindent=7,5mm%
%\hspace*{-7,5mm}
De modo punctandi: inventum elegans cujusdam von Siegen\protect\index{Namensregister}{\textso{Siegen}, Ludwig von 1609-1680}, quod ad suam \edtext{Calogra$\phi$icen}{\lemma{Calogra$\phi$icen}\Cfootnote{L. v. Siegen gilt als Erfinder der sog. Schabkunst oder Mezzotinto.\hspace{30mm}}} applicuit.\\
%\pend
%\pstart%
%\noindent%
%\hangindent=7,5mm%
\hspace*{-7,5mm}De figura rotunda apparente, luminosi cujuslibet e longinquo visi.\\
%\pend
%\pstart%
%\noindent%
%\hangindent=7,5mm%
\hspace*{-7,5mm}\edtext{\textit{De Sole Elliptico};
Scheinerus\protect\index{Namensregister}{\textso{Scheiner}, Christoph 1573-1650}%
}{\lemma{\textit{De Sole} [...] Scheinerus}\Cfootnote{\textsc{C. Scheiner}, \textit{Sol ellipticus}, Augsburg 1615.\cite{00342}}} et alii.\\
%\pend
%\pstart%
%\noindent%
%\hangindent=7,5mm%
\hspace*{-7,5mm}De opticis et catoptricis, dioptricis lusibus; e Geometria \edlabel{amoenior7}petitis.
Inprimis \edtext{Niceronii,\protect\index{Namensregister}{\textso{Niceron}, Jean-Fran\c{c}ois 1613-1646}
quomodo eadem res varia apparet ex diverso loco.%
}{\lemma{Niceronii [...] loco}\Cfootnote{\cite{00509}\textsc{J.-F. Niceron}, \textit{Thaumaturgus opticus}, Paris 1646.}}
\pend
\pstart%
\noindent%
\hangindent=7,5mm%
De crucibus\edlabel{amoenior8}\edtext{}{{\xxref{amoenior7}{amoenior8}}\lemma{}\Bfootnote{petitis. \textbar\ Inprimis [...] loco. \textit{erg.} \textbar\ Pons Catulli, Palladii,\protect\index{Namensregister}{\textso{Palladio} (Andrea di Pierro della Gondola) 1508-1580} etc. \textit{gestr.}\ \textbar\ De crucibus\textit{ L}}}
aliisque formis crystallinis ope ligni
\edtext{quercini,}{\lemma{quercini}\Bfootnote{\textit{erg. L}}}
ita ut volumus figurati et in aluminosa aqua mersi.\\
%\pend
%\pstart%
%\noindent%
%\hangindent=7,5mm%
\hspace*{-7,5mm}De prodigiosarum crucium causa physica
\edtext{dissertatio P. Kircheri.\protect\index{Namensregister}{\textso{Kircher}, Athanasius 1602-1680}}{\lemma{dissertatio P. Kircheri}\Cfootnote{\cite{00345}\textsc{A. Kircher}, \textit{De prodigiosis crucibus}, Rom 1661.}}
[151~v\textsuperscript{o}]
\pend
\count\Afootins=1500
\count\Bfootins=1500
\count\Cfootins=1500