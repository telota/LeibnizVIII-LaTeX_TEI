\hspace{-0.7mm}[14~v\textsuperscript{o}] \textit{that is fixed unter the table and frame of the quadrant\protect\index{Sachverzeichnis}{quadrans}, and of the other end a counterpoise of lead, tho ballance the weight of the whole apparatus, about the Quadrant\protect\index{Sachverzeichnis}{quadrans} upon the middle line, of the long Axis, then the Table and Quadrant\protect\index{Sachverzeichnis}{quadrans} is rectifi'd, so as to lye in the plain of the two celestial objects, whether planets or fixt stars, and by the small screws in the sockets, it is fixt in that plain.}\edtext{}{\lemma{\textit{plain}.}\Cfootnote{a.a.O., S. 67.}} Alia requisita \edtext{adaptationis}{\lemma{}\Bfootnote{adaptationis \textit{erg. L}}} facile fiunt ope exiguarum cochlearum in quadrante\protect\index{Sachverzeichnis}{quadrans} ipso. Tabula accommodata plano objectorum cum quadrante\protect\index{Sachverzeichnis}{quadrans} \textit{on it,}\edtext{}{\lemma{\textit{on it,}}\Cfootnote{a.a.O., S. 67.}} et omnibus exacte satis aequilibratis ope ponderum sub Tabula, et fixis dioptris, ad unum dictorum objectorum directis, dicta tabula et instrumentum continuant in eodem plano manere sine ulla alia incommoditate observatoris quanquam objecta semper mutent locum; et fixa dioptra\protect\index{Sachverzeichnis}{dioptra} manet directa ad objectum \edtext{unum}{\lemma{}\Bfootnote{unum \textit{erg. L}}} donec ope mobilis quaeratur objectum alterum. Quod ut efficiatur motu Tabulae et instrumenti; Horologium adaptatum est axi, quod eodem tempore ipsum revolvi faciat quo terra absolvit motionem diurnam, et ita sequetur semper motum apparentem stellarum fixarum quod ita fiet: circa partes quasdam Axis, ubi locus videbitur commodissimus fixetur octava pars rotae radii tripedalis, ejus \textit{Rim}\edtext{}{\lemma{\textit{Rim}}\Cfootnote{a.a.O., S. 68.}} (margo puto) exacte tornetur, et acies ejus secetur in dentes 360. tot enim sunt dimidia horae minuta, in octava parte integrae revolutionis, quanquam haec minuta \edtext{[halve]}{\lemma{}\Bfootnote{harve \textit{L \"{a}ndert Hrsg.}}} horae, quae respiciunt fixas, notabiliter breviores erunt quam solares. \textit{Then fit a worm or screw}\edtext{}{\lemma{\textit{screw}}\Cfootnote{a.a.O., S. 68.}} at \edtext{[hos]}{\lemma{}\Bfootnote{has \textit{ L \"{a}ndert Hrsg.}}} dentes, ita ut una \edtext{revolutione cochleae}{\lemma{revolutione}\Bfootnote{\textit{(1)}\ spirae \textit{(2)}\ cochleae \textit{L}}} facta in semiminuto, faciat \edtext{[unum]}{\lemma{}\Bfootnote{unam \textit{L \"{a}ndert Hrsg.}}} dentem moveri prorsum, \textit{the revolution of the worm is adjusted by a circular pendulum, wich is carried round by a flie}\edtext{}{\lemma{}\Afootnote{\textit{Oberhalb a flie}: un volant \Denarius\vspace{-6mm}}}\textit{, moved in the form of a one wheel'd jack from a swash toothed wheel,} \edtext{\textit{fastned upon}}{\lemma{\textit{fastned}}\Bfootnote{\textit{(1)}\ \textit{about} \textit{(2)}\ \textit{upon} \textit{L}}} \textit{the shank of the worm or screw above mentionn'd; the weight that carries round this wheel must hang, upon the shank of the worm, and must be about a 3}\textsuperscript{\textit{d}} \textit{or 4}\textsuperscript{\textit{th}} \textit{part of the weight of the quadrant\protect\index{Sachverzeichnis}{quadrans} and Table, that it may carry it round steadily and strongly and the circular pendulum must be so order'd, that the observator may at any time of his observation either shorten or produce the length thereof, so as to make it}
\pend
\newpage
\pstart\noindent \textit{move quicker or slower,}\edtext{}{\lemma{\textit{slower},}\Cfootnote{a.a.O., S. 68.}} pro ut res postulabit, quod fiet, \textit{by sliding the hole upon wich the pendulum makes its conical motion a little higher or lower, without lifting up or letting down the pendulum}\edtext{}{\lemma{\textit{pendulum}}\Cfootnote{a.a.O., S. 68.}} (exaltando aut demittendo cavitatem conicam pondere pendulo non ascendente vel descendente) vel etiam penduli filo nonnihil attracto uffgewunden, vel demisso, ope cylindri circa foramen aut apicem coni in quo movetur pendulum. De modo construendi pendulum circulare nihil amplius dico, et id alteri tempori servo, cum quibusdam aliis circa motum experimentis, quae mihi inveniendi ejus occasio fuere, anno 1665. Qua occasione non possum quin loquor de libro ab Hugenio\protect\index{Namensregister}{\textso {Huygens}, Christiaan (1629-1695)} edito, ubi etiam brevem exhibet descriptionem ejusmodi penduli circularis, me non nominato,\edtext{}{\lemma{nominato,}\Cfootnote{\textsc{C. Huygens}, \textit{Horologium oscillatorium}, Paris 1673, S. 159f.\cite{00123} (\textit{HO} XVIII, S. 360-365). Das erste Projekt einer Uhr mit einem Kreispendel stammt aus dem Jahr 1659. Vgl. dazu: \textit{Projet de 1659 d'une horologe \`{a} pendule conique}, \textit{HO} XVII, S. 85-91.}} perinde ac nihil ea res ad me pertineret, quanquam jam anno 1665. invenerim et in usum transtulerim; communicavi societati regiae ejus tam theoriam quam praxin, et speciatim explicabam isochronam motionem pilae penduli, in superficie parabolica, una cum modo Geometrico et mechanico, efficiendi ut in tali superficie moveatur. Hujus rei testes habeo societatis Regesta,\edtext{}{\lemma{Regesta,}\Cfootnote{\textsc{T. Birch}, \textit{History}, London 1757, Bd.~I, S. 97 und 108\cite{00154}; hier sind Hookes Untersuchungen zum Kreispendel f\"{u}r das Jahr 1666 festgehalten.}} et Morayus\protect\index{Namensregister}{\textso {Moray}, Robert (1608-1673)} dixit se Hugenio\protect\index{Namensregister}{\textso {Huygens}, Christiaan (1629-1695)} scripsisse sed de eo plura postea; (alio tractatu) quando examinabo alia in eo libro contenta, circa descensum corporum gravium inveniendum, et circa inventionem longitudinis locorum, productis aliis rationibus certioribus et practicabilioribus. Haec faciunt 