%\begin{ledgroupsized}[r]{120mm}
%\footnotesize 
%\pstart 
%\noindent\textbf{\"{U}berlieferung:}
%\pend
%\end{ledgroupsized}
%\begin{ledgroupsized}[r]{114mm}
%\footnotesize 
%\pstart \parindent -6mm
%\makebox[6mm][l]{\textit{L}}%
%Konzept: LH XXXV 13, 3 Bl. 81. 1 Bl. 2\textsuperscript{o}. 2 S. Wasserzeichen. Die letzten vier Zeilen auf Bl. 81~ v\textsuperscript{o} am linken Seitenrand, an der oberen Ecke beginnend.
%\pend
%\pstart\parindent -6mm
%\makebox[6mm][l]{\textit{E}}\textsc{H.-J. Hess},\cite{00188} \textit{Die unver\"{o}ffentlichten naturwissenschaftlichen und technischen Arbeiten von G.W. Leibniz aus der Zeit seines Parisaufenthaltes. Eine Kurzcharakteristik}. In: \textit{Studia Leibnitiana}, Supplementa XVII (1978) S. 202-205.
%\\Cc 2, Nr. 1503\pend
%\end{ledgroupsized}
%%\normalsize
%\vspace{5mm}
%
%\begin{ledgroup}
%\footnotesize 
%\pstart
%\noindent\footnotesize{\textbf{Datierungsgr\"{u}nde}: Das Wasserzeichen ist für die Monate März und April 1676 belegt.}
%\pend
%\end{ledgroup}
%\vspace{8mm}
%\pstart
%\begin{center}
%[81~r\textsuperscript{o}] De Arcanis Motus, et Mechanica\protect\index{Sachverzeichnis}{mechanica} ad puram\\ Geometriam\protect\index{Sachverzeichnis}{geometria pura} \edlabel{arcanis1}reducenda
%\end{center} 
\begin{ledgroupsized}[r]{120mm}
\footnotesize 
\pstart 
\noindent\textbf{\"{U}berlieferung:}
\pend
\end{ledgroupsized}
\begin{ledgroupsized}[r]{114mm}
\footnotesize 
\pstart \parindent -6mm
\makebox[6mm][l]{\textit{L}}%
Konzept: LH XXXV 13, 3 Bl. 81. 1 Bl. 2\textsuperscript{o}. 2 S. Wasserzeichen.
% Die letzten vier Zeilen auf Bl. 81~v\textsuperscript{o} am linken Seitenrand, an der oberen Ecke beginnend.
\newline%
Cc 2, Nr. 1503
\pend%
\pstart%
\parindent -6mm
\makebox[6mm][l]{\textit{E}}\textsc{H.-J. Hess},\cite{00188} \glqq Die unver\"{o}ffentlichten naturwissenschaftlichen und technischen Arbeiten von G.W. Leibniz aus der Zeit seines Parisaufenthaltes. Eine Kurzcharakteristik\grqq, \textit{Studia Leibnitiana. Supplementa} XVII (1978), S. 202-205.
\pend%
\end{ledgroupsized}
%\normalsize
\vspace{5mm}
\begin{ledgroup}
\footnotesize 
\pstart
\noindent\footnotesize{\textbf{Datierungsgr\"{u}nde:} Das Wasserzeichen ist für die Monate Februar bis September 1676 belegt.}
\pend
\end{ledgroup}
\vspace{8mm}
\pstart%
\noindent%
[81~r\textsuperscript{o}]
\pend%
\pstart%
\noindent%
\centering%
De Arcanis Motus, et Mechanica\protect\index{Sachverzeichnis}{mechanica} ad puram Geometriam\protect\index{Sachverzeichnis}{geometria pura} \edlabel{arcanis1}reducenda
\pend%
\vspace*{0.5em}%
\count\Bfootins=1000
\count\Cfootins=1000
\pstart%
\noindent%
\edtext{Elementa scientiae\protect\index{Sachverzeichnis}{scientia mechanicae} Mechanicae tum demum \edlabel{arcanis2}perfecta}{{\xxref{arcanis1}{arcanis2}}\lemma{reducenda}\Bfootnote{\textit{(1)} Ut Mech \textit{(2)} \textbar\ Elementa scientiae \textit{erg.} \textbar\  Mechanicae\protect\index{Sachverzeichnis}{mechanica|textit} tum demum \textit{(a)} ad puram Geometriam\protect\index{Sachverzeichnis}{geometria pura|textit} reducta \textit{(b)} perfecta \textit{L}}} videbuntur, cum ex datis sufficientibus, praedici poterit effectus, ope calculi\protect\index{Sachverzeichnis}{calculus} et Geometriae\protect\index{Sachverzeichnis}{geometria}. Hoc vero ut fiat, necesse est ut Leges Motus\protect\index{Sachverzeichnis}{lex motus}, quae hactenus variae visae sunt, ad unum quoddam principium reducantur, cujus ope ad Aequationes quasdam analyticas\protect\index{Sachverzeichnis}{aequationes analyticae} possit veniri. \edtext{Hactenus autem non nisi casus particulares propositos video. Mechanica\protect\index{Sachverzeichnis}{mechanica} ad nostrum usque seculum in sola aequiponderantium consideratione versabatur. Constituta enim semel notione centri gravitatis\protect\index{Sachverzeichnis}{centrum gravitatis}, ejusque usu ab \edtext{Archimede}{\lemma{Archimede}\Cfootnote{\textsc{Archimedes}, \cite{01010}\textit{De aequiponderantibus}.}} ostenso, libris de aequiponderantibus, deque iis quae in humido natant, non erat difficile ostendere, corporibus gravibus\protect\index{Sachverzeichnis}{corpus grave} utcunque compositis, aequilibrium\protect\index{Sachverzeichnis}{aequilibrium} esse, cum centrum gravitatis compositi descendere amplius non potest.}{\lemma{veniri.}\Bfootnote{\textit{(1)} Qui centrum gravitatis primi consideraverunt, aditum ad aequationes mechanicas\protect\index{Sachverzeichnis}{aequatio mechanica} aperuerunt, quoniam ostenderunt semper esse aequilibrium, axe librationis per centrum gravitatis \textbar\ corporis \textit{erg.} \textbar\ transeunte, aequilibrium\protect\index{Sachverzeichnis}{aequilibrium} autem aequationis genus est quoddam. \textit{(a)} Quod principium perfecit \textit{(b)} Archimedes\protect\index{Namensregister}{\textso{},|textit}, cum ostendit \textit{(aa)} locum habere in liquidis \textit{(bb)} corpus natans in humido eousque immergi, donec aquam sibi aequiponder \textit{(c)} Talis Archime \textit{(d)} Centrum autem \textit{(e)} Usum hujus principii, applicationemque ad corpora composita, ostendit Archimedes\protect\index{Namensregister}{\textso{},|textit} praeclaris demonstrationibus,\protect\index{Sachverzeichnis}{demonstratio} de iis quae in humido vehuntur; unde tandem regulam generalem condere non difficile fuit, \textit{(aa)} quod scilicet corpus aliquod non desce \textit{(bb)} corporibus gravibus, utcunque compositis nullum esse \textit{(aaa)} motum a gravi \textit{(bbb)} descensum a gravitate, cum centrum gravitatis compositi, descendere non potest. \textit{(aaaa)} Verum \textit{(bbbb)} Sed nondum his omnis rei Mechanicae\protect\index{Sachverzeichnis}{mechanica} ambitus continebatur; nam et \textit{(2)} Hactenus [...] video. \textit{(a)} Veteres \textit{(b)} Veterum Mechanica\protect\index{Sachverzeichnis}{mechanica} \textit{(aa)} ad corporum \textit{(bb)} ad solam considerationem aequiponderantium reducebatur; \textit{(aaa)} cum enim centrum gravitatis \textit{(bbb)} tota redibit aute \textit{(ccc)}  Mechanica [...] potest. \textit{L}}} Aequilibrium autem genus est quoddam aequationis. Verum quoniam his regulis vis tantum mortua gravium explicatur, non vero impetus\protect\index{Sachverzeichnis}{impetus} ille vivus et validus, qui durante aliquandiu motus libertate corpora etiam ultra aequilibrium effert, ideo de ictu, de acceleratione\protect\index{Sachverzeichnis}{acceleratio}, de oscillationibus\protect\index{Sachverzeichnis}{oscillatio}, de motu projectorum\protect\index{Sachverzeichnis}{motus projectorum} altum \edtext{apud veteres\protect\index{Sachverzeichnis}{veteres}}{\lemma{apud}\Bfootnote{\textit{(1)} omnes \textit{(2)} veteres \textit{L}}} silentium fuit. \edtext{Primus omnium}{\lemma{Primus}\Bfootnote{\textit{(1)} mortaliu \textit{(2)} omnium \textit{L}}} Galilaeus\protect\index{Namensregister}{\textso {Galilei}, Galileo (1564-1642)} mentem \edtext{altius sustulit, et limites ab}{\lemma{altius}\Bfootnote{\textit{(1)} sustulit, et positos a \textit{(2)} sustulit, et limites ab \textit{L}}} Archimede \protect\index{Namensregister}{\textso{Archimedes} (287-212 v. Chr.)} signatos transgressus \edtext{est, compositionibus motuum (quas Archimedes abstractis contemplationibus libaverat), in rerum natura consideratis}{\lemma{est,}\Bfootnote{\textit{(1)} explicata \textit{(2)} compositionibus [...] consideratis \textit{L}}}. Unde praeclara illa de motu uniformiter accelerato, deque compositione motus utriusque, quo curva parabolae describitur; et leges denique pendulorum\protect\index{Sachverzeichnis}{pendulum} quas nostro tempore Hugenius\protect\index{Namensregister}{\textso {Huygens}, Christiaan (1629-1695)} ad summam perfectionem perduxit. Hinc, jam nova quaedam aequatio mechanica\protect\index{Sachverzeichnis}{aequatio mechanica} detecta est, scilicet, corpus idem eandem velocitatem acquirere, si ex eadem altitudine descendat, inclinatione quacunque. 
\pend 
\count\Bfootins=1000
\count\Cfootins=1200
\pstart 
Ab eo \edtext{tempore cogitatum est de generalibus quibusdam principiis Mechanicis\protect\index{Sachverzeichnis}{principium mechanicum} condendis. Et plerique huc ivere, ut dicerent}{\lemma{tempore}\Bfootnote{\textit{(1)} doctis \textit{(2)} cogitatum [...] dicerent \textit{L}}} corporis molem ejus celeritate compensari. Celeritatem autem sumendam in directionis linea, et \edtext{ut complures}{\lemma{ut}\Bfootnote{\textit{(1)} plerique \textit{(2)} complures \textit{L}}} \edtext{enuntiant, iisdem opus esse viribus ut una libra attollatur ad centum pedes, quibus opus est ut centum librae attollantur}{\lemma{enuntiant,}\Bfootnote{\textit{(1)} tantundem opus esse virium ad unam libram attollendam ad centum pedes, quantum ad unam libram attollendam \textit{(2)} iisdem [...] attollantur \textit{L}}} ad unum \edtext{pedem. Satis enim videbant demonstrationes a centro gravitatis\protect\index{Sachverzeichnis}{centrum gravitatis} et aequilibrio petitas, non esse directas et ostensivas; quoniam non sumerentur a causa efficiente, causam autem efficientem phaenomenorum, utique in corporis magnitudine, et velocitate consistere debere, judicatu facile erat. Fassi sunt tamen hypothesin esse tantum probabili ratione et experimentorum\protect\index{Sachverzeichnis}{experimentum} successu nixam, non vero demonstratam; quare cum intimas rerum rationes non tenerent}{\lemma{pedem.}\Bfootnote{\textit{(1)} Fassi sunt tamen haec non nisi probabilia esse, et experimentis satis conformia; verum cum intimas \textbar\ eorum \textit{gestr.} \textbar\ rationes nondum satis, quantum judico, essent assecuti, \textit{(a)} saepe a \textit{(b)} agnoscebant \textit{(2)} Satis [...] non \textit{(a)} peterentur \textit{(b)} sumerentur a causa efficiente, \textit{(aa)} quam non aliam esse ab ipsa motus velocitate, et corporis magnitudine \textit{(bb)} causam [...] ratione \textbar\ et experimentorum successu \textit{erg.} \textbar\ nixam, [...] tenerent \textit{L}}}, mirum non est si in applicandis regulis lapsi sunt, aut certe rem non explicuere. \edtext{Quod ipsi Cartesio\protect\index{Namensregister}{\textso{Descartes} (Cartesius, des Cartes), Ren\'{e} 1596-1650} contigit cum leges}{\lemma{Quod}\Bfootnote{\textit{(1)} circa phaenomena cont \textit{(2)} ipsi Cartesio contigit cum leges \textit{L}}} concursuum tradere suscepisset, nam si secutus fuisset, hoc ratiocinandi filum, poterat eas tradere, prorsus quales nunc phaenomenis consentientes habemus, \edtext{nec materiam aut obstacula exteriora \edlabel{arcanis3}accusasset}{\lemma{nec}\Bfootnote{\textit{(1)} habuisset \textit{(2)} materiam aut obstacula exteriora accusasset. \textit{L}}}. \pend 
\pstart 
\edtext{Ab eo tempore experimentis \edlabel{arcanis4}homines}{{\xxref{arcanis3}{arcanis4}}\lemma{accusasset.}\Bfootnote{\textit{(1)} Ab hoc tempore \textit{(a)} plerique \textit{(b)} viri complures doctrina \textit{(2)} Ab eo tempore experimentis homines \textit{L}}} intentius incubuere, et non pauca eruerunt, quae praedici potuisse certum est, si vero ac generali principio constituto, caetera Geometricis ratiociniis tractata fuissent. Id vero distinctius tradere, et scientiam eadem opera \edtext{novis theorematis\protect\index{Sachverzeichnis}{theorema}}{\lemma{}\Bfootnote{novis \textbar\ mutatis \textit{gestr.} \textbar\ theorematis, \textit{L}}}, ante sumta experimenta conditis, locupletare operae pretium est. 
\pend 
\count\Bfootins=1000
\count\Cfootins=1200
\pstart Quemadmodum in Geometria\protect\index{Sachverzeichnis}{geometria} principium ratiocinandi sumi solet ab aequatione quae est, inter totum et omnes partes; ita in Mechanicis cuncta pendent ab aequatione inter causam plenam et effectum \edtext{integrum. Hinc}{\lemma{integrum.}\Bfootnote{ \textit{(1)} Et quema \textit{(2)} Totum \textit{(3)} Hinc \textit{ L}}} ut axioma Geometriae\protect\index{Sachverzeichnis}{axioma geometriae} primarium est, totum aequale esse omnibus partibus; ita axioma Mechanicae\protect\index{Sachverzeichnis}{axioma mechanicae} primarium \edtext{est, causae plenae, et effectus integri}{\lemma{est,}\Bfootnote{\textit{(1)} effectus tantum potest \textit{(2)} causae plenae, et effectus integri \textit{L}}} eadem potentia est. Utrumque axioma a Metaphysico demonstrandum est. Et illud quidem pendet ex \edtext{definitione totius}{\lemma{definitione}\Bfootnote{\textit{(1)} Majoris \textit{(2)} totius \textit{L}}} partis et aequalis; hoc vero ex definitione causae effectus et potentiae\protect\index{Sachverzeichnis}{potentia}. Explicandum est autem nonnihil, (nam demonstratio multas meditationes metaphysicas ab hoc loco alienas, pulcherrimas tamen requirit) ut intelligatur. Causa plena et effectus integer ita
