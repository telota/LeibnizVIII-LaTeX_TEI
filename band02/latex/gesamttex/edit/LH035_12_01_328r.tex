\begin{ledgroupsized}[r]{120mm}%
\footnotesize%
\pstart%
\noindent%
\textbf{\"{U}berlieferung:}%
\pend%
\end{ledgroupsized}%
%
\begin{ledgroupsized}[r]{114mm}%
\footnotesize%
\pstart%
\parindent -6mm%
\makebox[6mm][l]{\textit{L}}%
Notiz: XXXV 12, 1 Bl. 328-329.
1 Bog. 2\textsuperscript{o}.
6 Z. auf Bl. 328~r\textsuperscript{o}.
Der Bog. überliefert zudem die erste H\"{a}lfte von \textit{LSB} VII, 3 N. 8.
Wasserzeichen auf Bl. 328.\\%
Cc 2, Nr. 529
\pend%
\end{ledgroupsized}%
% \normalsize
\vspace*{5mm}%
%
\begin{ledgroup}%
\footnotesize%
\pstart%
\noindent\footnotesize{\textbf{Datierungsgr\"{u}nde:}
Die Datierung von \textit{LSB} VII, 3 N. 8 wird auch für das vorliegende, auf demselben Bogen überlieferte Stück N. 61 \"{u}bernommen.
Das Wasserzeichen im Textträger ist für die Zeitspanne vom Sommer 1672 bis zum Frühling 1673 belegt.%
}%
\pend%
\end{ledgroup}%
%
\vspace*{8mm}%
\pstart%
\normalsize%
\noindent%
\hangindent=7,5mm%
% [328~r\textsuperscript{o}]
[328~r\textsuperscript{o}] 
\edtext{Libavius\protect\index{Namensregister}{\textso{Libavius}, Andreas 1550-1616}
\textit{Alchem.} lib. 1. c. 14. p. 27.%
}{\lemma{Libavius [...] p. 27}\Cfootnote{\cite{01126}\textsc{A. Libavius}, \textit{Alchemia}, Frankfurt a.M. 1597, S. 26-28.}}
repraesentat
\edtext{ex And. Fachsio\protect\index{Namensregister}{\textso{Fachs}, Modestin ??-1575}%
}{\lemma{ex And. Fachsio}\Cfootnote{\cite{00045}\textsc{M. Fachs}, \textit{Probierb\"{u}chlein}, Leipzig 1595, S. 6%
%, listet Faktoren auf, die das Feuer im Ofen beeinflussen. Vgl. auch die Ausgabe Amsterdam 1669, S. 5-8
. Die Bezeichnung des Autors als \textit{And. Fachsius} ist erkl\"{a}rbar durch Verwechslung mit Andreas Libavius.}}
ea quae calorem\protect\index{Sachverzeichnis}{calor} reddunt irregularem.
\pend%
\pstart%
\noindent%
Florimond \edtext{Rapine.\protect\index{Namensregister}{\textso{Rapine}, Florimond 1579-1646} 1614. 1651.%
}{\lemma{Rapine. 1614. 1651}\Cfootnote{\cite{00085}\textsc{F. Rapine}, \textit{Recueil de tout ce qui s'est fait en l'assembl{\'e}e g{\'e}n{\'e}rale des estats tenus {\`a} Paris en l'an 1614}, Paris 1651.}}
\pend%
\pstart%
\noindent%
Vincent \edtext{Cabot.\protect\index{Namensregister}{\textso{Cabot}, Vincent 1560-1620}
26 lib. Pierre du Bosc.\protect\index{Namensregister}{\textso{Bosc}, Pierre du 1623-1692} 1630.%
}{\lemma{Cabot [...] 1630}\Cfootnote{\cite{00018}\textsc{V. Cabot}, \textit{Les Politiques}, par Pierre Bosc marchand libraire,
Toulouse 1630. Der Text ist eigentlich in 28 Bücher unterteilt.}}
\pend%
\pstart%
\noindent%
\edtext{\textit{De inventione Remediorum}\protect\index{Sachverzeichnis}{remedium}%
}{\lemma{\textit{De inventione Remediorum}}\Cfootnote{\cite{00090}\textsc{S. Santorio}, \textit{De remediorum inventione}, Genf 1631.}}
et \edtext{\textit{methodus}\protect\index{Sachverzeichnis}{methodus} etc.%
}{\lemma{\textit{methodus} etc.}\Cfootnote{\cite{00089}\textsc{S. Santorio}, \textit{Methodus vitandorum errorum omnium qui in arte medica contingunt}, Venedig 1603. Neuausgabe: Venedig 1630.\cite{01127}}}
Logica Medica\protect\index{Sachverzeichnis}{logica medica}.
\pend%