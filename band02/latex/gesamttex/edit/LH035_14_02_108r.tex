[108~r\textsuperscript{o}]
\textit{et la font continuer en\-suite beaucoup de semaines, et estant un medicament innocent, il ne peut servir qu'\`{a} des melancoliques\protect\index{Sachverzeichnis}{m\'{e}lancolique} et \`{a} des hypocondriaques\protect\index{Sachverzeichnis}{hypocondriaque} qui ayment d'estre amus\'{e} tous les jours par des breuuages.}\edlabel{Boccone24}
%
\edtext{%
\textit{La Methode est si approchante de celle de Chiaramonte,\protect\index{Namensregister}{\textso{Chiaramonti}, Scipione 1565-1652} que}
je croy que \edtext{cette poudre}{\lemma{cette}\Bfootnote{\textit{(1)}\ terre de sc \textit{(2)}\ poudre \textit{L}}} est aussi la m\^{e}me avec la terre de Bayra.\protect\index{Sachverzeichnis}{terre de Bayra}
}{\lemma{\textit{La methode} [...] Bayra}\Cfootnote{\cite{00318}a.a.O., S. 233.\hspace{20mm}}}
% Siehe dazu \textsc{G. A. Bianchi}, \textit{Dell' ammirabile facoltà ed effetti della polvere o elixir vitae di Girolamo Chiaramonte}, Florenz 1620, und die Antwort \textsc{G. Chiaramonte}, \textit{Dichiarazione contro il sommario metodo di Gio. Ant. Bianchi}, Genua 1627.
%
\edtext{Il y a encor en \textit{Sicile\protect\index{Ortsregister}{Sizilien} la \textso{poudre del Fondacaro}}\protect\index{Sachverzeichnis}{poudre del Fondacaro} \textit{qui est estim\'{e}e un medicament prodigieux pour les maladies} % }{\lemma{Il y a [...] \textit{maladies}}\Cfootnote{\cite{00318}a.a.O., S. 233.}}\edtext{
%
enracin\'{e}es, elle \textit{est} \edtext{[\textit{distribu\'{e}e}]}{\lemma{}\Bfootnote{\textit{distribu\'{e}} \textit{\ L \"{a}ndert Hrsg. nach Vorlage}}} \textit{par les Jesuites de Sicile\protect\index{Ortsregister}{Sizilien} comme un secret particulier, \`{a} la pesanteur d'une ou deux dragmes,} % }{\lemma{\textit{est} [\textit{distribu\'{e}e}] [...] \textit{dragmes}}\Cfootnote{\cite{00318}a.a.O., S. 233.}}\edtext{
%
elle purge \textit{par en bas et par} en \textit{haut avec irregularit\'{e} et} quelques \textit{fois avec violence.} % }{\lemma{\textit{par en bas} [...] \textit{violence}}\Cfootnote{\cite{00318}a.a.O., S. 234.}}\edtext{
%
Donn\'{e}e \`{a} propos elle pourroit estre utile, mais quelques fois elle \edtext{est pernicieuse.}{\lemma{est}\Bfootnote{\textit{(1)}\ exitiale \textit{(2)}\ pernicieuse. \textit{L}}}
On peut juger que c'est \textit{un melange de matieres arsenicales\protect\index{Sachverzeichnis}{matieres arsenicales} de la matiere de l'antimoine\protect\index{Sachverzeichnis}{antimoine}} % }{\lemma{\textit{un melange} [...] \textit{l'antimoine}}\Cfootnote{\cite{00318}a.a.O., S. 234.}}\edtext{
%
quelques Empyriques de Sicile\protect\index{Sachverzeichnis}{Empyriques de Sicile} l'appellent aussi pierre Bezoar mineral\protect\index{Sachverzeichnis}{Bezoar mineral}. Je croy que cette poudre de fondacaro\protect\index{Sachverzeichnis}{poudre de Fondacaro} a est\'{e} encor deguis\'{e}e \`{a} Rome\protect\index{Ortsregister}{Rom} par les J\'{e}suites du College Romain distribuent une poudre appell\'{e}e polvere diabolica et luy donnent des louanges eminentes, elle fait vomir avec violence, et je croy que c'est celle de Fondacaro.\protect\index{Sachverzeichnis}{poudre de Fondacaro}%
}{\lemma{Il y a [...] Fondacaro}\Cfootnote{\cite{00318}a.a.O., S. 233f. Zitate mit Auslassungen.}}
%
\edlabel{Boccone25}\edtext{}{{\xxref{Boccone25}{Boccone26}}\lemma{\textso{Bezoardicum} [...] \textit{precedantes}}\Cfootnote{\cite{00318}a.a.O., S. 235.}}% \edtext{
\textso{Bezoardicum Minerale}\protect\index{Sachverzeichnis}{Bezoar mineral}
tir\'{e} de l'antimoine\protect\index{Sachverzeichnis}{antimoine}
est d'une vertu diaphoretique\protect\index{Sachverzeichnis}{diaphoretique}. La pharmacoepoea de Londre\protect\index{Sachverzeichnis}{pharmacoepoea de London} en met 4 ou 5 fa\c{c}ons differentes. % }{\lemma{\textso{Bezoardicum} [...] differentes}\Cfootnote{\cite{00318}a.a.O., S. 235.}}
%
\textit{Lazarus Riverius\protect\index{Namensregister}{\textso{Rivi\`{e}re} (Riverus), Lazare 1589-1655}}
\edtext{\textit{met souuent dans}}{\lemma{\textit{met}}\Bfootnote{\textit{(1)}\ \textit{une} \textit{(2)}\ \textit{souuent} \textit{(a)}\ \textit{une e} \textit{(b)}\ \textit{dans} \textit{L}}}
\textit{ces ordonnances sudorifiques une espece de Bezoarticum mineral\protect\index{Sachverzeichnis}{Bezoar mineral}, dont je n'ay point de connoissance, si ce n'est une preparation d'Antimoine\protect\index{Sachverzeichnis}{antimoine} semblable aux precedantes.}\edlabel{Boccone26}
% \edtext{}{{\xxref{Boccone25}{Boccone26}}\lemma{\textit{Lazarus} [...] \textit{precedantes}}\Cfootnote{\cite{00318a.a.O., S. 235.}
% \hspace{3mm} }}
%
\edtext{\textit{Mons. Rasi\c{c}an Apotiquaire et Spagyrique\protect\index{Sachverzeichnis}{Spagyrist} fort estim\'{e} \`{a} Paris\protect\index{Ortsregister}{Paris} soutenoit que l'Ebur fossile\protect\index{Sachverzeichnis}{ebur fossile}, d\'{e}crit par Carolus Clusius\protect\index{Namensregister}{\textso{L'Ecluse} (Clusius), Charles de 1526-1609}, qui est appell\'{e} Lapis Arabicus\protect\index{Sachverzeichnis}{lapis Arabicus} par Caesalpinus\protect\index{Namensregister}{\textso{Caesalpino}, Andrea 1519-1603}, estoit la pierre Bezoar Minerale\protect\index{Sachverzeichnis}{Bezoar minerale} sive Bezoar fossile.\protect\index{Sachverzeichnis}{Bezoar fossile}}}{\lemma{\textit{Mons.} [...] \textit{fossile}}\Cfootnote{\cite{00318}a.a.O., S. 236.}}
%
\edtext{Les Galenistes\protect\index{Sachverzeichnis}{Galen} d'Italie\protect\index{Ortsregister}{Italien} \textit{au lieu du Bezoar Oriental\protect\index{Sachverzeichnis}{Bezoar oriental} de l'animal substituent franchement les dents de poissons de mer alterez et petrifiez, appell\'{e}s vulgairement langues de serpent.\protect\index{Sachverzeichnis}{langues de serpent}}}{\lemma{Les Galenistes [...] \textit{de serpent}}\Cfootnote{\cite{00318}a.a.O., S.~236. 
%, S. 236.
}}
%****
\edlabel{Boccone27}\edtext{}{{\xxref{Boccone27}{Boccone28}}\lemma{\textit{Quelques} [...] trouue}\Cfootnote{\cite{00318}a.a.O., S.~237-240. Zitat mit Auslassungen.}}%
\textit{Quelques vieux Apotiquaires et Medecins de Sicile\protect\index{Ortsregister}{Sizilien} m'ont rapport\'{e} qu'un vieillard nomm\'{e} Andr\'{e} Figluzzo\protect\index{Namensregister}{\textso{Figluzzo}, Andrea} de la ville de Monte Leone\protect\index{Ortsregister}{Monteleone}} \edtext{\textit{situ\'{e}e dans}}{\lemma{\textit{situ\'{e}e}}\Bfootnote{\textit{(1)}\ \textit{de la} \textit{(2)}\ \textit{dans} \textit{L}}} \textit{Calabre\protect\index{Ortsregister}{Kalabrien} interieure fut le premier qui monstra aux habitans de la ville de Catana}\protect\index{Ortsregister}{Katanien}% \edlabel{Boccone28}
%
% \edlabel{Boccone29}\edtext{}{{\xxref{Boccone29}{Boccone30}}\lemma{La pierre [...] Catagne}\Cfootnote{\cite{00318}a.a.O.%, S. 237.
% }}%
la pierre \textso{Bezoard Mineral de Sicile.}\protect\index{Sachverzeichnis}{Bezoar Mineral}\protect\index{Ortsregister}{Sizilien}
Il estoit apotiquaire, et avoit perdu son bien et boutique \`{a} l'embrasement du Vesuve\protect\index{Ortsregister}{Vesuv} 1630. Il se retira en Sicile\protect\index{Ortsregister}{Sizilien}, et demeura 3 ans proche de Catane\protect\index{Ortsregister}{Katanien} dans une contr\'{e}e appell\'{e}e Cortiglio del Porto\protect\index{Ortsregister}{Cortiglio del Porto}. J'ay appris de ses nouuelles et parl\'{e} \`{a} des gens qui l'avoient connu.
Il sortoit de Catagne\protect\index{Ortsregister}{Katanien}% \edlabel{Boccone30} 
%
% \edlabel{Boccone31}\edtext{}{{\xxref{Boccone31}{Boccone32}}\lemma{\textit{alloit} [...] \textit{retournoit}}\Cfootnote{\cite{00318}a.a.O.%, S. 237f.
% }}%
et \textit{alloit dans la}
% \edtext{[\textit{le}]}{\lemma{}\Bfootnote{\textit{la} \textit{\ L \"{a}ndert Hrsg.}}}
\textit{Comt\'{e} de Modica\protect\index{Ortsregister}{Modica} o\`{u} il y avoit un ruisseau o\`{u} ordinairement les femmes alloient laver}
\edtext{[\textit{leurs}]}{\lemma{}\Bfootnote{\textit{leur} \textit{\ L \"{a}ndert Hrsg.}}}
\textit{linges, le dit Andr\'{e} deguis\'{e} en} genre \textit{faisoit feinte d'y aller laver et emplissoit un sac de pierres, et s'en retournoit.} % \edlabel{Boccone32}
%
% \edtext{
Ce qui dura assez long temps. Il parut enfin \`{a} Catane\protect\index{Ortsregister}{Katanien}, fit rapport de la \edtext{chose au}{\lemma{chose}\Bfootnote{\textit{(1)}\ \`{a} \textit{(2)}\ au \textit{L}}} Sr. Andr. Lucca, docteur en Medecine et protomedico, car dans ces lieux personne 
% }{\lemma{Ce qui [...] personne}\Cfootnote{\cite{00318}a.a.O., S. 237f.}}
%
% \edtext{
\edtext{\textit{peut vendre ny distribuer}}{\lemma{\textit{peut}}\Bfootnote{\textit{(1)}\ \textit{distribuer ny} \textit{(2)}\ \textit{vendre ny distribuer} \textit{L}}} \textit{aucune drogue sans permission du premier Medecin;}
% }{\lemma{\textit{peut} [...] \textit{Medecin}}\Cfootnote{\cite{00318}a.a.O., S. 238.}}
%
% \edtext{
et luy donna de la pierre. Ce medecin ne le meprisa pas, fit des experiences dans l'Hospital de Catane\protect\index{Ortsregister}{Katanien}, la trouua dou\'{e}e de vertus, et en permit la distribution ronde et quelques fois ovale, de la grosseur d'un oeuf de pigeon, la couleur le plus souuent blanche, quelques fois un peu cendr\'{e}e.
% }{\lemma{et luy donna [...] cendr\'{e}e}\Cfootnote{\cite{00318}a.a.O., S. 238f.}}
%
% \edtext{
\textit{La surface tantost polie tantost rude, avec des petits boutons comme on voit au fruit appell\'{e} par les italiens Azzarolo\protect\index{Sachverzeichnis}{Azzarolo} et par les Latins Mespilus Aronia}.
Du \textit{goust du bol blanc\protect\index{Sachverzeichnis}{bol blanc d'Armenie} d'Armenie,\protect\index{Ortsregister}{Armenien}}
et de \textit{la terre Lemnie.\protect\index{Sachverzeichnis}{terre Lemnie}
La composition semblable au Bezoar Oriental\protect\index{Sachverzeichnis}{Bezoar Oriental} de l'animal ayant les couches de m\^{e}me:
et au centre de cette pierre Bezoar Mineral\protect\index{Sachverzeichnis}{Bezoar Mineral} on trouue un petit amas de sable, sur quoy la nature produit jusques \`{a} 8 ou 10 couches} %
\edtext{}{\lemma{}\Afootnote{\textit{Am Rand}: NB\vspace{-4mm}}}%
\textit{ainsi que l'on voit au Bezoar de l'animal;\protect\index{Sachverzeichnis}{Bezoar de l'animal}
et ce que l'on remarque icy, est que lors que l'amas de sable, qui est renferm\'{e} comme j'ay rapport\'{e} cy dessus dans le centre de cette pierre est en grand volume,}
% }{\lemma{\textit{La surface} [...] \textit{volume}}\Cfootnote{\cite{00318}a.a.O., S. 239.}}
%
% \edtext{
\textit{alors les couches sont en moindre nombre s\c{c}avoir 4.5.6. plus ou moins} \`{a} mesure \textit{qu'il y a du sable: et les couches sont tantost plus \'{e}paisses tantost plus minces selon la substance du Tophus\protect\index{Sachverzeichnis}{Tophus}, dont elles sont compos\'{e}es}: de plus \textit{les petites pierres ont autant de couches que les plus grosses.}
% }{\lemma{\textit{alors} [...] \textit{grosses}}\Cfootnote{\cite{00318}a.a.O.%, S. 239.}}
%
\edtext{[Elles]}{\lemma{}\Bfootnote{Il\textit{\ L \"{a}ndert Hrsg. nach Vorlage}}} \textit{sont} souuent
% \edtext{
\textit{fort} \edtext{[\textit{differentes}]}{\lemma{}\Bfootnote{differens \textit{L \"{a}ndert Hrsg. nach Vorlage}}} \textit{en poids}, quoyque \textit{d'egale grosseur, \`{a} cause qu'aucunes sont} fort \textit{fragiles et d'autres fixes et dures comme marbre dans les couches.}%
% }{\lemma{\textit{fort} [...] \textit{couches}}\Cfootnote{\cite{00318}a.a.O.%, S. 239.}}
%
\edtext{ J'estime}{\lemma{\textit{couches.}}\Bfootnote{\textit{(1)}\ Car \textit{(2)}\ J'estime \textit{L}}}
les fragiles les meilleures
% \edtext{
\textit{au lieu de l'amas de sable, on y trouue quelques fois de petits cailloux renfermez dans le Centre, quelques fois une matiere semblable au bitumen Judaicum\protect\index{Sachverzeichnis}{bitumen Judaicum}, ou au charbon de pierre et parfois quelque petite coquille.}
% }{\lemma{\textit{au lieu} [...] \textit{coquille}}\Cfootnote{\cite{00318}a.a.O., S. 240.}}
%
% \edtext{
\textit{Proche de la terre de Mililli\protect\index{Sachverzeichnis}{terre de Mililli} j'ay trouu\'{e} dans un lieu appell\'{e} S. Mauro\protect\index{Ortsregister}{S. Mauro} une espece de terre Lemnie\protect\index{Sachverzeichnis}{terre Lemnie}, semblable \`{a} celle de la ville de Nocera\protect\index{Ortsregister}{Nocera}, et de l'isle de Malthe\protect\index{Ortsregister}{Malta}: et nostre pierre Bezoar Mineral\protect\index{Sachverzeichnis}{Bezoar Mineral} estant mise en poudre est approchante \`{a} ces trois especes de terres alexipharmaques\protect\index{Sachverzeichnis}{terre alexipharmaque}. Cette pierre} de \textit{Bezoar Mineral\protect\index{Sachverzeichnis}{Bezoar Mineral}} de Sicile\protect\index{Ortsregister}{Sizilien} se trouue \textit{dans les trois promontoires de la Sicile\protect\index{Ortsregister}{Sizilien}. Mons. Rustici Medecin et philosophe}
% }{\lemma{\textit{Proche} [...] \textit{philosophe}}\Cfootnote{\cite{00318}a.a.O.%, S. 240.}}
%
% \edtext{
et Mons. Caffici apoticaire de Catagne\protect\index{Ortsregister}{Katanien} m'ont donn\'{e} des listes des lieux o\`{u} \edtext{on le trouue.\edlabel{Boccone28}}{\lemma{on}\Bfootnote{\textit{(1)}\ trouue le Bezoar\protect\index{Sachverzeichnis}{Bezoar} \textit{(2)}\ le trouue. \textit{L}}}
% }{\lemma{et Mons. [...] trouue}\Cfootnote{\cite{00318}a.a.O.%, S. 240.}}
%****
Sequentia hujus Epistolae ad Tulpium\protect\index{Namensregister}{\textso{Tulpius}, Nicolaus Petreus 1593-1674} etc. desunt in meo exemplari.
\pend
\count\Bfootins=1500
\count\Cfootins=1500
\count\Afootins=1500