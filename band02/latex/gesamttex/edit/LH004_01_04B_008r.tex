[8~r\textsuperscript{o}]
corpus unus erat canalis.
Testes erant albi, satis magni, sed vix \rule[-4mm]{0mm}{10mm}$\displaystyle\frac{1}{20}$ renum aequantes.
Cornibus matricis \edtext{}{\lemma{26-S.570.1 \hspace{1.8mm}appendebant.}\killnumber\Bfootnote{\textit{(1)}\ Cor erat colorat \textit{(2)}\ Cor erat coloris \textit{L}}}appendebant.
%%%vorsicht dieses edtext ist händisch verkürzt, damit der Seitenumbruch definiert werden kann. das edtext geht eigentlich von appendeband bis cor erat coloris
\pend%
\newpage
\pstart%
Cor erat coloris
satis albi pericardio involutum, sed pars sinistra pulmonum erat
\edtext{valde rubens}{\lemma{valde}\Bfootnote{\textit{(1)}\ rubicans \textit{(2)}\ rubens \textit{L}}}
pars autem dextra superior erat albicans, et inferior paulo magis rubea, non autem tantum quantum pars sinistra, quae minor erat dextra: erant autem hae duae partes plane distinctae et potius infra cor a tergo quam supra.
Sed pericardium si affuit, tam fuit tenue, ut vix notari potuerit.
Cor autem oris cum naso crassitiem aequabat: ejus ventriculi dextri supra sinistrum inflexio videbatur a basis latere
\edtext{sinistro (\phantom)\hspace{-1.2mm}unde}{\lemma{sinistro (\phantom)\hspace{-1.2mm}}\Bfootnote{\textit{(1)}\ ubi \textit{(2)}\ unde \textit{ L}}}
erat truncus aortae versus inferiora reflexus\phantom(\hspace{-1.2mm}) per anteriora versus mucronem dextrae partis revolvi, ubi erat ingressus cavae.
Nempe erat contrarium\edtext{}{\lemma{}\Afootnote{\textit{Am Rand}: \Denarius \vspace{-4mm}}}
a cava deorsum per anteriora sursum in truncum aortae descendentem ascendebat; erat autem hujus dextri ventriculi caro notabiliter magis rubens quam caro sinistri: manifestus et patentissimus erat meatus a dextro ventriculo in truncum aortae descendentem; ascendens autem vix notari poterat.
Aspera arteria erat a summo gutture ad cor usque valde longa, et ubique ejusdem crassitiei; in summo autem, ubi est
\edtext{cartilago scutiformis}{\lemma{cartilago}\Bfootnote{\textit{(1)}\ ensiformis \textit{(2)}\ scutiformis \textit{L}}}
erat multo crassior, nodi instar rotundi; et adhaerebant ei carunculae valde rubentes, quas pro tonsillis sumsi: epiglottis jam satis formata erat, et stylus in os immissus descendit per oesophagum inter spinam dorsi et asperam arteriam situm usque ad intestina. Cerebri substantia plane alba erat et subpallida sed intus in duobus anterioribus erat sanguis concretus nullo modo cerebro permistus[,] oculi pupilla rotunda erat et satis magna licet in adultis sit oblonga, an vero pupilla fuerit vel potius corneae pars transparens, quae ita rotunda apparuit, adhuc dubito, non enim uveam a cornea dividere potui.
Humor crystallinus valde magnus et fere rotundus erat: notavi etiam humorem vitreum sed nullum aqueum.
Omnia \edtext{autem oculi}{\lemma{}\Bfootnote{autem\ \textbar\ autem \textit{gestr.}\ \textbar\ oculi \textit{L}}}
interiora \edtext{valde}{\lemma{valde}\Bfootnote{\textit{erg. L}}}
transparebant: sola tunica exterior, in parte anteriore circa illud foramen rotundum, quod pro pupilla sumebamus nigrescebat, paulatimque minus nigrescebat, et diaphana evadebat versus posteriora, nec dum ulli erant processus citiores, avis alicujus
\edtext{[oculus]}{\lemma{oculis}\Bfootnote{\textit{L \"{a}ndert Hrsg.}}} esse videbatur.
\pend%
\pstart%
Membranae foetum involventes multo ulterius in sinistrum cornu quam in dextrum pertingebant,
adeo ut probem quod inquiunt, mares in dextro foemellas in sinistro latere gestari.
\pend%
\newpage
\pstart%
Hujus vituli crura et pedes non tam extensa erant, quam illa paulo majoris quem \edtext{olim}{\lemma{olim}\Cfootnote{Siehe oben S.~\pageref{pedibusque}.}} videram.
Unde conjicio illa fuisse inflexa initio, et omnium motuum et articulorum rudimenta tunc coepisse; postea autem aqua crescente in utero illa omnia se
\edtext{extendisse, et denuo}{\lemma{extendisse,}\Bfootnote{\textit{(1)}\ donec \textit{(2)}\ et denuo \textit{L}}}
foetu crescente illa se contraxisse.
\pend%
\vspace*{1.0em}% PR: Diesen leeren Zeilenabstand bitte behalten !!!
\pstart%
\noindent%
% \centering%
Observationum Anatomicarum compendium de partibus %\\
inferiori ventre contentis 1637.
\pend%
% \vspace*{0.5em}% PR: Rein provisorisch !!!
\pstart%
% \noindent%
Has omnes, peritonaeum involvit, quod constat membrana satis valida duplici, interiori et exteriore, inter quas renes, et arteria magna et vena cava collocantur, item productiones secundas habet, quibus vasa spermatica praeparantia ac deferentia involvuntur, cumque renes natent in foetus corpore, hinc patet istam membranam nonnisi postea produci.
\pend%
\pstart%
Arteriae umbilicales ab iliacis ad umbilicum venientes, et vena ab umbilico ad hepar;
ostendunt sanguinem a corde per aortam ad ilia primum descendisse, et inde ad umbilicum placentae uteri conjunctum rediisse.
\edtext{Ubi sanguini}{\lemma{Ubi}\Bfootnote{\textit{(1)}\ sanguinis \textit{(2)}\ sanguini \textit{L}}}
matris se permiscens reversus est ad hepar foetus
\edtext{per venam}{\lemma{per}\Bfootnote{\textit{(1)}\ sanguinem \textit{(2)}\ venam \textit{L}}} umbilicalem.
Urachus cum in homine non sit pervius ut in brutis, ostendit hominem minus serosi humoris habere et magis ad avium naturam accedere, quae non mingunt; foetusque ideo tunica allantoide etiam caret.
Connectuntur hae arteriae lateribus vesicae, quae ideo videntur ex eo tantum orta
\edtext{quod sanguis}{\lemma{quod}\Bfootnote{\textit{(1)}\ sanguinis \textit{(2)}\ sanguis \textit{L}}}
\edtext{foetus attingendo}{\lemma{foetus}\Bfootnote{\textit{(1)}\ attingit \textit{(2)}\ attingendo \textit{L}}}
in placenta matris sanguinem aliquid ibi de humiditate sua deposuerit renesque ibi ex eadem causa producti sunt, quippe nondum productis vel saltem auctis intestinis; ilia renes et hepar simul ad umbilicum, et cum illo ad placentam matris pertingebant.
\pend%
\pstart%
Omentum semper connectitur ventriculo, lieni et colo, interdum etiam dia$\phi$ragmati et hepati, caetera propendet veli instar supra intestina anterius:
nec videtur aliunde factum, quam ex vasis quae recipit et fulcit, ut illa in ventriculum, lienem, duodenum et colon deferat;
cum enim intestina nunc vacuentur, nunc inflentur, vasa ista non potuerunt ipsis adhaerere, cumque libera starent, circa ipsa
\edtext{secundae membranae}{\lemma{secundae}\Bfootnote{\textit{(1)}\ omenta eodem modo \textit{(2)}\ membranae \textit{L}}}
ex quibus omentum componitur, eodem modo quo peritonaeum factae sunt.
\pend%
%\count\Bfootins=1500
%\count\Cfootins=1500
%\count\Afootins=1500
\pstart%
Vena portae radices educit varias ex intestinis ventriculo, mesenterio, omentis, pancreate, liene et felle, itemque exiguam ex hepate,
unum etiam nempe vas breve educit e ventriculo per lienem:
Dico autem ipsam ex omnibus illis locis radices emittere, quia in illis arterias comites habet, nempe coeliacam vel mesentericam superiorem vel inferiorem
\pend
\newpage
\pstart\noindent quae in ejus extremitates sanguinem mittant, nempe vas breve arteriale, sanguinem acidum ex splene ad ventriculum defert, et vas breve venale succum ex ventriculo in splenem, ubi acescit, ramos autem% Hier endet Bl. 8r und beginnt Bl. 8v.