[3~r\textsuperscript{o}]%
\pend%
\pstart%
Man mus
\edtext{auf den Zustand der}{\lemma{auf}\Bfootnote{\textit{(1)} die \textit{(2)} den Zustand der \textit{L}}}
ordens Personen, die gemeiniglich gewi{\ss}e ihnen allen gemeine arten zu leben in diaet\protect\index{Sachverzeichnis}{Di\"{a}t}
und allen andren haben achtung geben, und daraus Consequentien ziehen.
\pend%
\pstart%
Man mus probiren ob bey dem Menschen nuzen zu schaffen wenn er solche thiere gem\"{a}stet die auf gewi{\ss}e ma{\ss}e mit Krauten[,] Thieren etc. gespeiset worden.
\pend%
\pstart%
Man mus allerhand Mittel versuchen an gewi{\ss}en menschen ob man sie durch eine richtige Kunst alt machen kan, umb daher ein modell vor andere zu nehmen.
\pend%
\pstart%
Man mus mit M\textsuperscript{r}
%
\edtext{Charas\protect\index{Namensregister}{\textso{Charas}, Moyse 1619-1698}}{\lemma{}\Afootnote{\textit{\"{U}ber} Charas: \Denarius \vspace{-4mm}}}
%
untersuchen causas mortis naturalis\protect\index{Sachverzeichnis}{causa mortis naturalis}, umb zu finden modos
\edtext{prolongandae vitae.}{\lemma{prolongandae vitae}\Cfootnote{\cite{00025}\textsc{M. Charas}, \textit{Sur la vipere}, Paris 1669, S. 129 und 135-137, stellt Vipernfleisch als Mittel der Lebensverl\"{a}ngerung dar.}}
%
\pend%
\pstart%
Man mus die Menschen aufs allergnauste examiniren von dem was sie gern e{\ss}en oder riechen, oder nicht, und die gradus delectationis\protect\index{Sachverzeichnis}{gradus delectationis}.
So mu{\ss} mans auch achtung geben quo genere toni musici quisque magis delectetur, wie bey denen so durch die tarantulas\protect\index{Sachverzeichnis}{tarantula} gestochen worden.
Item nach
\edtext{Platonis\protect\index{Namensregister}{\textso{Platon}, 427-347 v.Chr.} reguln%
}{\lemma{Platonis reguln}\Cfootnote{\cite{01137}\textsc{Platon}, \textit{Politeia} III, 398b-400b; \cite{01138}\textsc{Ders.}, \textit{Nomoi}, 653c-659c.}}
%
cum ait mutata musica mutari rem publicam.
\edtext{Ieder mus achtung geben, was das jenige sey, so ihn in der Welt am meisten delectire.}{\lemma{Ieder [...] delectire.}\Bfootnote{\textit{erg. L}}}
\pend%
\pstart%
Man mus die schedulas mortalitatis\protect\index{Sachverzeichnis}{schedula mortalitatis} in hochste mugliche perfection bringen; und nicht nur in gro{\ss}en stadten sondern uberall aufen lande
\edtext{machen, und dabey die differentias climatum, terrarum, aeris, etc. genau annotiren la{\ss}en%
}{\lemma{machen}\Bfootnote{\textit{(1)} la{\ss}en. \textit{(2)} , und [...] la{\ss}en \textit{L}}}
da werden viel admirable dinge heraus kommen. Gewi{\ss}en personen als dann mus man auftragen inductiones und observationes daraus zu machen.
\pend%
%\newpage
\pstart%
Man mus achtung geben auf effectus Astrologicos, ob zum exempel wahr was man sagt, da{\ss} wenn eine frau durante Eclipsi (solari)\protect\index{Sachverzeichnis}{eclipsis} gebehre, da{\ss} sie und das kind bleibe und was dergleichen traditiones mehr.
\pend%
%\newpage
\pstart%
So mus man auch der Calenderschreiber regeln vom bade, schr\"{o}pfen\protect\index{Sachverzeichnis}{Schr\"{o}pfen} aderla{\ss}en,\protect\index{Sachverzeichnis}{Aderlassen}
so sie auf den mond\protect\index{Sachverzeichnis}{Mond} und himlische Zeichen applicirt examiniren, nach den von
\edtext{Keplero[,]\protect\index{Namensregister}{\textso{Kepler}, Johannes 1571-1630}
Campanella,\protect\index{Namensregister}{\textso{Campanella}, Tommaso 1568-1639}%
}{\lemma{Keplero[,]}\Bfootnote{\textit{(1)} drey \textit{(2)} Campanella, \textit{L}}}
Trew,\protect\index{Namensregister}{\textso{Trew}, Abdias 1597-1669}
und andern gelehrten, vorgeschriebenen wegen.
\pend%
\pstart%
Man mus alle simplicia aus der ganzen welt zusammen kommen la{\ss}en, sie fortpflan\-zen auch an unsern orthen, $\langle$w$\rangle$elches zweifels ohne m\"{u}glich sie exa$\langle$m$\rangle$iniren.%
% Hier folgt Bl. 3v.