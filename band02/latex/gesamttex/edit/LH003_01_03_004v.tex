[4~v\textsuperscript{o}]
\pend
%\newpage
\pstart%
Es sind gewi{\ss}e consensus und communicationes unter den gliedern\protect\index{Sachverzeichnis}{Glieder} so ein lebendiger an sich selbst f\"{u}hlen, nicht aber an andren toden finden kan. Als zum exempel was
\edtext{genitalia\protect\index{Sachverzeichnis}{genitalia} und}{\lemma{genitalia}\Bfootnote{\textit{(1)} mit \textit{(2)} und \textit{L}}}
die planta pedis\protect\index{Sachverzeichnis}{planta pedis} mit dem haupt\protect\index{Sachverzeichnis}{Haupt} f\"{u}r connexionem haben befindet ein ieder. Die planta pedis\protect\index{Sachverzeichnis}{planta pedis} parum fricta facit hoc in capite\protect\index{Sachverzeichnis}{caput} exacte sentiri. Similiter de caeteris instituenda experimenta. Und kan
\edtext{seyn ist auch der Vernunfft gem\"{a}{\ss},}{\lemma{seyn}\Bfootnote{\textit{(1)} das die glie \textit{(2)} ist auch \textit{(a)} die glied \textit{(b)} der Vernunfft gem\"{a}{\ss}, \textit{L}}}
das die gliedma{\ss}en\protect\index{Sachverzeichnis}{Gliedma{\ss}en} die constantem proportionem unter einander halten auch eine mehrere sympathiam mit einander haben.
\pend%
\pstart%
Man mus suchen eine quantitat neuer aphorismorum\protect\index{Sachverzeichnis}{aphorismus} zu machen.
\pend%
\pstart%
Wer einen neuen bishehr unbekandten doch zutreffenden, (saltem plerumque) aphorismum\protect\index{Sachverzeichnis}{aphorismus} findet, soll ein gewi{\ss} praemium haben.%
%
\edtext{}{\lemma{}\Afootnote{\textit{Anschließend, m\"{o}glicherweise nicht von Leibniz' Hand:} Scholzius.\vspace{-8mm}\protect\index{Namensregister}{\textso{Scholz}, Lorenz 1552-1599}}}
%
\pend%
\pstart%
Dergleichen wer eine solidam rationem aphorismorum\protect\index{Sachverzeichnis}{aphorismus} jam notorum rationis antea incompertae finden kan.
\edtext{Hierzu siehe}{\lemma{Hierzu}\Bfootnote{\textit{(1)} bey \textit{(2)} siehe \textit{L}}}
\edtext{Claudium Campensium[,]\protect\index{Namensregister}{\textso{Campensis}, Claudius, Wirkjahre 1556-1562}%
}{\lemma{Claudium Campensium}\Cfootnote{\cite{00019}\textsc{C. Campensis}, \textit{Hippocratis aphorismi en nova interpretatione}, Leiden 1579.}}
%
\edtext{M\textsuperscript{r} de la Chambre,\protect\index{Namensregister}{\textso{Cureau de la Chambre}, Marin 1594-1669}%
}{\lemma{M\textsuperscript{r} de la Chambre}\Cfootnote{\cite{00031}\textsc{M. Cureau de la Chambre}, \textit{Novae methodi pro explicandis Hippocrate et Aristotele specimen}, Paris 1655.}}
%
\edtext{Antimum id  est Honoratum Fabry\protect\index{Namensregister}{\textso{Fabri}, Honor\'{e} 1607-1688}%
}{\lemma{Antimum [...] Fabry}\Cfootnote{Antimus Conygius ist eins der Pseudonyme, unter denen Honor\'{e} Fabri veröffentlichte%: \textsc{A. Conygius}, \textit{Pulvis Peruvianus}, Rom 1655
.}}
%
und andere in \textit{Aphorismos Hippocratis.}\protect\index{Namensregister}{\textso{Hippokrates} um 460-um 370 v.Chr.}
Adde \edtext{novos Aphorismos\protect\index{Sachverzeichnis}{aphorismus} additos \`{a} Laurentio Scholzio\protect\index{Namensregister}{\textso{Scholz}, Lorenz 1552-1599}%
}{\lemma{novos [...] Scholzio}\Cfootnote{\cite{00091}\textsc{L. Scholz}, \textit{Aphorismorum medicinalium sectiones octo}, Breslau 1589.}}
%
etc.
\pend%
\pstart%
Man mus auch proben anstellen was die vires
\edtext{[imaginationis]\protect\index{Sachverzeichnis}{vires imaginationis}}{\lemma{imaginationes}\Bfootnote{\textit{L \"{a}ndert Hrsg.}}}
und glaube des patienten\protect\index{Sachverzeichnis}{Patient} verm\"{o}gen.
Dahehr mu{\ss}en einem Medico\protect\index{Sachverzeichnis}{medicus} K\"{u}nste und mittel an die Hand gegeben werden den patienten\protect\index{Sachverzeichnis}{Patient} zu diesem und jenem zu \edlabel{bereden}bereden.%
\edtext{}{{\xxref{bereden}{mus}}\lemma{bereden.}\Bfootnote{\textit{(1)} Wenn man \textit{(2)} Man mus \textit{L}}}
\pend%
\pstart%
Man mus\edlabel{mus} sonderlich per ratiocinationes communicationes externorum membrorum\protect\index{Sachverzeichnis}{membrum externum} cum internis visceribus\protect\index{Sachverzeichnis}{viscera interna} finden, so kan man durch euserliche applicationen schohn ein gro{\ss}es thuen.
\pend%
\pstart%
Ich zweifle nicht da{\ss} liquores\protect\index{Sachverzeichnis}{liquor} zu finden so per syringem\protect\index{Sachverzeichnis}{syrinx} immissi calculum vesicae\protect\index{Sachverzeichnis}{calculus vesicae} solviren\protect\index{Sachverzeichnis}{solvieren}, auch \raisebox{-0.5ex}{\includegraphics[width=0.02\textwidth]{images/taros.pdf}}rum\protect\index{Sachverzeichnis}{tartarus} podagricum wegnehmen. Wenn diesem Methodo nachgegangen und alles aufgemuntert wird, wollen wir in 10 jahren wunder dinge beysammen haben.
\pend%
\pstart%
Nota alle frictiones: plantae pedis\protect\index{Sachverzeichnis}{planta pedis}, cutis\protect\index{Sachverzeichnis}{cutis} etc. fuhlet man oben in vertice am stercksten wie auch einigen dolorem\protect\index{Sachverzeichnis}{dolor}, wenn man starck druckt. Hinc ibi primum nervorum vel in vicinia.
\pend%
\pstart%
Amara\protect\index{Sachverzeichnis}{amara} guth contra febres\protect\index{Sachverzeichnis}{febris}
\pend%
\pstart%
acida\protect\index{Sachverzeichnis}{acida} guth contra pestem\protect\index{Sachverzeichnis}{pestis}.
\pend%
\pstart%
Ob die jenigen eines humoris\protect\index{Sachverzeichnis}{humor} seyn, so einerley exerrationes symmetriae certarum partium a symmetria ordinaria haben.
\pend%
\pstart%
Man mus probiren alle sorten der liquorum\protect\index{Sachverzeichnis}{liquor} sanguini\protect\index{Sachverzeichnis}{sanguis} injectorum.
\pend%
\pstart%
Man mus nicht aufhohren Proben, mit transfusione sanguinis\protect\index{Sachverzeichnis}{transfusio sanguinis} zu thuen, zum wenigsten in thieren, wie denn
\edtext{in England\protect\index{Ortsregister}{England} ein mattes pferd\protect\index{Sachverzeichnis}{Pferd} durch frisches Hamels-blut\protect\index{Sachverzeichnis}{Hammelblut} wieder kr\"{a}ftig worden.%
}{\lemma{in England [...] worden}\Cfootnote{Eine Bluttransfusion zwischen Tieren verschiedener Arten soll Jean-Baptiste Denis durchgeführt haben; siehe \cite{00158}\textit{PT}, II (1666), S. 559.}}
\pend%
\pstart%
Man mus probiren varia genera balneorum, denn alle
\edtext{balnea\protect\index{Sachverzeichnis}{balneum} sunt}{\lemma{balnea}\Bfootnote{\textit{(1)} sind \textit{(2)} sunt \textit{L}}}
quoddam genus infusionis per poros.\protect\index{Sachverzeichnis}{infusio per poros}
\pend%
\pstart%
Item varia genera oleorum\protect\index{Sachverzeichnis}{oleum} der salbung et eorum quae capiti\protect\index{Sachverzeichnis}{caput} aut alibi externe imponuntur. Item varias modificationes respirationis\protect\index{Sachverzeichnis}{respiratio} per varietatem aeris attracti.
\pend%
\pstart%
Item effectus varios immissorum variorum liquorum\protect\index{Sachverzeichnis}{liquor} per clysterem\protect\index{Sachverzeichnis}{clystera} in anum\protect\index{Sachverzeichnis}{anus} aut per syringem\protect\index{Sachverzeichnis}{syrinx} in pudenda\protect\index{Sachverzeichnis}{pudenda}.
\pend%
\pstart%
Item mit schropfen\protect\index{Sachverzeichnis}{Schr\"{o}pfen} konte mans so anstellen, da{\ss} der vollgezogene laskopf\protect\index{Sachverzeichnis}{Laskopf} ab und gleich etwas anders voll liquoris\protect\index{Sachverzeichnis}{liquor} den das corpus hingegen wieder an sich z\"{o}ge applicirt wurde. Item da{\ss}
\edtext{man aliquid}{\lemma{man}\Bfootnote{\textit{(1)} ein \textit{(2)} aliquid \textit{L}\hspace{-2mm}}}
liquore\protect\index{Sachverzeichnis}{liquor} certo plenum cuti\protect\index{Sachverzeichnis}{cutis} applicirt,
und denn las\protect\index{Sachverzeichnis}{Laskopf}
\edtext{k\"{o}pfe (geschlagen\phantom)\hspace{-1.2mm}}{\lemma{k\"{o}pfe (\phantom)\hspace{-1.2mm}}\Bfootnote{\textit{(1)} die \textit{(2)} geschlagen \textit{L}\hspace{-2mm}}}
oder ungeschlagen\phantom(\hspace{-1.2mm}) herumb applicirt, wurde sich dieses be{\ss}er in leib ziehen.
Mit varie applicirten las kopfen\protect\index{Sachverzeichnis}{Laskopf} kan am besten consensus partium\protect\index{Sachverzeichnis}{consensus partium} fuhlen. Man kan auch sanguini\protect\index{Sachverzeichnis}{sanguis} nicht nur infundiren liquores\protect\index{Sachverzeichnis}{liquor}, sondern auch inseriren corpora sicca.
\pend%
\pstart%
Man kondte sanguinem\protect\index{Sachverzeichnis}{sanguis}
\edtext{transfundendum erst}{\lemma{transfundendum}\Bfootnote{\textit{(1)} est \textit{(2)} erst \textit{L}\hspace{-2mm}}}
variis infusionibus\protect\index{Sachverzeichnis}{infusio} oder compressionibus nach belieben temperiren.
\pend%
\pstart%
Vasa corrosa\protect\index{Sachverzeichnis}{vas corrosum}, (si haec causa mortis naturalis\protect\index{Sachverzeichnis}{causa mortis naturalis}) sind nicht be{\ss}er als durch gewi{\ss}e balnea\protect\index{Sachverzeichnis}{balneum} zu stercken.
\pend%
\pstart%
Scribendae exactissimae historiae omnium longaevorum.
Adde \edtext{Meybomium\protect\index{Namensregister}{\textso{Meibom}, Heinrich, der J\"{u}ngere 1638-1700} \textit{De Longaevis.}%
}{\lemma{Meybomium \textit{De Longaevis}}\Cfootnote{\cite{00077}\textsc{H. Meibom}, \textit{Epistolae de longaevis}, Helmstedt 1664.}}
\pend%
\pstart%
Item omnium historia notanda, qui h$\langle$abent$\rangle$ aliquid extra ordinem, ut apoplexia,\protect\index{Sachverzeichnis}{apoplexia} epilepsia\protect\index{Sachverzeichnis}{epilepsia} etc. accidit.%
% Hier folgt Bl. 5r.