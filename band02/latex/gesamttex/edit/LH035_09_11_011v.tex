[11~v\textsuperscript{o}]
les temps\protect\index{Sachverzeichnis}{temps employ\'{e}} \edtext{dans lesquels ils sont parcourus}{\lemma{}\Bfootnote{dans [...] parcourus \textit{erg. L}}}
sont en raison reciproque des vitesses\protect\index{Sachverzeichnis}{vitesse} du mobile.
Or les vistesses sont en raison des espaces \`{a} parcourir $\displaystyle AE.$ $\displaystyle AB.$ $\displaystyle A(B)$\textso{ par la 1. prop.}
\edtext{Donc les temps dont chaque point du lieu doit estre parcouru}{\lemma{Donc}\Bfootnote{\textit{(1)}\ les augmentations  \textit{(a)}\ des  \textit{(b)}\ du temps \textit{(2)}\ les temps [...] du lieu  \textit{(a)}\ est  \textit{(b)}\ doit estre parcouru \textit{L}}}
seront \edtext{en raison}{\lemma{en}\Bfootnote{\textit{(1)}\ raisons \textit{(2)}\ raison \textit{L}}} \edtext{reciproque}{\lemma{reciproque}\Bfootnote{\textit{erg. L}}}
des espaces\protect\index{Sachverzeichnis}{espace parcouru} qui restent \`{a} parcourir.
\pend
\begin{Geometrico}% PR: Erste Zeile bitte hängend (more geometrico).
\footnotesize \textso{Corollaire[:]} Les dits temps employez \`{a} parcourir
\edtext{chaque endroit}{\lemma{chaque}\Bfootnote{\textit{(1)}\ point ou partie infiniment peti \textit{(2)}\ endroit \textit{L}}}
de l'espace $\displaystyle EA$ pourront estre representez
\end{Geometrico}
\count\Bfootins=900
\vspace{3mm}
\begin{Geometrico}
\edtext{Th. 2. Les m\^{e}mes conditions pos\'{e}es, le temps employ\'{e} croist \`{a} chaque endroit de l'espace en raison reciproque des parties de l'espace qui restent \`{a} parcourir.}{\lemma{Th. 2.}\Bfootnote{\textit{(1)}\ Les augmentations du temps  \textit{(a)}\ \`{a} chaque endroit du lieu sont comme les  \textit{(b)}\ croissent  \textit{(c)}\ parcourus croissent  \textit{(d)}\ employ\'{e}s croissent \`{a} chaque endroit du lieu  \textit{(aa)}\ comme les  \textit{(bb)}\ en raison reciproque des  \textit{(aaa)}\ espaces  \textit{(bbb)}\ parties de l'espace qui restent \`{a} parcourir. \textit{(2)}\ Les m\^{e}mes [...] \`{a} parcourir. \textit{L}}}
\end{Geometrico}
\vspace{2em}
\pstart
\noindent [\textit{Folgender kleingedruckter Text gestrichen:}]
\pend
\vspace{0.5em}
\pstart
\footnotesize
\noindent
Il est manifeste
\edtext{que dans le mouuement il n'y a point partie du lieu si petite, qu'il ne faille du temps pour la parcourir, et par consequent que le temps de la course s'augmente \`{a} chaque endroit du lieu}{\lemma{que}\Bfootnote{\textit{(1)}\ les vitesses se deminuent en chaque endroit du lieu, et par consequent les temps \textit{(2)}\ dans le [...] a point  \textit{(a)}\ de moment o\`{u} l'espace parcouru ne s'augmente, et  \textit{(b)}\ partie du temps si petite, o\`{u} le  \textit{(c)}\ partie du lieu si petite,  \textit{(aa)}\ o\`{u}  \textit{(bb)}\ qu'il ne  \textit{(aaa)}\ va  \textit{(bbb)}\ faille  \textit{(aaaa)}\ quelque  \textit{(bbbb)}\ du temps [...] du lieu. \textit{L}}}\edtext{. Et si cet endroit est un point, ou une partie moindre qu'aucune donn\'{e}e}{\lemma{lieu.}\Bfootnote{\textit{(1)}\ Et si la partie est moindre qu'aucun \textit{(2)}\ Et si [...] une partie  \textit{(a)}\ si petite qu'elle ne puisse estre expliqu\'{e}e en nombres,  \textit{(b)}\ moindre [...] donn\'{e}e, \textit{L}}}\edtext{, la partie du temps}{\lemma{donn\'{e}e,}\Bfootnote{\textit{(1)}\ les augmentations du temps \textit{(2)}\  la partie du temps \textit{L}}}
necessaire \`{a} le parcourir, sera
\edtext{aussi moindre qu'aucune donn\'{e}e}{\lemma{aussi}\Bfootnote{\textit{(1)}\ infiniment petite \textit{(2)}\ moindre qu'aucune donn\'{e}e, \textit{L}}},
et c'est ce
\edtext{que j'appelle}{\lemma{que}\Bfootnote{\textit{(1)}\ nous \textit{(2)}\ j'appelle \textit{L}}}\textso{ l'augmentation du temps \`{a} chaque endroit du }\edtext{\textso{lieu.}}{\lemma{}\Bfootnote{\textso{lieu}\ \textbar\ ou l'accession continuelle au temps \textit{gestr.}\ \textbar\ . \textit{L}}}
Or pour venir \`{a} la proposition, soit l'espace $\displaystyle EA$ divis\'{e} en
\pend
\count\Bfootins=800
\vspace{3mm}
\pstart
\normalsize
\noindent
Soit l'espace $\displaystyle EA$ divis\'{e} en parties $\displaystyle EB.$ $\displaystyle B(B).$ $\displaystyle (B)P$
\edtext{moindres qu'aucune donn\'{e}e}{\lemma{moindres}\Bfootnote{\textit{(1)}\ qu'aucunes donn\'{e}es \textit{(2)}\ qu'aucune donn\'{e}e, \textit{L}}},
et \'{e}gales entre elles.
Je dis que les
\edtext{parties du temps (qui sont aussi moindres que toutes celles qu'on puisse donner)}{\lemma{parties}\Bfootnote{\textit{(1)}\ moindres qu'aucunes \textit{(2) }\ du temps [...] donner) \textit{L}}}
dont elles seront parcourues,
\edtext{et qui sont les augmentations du temps \`{a} chaque endroit de l'espace}{\lemma{}\Bfootnote{et qui [...] l'espace \textit{erg. L}}}
sont en \edtext{raison}{\lemma{}\Bfootnote{raison \textit{erg. L}}}
reciproque des espaces qui restent \`{a} parcourir $\displaystyle AE.$ $\displaystyle AB.$ $\displaystyle A(B)$.
Car \edtext{les dites parties du temps}{\lemma{les}\Bfootnote{\textit{(1)}\ dits temps \textit{(2)}\ dites [...] temps \textit{L}}}
sont en raison reciproque des
\edtext{droites $\displaystyle GE.$ $\displaystyle CB.$ $\displaystyle (C)(B)$ ou des}{\lemma{}\Bfootnote{droites [...] ou des \textit{erg. L}}}
\edtext{vitesses qui sont residues quand le mobile est arriv\'{e} aux dites}{\lemma{vitesses}\Bfootnote{\textit{(1)}\ du mobile dans les  \textit{(a)}\ points $\displaystyle E.$ $\displaystyle B.$ $\displaystyle B.$ \textit{(b)}\ dites \textit{(2)}\ qui sont [...] aux dites \textit{L}}}
parties du \edtext{lieu,
(:~par ce que generalement les espaces estant egaux $\displaystyle EB$, $\displaystyle B(B)$, $\displaystyle (B)P$,
les temps dans lesquels ils sont parcourus sont en raison reciproque des vitesses~:)}{\lemma{lieu,}\Bfootnote{\textit{(1)}\ ou en raison reci \textit{(2)}\ (:~par ce [...] temps\ \textbar\ dans [...] parcourus \textit{ erg.}\ \textbar\ sont [...] vitesses~:) \textit{L}}}\edtext{ or par la prop. 1. les vitesses residues sont en raison reciproque des espaces \`{a} parcourir, donc les dites parties ou augmentations continuelles du temps sont en raison reciproque des espaces qui restent \`{a} parcourir}{\lemma{vitesses~:)}\Bfootnote{\textit{(1)}\ c'est \`{a} dire en raison reciproque des droites $\displaystyle GE.$ $\displaystyle CB.$ $\displaystyle (C)(B)$  \textit{(a)}\ ou des  \textit{(b)}\ et par consequent par la prop. 1. en raison reciproque des droites $\displaystyle AE$ \textit{(2)}\ or par [...] dites parties  \textit{(a)}\ du temps sont en raison reciproque  \textit{(b)}\ ou augmentations [...] \`{a} parcourir. \textit{L}}}.\edlabel{35.09.11_011v_01}
\pend
\begin{Geometrico}
Th. 3.\edlabel{35.09.11_011v_02}\edtext{}{{\xxref{35.09.11_011v_01}{35.09.11_011v_02}}\lemma{parcourir.}\Bfootnote{\textit{(1)}\ \textso{Corollaire} \textit{(2)}\ Th. 3. \textit{L}}} Les dites augmentations du temps \`{a} chaque
\edtext{endroit de l'espace pourront estre}{\lemma{endroit}\Bfootnote{\textit{(1)}\ du lieu sont \textit{(2)}\ de l'espace pourront estre \textit{L}}}
represent\'{e}es par les appliqu\'{e}es
\edtext{$\displaystyle EF.$ $\displaystyle BD.$ $\displaystyle (B)(D)$}{\lemma{}\Bfootnote{$\displaystyle EF.$ $\displaystyle BD.$ $\displaystyle (B)(D)$ \textit{erg. L}}}
de \edtext{l'Hyperbole $\displaystyle FD(D)Q$ men\'{e}es sur $\displaystyle AE.$ l'espace dans lequel tout le mouuement se doit faire, et qui est partie de l'Asymptote de l'Hyperbole, dont le centre $\displaystyle A$ est le point de repos}{\lemma{l'Hyperbole}\Bfootnote{\textit{(1)} \`{a} l' Asymptote, dont \textit{(2)} dont  \textit{(a)} les Asymptote  \textit{(b)} le centre \textit{(3)} $\displaystyle FD(D)Q$ men\'{e}es sur  \textit{(a)} l'Asymptote, $\displaystyle AE$. Le Centre de l'Hyperbole estant le point du repos, et dont  \textit{(aa)} une partie $\displaystyle AE$ est  \textit{(bb)} le point  \textit{(b)} $\displaystyle AE.$ l'espace dans lequel\ \textbar\ tout \textit{erg.} \textbar\ le mouuement se doit faire,  \textit{(aa)}\ et le centre de l'Hyperbole est le point du repos  \textit{(bb)} et qui [...] de repos. \textit{L}}}.
\end{Geometrico}
\pstart
\noindent
Cela se prouue aisement car \edtext{[par]}{\lemma{}\Bfootnote{par \textit{gestr. L} \textbar\ \textit{erg. Hrsg.}}}
la \edtext{proposition precedente}{\lemma{proposition}\Bfootnote{\textit{(1)}\ men\'{e}e \textit{(2)}\ precedente, \textit{L}}},
les augmentations du temps
\edtext{sont en raison}{\lemma{sont}\Bfootnote{\textit{(1)}\ comme \textit{(2)}\  en raison \textit{L}}}
reciproque des espaces $\displaystyle AE.$ $\displaystyle AB.$
\edtext{$\displaystyle A(B)$,
et par consequent ces augmentations estant [represent\'{e}es]\edtext{}{\Bfootnote{represent\'{e}s \textit{\ L \"{a}ndert Hrsg.}}}
par des lignes,
[12~r\textsuperscript{o}]
$\displaystyle EF$, $\displaystyle BD.$}{\lemma{$\displaystyle A(B)$}\Bfootnote{\textit{(1)}\ . Or les dites applique\'{e}es $\displaystyle EF$, $\displaystyle BD$ \textit{(2)}\ . Or le \textit{(3)}\ , et par [...] des lignes, $\displaystyle EF$,  \textit{(a)}\ $\displaystyle AB.$  \textit{(b)}\  $\displaystyle BD.$ \textit{L}}}
$\displaystyle (B)(D)$ les rectangles $\displaystyle AEF$, $\displaystyle ABD.$
[$\displaystyle A(B)(D)$]\edtext{}{\Bfootnote{$\displaystyle A(B)D$ \textit{\ L \"{a}ndert Hrsg.}}}
seront egaux:
Et par consequent la courbe dans la quelle les points $\displaystyle F.$ $\displaystyle D.$ $\displaystyle (D)$ tomberont, sera l'hyperbole.
\pend
