      
               
                \begin{ledgroupsized}[r]{120mm}
                \footnotesize 
                \pstart                
                \noindent\textbf{\"{U}berlieferung:}   
                \pend
                \end{ledgroupsized}
            
              
                            \begin{ledgroupsized}[r]{114mm}
                            \footnotesize 
                            \pstart \parindent -6mm
                            \makebox[6mm][l]{\textit{L}}Aufzeichnung: LH XXXV 15, 6 Bl. 62.
1 Bl. 4\textsuperscript{o}, beschnitten.
2 S. Textfolge: Bl.~62~r\textsuperscript{o}, rechte Sp.; Bl.~62~v\textsuperscript{o}, ganzseitig; Bl.~62~r\textsuperscript{o}, linke Sp.
In der Mitte von Bl. 62~r\textsuperscript{o} quer zum Text eine Mitteilung Hermann Andreas Lassers%
\protect\index{Namensregister}{\textso{Lasser} (Lassart), Hermann Andreas, um 1675} an Leibniz:
\glqq
Ich schicke hierbeij das begerte buch,
wie auch Etliche von den Philosophischen Zettulein,
m\"{o}chte w\"{u}nschen da{\ss} sie der Hr Dr. folgend{\ss} zeichnen k\"{o}nte,
damitt der Hr Eisenwirdt\protect\index{Namensregister}{\textso{Eisenwirdt}, ????} Etwas zu thun bekompt.
E{\ss} ist  bald geschehen.
Neue{\ss} hab diesmahl nicht{\ss}.
Wann der Hr Dr.\protect\index{Namensregister}{\textso{Leibniz}, Gottfried Wilhelm 1646-1716} das Corpus juris nicht mehr brauchet,
m\"{o}chte ich{\ss} gern haben.%
\grqq%
%~Leibnizens \textit{Corpus Juris Civilis reconcinnatum}\cite{00112} gilt heute als verschollen (siehe dazu \textit{LSB} II,~2, S.~XXI-XXIII).
\\KK 1, Nr. 185, Nr. 194 A \pend
\end{ledgroupsized}
                %\normalsize
                \vspace*{5mm}
                \begin{ledgroup}
                \footnotesize 
                \pstart
            \noindent\footnotesize{\textbf{Datierungsgr\"{u}nde}: Im vorliegenden Stück N.~85 % sowie in der Aufzeichnung N.~86 %?? = LH XXXV 15, 6 Bl. 61 = De ratione efficiendi motus uniformes duorum mobilium findet. 
entwickelt Leibniz unter dem Begriff eines \textit{Horologium mera vi elastica}
\"{U}berlegungen zur Stabilisierung des Ganges einer Uhr.
Der ähnliche Gedanke eines \textit{Horologium elasticum},
dessen Ganggenauigkeit nicht von der Rollbewegung eines Schiffes gestört wird,
ist im Stück \cite{01072}\textit{LSB} VIII,~1 N.~6\textsubscript{2} anzutreffen,
welches editorisch auf die zweite H\"{a}lfte 1672 datiert wurde.
Diese Datierung lässt sich aufgrund der inhaltlichen Verwandtschaft auch für N.~85 übernehmen.}
                \pend
                \end{ledgroup}
            
                \vspace*{8mm}
                \pstart 
                \normalsize \noindent
            [62~r\textsuperscript{o}] \\
%            \selectlanguage{ngerman}Ich schicke hierbeij das begerte buch, wie auch Etliche von den Philosophischen Zettulein, m\"{o}chte w\"{u}nschen da{\ss} sie der Hr Dr.\protect\index{Namensregister}{\textso  {Leibniz}, Gottfried Wilhelm (1646-1716)} folgend{\ss} zeichnen k\"{o}nte, damitt der Hr Eisenwirdt\protect\index{Namensregister}{\textso  {},} Etwas zu thun bekompt. E{\ss} ist  bald geschehen. Neue{\ss} hab diesmahl nicht{\ss}.\pend \pstart  Wann der Hr Dr.\protect\index{Namensregister}{\textso  {Leibniz}, Gottfried Wilhelm (1646-1716)} das \edtext{Corpus juris}{\lemma{Corpus juris}\Cfootnote{\textsc{Leibnitii }\cite{00112}\title{Corpus Juris Civilis reconcinnatum}. Der Text gilt heute als verschollen. Vgl. dazu \textit{LSB} II, 2, S. XXI-XXIII.}} nicht mehr brauchet, m\"{o}chte ich{\ss} gern haben.\pend\selectlanguage{latin}
%           \vspace*{4mm}
%\pstart 
            %\begin{center}                   
               \begin{tabular}[t]{llllll} 
               \hspace{20mm} & &\edtext{} 1. minut. & 1. schritt & 30. minut.\\
                \hspace{20mm}60. minuten\\ 
 \hspace{20mm}&& 2. \hspace{3mm}\textemdash & 1. schritt & & \\
 \hspace{20mm}1. minut. & 2. schritt & & & \\
 \hspace{20mm}{1. minut.\renewcommand*{\raggedleftmarginnote}{} \reversemarginpar\marginnote{\scriptsize\hspace{46mm}20}}&1. schritt & & & \\
           \end{tabular}
                                    %\end{center}
\pend
 \vspace*{4mm}
%\newpage
\pstart \noindent Includens \setline{21}absolvit circulum in 1 minut. inclusum in 2. minuta. Ergo \selectlanguage{ngerman}wenn zwey gezeichnete puncte in beyden zugleich aus lauffen, komt der geschwindere  rumb, wann der langsamere  halb. Ergo weil  der langsame  die andere  helfte absolvirt  komt der geschwinde wieder einmahl  rumb und erholen einander wieder
%\selectlanguage{latin}
 in loco priore  post 2. revolutiones. Ergo si  augeatur utriusque celeritas\protect\index{Sachverzeichnis}{celeritas} omnia manebunt similia, \edtext{minuto proportione}{\lemma{similia,}\Bfootnote{\textit{(1)} aucto propo \textit{(2)} minuto proportione \textit{L}}}  temporis intervallo, v.g. si una revolutio fiat  minuto 2\textsuperscript{do}, erit  assecutio 2. minutis secundis.  Ponatur revolutio  augeri, ut fiat dimidio minuto secundo, erit assecutio integro. Sed quomodo  efficiemus [62~v\textsuperscript{o}] ut eadem maneat assecutio aucta licet celeritate\protect\index{Sachverzeichnis}{celeritas}? Id non aliter fieri potest  quam si magis augeatur celeritas\protect\index{Sachverzeichnis}{celeritas} tardioris quam celerioris. Erit  hoc factu difficillimum, imo et computatu. De quo alias cogitandum.  Si celeritates\protect\index{Sachverzeichnis}{celeritas} augerentur proportione arithmetica res esset  effecta. Pone celerius conficere uno minuto 60 gran. tardius  uno minuto 30 gran. augeatur celeritas\protect\index{Sachverzeichnis}{celeritas} ita ut utrique addatur 1.  gran. erit assecutio semper aequivelox. Sed qua arte qua  machina hoc efficietur. Hic exerceant se Analytici, qui sibi videntur  ope Analyticae suae quidvis efficere posse solvantque \edtext{aut ad impossibile reducant}{\lemma{}\Bfootnote{aut [...] reducant \textit{erg.}\textit{ L}}}. Mihi hoc problema, machinam efficere in qua duo inaequalis velocitatis\protect\index{Sachverzeichnis}{velocitas} ita  sibi applicata sint, ut quantum uni accedit velocitatis\protect\index{Sachverzeichnis}{velocitas} adimiturque  progressione arithmetica, tantundem alteri quoque addatur  adimaturque, id est ut assecutiones sint semper aequiveloces.
\pend 
\pstart Caeterum unica et solida ratio rei assequendae haec videtur  esse, ut quanto magis intenditur celeritas\protect\index{Sachverzeichnis}{celeritas}, vel oneretur, vel  potentia contrahatur, vel onus elongetur. Pone gyros aliquos aucta celeritate\protect\index{Sachverzeichnis}{celeritas} revolvi, et ita pondus\protect\index{Sachverzeichnis}{pondus} magis elongari, pone contra eosdem tarditate contrahi, quod non videtur difficile machinatu. Forte  satius potentia celeritatem\protect\index{Sachverzeichnis}{celeritas} minui, quam pondus\protect\index{Sachverzeichnis}{pondus} augeri. Quia pondus\protect\index{Sachverzeichnis}{pondus} rem reducit ad principium gravitatis\protect\index{Sachverzeichnis}{gravitas}, quod gravitationem non  fert. Potius ergo aucta tarditate augeatur magnitudo circuli \edtext{quo}{\lemma{circuli}\Bfootnote{\textit{(1)}\ quem \textit{(2)}\ in \textit{(3)}\ quo \textit{L}}} agit elater\protect\index{Sachverzeichnis}{elater}, ita enim augebitur effectus celeritas\protect\index{Sachverzeichnis}{celeritas}.  Item potest fortasse construi horologium\protect\index{Sachverzeichnis}{horologium} mera vi Elastica\protect\index{Sachverzeichnis}{vis elastica}, quod componatur ex multis arculis, seu tensiunculis se restituentibus, ita ut possint  una hora forte fieri tales ultra 36000 displosiunculae. Quaelibet displosiuncula  sive celer sive tarda (nisi forte sunt aequidiuturnae) fiet ita parvo tempore, ut  differentia sit imperceptibilis, et vix possit errari nisi in Centesima millesima horae parte. Res ita geri potest, ut quilibet arculus se extendens liberet et alium  ubi semel ad summum pervenit, imo paulo ante totam extensionem fili, ne contractio aeris rem turbet. \edtext{Interea}{\lemma{turbet.}\Bfootnote{\textit{(1)}\ Idem \textit{(2)}\ Interea \textit{L}\ }} ab altero latere retendantur ob vim\protect\index{Sachverzeichnis}{vis elastica}  elasticam \edtext{aliam}{\lemma{elasticam}\Bfootnote{\textit{(1)}\ ibi majorem \textit{(2)}\ aliam \textit{L}\ }} applicatam. Forte et effici inde potest m.p.  possent omnes isti arculi esse in uno circulo, et retendantur in arcus\protect\index{Sachverzeichnis}{arcus}   ex lineis rectis, circulo compresso. \edtext{Numerentur}{\lemma{compresso.}\Bfootnote{\textit{(1)}\ Talis \textit{(2)}\ Numerentur \textit{L}\ }} displosiones in promoto quodam  instrumento accurate subdiviso, saltem pro uno minuto primo. Separatim alio circulo  notentur \edtext{minuta, alio horae.}{\lemma{notentur}\Bfootnote{\textit{(1)}\ horae, alio \textit{(2)}\ minuta, alio horae. \textit{L}}} [62~r\textsuperscript{o}]
\pend 
%\newpage
                   \pstart
Ecce tandem  rationem novam  et ingeniosam  ut justo tardius motum Elasticum\protect\index{Sachverzeichnis}{elasticum} levetur  vel acceleretur, justo celerius  tardetur. Nimirum: moveatur ab Elastico\protect\index{Sachverzeichnis}{elasticum} rota, haec  movet alias, usque ad rotam,  cui annexum aliquid quod  primam vi elastica\protect\index{Sachverzeichnis}{vis elastica} in arctum  \edtext{contrahere potest.}{\lemma{contrahere}\Bfootnote{\textit{(1)}\ , saltem radium ejus \textit{(2)}\ potest. \textit{L}}} Ergo si rota  prima movetur  justo tardius, ultima  proportionaliter movebitur (potest enim fieri quantacunque ultimae celeritas\protect\index{Sachverzeichnis}{celeritas}  propter magnitudinem, si alteri eccentricae concentrica radio longiore.)  tardius, ita ut remittat de rotae  primae retentaculo, ac proinde  eam relinquat se vi Elastica\protect\index{Sachverzeichnis}{vis elastica}  sua facere tanto longiorem  et ita motum extremi,  ac proinde caeterarum rotarum  celeriorem. Sin prima  movetur celerius, ultima fortius attrahet retentaculum,  et ita faciet prima in arctatiorem seu breviorem  ac proinde motum in  peripheria ejus tardiorem.  Ita omnia gerentur mero Elatere\protect\index{Sachverzeichnis}{elater}. Et duraturam  machinae exactitudinem  quamdiu restituendi se vis  in rota prima non languescet notabiliter. Loco vis\protect\index{Sachverzeichnis}{vis elastica}  Elasticae hic posset pondus\protect\index{Sachverzeichnis}{pondus}  movendum esse rotae compressiones\protect\index{Sachverzeichnis}{compressio}.  Nam alioquin vis Elastica\protect\index{Sachverzeichnis}{vis elastica} rotae  primae imminuta non reaget satis, nec manebit in aequilibrio\protect\index{Sachverzeichnis}{aequilibrium} cum vi rotae  ultimae, quanquam non opus sit aequilibrio\protect\index{Sachverzeichnis}{aequilibrium} nisi ut rota prima tardius incedente,  proportionaliter dilatet. Proportionem reperire difficile erit.\pend 
 


 


 


 


 

