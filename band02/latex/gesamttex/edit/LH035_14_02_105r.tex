\begin{ledgroupsized}[r]{114mm}%
\footnotesize%
\pstart \parindent -6mm%
\makebox[6mm][l]{\textit{L}}%
Aufzeichnung mit Auszügen aus unbekannter Vorlage: LH XXXV 14, 2 Bl. 105-107.
1~Bog. (Bl.~105-106) und 1 Bl. 2\textsuperscript{o}
(Bl.~105 weist zwei quere Ausschnitte auf;
Bl.~107 ist unregelmäßig beschnitten).
Etwa 3 S. auf Bl. 105~r\textsuperscript{o}, 106~v\textsuperscript{o} und 107~r\textsuperscript{o}.
Bl.~105~v\textsuperscript{o}, 106~r\textsuperscript{o} und 107~v\textsuperscript{o} sind leer.
Auf jedem Blatt je ein Wasserzeichen.
Bl.~105-107 sind ferner von dem aus Bl.~104 und 108 bestehenden Bogen eingeschlossen, welcher N. 59 % = LH 035,14,02_104,108 = Extraits de lettres de Mons. Boccone
überliefert.\\%
Cc 2, Nr. 1366 B%; Bl. 108 kein Eintrag KK 1, Cc 2
\pend%
\end{ledgroupsized}%
% \normalsize
\vspace*{5mm}%
\begin{ledgroup}%
\footnotesize%
\pstart%
\noindent%
\footnotesize{\textbf{Datierungsgr\"{u}nde:}
Die Wasserzeichen sind denen gleich, die in den Textträgern der Stücke N.~54 und N.~58 vorkommen.
Diese letzteren lassen sich auf die Monate Februar bis September 1676 datieren.
% Zwei Teile, Bl. 104 und 108 Excerpte aus Boccone 1674, nicht fortlaufend, Bl. 105-107 Notizen unbekannter Herkunft??. Die beiden Wasserzeichen in den Bl. 105, 106 und 107 k\"{o}nnen durch einen von Leibniz datierten Bogen (LH XXXVII 6 Bl. 14f., 18. III. 1676) auf M\"{a}rz 1676 datiert werden. Da auf diesen Bl\"{a}ttern die Lekt\"{u}re des Boccone noch als desideratum vermerkt wird??, sind die Exzerpte auf Bl. 104 und 108 danach enstanden. Mit dem Erscheinungsdatum 1674 ist ein terminus post quem gegeben. Erw\"{a}hnungen dieses Buches (III, 2; S. 391. 529) in Briefen an Leibniz geben nicht Aufschluss \"{u}ber Leibniz' Kenntnisstand. Auf Grund des fortlaufenden Textes und des gemeinsamen Texttr\"{a}gers kann angenommen werden, dass Bl. 104 und 108 zur gleichen Zeit beschrieben wurden. Das Wasserzeichen in Bl. 108 kann durch ?? auf ?? datiert werden. Daher k\"{o}nnen diese Ausz\"{u}ge durch inhaltliche \"{U}berlegungen auf ??, und durch den Texttr\"{a}ger auf ?? datiert werden.
}%
\pend%
\end{ledgroup}%
\vspace*{8mm}%
\count\Bfootins=1000
\count\Cfootins=1000
\count\Afootins=1200
\pstart%
\normalsize
\noindent%
[105~r\textsuperscript{o}]
Lettre \`{a} Mons. Oldenbourg\protect\index{Namensregister}{\textso{Oldenburg} (Grubendol), Heinrich 1618-1677} de Mons. Boccone\protect\index{Namensregister}{\textso{Boccone}, Paolo 1633-1704} \edtext{parle du corail\protect\index{Sachverzeichnis}{corail}}{\lemma{}\Bfootnote{parle du corail\protect\index{Sachverzeichnis}{corail} \textit{erg. L}}}. Chambre d'Anatomie \`{a} Delpht\protect\index{Ortsregister}{Delft} des chirurgiens de la ville. Il y a un rhinoceros embaum\'{e} entier. Mons. Cornelius Gravesande\protect\index{Namensregister}{\textso{`s-Gravesande}, Cornelius 1631-1691} doct. Medec. qui a succed\'{e} \`{a} la place de
\edtext{Regnero de Graef,\protect\index{Namensregister}{\textso{Graaf} (Graeff), Reinier de 1641-1673}}{\lemma{}\Bfootnote{Regnero\ \textbar\ de \textit{erg.}\ \textbar\ Graef,\ \textit{L}}}
\`{a} Mons. Sylvestre Buonfioli\protect\index{Namensregister}{\textso{Bonfioli}, Silvestro (??-??)} Anatomiste, Mathematicien medecin et s\c{c}avant philosophe \`{a} Boulogne.\protect\index{Ortsregister}{Bologna}
\pend%
\pstart%
Mons. Jannone, herboriste \`{a} Boulogne\protect\index{Ortsregister}{Bologna} habile dans ces plantes.
J'ay \edtext{veu il y a}{\lemma{veu}\Bfootnote{\textit{(1)}\ derniere m \textit{(2)}\ il y a \textit{L}}}
quelques mois \`{a} Bruxelles\protect\index{Ortsregister}{Br\"{u}ssel} chez Mons. Jean Herman,\protect\index{Namensregister}{\textso{Herman}, Jean}
Apothicaire et Herboriste fort habile, le catalogue des plantes nouuelles que Mons. Jannone
temoignoit dans ses lettres vouloir faire imprimer en l'an 1673. J'espere qu'il sera publi\'{e} \`{a} present.
\pend%
\pstart%
Personne nous a monstr\'{e} encor la methode de rechercher les proprietez des \mbox{plantes} par quelque experience particuliere.
On a dans le nort l'usage de quelques plantes, dont nous ignorons l'utilit\'{e} en Italie,\protect\index{Ortsregister}{Italien}
comme la Sophia, et la \edtext{jacea nigra.\protect\index{Sachverzeichnis}{jacea nigra}}{\lemma{jacea nigra}\Cfootnote{Wohl Centaurea jacea nigra oder Schwarze Flockenblume.}}
La voye des signatures peu seure.
%
Rolfinckius % \edtext{}{\lemma{Rolfinckius}\Cfootnote{\textsc{W.~Rolfinck}, \textit{De vegetabilibus}, Jena 1670.}}\cite{00319}\protect\index{Namensregister}{\textso{Rolfinck}, Werner (1599-1673)} \textit{De vegetabilibus} 1670 Jenae 4\textsuperscript{o}
refert quaedam \edtext{loca Avicennae\protect\index{Namensregister}{\textso{ibn-S\=\i n\={a}} (Avicenna), Ab\={u} Al\=\i\ al-\d{H}usain ibn Abdull\={a}h ca. 980-1037}}{\lemma{loca Avicennae}\Cfootnote{\cite{00319}\textsc{W.~Rolfinck}, \textit{De vegetabilibus}, Jena 1670, S.~197, % 198-212. 
verweist auf \cite{01167}\textsc{Avicenna}, \textit{Canon}, lib. II, tract. I, cap.~II, §~3.}} de modo investigandi haec particularia.
%
Adde \edtext{Porta}{\lemma{Porta}\Cfootnote{\cite{00320}\textsc{G. B. Della Porta}, \textit{Magia naturalis}, Leiden 1650, S. 446.}}\protect\index{Namensregister}{\textso{Della Porta}, Giovanni Battista 1535-1615} in \textit{Magia} pag. 446 edit. Lugd. Bat. 1650 titulo \textit{quomodo virtutes plantarum vestigandae}.
%
Item \protect\index{Namensregister}{\textso{Tacke} (Tachenius), Otto 1610-1680}\edtext{Tachen}{\lemma{Tachen}\Cfootnote{\cite{00479}\textsc{O. Tachenius}, \textit{Hippocrates chymicus}, Braunschweig 1668.% Vgl. A II, 1 (2006), 160f.
}} \textit{Hippoc.}
%
\edtext{Willis}{\lemma{Willis}\Cfootnote{\cite{00322}\textsc{T. Willis}, \textit{Pharmaceutice rationalis}, Oxford 1674.}}\protect\index{Namensregister}{\textso{Willis}, Thomas 1621-1675} \textit{Pharmaceutice rationalis}
%
et \edtext{Pechlin}{\lemma{Pechlin}\Cfootnote{\cite{00499}\textsc{J. N. Pechlin}, \textit{De purgantium facultatibus}, Leiden 1672.}}\protect\index{Namensregister}{\textso{Pechlin}, Johann Nicolas 1644-1706} dans une exercitation.
\pend%
\count\Bfootins=1200
\count\Cfootins=1200
\count\Afootins=1200
\newpage
\pstart%
\edtext{\edlabel{035,14,02_105r_01}\textit{Recherches}}{\lemma{\textit{Recherches}}\Cfootnote{\cite{00318}\textsc{P. Boccone}, \textit{Recherches}, Amsterdam 1674.}} \textit{et observations naturelles} imprim\'{e}es \`{a} Amsterdam\protect\index{Ortsregister}{Amsterdam} 1674 chez Jean Jansson\protect\index{Namensregister}{\textso{Janssonius}, Johannes 1588-1664} a Waesbergue.\protect\index{Ortsregister}{Waesberghe}\edlabel{035,14,02_105r_02}
\pend%
%
\pstart%
Mons. Van der Meer Medecin et Apothicaire de la ville de Delpht.\protect\index{Ortsregister}{Delft}
\pend%
%
\pstart%
Pseudo-corallium\protect\index{Sachverzeichnis}{corail} album fungosum d'Imperatus\protect\index{Namensregister}{\textso{Imperato}, Ferrante 1550-1631} est rare, je ne l'ay veu dans le naturel, qu'une fois dans la ville d'Agrigentum\protect\index{Ortsregister}{Agrigent} 1668 chez Mr Rocco Pinzellone\protect\index{Namensregister}{\textso{Pinzellone}, Rocco} Apothicaire.
\pend%
%
\pstart%
Jean Brayne\protect\index{Namensregister}{\textso{Brayne}, Jean} droguiste fort curieux \`{a} Amsterdam\protect\index{Ortsregister}{Amsterdam} dans la rue de Nest.
\pend%
%
\pstart%
Volckert Janse\protect\index{Namensregister}{\textso{Janse}, Volckert} marchant \`{a} Amsterdam\protect\index{Ortsregister}{Amsterdam} curieux des choses naturelles Mons. Cognart apothiquaire \`{a} Rouen\protect\index{Ortsregister}{Rouen} curieux.
\pend%
%
\pstart%
Mons. Montalbani\protect\index{Namensregister}{\textso{Montalbano}, Ovidio 1601-1671} professeur de Mathematique (+ J'ose, etc. +) avoit dessein de faire imprimer des restes d'Aldrovandi\protect\index{Namensregister}{\textso{Aldrovandi}, Ulisse 1522-1605}, il a donn\'{e} d\'{e}ja la \edtext{\textit{dendrologia}.}{\lemma{\textit{dendrologia}}\Cfootnote{\cite{00323}\textsc{U. Aldrovandi}, \textit{Dendrologia. Ovidius Montalbanus collegit}, Frankfurt 1671.}}
\pend%
%
\pstart%
Mr Joach. Jean Nuiz.\protect\index{Namensregister}{\textso{Nuiz}, Joachim Jean}
\pend%
%
\pstart%
Touchant le pumex\protect\index{Sachverzeichnis}{pumex} ou pierre ponce des orfeuures lettre de Mons. Boccone.\protect\index{Namensregister}{\textso{Boccone}, Paolo 1633-1704}
\pend%
%
\pstart%
Mons. des Jardins, docteur en Medecine \`{a} Bruxelles.\protect\index{Ortsregister}{Br\"{u}ssel}
\pend%
%
\pstart%
Mons. Lewenhoeck\protect\index{Namensregister}{\textso{Van Leeuwenhoek}, Antoni 1632-1723} \`{a} Delpht.\protect\index{Ortsregister}{Delft}
\pend%
%
\pstart%
Cornelius Mayer\protect\index{Namensregister}{\textso{Mayer}, Cornelius} Medecin a un oculus mundi ou opal qui plong\'{e} dans l'eau devient transparent. J'ay veu quelques morceaux de cristal de roche de la longueur et de l'epaisseur d'un pouce, \edtext{les quels}{\lemma{les}\Bfootnote{\textit{(1)}\ quelques \textit{(2)}\ quels \textit{L}}} avoient des cavitez dans le milieu qui renfermoient quelque portion d'air et d'eau claire c'estoit une piece rare \`{a} voir, \`{a} cause que toutes les fois qu'on la remuoit et qu'on tournoit entre deux doigts un de ces morceaux de cristal au travers du jour, on y remarquoit dans les cavitez une boule d'eau, la quelle \`{a} proportion du mouuement qu'on donnoit \`{a} ce corps solide, \edtext{[changeoit]}{\lemma{}\Bfootnote{changoit\textit{\ L \"{a}ndert Hrsg.}}} de place (+ se troubloit, \`{a} cause d'un peu de terre, qui y estoit encor +) et par apres s'\'{e}claircissoit.
\pend%
%
\pstart%
Estant dans l'isle d'Elbe\protect\index{Ortsregister}{Elba} j'ay remarqu\'{e} dans le bol rouge des petits morceaux de fer,
le plus souuent chaque morceau de fer avoit des cavitez de figure angulaire dans la surface qui repondent \`{a} celles qu'on voit dans ces corps metalliques,
\edtext{[appell\'{e}s]}{\lemma{appell\'{e}es}\Bfootnote{\textit{L ändert Hrsg.}}}
\edtext{par \protect\index{Namensregister}{\textso{Imperato}, Ferrante 1550-1631}
Imperatus}{\lemma{par Imperatus}\Cfootnote{\cite{00500}\textsc{F. Imperato}, \textit{Historia naturale}, Venedig 1672, lib. XVI, cap. 13, S.~401f.}}
Glebe di ferro et
\edtext{[suoi]}{\lemma{suo}\Bfootnote{\textit{L ändert Hrsg.}}}
ingemmamenti lib. 16, \textit{Hist. nat.}
%
Et comme cela est fort regulier dans cette espece de bol, nous pouuons dire que cette demonstration confirme l'opinion de Mons. Boyle\protect\index{Namensregister}{\textso{Boyle}, Robert 1627-1691} de gemmis, fuisse olim gemmas ex parte saltem liquidas et pellucidas. Adde effectum boli \edtext{illius}{\lemma{illius}\Bfootnote{\textit{erg. L}}} astringentem a ferro.
\pend%
\newpage
\pstart%
Eremitage de S. Cire\protect\index{Ortsregister}{Saint-Cyr-au-Mont-d'Or} proche de Lion\protect\index{Ortsregister}{Lyon}. Il y a une prodigieuse quantit\'{e} de Belemnites, et les paysans les appellent des quilles, \`{a} cause de leur figure sont bitumineux. Puisqu'on dit qu'ils sont bons pour la gravelle, on les pourroit preparer comme Ludus Paracelsi\protect\index{Sachverzeichnis}{Ludus Paracelsi} dont a parl\'{e} Helmont\protect\index{Namensregister}{\textso{Van Helmont}, Johan Baptista 1580-1644}.
Estant \`{a} Haure\protect\index{Ortsregister}{Le Havre} de grace et me promenant j'ay trouu\'{e} de petits cailloux, \edtext{agathes\protect\index{Sachverzeichnis}{agate} et}{\lemma{agathes}\Bfootnote{\textit{(1)}\ des \textit{(2)}\ et \textit{L}}} autres especes, ferm\'{e} blanche, comme bol, dedans; Mons. Tauron qui s'y trouue, homme d'esprit et curieux me dit qu'une personne fort ingenieuse a compos\'{e} un cabinet de ces morceaux d'Agathe\protect\index{Sachverzeichnis}{agate} qu'il avoit tri\'{e}s des cailloux qu'il avoit \edtext{[cass\'{e}s]}{\lemma{}\Bfootnote{cass\'{e} \textit{L \"{a}ndert Hrsg.}}}.
(J'ay trouu\'{e} aussi comme des cristaux dans ces pierres).
Cette m\^{e}me personne avoit une adresse merveilleuse \`{a} casser un
\edtext{bizet \`{a} plat d'un coup de [poigne] en}{\lemma{bizet \`{a}}\Bfootnote{\textit{(1)}\ coups de poign \textit{(2)}\ plat \textit{(3)}\ plat d'un coup de \textbar\ poign \textit{\"{a}ndert Hrsg.} \textbar\ en \textit{L}}}
suspendant le bizet avec deux doigts sur la surface de la terre. En montant la riviere pour venir \`{a} Rouen\protect\index{Ortsregister}{Rouen} je trouuay une partie de ces cailloux qui faisoient du bruit en les \edtext{secouant}{\lemma{les}\Bfootnote{\textit{(1)}\ remuant \textit{(2)}\ secouant \textit{L}}} ce que le vulgaire appelle Corail noir\protect\index{Sachverzeichnis}{corail}, n'est qu'un vray Antipates.
\pend%
%
\pstart%
Mons. Alex. Strayti\protect\index{Namensregister}{\textso{Strayti}, Alexandre} de Trapani.\protect\index{Ortsregister}{Trapani}
\pend%
\pstart%
Ceux qui ont la peine de comprendre Faba Egyptia \edtext{des anciens}{\lemma{des anciens}\Bfootnote{\textit{erg. L}}} après ce que dit \edtext{Clusius}{\lemma{Clusius}\Cfootnote{\cite{00489}\textsc{C. de L'Ecluse}, \textit{Exoticorum libri}, Leiden 1605, S.~32.}}\protect\index{Namensregister}{\textso{L'Ecluse} (Clusius), Charles de 1526-1609}
 lib. 2 cap. 13
et \edtext{Theophrastus}{\lemma{Theophrastus}\Cfootnote{\cite{00325}\textsc{Theophrast}, \textit{De historia plantarum}, Amsterdam 1644, S.~448.}}\protect\index{Namensregister}{\textso{Theophrast} 371-287 v.Chr.} cum notis Bodaei a Stapel\protect\index{Namensregister}{\textso{Bodaeus Van Stapel}, Johannes 1602-1636} pag. 448,
pourront voir le fruit de la m\^{e}me faba Egyptia dans son entier dans le cabinet de Mons. de S. Victor \`{a} Bruxelles\protect\index{Ortsregister}{Br\"{u}ssel}, confirm\'{e}e par le fidel rapport de Justus Heurnius\protect\index{Namensregister}{\textso{Heurnius}, Justus 1587-1652} sous le nom de
\edtext{Nymphaea glandifera Batavica}{\lemma{Nymphaea}\Bfootnote{\textit{(1)}\ Buccifera \textit{(2)}\ glandifera Batavica \textit{L}}}
Javorum.
\pend
\pstart
Cabinet de M. Justus Roeters\protect\index{Namensregister}{\textso{Roeters}, Justus} Conseiller de la ville d'Amsterdam.\protect\index{Ortsregister}{Amsterdam}
\pend%
\pstart%
Mons. Porree \edtext{[marchand]}{\lemma{}\Bfootnote{marchande \textit{ L \"{a}ndert Hrsg.}}} de Rouen \protect\index{Ortsregister}{Rouen} \edtext{de la}{\lemma{}\Bfootnote{de la \textit{erg. L}}} connaissance de Mons. Boccone.\protect\index{Namensregister}{\textso{Boccone}, Paolo 1633-1704}
[106~v\textsuperscript{o}]%
\pend%
\count\Bfootins=1500
\count\Cfootins=1500
\count\Afootins=1500
