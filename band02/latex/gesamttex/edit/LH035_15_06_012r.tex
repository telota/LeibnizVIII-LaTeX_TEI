[12~r\textsuperscript{o}] Pars 2\textsuperscript{da} Excerptorum ex Hookio\protect\index{Namensregister}{\textso{Hooke}, Robert (1635-1703)} contra Hevelium\protect\index{Namensregister}{\textso{Hevelius}, Johannes (1611-1687)} 
\pend 
\count\Afootins=1200
\count\Bfootins=1200
\count\Cfootins=1200
\pstart Hactenus Hookius\protect\index{Namensregister}{\textso{Hooke}, Robert (1635-1703)} examinavit Instrumenta\protect\index{Sachverzeichnis}{instrumentum Hevelii} Hevelii\protect\index{Namensregister}{\textso{Hevelius}, Johannes (1611-1687)}, \edtext{nunc de suis quoque methodis loquitur. Ait}{\lemma{nunc}\Bfootnote{\textit{(1)}\ ait \textit{(2)}\ de suis [...] Ait \textit{L}}} se invenisse et in exiguis modulis tentasse aliquot \textit{scores}\edtext{}{\lemma{\textit{scores}}\Cfootnote{a.a.O., S. 44.}} (vicenas) modorum perficiendi instrumenta pro sumendis angulis, distantiis, altitudinibus, tabellis etc. quorum omnium usus esse possit in terra, quorundam et in mari. Et ultra eas quas expertus sit rationes, posse se describere aliquot centenas, (2, a 300) quorum quilibet sit aeque accuratus ac Hevelii largissimus et quidam 30, 40, imo 60 vicibus accuratiores. Omnes tamen a se invicem differentes, in quibusdam partibus essentialibus. Assevero me ultra 20 methodos (\textit{contrived}\edtext{}{\lemma{\textit{contrived}}\Cfootnote{a.a.O., S. 44.}}) adhibuisse dividendi instrumenta, quarum quaelibet aeque ab alia qualibet distincta, quam via Heveliana a via diagonalium et tamen quaelibet earum capax ejusdem minimum certitudinis et exactitudinis et quaedam, centies majoris. Habeo plus, quam duodenam modorum adaptandi instrumenta ad perpendicularitatem et horizontalitatem; omnia aeque exacta ac perpendiculum commune, et quaedam multo magis, et ad exactitudinem progredientia \textso{quantamvis}, valde differentia a se invicem. Habeo totidem differentia dioptrarum\protect\index{Sachverzeichnis}{dioptra} genera pro augendis, dirigendis, accommodandis et certificandis dioptris, quorum quaedam applicandae ad usus particulares, quaedam ad omnes, quarum ope etiam haec pars perfici potest ad certitudinem quantamvis. Varias habeo rationes fixandi instrumenta, et accommodandi pro hoc aut alio aut variis usibus. Varias habeo vias mechanicas pro ipsis illis machinis elaborandis magna facilitate, et certitudine: cognitio non minus utilis quam theoria et usus jam elaboratorum. Cum sint adeo pauci in mundo, qui possint hoc aut velint. Habeo viam mechanicam calculandi et efficiendi operationes Arithmeticas longe promtiorem et certiorem, quam fieri possit per logarithmos. Quod implet totum negotium dimetiendi angulos. 
\pend 
\pstart Describit ergo instrumentum pro angulis et distantiis coelestibus sumendis, quod magnitudine aucta tantae capax est exactitudinis, quantam aer et atmosphaera permittunt. 
\pend 
\newpage
\pstart Dioptra\protect\index{Sachverzeichnis}{dioptra} Hookii Telescopia\protect\index{Sachverzeichnis}{telescopium} ex duabus lentibus convexis\protect\index{Sachverzeichnis}{lens convexa}, in tubo vel pyxide quadrata, ponatur oculus ibi, \edtext{ubi totum vitrum oculare\protect\index{Sachverzeichnis}{ocular} ab objecto}{\lemma{}\Bfootnote{ubi \textbar\ ubi \textit{streicht Hrsg.} \textbar\ totum \textit{(1)}\ objectum a \textit{(2)}\ vitrum oculare\protect\index{Sachverzeichnis}{ocular} ab objecto \textit{L}}} repletum videt. In foco ponantur duo fila se decussantia, idque ita dignosces, si moto oculo moveri super objecto apparent fila non sint in foco, si fixa manent[,] sunt, ibi fila etsi subtilissima ex tela serica, apparent ut fila crassa tracti. 
\pend 
\pstart \textso{Modus dividendi}. Modo Tychonis\protect\index{Namensregister}{\textso{Brahe}, Tycho (1546-1601)} et Hevelii\protect\index{Namensregister}{\textso{Hevelius}, Johannes (1611-1687)} opus est 150 pedum radio pro secundis. Hic sufficit radius 3 pedum. Imo in tali instrumento non bene distinguentur divisiones nudo visu, hoc tam facile nudo visu quam facile cuilibet videre decimam pollicis partem. In instrumento 150 pedum minutum vix est semipollex, et secundum \rule[-4mm]{0mm}{10mm}$\displaystyle\frac{1}{120}$ pollicis quanquam autem Hevelius multa se praestare putet via Nonii\protect\index{Namensregister}{\textso{Nu\~{n}ez}, Pedro (1502-1578)}, Vernerii\protect\index{Namensregister}{\textso{Vernier}, Pierre (1580-1637)} vel Hedraei\protect\index{Namensregister}{\textso{Hedraeus}, Bengt (1608-1659)}, sed credo his quae dixit consideratis aliter censurum. Cum radius 10 pedum sit tantum hujus (150 pedum) pars \rule[-4mm]{0mm}{10mm}$\displaystyle\frac{1}{15}$ unde secundum minutum $\displaystyle\frac{1}{15,\smallfrown 120}$. Utitur Hookius\protect\index{Namensregister}{\textso{Hooke}, Robert (1635-1703)} \edtext{cochleis quas}{\lemma{cochleis}\Bfootnote{\textit{(1)}\ quarum \textit{(2)}\ quas \textit{L}}} circumagit ope indicis prorsus ut mihi relatum est fecisse jam \edtext{Hedraeum\protect\index{Namensregister}{\textso{Hedraeus}, Bengt (1608-1659)}. Habet}{\lemma{Hedraeum.}\Bfootnote{\textit{(1)}\ Ut legi facilius possint cha \textit{(2)}\ Hinc etiam facile na \textit{(3)}\  Habet \textit{L}}} ea methodus illud quoque compendium, \edtext{ut divisorii numeri non divisionibus}{\lemma{ut}\Bfootnote{\textit{(1)}\ divisoriae lineae non ipsi \textit{(2)}\ divisorii numeri non divisionibus \textit{L}}} illis subtilissimis ascribantur sed in indice revolutionis legantur; qui rursus quantalibet subtilitate subdivisus intelligi potest. Utque in indice facilius legantur potest extremum indicis secum ducere lentem, quae augeat characteres. 
\pend 
\pstart Aliud commodum proponit Hookius\protect\index{Namensregister}{\textso{Hooke}, Robert (1635-1703)}, quod est observatorem suum idque uno oculi \edtext{ictu dirigere instrumentum}{\lemma{ictu}\Bfootnote{\textit{(1)}\ praestare quantam \textit{(2)}\ dirigere instrumentum \textit{L}}} ad duo simul objecti loca vel quocunque angulo a se invicem remota sint, aut etiamsi sint in linea recta opposita; hoc est unum ex primariis auxiliis observationum, cum alia instrumenta duos requirant observatores pro sumendis coelestibus distantiis, et Tycho\protect\index{Namensregister}{\textso{Brahe}, Tycho (1546-1601)} illis usus sit quatuor, ubi opus erat concursu tam accurato, ut vel unius error omnia vitiaret. Et instrumenta pro uno observatore; qua duplici opus observatione, ob motum et mille mutationum incommoda ab omnibus relicta sunt. Hic vero nihil aliud faciendum est, quam ut ope cochleae moveatur instrumenti bracchium, donec sentiat ambo objecta se simul tangere, et in his punctis eorum distantiam metietur. Hoc quivis facile discet intellecta methodo adaptandi duo Telescopia\protect\index{Sachverzeichnis}{telescopium}, \textit{that by looking} \makebox[1.0\textwidth][s]{\textit{in at one small hole in the side of one of them, he will be able to see both those objects}}
\pend
\newpage
\pstart \noindent \textit{distinctly, to wich they are directed, how much soever separated.}\edtext{}{\lemma{\textit{separated}.}\Cfootnote{a.a.O., S. 56.}} Junge duo telescopia\protect\index{Sachverzeichnis}{telescopium} ope juncturae cavae, seu pertusae, cujus cavitas \rule[-4mm]{0mm}{10mm}$\displaystyle\frac{3}{4}$ cavitatis tuborum ipsorum Telescopiorum\protect\index{Sachverzeichnis}{telescopium}, et directe contrarie hujus cavitatis \edtext{in junctura}{\lemma{}\Bfootnote{in \textbar\ ipsa \textit{gestr.}\ \textbar\ junctura \textit{L}}} fiat junctura exigua circiter magnitudine nigerrimae partis pupillae oculi, ita ut oculus respiciens in hanc cavitatem possit videre perpendiculariter in tubum inferiorem, inde oblique pone duo \edtext{[frusta]}{\lemma{}\Bfootnote{frustra \textit{L \"{a}ndert Hrsg.}}} metalli reflectentis bene posita, ita ut reflectant axem utriusque tubi ad angulos rectos. Quod fit fixando planum laminarum sive Tabularum (\textit{plates}\edtext{}{\lemma{\textit{plates}}\Cfootnote{a.a.O., S. 56.}}) inclinatum ad illam axem angulo 45 graduum. Effice ut superior lamina reflexiva (\textit{reflex plate}\edtext{}{\lemma{\textit{reflex plate}}\Cfootnote{a.a.O., S. 56.}}) perveniat a superiore tubi latere, eo usque ut tangat \edtext{axem vel medium tubi}{\lemma{axem}\Bfootnote{\textit{(1)}\ Tubi medii \textit{(2)}\ vel medium tubi \textit{L}}} et effice ut inferior se extendat per totum Tubum, a summo ad fundum, et a latere uno ad aliud. Apparebit debite locata esse, \edtext{si introspiciendo}{\lemma{si}\Bfootnote{\textit{(1)} inspiciendo \textit{(2)} introspiciendo \textit{L}}} per foramen exiguum contra centrum juncturae duo foramina rotunda tubi apparebunt oculo coalescere in unum, ita ut oculus simul directe pervideat longitudines utriusque. Quo facto his tubis apta duo Telescopia\protect\index{Sachverzeichnis}{telescopium} cum convexis ocularibus\protect\index{Sachverzeichnis}{ocular}