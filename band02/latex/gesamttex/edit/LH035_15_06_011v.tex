%[11~v\textsuperscript{o}] 
\hspace{-1.8mm}ut definiat limbum solis, et ita facile esse discernere, quando is circulus perfecte repletus est figura solis per superiorem dioptram\protect\index{Sachverzeichnis}{dioptra} admissa. Respondeo hoc videri probabile et facile, et ita credi atque asseri ab omnibus Opticae scriptoribus. Sed rem plane aliter se habere. Nam praeterquam quod apud omnes in confesso est, penumbram\protect\index{Sachverzeichnis}{penumbra} hujus circuli minimum tam esse crassam quam diametrum superioris foraminis per quod trajectus est, quod non potest esse minus minuto: praeter hoc inquam, rem aliter demonstrat experientia, et quod limbus imaginis pictae super inferiore dioptra\protect\index{Sachverzeichnis}{dioptra}, terminatus est penumbra\protect\index{Sachverzeichnis}{penumbra}, quae est aliquando quinquies vel sexies crassior diametro foraminis, et quod est mirabilius, quo minus est foramen eo crassior est penumbra\protect\index{Sachverzeichnis}{penumbra}, \textit{and the bigger (to a certain degre) the less}\edtext{}{\lemma{\textit{less}}\Cfootnote{a.a.O., S. 35.}}. Sed nulla est crassities quae eam plane auferat, et diameter solis\protect\index{Sachverzeichnis}{diameter solis} ea ratione sumtus est aliquando crassior, aliquando tenuior, quam opus est, idque satis notabiliter. Sed de hac aliisque lucis miris proprietatibus alias \edlabel{hookius1}dicam. 
\pend 
\pstart \edtext{Hevelius\edlabel{hookius2}}{{\xxref{hookius1}{hookius2}}\lemma{dicam.}\Bfootnote{\textit{(1)}\ Hookius ope duarum cochlearum manualium \textit{(2)}\ Hevelius \textit{(3)}\ Hevelius \textit{L}}} usus refert quos admirabiles ait, cochlearum imo in locando et fixando quadrante\protect\index{Sachverzeichnis}{quadrans}, 2\textsuperscript{o} in danda motione qua sequamur instrumento solem et fixas in diurno eorum motu, 3\textsuperscript{o} in subdivisionibus graduum usque ad 2\textsuperscript{da} minuta. Sed putat Hookius\protect\index{Namensregister}{\textso{Hooke}, Robert (1635-1703)} pro sequendo stellarum motu, incommodum esse usum cochlearum et rectius adhiberi Automaton\protect\index{Sachverzeichnis}{automaton exiguum} \edtext{exiguum, cujus ope instrumentum semel positum ad stellae Azimuth\protect\index{Sachverzeichnis}{azimuth} exacte}{\lemma{exiguum,}\Bfootnote{\textit{(1)}\ quod satis exacte \textit{(2)}\ cujus [...] exacte \textit{L}}} pro aliquot horis sequatur stellae motum. Quod attinet subdivisiones ope cochlearum, addit \edtext{Hookius\protect\index{Namensregister}{\textso{Hooke}, Robert (1635-1703)} id}{\lemma{Hookius}\Bfootnote{\textit{(1)}\ non \textit{(2)}\ id \textit{L}}} verum esse, sed non bene factum ab Hevelio\protect\index{Namensregister}{\textso{Hevelius}, Johannes (1611-1687)} quia non est certus an ab initio suam cochleam ad certum aliquem gradum fixerit concludit Hookius\protect\index{Namensregister}{\textso{Hooke}, Robert (1635-1703)} de primario illo Hevelii\protect\index{Namensregister}{\textso{Hevelius}, Johannes (1611-1687)} quadrante\protect\index{Sachverzeichnis}{quadrans} (quem ille \edtext{pag. 184}{\lemma{pag. 184.}\Cfootnote{\textsc{J. Hevelius}, \textit{Machina Coelestis}, Danzig 1673\cite{00329}, S. 184, Vergleich der Quadranten\protect\index{Sachverzeichnis}{Quadrant} des Hevelius und des Brahe.}} commendat) videri \textit{the frame}\edtext{}{\lemma{\textit{the frame}}\Cfootnote{\textsc{R. Hooke}, \textit{Animadversions}, S. 37.}} structuram instrumenti ejus valde bonam, et ope quarundam additionum, ut quoad dioptras\protect\index{Sachverzeichnis}{dioptra}, divisiones, perpendicula et erectiones, debere tam fieri bonam, quam opus est pro ullo usu Astronomico; et quadragies meliore quam nunc factum et descriptum est ab \edtext{Hevelio\protect\index{Namensregister}{\textso{Hevelius}, Johannes (1611-1687)}. Nam}{\lemma{Hevelio.}\Bfootnote{\textit{(1)}\ Sed \textit{(2)}\ Nam \textit{L}}} ita ut est, et ab Hevelio\protect\index{Namensregister}{\textso{Hevelius}, Johannes (1611-1687)} datur, non est melius largo instrumento quod Tycho\protect\index{Namensregister}{\textso{Brahe}, Tycho (1546-1601)} adhibuerat 100 abhinc annis, et inferius ejus quadrante\protect\index{Sachverzeichnis}{quadrans} murali pro sumendis altitudinibus meridianis\protect\index{Sachverzeichnis}{meridian}. 
\pend 
\pstart Ingenue fatetur \edtext{Hevelius}{\lemma{fatetur}\Bfootnote{\textit{(1)}\ Tycho \textit{(2)}\ Hevelius \textit{L}}} difficultatem in sumendis distantiis fixarum a luna; quod a nulla alia re provenit, quam imperfectione visus communis et omnes difficultates evanescent, si dioptrae\protect\index{Sachverzeichnis}{dioptra} fiant aliter (Telescopice\protect\index{Sachverzeichnis}{telescopium}). Et videtur longe majorem adhuc facere difficultatem, sumere distantiam solis a \venus\ visa diurno tempore, sed ego mox dicam methodum per quam non tantum facile sit sumere eam a \venus, sed et a \mars\ et \jupiter, \textit{nay, from several of the fix Stars.}\edtext{}{\lemma{\textit{Stars}.}\Cfootnote{a.a.O., S. 38.}} 
\pend 
\count\Afootins=1500
\count\Bfootins=1500
\count\Cfootins=1500
\pstart Non dubito Hevelium\protect\index{Namensregister}{\textso{Hevelius}, Johannes (1611-1687)} exactissimum fuisse, quantum nudo visu licet. Sed optarem videre distantias quas dicit sumsisse octo fixarum prope Eclipticam\protect\index{Sachverzeichnis}{ecliptica}, nempe: \textit{Lucidae Arietis, et Palilicii, Palilicii et Pollucis, Pollucis et Reguli, Reguli et Spicae, Spicae et in manu Serpentarii, in manu Serpentarii et Aquilae, Aquilae et Marchab, Marchab et Lucidae Arietis,}\edtext{}{\lemma{\textit{Arietis},}\Cfootnote{a.a.O., S. 38.}} idque tanta exactitudine, ut non defuerit ei secundum minutum in tota circumferentia circuli coelestis, octo observationibus sumta. Quod mihi videtur una ex maximis asseverationibus quas unquam viderim; et ausim geometrice demonstrare, quod fuerit ipsi certe, et instrumentis quibus usus est, impossibile fuisse facere vel unicam observationem 30 secundorum certitudine, unde fit ut in toto circulo vel 240 secundis certus fuerit, vel 4 minutis. Superest ut addam quaedam de \textso{apparatu meo}\protect\index{Sachverzeichnis}{apparatus Hookii}, ut patientiam lectoris compensem ac subjiciam, inquit, antea Hevelii\protect\index{Namensregister}{\textso{Hevelius}, Johannes (1611-1687)} \edtext{Epistolam}{\lemma{Epistolam}\Cfootnote{a.a.O., S. 39-41.}} ad Oldenburgium\protect\index{Namensregister}{\textso{Oldenburg}, Heinrich (ca. 1619-1677)}. Ait Hevelius\protect\index{Namensregister}{\textso{Hevelius}, Johannes (1611-1687)} in illa: \textit{Hocce penitus mihi imaginor, si totum istud negotium dioptris Telescopicis\protect\index{Sachverzeichnis}{telescopium} suscepissem, quod non solum plurimos annos examinibus trivissem, sed spe sine dubio varia via, de qua non est hic disserendi locus, cecidissem. Exinde gratulor mihi me ad eam sententiam nondum transiisse, ac mea methodo universa perfecisse}\edtext{}{\lemma{\textit{perfecisse}}\Cfootnote{a.a.O., S. 40.}}. Ubi viderint meas observationes judicent. \edtext{Integrum cuilibet}{\lemma{}\Bfootnote{Integrum \textbar\ postea \textit{ gestr.}\ \textbar\ cuilibet \textit{ L}}} erit, vel alium plane sua methodo catalogum construere adhibitis \textit{tot centenis novis fixis hactenus neglectis}\edtext{}{\lemma{\textit{neglectis}}\Cfootnote{a.a.O., S. 41.}}. Sed non video an cura haec quenquam adhuc serio tangat, facile est \textit{unam alteramve stellam ope Telescopii\protect\index{Sachverzeichnis}{telescopium} vel Telescopicarum\protect\index{Sachverzeichnis}{telescopium} dioptrarum\protect\index{Sachverzeichnis}{dioptra}, dum praecipuas ac majores fixas earumque intercapedines sumimus correctas ad debitum locum deducere, tum nonnunquam distantias nonnullarum stellarum capere, haec ludicra sunt. Sed omnes conjunctim secundum longum ac latum restituere, tum ductu continuo singulis serenis diebus ac noctibus tam altitudinum solarium quam reliquarum stellarum operam dare easque orbi exponere, ut pateat instrumentorum harmonia ac motuum certitudo, hoc artis, hoc laboris est. Quando observationes 20 vel 30 annorum spatio ab utraque parte}\edtext{}{\lemma{\textit{parte}}\Cfootnote{a.a.O., S. 41.}} habebimus, res clarior erit. \textit{Interea quilibet fruatur suo ingenio}\edtext{}{\lemma{\textit{ingenio}}\Cfootnote{a.a.O., S. 41.}} etc. Respondet Hookius\protect\index{Namensregister}{\textso{Hooke}, Robert (1635-1703)} errare Hevelium\protect\index{Namensregister}{\textso{Hevelius}, Johannes (1611-1687)} in eo, quod putat \edtext{ad}{\lemma{}\Bfootnote{ad \textit{erg. L}}} quamvis telescopicam\protect\index{Sachverzeichnis}{telescopium} observationem opus esse novo examine unde videri Hevelium\protect\index{Namensregister}{\textso{Hevelius}, Johannes (1611-1687)} non habere veram notionem de observationibus ejusmodi. \edtext{Usus sum}{\lemma{ejusmodi.}\Bfootnote{\textit{(1)}\ Item qu \textit{(2) }\ Usus sum \textit{L}}} quadrante\protect\index{Sachverzeichnis}{quadrans} sex pedum, instructo duabus dioptricis telescopicis\protect\index{Sachverzeichnis}{telescopium}, pro examinandis motibus cometis\protect\index{Sachverzeichnis}{cometa} anni 1665. Et cum eandem rem melius fecerim quadrante\protect\index{Sachverzeichnis}{quadrans} radii 6 pollicum, quam ille possit quadrante\protect\index{Sachverzeichnis}{quadrans} sex pedum, puto concludere minimum debuisse, eandem rem decies melius fieri posse in radio sex pedum. Instrumenta telescopicarum\protect\index{Sachverzeichnis}{telescopium} dioptricarum ope ad minuta tertia proferri possunt. Plus faciam unius \edtext{pedis radio}{\lemma{unius}\Bfootnote{\textit{(1)}\ instrumenti radii \textit{(2)}\ pedis radio \textit{L}}} sextante\protect\index{Sachverzeichnis}{sextans} aut quadrante\protect\index{Sachverzeichnis}{quadrans}, quam ille ope 60 pedum simplicibus dioptris. 
\pend 
\count\Afootins=1500
\count\Bfootins=1500
\count\Cfootins=1500
\pstart