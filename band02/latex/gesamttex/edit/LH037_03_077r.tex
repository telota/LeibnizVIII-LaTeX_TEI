\vspace{-1mm}
\begin{Ueberlieferung}% 
{\textit{L}}Konzept: LH XXXVII 3 Bl. 77-78. 1 Bog. 2\textsuperscript{o}. 1 S. auf Bl. 77~r\textsuperscript{o}. Bl. 77~v\textsuperscript{o} leer. Bl. 78 überliefert N.~45\textsubscript{2}.
% DE VESTIBUS CONJUGATIS
Je ein verschiedenes Wasserzeichen auf Bl. 77 und 78.\\%
Cc 2, Nr. 1213 A
\end{Ueberlieferung}
%
%\vspace*{4mm}
% \begin{Datierungsgruende}%
% ?? Rest eines Wasserzeichens
% \end{Datierungsgruende}
%
\vspace{5.5mm}
\count\Afootins=1000
\count\Bfootins=1000
\count\Cfootins=1000
%\pstart
%\noindent [77~r\textsuperscript{o}] 
%\pend
\pstart
\begin{center}De vectibus\protect\index{Sachverzeichnis}{vectis} conjugatis\end{center}
\pend
\vspace{0.3em}
\pstart
\noindent [77~r\textsuperscript{o}] 
\textso{Vectes\protect\index{Sachverzeichnis}{vectis} conjugati} sunt quorum diversa sunt centra, 
et uno circa suum centrum moto contingit, ut alter  quoque circa suum centrum moveatur;  ita in fig. 1. sunt vectes\protect\index{Sachverzeichnis}{vectis} conjugati \textit{AC}, et \textit{BC} quorum centra \textit{A} contactus in \textit{C} ita ut vectis\protect\index{Sachverzeichnis}{vectis} \textit{BC} non possit elevari quin \textit{AC} elevetur, nec \textit{AC} deprimi quin \textit{BC} 
\edtext{deprimatur. Potentiae\protect\index{Sachverzeichnis}{potentia}}{\lemma{deprimatur.}\Bfootnote{\textit{(1)}\ Intelligatur autem pondus\protect\index{Sachverzeichnis}{pondus} \textit{a} \textit{(2)}\  Potentiae\protect\index{Sachverzeichnis}{potentia} \textit{L}}} 
variis modis applicari possunt, sed experimenti faciendi causa  optimum est, potentiam\protect\index{Sachverzeichnis}{potentia} ad vectem\protect\index{Sachverzeichnis}{vectis}
\edtext{vel libram}{\lemma{}\Bfootnote{vel libram \textit{erg.} \textit{L}}}
\textit{B} applicatam esse pondus\protect\index{Sachverzeichnis}{pondus}
\edtext{applicatum ejus brachio\protect\index{Sachverzeichnis}{brachium} inferiori}{\lemma{applicatum}\Bfootnote{ \textit{(1)}\ opposito \textit{(2)} ejus brachio inferiori \textit{L}}} in
$\theta$
et potentiam\protect\index{Sachverzeichnis}{potentia} applicatam  
\edtext{ad vectem\protect\index{Sachverzeichnis}{vectis}}{\lemma{ad}\Bfootnote{\textit{(1)}\ libram \textit{(2)}\ vectem\protect\index{Sachverzeichnis}{vectis} \textit{L}}} 
\textit{A}, applicari superiori ejus brachio\protect\index{Sachverzeichnis}{brachium} in
$\eta$.
Ita sibi obnitentur, nam si velis potentiam\protect\index{Sachverzeichnis}{potentia} locare in ($\eta$) vel ($\theta$) opus erit levitate, et machina\protect\index{Sachverzeichnis}{machina} in aqua  locanda est,  
\edtext{nisi animalia}{\lemma{nisi}\Bfootnote{\textit{(1)}\ homines \textit{(2)}\ animalia \textit{L}}} 
aut Elateria\protect\index{Sachverzeichnis}{elaterium}, aut trochleas applicare velis.
\pend
\pstart
Jam ex datis rectis $A\eta$, $B\theta$, \textit{AC}, \textit{BC} angulisque \textit{CAB}, \textit{CBA}, ac  ponderibus\protect\index{Sachverzeichnis}{pondus} $\eta, \theta$ investigare virium rationem, quam alterum pondus\protect\index{Sachverzeichnis}{pondus} alterius ratione habet, ac  proinde ex datis rectis, et potentiis\protect\index{Sachverzeichnis}{potentia}, investigare angulos, ex datis lineis et angulis potentias\protect\index{Sachverzeichnis}{potentia}; ex caeteris datis  investigare distantias potentiarum\protect\index{Sachverzeichnis}{potentia} a suis centris; aut ex caeteris datis investigare distantias puncti contactus a centris, aut potentiis\protect\index{Sachverzeichnis}{potentia}: ut  potentia\protect\index{Sachverzeichnis}{potentia} altera in alteram virium rationem  habeat datam, vel etiam ut fiat aequilibrium\protect\index{Sachverzeichnis}{aequilibrium} aliaque  
\edtext{infinita problemata\protect\index{Sachverzeichnis}{problema}}{\lemma{infinita}\Bfootnote{\textit{(1)}\ theorema\protect\index{Sachverzeichnis}{theorema} \textit{(2)}\ problemata\protect\index{Sachverzeichnis}{problema} \textit{L}}} solvere quae ex  horum combinatione nasci possint; res non adeo facilis ac parata
\edtext{est; et fateor ac theorema\protect\index{Sachverzeichnis}{theorema} generale, cujus unius ope possint omnes casus solvi, non parum}{\lemma{est;}\Bfootnote{\textit{(1)} ac theorema\protect\index{Sachverzeichnis}{theorema} generale, \textit{(a)} ex quo \textit{(b)} quod \textit{(c)} cujus unius ope \textit{(aa)} solvi \textit{(bb)} possint omnes casus solvi, \textit{(2)} \textbar\ et fateor \textit{erg.} \textbar\ ac theorema [...] solvi, \textit{(a)} non sine \textit{(b)} non parum\ \textit{L}}}
negotii mihi facessisse.
Juveni\protect\index{Sachverzeichnis}{juvenis} tamen, atque ita omnino id mihi enuntiare posse 
\edtext{videor ut mox}{\lemma{videor}\Bfootnote{\textit{(1)}\ praecedente \textit{(2)}\ hac figurae praeparatione, ut \textit{(3)}\  . In figura \textit{(4)}\ ut mox \textit{L}}} 
praecedente figurae praeparatione sequetur.
In figura \edtext{quam vides}{\lemma{}\Bfootnote{quam vides \textit{erg.} \textit{L}}} \edtext{recta}{\lemma{}\Bfootnote{recta \textit{ erg.} \textit{ L}}} \textit{CL}
perpendicularis ad 
\edtext{\textit{BC} perducatur}{\lemma{\textit{BC}}\Bfootnote{ \textit{(1)}\ vectem\protect\index{Sachverzeichnis}{vectis} \textit{(a)} moventem \textit{(b)} inferiorem \textit{(2)}\ perducatur \textit{L}}} 
dum ipsi \textit{AB} occurrat in \textit{L}.  
Ipsas $\eta\xi$ et $\theta\lambda$ appellabimus \textso{altitudines}.
\pend
\count\Afootins=1500
\count\Bfootins=1500
\count\Cfootins=1500


