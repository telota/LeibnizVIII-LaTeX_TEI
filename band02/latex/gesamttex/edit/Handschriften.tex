\thispagestyle{empty}
\vspace{3.0ex}
\begin{center}\uppercase{\normalsize Fundstellen}\end{center}
\footnotesize
Folgendes Register verzeichnet sämtliche im vorliegenden Band edierten Hand- und Druckschriften, 
alpha\-betisch geordnet nach \mbox{Fund\-ort} und Signatur.\\[4.0ex]%
%%
\textsc{Druckschrift}\\
\vspace{-3mm}
\setlength{\columnseprule}{0.4pt}
\renewcommand*{\chapter}{\OrigChapter}
\setlength\LTleft{\fill} \setlength\LTright{\fill}
\begin{longtable}{ll}
\footnotesize
\textit{Journal général de l'Instruction publique et des cultes} XXVI (1857), Nr. 32, S. 235f. & N. 81\\%% = ex N. 93 = RK ?????.
\end{longtable}
\vspace{3.0ex}
%%
\noindent
\textsc{Göttingen}, \textit{Niedersächsische Staats- und Universitätsbibliothek}\\ 
\vspace{-3mm}
\setlength{\columnseprule}{0.4pt}
\renewcommand*{\chapter}{\OrigChapter}
\setlength\LTleft{\fill} \setlength\LTright{\fill}
\begin{longtable}{ll}
\footnotesize
8 MED PRACT 96/37 & N. 67\\%% = ex N. 84 = RK 62029.
\end{longtable}
\vspace{3.0ex}
%%
\noindent
\textsc{Göttingen}, \textit{Stadtarchiv}\\ 
\vspace{-3mm}
\setlength{\columnseprule}{0.4pt}
\renewcommand*{\chapter}{\OrigChapter}
\setlength\LTleft{\fill} \setlength\LTright{\fill}
\begin{longtable}{ll}
\footnotesize
MSL Nr. 12 Bl. 19 & N. 66\\%% = ex N. 74 = RK 60389.
\end{longtable}
\vspace{3.0ex}
%%
\noindent
\textsc{Hannover}, \textit{Gottfried Wilhelm Leibniz Bibliothek \textendash\ Niedersächsische Landesbibliothek}\\
\vspace{-3mm}
\setlength{\columnseprule}{0.4pt}
\renewcommand*{\chapter}{\OrigChapter}
\setlength\LTleft{\fill} \setlength\LTright{\fill}
\begin{longtable}{llll}
\footnotesize
LBr & 719a          & Bl. 3\textendash 4 & N. 57\\%% = ex N. 64 = Cc2 00 = RK 41946.
LH III & 1, 3 & Bl. 1\textendash 8 & N. 70\\%% = ex N. 75 = KK1 975.
LH III & 1, 3 & Bl. 9 & N. 69\\%% = ex N. 76 = KK1 976.
% LH III & 3, 3a & Bl. 11\textendash 12 & N. ***\\%%    N. *77 = Cc2 1562.
% LH III & 3, 3a & Bl. 19\textendash 22 & N. ***\\%%    N. *78 = KK1 978.
LH III & 4, 3a & Bl. 1 & N. 76\\%% = ex  N. 79 = Cc2 1323 A (tlw.).
LH III & 4, 3a & Bl. 1 & N. 82\\%% = ex N. 96 = Cc2 1323 A (tlw.).
LH III & 4, 8a & Bl. 1 & N. 74\\%% = ex N. 80 = Cc2 1563.
LH III & 5 & Bl. 49 & N. 76\\%% = ex N. 79 = Cc2 1323 B.
LH III & 5 & Bl. 55 & N. 71\\%% = ex N. 81 = KK1 977.
LH III & 5 & Bl. 56 & N. 75\\%% = ex N. 82 = Cc2 1271.
LH III & 5 & Bl. 67\textendash 68 & N. 68\\%% = ex N. 83 = KK1 979.
LH III & 5 & Bl. 86\textendash 87 & N. 73\\%% = ex N. 85 = Cc2 430.
% LH III & 5 & Bl. 88 & N. ***\\%%    N. *86 = Cc2 340.
LH III & 5 & Bl. 89 & N. 72\\%% = ex N. 87 = Cc2 869.
LH IV & 1, 4b & Bl. 3\textendash 12 & N. 58\\%% = ex N. 68 = Cc2 1322 A-E (D tlw.).
LH IV & 1, 4b & Bl. 9\textendash 10 & N. 6\\%% = ex N. 62 = Cc2 1322 D (tlw.) = RK 60606.
LH IV & 1, 4b & Bl. 13\textendash 14 & N. 54\\%% = ex N. 68 = Cc2 1324.
LH XXXV & 2, 1 & Bl. 273 & N. 95\\%% = ex N. 06 = Cc2 897 (tlw.).
LH XXXV & 5, 23 & Bl. 14 & N. 96\\%% = ex N. 95 = Cc2 00 = RK 58394.??
LH XXXV & 8, 30 & Bl. 151 & N. 11\\%% = ex N. 54 = Cc2 939.
LH XXXV & 9, 11 & Bl. 1\textendash 2 & N. 36\textsubscript{1}\\%% = ex N. 31/1 = Cc2 1189 A, C.
LH XXXV & 9, 11 & Bl. 3\textendash 4 & N. 36\textsubscript{2}\\%% = ex N. 31/2 = Cc2 1189 B, D-G.
LH XXXV & 9, 11 & Bl. 5\textendash 8 & N. 34\textsubscript{1}\\%% = ex N. 29/1 = Cc2 965 A-C, G-H.
LH XXXV & 9, 11 & Bl. 7\textendash 10 & N. 34\textsubscript{2}\\%% = ex N. 29/2 = Cc2 965 D, K.
LH XXXV & 9, 11 & Bl. 9\textendash 12 & N. 34\textsubscript{3}\\%% = ex N. 29/3 = Cc2 965 E, J.
LH XXXV & 9, 11 & Bl. 11\textendash 14 & N. 34\textsubscript{4}\\%% = ex N. 29/4 = Cc2 965 F.
LH XXXV & 9, 11 & Bl. 13\textendash 14 & N. 34\textsubscript{5}\\%% = ex N. 29/5 = Cc2 00 = (RK 41140).
LH XXXV & 10, 9 & Bl. 1 & N. 28\textsubscript{4}\\%% = ex N. 22/4 = Cc2 1190 A.
LH XXXV & 10, 9 & Bl. 2 & N. 28\textsubscript{3}\\%% = ex N. 22/3 = Cc2 1190 B.
LH XXXV & 10, 9 & Bl. 3\textendash 4 & N. 5\\%% = ex N. 61 = Cc2 1191 (tlw.) (3r).
LH XXXV & 10, 9 & Bl. 3\textendash 4 & N. 28\textsubscript{1}\\%% = ex N. 22/1 = Cc2 1190 D (3v).
LH XXXV & 10, 9 & Bl. 3\textendash 4 & N. 28\textsubscript{2}\\%% = ex N. 22/2 = Cc2 1190 C (3v).
LH XXXV & 10, 9 & Bl. 3\textendash 4 & N. 28\textsubscript{5}\\%% = ex N. 22/5 = Cc2 1191 (tlw.) (3r).
LH XXXV & 10, 9 & Bl. 3\textendash 4 & N. 28\textsubscript{6}\\%% = ex N. 22/6 = Cc2 1192 A-B (4r/v).
LH XXXV & 11, 13 & Bl. 9 & N. 89\\%% = ex N. 10 = Cc2 914.
LH XXXV & 12, 1 & Bl. 328-329 & N. 61\\%% = ex N. 69 = Cc2 529.
LH XXXV & 12, 2 & Bl. 62 & N. 16\\%% = N. 16 = Cc2 543 (tlw.).
LH XXXV & 12, 2 & Bl. 150 & N. 99\\%% = ex N. 72 = Cc2 1516.
LH XXXV & 13, 2c & Bl. 144 & N. 40\\%% = ex N. 03 = Cc2 00 = RK 57993.
LH XXXV & 13, 3 & Bl. 35 & N. 18\\%% = ex N. 35 = Cc2 974.
LH XXXV & 13, 3 & Bl. 81 & N. 12\\%% = ex N. 24 = Cc2 1503.
LH XXXV & 13, 3 & Bl. 261\textendash 262 & N. 35\\%% = ex N. 30 = Cc2 947.
LH XXXV & 14, 2 & Bl. 51 & N. 39\\%% = ex N. 56 = Cc2 00 = RK 58246.
LH XXXV & 14, 2 & Bl. 53 & N. 53\\%% = ex N. 01 = Cc2 509.
LH XXXV & 14, 2 & Bl. 103 & N. 56\\%% = ex N. 66 = Cc2 1367.
LH XXXV & 14, 2 & Bl. 104, 108 & N. 59\\%% = ex N. ** = Cc2 1366 A = Extraits de lettres de Mons. Boccone
LH XXXV & 14, 2 & Bl. 105\textendash 107 & N. 60\\%% = ex N. ** = Cc2 1366 B = Notizen zur Botanik
LH XXXV & 14, 2 & Bl. 109\textendash 111 & N. 1\\%% = ex N. 65 (Gassendi) = Cc2 502.
LH XXXV & 14, 2 & Bl. 112\textendash 115 & N. 50\\%% = ex N. 34 (Mariotte) = Cc2 942 A-B.
LH XXXV & 14, 2 & Bl. 114\textendash 115 & N. 9\\%% = ex N. 5 (Wallis 2) = Cc2 941 B.
LH XXXV & 14, 2 & Bl. 114\textendash 115 & N. 80\\%% = Einkf. = Cc2 00.
LH XXXV & 14, 2 & Bl. 116, 125\textendash 126 & N. 35\\%% = ex N. 30 = Cc2 947.
LH XXXV & 14, 2 & Bl. 117\textendash 124 & N. 8\\%% = ex N. 5 (Wallis 1) = Cc2 941 A.
LH XXXV & 14, 2 & Bl. 127\textendash 128 & N. 7\\%% = ex N. 18 = Cc2 423.
LH XXXV & 14, 2 & Bl. 135, 138\textendash 158 & N. 55\\%% = N. 63 = Cc2 00 = RK 55598.
LH XXXV & 15, 6 & Bl. 9\textendash 16 & N. 2\\%% = N. 02 = Cc2 921.
LH XXXV & 15, 6 & Bl. 58 & N. 84\\%% = ex N. 08 = KK1 194 B.
LH XXXV & 15, 6 & Bl. 59 & N. 83\\%% = ex N. 07 = KK1 194 C.
LH XXXV & 15, 6 & Bl. 60 & N. 87\\%% = ex N. 09 = KK1 194 D.
LH XXXV & 15, 6 & Bl. 61 & N. 86\\%% = ex N. 12 = KK1 194 E.
LH XXXV & 15, 6 & Bl. 62 & N. 85\\%% = ex N. 11 = KK1 185 + 194 A.
LH XXXVI &     & Bl. 130\textendash 131 & N. 78\\%% = ex N. 91 = Cc2 508.
LH XXXVII & 3 & Bl. 16 & N. 4\\%% = ex N. 60 = Cc2 485.
LH XXXVII & 3 & Bl. 77\textendash 78 & N. 45\textsubscript{2}\\%% = ex N. 52/2 = Cc2 1213 D.
LH XXXVII & 3 & Bl. 77\textendash 78 & N. 45\textsubscript{3}\\%% = ex N. 52/3 = Cc2 1213 A.
LH XXXVII & 3 & Bl. 79 & N. 45\textsubscript{4}\\%% = ex N. 52/4 = Cc2 1213 C.
LH XXXVII & 3 & Bl. 80 & N. 45\textsubscript{1}\\%% = ex N. 52/1 = Cc2 1213 B.
LH XXXVII & 3 & Bl. 84\textendash 85 & N. 62\\%% = ex N. 73 (Athanor) = RK 60204.
LH XXXVII & 3 & Bl. 86 & N. 97\textsubscript{1}\\%% = ex N. 42/1 = Cc2 1133 A.
LH XXXVII & 3 & Bl. 87 & N. 97\textsubscript{2}\\%% = ex N. 42/2 = Cc2 1142.
LH XXXVII & 3 & Bl. 88 & N. 97\textsubscript{4}\\%% = ex N. 42/4 = Cc2 1141.
LH XXXVII & 3 & Bl. 89 & N. 65\\%% = ex N. 88 = Cc2 1179.
LH XXXVII & 3 & Bl. 162\textendash 163 & N. 48\\%% = ex N. 55 = Cc2 480 A-B.
LH XXXVII & 4 & Bl. 34 & N. 49\\%% = ex N. 04 = Cc2 482.
LH XXXVII & 4 & Bl. 49\textendash 50 & N. 42\\%% = ex N. 40 = Cc2 972.
LH XXXVII & 4 & Bl. 49\textendash 50 & N. 43\\%% = ex N. 53 = Cc2 973 (tlw.).
LH XXXVII & 4 & Bl. 49\textendash 50 & N. 88\\%% = ex N. 94 = Cc2 973 (tlw.).
LH XXXVII & 4 & Bl. 51\textendash 52 & N. 26\\%% = ex N. 48-49 (F/8) = Cc2 967 A.
LH XXXVII & 4 & Bl. 61\textendash 62 & N. 29\\%% = ex N. 37 = Cc2 1504.
LH XXXVII & 5 & Bl. 4\textendash 5, 8\textendash 9 & N. 30\\%% = ex N. 25 = Cc2 945 A.
LH XXXVII & 5 & Bl. 6\textendash 7, 10\textendash 11 & N. 32\\%% = ex N. 27 = Cc2 945 C.
LH XXXVII & 5 & Bl. 6\textendash 7 & N. 33\\%% = ex N. 28 = Cc2 945 E.
LH XXXVII & 5 & Bl. 12 & N. 31\textsubscript{1}\\%% = ex N. 26/1 = Cc2 944.
LH XXXVII & 5 & Bl. 12 & N. 31\textsubscript{2}\\%% = ex N. 26/2 = Cc2 945 B.
LH XXXVII & 5 & Bl. 12 & N. 31\textsubscript{3}\\%% = ex N. 26/3 = Cc2 946.
LH XXXVII & 5 & Bl. 56 & N. 17\textsubscript{1}\\%% = ex N. 15/1 = Cc2 975 A.
LH XXXVII & 5 & Bl. 56 & N. 17\textsubscript{2}\\%% = ex N. 15/2 = Cc2 975 B.
LH XXXVII & 5 & Bl. 57 & N. 92\\%% = ex N. 43 = Cc2 836.
LH XXXVII & 5 & Bl. 58\textendash 59 & N. 93\\%% = ex N. 44 = Cc2 837.
LH XXXVII & 5 & Bl. 92\textendash 93 & N. 94\\%% = ex N. 17 = Cc2 838.
LH XXXVII & 5 & Bl. 120 & N. 27\\%% = ex N. 21 = Cc2 00 = RK 60336.
LH XXXVII & 5 & Bl. 126 & N. 52\\%% = ex N. 36 = Cc2 964 B.
LH XXXVII & 5 & Bl. 127 & N. 37\\%% = ex N. 32 = Cc2 965 L.
LH XXXVII & 5 & Bl. 128\textendash 129 & N. 15\\%% = ex N. 45 = Cc2 969.
LH XXXVII & 5 & Bl. 130 & N. 14\\%% = ex N. 46 = Cc2 976.
LH XXXVII & 5 & Bl. 135\textendash 136 & N. 41\\%%  = ex N. 47 = Cc2 541.
LH XXXVII & 5 & Bl. 139 & N. 51\\%% = ex N. 41 = Cc2 943.
LH XXXVII & 5 & Bl. 142 & N. 38\\%% = ex N. 33 = Cc2 948.
LH XXXVII & 5 & Bl. 201, 204 & N. 19\\%% = ex N. 48-49 (F/1) = Cc2 971 A.
LH XXXVII & 5 & Bl. 201, 204 & N. 20\\%% = ex N. 48-49 (F/2) = Cc2 971 B.
LH XXXVII & 5 & Bl. 202\textendash 203 & N. 21\\%% = ex N. 48-49 (F/3) = Cc2 967 B.
LH XXXVII & 5 & Bl. 207\textendash 208 & N. 25\\%% = ex N. 48-49 (F/7) = Cc2 968 A.
LH XXXVII & 5 & Bl. 209 & N. 22\\%% = ex N. 48-49 (F/4) = Cc2 968 D.
LH XXXVII & 5 & Bl. 210\textendash 211 & N. 23\\%% = ex N. 48-49 (F/5) = Cc2 968 B.
LH XXXVII & 5 & Bl. 210\textendash 211 & N. 24\\%% = ex N. 48-49 (F/6) = Cc2 968 C.
LH XXXVII & 5 & Bl. 215 & N. 10\\%% = ex N. 50 = Cc2 835.
LH XXXVII & 5 & Bl. 216 & N. 98\\%% = ex N. 51 = Cc2 1187.
LH XXXVII & 6 & Bl. 3\textendash 4 & N. 63\\%% = ex N. 70 = Cc2 1054 (tlw.).
LH XXXVII & 6 & Bl. 3\textendash 4 & N. 64\\%% = ex N. 71 = Cc2 1054 (tlw.).
LH XXXVIII &       & Bl. 24 & N. 97\textsubscript{3}\\%% = ex N. 42/3 = Cc2 1133 B.
LH XXXVIII &       & Bl. 25 & N. 28\textsubscript{7}\\%% = ex N. 23 = 1192 C.
LH XXXVIII &       & Bl. 170\textendash 171 & N. 90\\%% = ex N. 13 = Cc2 00 = RK 55810 [A].
LH XXXVIII &       & Bl. 170\textendash 171 & N. 91\\%% = ex N. 14 = Cc2 00 = RK 55810 [B].
LH XLI & 2 & Bl. 9 & N. 77\\%% = ex N. 89 = Cc2 00 = RK 53211.
LH XLII & 1 & Bl. 21 & N. 79\\%% = N. 90 = RK 60607.
%%  MARGINALIEN
\multicolumn{3}{l}{Leibn. Marg. 28} & N. 47\\%% = ex N. 19 = RK 62031.
\multicolumn{3}{l}{Leibn. Marg. 66} & N. 44\\%% = ex N. 20 = RK 62072.
\multicolumn{3}{l}{Leibn. Marg. 126} & N. 13\\%% = ex N. 39 = RK 62051.
\multicolumn{3}{l}{Leibn. Marg. 174} & N. 3\\%% = ex N. 39 = RK 62071.
% \multicolumn{3}{l}{Leibn. Marg. 177} & N. ***\\%% N. *92 = RK 62070.
\multicolumn{3}{l}{Nm \textendash\ A 10003} & N. 46\\%% = ex N. 38 = RK 62068.
\end{longtable}
% \noindent Die letzte Zeile enthält die Signatur eines Handexemplars mit Leibniz' Anstreichungen und Anmerkungen, der nicht durch die Signatur \glqq Leibn. Marg.\grqq\ ausgewiesen ist.% \\[1.0ex]
%\newpage% PR: Rein provisorisch!
%%%%
%%%% Erwähnte Handschriften
%%%%
% \begin{center} \uppercase{Erw\"{a}hnte Leibniz-Handschriften}\end{center}
% Dieses Verzeichnis umfa{\ss}t die in den \"{U}berlieferungen und Erl\"{a}uterungen erw\"{a}hnten, nicht edierten Handschriften. Es ist nach Cc 2-Nummern und Handschriftensignaturen geordnet und verweist auf die Nummern des vorliegenden Bandes.\\
% \begin{center}
% \begin{tabular}{llll}
% Cc 2, Nr. & LH, Nr. & & N.\\[0.5ex]
% ???? & RRR xx,xx & Bl. yyyyy & ***\\
% ???? & RRR xx,xx & Bl. yyyyy & ***\\
% ???? & RRR xx,xx & Bl. yyyyy & ***\\
% ???? & RRR xx,xx & Bl. yyyyy & ***\\
% \end{tabular}
% \end{center}
% \clearpage% PR: Rein provisorisch!
%
% \vspace{3.0ex}
% 
%\clearpage
%%
%% 
%% Hier endet das Handschriftenverzeichnis. 
%%%%
% \newpage%
\vspace{4.0ex}%
%%%%%%%%%%%%%%%%% ERWÄHNTE HANDSCHRIFTEN
%
\begin{center} \uppercase{Erwähnte Leibniz-Handschriften}\end{center}
%
Im Folgenden sind die in den Köpfen oder Erläuterungen erwähnten, nicht edierten Leibniz-Handschriften verzeichnet.
Das Register ist nach Fundort und Signatur geordnet und verweist auf die entsprechenden Katalogeinträge (wenn vorhanden) sowie auf die Stücke im Band, in denen die jeweilige Handschrift erwähnt wird.
%
\\[4.0ex]%[1.0ex]% PR: Rein provisorisch!
% \begin{multicols}{1}
% \centering\footnotesize{\uppercase{Erwähnte Leibniz-Handschriften}}% Doppelt-Backslash muss gelöscht werden. 
 % \setlength{\columnseprule}{0.4pt}
% \\[2.0ex]%[1.0ex]\vspace{4.0ex}% PR: Rein provisorisch!
% \centering%
\begin{tabular}{lllll}
\textsc{Hannover}, \textit{GWLB} & LH XXXV 9, 5 & Bl. 26 & Cc 2, Nr. 00 & N. 36\\%%
 & LH XXXV 12, 1 & Bl. 280-281 & Cc 2, Nr. 787 & N. 57%%
% Doppelt-Backslash muss vor dem Befehl \end{tabular} gelöscht werden
\end{tabular}
% \columnbreak
% \end{multicols}
% \noindent
% \vspace{1.0ex}% PR: Rein provisorisch!
%%
\clearpage
%%
%% 
%% Hier enden die Handschriftenverzeichnisse und die Konkordanzen. Es folgen die Siglenverzeichnisse.