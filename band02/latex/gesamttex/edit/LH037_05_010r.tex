\begin{ledgroupsized}[r]{120mm}
\footnotesize
\pstart
\noindent\textbf{\"{U}berlieferung:}
\pend
\end{ledgroupsized}            
\begin{ledgroupsized}[r]{114mm}
\footnotesize
\pstart \parindent -6mm
\makebox[6mm][l]{\textit{L}}Konzept: LH XXXVII 5 Bl. 6-7, 10-11. 2 Bog. 4\textsuperscript{o}. 6 S. Textfolge: Bl. 10, 11 und 6 (ein Kustos am Ende von Bl. 11~v\textsuperscript{o} verweist auf den Anfang von Bl.~6~r\textsuperscript{o}). Auf B.~7~r\textsuperscript{o} finden sich N.~33 sowie Rechnungen (Cc 2, Nr.~945~D), welche in \textit{LSB} VII ediert werden. Bl.~7~v\textsuperscript{o} ist leer. Die Bogen, s\"{a}mtlich durch Papiererhaltungsma{\ss}nahmen gesichert, tragen mittig jeweils verschiedene Wasserzeichentypen. \\Cc 2, Nr. 945 C
\pend
\end{ledgroupsized}

%\normalsize
\vspace*{5mm}
\begin{ledgroup}
\footnotesize 
\pstart
\noindent\footnotesize{\textbf{Datierungsgr\"{u}nde}: An einer Stelle (S.~\refpassage{037,05_011r_scheda-1}{037,05_011r_scheda-2}) verweist der eigenh\"{a}ndig datierte Text allem Anschein nach auf N.~31\textsubscript{2}. Demgem\"{a}{\ss} muss N. 32 sp\"{a}ter als N. 31\textsubscript{2} entstanden sein.}
\pend
\end{ledgroup}

\vspace*{8mm}
\pstart 
\noindent [10~r\textsuperscript{o}]
\pend
\count\Bfootins=1200
\count\Afootins=1200
\count\Cfootins=1200
\pstart
\centering
\textso{ De detrimento motus}\protect\index{Sachverzeichnis}{detrimentum motus}\textso{ pars 2} \textsuperscript{\textso{da}}\textso{ }\edtext{}{\lemma{pars 2\textsuperscript{da}}\Cfootnote{Es ist nicht klar, auf welches St\"{u}ck die Bezeichnung \textit{pars secunda} Bezug nimmt.}}April 1675
\pend
\count\Bfootins=1000
\count\Cfootins=1000
\pstart
\vspace*{0.5em}% PR: nur provisorisch
\noindent
Mechanici scriptores plerique olim, non nisi de quinque Machinis Fundamentalibus\protect\index{Sachverzeichnis}{machinae fundamentales},
ut vocant 
% PR: getilgter Kommentar: \edtext{}{\lemma{ut vocant}\Cfootnote{????Vgl. z.B. \cite{01009}\textsc{Guidobaldo del Monte,} \textit{Mechanicorum liber}, Pesaro 1577. Die Kapitel dieser Abhandlung tragen als \"{U}berschriften die Namen der von Leibniz erw\"{a}hnten f\"{u}nf einfachen Maschinen.????}}
loquebantur, Vecte\protect\index{Sachverzeichnis}{vectis}, Cuneo\protect\index{Sachverzeichnis}{cuneus}, Axe in Peritrochio\protect\index{Sachverzeichnis}{axis in peritrochio}, Trochlea\protect\index{Sachverzeichnis}{trochlea}, Cochlea\protect\index{Sachverzeichnis}{cochlea}:
de Libra\protect\index{Sachverzeichnis}{libra} etiam et Hydrostaticis\protect\index{Sachverzeichnis}{hydrostatica}, quaedam
\edtext{ex Archimede\protect\index{Namensregister}{\textso{Archimedes}, 287-212 v. Chr.}}{\lemma{ex Archimede}\Cfootnote{\cite{01010}\textsc{Archimedes}, \textit{De aequiponderantibus}; \cite{01011}\textit{De corporibus fluitantibus}.}}
petita \edtext{adjiciebant.}{\lemma{adjiciebant}\Cfootnote{Vgl. z.B. \cite{01012}\textsc{Guidobaldo del Monte}, \textit{In duos Archimedis aequeponderatium libros paraphrasis scholiis illustrata}, Pesaro 1588. \cite{01013}F. \textsc{Commandino}, \textit{Archimedis de iis quae vehuntur in aqua libri duo restituti et commentariis illustrati}, Bologna 1565.}}
Caetera, id est potissimam negotii partem industriae atque experientiae artificum relinquebant.
Primus \edtext{Galilaeus\protect\index{Namensregister}{\textso{Galilei} (Galilaeus, Galileus), Galileo 1564-1642} aliquid}{\lemma{Galilaeus}\Bfootnote{\textit{(1)}\ magnum \textit{(2)}\ aliquid \textit{L}}}
adjecit Archimedi\protect\index{Namensregister}{\textso{Archimedes}, 287-212 v. Chr.}, quod memoratu dignum esset:
cum firmitates solidorum, et impetum\protect\index{Sachverzeichnis}{impetus} illum, quem ex ipso
\edtext{motu gravia}{\lemma{motu}\Bfootnote{\textit{(1)}\ corpora \textit{(2)}\ gravia \textit{L}\ }}
concipiunt \edtext{calculo subjecisset.}{\lemma{calculo subjecisset}\Cfootnote{\cite{00050}\cite{00048}G. \textsc{Galilei}, \textit{Discorsi}, Leiden 1638.}}
Nostro tempore Mathematici insignes
\edtext{feliciter}{\lemma{}\Bfootnote{feliciter \textit{erg.} \textit{L}\ }}
laborant in Elatere\protect\index{Sachverzeichnis}{elater}
\edtext{penitus eruendo}{\lemma{penitus}\Bfootnote{\textit{(1)}\ detegendo \textit{(2)}\ eruendo \textit{L}\ }};
quo in genere et
\edtext{a me}{\lemma{a me}\Cfootnote{Stelle nicht nachgewiesen.}}
\edtext{observata sunt}{\lemma{}\Bfootnote{observata  \textbar\ quoque praestita \textit{gestr.}\ \textbar\ sunt \textit{L}}}
non pauca.
Unum argumentum video intactum Geometris, calculo tamen inprimis dignum esse; quod\textso{ Detrimentum }\protect\index{Sachverzeichnis}{detrimentum}appello.
Constat rotas,
\edtext{funes, currus[,] naves}{\lemma{funes,}\Bfootnote{\textit{(1)}\ naves motas, \textit{(2)}\ currus naves, \textit{L}}},
\edtext{libramenta\protect\index{Sachverzeichnis}{libramentum}}{\lemma{}\Bfootnote{libramenta\protect\index{Sachverzeichnis}{libramentum} \textit{ erg.} \textit{ L}}}
ipso \edtext{contactu axium, trochlearum, pavimenti, medii liquidi}{\lemma{contactu}\Bfootnote{\textit{(1)}\ medii, fundi \textit{(2)}\ axium [...] liquidi \textit{L}}}
plurimum retardari: et inanes eorum conatus a rerum natura deludi,
\edtext{qui vulgaribus Mechanicorum praeceptis freti}{\lemma{qui}\Bfootnote{\textit{(1)}\ contactu corporum in punctis freti \textit{(2)}\ vulgaribus [...] freti \textit{L}}},
calculo male subducto ingentia opera viribus non suffecturis aggrediuntur:
Id ergo nunc agendum est sedulo, tum ut calculari possit hoc virium detrimentum, tum ut machinis quantumlicet emendatis, evitetur.
\edtext{Scimus ingenioso clarissimi Perralti\textso{ Barulco,} commentariis ad Vitruvium
\edtext{adjecto,}{\lemma{adjecto}\Cfootnote{\cite{01014}\textsc{Vitruvius}, \textit{Les dix livres d'architecture}, hrsg. von C. \textsc{Perrault}, Paris 1673, l. X, ch. V, S. 280f. und 324f. Keine der dort beschriebenen Maschinen wird allerdings \textit{barulcus} genannt. F\"{u}r diesen auf Heron von Alexandria zur\"{u}ckgehenden Begriff siehe vielmehr \cite{01015}\textsc{Pappus}, \textit{Mathematica collectio}, l.~VIII, probl. VI, prop. X.}}}{\lemma{Scimus}\Bfootnote{\textit{(1)}\ ingeniosum extare clarissimi Perralti\protect\index{Namensregister}{\textso{Perrault} (Perraltus), Claude 1613-1688} Barulcon \textit{(2)}\ ingenioso [...] \textso{Barulco}, \textit{(a)}\ quod commentariis ad Vitruvium\protect\index{Namensregister}{\textso{Vitruvius} Pollio, Marcus ca. 70-10 v. Chr.} adjecit \textit{(b)}\ commentariis [...] adjecto, \textit{L}}}
ingenti virium lucro maximam detrimenti partem evitari;
et\textso{ antlia }\protect\index{Sachverzeichnis}{antlia}ingeniosi cujusdam juvenis
\edtext{adhibito}{\lemma{adhibito}\Cfootnote{Stelle nicht nachgewiesen.}}
Torricelliano invento summa facilitate aquam haurit;
et\textso{ currus }qui planam ipse sibi viam substernit, nulla itineris asperitate retardatur;
et\textso{ libra }haberi \edtext{potest, cujus summae subtilitati}{\lemma{potest,}\Bfootnote{\textit{(1)}\ in qua \textit{(2)}\ tantae subtilitatis, ut \textit{(3)}\ cujus summae subtilitati \textit{L}}}
nihil omnino decedat attritu\protect\index{Sachverzeichnis}{attritus} motus circa axem.
\edtext{Et\textso{ Rotarum dentatarum}}{\lemma{Et}\Bfootnote{\textit{(1)}\ dentium \textit{(2)}\ \textso{Rotarum dentatarum} \textit{L}}}
inventa est \edtext{forma qua dens a dente aequali semper facilitate moveatur: quod utile fuisset tum inprimis cum pendulorum usus ignoraretur}{\lemma{forma}\Bfootnote{\textit{(1)}\ quae ante pendula\protect\index{Sachverzeichnis}{pendulum} reperta \textit{(2)}\ qua \textit{(a)}\ aequalis  \textit{(aa)}\ difficultate \textit{(bb)}\ semper rotarum a rotis agendarum \textit{(b)}\ dens [...] cum \textit{(aa)}\ pendulo\protect\index{Sachverzeichnis}{pendulum} \textit{(bb)}\ pendulorum usus ignoraretur. \textit{L}}}.
\pend
\count\Bfootins=1000
\count\Cfootins=1000
\pstart
Quanquam enim \edtext{Aristoteles\protect\index{Namensregister}{\textso{Aristoteles}, 384-322 v. Chr.}}{\lemma{Aristoteles}\Cfootnote{Stelle nicht nachgewiesen.}}
\edtext{crederet\textso{ detrimenti }vitium supra}{\lemma{crederet}\Bfootnote{\textit{(1)}\ hoc ma \textit{(2)}\ \textso{detrimenti} vitium supra \textit{L}}}
remedium esse; ostendunt tamen magis magisque ingenia seculi
\edtext{nostri, vix esse de ulla re desperandum. Quoniam}{\lemma{nostri,}\Bfootnote{%
\textit{(1)}\ non esse quod de ulla re desperemus %
\textit{(2)}\ vix [...] desperandum. %
\textit{(a)}\ Et %
\textit{(b)}\ Quoniam \textit{L}}}
tamen certum est non posse omnem omnino attritum evitari:
nam et projecta ab aere\protect\index{Sachverzeichnis}{aer} tardantur, et naves ab
\edtext{aqua, et funes ab orbiculis et orbiculi rotaeque ab axibus}{\lemma{aqua,}\Bfootnote{\textit{(1)}\ et rotae ab axibus \textit{(2)}\ et funes [...] ab axibus; \textit{L}}};
ideo \edtext{detrimenta virium}{\lemma{detrimenta}\Bfootnote{\textit{(1)}\ celeritatis \textit{(2)}\ virium \textit{L}}}
sub calculum vocari rei Mechanicae interesse
\edtext{putavi.\\
%\newpage
\hspace*{7,5mm}
Theoremata}{\lemma{}\Bfootnote{putavi. \textbar\ Rem prima fronte facilem aggressus tam abstrusam reperi et profundam, ut non jam amplius mirarer intactam. \textit{gestr.}\ \textbar\ Theoremata \textit{L}}}
autem reperi expectatione
\edtext{pulchriora in}{\lemma{pulchriora}\Bfootnote{\textit{(1)}\ ex \textit{(2)}\ in \textit{L}}}
quibus illud eminet: vim
\edtext{per se}{\lemma{}\Bfootnote{per se \textit{erg.} \textit{L}}}
aequabilem a medio homogeneo\protect\index{Sachverzeichnis}{medium homogeneum} diminui
\edtext{progressione Geometrica, cum \mbox{tamen} progressionem Geometricam nemo quod sciam hactenus exhibuerit motu quodam physico. Gravium}{\lemma{progressione}\Bfootnote{%
\textit{(1)}\ Arithmetica %
\textit{(2)}\ Geometrica, %
\textit{(a)}\ progressionem autem Geometricam nemo quod sciam hactenus spatio quodam exhibuit in rerum natura %
\textit{(b)}\ cum [...] physico. %
\textit{(aa)}\ Cujus rei %
\textit{(bb)}\ Ratio est difficultatis, quod progressio Geometrica solet crescere per intervalla. %
\textit{(cc)}\ Gravium \textit{L}}}
acceleratio\protect\index{Sachverzeichnis}{acceleratio gravium}\textso{ Arithmeticae }subest%
% [10~v\textsuperscript{o}]
% \pend