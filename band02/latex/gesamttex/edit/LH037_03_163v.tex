% [163~v\textsuperscript{o}]
\pstart%
Si sint duo corpora\protect\index{Sachverzeichnis}{corpora cylindrica} \edtext{cylindrica axes $AB$ et $CD$}{\lemma{cylindrica}\Bfootnote{\textit{(1)} $AB$ et $CD$ \textit{(2)} axes $AB$ et $CD$ \textit{L}}} habentia perpendiculares basibus $A$ et $B$ item $C$ et $D$ basium seu latitudinum aequalium, corpus vero $AB$ sit longius \edtext{utcunque}{\lemma{utcunque}\Bfootnote{\textit{erg. L}}} corpore $CD$ et corpora cylindrica\protect\index{Sachverzeichnis}{corpora cylindrica} ita sint locata, ut eorum inter se bases sint parallelae, axes autem in una linea recta $AD$ et corpora in ea recta axes connectente, integris basibus sibi occurrere intelligantur, motu aequiveloce; ajo fore ut post concursum\protect\index{Sachverzeichnis}{concursus} utrumque concurrentium quiescat in loco concursus\protect\index{Sachverzeichnis}{concursus}. \pend \pstart Nam momento concursus\protect\index{Sachverzeichnis}{concursus} utrumque ab altero impellitur, id est alterum conatur in alterius locum. Conatur, id est \edtext{incipit intrare}{\lemma{incipit}\Bfootnote{\textit{(1)} inesse \textit{(2)} intrare, \textit{L}}}, (nam per \edtext{alibi}{\lemma{alibi}\Cfootnote{Etwa \cite{00259}\textit{Theoria motus abstracti}, \textit{LSB} VI, 2 N.~41, S.~265.}} ostensa, conatus\protect\index{Sachverzeichnis}{conatus} omnis est initium motus, etsi minus quolibet assignabili) ergo incipit alterum expellere seu abripere secum, alterum ergo incipit, seu quod idem est conatur \edtext{exire, nec}{\lemma{exire,}\Bfootnote{\textit{(1)} utrum \textit{(2)} nec \textit{L}}} refert unum altero esse longius, nam unum non potest incipere intrare in alterius locum, \edtext{quia alterum}{\lemma{quia}\Bfootnote{\textit{(1)} ipsum \textit{(2)} alterum \textit{L}}} totum quantaecunque sit longitudinis abripere conetur ut qui baculum\protect\index{Sachverzeichnis}{baculus} in uno extremo impellit, totum impellit, longitudinis licet quantaecunque.
Utrumque ergo conatum\protect\index{Sachverzeichnis}{conatus} habet et suum et alienum, ac proinde duos habet conatus aequales in diversa.
Ergo quiescet.
\pend 
\count\Afootins=1500
\count\Bfootins=1500
\count\Cfootins=1500