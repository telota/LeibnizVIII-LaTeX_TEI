\count\Afootins=1200
\count\Bfootins=1200
\count\Cfootins=1200
\pstart Hevelius\protect\index{Namensregister}{\textso{Hevelius}, Johannes (1611-1687)} ipse suas fecit divisiones sui Quadrantis\protect\index{Sachverzeichnis}{quadrans} aenei quem primo describit. Subdivisiones fecit modo Tychonico ductis circulis parallelis, sed distantias circulorum parallelorum fecit aequales, \edtext{in Tychone\protect\index{Namensregister}{\textso{Brahe}, Tycho (1546-1601)}}{\lemma{}\Bfootnote{in Tychone\protect\index{Namensregister}{\textso{Brahe}, Tycho (1546-1601)} \textit{erg. L}}} accuratius fecisset, ponendo eorum distantias \textit{according to the proportions of the differences of the secants of some ten minutes}, \edtext{[\textit{next}]}{\lemma{}\Bfootnote{\textit{nex L \"{a}ndert Hrsg.}}} \textit{successively following one another in some degree of the} \edtext{\textit{quadrant\protect\index{Sachverzeichnis}{quadrans}}}{\lemma{\textit{quadrant}}\Cfootnote{\textsc{R. Hooke}, \textit{Animadversions}, S. 12.}} \textit{wich} \edtext{[\textit{is}]}{\lemma{}\Bfootnote{\textit{is: erg. Hrsg. nach Vorlage}}} \textit{easie to determine from the distance of the two extream or bounding} \edtext{\textit{circles}}{\lemma{\textit{circles}}\Cfootnote{a.a.O., S. 12.}}. Sed modo spatium in quo jacent circuli sit valde largum, et modo partes graduum distinguendae sint exiguae; error ab aequalibus \edtext{distantiis contemnendus}{\lemma{}\Bfootnote{distantiis \textbar\ non \textit{gestr.}\ \textbar\ contemnendus \textit{L}}} praesertim pro nudo visu. Modus Hookii\protect\index{Namensregister}{\textso{Hooke}, Robert (1635-1703)} quo labor ad partem nonagesimam reduci potest: pro divisione per diagonales; nimirum divisio unius gradus serviet pro omnibus 90. Certior est quoque, et exactior. Est autem talis: sume frustum tenue, \textit{of a lookings glas} \edtext{\textit{platte}}{\lemma{\textit{platte}}\Cfootnote{a.a.O., S. 13.}}; speculi plani. Ab utraque parte lene politum ac laevigatum et satis largum d'un sens (of \textit{one} \edtext{\textit{way}}{\lemma{\textit{way}}\Cfootnote{a.a.O., S. 13.}},) ut tegere possit omnem illam quadrantis\protect\index{Sachverzeichnis}{quadrans} partem in qua diagonales fieri necesse est. In alteram autem plagam, seu alio sensu, teget duos aut tres gradus quadrantis\protect\index{Sachverzeichnis}{quadrans} \textit{(: This I do the bigger that the sides of the arm may} \edtext{\textit{not}}{\lemma{\textit{not}}\Cfootnote{a.a.O., S. 13.}} \edtext{[inumbrare]}{\lemma{}\Bfootnote{inumbare \textit{L \"{a}ndert Hrsg.}}} et obnigrare divisiones et numerationes. Vult credo dicere ideo \edtext{hanc vitream tabulam}{\lemma{hanc}\Bfootnote{\textit{(1)}\ laminam \textit{(2)}\ vitream tabulam \textit{L}}} a se tam fieri largam, ut sustentacula ejus sint satis remota ab illis numeris, quibus videndis opus habemus, quales sunt ipsi qui loco quo utimur, in quo dioptra\protect\index{Sachverzeichnis}{dioptra} ojectiva est, proximi sunt; ne tegant divisiones. :) Hanc jam tabulam prorsus ita divide, ut Hevelius\protect\index{Namensregister}{\textso{Hevelius}, Johannes (1611-1687)} divisit \edtext{[ipsum]}{\lemma{}\Bfootnote{ipsam \textit{L \"{a}ndert Hrsg.}}} ubique quadrantem\protect\index{Sachverzeichnis}{quadrans}, nisi quod si paulo largius est spatium illud, proportione \edtext{ad radium}{\lemma{ad}\Bfootnote{\textit{(1)}\ gradum \textit{(2)}\ radium \textit{L}}} circuli paralleli non sint aequidistantes, sed secundum Tabulam tangentium aut secantium naturalium elaborati. Has divisiones facies circinis quorum pedum extrema adamantibus instructa, qualibus utuntur et vitrarii. Itaque divisiones fac, et duc lineas et pone \textit{in the frame of the} \edtext{\textit{ruler}}{\lemma{\textit{ruler}}\Cfootnote{a.a.O., S. 14.}} (formam vel regulae) ita ut latus linearum immediate tangat quadrantem\protect\index{Sachverzeichnis}{quadrans}. Ipse aeneus quadrans divisus sit in 90 partes aequales vel gradus et ex quolibet puncto divisionis rectae ad circumferentiam ductae per totam quadrantis\protect\index{Sachverzeichnis}{quadrans} faciem seu usque ad ipsum centrum \edtext{vel saltem quousque pertingunt}{\lemma{centrum}\Bfootnote{\textit{(1)}\ si scilicet eousque pertingunt \textit{(2)}\ vel saltem quousque pertingunt \textit{L}}} divisiones pro diagonalibus in vitro. \edtext{\textit{The frame}}{\lemma{\textit{The frame}}\Cfootnote{a.a.O., S. 14.}}, (forma sustentaculum) cujus ope movetur vitrum, est conveniens cavitas relicta in mobili brachio quadrantis\protect\index{Sachverzeichnis}{quadrans}. Figura, inquit, haec reddet clariora. Distantiae parallelorum circulorum secundum numeros tangentium et secantium naturales \edtext{[sumentur]}{\lemma{}\Bfootnote{sumetur \textit{L \"{a}ndert Hrsg.}}} \textit{with a pair of} \edtext{\textit{compasses}}{\lemma{\textit{compasses}}\Cfootnote{a.a.O., S. 14.}}, pari circinorum (+ cur pari seu duobus? +) \textit{contrived like} \edtext{\textit{beam-compasses\protect\index{Sachverzeichnis}{beam-compass}}}{\lemma{\textit{beam-compasses}}\Cfootnote{a.a.O., S. 14.}} (beam, statera, bilanx). Sed quorum puncta ad distantiam datam ponuntur ope cochleae quae movetur super una parte \edtext{\textit{of the beam}}{\lemma{\textit{of the beam}}\Cfootnote{a.a.O., S. 14.}}. Quod forte alibi describam clarius. 
\pend 
\count\Bfootins=1500
\count\Cfootins=1500
\pstart Si haec ratio non placet, pergit Hookius\protect\index{Namensregister}{\textso{Hooke}, Robert (1635-1703)}, \textso{alia} adhiberi potest, cujus ope feci exigua instrumenta valde exiguarum divisionum, valdeque exacta et facilia. Primum limbum (planum, latus) quadrantis\protect\index{Sachverzeichnis}{quadrans} dividendi reddo summe planum, \edtext{inde super}{\lemma{}\Bfootnote{inde \textbar\ inde \textit{streicht Hrsg.} \textbar\ super \textit{L}}} eo describo circulum ita levem ac subtilem, ut non nisi discerni possit, et ope Laminae divisoriae communis satis longae, radii decem pedum, divido in 90 partes, inde singulari artificio punctorum quorundam curiosorum, \textit{that strikes with a} \edtext{\textit{spring}}{\lemma{\textit{spring}}\Cfootnote{a.a.O., S. 14.}} (quae ope elaterii tundunt seu percutiunt) quae in alio discursu describo notantur gradus in lamina per curiosas exiguas, rotundas, profundas cavitates; haec per aliam lineam extra, divisam et figuratam communi more, distinguuntur et numerantur per figuras (+ numeros, cyphras +) communi more. Inde pro subdivisionibus facio exiguum \edtext{\textit{Hold-fast\protect\index{Sachverzeichnis}{hold-fast}}}{\lemma{\textit{Hold-fast}}\Cfootnote{a.a.O., S. 14.}}, (Tenaculum\protect\index{Sachverzeichnis}{tenaculum}, Estoc) firmatum per coch\-leam, fixum ad mobile quadrantis\protect\index{Sachverzeichnis}{quadrans} brachium, quod inservit ad tenendum extremum diagonalis crinis vel capilli, cujus alterum extremum super gradum \edtext{supplementarium donec}{\lemma{}\Bfootnote{supplementarium \textbar\ (. \Denarius\ .) \textit{gestr.}\ \textbar\ donec \textit{L}}} directe jaceat super aliqua cavitate subtilium divisionum limbi quadrantis\protect\index{Sachverzeichnis}{quadrans}. Hoc dat subdivisiones quadrantis\protect\index{Sachverzeichnis}{quadrans} quanta volo exactitudine. \textso{Supplementarius gradus} est gradus ingentis circuli, positus super exigua regula mobili, fixus in latere brachii mobilis, cujus magnitudo et distantia hac proportione invenitur; scilicet ut extremum inter distantiam exigui tenaculi et circuli punctati, est ad radium ejus circuli ita fiat distantia inter dictum extremum et supplementarium circulum, ad hujus circuli radium. Subjicit descriptionem, cum figura, sed figuram non invenio in exemplari meo, forte, quia deest una ex Tabulis autographicis. Describam tamen. Sit $aaa$ in figura \edtext{[32\textsuperscript{da}]}{\lemma{}\Bfootnote{30\textsuperscript{ma}\textit{\ L und Vorlage, \"{a}ndert Hrsg. nach Vorlage S. 78}}} repraesentans quadrantem\protect\index{Sachverzeichnis}{quadrans}, $b.bb$ subtilem circulum interiorem, descriptum super limbo quadrantis\protect\index{Sachverzeichnis}{quadrans} ex centro $l$ quem ope largi quadrantis\protect\index{Sachverzeichnis}{quadrans} radii decempedalis divido in gradus, et ope puncti elastici tundo in eo totidem exigua puncta, et numero 90 incipiendo ab $m$ et numerando ad $i$. Pone $dd$ repraesentare brachium mobile. \textit{cc} \edtext{\textit{the Hold-fast\protect\index{Sachverzeichnis}{hold-fast}}}{\lemma{\textit{the Hold-fast}}\Cfootnote{a.a.O., S. 15.}}, tenaculum\protect\index{Sachverzeichnis}{tenaculum}; fixum super latere ejus brachii. Hoc tenaculum ope exiguae cochleae tenet subtilem capillum, at $k$. \pend 
\pstart $ee$ est exigua regula fixa ad angulos rectos cum linea $lkf$. In hac linea (per puncta $l$ et $k$ \edtext{[\textit{through}]}{\lemma{}\Bfootnote{\textit{trough L \"{a}ndert Hrsg.}}} \textit{the points $l$ and $k$.)}\edtext{}{\lemma{\textit{and} $k$.)}\Cfootnote{a.a.O., S. 15.}} sumo punctum, ut $f$ et per $fI$ describo (+ distingue $l.i.I.$ +) partem circuli $fg$ cujus centrum est alicubi in linea $fkl$ producta, quod invenio resolvendo hanc proportionem, ut $ki$ ad $li$ ita $kf$ erit ad radium supplementarii circuli $fg$ quod cadet alicubi in $fkl$ productam, versus $l$. Inde sume gradum ejus circuli, qui extendetur ab $f$ ad $g$ et divide in divisiones tam minutas quam videtur opus, et numera ab $f$ ad $g$ nunc invenias quem angulum dioptra\protect\index{Sachverzeichnis}{dioptra} $dd$ facit cum dioptra\protect\index{Sachverzeichnis}{dioptra} $mm$ extendo pilum $hk$ donec inveniam jacere super proximo divisionis puncto versus dextram, et observo in regula $ee$ quae pars gradus ibi notata, et ad circulum $bbb$ quis gradus ibi notatus, summa utriusque dat veram mensuram anguli $ddlm$. Sed haec inquit obiter, quae fusius describam in alio discursu, ubi varias exhibebo rationes mechanicas et practicas dividendi accurate lineas in ullum numerum assignabilem partium proportionalium. Sed redeamus inquit ad quadrantem\protect\index{Sachverzeichnis}{quadrans} Hevelii\protect\index{Namensregister}{\textso{Hevelius}, Johannes (1611-1687)}. Is movetur ope cochlearum, sed alterandus pro variis azimuthis\protect\index{Sachverzeichnis}{azimuth}, quod nos infra evitabimus. \edtext{Idem Hookius\protect\index{Namensregister}{\textso{Hooke}, Robert (1635-1703)}}{\lemma{Idem}\Bfootnote{\textit{(1)}\ Hevelius \textit{(2)}\ Hookius \textit{(3)}\ Hookius \textit{L}}}, ego mox docebo modum quo unus observator potest sumere distantias in semicirculo exactius [11~r\textsuperscript{o}] multo quam possint duo, adeoque erit usui insigni pro nautis et astronomis. 
\pend 