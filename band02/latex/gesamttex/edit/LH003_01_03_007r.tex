[7~r\textsuperscript{o}]
\pend%
\newpage
\pstart%
Dieweil ich daf\"{u}r halte der geschmack\protect\index{Sachverzeichnis}{Geschmack} sey das beste instrument, die Natur der dinge zu erfahren,\edtext{}{\lemma{}\Afootnote{\textit{Am Rand:} En mâchant le cosmetique de la nacre de perles, découuert}}
\edtext{[also]}{\lemma{als}\Bfootnote{\textit{L \"{a}ndert Hrsg.}}}
mu{\ss} man alle mittel suchen, dadurch gewi{\ss}e
\edtext{Menschen zu einem in hoch\-sten grad subtilen schmack\protect\index{Sachverzeichnis}{Geschmack} gelangen.%
}{\lemma{Menschen}\Bfootnote{\textit{(1)} bestellet \textit{(2)} zu einem [...] gelangen. \textit{L}}}
Nun ist bekand, da{\ss} die Menschen so nur wa{\ss}er trincken, so subtil im geschmack seyn, da{\ss} sie auch ein wa{\ss}er vom andern am geschmack\protect\index{Sachverzeichnis}{Geschmack} unterscheiden k\"{o}nnen, welches andere nicht verm\"{o}gen. Derowegen mus man gewi{\ss}e menschen mit fast insipidis\protect\index{Sachverzeichnis}{insipidus}, als wa{\ss}er und
\edtext{brodt, oder mehl auf}{\lemma{brodt,}\Bfootnote{\textit{(1)} auff \textit{(2)} oder mehl auf \textit{L}}}
Tartarische manier nehren. Diese weil sie auch solche andern Menschen pro insipidis gehaltene Dinge unterscheiden k\"{o}nnen, werden die sapida\protect\index{Sachverzeichnis}{sapidus} vielmehr subtiliter unterscheiden. Hiehehr gehoren auch die k\"{u}nste der weinh\"{a}ndler, umb einen reinen schmack\protect\index{Sachverzeichnis}{Geschmack} zu haben. Man soll allemahl wa{\ss}er vorhehr kosten, ehe man sonst etwas kosten will.
\pend%
\pstart%
Wenn man die observationes des geschmacks\protect\index{Sachverzeichnis}{Geschmack} etwa mit einem gewi{\ss}en instrumento als menstruo\protect\index{Sachverzeichnis}{menstruum} salino etc. concordant funden,
\edtext{so kan}{\lemma{so}\Bfootnote{\textit{(1)} mus \textit{(2)} kan \textit{L}}}
man alsdann des instruments sich anstats geschmacks\protect\index{Sachverzeichnis}{Geschmack} gebrauchen. Gleichwie wenn man einmahl wei{\ss} da{\ss} ein wa{\ss}er gesalzen sey, kan man aus dem gewicht gradum salsedinis\protect\index{Sachverzeichnis}{gradus salsedinis} ohne schmack\protect\index{Sachverzeichnis}{Geschmack} determiniren.
\pend%
\pstart%
Man mus gewi{\ss}e Menschen in der Republick halten die im geruch\protect\index{Sachverzeichnis}{Geruch} exquisit seyn[,]
gewi{\ss}e menschen die im f\"{u}hlen\protect\index{Sachverzeichnis}{F\"{u}hlen},
wie der \edtext{blinde beym Boyle.\protect\index{Namensregister}{\textso{Boyle}, Robert 1627-1691}%
}{\lemma{blinde beym Boyle}\Cfootnote{\cite{00014}\textsc{R. Boyle}, \textit{Experiments and considerations  touching colours}, London 1664, S.~41-49.}}
%
Solche abtheilungen der menschen sind nothiger als die abtheilungen der handwerge.
\pend%
\pstart%
Man mu{\ss} die bucher so die leute aufmuntern ad realia offt aufflegen unter die leute austheilen in viele sprachen vertiren la{\ss}en. Den kinders la{\ss}en beyzeiten in schuhlen proponiren,
ut \edtext{Vives,\protect\index{Namensregister}{\textso{Vives}, Juan Luis 1492-1540}%
}{\lemma{Vives}\Cfootnote{Siehe etwa \cite{01141}\textsc{J.L. Vives}, \textit{De ratione studii puerilis}, Oxford 1523; \cite{01142}\textsc{Ders.}, \textit{De institutione feminae christianae}, Oxford 1523.}}
%
\edtext{Baconus,\protect\index{Namensregister}{\textso{Bacon}, Francis 1561-1626}%
}{\lemma{Baconus}\Cfootnote{\cite{00270}\textsc{F. Bacon}, \textit{Instauratio magna}, London 1620.}}
%
\edtext{Cartesii\protect\index{Namensregister}{\textso{Descartes}, Ren\'{e} (1596-1650)} methodus.%
}{\lemma{Cartesii methodus}\Cfootnote{\cite{01140}\textsc{R. Descartes}, \textit{Discours de la m\'{e}thode}, Leiden 1637.}}
\pend%
\pstart%
\edtext{Turcae opio\protect\index{Sachverzeichnis}{opium} ad hilaritatem uti solent[,]
putant enim \textit{colorem faciei\protect\index{Sachverzeichnis}{color faciei} egregium inducere, hominisque animum recreare},
ut qui \textit{semel eo usus sit, nunquam non delectetur.}
Sorantius\protect\index{Namensregister}{\textso{Soranzo}, Lazzaro, vor 1572-1602} \textit{Ottomanno} p. 2. n. 49. p. 63.%
}{\lemma{Turcae [...] p. 63}\Cfootnote{\cite{00097}\textsc{L. Soranzo}, \textit{Ottomannus sive De imperio Turcico}, pars II, n. XLIX, in H.~\textsc{Conring} (Hrsg.), \textit{De bello contra Turcas prudenter gerendo}, Helmstedt 1664, S. 63.}}%
% Hier folgt Bl. 7v.