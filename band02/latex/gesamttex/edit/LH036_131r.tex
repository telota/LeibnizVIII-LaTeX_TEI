[131~r\textsuperscript{o}]
\pend%
\vspace*{8mm}%
\pstart%
\normalsize%
\noindent%
\centering%
\lbrack\textit{Teil 2}\rbrack%
\pend%
\vspace*{0.5em}% PR: Rein provisorisch !!!
\pstart%
\noindent%
\centering
Onomastici rerum militarium ad hodiernam consuetudinem pars altera
\pend%
\vspace*{0.5em}
\pstart%
\textso{Adjoutant}. adjutor. praefecti vigilum adjutor.
\pend%
\pstart%
\textso{Auditor generale}. militarium causarum generalis quaesitor.
\pend%
\pstart%
\textso{Capitaine}. centurio.
\pend%
\pstart%
\textso{Capitaine} de campagne. centurio castrensis.
\pend%
\pstart%
\textso{Capitaine de la garde}. praefectus praetoriano militi.
\pend%
\pstart%
\textso{Caporale}. decurio.
\pend%
\pstart%
\textso{Cavalli da bagaglio}. Equi sarcinarii.
\pend%
\pstart%
\textso{Colonel} praefectus legioni. chiliarchus. Tribunus.
\pend%
\pstart%
\textso{Commissario generale della Cavalleria}. Equitum commissarius.
\edtext{commissorum in Equestri}{\lemma{commissorum}\Bfootnote{\textit{(1)} Equitum \textit{(2)} in Equestri \textit{ L}}}
militia curator.
\pend%
%\newpage
\pstart%
\textso{Compagnia} di cavalli volante. Expedita levis armaturae turma.
\pend%
\pstart%
\textso{Cornetta}. Equestre vexillum.
\pend%
\pstart%
\textso{Cornetta del generale}. Labarum. Imperatorum vexillum.
\pend%
\pstart%
\textso{Dragons}. dimachae.
\pend%
\pstart%
\textso{General}. Imperator.
\pend%
\pstart%
\textso{Ingeniero}. Machinator bellicus. a bellicis machinamentis. Machinali scientia clarus.
\pend%
\pstart%
\textso{Gouuerneur}. praefectus oppidi.
\pend%
\pstart%
\textso{Lieutenant}. Optio.
\pend%
\pstart%
\textso{Moschetti}. majores sclopi. Tubi furcillis librari soliti.
\pend%
\pstart%
\textso{Moschettieri}. majores sclopetarii, sclopetarii furcillis sclopo, librantes.
\pend%
\pstart%
\textso{Pistola}. fistula ferrea.
\pend%
\pstart%
\textso{officiers}. praefecti tribuni.
\pend%
\pstart%
\textso{Polvere d'archibugio}. pulvis bellicus.
\pend%
\pstart%
\textso{Prevost}. quaesitor militaris.
\pend%
\pstart%
\textso{Un pont sur pilotis}. pons sublicius.
\pend%
\pstart%
\textso{porterseigne}. vaender. Aquilifer. vexillarius (signifer)
\pend%
\pstart%
\textso{Punta del baloardo}. rostrum propugnaculi.
\pend%
\pstart%
\textso{Rinforzi}. subsidia copiae subsidiariae.
\pend%
\pstart%
\textso{Ronda}. circitor.
\pend%
\pstart%
\textso{Salvo condotto}. fides publica. assertiae litterae. liberi commeatus tessera.
\pend%
\pstart%
\textso{Sergeant major General}. summus vigilum praefectus.
\pend%
\pstart%
\textso{Sergeant}. satelles.
\pend%
\pstart%
\textso{Sergeante d'una compania}. instructor centuriae.
\pend%
\pstart%
\textso{Sergeante d'un terzo}. legionis instructor.
\pend%
\vspace*{1.0em}%
\pstart%
\noindent%
Nav.
\pend%
\vspace*{0.5em}
\pstart%
\textso{Admiral}. Navis praetoria.
\pend%
\pstart%
\textso{Vice Admiral}. vicaria. seu quae locum dignitate proximum tenet.
\pend%
\pstart%
\textso{Schout by nacht}. quae tertio gradu in classe censetur.
\pend%
\pstart%
Totam nomenclaturam navium veterum et recentium
vide \edtext{apud Ricciolum\protect\index{Namensregister}{\textso{Riccioli}, Giambattista 1598-1671}
in \title{Geo\-graphia reformata} lib. 10. cap. 35. pag. 526. etc.%
}{\lemma{vide [...] etc.}\Cfootnote{\cite{01046}\textsc{G. Riccioli}, \title{Geographiae et hydrographiae reformatae libri duodecim}, Bologna 1661, S.~526-529 (Schiffnamen) und S.~530-533 (Namen von Schiffteilen).}}
\pend%
\count\Bfootins=1500
\count\Cfootins=1500
\count\Afootins=1500
%%%%  PR: Hier endet das Stück.