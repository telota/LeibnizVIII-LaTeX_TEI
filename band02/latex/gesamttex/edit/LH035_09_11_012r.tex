\count\Bfootins=800
\begin{Geometrico}% PR: Erste Zeile bitte hängend (more geometrico).
Th. 4.
Le m\^{e}me estant
\edtext{pos\'{e}, les temps employez}{\lemma{pos\'{e},}\Bfootnote{\textit{(1)}\ les espaces \`{a} parcou \textit{(2)}\ les temps employez \textit{L}}}
seront repraesentez \edtext{par les portions hyperboliques [comprises] entre}{\lemma{par les}\Bfootnote{%
\textit{(1)}\ espaces hyperboliques %
\textit{(a)}\ en %
\textit{(b)}\ comprises %
\textit{(c)}\ compris entre %
\textit{(2)}\ portions hyperboliques compris entre \textit{L ändert Hrsg.}}}
deux \edtext{ordonn\'{e}es \`{a} l'asymptote, $\displaystyle EBA$ dont}{\lemma{ordonn\'{e}es}\Bfootnote{\textit{(1)}\ dont \textit{(2)}\ \`{a} l'asymptote,\ \textbar\ $\displaystyle EBA$ \textit{ erg.}\ \textbar\  dont \textit{L}}}
l'une \edtext{$\displaystyle EF$}{\lemma{}\Bfootnote{$\displaystyle EF$ \textit{erg. L}}}
passe par le point dont le mobile\protect\index{Sachverzeichnis}{mobile} est parti, $\displaystyle E$
et l'autre $\displaystyle BD$ ou $\displaystyle (B)(D)$ passe par le point
\edtext{$\displaystyle B$ ou $\displaystyle (B)$}{\lemma{$\displaystyle B$ ou $\displaystyle (B)$}\Bfootnote{\textit{erg. L}}}
o\`{u} il est arriv\'{e}.
\end{Geometrico}
%\newpage
\pstart
\noindent
Car par la precedente, les temps croissent
\edtext{comme $\displaystyle EF.$ $\displaystyle BD.$ $\displaystyle (B)(D)$ etc. ordonn\'{e}es}{\lemma{comme}\Bfootnote{\textit{(1)}\ les ap \textit{(2)}\ $\displaystyle EF.$ $\displaystyle BD.$ $\displaystyle (B)(D)$ etc.  \textit{(a)}\ appliqu\'{e}es  \textit{(b)}\ ordonn\'{e}es \textit{L}}}
\`{a} l'asymptote de
\edtext{l'hyperbole, ou comme les rectangles $\displaystyle FEB.$ $\displaystyle DB(B).$ $\displaystyle (D)(B)P$ dont les longueurs sont les dites ordonn\'{e}es, et la largeur constante est une portion infiniment petite de l'espace $\displaystyle EA$, s\c{c}avoir $\displaystyle EB$ egale \`{a} $\displaystyle B(B)$ ou $\displaystyle (B)P$ (parce que les rectangles dont les largeurs sont les m\^{e}mes sont en raison des longueurs) ou comme les espaces $\displaystyle FEBDF.$ $\displaystyle DB(B)(D)D.$ $\displaystyle (D)(B)PQ(D)$ parce que la difference entre ces espaces et les rectangles susdits est de nulle consideration quand les intervalles $\displaystyle EB.$ $\displaystyle B(B)$ etc. sont infiniment petits}{\lemma{l'hyperbole,}\Bfootnote{\textit{(1)}\ ou comme les espaces infiniment petits, $\displaystyle FEBDF.$ $\displaystyle DB(B)(D)D.$ $\displaystyle (D)(B)PQ(D)$ dont la largeur $\displaystyle EB$ egale \`{a} $\displaystyle B(B)$  \textit{(a)}\ etc.  \textit{(b)}\ ou $\displaystyle (B)P$ etc. est infiniment petite, et dont la longueur est \textit{(2)}\ ou comme [...] dont  \textit{(a)}\ la longeur est  \textit{(b)}\ les longueurs [...] ordonn\'{e}es, et  \textit{(aa)}\ les largeurs infiniment petites  \textit{(bb)}\ la  \textit{(aaa)}\ constante  \textit{(bbb)}\ largeur constante est  \textit{(aaaa)}\ la  \textit{(bbbb)}\ infi  \textit{(cccc)}\ la  \textit{(dddd)}\ $\displaystyle EB$ egale  \textit{(eeee)}\ la  \textit{(ffff)}\ une portion [...] parce que  \textit{(aaaaa)}\ le res  \textit{(bbbbb)}\ ces espaces  \textit{(ccccc)}\ la difference [...] est  \textit{(aaaaa a)}\ de nulle  \textit{(bbbbb b)}\ infini  \textit{(ccccc c)}\ de nulle [...] petits. \textit{L}}}.
Or les \edtext{sommes de ces}{\lemma{sommes}\Bfootnote{\textit{(1)}\ des \textit{(2)}\ de ces \textit{L}}}
\edtext{espaces de}{\lemma{espaces}\Bfootnote{ \textbar\ hyperboliques \textit{gestr.}\ \textbar\ de \textit{L}}}
largeur infiniment
\edtext{petite qui representent les accroissemens du temps, sont les portions}%
{\lemma{petite}\Bfootnote{%
\textit{(1)}\ sont %
\textit{(2)}\ qui representent [...] sont les %
\textit{(a)}\ espaces %
\textit{(b)}\ portions \textit{L}}}
Hyperboliques dont il est  parl\'{e} dans nostre proposition,
s\c{c}avoir $\displaystyle FEBDF$ ou $\displaystyle FE(B)(D)F$ ou $\displaystyle FEPQF$,
\edtext{donc ces portions}{\lemma{donc}\Bfootnote{%
\textit{(1)}\ les %
\textit{(2)}\ ces %
\textit{(a)}\ espaces %
\textit{(b)}\ portions \textit{L}}}
Hyperboliques repraesenteront les sommes des accroissemens
\edtext{ou particelles}{\lemma{ou particelles}\Bfootnote{\textit{erg. L}}}
du temps; c'est \`{a} dire tout le temps qui a est\'{e} employ\'{e} depuis le commencement du mouuement,
jusqu' au point o\`{u}
\edtext{le}{\lemma{}\Bfootnote{le \textit{erg. L}}}
mobile se trouue, $\displaystyle B$ ou $\displaystyle (B)$ ou $\displaystyle P$.
\pend
%\newpage
\begin{Geometrico}
\edtext{Th. 5.
Si le mouuement d'un corps}{\lemma{Th. 5.}\Bfootnote{\textit{(1)}\ Si l \textit{(2)}\ Un corps dont le mouuement \textit{(3)}\ Si [...] corps \textit{L}}}
\edtext{est uniforme}{\lemma{est}\Bfootnote{\textit{(1)}\ uniformement \textit{(2)}\ uniforme \textit{L}}}
en soy \edtext{m\^{e}me, mais}{\lemma{m\^{e}me,}\Bfootnote{\textit{(1)}\ mais \textit{(2)}\ estant \textit{(3)}\ mais \textit{L}}}
\edtext{retard\'{e} egalement}{\lemma{retard\'{e}}\Bfootnote{\textit{(1)}\ uniformement \textit{(2)}\ egalement \textit{L}}}
par le lieu o\`{u} il passe[,]
les espaces qui restent \`{a} parcourir jusqu'au point de repos estant comme les nombres,
les temps employez
\edtext{d\'{e}j\`{a}, aussi bien que les temps qui restent \`{a} employer [seront]}{\lemma{d\'{e}j\`{a},}\Bfootnote{\textit{(1)}\ seront \textit{(2)}\ aussi [...] employer \textit{L}\  \textbar\ seront \textit{erg. Hrsg.}}}
comme [leurs]\edtext{}{\Bfootnote{leur \textit{\ L \"{a}ndert Hrsg.}}}
Logarithmes;
ou qui est la m\^{e}me chose,
les temps employez d\'{e}j\`{a} croissant
\edtext{et les temps qui restent \`{a} employer, d\'{e}croissant,}{\lemma{}\Bfootnote{et les temps [...] d\'{e}croissant, \textit{erg. L}}}
en progression Arithmetique;
les espaces \`{a} parcourir d\'{e}croistront en progression Geometrique.
\end{Geometrico}
%\newpage
\pstart 
\noindent
Car les temps employez d\'{e}j\`{a} \edtext{sont comme les portions}{%
\lemma{sont}\Bfootnote{%
\textbar\ comme \textit{erg.} \textbar\ les %
\textit{(1)} espaces %
\textit{(2)} portions \textit{L}}}
Hyperboliques $\displaystyle FEBDF.$ $\displaystyle DB(B)(D)D.$ $\displaystyle (D)(B)PQ(D)$ par la precedente
\edtext{et par consequent les temps qui restent \`{a} employer seront comme les portions hyperboliques $\displaystyle DBPQD.$
 $\displaystyle (D)(B)PQ(D)$}{\lemma{}\Bfootnote{et par consequent [...] comme les \textit{(1)}\ espaces \textit{(2)}\ portions hyperboliques  \textit{(a)}\ $\displaystyle PB(B)(D)D.$ $\displaystyle (D)(B)PQ(D)$ etc.  \textit{(b)}\ $\displaystyle DBPQD.$ $\displaystyle (D)(B)PQ(D)$ \textit{erg. L}}}.
Or les espaces \`{a} parcourir ou les droites $\displaystyle BA.$ $\displaystyle (B)A.$ $\displaystyle PA$ estant en progression Geometrique,
les differences \edtext{des dites portions}{\lemma{des}\Bfootnote{\textit{(1)}\ dits espaces \textit{(2)}\ dites portions \textit{L}}}
Hyperboliques, ou les espaces de largeur infiniment petite,
s\c{c}avoir $\displaystyle FEBDF.$ $\displaystyle DB(B)(D)D.$ $\displaystyle (D)(B)PQ(D)$ sont egaux entre
\edtext{eux (:~comme il}{\lemma{eux (:}\Bfootnote{\textit{(1)}\ par ce qui \textit{(2)}\ comme il \textit{L}}}
a \edtext{est\'{e} d\'{e}couuert}{\lemma{est\'{e}}\Bfootnote{\textit{(1)}\ invent\'{e} \textit{(2)}\ d\'{e}couuert \textit{L}}}
par le Pere \edtext{Gregoire de S. Vincent\protect\index{Namensregister}{\textso{Saint-Vincent}, Gr\'{e}goire de (Gregorius a S. Vincentio) S.J. 1584-1667}}{\lemma{Gregoire de S. Vincent}\Cfootnote{\cite{00316}\textit{Opus geometricum}, Antwerpen 1647, lib. VI, prop. 129, S. 596f.}}~:)
dont \edtext{les dites portions}{\lemma{les}\Bfootnote{\textit{(1)}\ dites \textit{(2)}\ dits espaces \textit{(3)}\ dites portions \textit{L}}}
Hyperboliques, ou les temps employez
\edtext{ou \`{a} employer}{\lemma{}\Bfootnote{ou \`{a} employer \textit{erg. L}}},
qu'ils representent
sont en progression Arithmetique;
\makebox[1.0\textwidth][s]{les espaces \`{a} parcourir estant en progression
\edtext{Geometrique; et en renversant}{\lemma{Geometrique;}\Bfootnote{\textit{(1)}\ c'est \`{a} \textit{(2)}\ et en renversant, \textit{L}}},
ceux cy}
\pend
\newpage
\pstart\noindent estant comme les nombres,
ceux l\`{a} seront comme
[leurs]\edtext{}{\Bfootnote{leur\textit{\ L \"{a}ndert Hrsg.}}}
logarithmes.
Ce qu'il falloit d\'{e}monstrer.
\pend
\pstart
Il faut \edtext{remarquer qu'on peut conceuuoir deux sortes de Logarithmes ou termes de progression arithmetique \`{a} l'egard des termes de progression Geometrique, car ou les termes et [les] logarithmes croissent tous deux, ou les uns croissent les autres d\'{e}croissent. Ceux qui croissent avec les termes se trouuent dans les tables}{\lemma{remarquer}\Bfootnote{\textit{(1)}\ que ces Logarithmes sont un peu differens de ceux qui se trouuent dans les tables \textit{(2)}\ qu'on [...] les termes et\ \textbar\ les \textit{erg. Hrsg.}\ \textbar\ logarithmes [...] les tables \textit{L}}}
des Logarithmes des nombres absolus.
Car une droite comme $\displaystyle AP$ estant prise pour l'unit\'{e},
\edtext{les portions}{\lemma{les}\Bfootnote{\textit{(1)}\ espaces \textit{(2)}\ portions \textit{L}}}
Hyperboliques,
[prises]\edtext{}{\Bfootnote{pris\textit{\ L \"{a}ndert Hrsg.}}}
d'une maniere oppos\'{e}e \`{a} celle
\edtext{dont j'avois}{\lemma{dont}\Bfootnote{\textit{(1)}\ j'ay \textit{(2)}\ j'avois \textit{L}}}
parl\'{e} \edtext{dans la proposition precedente}{\lemma{}\Bfootnote{dans [...] precedente \textit{erg. L}}};
\edtext{s\c{c}avoir les portions Hyperboliques $\displaystyle DBPQD.$ $\displaystyle (D)(B)PQ(D)$}{\lemma{s\c{c}avoir}\Bfootnote{\textit{(1)}\ $\displaystyle QP(B)(D)Q.$ $\displaystyle (D)(B)BD(D)$ \textit{(2)}\  $\displaystyle DBEFD.$ \textit{(3)}\ $\displaystyle QPEFQ.$ \textit{(4)}\ $\displaystyle QP(B)(D)Q.$ $\displaystyle QPBDQ.$ $\displaystyle QBEFQ$ \textit{(5)}\ $\displaystyle DB(B)D.$ $\displaystyle (D)(B)PQ(D)$ \textit{(6)}\ les  \textit{(a)}\ espaces  \textit{(b)}\ portions [...] $\displaystyle (D)(B)PQ(D)$ \textit{L}}}
seront les logarithmes des raisons des
\edtext{nombres $\displaystyle AB.$ $\displaystyle A(B)$}{\lemma{nombres}\Bfootnote{%
\textit{(1)}\ $\displaystyle A(B).$ %
\textit{(2)}\ $\displaystyle AB.$ %
\textit{(a)}\ $\displaystyle AE$ %
\textit{(b)}\ $\displaystyle A(B)$ \textit{L}}}
\count\Bfootins=1500% [12~v\textsuperscript{o}]
% \pend