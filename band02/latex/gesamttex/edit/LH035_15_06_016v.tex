[16~v\textsuperscript{o}] \textso{VIII}\textsuperscript{\textso{vus}} \textso{usus} pro diametris solis\protect\index{Sachverzeichnis}{diametrus solis}, $\rightmoon$\ planetarum, ad minutum usque secundum, et distantia \textit{of the smaller appearing planets from the }\edtext{\textit{fixt stars}}{\lemma{\textit{fixt}}\Bfootnote{\textit{(1)}\ \textit{sterns} \textit{(2)}\ \textit{stars} \textit{L}}}\textit{, near} \edtext{\textit{adjoyning.}}{\lemma{\textit{adjoyning}.}\Cfootnote{a.a.O., S. 77.}} Sed quoniam pro hoc usu videbitur res forte paulo operosior et ob brevitatem Tubi, justo arctior, ideo inveni instrumentum radii sextuplo majoris, \textit{wich will take in an angle of about 5 degrees} (credo dicere vult totum instrumentum non esse quadrantem\protect\index{Sachverzeichnis}{quadrans} neque octantem etc. sed non nisi 5 graduum) \textit{and yet take in the whole
\edtext{angle by one glance of the
\edtext{eye%
}{\lemma{\textit{eye}}\Cfootnote{a.a.O., S. 78.}}%
}{\lemma{\textit{angle}}\Bfootnote{\textit{(1)}\ \textit{of about 5 degrees} \textit{(2)}\ \textit{by} [...] \textit{eye} \textit{L}}}%
}
%\textit{and yet take in the whole angle by one glance of the eye}\edtext{}{\lemma{\textit{angle}}\Bfootnote{\textit{(1)}\ \textit{of about 5 degrees} \textit{(2)}\ \textit{by} [...] \textit{eye} \textit{L}}}\edtext{}{\lemma{\textit{eye}}\Cfootnote{a.a.O., S. 78.}} 
(uno oculi ictu) et determinare ejus mensuram, ad partem secundo minorem. Inveni etiam et feci novum helioscopium, cujus ope corpus solis intueri \edtext{possumus oculorum offensa, non majore}{\lemma{possumus}\Bfootnote{\textit{(1)}\ sine ulla \textit{(2)}\ oculorum offensa, non majore \textit{L}}} quam si albae chartae folium intueremur, res magni usus pro observationibus \pgrk{f}ysicis in sole. Haec in sequentibus quibusdam chartis \edlabel{hooke1}describam. 
\pend 
\pstart \textso{IX}\textsuperscript{\textso{us}}\textso{ usus}\edlabel{hooke2}\edtext{}{{\xxref{hooke1}{hooke2}}\lemma{describam.}\Bfootnote{\textit{(1)}\ Nonus \textit{(2)}\ \textso{IX}\textsuperscript{\textso{us}}\textso{ usus} \textit{L}}} pro exacte sumenda libella aquis de loco in locum. Ducendis aliisque infinitis sub hoc capite philosophicis \edtext{experimentis praestandis,}{\lemma{}\Bfootnote{experimentis \textbar\ saepe \textit{gestr.}\ \textbar\ praestandis, \textit{L}}} vix alia ratione possibilibus circa frangibilitatem aeris circumterrestris, quo loca dissita modo supra modo infra horizontem apparent. Eadem arte etiam inveniri \edtext{potest terrae rotunditas}{\lemma{potest}\Bfootnote{\textit{(1)}\ longitudo \textit{(2)}\ terrae rotunditas \textit{L}}} vera, quod nullis aliis cognitis libellis possit. Multa alia nominari possent, sed haec impraesentiarum videntur suffectura. 
\pend
\count\Afootins=1500
\count\Bfootins=1500
\count\Cfootins=1500 