\begin{ledgroupsized}[r]{120mm}%
\footnotesize%
\pstart%
\noindent\textbf{\"{U}berlieferung:}%
\pend%
\end{ledgroupsized}%
\begin{ledgroupsized}[r]{114mm}%
\footnotesize%
\pstart%
\parindent -6mm%
\makebox[6mm][l]{\textit{L}}%
Auszüge aus \cite{01081}\textsc{A. Kircher}, \textit{Magneticum naturae regnum}, Amsterdam 1667:
\textsc{Göttingen}, Stadtarchiv, MSL Nr.~12,
 Bl. 19.
1 Bl. 8\textsuperscript{o}. 1 S. auf Bl. 19~r\textsuperscript{o}. Bl. 19~v\textsuperscript{o} leer (bis auf eine fremdhändige Zahl).
Am Fuß von Bl. 19~r\textsuperscript{o} eine gegenläufige fremdhändige Textzeile: \textit{§~7 sicut et ult.: 1031}.%
\newline%
Cc 2, Nr. 00%
\pend%
\end{ledgroupsized}%
%
\vspace*{5mm}%
\begin{ledgroup}%
\footnotesize%
\pstart%
\noindent%
\footnotesize{%
\textbf{Datierungsgr\"{u}nde:}
Beim Textträger des vorliegenden Stücks handelt es sich sehr wahrscheinlich -- wie es den Spuren fremdhändigen Textes zu entnehmen ist -- um Papier für das \textit{Corpus juris reconcinnatum} (siehe hierzu \textit{LSB} VI, 2, S. XXI\,f.).
Die Auszüge dürften daher aus der Mainzer Zeit stammen und nach Beginn der Arbeiten am \title{Corpus juris reconcinnatum} angefertigt worden sein.%
}%
\pend%
\end{ledgroup}%
%
%
\vspace*{8mm}%
\count\Bfootins=1200
\count\Cfootins=1200
\count\Afootins=1200
\pstart%
\normalsize%
\noindent%
[19~r\textsuperscript{o}] 
\edtext{\cite{01081}Kircher. \title{Magnet. Nat. Regn.}
\edtext{Sect. 2.}{\lemma{Sect. 2.}\Bfootnote{\textit{erg. L}}}
Cap. 5.%
}{\lemma{Kircher [...] Cap. 5}\Cfootnote{%
\cite{01081}\textsc{A. Kircher}, \textit{Magneticum naturae regnum}, sectio II, cap. 5, Rom 1667.
Leibniz zitiert die im selben Jahr in Amsterdam erschienene Ausgabe des Werkes,
die nicht seitenidentisch mit der römischen Aus\-gabe ist.}}
%
\edtext{p. 68. Bufo\protect\index{Sachverzeichnis}{bufo}
seu Rubeta\protect\index{Sachverzeichnis}{rubeta} sole exiccata et tuberibus\protect\index{Sachverzeichnis}{tuber} applicata pestiferis,
infectum peste\protect\index{Sachverzeichnis}{pestis} liberat, ut in nostro \cite{01082}\title{Scrutinio physico medico de peste} docuimus.%
}{\lemma{p. 68 [...] docuimus}\Cfootnote{\cite{01081}a.a.O., S. 68.
Siehe zudem \cite{01082}\textsc{A. Kircher}, \textit{Scrutinium physico-medicum contagiosae Luis, quae pestis dicitur}, Rom 1658.}}
%
\pend%
\pstart%
\edtext{Sect. 2.}{\lemma{Sect. 2.}\Bfootnote{\textit{erg. L}}} \edtext{p.\textso{ 70.}
\textit{Erphordiae\protect\index{Ortsregister}{Erfurt}
in Germania\protect\index{Ortsregister}{Deutschland}
seplasiarium\protect\index{Sachverzeichnis}{seplasiarius} quendam morsu viperae\protect\index{Sachverzeichnis}{vipera} ad extremum vitae discrimen} adductum \textit{cum omnibus adhibitis antidotis\protect\index{Sachverzeichnis}{antidotum} nulla alia venenum\protect\index{Sachverzeichnis}{venenum} pellendi remedium\protect\index{Sachverzeichnis}{remedium} superessent, carnium\protect\index{Sachverzeichnis}{caro} ejusdem viperae\protect\index{Sachverzeichnis}{vipera}
%
quae momorderat esu, perfecte sanatum vidi.}\edtext{}{\lemma{}\Afootnote{\textit{Am Absatzende:} Non puto referre multum istud: \textit{quae momorderat.}\vspace{-8mm}}}%
}{\lemma{p. \textso{70} [...] \textit{vidi}}\Cfootnote{\cite{01081}\textsc{A.~Kircher}, \textit{Magneticum naturae regnum}, sectio II, cap. 5, Rom 1667, S.~70.}}
%
\pend%
\pstart%
\edtext{Sect. 2.}{\lemma{Sect. 2.}\Bfootnote{\textit{erg. L}}} \edtext{p.\textso{ 82.}
\textit{Reperitur in India}\protect\index{Ortsregister}{Indien}
virulentissimus serpens,\protect\index{Sachverzeichnis}{serpens}
\textit{quem ego Basiliscum indicum} voco;
necat \edtext{afflatu, sed ut}{\lemma{afflatu,}\Bfootnote{\textit{(1)}\ nec \textit{(2)}\ sed ut \textit{ L}}}
vitari possit natura ejus caudae velut crepundia\protect\index{Sachverzeichnis}{crepundia}
annexuit, vocant campanam aut tintinnabulum\protect\index{Sachverzeichnis}{tintinnabulum}
unde insurgens, strepitu monet.%
}{\lemma{p. \textso{82} [...] monet}\Cfootnote{\cite{01081}a.a.O., S. 82.}}
\pend%
\pstart%
\edtext{\textit{P. Emanuel Luisius Romam advena}\protect\index{Ortsregister}{Rom}
unum \textit{secum attulit, mirum in modum ad primum attactum perstrepens},
est enim \textit{vesicaria quaedam textura\protect\index{Sachverzeichnis}{textura}}%
}{\lemma{\textit{P. Emanuel} [...] \textit{textura}}\Cfootnote{\cite{01081}a.a.O., S. 83.\hspace{20mm}}}
%
instar \edtext{pellis diaphanae vesicae\protect\index{Sachverzeichnis}{vesica}
suillae\protect\index{Sachverzeichnis}{suilla}
piso\protect\index{Sachverzeichnis}{pisum}
indito et agitato mirum strepitum\protect\index{Sachverzeichnis}{strepitus} edentis.
Lusitanis serpens:\protect\index{Sachverzeichnis}{serpens}
\textit{La cobra de cascavel.}
Crependia\protect\index{Sachverzeichnis}{crependia} praebent remedium praesentissimum
contra multa venena,\protect\index{Sachverzeichnis}{venenum}
et fuit inprimis praesentissimum remedium\protect\index{Sachverzeichnis}{remedium}
contra Epilepsiam,\protect\index{Sachverzeichnis}{epilepsia}
\textit{si vel collo affixum}
\edtext{[\textit{portetur}]}{\lemma{potetur}\Bfootnote{\textit{ L \"{a}ndert Hrsg. nach Vorlage}}},
\textit{vel aqua\protect\index{Sachverzeichnis}{aqua} cardiaca
dilutum potu\protect\index{Sachverzeichnis}{potus} sumatur.}%
}{\lemma{pellis [...] \textit{sumatur}}\Cfootnote{\cite{01081}a.a.O., S. 84.}}
%
\pend%
\count\Bfootins=1500
\count\Cfootins=1500
\count\Afootins=1500