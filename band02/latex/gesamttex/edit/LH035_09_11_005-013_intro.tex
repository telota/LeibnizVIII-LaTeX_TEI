\footnotesize 
\pstart
\noindent                
Bei den folgenden f\"{u}nf St\"{u}cken handelt es sich um eine Gruppe von Texten, die sowohl dem Inhalt als auch der Entstehung nach eine geschlossene Einheit bilden. Das mechanische Ph\"{a}nomen der Reibung wird dort als Ursache der gleichm\"{a}ßigen Verz\"{o}gerung eines sich durch ein homogenes Medium hindurch bewegenden K\"{o}rpers betrachtet. Leibniz ist vornehmlich um eine geometrische Beschreibung des Sachver\-haltes bestrebt, f\"{u}r die er erneut die logarithmische Funktion verwendet; Hintergrund der Untersuchung ist erkl\"{a}rtermaßen Galileis Darstellung der gleichm\"{a}ßigen Beschleunigung fallender K\"{o}rper. In N. 34\textsubscript{4} unterscheidet er ferner explizit zwischen zwei f\"{u}r die Verz\"{o}gerung verantwortlichen Widerstandsarten des Mediums: einer von der Geschwindigkeit des beweglichen K\"{o}rpers unabh\"{a}ngigen \textit{r\'{e}sistance absolue} und einer zu dessen Geschwindigkeit proportionalen \textit{r\'{e}sistance respective}. Die Gruppe ist im Mai 1675 entstanden: S\"{a}mtliche Texttr\"{a}ger sind der Reihe nach eigenh\"{a}ndig nummeriert und datiert; zudem weisen sie den gleichen Wasserzeichentypus auf. Leibniz hat der Untersuchung im Laufe der Bearbeitung verschiedene \"{U}berschriften verliehen, die bei den folgenden Einzelst\"{u}cken wiedergegeben werden. Die sp\"{a}tere N. 36 ist als Weiterentwicklung von N. 34 anzusehen.
\pend
\normalsize