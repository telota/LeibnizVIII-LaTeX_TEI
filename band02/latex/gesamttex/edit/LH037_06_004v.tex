\begin{ledgroupsized}[r]{120mm}%
\footnotesize%
\pstart%
\noindent\textbf{\"{U}berlieferung:}%
\pend%
\end{ledgroupsized}%
%
\begin{ledgroupsized}[r]{114mm}%
\footnotesize%
\pstart%
\parindent -6mm%
\makebox[6mm][l]{\textit{L}}%
Aufzeichnung: LH XXXVII 6 Bl. 3-4.
1 Bog. 2\textsuperscript{o}. 1 S. auf Bl. 4~v\textsuperscript{o} und letzte 11 Z. auf Bl.~3~r\textsuperscript{o}.
Bl. 3~v\textsuperscript{o} und 4~r\textsuperscript{o} leer.
Bl. 3~r\textsuperscript{o} überliefert zudem N.~64.% LH 037,06_003r
\newline%
Cc 2, Nr. 1054 (tlw.)%
\pend%
\end{ledgroupsized}%
% \normalsize
\vspace*{5mm}%
%
\begin{ledgroup}%
\footnotesize
\pstart
\noindent\footnotesize{\textbf{Datierungsgr\"{u}nde:}
Das vorliegende Stück ist auf demselben Textträger überliefert wie das von Leibniz eigenhändig auf September 1675 datierte Stück N.~64. % LH 037,06_003r
% Dieser Text ist nach Inhalt und Anordnung auf dem Blatt in direkter Umgebung zu dem von Leibniz datierten Text auf der Vorderseite entstanden, so da{\ss} die gleiche Entstehungszeit angenommen werden kann.%
}%
\pend%
\end{ledgroup}%
%
\vspace{6mm}%
\pstart
\noindent
[4~v\textsuperscript{o}]
\pend
\pstart%
\normalsize%
\centering%
% [4 v\textsuperscript{o}]
\textso{Ars conficiendi omnis generis gemmas}% Bitte als Überschrift formattieren.
\pend%
\vspace{2mm}% PR: Rein provisorisch !!!
\count\Bfootins=1200
\count\Cfootins=1200
\count\Afootins=1000
\pstart%
\noindent%
\edtext{}{\lemma{}\Afootnote{\textit{Am Rand:} (1)}}%
Wie man das Cristall preparirn soll. Nim schohnen Berg Cristall\protect\index{Sachverzeichnis}{Bergkristall}, so viel Du wilt, in einem reinen Tiegel\protect\index{Sachverzeichnis}{Tiegel} decke ihn zu mit einem andern Tiegel\protect\index{Sachverzeichnis}{Tiegel}
\edtext{dass}{\lemma{}\Afootnote{\hspace{-1.8mm}\textit{Am Rand:} quoad  corpus}}
kein staub oder aschen darein falle. Seze dasselbige in ein Kohlfe\"{u}er und la{\ss} es wohl durch gl\"{u}hen, hernach nim es aus dem feuer und sch\"{u}tte es also gl\"{u}end in dein gross geschirr mit reinem kalten wa{\ss}er. Thue es wieder in Tiegel\protect\index{Sachverzeichnis}{Tiegel} und gl\"{u}e es wieder, und lesche es im wa{\ss}er ab, da{\ss} thue zu 12 mahlen nacheinander.
Wann es dann gnug calcinirt ist, so truckne den Cristall sauber, und reibe ihn auff einen harten porphyr-stein zum aller subtilsten staub-pulver. Es muss in keinen metallinen M\"{o}rser gestossenn noch auff einen weichen marmorstein gerieben werden sonst wird es vom Metall oder weichen marmorstein soglich darunter menget, verunreiniget, und ist zu dieser arbeit nicht tauglich, also siehe wohl zu da{\ss} alles auffs reineste und subtilste gemacht werde, und verschaffe dir also dieses pulvers eine zimliche quantit\"{a}t, denn die{\ss} ist aller edelgestein prima materia.
\pend%
\count\Afootins=1200
\pstart%
\textso{Smaragden}\protect\index{Sachverzeichnis}{Smaragd}\textso{ zu machen.}
Nim des preparirten Cristals 2 Unzen; Minie 4 unzen diese zwey vermische auffs beste als immer m\"{o}glich auff einem harten reibestein thue dazu 45 gran gr\"{u}nspan. und Croci \mars\textsuperscript{tis} 8 gran nach fleissiger Vermischung thue die materi in einen reinen tiegel, doch das der tiegel nicht viel \"{u}ber die helfte voll sey; denn es thut sich anfangs hoch auff damit es nicht mag \"{u}berlauffen; der tiegel mus auch mit einem Deckel verlutirt werden, dann seze es in einen ofen und gieb ihm starck fe\"{u}er mit d\"{u}rrem holz, bey 24 stunden, und siehe wohl zu, dass es in gleicher hize stehe, als ob man gold schmelzte, dann las es wohl erkalten und mache den tiegel auff, wann du nun siehest dass es schohn clar und durchsichtig geschmolzen so schlag den Tiegel\protect\index{Sachverzeichnis}{Tiegel} enzwey, so wirstu finden einen sch\"{o}hnen Smaragd\protect\index{Sachverzeichnis}{Smaragd}. Ist es aber noch blasericht, so verlutir den tiegel wieder oder seze ihn wieder in den Windofen, bis es sch\"{o}hn clar und rein wird, und diese manier soltu mercken bey allen farben, dann es ist einerley arbeit und handgriff.
\pend%
\pstart%
Eine\textso{ ander art schmaragd }\protect\index{Sachverzeichnis}{Smaragd}zu machen.
Nim des bereiteten Cristals 1. Unzen Minii 6 $\displaystyle\frac{1}{2}$\rule[-4mm]{0mm}{10mm} Unz. mische es wohl, thue dazu 75 gran gr\"{u}nspann,
\edtext{Croci}{\lemma{}\Afootnote{\textit{Am Rand:} In his omnibus minimum minii seu vitrii quod facit nimis ponderosum sine necessitate.\vspace{-4mm}}}
\mars\textsuperscript{tis} 10 gran: procedire wie oben, so wirstu haben einen schohnen hochfarbig Smaragd\protect\index{Sachverzeichnis}{Smaragd} zu kleinen steinen geschliffen, werden sich auffen folio unglaublich sch\"{o}hn praesentirn.
\pend%
\pstart%
Aliter \textrecipe.\ deines Cristals 2 unzen. Minii 7 unzen und zu ieden unzen sez 10 gran gr\"{u}nspan mische es wohl durcheinander, dann thue noch dabey 10 gran croci tis, schmelze es zusammen wie oben.
\pend%
\pstart%
\textso{Adhuc aliter.}
Nim des Cristallpulvers 2 Unzen. Minii 6 Unzen. Seze ieder Unzen 10 gran gr\"{u}nspan zu, ohne den crocum Martis, schmelze es.
\pend%
\pstart%
\textso{Topasius}\protect\index{Sachverzeichnis}{topasius.}
Cristall 2 Unzen Minii 7 unzen schmelz es.
\pend%
\pstart%
\textso{Chrysolithus orientalis.}
Cristall 2 unzen, Minii 8 unzen. Croci \mars\textsuperscript{tis} 12 gran, las es zimlich lang im fluss stehen.
\pend%
\pstart%
\textso{Himmelblau.}
Cristall 2 Unzen, \edtext{minii 5 unzen, Zapherae 21 gr.}{\lemma{minii 5 unzen,}\Bfootnote{\textit{(1)}\ smalt 26 gr \textit{(2)}\ Zapherae 21 gr. \textit{L}}}
\pend%
\pstart%
\textso{Violet himmelfarb.}
Cristall 2 unzen, minii 4 $\displaystyle\frac{1}{2}$\rule[-4mm]{0mm}{10mm} unz. Smalt 26 gran.
[3~r\textsuperscript{o}]
\pend%
\pstart%
\textso{Oriental saphir.}
Cristall 2. unz., minii 6 unz. Zapherae 2 scrup. Magnesiae pedemontanae 6 gr.
\pend%
\pstart%
\textso{Ein hochfarbig Saphir.}
Cristall 2 unz. minii 5 unz. Zafferae 42 gr. Magnesiae 8 gr.
\pend%
\pstart%
\textso{Granatus orientalis.}
Cristall 2 unz. minii 6 unz. magnesiae 16 gran. Zafferae 2 gran.
\pend%
\pstart%
\textso{Granatus sehr hochf\"{a}rbig.}
Cristall 2 unz. minii 5 $\displaystyle\frac{1}{2}$%\rule[-4mm]{0mm}{10mm}
 unz., magnesiae 15 gr. Zapherae 4. gr.
\pend%
\newpage% PR: Rein provisorisch !!!
\pstart%
\textso{Aliter:}
\edtext{Cristall 2 unz. minii 5 unz. magnesiae}{%
\lemma{Cristall 2 unz.}\Bfootnote{%
\textit{(1)}\ magnesiae %
\textit{(2)}\ minii 5 unz. magnesiae %
\textit{L}}} 35 gr. Zaph. 4. gr.
\pend%
\pstart%
\textso{Crocum \mars\textsuperscript{tis}}\protect\index{Sachverzeichnis}{crocus martis} zu machen
\edtext{pro gemmis.}{\lemma{pro gemmis}\Bfootnote{\textit{erg. L}}}
Nim feilsp\"{a}n vom besten stahl, mische es wohl in einem verglasierten geschirr mit guthem distillirten essig,\protect\index{Sachverzeichnis}{Essig}
dann seze das geschirr in die sonne, las wieder trocken werden denn zerreibe es wieder und befeuchte es wieder mit essig,\protect\index{Sachverzeichnis}{Essig}
und lass wieder trocken werden dass thue so lange bis die feilspene alle zu einem subtilen pulver sind worden,
ander farben wie ziegel-m\"{a}hl, so hastu crocum \mars\textsuperscript{tis}.\protect\index{Sachverzeichnis}{crocus martis}
\pend%
\count\Bfootins=1500
\count\Cfootins=1500
\count\Afootins=1500