\begin{ledgroupsized}[r]{120mm}
\footnotesize
\pstart
\noindent\textbf{\"{U}berlieferung:}
\pend
\end{ledgroupsized}
%
\begin{ledgroupsized}[r]{114mm}
\footnotesize
\pstart
\parindent -6mm
\makebox[6mm][l]{\textit{L}}Reinschrift mit Verbesserungen und Ergänzungen:
LH XXXV 9, 11 Bl. 1-2.
1 Bog. 2\textsuperscript{o}.
4~S.
Ein jeweils verschiedenes Wasserzeichen auf Bl.~1 und Bl.~2.
Der Text wird editorisch in drei Teile unterteilt,
die als verschiedene Redaktionsstufen gedeutet werden k\"{o}nnten.%
 \\Cc 2, Nr. 1189 A, C
\pend
\end{ledgroupsized}
\vspace*{8mm}
\pstart
\noindent
[1~r\textsuperscript{o}] \hspace{41mm}
DU FROTTEMENT.
\pend
\pstart
\centering Essais Geometriques en fait de Mechanique.
\pend \vspace{1em}
\pstart
\centering
[\textit{Teil 1}]
\pend
\count\Bfootins=1000
\pstart
\noindent
Les Mathematiciens n'ont pas encor donn\'{e} des regles sur cette matiere, et ceux qui ont fait des traitez de
Mechanique, n'en parlent qu'en passant, et pour la renvoyer \`{a} l'experience des ouuriers. Il est constant
toutes fois que souuent des projets bien conceus ont avort\'{e} \`{a} cause de la perte de la force mouuante,
dont une grande partie avoit
\edtext{est\'{e} employ\'{e}e \`{a} surmonter}{\lemma{est\'{e}}\Bfootnote{\textit{(1)}\ absorb\'{e}e par \textit{(2)}\ employ\'{e}e \`{a} surmonter \textit{L}}}
le frottement\protect\index{Sachverzeichnis}{frottement} des pieces de la machine. On s\c{c}ait 
\edtext{que les machines qui servent \`{a} lever de grands fardeaux, les pompes\protect\index{Sachverzeichnis}{pompe},}{\lemma{que}\Bfootnote{\textit{(1)}\ les balances, les pompes, les f \textit{(2)}\ les machines [...] fardeaux,  \textit{(a)}\ les horloges  \textit{(b)}\ les pompes\protect\index{Sachverzeichnis}{pompe}, \textit{L}}}
les chariots et autres voitures,
\edtext{y sont interess\'{e}es}{\lemma{}\Bfootnote{y sont interess\'{e}es: \textit{erg. L}}}:
et on a cherch\'{e} et trouu\'{e}
\edtext{depuis peu}{\lemma{}\Bfootnote{depuis peu \textit{erg. L}}}
quelques inventions propres \`{a} eviter ou diminuer cette perte.
Monsieur \edtext{Perrault\protect\index{Namensregister}{\textso{Perrault} (Perraltus, Perraut), Claude 1613-1688}}{\lemma{Perrault}\Cfootnote{\cite{01014}\textsc{Vitruvius}, \textit{Les dix livres d'architecture}, hrsg. von C. \textsc{Perrault}, Paris 1673, l. X, ch. V, S. 280f. und 324f. Keine der dort beschriebenen Maschinen wird allerdings \textit{barulcus} genannt. F\"{u}r diesen auf Heron von Alexandria zur\"{u}ckgehenden Begriff siehe vielmehr \cite{01015}\textsc{Pappus}, \textit{Mathematica collectio}, l. VIII, probl. VI, prop. X.}}
a publi\'{e} dans son Vitruue\protect\index{Namensregister}{\textso{Vitruvius Pollio}, Marcus ca. 70-10 v. Chr.}
une esp\`{e}ce de Machine \`{a} lever des fardeaux ou\textso{ Barulcum,} o\`{u} il n'y a quasi point de frottement.
\edtext{On a present\'{e}}{\lemma{On a present\'{e}}\Cfootnote{Quelle nicht nachgewiesen.}}
\`{a} l'Academie Royale\protect\index{Sachverzeichnis}{Acad\'{e}mie Royale des Sciences}
une\textso{ pompe }tres ingenieuse,
o\`{u} le principe de Torricelli\protect\index{Namensregister}{\textso{Torricelli} (Torricellius), Evangelista 1608-1647}
est appliqu\'{e} \`{a} la m\^{e}me
\edtext{fin, le Mercure m\^{e}me servant de bouchon.
La pens\'{e}e de
\edtext{celuy}{\lemma{celuy}\Cfootnote{Quelle nicht nachgewiesen. }}
qui a fait faire}{\lemma{fin}\Bfootnote{\textit{(1)}\ . Celuy qui a eu la pens\'{e}e de faire \textit{(2)}\ , le Mercure [...] faire}}
des\textso{ chariots }qui se fournissent eux m\^{e}mes des planches pour marcher l\`{a}
\edtext{dessus doucement,}{\lemma{}\Bfootnote{dessus\ \textbar\ tres \textit{gestr.}\ \textbar\ doucement, \textit{L}}}
n'a \edtext{pas est\'{e} mauvaise.}{\lemma{pas}\Bfootnote{\textit{(1)}\ mal reussi \textit{(2)}\ est\'{e} mauvaise.}}
Et je croy qu'on trouuera avec le temps de semblables remedes pour toute autre
sorte de mouuements. Cependant l'estime de la perte faite par le frottement ne laisse pas
d'estre \edtext{utile}{\lemma{}\Bfootnote{utile \textit{erg. L}}},
et sans parler des vaisseaux, qui marchent dans de l'eau avec quelque difficult\'{e}, il est
constant que les corps jettez\protect\index{Sachverzeichnis}{corps jet\'{e}} sont retard\'{e}s notablement par la resistence de l'air\protect\index{Sachverzeichnis}{resistance de l'air}.
Et comme il y a de l'apparence que les hommes trouueront un jour des regles assez justes pour [la]\edtext{}{\lemma{la}\Bfootnote{\textit{erg. Hrsg.}}}
donner dans un point propos\'{e}; il est ais\'{e} de juger que ce ne sera qu'apr\`{e}s que le frottement sera reduit
en regles\edtext{, quoyque cependant un long usage des personnes qui s'y sont exerc\'{e}es d\`{e}s leurs jeunesse puisse suppleer \`{a} ce defaut}{\lemma{}\Bfootnote{, quoyque [...] defaut \textit{erg. L}}}.
\pend
\count\Bfootins=1500
\vspace*{3mm} 
\begin{Geometrico}
% PR: Erste Zeile bitte hängend (more geometrico).
%\edlabel{LH035,09,11_001r_z1} 
\textso{Le Frottement} \edlabel{LH035,09,11_001r_z1}est la resistence du lieu\protect\index{Sachverzeichnis}{resistance du lieu} par o\`{u} le mobile passe.\\% PR: Diesen Absatz bitte links ganz einrücken. 
J'entends par le\textso{ Lieu }la surface du corps ambient ou environnant,
entierement ou en partie
\edtext{comme \edtext{Aristote\protect\index{Namensregister}{\textso{Aristoteles}, 384-322 v. Chr.}}{\lemma{Aristote}\Cfootnote{\cite{00235}\textit{Phys.} IV 4, 212a2-30.}} l'a defini}{\lemma{}\Bfootnote{comme [...] defini \textit{erg. L}}}.
\end{Geometrico}
\pstart
\noindent% PR: Diesen Absatz bitte gar nicht einrücken.
Cette\textso{ Resistence }se fait par la complication de deux causes, et c'est pourquoy elle est aussi de deux
especes,\textso{ absolue,} et\textso{ respective.}
Je veux traiter icy de la resistence absolue\protect\index{Sachverzeichnis}{resistance absolue}, et je me reserve
de parler de la respective\protect\index{Sachverzeichnis}{resistance respective} dans un
\edtext{autre cahier,}{\lemma{autre cahier}\Cfootnote{Vermutlich N. 35.}}
o\`{u} j'expliqueray la difference qu'il y a entre ces deux Resistences, et leurs origines.\edlabel{LH035,09,11_001r_z2}
\pend
\pstart
\vspace{1em} \begin{center} 
Premiere section\\
De la\textso{ Resistence absolue,} qui se trouue\\dans le frottement et qui est tousjours la m\^{e}me quelque vitesse que le mobile\\puisse avoir
\end{center} \pend
\begin{Geometrico}
% PR: Erste Zeile bitte hängend (more geometrico).
\textso{%
Acceleration, \protect\index{Sachverzeichnis}{acc\'{e}l\'{e}ration}}\edtext{\textso{ou}}{\lemma{}\Bfootnote{\textso{ou} \textit{erg. L}}}\textso{ Retardation\protect\index{Sachverzeichnis}{retardation}}\textso{ \'{e}gale selon les lieux }\edtext{\textso{\lbrack temps\rbrack}}{\lemma{\textso{\lbrack temps\rbrack}\! }\Cfootnote{Die eckigen Klammern stammen von Leibniz.}}\textso{
}est une addition ou soubstraction continuelle
d'un m\^{e}me degrez de vitesse\protect\index{Sachverzeichnis}{degr\'{e} de vitesse}, \`{a} chaque point du lieu
\edtext{\lbrack\`{a} chaque moment du temps\rbrack}{\lemma{\lbrack\`{a} chaque moment du temps\rbrack}\Cfootnote{Die eckigen Klammern stammen von Leibniz.}}.
\end{Geometrico}
%\newpage
\pstart
\noindent% PR: Diesen Absatz bitte gar nicht einrücken.
Celle qui est selon les temps a est\'{e} employ\'{e}e par
\edtext{Galilei\protect\index{Namensregister}{\textso{Galilei} (Galilaeus, Galileus), Galileo 1564-1642}}{\lemma{Galilei}\Cfootnote{\cite{00050}\textit{Discorsi}, Leiden 1638, S. 157f. und 163-165 (\cite{00048}\textit{GO} VIII, S. 197f. und 202-204).}}
\`{a} l'explication de la descente des corps pesans.
Mais celle qui se fait selon les lieux n'a pas encor est\'{e} reduite au calcul \`{a} ce
\edtext{que j'en ay p\^{u} apprendre}{\lemma{que}\Bfootnote{\textit{(1)}\ je s\c{c}ache \textit{(2)}\ j'en [...] apprendre:}}:
Quoyque \edtext{plusieurs}{\lemma{plusieurs}\Cfootnote{Vermutliche Anspielung auf \cite{01022}\textsc{P. Le Cazre}, \textit{Physica demonstratio}, Paris 1645. Leibniz' eigenhändige Randbemerkungen befinden sich in seinem Handexemplar von Le Cazres\protect\index{Namensregister}{\textso{Le Cazre} (Cazreus), Pierre 1589-1664} Abhandlung; siehe N. 13.}}
l'ayent cr\^{u} preferable \`{a} celle de Galilei\protect\index{Namensregister}{\textso{Galilei} (Galilaeus, Galileus), Galileo 1564-1642}, pour expliquer m\^{e}me la dite descente.
Je ne suis pas de leur opinion, et il me suffit de la pouuoir appliquer au frottement.
\pend
\count\Bfootins=1500
\pstart% PR: Normal einrücken, bitte.
Theoreme I.
\pend
\pstart
\noindent% PR: Diesen Absatz bitte gar nicht einrücken.
\sloppy
\textso{Un corps dont le mouuement est uniforme en soy m\^{e}me estant retard\'{e} \'{e}galement \`{a} chaque endroit
du lieu o\`{u} il passe; les vistesses residues sont entre elles, comme les espaces qui restent \`{a} parcourir.}
\pend
\pstart
\noindent% PR: Diesen Absatz bitte gar nicht einrücken.
Dans la\edtext{\textso{ I. fig. }}{\lemma{\textso{I. fig.}\! }\Cfootnote{Siehe [\textit{Fig. 1}].}}soit
un mobile $\displaystyle M$ qui parcoureroit la ligne $\displaystyle EA$ avec la vistesse uniforme\protect\index{Sachverzeichnis}{vitesse uniforme} represent\'{e}e
par $\displaystyle EG$, et par consequent avec un mouuement, qui seroit represent\'{e} tout entier par $\displaystyle EG$ appliqu\'{e}e
\`{a} tous les points $\displaystyle B.$ $\displaystyle (B)$
\edtext{etc.}{\lemma{}\Bfootnote{etc. \textit{erg. L}}}
de la dite ligne $\displaystyle EA$, ou par le Rectangle $\displaystyle GEA$, si chaque point $\displaystyle B.$ $\displaystyle (B)$ etc.
ne diminuoit
\edtext{\'{e}galement la vitesse du mobile}{\lemma{\'{e}galement}\Bfootnote{\textit{(1)}\ sa vitesse \textit{(2)}\ la vitesse du mobile. \textit{L}}}.
Donc les vistesses decroissant \'{e}galement jusqu'au
\edtext{repos dans}{\lemma{repos}\Bfootnote{\textit{(1)}\ en \textit{(2)}\ dans \textit{L}}}
$\displaystyle A.$ celles qui resteront en chaque point $\displaystyle B.$ $\displaystyle (B)$ etc.
seront represent\'{e}es par les appliqu\'{e}es du Triangle,
\edtext{$\displaystyle GEA.$ s\c{c}avoir par $\displaystyle CB$ ou $\displaystyle (C)(B)$ etc.}{\lemma{$\displaystyle GEA.$}\Bfootnote{\textit{(1)}\ $\displaystyle CB.$ $\displaystyle (C)(B)$ represent\'{e}es par \textit{(2)}\ s\c{c}avoir [...] etc.}}
paralleles \`{a} la
\edtext{base $\displaystyle EG$.}{\lemma{base $\displaystyle EG$}\Cfootnote{Bei der gleichf\"{o}rmigen Bewegung von $\displaystyle M$ bezeichnet $\displaystyle GEA$ ein in [\textit{Fig. 1}] nicht gezeichnetes Viereck; bei der gleichf\"{o}rmig verz\"{o}gerten Bewegung von $\displaystyle M$ bezeichnet $\displaystyle GEA$ das gezeichnete gleichnamige Dreieck.}}
Or $\displaystyle CB.$ $\displaystyle (C)(B)$ sont comme $\displaystyle AB.$ $\displaystyle A(B)$ espaces qui restent \`{a} parcourir.
Donc les vistesses residues sont comme les espaces qui restent \`{a} parcourir.
\pend
\pstart% PR: Normal einrücken, bitte.
Theoreme II.
\pend
\pstart
\noindent% PR: Diesen Absatz bitte gar nicht einrücken.
\sloppy
\textso{Les m\^{e}mes conditions estant pos\'{e}es, le temps employ\'{e} croist \`{a} chaque endroit de l'espace
en raison reciproque des espaces qui restent \`{a} parcourir.}
\pend
\pstart
\noindent% PR: Diesen Absatz bitte gar nicht einrücken.
Car generalement les accroissemens du temps en chaque endroit du lieu, sont en raison reciproque des vistesses que
[le]\edtext{}{\lemma{}\Bfootnote{le \textit{erg. Hrsg.}}}
mobile y a,%
\edtext{\textso{ par le Lemme suivant[;]}}{\lemma{}\Bfootnote{\textso{par} [...] \textso{suivant} \textit{erg. L}}}
or icy ces vistesses sont en raison des espaces qui restent \`{a} parcourir,
par le\textso{ th. I.[;]}
\edtext{donc les dits accroissements du temps}{\lemma{donc}\Bfootnote{\textit{(1)}\ les accroissements du temps susdits \textit{(2)}\ les dits [...] temps, \textit{L}}},
seront en raison reciproque des dits espaces.
[1~v\textsuperscript{o}]
\pend