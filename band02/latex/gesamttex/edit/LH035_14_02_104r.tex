\begin{ledgroupsized}[r]{114mm}%
\footnotesize%
\pstart \parindent -6mm%
\makebox[6mm][l]{\textit{L}}%
Auszüge mit Bemerkungen aus \cite{00318}\textsc{P. Boccone}, \textit{Recherces et observations naturelles}, \mbox{Amsterdam} 1674:
LH XXXV 14, 2 Bl. 104, 108. 1 Bog. 2\textsuperscript{o}. Etwa 3\,\nicefrac{1}{2} S.
Textfolge: Bl. 104~r\textsuperscript{o} (mit Textüberhängen
 auf Bl. 104~v\textsuperscript{o}), Bl. 108~v\textsuperscript{o} und Bl. 108~r\textsuperscript{o}.
% Spuren eines Wasserzeichens auf Bl. 108.
Der Bogen umschließt ferner Bl. 105-107, auf denen N.~60 % = LH 35,14,02_105-107 = Notizen zur Botanik
überliefert ist.\\%
Cc 2, Nr. 1366 A %; Bl. 108 kein Eintrag KK 1, Cc 2
\pend%
\end{ledgroupsized}%
% \normalsize
\vspace{4mm}%
\begin{ledgroup}%
\footnotesize%
\pstart%
\noindent%
\footnotesize{\textbf{Datierungsgr\"{u}nde:}
Paolo Boccone und seine \textit{Recherces et observations naturelles} werden auch in dem inhaltlich verwandten Stück N.~60 % = LH 35,14,02_105-107 = Notizen zur Botanik
erwähnt (siehe etwa S. \refpassage{035,14,02_105r_01}{035,14,02_105r_02}). Dies dürfte auf eine gemeinsame Entstehungszeit hinweisen, so dass die Datierung von N.~60
% = LH 35,14,02_105-107 = Notizen zur Botanik
hier übernommen wird.
% Die Spuren des Wasserzeichens könnten diesen Datierungsvorschlag bekräftigen.
% Zwei Teile, Bl. 104 und 108 Excerpte aus Boccone 1674, nicht fortlaufend, Bl. 105-107 Notizen unbekannter Herkunft??. Die beiden Wasserzeichen in den Bl. 105, 106 und 107 k\"{o}nnen durch einen von Leibniz datierten Bogen (LH XXXVII 6 Bl. 14f., 18. III. 1676) auf M\"{a}rz 1676 datiert werden. Da auf diesen Bl\"{a}ttern die Lekt\"{u}re des Boccone noch als desideratum vermerkt wird??, sind die Exzerpte auf Bl. 104 und 108 danach enstanden. Mit dem Erscheinungsdatum 1674 ist ein terminus post quem gegeben. Erw\"{a}hnungen dieses Buches (III, 2; S. 391. 529) in Briefen an Leibniz geben nicht Aufschluss \"{u}ber Leibniz' Kenntnisstand. Auf Grund des fortlaufenden Textes und des gemeinsamen Texttr\"{a}gers kann angenommen werden, dass Bl. 104 und 108 zur gleichen Zeit beschrieben wurden. Das Wasserzeichen in Bl. 108 kann durch ?? auf ?? datiert werden. Daher k\"{o}nnen diese Ausz\"{u}ge durch inhaltliche \"{U}berlegungen auf ??, und durch den Texttr\"{a}ger auf ?? datiert werden.
}%
\pend%
\end{ledgroup}%
\vspace{4mm}%
\pstart
\noindent%
 [104~r\textsuperscript{o}]
 \pend
 \count\Bfootins=1100
\count\Cfootins=1100
\count\Afootins=1200
\pstart%
\noindent
\centering%
\hspace{11,5mm}
\textso{Extraits de }\edtext{\textso{lettres de Mons. Boccone}
\newline%
\textso{imprim\'{e}es en Hollande.}\protect\index{Ortsregister}{Holland}% 
}{\lemma{\textso{lettres} [...] \textso{Hollande}}\Cfootnote{\cite{00318}\textsc{P. Boccone}, \textit{Recherches et observations naturelles}, Amsterdam 1674.}}
\pend%
\vspace{2mm}
\pstart%
\noindent%
\protect\index{Namensregister}{\textso{Boccone}, Paolo 1633-1704}\textso{Boccone \`{a} Mons. Pierre Guisony}\protect\index{Namensregister}{\textso{Guisony}, Pierre 17. Jhd.}\textso{ Medecin \`{a} Avignon.}\protect\index{Ortsregister}{Avignon}
\edtext{\textit{Adressez vos lettres \`{a} Mons. Tardy pour} m'estre rendues \`{a} Lyon.}{\lemma{\textit{Adressez} [...] Lyon}\Cfootnote{\cite{00318}%\textsc{P. Boccone}, \textit{Recherches}, Amsterdam 1674, 
a.a.O., S. 1.\cite{00318}}}
%
\edtext{Le \edtext{vray}{\lemma{vray}\Bfootnote{\textit{erg. L}}}
Corail\protect\index{Sachverzeichnis}{corail} blanc et rouge de Dioscoride
doiuuent \textit{estre mis sous le genre des pierres:}
selon la definition que les plantes croissent per intus susceptionem, les autres choses per accretionem.%
}{\lemma{Le vray [...] accretionem}\Cfootnote{a.a.O., S. 1f.\cite{00318} %
% In den bekannten Ausgaben des Dioscorides gehört die Beschreibung der Korallen zum Abschnitt f\"{u}r mineralische Medikamente, obwohl sie als Gew\"{a}chse bezeichnet werden. Siehe \textsc{Dioscorides}, \textit{Materia medica}, Wiener Dioskurides, f. 391~v\textsuperscript{o}. WELCHE AUSGABE ???
Siehe \cite{01117}\textsc{Dioskurides}, \textit{De materia medica}.}}
\edtext{\textit{Le Corail\protect\index{Sachverzeichnis}{corail} n'a point de}
%}{\lemma{\textit{Le Corail} [...] \textit{de}}\Cfootnote{a.a.O., S. 3.}}
semence, et %\edtext{
\textit{quoyque veuillent dire les Apothicaires de Marseille\protect\index{Ortsregister}{Marseille} de leur fleur de Corail,\protect\index{Sachverzeichnis}{corail}
ce ne sont selon ma pens\'{e}e et mon observation, que les extremitez de cette pierre qui sont arrondies et perc\'{e}es de plusieurs pores estoilez.}%
}{\lemma{\textit{Le Corail} [...] \textit{estoilez}}\Cfootnote{\cite{00318}\textsc{P. Boccone}, \textit{Recherches}, Amsterdam 1674, S. 3.}}
\edtext{Estant present \`{a} la peche du Corail\protect\index{Sachverzeichnis}{corail} je remarquai l\`{a} dedans,
\textit{une humeur que je crois estre son levain,} je croy qu'il croit par la sublimation et application de ce levain.}{\lemma{Estant [...] levain}\Cfootnote{a.a.O., S. 4.\cite{00318}}}
\pend%
\count\Bfootins=1200
\count\Cfootins=1200
\count\Afootins=1200
\pstart%
\edtext{\textso{Boccone }\protect\index{Namensregister}{\textso{Boccone}, Paolo 1633-1704}\textso{\`{a} }\textit{\textso{Mons. Alexandre Marchetti professeur de Mathematiques }}\protect\index{Namensregister}{\textso{Marchetti}, Alessandro 1633-1714}\textit{\`{a} Pise.}\protect\index{Ortsregister}{Pisa}
%}{\lemma{\textit{\textso{Mons}.} [...] \textit{Pise.}}\Cfootnote{a.a.O., S. 6.}}\edtext{
Estant present \`{a} la peche de \edtext{corail}{\lemma{de}\Bfootnote{\textit{(1)}\ Cristal \textit{(2)}\ corail, \textit{L}}}, \textit{dans le phare de Messine}\protect\index{Ortsregister}{Messina}
%}{\lemma{\textit{dans} [...] \textit{Messine}}\Cfootnote{a.a.O., S. 6.}}
j'ay remarqu\'{e} que le corail sort \textit{de la mer dur excepte aux extremitez arrondies parce qu'elles sont gonfl\'{e}es, tendres, et rendent une petite quantit\'{e} d'humeur lact\'{e}e\protect\index{Sachverzeichnis}{humeur lact\'{e}e}.}
%\edtext{}{\lemma{\textit{de la} [...] \textit{lact\'{e}e.}}\Cfootnote{a.a.O., S. 6f.}}\edtext{
Ces bouts ne se trouuent pas dans les auteurs ny peints, ny d\'{e}crits,
\textit{parce qu'ils n'ont pas song\'{e} de les tirer de l'eau avec diligence; et je trouue que ce corail\protect\index{Sachverzeichnis}{corail} embarrass\'{e} dans les fil\'{e}s perd ais\'{e}ment ses bouts tendres.}%
}{\lemma{\textso{Boccone} [...] \textit{tendres}}\Cfootnote{a.a.O., S. 6f.\cite{00318} Zitat mit Auslassungen.}}
%
Mons. \edtext{Swammerdam\protect\index{Namensregister}{\textso{Swammerdam}, Jan 1637-1680} pourtant \'{e}crit \`{a} Mons. Boccone\protect\index{Namensregister}{\textso{Boccone}, Paolo 1633-1704}, pag. 161. que Gassendi\protect\index{Namensregister}{\textso{Gassendi} (Gassendus), Pierre 1592-1655}
 en parle in vita Peireskii.\protect\index{Namensregister}{\textso{Fabri de Peiresc}, Nicolas-Claude 1580-1637}
}{\lemma{Swammerdam [...] Peireskii}\Cfootnote{a.a.O., S. 161.\cite{00318} Dort Hinweis auf \cite{00051}\textsc{P. Gassendi}, \textit{Nicolai Claudii Fabricii de Peiresc vita}, Den Haag 1655 (\cite{01029}\textit{GOO} V, S.~237-362).}}
\pend%
\count\Bfootins=1200
\count\Cfootins=1200
\pstart%
\edtext{\textso{Boccone }\protect\index{Namensregister}{\textso{Boccone}, Paolo 1633-1704}\textso{\`{a} M. Denis sur le Corail.}\protect\index{Sachverzeichnis}{corail}}{\lemma{\textso{Boccone} [...] \textso{Corail}}\Cfootnote{\cite{00318}\textsc{P. Boccone}, \textit{Recherches}, Amsterdam 1674, S. 13-17.}}
\pend%
\pstart%
\edtext{\lemma{\textit{de vegetatione} [...] \textit{silentium apud}}\Cfootnote{a.a.O., S. 18.}}\edtext{
Cartesium.\protect\index{Namensregister}{\textso{Descartes} (Cartesius, des Cartes), Ren\'{e} 1596-1650}
Credo tamen si quis \textit{systema vegetationis secundum genuina $\phi$ysices principia} prosequeretur,
\textit{non minus illud methodo geometrica} posset \textit{demonstrare}, ac \textit{corporis animalis oeconomiam.}
%}{\lemma{\textit{systema} [...] \textit{oeconomiam}}\Cfootnote{a.a.O., S. 18.}}\edtext{
Assentitur Bocconio\protect\index{Namensregister}{\textso{Boccone}, Paolo 1633-1704} Corallum non esse plantam, non magis quam arborem philosophicam Chymicorum\protect\index{Sachverzeichnis}{arbor philosophica Chymicorum} ex \protect$\mercury$ et \protect$\rightmoon$ cupelleti in \protect\includegraphics[width=0.02\textwidth]{images/salpeter2.pdf} \textit{dissolutorum et aquae communi deinde innatantium subsidentia} et \textit{nexu: idem} accipit \textit{in cryptis subterraneis, ubi lento stillicidio} formantur \textit{arbores minerales}
%}{\lemma{\textit{dissolutorum} [...] \textit{minerales}}\Cfootnote{a.a.O., S. 19.}}\edtext{ 
et variae figurae.
Adde egregium \textit{veritatis criterium, nimirum corallii\protect\index{Sachverzeichnis}{corallium} salem per deliquium in cella vinaria solutum, in experimentum jucundissimum asservo in}
\edtext{\textit{Musaeo}, nam \textit{ubi corallio}}{\lemma{\textit{Musaeo},}\Bfootnote{\textit{(1)}\ is enim \textit{(2)}\ nam \textit{ubi} \textit{(a)}\ hae in i \textit{(b)}\ \textit{corallio} \textit{L}}}
\textit{sale praegnans liquor, prae calore tempestatis} paulatim \textit{evaporat, concrescit illico reliquum} et infinitas \textit{imitatur perticas; verius sylvam dixissem.}%
}{\lemma{Petri [...] \textit{dixissem}}\Cfootnote{\cite{00318}a.a.O., S. 18f. Zitat mit Auslassungen.}}
\pend%
\count\Afootins=1500
\count\Bfootins=1500
\count\Cfootins=1500
\pstart%
Mons. Swammerdam\protect\index{Namensregister}{\textso{Swammerdam}, Jan 1637-1680} \`{a} Mons. Boccone,\protect\index{Namensregister}{\textso{Boccone}, Paolo 1633-1704}
\edtext{\textit{pour separer et pour faire paroistre les Boules angulaires qui composent la cro\^{u}te du vray corail, il faut mettre des morceaux de corail\protect\index{Sachverzeichnis}{corail} avec leur crouste, dans la lessive vulgaire, ou dans quelque eau douce, mesl\'{e}e d'un peu de savon ordinaire d'Hollande\protect\index{Ortsregister}{Holland}, et le faire chauffer jusqu'\`{a} ce que la crouste se puisse detacher de la surface du corail.\protect\index{Sachverzeichnis}{corail}
Cela fait il faut arracher la crouste avec une petite vergette dans de l'eau de pluye, qui soit chaude, afin d'oster les parties de sel, que la lessive peut avoir laiss\'{e}. Lors qu'on est asseur\'{e} que le Tartre Coralin\protect\index{Sachverzeichnis}{Tartre Coralin}, ou amas de boules est tendre, ce qu'on peut eprouuer avec le doigt, il le faut mettre dans une goutte d'eau claire, le frotter, et le diviser avec un petit pinceau, afin d'avoir les petits corps plus divisez. Apres donc avoir bien lav\'{e} ces petites boules, prenez les, et les versez sur un morceau de verre clair, mince, et propre en cela, et apres avoir coul\'{e} vostre eau l\`{a} dessus, et l'avoir sech\'{e} sur le verre, vous pourrez observer avec le microscope\protect\index{Sachverzeichnis}{Mikroskop}, (tenant le verre oppos\'{e} au jour) les moindres parties de la cro\`{u}te divis\'{e}e, qui sont demeur\'{e}es attach\'{e}es sur la superficie du verre. Cette methode dont je me suis tousjours servi}, sert aussi \`{a} examiner beaucoup d'autres choses.}{\lemma{\textit{pour separer} [...] d'autres choses.}\Cfootnote{\cite{00318}a.a.O., S.~161f.}}
\edtext{Ainsi j'ay trouu\'{e} que chaque partie de la crouste \textit{est compos\'{e}e environ de dix boulles angulaires et cristallines, parfois l'on en trouue moins, et parfois d'avantage.}
\edtext{}{{\xxref{35.14.02_104v_A-1}{35.14.02_104v_A-2}}{\lemma{\textit{La couleur} [...] exemplaire}\Cfootnote{Textüberhang auf Bl.~104~v\textsuperscript{o}.}}}%
\edlabel{35.14.02_104v_A-1}%
\textit{La couleur de ces boules est approchante du rubis blanchastre.
Leur figure est tous\-jours en angle, quoyqu'elle me semble tantost ronde, tantost moins ronde, et m\^{e}me angulaire,
selon la reflexion de la lumiere qui passe par ses angles.
Neantmoins il me semble que je puis tousjours conter cinq angles.
Or dans} une \textit{petite partie qui estoit la huitieme partie d'un grain de Centaurium minus\protect\index{Sachverzeichnis}{Centaurium minus} ou de la plante dite Exacon\protect\index{Sachverzeichnis}{exacon},
sont renferm\'{e}es comme je} viens de dire tantost plus tantost moins de boules, quelques fois en quarr\'{e}, cylindre, et le plus souuent en croix,
\textit{quelques fois en croix de Lorraine.}\protect\index{Ortsregister}{Lothringen}%
}{\lemma{Ainsi [...] \textit{Lorraine}}\Cfootnote{\cite{00318}a.a.O., S.~160.}} 
%%%%Bei folgender Fußnote ist das edtext künstlich verkürzt, da es sonst Probleme mit dem Seitensatz gibt. ACHTUNG das Ende der Fußnote ist händisch ergänzt
\edtext{\textit{Le vray Corail\protect\index{Sachverzeichnis}{corail}}}{\lemma{15-S. 590.1 \hspace{1.8mm}\textit{Le vray} [...] \textit{serr\'{e}e}}\killnumber\Cfootnote{\cite{00318}a.a.O., S.~162f.}} \textit{sans crouste estant tremp\'{e} dans de l'eau forte, se consume peu \`{a} peu, car l'eau forte ronge egalement les rides, qui sont dans la surface, sans pourtant que les boules angulaires se perdent totalement, ny qu'elles changent aucunement de couleur.
Le m\^{e}me corail bouilli dans de la lessive ne souffre aucune alteration remarquable.
Mais si vous le portez sur un charbon ardent, vous le rendrez incontinent blanc. Et s'il n'a point touch\'{e} l'eau forte, ny de la lessive, il deviendra jaunastre sur un charbon ardent\protect\index{Sachverzeichnis}{charbon ardent}.
\textso{Le corail rouge }\protect\index{Sachverzeichnis}{corail}et haut en couleur, mis en poudre fort grossiere, et mesl\'{e} \textso{avec de la cire vierge fondue} jusqu'\`{a} la hauteur d'un pouce, devient dans deux heures, si%
\edtext{}{\lemma{}\Afootnote{\textit{Am Rand}: NB \vspace{0mm}}}
on continue la digestion\protect\index{Sachverzeichnis}{digestion} premierement jaunastre, et apres tout blanc, la cire demeurant quelque peu teinte d'une couleur rougeastre, ce qui arrive aussi quand on fond la cire toute seule, car cela ne vient seulement que de la digestion. C'est pourquoy si on observe \`{a} ce temps le corail\protect\index{Sachverzeichnis}{corail} avec un microscope\protect\index{Sachverzeichnis}{Mikroskop}, on ne} \edtext{[\textit{voit}]}{\lemma{vaut}\Bfootnote{\textit{L \"{a}ndert Hrsg.}}} \textit{aucune alteration, que le simple changement de couleur, ce qui fait qu'il est impossible d'en tirer ainsi aucune \textso{teinture}.
La crouste de Corail\protect\index{Sachverzeichnis}{corail} ne se}
\pend
\count\Afootins=1200
\count\Bfootins=1000
\count\Cfootins=1200
\newpage
\pstart \noindent \textit{dissout aucunement dans cette digestion,\protect\index{Sachverzeichnis}{digestion}
mais au contraire elle devient plus serr\'{e}e.}
%%%%%%%%bis hierher geht eigentlich das edtext Le Vray Corall....%
\edtext{\textit{Pour observer exactement tout, il faut, comme j'ay dit, avoir des morceaux de verre ou de glace,
unis, plats, et fort delicats, et sans aucuns grains,
et l\`{a} dessus appliquer seulement avec de l'eau claire les corps ou boules corallines, qu'on souhaite d'examiner.
Ayant tout dispos\'{e} de la sorte, il faut employer le microscope,\protect\index{Sachverzeichnis}{Mikroskop}
pour voir clairement \`{a} travers de la lumi\`{e}re, la figure et l'arrangement des petits corps cristallins, qu'on a attachez sur le verre}%
}{\lemma{\textit{Pour observer} [...] \textit{le verre}}\Cfootnote{\cite{00318}a.a.O., S.~169f.}}%
\edtext{.\edtext{ Touchant \textit{le}}{\lemma{\textit{verre.}}\Bfootnote{\textit{(1)} \textit{Le}\ \textit{(2)} Touchant \textit{le}\ \textit{L}}}
\textit{lait dont sont remplis les bouts} de \textit{corail,
j'ose presque dire qu'en tombant dans l'eau de la mer il}
\edtext{[\textit{fait}]}{\lemma{font}\Bfootnote{\textit{L ändert Hrsg. nach Vorlage}}}
\textit{peut estre precipiter les parties salines, des quelles après se produit la crouste des boules cristallines ou angulaires
qui sont la premiere application du corail;
ce qui peut estre \'{e}claircy entre autres choses par l'argent de coupelle\protect\index{Sachverzeichnis}{argent de coupelle}
dissout avec de l'eau forte\protect\index{Sachverzeichnis}{eau forte}}[,]
\textit{la quelle estant precipit\'{e}e par le cuivre,
laisse tomber une infinit\'{e} de petits batons, qui estant rangez ensemble produisent en peu de temps}[,]
\textit{des ramifications admirables, en forme d'un petit arbre d'argent\protect\index{Sachverzeichnis}{argent}
couch\'{e} sur un morceau de verre plat
\edtext{o\`{u} a est\'{e}}{\lemma{}\Bfootnote{\textit{o\`{u}\ \textbar\ il gestr.\ \textbar\ a esté\ L}}}
vers\'{e} l'argent\protect\index{Sachverzeichnis}{argent} dissout.}
Je croy que les petites boules cristallines se trouueront aussi dans ce lait,
ainsi nous aurions trouu\'{e} \textit{la vraye semence ou le vray commencement du corail.}\protect\index{Sachverzeichnis}{corail}%
}{\lemma{Touchant [...] \textit{du corail}}\Cfootnote{\cite{00318}a.a.O., S.~170f.}}
Mons. Swammerdam\protect\index{Namensregister}{\textso{Swammerdam}, Jan 1637-1680} avoit dit ce que je viens de transcrire dans \edtext{sa premi\`{e}re lettre.}{\lemma{premi\`{e}re lettre}\Cfootnote{\cite{00318}a.a.O., S.~154-172 (Swammerdams erster Brief an Boccone über den Ursprung und die Anatomie des Koralles).}}
%%
\edtext{Cependant Mons. Oldenbourg\protect\index{Namensregister}{\textso{Oldenburg} (Grubendol), Heinrich 1618-1677} ayant renvoy\'{e} \`{a} Mons. Boccone\protect\index{Namensregister}{\textso{Boccone}, Paolo 1633-1704} les trois pointes ou bouts de Corail\protect\index{Sachverzeichnis}{corail} qu'il y avoit laissez, Mons. Swammerdam\protect\index{Namensregister}{\textso{Swammerdam}, Jan 1637-1680} les examina avec le Microscope\protect\index{Sachverzeichnis}{Mikroskop}.}{\lemma{Cependant [...] Microscope}\Cfootnote{\cite{00318}a.a.O., S.~173f. (aus Swammerdams zweitem Brief an Boccone über den Ursprung und die Anatomie des Koralles, a.a.O., S.~173-180.)}}
%%
\edtext{\textit{Quand on coupe} quelques grands morceaux de la crouste \textit{on}
\edtext{[\textit{trouve}]}{\lemma{\textit{trouve}}\Bfootnote{\textit{erg. Hrsg. nach Vorlage}}}
\edtext{\textit{tousjours} des \textit{cellules. Les}}{\lemma{\textit{tousjours}}\Bfootnote{\textit{(1)}\ \textit{les cellules plus grandes que celles qui sont ferm\'{e}es} \textit{(2)}\ des \textit{cellules. Les}\ \textit{L}}}
\textit{grandes cellules sont remplies de membranes jaunes ou d'une mati\`{e}re jaunastre fort tendre et qui se divisoit comme un jaune d'oeuf quand il est bouilli.}%
}{\lemma{\textit{Quand} [...] \textit{bouilli}}\Cfootnote{\cite{00318}a.a.O., S.~175. Zitat mit Auslassungen.}}
%%
\edtext{\textit{Ces membranes jaunes ne sont autre chose que} le \textit{lait\protect\index{Sachverzeichnis}{lait}
ou Levain\protect\index{Sachverzeichnis}{levain} coagul\'{e} dans les cellules du bout du corail.\protect\index{Sachverzeichnis}{corail}
J'ay gout\'{e} ce lait, il est un peu piquant, et tire sur le vinaigre};
enfin ayant examin\'{e} ces membranes \textit{avec un bon microscope\protect\index{Sachverzeichnis}{Mikroskop}
j'ay remarqu\'{e} clairement une grande quantit\'{e} des boulles cristallines} susdites.
Ce qui me confirme dans la pens\'{e}e susdite.}{\lemma{\textit{Ces membranes} [...] susdite}\Cfootnote{\cite{00318}a.a.O., S.~176. Zitat mit Auslassungen.}}
%%
Le reste de la lettre de
\edtext{Mons. Swammerdam}{\lemma{Mons.}\Bfootnote{\textit{(1)}\ Schwammerdam \textit{(2)}\ Swammerdam \textit{L}}}
manquoit dans mon exemplaire.%
\edlabel{35.14.02_104v_A-2}%
\pend%
\count\Bfootins=1200
\pstart%
\edtext{\textso{Boccone }\protect\index{Namensregister}{\textso{Boccone}, Paolo 1633-1704}%\edlabel{Boccone1}
\textso{Epist. pag. 89.}
% }{\lemma{\textso{Epist. pag. 89}}\Cfootnote{a.a.O., S. 89.\cite{00318}}}\edtext{
\textit{ayant laissé sur la table de ma chambre les ra\-cines de \edtext{l'umbilicus Veneris\protect\index{Sachverzeichnis}{umbilicus Veneris}}{\lemma{umbilicus Veneris}\Cfootnote{Bl\"{u}ten\-pflanze aus der Familie der Crassulaceae (Dickblattgew\"{a}chse).}}, et celles de \edtext{Nardus Montana\protect\index{Sachverzeichnis}{Nardus Montana}}{\lemma{Nardus Montana}\Cfootnote{Bezeichnung f\"{u}r einige Gew\"{a}chse aus der Familie der Valerianaceae oder Baldriangew\"{a}chse.}}, au bout de deux mois ces racines germerent, comme si elles avoient est\'{e} plant\'{e}es dans la terre, et chez un Apoticaire de mes amis de la ville de Pise\protect\index{Ortsregister}{Pisa} appell\'{e} Andrea Vestri\protect\index{Namensregister}{\textso{Vestri}, Andrea} j'ay observ\'{e} quelque chose de plus singulier et de plus surprenant, s\c{c}avoir que l'oignon de la squille\protect\index{Sachverzeichnis}{squille} ayant est\'{e} coup\'{e} en divers morceaux enfilez ensemble et suspendus en l'air dans sa boutique, afin qu'ils se sechassent, et qu'on les put garder pour}%
% }{\lemma{\textit{ayant laisse} [...] \textit{garder pour}}\Cfootnote{a.a.O., S. 89f.\cite{00318}}}\edtext{
la composition de l'acetum squilliticum,\protect\index{Sachverzeichnis}{acetum squilliticum}
\textit{quelques uns de ces petits morceaux, au lieu de se secher quelques mois apres pousserent et produisirent vers leurs extremitez quelques petits oignons, avec leurs feuilles semblables avec des petites echalotes.}}%
{\lemma{\textso{Boccone} [...] \textit{echalotes}}\Cfootnote{a.a.O., S. 89f.\cite{00318}}}
%
\edlabel{Boccone0}%
\edtext{}{{\xxref{Boccone0}{Boccone4}}\lemma{\textit{La seule} [...] \textit{feuille}}\Cfootnote{a.a.O.\cite{00318}, S. 90f. Zitat mit Auslassungen.}}% 
\textit{La seule feuille d'opuntia\protect\index{Sachverzeichnis}{opuntia}
estant mise \`{a} moiti\'{e} dans la terre produit des feuilles des fleurs, et du fruit,
et elle sert de semence, de racine, et de partie tubereuse.
Cette plante abonde aussi en humeur glutineuse.
En Italie\protect\index{Ortsregister}{Italien} ce n'est} pas \textit{une chose extraordinaire
de voir l'Alo\`{e}s Africana,\protect\index{Sachverzeichnis}{Aloe africana}
et la meme squille, suspendues au plancher durant l'espace de plusieurs ann\'{e}es continuelles,
sans estre d\'{e}tach\'{e}es germer et jetter des feuilles, et des fleurs:}
ainsi \textit{sans toucher la terre.}
%%
\edlabel{Boccone3}Et%
\edtext{}{{\xxref{Boccone3}{Boccone4}}\lemma{Et  [...] \textit{feuille}}\Cfootnote{Textüber\-hang auf Bl.~104~v\textsuperscript{o}.}}
\edtext{[\textit{en}]}{\lemma{\textit{en}}\Bfootnote{\textit{erg. Hrsg. nach Vorlage}}}
\textit{coupant des oranges\protect\index{Sachverzeichnis}{orange} nouuellement}
\edtext{[\textit{cueillies}]}{\lemma{\textit{cuillies}}\Bfootnote{\textit{L \"{a}ndert Hrsg. nach Vorlage}}}
\textit{par le milieu bien souuent j'y ay rencontr\'{e} que les grains avoient germ\'{e}s,
sans que les oranges\protect\index{Sachverzeichnis}{orange} eussent est\'{e} enterrez,
ny ouuertes, ny pourries, au contraire elles estoient bonnes \`{a} manger.
Pour n'oublier pas une autre experience que j'ay trouu\'{e}e fort agreable,
laquelle est qu'ayant mis pour secher dans un liure, la tige de la plante,
appell\'{e}e par Clusius\protect\index{Namensregister}{\textso{L'Ecluse} (Clusius), Charles de 1526-1609}}
\edtext{[\textit{Hemerocallis}]}{\lemma{\textit{Hemercallis}}\Bfootnote{\textit{L \"{a}ndert Hrsg. nach Vorlage}}}
\textit{Valentina\protect\index{Sachverzeichnis}{Hemerocallis Valentina}, qui avoit au bout une grosse gousse meure,
remplie de grains noirs, la gousse ayant crev\'{e}, les grains qui en sont sortis ont germ\'{e}s entre les deux feuilles du liure.
Il y avoit entre autres 6 grains.
Chacun desquels avoit jett\'{e} une racine blanche comme une petite fibre, qui estoit longue d'une demie once,
et par le haut avoit produit une feuille verte, semblable au Gramen de la grandeur de deux onces et demie,
et la d\'{e}pouille des grains demeuroit tousjours attach\'{e}e entre la racine et la feuille.}%
\edlabel{Boccone4}%
\pend%
\pstart%
\edtext{Remarques de Mons. Boccone\protect\index{Namensregister}{\textso{Boccone}, Paolo 1633-1704} touchant les figures des Plantes
page 92 %}{\lemma{page 92}\Cfootnote{%\cite{00318}\textsc{P. Boccone}, \textit{Recherches}, Amsterdam 1674
%a.a.O., S. 92.\cite{00318}}}\edtext{
de ses \textit{observations}.
Apres avoir trait\'{e} au long de la figure ronde, ou des plantes bulbeuses\protect\index{Sachverzeichnis}{plantes bulbeuses}, il touche les figures
\textit{triangulaire et spirale.
Nous observons, dit il, au milieu des plantes Bulbeuses\protect\index{Sachverzeichnis}{plantes bulbeuses} le moly Pesariense\protect\index{Sachverzeichnis}{moly Pesariense}
decrit par \edtext{Pona\protect\index{Namensregister}{\textso{Pona}, Giovanni 1565-1630}}{\lemma{Pona}\Cfootnote{\cite{00498}\textsc{G. Pona}, \textit{Monte Baldo}, Venedig 1617, S.~22-24.}}
qui produit tousjours sa tige de la figure triangulaire et l'orchis spiralis major et minor d\'{e}crit par Lobelius\protect\index{Namensregister}{\textso{L'Obel} (Lobelius), Matthias de 1538-1618} produit l'extremit\'{e} de la tige de figure spirale.}%
}{\lemma{Remarques [...] \textit{spirale}}\Cfootnote{\cite{00318}a.a.O., S. 92.%\textsc{P. Boccone}, \textit{Recherches}, Amsterdam 1674, S. 92.
}}
%
\edtext{\textit{Il seroit apropos d'examiner les oignons et les petites loges ou cellules par o\`{u} la tige sort, et \`{a} son origine, pour voir quelle impression et quelle figure elles donnent de la tige de l'orchis et du Moly susdits. Entre les plantes non bulbeuses nous voyons quelques especes de Gramen, de \edtext{juncus\protect\index{Sachverzeichnis}{juncus}}{\lemma{juncus}\Cfootnote{Binsengew\"{a}chse oder Juncaceae.}} et de \edtext{cyperus\protect\index{Sachverzeichnis}{cyperus}}{\lemma{cyperus}\Cfootnote{Riedgr\"{a}ser oder Cyperaceae.}} qui ont la tige triangulaire, sans s\c{c}avoir d'ou procede la mani\`{e}re de cette figure. Et de plus parmy les plantes maritimes on trouue un Alga\protect\index{Sachverzeichnis}{alga} ou fucus maritimus\protect\index{Sachverzeichnis}{fucus maritimus} atro-purpureo colore donatus, qui produit la \mbox{tige} et toutes ses branches de figure spirale; et parce qu'elle est reguli\`{e}re dans toutes les branches, on ne peut pas dire, que c'est une monstruosit\'{e}, ny jeu de la nature, comme il arrive souuent, dans quelques especes, de chicoree\protect\index{Sachverzeichnis}{chicor\'{e}e}, de chardon\protect\index{Sachverzeichnis}{chardon}, de Genest\protect\index{Sachverzeichnis}{genest}, et de rubea major\protect\index{Sachverzeichnis}{rubea major}, et cela vraisemblablement par le transport promt et violent des sucs aux parties superieures.}%
}{\lemma{\textit{Il seroit} [...] \textit{superieures}}\Cfootnote{\cite{00318}\textsc{P. Boccone}, \textit{Recherches}, Amsterdam 1674, S. 92f.}}
%
\pend%
%\newpage
\count\Bfootins=1200
\count\Cfootins=1200
\pstart%
\edtext{}{{\xxref{Boccone5}{Boccone6}}\lemma{\textit{Il y a} [...] \textit{d'autres}}\Cfootnote{a.a.O., S.~93.\cite{00318}}}%
\textit{Il\edlabel{Boccone5}
y a plusieurs figures spirales dans la mechanique,
il y en a aussi dans les ventricules du coeur:}
%
\edtext{}{{\xxref{Boccone7}{Boccone8}}\lemma{\textit{il y a} [...] \textit{plantes}}\Cfootnote{Textüberhang auf Bl.~104v\textsuperscript{o}.}}%
\textit{il\edlabel{Boccone7}
y a une figure triangulaire dans un des muscles du bras, appell\'{e} trapeze,
qui prend pour origine de l'occiput de 5 epines inferieures du col, et des 8 ou 9 superieures du dos \`{a} la base de l'omoplate.
Et il y en a une infinit\'{e} d'autres.}\edlabel{Boccone6}
%
\edtext{\textit{Si nous pouuions en examinant leur necessite et leur usage trouuer quelque rapport vraysemblable aux plantes,
et l'appliquer en suite \`{a} leur figure, on}
%%
\edtext{trouueroit}{\lemma{trouueroit}\Cfootnote{In der Vorlage \textit{decouvriroit.}}}
%%
\textit{sans doute quelque chose de fort utile}
%%%
\edtext{\textit{aux} philosophes.}{\lemma{\textit{aux}}\Bfootnote{\textit{(1)}\ \textit{gens de lettres} \textit{(2)}\ philosophes. \textit{L}}}
%%%
\textit{Dans ce dessein il seroit necessaire d'examiner quelle proportion il y a
entre les resistences des cones et des pyramides triangulaires des bases isoperimetres inscrites,
pour s'en servir \`{a} raisonner, ou demonstrer la fermet\'{e} et la necessit\'{e} de la figure reguliere des plantes.}\edlabel{Boccone8}%
}{\lemma{\textit{Si nous} [...] \textit{plantes}}\Cfootnote{\cite{00318}a.a.O., S. 93f.}}
\pend%
\pstart%
Mons. \edtext{Fayon \`{a} Mons. Boccone\protect\index{Namensregister}{\textso{Boccone}, Paolo 1633-1704} pag. 99. %}{\lemma{pag. 99}\Cfootnote{%\textsc{P. Boccone}, \textit{Recherches}, 
% a.a.O., S. 99.\cite{00318}}}\edtext{
\textit{La vertu de germer se trouue dans plusieurs endroits de la plante,} %}{\lemma{\textit{La vertu} [...] \textit{de la plante}}\Cfootnote{a.a.O.%, S. 99.}}\edtext{
\textit{comme vous avez remarqu\'{e} dans les tranch\'{e}es de l'oignon de} \edtext{\textit{squille}\protect\index{Sachverzeichnis}{Scilla maritima}}{\lemma{\textit{squille}}\Cfootnote{Scilla maritima oder Meerzwiebel.}}, \textit{dont chaque extremit\'{e} vous fit} \edtext{[\textit{\'{e}clore}]}{\lemma{}\Bfootnote{\textit{\'{e}clorre} \textit{\ L \"{a}ndert Hrsg. nach Vorlage}}} \textit{un jet, ce que j'ay pareillement observ\'{e} dans la racine de la fleur de la passion\protect\index{Sachverzeichnis}{fleur de passion (Passiflora)} qui estant coup\'{e}e par rouelles produit autant de plantes de m\^{e}me esp\`{e}ce que l'on trouue de morceaux, comme fait l'opuntia\protect\index{Sachverzeichnis}{opuntia} par chaque feuille replant\'{e}e, et le sedum arborescens par ses branches ce qui arrive aussi au cresson des prez\protect\index{Sachverzeichnis}{cresson des pr\'{e}s (Cardamine pratensis)} \`{a} simple et double fleur, dont une feuille \`{a} moiti\'{e} enterr\'{e}e fournit une plante entiere; et \`{a} la petite bistorte\protect\index{Sachverzeichnis}{bistorta} des Alpes\protect\index{Ortsregister}{Alpen}, qui se multiplie aussi aisement par les boutons de ses fleurs que par sa graine.}}{\lemma{Fayon [...] \textit{graine}}\Cfootnote{\cite{00318}a.a.O., S. 99f. Zitat mit Auslassung.}}
\pend%
\pstart%
\textso{Mons. }\edtext{\textso{Fagon \`{a} Mons. Boccone }\protect\index{Namensregister}{\textso{Boccone}, Paolo 1633-1704}%
\textso{pag. 102. }% }{\lemma{\textso{pag. 102}}\Cfootnote{%\textsc{P. Boccone}, \textit{Recherches}, 
%a.a.O., S. 102.\cite{00318}}}\edtext{
\textit{Les vers et autres bestioles rongeans les \'{e}corces} des plantes \textit{et ouurant les conduits par lesquels se porte la nourriture, donnent lieu} \edtext{\textit{\`{a} la seue}
(+~succus plantae credo~+)
\textit{de s'echapper}}{\lemma{\textit{\`{a} la}}\Bfootnote{\textit{(1)}\ plante \textit{de s'\'{e}chapper} \textit{(2)}\ \textit{seue} [...] \textit{s'echapper} \textit{L}}} \textit{et de former en se coagulant ces sortes de boules se reduisant en rond plus tost qu'en une autre figure, \`{a} cause d'une \'{e}gale compression du corps qui l'environne, et le retient} \edtext{\textit{autant d'un}}{\lemma{\textit{autant}}\Bfootnote{\textit{(1)}\ de \textit{(2)}\ \textit{d'un} \textit{L}}} \textit{cost\'{e} que de l'autre. Ce qui arrive pareillement aux branches des rouures\protect\index{Sachverzeichnis}{rouure (Quercus petraea)} et du Kermes\protect\index{Sachverzeichnis}{kerm\`{e}s}, sur lesquels certains vermisseaux qui piquent leur \'{e}corce, font naistre les noix de galle\protect\index{Sachverzeichnis}{noix de galle} et les} \edtext{\textit{grains d'\'{e}carlate}\protect\index{Sachverzeichnis}{kerm\`{e}s}}{\lemma{\textit{grains d'\'{e}carlate}}\Cfootnote{Pflanzen, auf denen Kermes-Schildl\"{a}use leben.}}, \textit{et de semblables animaux effleurant la membrane des feuilles de chesne\protect\index{Sachverzeichnis}{ch\^{e}ne} et du lierre terrestre\protect\index{Sachverzeichnis}{lierre terrestre (Glechoma hederacea)} font paroistre} \textso{des \textit{fausses noix de galle.\protect\index{Sachverzeichnis}{noix de galle}}}%
}{\lemma{\textso{Fagon} [...] \textit{\textso{galle}}}\Cfootnote{\cite{00318}a.a.O., S. 102.}}
\pend%
%
\pstart%
\edtext{D'Huisseau\protect\index{Namensregister}{\textso{Huisseau}, Isaac de 1628-1670} \`{a} Mons. Boccone\protect\index{Namensregister}{\textso{Boccone}, Paolo 1633-1704} \edtext{pag. 116.}{\lemma{pag. 116.}\Bfootnote{\textit{erg. L}}} % }{\lemma{pag. 116}\Cfootnote{%\textsc{P. Boccone}, \textit{Recherches}, 
% a.a.O., S. 110.\cite{00318}}}\edtext{
\textit{J'ay souuent fait l'experience de prendre une \'{e}ponge des plus fines, et apres l'avoir fait extraordinairement dessescher je la laissois des jours entiers cach\'{e}e de dans l'eau sans qu'elle s'en remplit, jusqu'\`{a} ce que par divers mouuemens et compressions dedans mes mains je reveillois sa facult\'{e} naturelle, et l'humectant peu \`{a} peu je faisois rouurir tous ses pores pour donner un libre passage \`{a} l'eau.}%
}{\lemma{D'Huisseau [...] \textit{l'eau.}}\Cfootnote{\cite{00318}a.a.O., S. 110f.}}
\pend%
%
\count\Bfootins=1200
\count\Cfootins=1200
\pstart%
Boccone\protect\index{Namensregister}{\textso{Boccone}, Paolo 1633-1704} dans \edtext{la}{\lemma{dans}\Bfootnote{\textit{(1)}\ une \textit{(2)}\ la \textit{L}}} \edtext{13\textsuperscript{me} lettre,}{\lemma{13\textsuperscript{me} lettre}\Cfootnote{%\textsc{P. Boccone}, \textit{Recherches}, 
a.a.O., S. 118-124.\cite{00318}}}
qui est \`{a} Mons. Stenone,\protect\index{Namensregister}{\textso{Stensen} (Steno), Niels 1638-1686}
entreprend d'expliquer mechaniquement les pierres \'{e}toil\'{e}es,
et \edtext{\textit{les productions marines qu'Aldrovandus\protect\index{Namensregister}{\textso{Aldrovandi}, Ulisse 1522-1605} appelle $\psi$eudo-Corallium\protect\index{Sachverzeichnis}{corail} album fungosum, avec celles que Ferrante Imperatus\protect\index{Namensregister}{\textso{Imperato}, Ferrante 1550-1631} nomme millepora\protect\index{Sachverzeichnis}{millepora} et Madrepora;\protect\index{Sachverzeichnis}{madrepora}}
o\`{u} il explique la raison de ces \'{e}toiles; ce ne sont que des tuyaux ou cellules disposez en rond.%
}{\lemma{\textit{les productions} [...]  en rond}\Cfootnote{\cite{00318}a.a.O., S.~119f.}}
%
\edtext{\textit{La Tubularia\protect\index{Sachverzeichnis}{tubularia}, ou Alcyonium Milesium d'Imperatus\protect\index{Namensregister}{\textso{Imperato}, Ferrante 1550-1631} n'est autre chose que des tuyaux d\'{e}licats rouges, affermis par quelque matiere homogene.}%
}{\lemma{\textit{La Tubularia} [...] \textit{homogene}}\Cfootnote{a.a.O., S. 122.\cite{00318}}}
%
\edtext{\textit{J'ay remarqu\'{e} \`{a} l'entr\'{e}e de quelques degrez des maisons d'Amsterdam\protect\index{Ortsregister}{Amsterdam} qu'il y a beaucoup de tuyaux coralloeides\protect\index{Sachverzeichnis}{corail} renfermez dans des pierres bleues dont ils sont bastis et des marques \'{e}toiles:}
%}{\lemma{\textit{J'ay remarqu\'{e}} [...] \textit{\'{e}toiles}}\Cfootnote{\cite{00318}a.a.O., S. 124.}}\edtext{
\textit{on dit qu'}elles \textit{sont port\'{e}es de Bruxelles\protect\index{Ortsregister}{Br\"{u}ssel}, du cost\'{e} de Nivel}
etc. en Flamand Blaeuwesteen.}{\lemma{\textit{J'ay remarqu\'{e}} [...] Blaeuwesteen}\Cfootnote{\cite{00318}a.a.O., S. 124.}}
\pend%
%
\pstart%
Le Fusin\protect\index{Sachverzeichnis}{fusain (Euonymus europaeus)} (pag. 129
\edtext{Lettre de Mons. Moran}{\lemma{Lettre [...] Moran}\Cfootnote{% \textsc{P. Boccone}, \textit{Recherches}, 
a.a.O., S. 125-134.\cite{00318}\hspace{-3mm}}}
\`{a} M. Boccone\protect\index{Namensregister}{\textso{Boccone}, Paolo 1633-1704})
\textit{a\edlabel{Boccone9} seul le privilege entre les arbres et arbrisseaux d'avoir ses rameaux quarrez.
Un certain moly\protect\index{Sachverzeichnis}{moly}, et le souchet ont} \edtext{[\textit{leurs}]}{\lemma{\textit{leur}}\Bfootnote{\textit{\ L \"{a}ndert Hrsg.}}}
\textit{tiges triangulaires.}
\textit{L'ortie,}\protect\index{Sachverzeichnis}{ortie (Urtica)}
\textit{la menthe\protect\index{Sachverzeichnis}{menthe}, le marrube\protect\index{Sachverzeichnis}{marrube (Marrubium)} et plusieurs autres l'ont \edlabel{Boccone10}quarr\'{e}e.}\edtext{}{{\xxref{Boccone9}{Boccone10}}\lemma{\textit{a seul} [...] \textit{quarr\'{e}e}}\Cfootnote{a.a.O., S. 129f.\cite{00318}\hspace{-3mm}}}\edtext{}{\lemma{\textit{ortie}}\Cfootnote{Brennnessel.}}
%
\edtext{\textit{Les angles \`{a} la graine de l'herbe de Staphisagria};\protect\index{Sachverzeichnis}{Staphisagria}
la graine du Myrrhis canel\'{e}e.\protect\index{Sachverzeichnis}{Myrrhis canel}%
}{\lemma{\textit{Les angles} [...] canel\'{e}e}\Cfootnote{a.a.O., S. 130.\cite{00318}}}
%
\edtext{Dans \edtext{\textit{le germe}}{\lemma{Dans}\Bfootnote{\textit{(1)}\ \textit{la ge} \textit{(2)}\ \textit{le germe} \textit{L}}}
\textit{on remarque par le secours du microscope comme un raccourcy de la plante et on y voit son \'{e}bauche et les premiers lin\'{e}amens comme vous} (\'{e}crit Moran \`{a} M. Boccone),\protect\index{Namensregister}{\textso{Boccone}, Paolo 1633-1704} \textit{avez observ\'{e} dans la semence de convolvulus, Highmorus\protect\index{Namensregister}{\textso{Highmore}, Nathanael 1613-1685} a remarqu\'{e} la m\`{e}me chose dans beaucoup de grains comme dans celle du chou, de la moutarde, des febues et surtout dans les semences des deux Erables\protect\index{Sachverzeichnis}{erable} grand et petit; et du fresne\protect\index{Sachverzeichnis}{fr\`{e}ne} o\`{u} il dit que l'on appercoit deux feuilles fort minces pli\'{e}es au tour d'une tige tres deli\'{e}e, comme dans les avellaines\protect\index{Sachverzeichnis}{aveline (Corylus avellana)} et les noix, on en decouure 4 petites entortill\'{e}es, qui enveloppent une tige, ce qui prouue ce que Joseph Scaliger\protect\index{Namensregister}{\textso{Scaliger}, Joseph Juste 1540-1609} a autres fois avanc\'{e} que les plantes engendroient, lorsqu'elles produisoient} \edtext{[\textit{leurs}]}{\lemma{}\Bfootnote{\textit{leur\ L ändert Hrsg.}}} \textit{semences.}%
}{\lemma{Dans [...] \textit{semences}}\Cfootnote{a.a.O., S.~131f.\cite{00318}}}
%
\edtext{\textit{Les estranges figures des Mandragores\protect\index{Sachverzeichnis}{mandragore (Mandragora officinarum)} et des Brionies\protect\index{Sachverzeichnis}{brionie} qui representent des hommes et de cette racce, qu'on a veu depuis peu en Allemagne\protect\index{Ortsregister}{Deutschland}}, donc \textit{la racine ressemble \`{a} une femme}}{\lemma{\textit{Les estranges} [...] \textit{femme}}\Cfootnote{a.a.O., S. 133.}}
(Moran \`{a} Boccone\protect\index{Namensregister}{\textso{Boccone}, Paolo 1633-1704} 1672 pag. 133).
%\edtext{}{\lemma{pag. 133}\Cfootnote{\cite{00318}a.a.O.%\textsc{P. Boccone}, \textit{Recherches}, S. 133.}}
\pend%
%
\pstart%
Mons. \edtext{Steno\protect\index{Namensregister}{\textso{Stensen} (Steno), Niels 1638-1686}
 \`{a} M. Boccone\protect\index{Namensregister}{\textso{Boccone}, Paolo 1633-1704} \textit{passant par Inspruck\protect\index{Ortsregister}{Innsbruck} l'ann\'{e}e 1669, je vis chez Mons. Pandolfini des coquilles mesl\'{e}es avec} \edtext{l'Astroites}{\lemma{\textit{avec}}\Bfootnote{\textit{(1)}\ des Astroites \textit{(2)}\ l'Astroites, \textit{L}}}, \textit{qui avoient est\'{e} trouu\'{e}es aupres de Salzbourg\protect\index{Ortsregister}{Salzburg}, d'o\`{u} j'ay conjectur\'{e} qu'elles sont des restes et des effects du grand deluge.}%
}{\lemma{Steno [...] \textit{deluge}}\Cfootnote{a.a.O., S.~136f.\cite{00318}}}
%
Mons. \edtext{Boccone\protect\index{Namensregister}{\textso{Boccone}, Paolo 1633-1704} r\'{e}pond \`{a} Mons. Steno\protect\index{Namensregister}{\textso{Stensen} (Steno), Niels 1638-1686}, qu'il a \textit{veu entre les mains d'un apoticaire religieux de la chartreuse de Pise\protect\index{Ortsregister}{Pisa} un sel avec des marques disposez en rayons, qui formoient en chaque morceau une \'{e}toile, et il me dit que c'estoit le sel d'\'{e}tain\protect\index{Sachverzeichnis}{sel d'\'{e}tain}.%
}}{\lemma{Boccone [...] \textit{d'\'{e}tain}}\Cfootnote{a.a.O., S. 139.\cite{00318}}}
\pend
\pstart%
\edtext{La seche.\protect\index{Sachverzeichnis}{seiche} Sepia.\protect\index{Sachverzeichnis}{Sepia}}{\lemma{Sepia}\Cfootnote{%\textsc{P. Boccone}, \textit{Recherches}, 
a.a.O., S. 145.\cite{00318}\hspace{20mm}}}
[108~v\textsuperscript{o}]
\pend
\count\Bfootins=1500
\count\Cfootins=1500
\count\Afootins=1500
% \pstart