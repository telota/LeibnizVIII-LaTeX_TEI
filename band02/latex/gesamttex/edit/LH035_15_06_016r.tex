[16~r\textsuperscript{o}] Pars IV\textsuperscript{ta} excerptorum ex Hookio\protect\index{Namensregister}{\textso{Hooke}, Robert (1635-1703)} contra Hevelium\protect\index{Namensregister}{\textso{Hevelius}, Johannes (1611-1687)}
\pend 
\count\Afootins=1200
\count\Bfootins=1000
\count\Cfootins=1000 
\pstart Quaeret tandem aliquis cui bono omnis ista subtilitas. Respondeo quanquam in plurimis communibus casibus nullius sit usus; est tamen valoris infiniti in genere pro provehenda, Geographia, Astronomia, Navigatione, philosophia, physica etc. et in specie ut quaedam allegem.
\pend 
\pstart \textso{Primo} ope hujus Instrumenti exacte potest \textso{refractio} aeris sumi ab horizonte ad Zenith usque, quo facto non tantum rectificantur omnes observationes, quot in quibus\-dam observationibus, inprimis parallaxium, absolute necessarium est, sed et dabit nobis \textso{forte} nova media, judicandi de natura et qualitatibus aeris, pro variis anni temporibus, et temperatura etiam futura. Certum enim est \textso{non} minus in refractionis gradibus, quam caloris ac frigoris gravitatis et levitatis rari ac densi aerem variare. Adeo ut saepe forte caeteris apparentibus iisdem, apparere possit mutata refrangibilitas. Quod fit forte a mutationibus superiorum ejus regionum quae opus habent aliquot dierum descensu ac fermentatione donec ad ejus \edtext{fundum terrae}{\lemma{fundum}\Bfootnote{\textit{(1)}\ aeris \textit{(2)}\ terrae \textit{L}}} propinquum perveniant. Sed de hoc amplius alibi.
\pend 
\pstart \textso{2}\textsuperscript{\textso{dus}} usus est pro determinandis locis fixarum, earumque longitudinum ac latitudinum, et distantiarum a se invicem, earum inprimis quae intra Zodiacum, unde brevi judicabitur, an haec corpora quae adeo fixa et constantia apparent varient situs inter se; cujus \edtext{credendi}{\lemma{}\Bfootnote{credendi \textit{erg. L}}} fundamenta habeo non nulla.
\pend 
\pstart \textso{Tertius usus} pro determinandis locis planetarum eorumque appulsibus ad fixas, quo facto non tantum Astronomia perficietur, sed et longitudo locorum terrestrium, (res summi usus etiam pro commercio et navigatione) consequetur, quod sine ejusmodi instrumentis frustra expectabitur a coelo.
\pend 
\newpage
\pstart \textso{IV}\textsuperscript{\textso{tus}}\textso{ usus} latitudinum locorum determinatio usque ad secundum, quo posito apparebit an latitudo variet non minus ac magnes, quod non sine probabilitate conjecere quidam,\pend 
%\newpage
\count\Afootins=1200
\count\Bfootins=1200
\count\Cfootins=1200 
\pstart \textso{5}\textsuperscript{\textso{tus}} \edtext{\textso{usus} quas influentias}{\lemma{\textso{usus}}\Bfootnote{\textit{(1)}\ quos influxus, \textit{(2)}\ quas influentias \textit{L}}} in terram habeant appropinquationes aut recessus planetarum quoad ejus motionem periodicam, et vicissim terra in motus planetarum; producendis motibus, qui hactenus Hypotheses et calculos Astronomorum confudere. \textso{VI}\textsuperscript{\textso{tus}}\textso{ usus} pro mensuranda gradus quantitate in terra. Optimum in hoc genere experimentum, quod nunc extet in mundo, \edtext{est quod}{\lemma{est}\Bfootnote{\textit{(1)}\ celuy \textit{(2)}\ that \textit{(3)}\ quod \textit{L}}} a Mr Norwood\protect\index{Namensregister}{\textso{},} factum inter Londinum\protect\index{Ortsregister}{London} et Eboracum (York)\protect\index{Ortsregister}{York} sed cum examinamus quibus ille usus instrumentis, invenimus non fuisse certum ad minutum \edtext{usque primum}{\lemma{usque}\Bfootnote{\textit{(1)}\ secundum \textit{(2)}\ primum \textit{L}}} latitudinis suae, et proinde \edtext{ad duo}{\lemma{ad}\Bfootnote{\textit{(1)}\ duos \textit{(2)}\ duo \textit{L}}} usque milia (Anglica) non fuisse certum magnitudinis gradus; unde nec potuit Mensurae universali servire. Sed latitudinibus usque ad secundum minutum sumtis, error in 150 milibus non \edtext{erit nisi}{\lemma{}\Bfootnote{erit \ \textbar\ non erit \textit{streicht Hrsg.} \ \textbar \ nisi \textit{L }\ }} 30\textsuperscript{ma} pars miliaris (Angli) ac proinde pes, \textit{or yard, or rod, this way stated}\edtext{}{\lemma{\textit{stated}}\Cfootnote{a.a.O., S. 77.}} non \edtext{potest variare}{\lemma{potest}\Bfootnote{\textit{(1)}\ errare \textit{(2)}\ variare \textit{L}}} \rule[-4mm]{0mm}{10mm}$\displaystyle\frac{1}{6000\text{\textsuperscript{ma}}}$ parte suae magnitudinis quod sufficit pro \textso{mensura communi}, ad quam aliae totius mundi referendae. \textit{This was the occasion of the contriving and making thereof His Sacred Majesty}\protect\index{Namensregister}{\textso{Karl II.}, K\"{o}nig von England (1660-1685)} \textit{having commanded me to see that experiment accurately performed,}\edtext{}{\lemma{\textit{performed},}\Cfootnote{a.a.O., S. 77.}} \textit{hat not my indisposition of health prevented.}\edtext{}{\lemma{\textit{prevented}.}\Cfootnote{a.a.O., S. 77.}} 
\pend 
\pstart \textso{VII}\textsuperscript{\textso{mus}} pro mensurandis exacte in linea recta duorum locorum distantiis. Hoc admirabili exactitudine fiet sumendo angulos hoc instrumento, si longitudinem aliquam exacte mensuratam demus, ita ut vix ullis \edtext{aliis cognitis mediis}{\lemma{aliis}\Bfootnote{\textit{(1)}\ totius mund \textit{(2)}\ cognitis mediis \textit{L}}} possibile sit ad eam exactitudinem perveniri. Hac arte etiam distantia navis in mari inveniri potest exactius quam ulla alia via, ex una aut duabus stationibus, et aliorum philosophicorum tentaminum multitudo non aliter practicabilium tolerabilis ita habebitur executio. 
\pend 
\count\Afootins=1000
\count\Bfootins=1000
\count\Cfootins=1000 
\pstart 