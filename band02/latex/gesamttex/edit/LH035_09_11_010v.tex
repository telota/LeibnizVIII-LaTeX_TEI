\hspace{-1.3mm}[10~v\textsuperscript{o}]
conceuuons \`{a} present que celuy
\edtext{qui rame, fasse un nouuel effort egal au premier}{\lemma{qui rame,}\Bfootnote{\textit{(1)}\ donne \textit{(2)}\ fasse [...] premier \textit{L}}} \`{a} chaque seconde
\edtext{precisement, alors le mouuement du bateau sera}{\lemma{precisement,}\Bfootnote{\textit{(1)}\ le bateau s \textit{(2)}\ alors [...] sera \textit{L}}}
acceler\'{e} uniformement selon les temps; et si au lieu d'une seconde nous conceuuions une partie du temps incomparablement plus petite,
et qui puisse tenir lieu d'un moment physique,
nous pourrions dire que le bateau iroit de
\edtext{m\^{e}me que les corps pesans descendent selon la premiere supposition qui est celle de Galilaei\protect\index{Namensregister}{\textso{Galilei} (Galilaeus, Galileus), Galileo 1564-1642|textit}}{\lemma{m\^{e}me}\Bfootnote{\textit{(1)}\ que Galilaei suppose que \textit{(2)}\ que [...] pesans  \textit{(a)}\ iront  \textit{(b)}\ descendent [...] Galilaei. \textit{L}}}.
Mais si au contraire le rameur prenoit garde non pas aux temps,
mais aux espaces, et s'il pouuoit donner reglement une nouuelle impression toutes les fois,
qu'il verroit le bateau avancer d'un pied outre l'espace qu'il a d\'{e}ja fait;
ou si au lieu d'un pied
\edtext{on prennoit}{\lemma{on}\Bfootnote{\textit{(1)}\ prendroit \textit{(2)}\ prennoit \textit{L}}}
un espace incomparablement plus petit;
la seconde supposition auroit lieu et il arriveroit
(quoyque d'un maniere renvers\'{e}e)
\`{a} l'\'{e}gard de l'acceleration\protect\index{Sachverzeichnis}{acc\'{e}l\'{e}ration} des corps
\edtext{pesans durant la descente}{\lemma{pesans}\Bfootnote{\textit{(1)}\ par la pa \textit{(2)}\ dans \textit{(3)}\ durant la descente, \textit{L}}},
ce que je demonstreray \`{a}
\edtext{l'\'{e}gard de la retardation du mouuement}{\lemma{l'\'{e}gard}\Bfootnote{\textit{(1)}\ des corps \textit{(2)}\ des Mouuemens diminu\'{e}s uniformement \textit{(3)}\ de la [...] mouuement \textit{L}}}
par un frottement\protect\index{Sachverzeichnis}{frottement} \'{e}gal.
\pend
\pstart
Il est vray que je suis persuad\'{e} de la verit\'{e} de la supposition de Galilaei\protect\index{Namensregister}{\textso{Galilei} (Galilaeus, Galileus), Galileo 1564-1642},
et que je croy d'en avoir
\edtext{une espece}{\lemma{une}\Bfootnote{\textit{(1)}\ fa\c{c}on \textit{(2)}\ espece \textit{L}}}
de \edtext{demonstration \`{a} priori}{\lemma{demonstration \`{a} priori}\Cfootnote{Möglicherweise Anspielung auf N.~15.}}
qui m'a determin\'{e} en sa faveur:
mais j'ay cr\^{u} qu'il estoit \`{a} propos icy[,] puisque une
\edtext{autre occasion}{\lemma{autre occasion}\Cfootnote{Nicht nachgewie\-sen.}}
m'a oblig\'{e} de reduire la seconde supposition aux loix de Geometrie,
de faire cette remarque qui peut estre ne paroistra pas inutile \`{a} ceux qui voudront s'\'{e}claircir entierement sur une
\edtext{matiere si}{\lemma{matiere}\Bfootnote{\textit{(1)}\ aussi \textit{(2)}\ si \textit{L}}}
considerable.
D'autant plus que de tous ceux qui ont cr\^{u} donner
\edtext{des suppositions}{\lemma{des}\Bfootnote{\textit{(1)}\ hypotheses \textit{(2)}\ suppositions \textit{L}}}
differentes de celle de Galilaei\protect\index{Namensregister}{\textso{Galilei} (Galilaeus, Galileus), Galileo 1564-1642}
il n'y en a point, qui
\edtext{ait approfondi celle dont je parle,}{\lemma{ait}\Bfootnote{%
\textit{(1)}\ p\^{u} approfondir %
\textit{(2)}\ approfondi %
\textit{(a)}\ celle-cy \textbar\ dont je vais p \textit{erg.} \textbar\ %
\textit{(b)}\ celle [...] parle, \textit{L}}}
quoyque [effectivement]\edtext{}{\Bfootnote{effectiment\textit{\ L \"{a}ndert Hrsg.}}}
il n'y ait \edtext{qu'elle qu'on}{\lemma{qu'elle}\Bfootnote{\textit{(1)}\ qui \textit{(2)}\ qu'on \textit{L}}}
puisse embrasser raisonnablement, en quittant la premiere.
\pend
\pstart
Or comme il faut diriger toutes les recherches \`{a} l'usage de la vie,
je diray en peu de mots ce qui m'a fait penser \`{a}
\edtext{celle-cy. Il y a de l'apparence que le mouuement des corps jettez pourra}{\lemma{celle-cy.}\Bfootnote{\textit{(1)}\ On ne doute pas que le mouuement des corps projettez pui \textit{(2)}\ Il y a [...] pourra \textit{L}}}
estre regl\'{e} entierement
\edtext{avec le temps}{\lemma{}\Bfootnote{avec le temps \textit{erg. L}}}.
Galilaei\protect\index{Namensregister}{\textso{Galilei} (Galilaeus, Galileus), Galileo 1564-1642}
est all\'{e} fort avant, mais le frottement des
\edtext{corps, ou}{\lemma{corps,}\Bfootnote{\textit{(1)}\ et \textit{(2)}\ ou \textit{L}}}
la resistence de l'air\protect\index{Sachverzeichnis}{r\'{e}sistance de l'air} n'y entre pas en ligne de
\edtext{conte. Et generalement les Mathematiciens jusqu'icy ont pris cet accident pour une imperfection de la matiere plustost que pour une qualit\'{e} constante et susceptible de calcul. Je}{\lemma{conte.}\Bfootnote{%
\textit{(1)}\ Et comme personne \`{a} ce que je s\c{c}ache a trait\'{e} cette matiere %
\textit{(2)}\ Or %
\textit{(3)}\ Et generalement [...] ont %
\textit{(a)}\ trait\'{e} %
\textit{(aa)}\ cette ma %
\textit{(bb)}\ ce sujet %
\textit{(b)}\ neglig\'{e} ce sujet comme %
\textit{(c)}\ pris cet accident %
\textit{(aa)}\ , comme %
\textit{(bb)}\ pour une [...] plustost %
\textit{(aaa)}\ qu'une %
\textit{(bbb)}\ que com %
\textit{(ccc)}\ que pour [...] de calcul. %
\textit{(aaaa)}\ Il %
\textit{(bbbb)}\ Je \textit{L}}}
leur avoue qu'il y aura tousjours quelques petites inegalitez qui dependent du hazard.
Mais il ne faut pas laisser pour cela de determiner ce qu'il y a de constant et
\edtext{d'ordinaire, et d'avancer autant qu'on peut.}{\lemma{d'ordinaire, et d'}\Bfootnote{%
\textit{(1)}\ aller aussi avant qu'on peut \textit{(2)}\ avancer autant qu'on peut. \textit{L}}}
\pend
\newpage
\count\Bfootins=800
\pstart
Quand un \edtext{corps marche le long d'un autre}{\lemma{corps}\Bfootnote{\textit{(1)}\ frotte contre un autre \textit{(2)}\ marche le long d'un autre \textit{L}}}
avec quelque difficult\'{e}, on se peut
\edtext{imaginer [une]}{\lemma{}\Bfootnote{imaginer\ \textbar\ une \textit{gestr. L}\ \textbar\ une\ \textit{erg. Hrsg.}}}
quantit\'{e} de pointes ou
\edtext{eminences sur la surface de celuy}{\lemma{eminences}\Bfootnote{\textit{(1)}\ sur celuy \textit{(2)}\ sur [...] celuy \textit{L}}}
qui resiste au mouuement de l'autre,
lesquelles se plient et se remettent,
et on \edtext{peut representer cet effect}{\lemma{peut}\Bfootnote{%
\textit{(1)}\ expliquer \textit{(2)}\ representer %
\textit{(a)}\ cecy %
\textit{(b)}\ cet effect \textit{L}}}
mechaniquement par des chevilles\protect\index{Sachverzeichnis}{cheville} ou dens\protect\index{Sachverzeichnis}{dent}
qui marchent dans des charnieres\protect\index{Sachverzeichnis}{charni\`{e}re},
\edtext{et}{\lemma{}\Bfootnote{et \textit{erg. L}}}
qui se peuuent plier et remettre par le moyen de quelques ressorts\protect\index{Sachverzeichnis}{ressort} ou quelques
[bascules\protect\index{Sachverzeichnis}{bascule}]\edtext{}{\Bfootnote{bassecoules\textit{\ L \"{a}ndert Hrsg.}}}
appliqu\'{e}es.
Cela pos\'{e} il est ais\'{e} de conceuuoir comment le mouuement du corps qui passe est retard\'{e} par la
\edtext{pesanteur\protect\index{Sachverzeichnis}{pesanteur} de ce petit poids ou par le ressort}{\lemma{pesanteur}\Bfootnote{\textit{(1)}\ ou par le ressort de ce petit poids \textit{(2)}\ de ce [...] ressort, \textit{L}}},
qu'il doit \edtext{lever ou bander chemin faisant}{\lemma{lever}\Bfootnote{\textit{(1)}\ chemin faisant \textit{(2)}\ ou [...] faisant. \textit{L}}}.
Et n'ayant \'{e}gard qu'\`{a} cette force\protect\index{Sachverzeichnis}{force} de la pesanteur,
ou plus\edtext{tost}{\lemma{}\Bfootnote{tost \textit{erg. L}}}
au ressort (:~car le bruit que les corps font en frottant les uns contre les autres en rend t\'{e}moignage~:)
c'est comme \edtext{si cet}{\lemma{si}\Bfootnote{\textit{(1)}\ le \textit{(2)}\  cet \textit{L}}}
aether\protect\index{Sachverzeichnis}{\'{e}ther} ou liquide general,
dont le mouuement est cause de la pesanteur ou du ressort,
\edtext{donnoit au corps qui est en mouuement, en sens contraire \`{a} celuy dans lequel il est m\^{u}, autant de chocs egaux entre eux}{\lemma{donnoit}\Bfootnote{\textit{(1)}\ autant de chocs  \textit{(a)}\ uniformes  \textit{(b)}\ \'{e}gaux entre eux \textit{(2)}\ au corps [...] mouuement,  \textit{(a)}\ \`{a} contresens de son mou  \textit{(b)}\ en sens contraire  \textit{(aa)}\ de mou  \textit{(bb)}\ \`{a} celuy [...] entre eux, \textit{L}}},
contraires \`{a} son mouuement, qu'il y a des pointes \`{a} plier.
Parce que je suppose ces pointes et ces ressorts egaux entre eux,
et pli\'{e}s l'une fois autant que l'autre,
c'est \`{a} dire autant qu'il faut pour laisser passer le corps.
Et pour%
% [11~r\textsuperscript{o}]
% \pend