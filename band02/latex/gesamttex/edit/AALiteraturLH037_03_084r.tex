\begin{ledgroupsized}[r]{120mm}%
\footnotesize%
\pstart%
\noindent\textbf{\"{U}berlieferung:}%
\pend%
\end{ledgroupsized}%
%
\begin{ledgroupsized}[r]{114mm}%
\footnotesize%
\pstart%
\parindent -6mm%
\makebox[6mm][l]{\textit{L}}%
Konzept: LH XXXVII 3 Bl. 84-85.
1 Bog. 2\textsuperscript{o}. 3 S. einspaltig.
Bl. 85~v\textsuperscript{o} leer.
Rand beschädigt mit geringem Textverlust auf Bl. 84~v\textsuperscript{o}.
Je ein verschiedenes Wasserzeichen auf jedem Blatt.%
\newline%
Cc 2, Nr. 00%
\pend%
\end{ledgroupsized}%
%
% \normalsize
\vspace*{5mm}%
\begin{ledgroup}%
\footnotesize%
\pstart%
\noindent%
\footnotesize{%
\textbf{Datierungsgr\"{u}nde:}
Die Wasserzeichen im Textträger des vorliegenden Stücks sind für den Zeit\-raum von Frühjahr 1672 (\cite{01071}\textit{LSB} VI, 3 N. 2) bis Herbst 1673 (\cite{00265}\textit{LSB} VIII, 1 N.~21) belegt.
Im Text wird aber ein \textit{Regimen ignis} bzw. ein \textit{Ignis moderator} erwähnt, den Cornelius Drebbel erfunden habe.
Diese Erfindung ist nur handschriftlich durch zwei deutsche Berichte aus dem späten 17. Jahrhundert belegt.
Von ihr könnte Leibniz während seines Aufenthalts in London womöglich durch Nachfahren Drebbels erfahren haben,
weshalb das vorliegende Stück nicht vor Februar 1673 entstanden sein dürfte. % Unter den sonst nur bibliographischen Notizen N. ?? ist die einzige Inhaltsangabe, Libavius gebe die bei Fach aufgef\"{u}hrten Gr\"{u}nde f\"{u}r Unbest\"{a}ndigkeit eines Feuers in einem Ofen wieder, auffallend. Dazu besteht eine deutliche \"{U}bereinstimmung der technischen Ausdr\"{u}cke zwischen dem vorliegenden Text und Libavius' Beschreibung. Eine weitere Besch\"{a}ftigung Leibniz' mit einem Ofen f\"{u}r chemische Experimente ist in der Pariser Zeit nicht nachweisbar. Daher ist der vorliegende Text wahrscheinlich als Reaktion auf die Lekt\"{u}re Libavius' entstanden. Diese Zuordnung wird durch das Wasserzeichen in Bl. 85 (datiert auf Fr\"{u}hjahr bis Herbst 1672, vgl. VI, 3, N. 2) best\"{a}tigt. Der vorliegende Text kann daher auf die gleiche Zeit wie die bibliographischen Notizen datiert werden.
}%
\pend%
\end{ledgroup}%
%
\vspace*{8mm}%
\pstart
\noindent
[84~r\textsuperscript{o}]
\pend
\pstart%
\normalsize%
\noindent%
\centering%
% [84~r\textsuperscript{o}]
Athanor seu Fornax\protect\index{Sachverzeichnis}{fornax} Philosophorum,\\ in praescriptum caloris\protect\index{Sachverzeichnis}{calor} gradum se sua sponte restituens.% PR: Bitte als Überschrift gestalten. Danke.
\pend%
\vspace*{1em}% PR: Rein provisorisch !!!
\count\Bfootins=1200
\count\Cfootins=1200
\count\Afootins=1200
\pstart%
\noindent%
Diu \edtext{multumque Philosophi}{\lemma{}\Bfootnote{multumque\ \textbar\ ab omnia memoria \textit{gestr.}\ \textbar\ Philosophi \textit{ L}}}
Chemici fornacis\protect\index{Sachverzeichnis}{fornax} tale genus quaesivere, quod praescriptas semel ab Artifice Leges servaret, jussumque caloris\protect\index{Sachverzeichnis}{calor} gradum constanter teneret, aut si quo casu metas excessisset, in viam a seipso revocaretur.
\pend%
\pstart%
Hoc \edtext{inter Philosophorum}{\lemma{}\Bfootnote{inter\ \textbar\ caetera \textit{gestr.} \textbar\ Philosophorum \textit{ L}}} Veterum arcana fuisse narrant: sed certior fama est Cornelium\protect\index{Namensregister}{\textso{Drebbel} (Drebelius), Cornelis 1572-1633} 
\edtext{Drebelium Alcmariensem Batavum\protect\index{Namensregister}{\textso{Drebbel} (Drebelius), Cornelis 1572-1633}
inter caetera praeclara inventa hunc quoque Ignis moderatorem}{%
\lemma{Drebelium [...] moderatorem}\Cfootnote{\textsc{C. Drebbel}, \cite{01154}\textit{Beschreibung Seiners Circulir Ofens com}[\textit{m}]\textit{unic}[\textit{ata}] \textit{a D. Reger.} Ms. hrsg. in
\textsc{V. Keller}, \glqq Re-entangling the Thermometer: Cornelis Drebbel's Description of his Self-regulating Oven, the Regiment of Fire, and the Early History of Temperature\grqq, \title{Nuncius} 28 (2013), S.~266-270.}}
assecutum fuisse:
\pend%
\pstart%
\centering%
\edtext{Qui regere et certas sciret dare jussus \edlabel{habenas}habenas.}{\lemma{Qui [...] habenas}\Cfootnote{Nach \cite{00246}\textsc{Vergil}, \textit{Aeneis} I, 62f.% (qui foedere certo / et premere et laxas sciret dare iussus habenas).
}}%
\edtext{}{{\xxref{habenas}{constat}}\lemma{habenas.}\Bfootnote{\textit{(1)}\ Nullum dubium est \textit{(a)}\ maximi usus \textit{ (b) }\ maximae utilitatis \textit{(2)}\ Nec dubium est maximae utilitatis fore, si quando penitus detegeretur, et in usum revocaretur \textit{(3)}\  Utilitates hujus Fornacis\ \textbar\ insignes \textit{erg.}\ \textbar\ maximas fore constat. \textit{L}}}
\pend%
\pstart%
Utilitates hujus Fornacis insignes maximas fore constat.\edlabel{constat}
%
Hactenus \edtext{enim certi caloris gradus nec \textso{definiri} potuere a Chemicis,}{%
\lemma{enim}\Bfootnote{%
\textit{(1)}\ definitum ignis gradum nemo Chemicorum aut sibi potuit praestituere %
\textit{(2)}\ certus %
\textit{(3)}\ certi [...] Chemicis, \textit{L}}}
nec \textso{servari.}
\pend%
\pstart%
Non \textso{definiri,} etsi enim alii quatuor, alii octo gradus numerent, et intermedios alii
\edtext{rursus accuratius}{\lemma{rursus}\Bfootnote{\textit{(1)}\ accurate \textit{(2)}\ accuratius \textit{L}}}
subdistinguant.
Nemo tamen Chemicorum unquam dicere potuit: ego tanto caloris\protect\index{Sachverzeichnis}{calor} gradu, tantum
\edtext{peregi, ex tali tantoque corpore tantum talis spiritus tanto tempore tali modo $\langle$ob$\rangle$tinui.}{\lemma{peregi}\Bfootnote{\textit{(1)}\ . Nunc enim descrescit, nunc crescit calor\protect\index{Sachverzeichnis}{calor}. \textit{(2)}\ , ex \textit{(a)}\ tanto \textit{(b)}\ tali [...] tempore\ \textbar\ tali modo \textit{erg.}\ \textbar\ $\langle$ob$\rangle$tinui. \textit{L}}}
Nam datus semel caloris\protect\index{Sachverzeichnis}{calor} gradus crescit ob ipsam durationem, non
\edtext{sine acceleratione}{\lemma{sine}\Bfootnote{\textit{(1)}\ aliquo accelerationis genere \textit{(2)}\ acceleratione quadam.
\textit{L}}}
quadam. Quare etsi eandem Registri aperturam relinquas, etsi tantundem alimenti subministres igni, calor\protect\index{Sachverzeichnis}{calor} tamen idem non erit, \edtext{sed major}{\lemma{sed major}\Bfootnote{\textit{erg. L}}}
quemadmodum enim in motu gravium impetus prior novo in quolibet momento accedente continue augetur, ita calor\protect\index{Sachverzeichnis}{calor} quoque prior cum nondum penitus evaporaverit, cum novo succedente
\edtext{in unam summam}{\lemma{in}\Bfootnote{\textit{(1)}\ unum conjung \textit{(2)}\ unam summam \textit{L}}}
conjungendus est.
\edtext{Contra, si nihil addas non idem manebit, sed continue decrescet calor\protect\index{Sachverzeichnis}{calor}, ob evaporationem.}{\lemma{}\Bfootnote{Contra, [...]  evaporationem. \textit{erg. L}}}%
\edtext{%%
\newline%%
\indent%%
Haec faciunt}{\lemma{evaporationem.}\Bfootnote{\textit{(1)}\ Hoc facit \textit{(2)}\ Haec faciunt \textit{L}}}
ut hactenus impossibile fuerit, etiam scientissimo candidissimoque Chemico experimentum aliquod suum ita tradere, ut alius eum possit imitari, imo ut, semper ipse
\edtext{se. Quia:}{\lemma{se.}\Bfootnote{\textit{(1)}\ Quare \textit{(2)}\ Quia: \textit{L}}}
licet materiam vasaque accuratissime annotet, restabit tamen aliis omnibus efficacius \textso{Regimen ignis,}
quod nec alios docere, nec sibi ipsi satis retinere potest.
Unde tot \edtext{praeclara experimenta}{\lemma{}\Bfootnote{praeclara\ \textbar\ veraque \textit{gestr.}\ \textbar\ experimenta \textit{L}}}
\edtext{intercidere, et veraces}{\lemma{intercidere,}\Bfootnote{\textit{(1)}\ tot \textit{(2)}\ et \textit{(a)}\ viri \textit{(b)}\ veraces \textit{L}}}
etiam scriptores in imposturae suspicionem adducti sunt, aliis eorum formulas sive processus frustra tentantibus,
donec nonnumquam aliquis sive felicior sive diligentior aliis repertus est, qui eorum famam vindicavit,
quemadmodum Basilii\protect\index{Namensregister}{\textso{Thoelde}, Johann (1565-1614)}
Kerckringius,\protect\index{Namensregister}{\textso{Kerckring}, Theodor (1640-1693)}
Helmontii\protect\index{Namensregister}{\textso{Helmont}, Jan Baptista van (1579-1644)}
Boylius.\protect\index{Namensregister}{\textso{Boyle}, Robert 1672-1691}
\pend%
\pstart%
Sed multo minus \textso{servari} hactenus a quoquam certum quoddam \textso{Ignis Regimen} potuit, quis
\edtext{enim illa}{\lemma{enim}\Bfootnote{\textit{(1)}\ illos \textit{(2)}\ illa \textit{L}}}
calorum\protect\index{Sachverzeichnis}{calor} incrementa
\edtext{aut decrementa,}{\lemma{aut decrementa}\Bfootnote{\textit{erg.} \textit{ L}}}
illas caloris\protect\index{Sachverzeichnis}{calor} primi secundi tertiique agglomerationes aut evaporationes ad calculos revocet?
Neque enim\edtext{, ut dixi,}{\lemma{, ut dixi,}\Bfootnote{\textit{erg. L}}}
sufficit, tantundem materiae aperturaeque igni dare, ita enim infallibiliter non
\edtext{idem calor}{\lemma{idem}\Bfootnote{\textit{(1)}\ ignis \textit{(2)}\ calor \textit{L}}}
servabitur, sed continue
\edlabel{augebitur}augebitur.%
\edtext{}{{\xxref{augebitur}{maximam}}\lemma{augebitur.}\Bfootnote{\textit{(1)}\ Omnia \textit{(2)}\ Maximam \textit{L}}}%
\pend%
\pstart%
Maximam\edlabel{maximam}
autem Artis partem in Regimine Ignis sitam esse, summi artifices ubique ingeminant.
Et constat pro diverso ignis gradu ex eodem corpore innumerabilia rerum genera, volatilitate%
% Hier folgt jetzt Bl. 84v.