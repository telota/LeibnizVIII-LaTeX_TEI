      
               
                \begin{ledgroupsized}[r]{120mm}
                \footnotesize 
                \pstart                
                \noindent\textbf{\"{U}berlieferung:}   
                \pend
                \end{ledgroupsized}
            
              
                             \begin{ledgroupsized}[r]{114mm}
                            \footnotesize 
                            \pstart \parindent -6mm
                            \makebox[6mm][l]{\textit{L}}Aufzeichnung: LH XXXVIII Bl. 170-171.
1 Bog. 4\textsuperscript{o}, beschnitten.
\nicefrac{1}{3} S. auf Bl.~171~r\textsuperscript{o}.
Bl.~171~v\textsuperscript{o} leer.
Bl.~170 überliefert N.~91.%?? = LH038_170 (De horologiis pendulis)
% = LH038_170r (De horologiis pendulis)
 \\Cc 2, Nr. 00 \pend
                            \end{ledgroupsized}
                %\normalsize
                \vspace*{5mm}
                \begin{ledgroup}
                \footnotesize 
                \pstart
            \noindent\footnotesize{\textbf{Datierungsgr\"{u}nde}: Das vorliegende St\"{u}ck N.~90 ist auf demselben Textträger überliefert wie das auf die Monate April bis Mai 1673 datierbare Stück N.~91.
%?? = LH038_170r (De horologiis pendulis)
Beide Aufzeichnungen handeln zudem von der Ganggenauigkeit von Pendeluhren.
Aus diesen beiden Gründen wird die Datierung von N.~91 %
auch für N.~90 %?? = vorliegendes Stück
vorgeschlagen.}
                \pend
                \end{ledgroup}
            
                \vspace*{8mm}
                \pstart 
                \normalsize
            \noindent[171~r\textsuperscript{o}]
In Horologio\protect\index{Sachverzeichnis}{horologium} communi,
pars quae Germanis vocatur inquies\protect\index{Sachverzeichnis}{inquies} impetum rotae moratur,
dum simul
\edtext{contrariis dentibus}{\lemma{contrariis}\Bfootnote{\textit{(1)}\ parte \textit{(2)}\ dentibus \textit{L}}}
illiditur;
sed quia impetus concepti pars magna perditur ea ratione;
cogitavi an non satius sit, rotam vi sua elateriolum\protect\index{Sachverzeichnis}{elaterium} tendere,
atque ita ubi ipsum ad certum perduxit terminum,
fracta vi sua impeditum
\edtext{teneri, maxima vis}{\lemma{teneri,}\Bfootnote{\textit{(1)}\ magna vis \textit{(2)}\ maxima vis \textit{L}}}
parte hoc modo conservata.
Et posset hoc Elastrum esse additum ipsi illi inquieti,
cujus dum extrema in diversa pelluntur,
posset tendi Elastrum in medio, ad certum usque terminum.
Sed pendulum staticum vel Elasticum mox reversum liberabit hoc Elaterium,
et ab eo ictum accipiet.
Quo peracto rota quoque horologii Elateriolum inquietis denuo
\edtext{tendet.
\newline%
\indent%
\hspace{-2mm}Pendulum}{\lemma{tendet.}\Bfootnote{\textit{(1)}\ In \textit{(2)}\ Pendulum \textit{L}}}
staticum \hspace{-0.15mm}affigi \hspace{-0.2mm}solet \hspace{-0.15mm}libramento,
\hspace{-0.15mm}at \hspace{-0.1mm}Elasticum \edtext{tenet \hspace{-0.15mm}arborem}{\lemma{tenet}\Bfootnote{\textit{(1)}\ arb \textit{(2)}\ in medio \textit{(3)}\ arborem \textit{L}}} \hspace{-0.15mm}rotae \hspace{-0.1mm}ser\-ra\-tae.\pend
 


 

