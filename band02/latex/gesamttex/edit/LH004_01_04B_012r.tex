% [12~r\textsuperscript{o}]
\pstart%
Bis repetito experimento inveni in ovo, cui tantum per septem integros dies gallina incubuerat, non rostrum pulli esse formatum, sed in partem capitis posteriorem valde tumidum esse, post octavum autem diem plane rostrum esse formatum et fissum; ita ut immittendo aciculae caput in foramen, sine ulla difficultate usque ad posteriorem capitis partem, ubi tumor fuerat, perveniret illum autem tumorem esse valde imminutum.
\pend%
\pstart%
Notavi etiam nono die nulla adhuc esse intestina, sed ventriculum occupare infimam ventris capacitatem supra hunc esse hepar et cor, nihilque amplius; caput crassius erat reliquo corpore et collum erat longius reliquo corpore, pterygium sive cauda etiam longa erat imo longior quam pedes, musculi in pectore nulli adhuc apparebant sed spina dorsi omnium prima post caput formatur.
\pend%
\pstart%
Quantum notare potui ex dissectione pullorum plus quam triginta omnis aetatis quos ex ovis eduxi, die 2\textsuperscript{da} incipit aliquid apparere, hoc est cor est formatum et sanguinem versus superficiem tam albuminis quam vitelli mittit. 3\textsuperscript{tia} die caput et spina dorsi ad extremitatem pterygii usque formata sunt. 5\textsuperscript{ta} die cor optime videtur pulsare, et infra ipsum apparet ventriculus albus pedes et alae etiam apparent sed pterygium longius est quam pedes. Cerebellum vero valde tumet, nec non partes cerebri anteriores
\edtext{oculi vero}{\lemma{oculi}\Bfootnote{\textit{(1)}\ autem \textit{(2)}\ vero \textit{L}}}
etiam tertia die formati sunt; paulo post septimum diem rostrum formari incipit et cerebellum, itemque et cerebrum et spina dorsi detumescunt. Decimo die apparet etiam hepar, et fel partim hepati adhaerens, partim etiam ventriculo, ex quo illud punctum \makebox[1.0\textwidth][s]{viride, quod pro felle sumendum puto, videtur esse vehiculum quo intestina ex ventriculo}
\pend
\newpage
\pstart\noindent egrediuntur. Cor est tunc insigne, nondum hepar valde magnum, ventriculus juxta caudam. Die 12 etiam lien a sinistra parte supra fel ventriculo et hepati conjunctum notari potest. Die 15. 16. 17. et 19 notavi eadem omnia nec multo plura imo in pullo 19 dierum, qui biduo post debuisset excludi, nondum ullam partem vitelli notabam, sed ejus intestina magnam partem extra ejus ventrem erant ovi vitello adjuncta, adeo ut existimem duobus ultimis diebus totum vitellum una cum residuis intestinis, ingredi ventrem pulli.
\pend%
\pstart%
In ovis in quibus pulli erant 16 vel 19 dierum apparebat placenta quaedam oblonga quae ex materia putaminibus ovorum simili facta videbatur.
\pend%
\pstart%
Umbilicos quidem duos sive vasa ad umbilicum duo insignia notavi unum ex albumine aliud ex vitello, sed non vidi vasa ex albumine aliud accedere quam pellem pulli, nec vitelli vasa aliud quam unum ex intestinis extra pullum existentibus adire.%
\edtext{}{\lemma{}\Afootnote{\textit{Am Rand:} \Denarius}}
% Bl. 12v ist leer.
% [13~r\textsuperscript{o}]
\pend%
\count\Bfootins=1500
\count\Cfootins=1500
\count\Afootins=1500