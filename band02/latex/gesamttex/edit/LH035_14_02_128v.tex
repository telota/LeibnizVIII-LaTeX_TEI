\begin{ledgroupsized}[r]{120mm}
\footnotesize 
\pstart 
\noindent\textbf{\"{U}berlieferung:}
\pend
\end{ledgroupsized}
\begin{ledgroupsized}[r]{114mm}
\footnotesize 
\pstart \parindent -6mm
\makebox[6mm][l]{\textit{L}}Auszüge mit Bemerkungen aus I. G. \textsc{Pardies},\cite{00296} \title{La statique ou la science des forces mouvantes}, Paris 1673: LH XXXV 14, 2 Bl. 127-128. 1 Bog. 2\textsuperscript{o}. 2 S. Textfolge: Bl. 128 v\textsuperscript{o}, Bl. 127 r\textsuperscript{o}. Bl. 128 r\textsuperscript{o} und 129~v\textsuperscript{o} leer. Wasserzeichen. \\Cc 2, Nr. 423 \pend
\end{ledgroupsized}
%\normalsize
\vspace*{5mm}

\begin{ledgroup}
\footnotesize 
\pstart
\noindent\footnotesize{\textbf{Datierungsgr\"{u}nde}: In einem Brief an Oldenburg (\textit{LSB} III, 1 N. 17) erw\"{a}hnt Leibniz am 26. April 1673 drei kleinere Schriften von Pardies, die sich kurz nach dessen Tod noch im Druck bef\"{a}nden; am 24. und 26. Mai 1673 (\textit{LSB} III, 1 N. 20 und N. 22) berichtet er Oldenburg von der posthum erschienenen \textit{Statique}. Eine Entstehung der vorliegenden Exzerpte im Mai 1673 deckt sich zeitlich mit dem Wasserzeichen, das für die Monate März bis Mai 1673 belegt ist.}
\pend
\end{ledgroup}

\vspace*{8mm}
\pstart 
\normalsize
\noindent [128~v\textsuperscript{o}] \textit{Statique ou la science \edtext{des forces mouuantes}{\lemma{science}\Bfootnote%
{\textit{(1)}\ du mouuement du \textit{(2)}\ \textit{des forces mouuantes} \textit{L}}}} par le R. P. Ignace Gaston Pardies\protect\index{Namensregister}{\textso{Pardies}, Ignace Gaston SJ 1636-1673} de la Compagnie de Jesus \`{a} Paris chez Seb. Mabre-Cramoisy\protect\index{Namensregister}{\textso{Mabre-Cramoisy}, S\'{e}bastien, 1669-1678} imprimeur du Roy, rue S. Jacques, aux Cicognes 1673. 12\textsuperscript{o}. \edtext{Ce}{\lemma{12\textsuperscript{o}.}\Bfootnote{\textit{(1)} C'est \textit{(2)} Ce \textit{L}}} trait\'{e} est une suite de son trait\'{e} du mouuement local. Miror eum dissimulare nomen P. Fabri\protect\index{Namensregister}{\textso{Fabri}, Honor\'{e} 1607-1688} quem etiam extranei nominant libenter. Sed scilicet neminem nominat, nisi coactus. De Wallisii\protect\index{Namensregister}{\textso{Wallis} (Wallisius), John 1616-1703} opere fictis in speciem laudibus, ita loquitur, ut appareat ab eo non magni \edtext{fieri.}{\lemma{fieri.}\Cfootnote{\cite{00296}\textsc{I. G. Pardies}, \textit{La Statique}, Paris 1673, Vorwort.}} 
\pend
\pstart Compendium totius operis Mechanici patris Pardies\protect\index{Namensregister}{\textso{Pardies}, Ignace Gaston 1636-1673} 1. de Motu in genere ejusque productione, conservatione, communicatione; de legibus percussionis, de regulis reflexionis. Idque corporibus sine omni motus impedimento \edtext{consideratis.}{\lemma{consideratis.}\Cfootnote{\cite{00296}\textsc{I. G. Pardies}, a.a.O., Vorwort.}} Discursus 2. agit de motu corporum motui resistentium, seu des forces mouuantes. Omnia reducantur ad vectem, aut libram, ostenditur impossibilitas motus perennis pure mechanici, de corporibus suspensis a duobus terminis vel uno tantum affixis; de modo quo se rumpunt, de figura, \edtext{in}{\lemma{}\Bfootnote{in \textit{erg.} \textit{L}}} qua incurvantur; de viribus quibus Turres et pyramides resistunt vento; de loco maximae debilitatis, de figuris quibus aequaliter resisterent. Regulae generales de resistentia corporum, earumque applicatione ad casus particulares, et hoc inprimis exemplo \edtext{navis.}{\lemma{navis.}\Cfootnote{\cite{00296}\textsc{I. G. Pardies}, a.a.O., Vorwort.}} 
\pend 
%\newpage
\count\Cfootins=1200
\count\Bfootins=1200
\pstart Tertius discursus, de motu gravium, de ratione augmentationis, ubi excutitur
\edtext{disputatio inter Galilaeum%
\protect\index{Namensregister}{\textso{Galilei} (Galilaeus, Galileus), Galileo 1564-1642}
et postea Balianum[:]%
\protect\index{Namensregister}{\textso{Baliani}, Giovanni Battista 1582-1666}
ei definitionis suae, in applicatione scilicet ad naturam gravium,
controversiam movit alia progressione motus assignata.%
}{\lemma{disputatio [...] assignata}\Cfootnote{%
\cite{00050}\textsc{G. Galilei}, \textit{Discorsi}, Leiden 1638, S.~172 (\textit{GO}\cite{00048} VIII, S.~210).
\cite{00006}\textsc{G. B. Baliani}, \textit{De motu gravium}, Genua 1646, S.~79.}}
Unde secuta diuturna \edtext{contestatio inter Gassendum%
\protect\index{Namensregister}{\textso{Gassendi} (Gassendus), Pierre 1592-1655}
et le P. le Cazre,\protect\index{Namensregister}{\textso{Le Cazre} (Cazreus), Pierre 1589-1664}%
}{\lemma{contestatio [...] Cazre}\Cfootnote{%
\cite{00297}\textsc{P. Gassendi}, \textit{De proportione}, Paris 1646 (\textit{GOO}\cite{01029} III, S. 564-650).
\cite{01022}\textsc{P. Le Cazre}, \textit{Physica demonstratio}, Paris 1645.}}
donec res terminata videbatur per
\protect\index{Namensregister}{\textso{Huygens} (Hugenius, Ugenius, Hugens, Huguens), Christiaan 1629-1695}%
\edtext{Hugenium,}{\lemma{Hugenium}\Cfootnote{%
\cite{00123}\textsc{C. Huygens}, \textit{Horologium oscillatorium}, Paris 1673, Teil II, bes. S.~24f. (\textit{HO}\cite{00113} XVII, S.~131f.).}}
P. Billium,\protect\index{Namensregister}{\textso{Billy}, Jacques de 1602-1679}
qui demonstrabat progressionem Baliani\protect\index{Namensregister}{\textso{Baliani}, Giovanni Battista 1582-1666} esse impossibilem[,]
et Fermatium,\protect\index{Namensregister}{\textso{Fermat}, Pierre de 1601-1665}
qui ostendit aeternitate minimum opus esse
\edtext{corpori ex}{\lemma{corpori}\Bfootnote{\textit{(1)} per \textit{(2)} ex \textit{L}}}
pedis altitudine hac proportione
\edtext{descensuro. Cum}{\lemma{descensuro.}\Bfootnote{\textit{(1)} At \textit{(2)} Cum \textit{L}}}
P. Lalovera\protect\index{Namensregister}{\textso{Lalouvère} (La Loubère, Lalovera), Antoine de 1600-1664}
notus Geometricis inventis apparuit nova salvandi Baliani\protect\index{Namensregister}{\textso{Baliani}, Giovanni Battista 1582-1666}
adhibita ratione, quae ita elegans apparuit ut nec ipse Fermatius\protect\index{Namensregister}{\textso{Fermat}, Pierre de 1601-1665}
inveniret, quod contradiceret. Sed haec in dissertatione nostra, inquit P. Pardies\protect\index{Namensregister}{\textso{Pardies}, Ignace Gaston 1636-1673} explicabuntur, ubi apparebit, istud primum pondus, seu determinatum celeritatis gradum, cui innititur demonstratio Laloverae\protect\index{Namensregister}{\textso{Lalouv\`ere} Lalouvère (La Loubère, Lalovera), Antoine de 1600-1664}, subsistere non posse. Explicatur et similis progressio quae in motu brachii, aut pedis aut instrumenti quo ferimus, reperitur. Explicatur et aliud progressionis genus quo pilae tormentariae, aut \edtext{sagittae moventur}{\lemma{sagittae}\Bfootnote{\textit{(1)}\ pro \textit{(2)} moventur \textit{L}}} Examinatur et motus superficierum inclinatarum. Ubi demonstratur illa tam aestimata propositio quam scio et ab Hugenio
\protect\index{Namensregister}{\textso{Huygens} (Hugenius, Ugenius, Hugens, Huguens), Christiaan 1629-1695}
demonstratam, de uniformitate motus in \edtext{Cycloeide.}{\lemma{Cycloeide.}\Cfootnote{\cite{00296}\textsc{I. G. Pardies}, \textit{La Statique}, Paris 1673, Vorwort.}} (+ dicendum erat quis primus observator fuerit propositionis +) Quarta dissertatio est de motu corporum liquidorum, ubi sine novis suppositionibus omnia demonstrantur quae in eorum celeritate, viribus, directione, figura, eveniunt. Ubi et pneumatica, Elateriique vires, pulveris \edtext{pyrii.}{\lemma{pyrii.}\Cfootnote{\cite{00296}\textsc{I. G. Pardies}, a.a.O., Vorwort.}} 5\textsuperscript{ta} dissertatio est de vibrationibus, ut penduli, cordarum tensarum Elateriorum, describitur pendulum \edtext{ibi omnes vibrationes}{\lemma{}\Bfootnote{ibi omnes vibrationes \textit{erg. L}}} aequalis durationis, ostenditur omnes vibrationes chordae esse isochronas, et vibrationes duarum chordarum aequalis crassitiei et tensionis, esse in reciproca ratione longitudinum, cum pendula sint solum in ratione subdupla; si cordae
\edtext{aequales, vibrationes esse in subdupla virium}{\lemma{aequales,}\Bfootnote{\textit{(1)}\ vires esse \textit{(2)}\ vibrationes [...] virium \textit{L}}}
\edtext{tendentium.}{\lemma{tendentium.}\Cfootnote{\cite{00296}\textsc{I. G. Pardies}, a.a.O., Vorwort.}} Sexta dissertatio est de motu Undulationis, \edtext{qualis circulorum,}{\lemma{}\Bfootnote{qualis \textbar\ fit \textit{gestr.}\ \textbar\ circulorum, \textit{L}}} laxo, in aquam injecto. Idem in aere facere sonum, et in aere subtiliore lumen. (+ Ego, credo calorem propagari ut sonum per motum liquidi, at non \edtext{lumen. Praeterea}{\lemma{lumen.}\Bfootnote{\textit{(1)}\ Lumen \textit{(2)}\ Praeterea \textit{L}}} undulatio forte in superficie tantum. +[\thinspace )\thinspace] His passim inserentur, mira artis opera de ducta aquarum, de Molendinis semper euntibus, levandis aquis. De proportione pulveris Minis necessariis, et tormentis. De jaciendis secure bombis de longitudine canonis inferendo quantum fieri potest ictu, de novis \edtext{machinis ad voluptatem}{\lemma{machinis}\Bfootnote{\textit{(1)} aptis ad \textit{(2)}\ ad jucundi \textit{(3) } ad voluptatem \textit{L}}} de motu quodam perpetuo de impossibilitate motus Atomorum Epicuri, motum coelorum non esse ab intrinseco, Systema Tychonis\protect\index{Namensregister}{\textso{Brahe} (Tycho), Tycho 1546-1601} posse Mechanice explicari; de ratione mechanica explicandi corporis duritatem. De fluxu et refluxu maris origine \edtext{fontium.}{\lemma{fontium.}\Cfootnote{\cite{00296}\textsc{I. G. Pardies}, a.a.O., Vorwort.}}
\pend 
\count\Bfootins=1200
\count\Cfootins=1200
%\edtext{fontium.}{\lemma{fontium.}\Cfootnote{\cite{00296}\textsc{I. G. Pardies}, a.a.O., Vorwort.}} 
%\pend 
%\count\Bfootins=1200
%\count\Cfootins=1200
%\pstart
 

