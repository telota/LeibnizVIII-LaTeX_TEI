\begin{Ueberlieferung}% 
{\textit{L}}Aufzeichnung: LH XXXVII 3 Bl. 87. 1 Bl. 19 x 7 cm. 14 Z. auf Bl. 87~r\textsuperscript{o}. Bl. 87~v\textsuperscript{o} leer. Blatt oben und unten beschnitten. Ein Wasserzeichen.\\%
Cc 2, Nr. 1142 
\end{Ueberlieferung}
%
%\begin{Datierungsgruende}%
%Siehe Einleitung zu N. ???.
%\end{Datierungsgruende}
%
\count\Afootins=1200
\count\Bfootins=1200
\count\Cfootins=1200
\pstartfirst%
% [87~r\textsuperscript{o}]
[87~r\textsuperscript{o}]
Mons. le duc de Roannez\protect\index{Namenregister}{\textso{Roanez}, Artus Gouffier de 1627-1696} m'a dit que les eaux font des grands sauts quelques fois, \`{a} ces endroits on a mis des digues,\protect\index{Sachverzeichnis}{digue} et des pertuis\protect\index{Sachverzeichnis}{pertuis}. L'eau qui passe sur la digue\protect\index{Sachverzeichnis}{digue} ou plus tost degrez, et tombe par apres ne cave point comme il est ais\'{e} de juger. Mais celle qui passe le pertuis\protect\index{Sachverzeichnis}{pertuis} cave le fonds bien tost; \`{a} cause de la quantit\'{e} et de la vitesse avec laquelle elle va raser la terre, au lieu que celle qui passe le degrez ou cataracte ne tombe que sur une autre eau. Je croy que dans un tuyau l'eau de milieu ira plus viste, la force de l'eau\protect\index{Sachverzeichnis}{eau} contre toute la digue\protect\index{Sachverzeichnis}{digue} sera celle de son poids\protect\index{Sachverzeichnis}{poids}, (la liquide ayant le double de vistesse, a le quadruple de \edtext{force, et ayant}{\lemma{force,}\Bfootnote{%
\textit{(1)}\ \`{a} cause %
\textit{(2)}\ et \textbar\ ainsi \textit{gestr.} \textbar\ ayant \textit{L}}}
le triple de vistesse aura le noncuple de force[)]
la raison par ce que le degrez de vistesse est augment\'{e},
mais aussi en même temps la quantit\'{e} de la matiere qui passe. Les pertuis\protect\index{Sachverzeichnis}{pertuis} font que l'eau ne fait pas le saut\protect\index{Sachverzeichnis}{saut} trop rudement. Car alors on a de la peine \`{a} y faire remonter les bateaux.\protect\index{Sachverzeichnis}{bateau} Si le pertuis\protect\index{Sachverzeichnis}{pertuis} est plus long, on gagnera de toutes les manieres, le saut sera plus doux et la pente moindre. Il passera moins d'eau, et avec moins de vistesse, les pertuis\protect\index{Sachverzeichnis}{pertuis} \`{a} cause de la vistesse font perdre une grande quantit\'{e} d'eau\protect\index{Sachverzeichnis}{eau}. Rigoles\protect\index{Sachverzeichnis}{rigole} dans des canaux\protect\index{Sachverzeichnis}{canal} de pierre bien unis, de 200 pas, proche Paris\protect\index{Ortsregister}{Paris}, comme \`{a} Berny\protect\index{Ortsregister}{Berny-Rivi\`{e}re} et dans autres maisons de campagne propres \`{a} faire des experiences\protect\index{Sachverzeichnis}{experience} en ces matieres.  
\pend 
\count\Afootins=1500
\count\Bfootins=1500
\count\Cfootins=1500
%\newpage
 

