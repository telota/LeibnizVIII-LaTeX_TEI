\centering [\textit{Teil 2}]
\pend
\count\Bfootins=1000
\count\Cfootins=1000
%\vspace*{0,5em}
\pstart
\noindent
Pulchrum satis \edtext{foret, ope}{\lemma{foret,}\Bfootnote{\textit{(1)}\ opus \textit{(2)}\ ope \textit{L}}}
motus gravium accelerati,
exhibere motum uniformem, id pendulum\protect\index{Sachverzeichnis}{pendulum} ita suspendere,
ut \edtext{non vibrationes sint aequidiuturnae, sed motus ponderis\protect\index{Sachverzeichnis}{pondus} suspensi semper aequivelox.}{\lemma{non}\Bfootnote{\textit{(1)}\ tantum vibrationes, sed ipsi motus p \textit{(2)}\ vibrationes [...] aequivelox. \textit{L}}}
Ita ut motus quanto crescit \edtext{magis (inde a certo quodam loco)}{\lemma{magis}\Bfootnote{\textit{(1)}\ quo ad \textit{(2)}\ (inde [...] loco) \textit{L}}}
eo magis \edtext{oneretur onus penduli}{\lemma{oneretur}\Bfootnote{\textit{(1)}\ pendulum \textit{(2)}\ onus penduli. \textit{L}}}.
\pend
%\newpage
\pstart
Nec contemnendum foret, ita suspendere pendulum intra duas laminas,
ut motu suo lineam describat rectam, ope scilicet compositionis
motuum, agitando scilicet etiam ipsas laminas.
\pend
\pstart
Voyons \edtext{ce qui arrivera si}{\lemma{ce qui}\Bfootnote{\textit{(1)}\ arrive si \textit{(2)}\ arrivera si \textit{L}}}
un balancier \edtext{ou volant\protect\index{Sachverzeichnis}{volant}}{\lemma{}\Bfootnote{ou volant \textit{erg.} \textit{L}}} 
qui \edtext{est en bransle rencontre}{\lemma{est}\Bfootnote{\textit{(1)}\ bransl\'{e} rencontre \textit{(2)}\ en bransle rencontre \textit{L}}}
un poids\protect\index{Sachverzeichnis}{poids} qu'il doit lever. Son mouuement
\edtext{present}{\lemma{}\Bfootnote{present \textit{erg.} \textit{L}}}
ne vient ny de la pesanteur\protect\index{Sachverzeichnis}{pesanteur} ny du 
\edtext{ressort,\protect\index{Sachverzeichnis}{ressort} mais de la nature du mouuement
qui se continue aussi bien que de l'ondulation des parties des liquides, qui sont}{\lemma{ressort,}\Bfootnote{%
\textit{(1)}\ mais de l'ondulation %
\textit{(2)}\ car %
\textit{(3)}\ mais [...] mouuement %
\textbar\ qui se continue \textit{erg.} \textbar\ aussi [...] l'ondulation %
\textit{(a)}\ des liquides qui soit %
\textit{(b)}\ des parties [...] sont \textit{L}}}
me\^{u}es
\edtext{correspondamment. Supposons donc qu'il rencontre}{\lemma{correspondamment}\Bfootnote{%
\textit{(1)}\ , et du %
\textit{(2)}\ . Supposons %
\textit{(a)}\ qu'il rencontre donc %
\textit{(b)}\ donc qu'il rencontre \textit{L}}}
un poids \`{a} lever.
Il faut que la force\protect\index{Sachverzeichnis}{force} de son mouuement, qui est un reste de la force de la premiere impression, soit plus grande que la force du poids \`{a} lever: ce qui est tousjours parce
\edtext{que l'impression du balancier}{\lemma{que}\Bfootnote{\textit{(1)}\ le mouuement du balancie \textit{(2)}\ l'impression du balancier \textit{L}}}
a tousjours est\'{e} un mouuement acceler\'{e}.
Il levera donc ce poids; mais d'autant
\edtext{[plus]}{\lemma{}\Bfootnote{plus \textit{erg. Hrsg.}}}
viste qu'il le leve d'autant plus de
\edtext{resistence trouuera-t-il}{\lemma{resistence}\Bfootnote{\textit{(1)}\ trouueroit-il \textit{(2)}\ trouuera-t-il. \textit{L}}}.
Or cette resistence ne s\c{c}auroit faire autre chose, que retarder son mouuement.
Car ce n'est pas comme quand deux corps pesans se rencontrent dont l'un monte dans de l'eau, par exemple, l'autre descend:
car la vitesse\protect\index{Sachverzeichnis}{vitesse} n'en est pas diminu\'{e},
(:~si non autant que le mouuement est une continuation ou bransle~:).
Et la demonstration est manifeste, par ce qu'icy il y a autant de matiere mue avant qu'apr\`{e}s la resistence,
il faut donc que la resistence ne fasse que deminuer la vitesse,
car sans cela elle ne feroit rien du tout, donc les deminutions seront comme les vitesses. [7~v\textsuperscript{o}]
\pend
\pstart
\vspace{1em}% PR: Provisorisch!