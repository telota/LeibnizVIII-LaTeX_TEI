[1~v\textsuperscript{o}]
\edtext{regeln}{\lemma{regeln}\Cfootnote{\cite{00088}\textsc{S. Santorio}, \textit{De statica medicina}, Venedig 1614.}}
gebracht worden.
%
Welche denn reassumirt, prosequirt, und auf alle particularia accommodirt werden mu{\ss}.
\pend%
\pstart%
Man k\"{o}ndte auch wohl Experimenta Medica Elastica\protect\index{Sachverzeichnis}{medica elastica} anstellen von Vermehr oder Verminderung der Kr\"{a}ffte des Menschen, so an spannung eines bogens oder wurffs weitig\-keit am besten aber an langer ausdaurung gewi{\ss}er arbeit,
\edtext{[als]}{\lemma{dls}\Bfootnote{\textit{L \"{a}ndert Hrsg.}}}
des gehens, tragens etc. zu probiren, alleine es thut hier den exercitium das beste, da{\ss} also nicht wohl operae pretium hier gnugsame untersuchung zu thun.
\pend%
\pstart%
Zur observation des Pulses\protect\index{Sachverzeichnis}{Puls} gehohret die observation der Warme\protect\index{Sachverzeichnis}{W\"{a}rme} und Kalte\protect\index{Sachverzeichnis}{K\"{a}lte} der hande\protect\index{Sachverzeichnis}{Hand} an einem exacten wohl verbesserten Thermometro\protect\index{Sachverzeichnis}{thermometrum}. Denn mancher mensch kalte, mancher warme h\"{a}nde\protect\index{Sachverzeichnis}{Hand} von natur hat, mehr oder weniger nach seiner constitution\protect\index{Sachverzeichnis}{Konstitution}. 
\pend%
\pstart%
Das Thermometrum\protect\index{Sachverzeichnis}{thermometrum} aber mus rectificirt werden, sowohl nach P. Eschinardi\protect\index{Namensregister}{\textso{Eschinardi}, Francesco 1623-1703}
\edtext{erinnerung,}{\lemma{erinnerung}\Cfootnote{\cite{00042}\textsc{F. Eschinardi}, \glqq Difetti de' termometri\grqq, \textit{Giornale de' Letterati}, 27. Februar 1670, S. 22f.}}
%
als auch nach der proposition so in England\protect\index{Ortsregister}{England} mit einem Thermometro\protect\index{Sachverzeichnis}{thermometrum} circulari gethan worden, wie die
\edtext{\textit{Historia societatis}}{\lemma{\textit{Historia societatis}}\Cfootnote{\cite{00098}\textsc{T.~Sprat}, \textit{History of the Royal Society}, London 1667, S. 313.}}
%
\edtext{\edlabel{erzehlet}erzehlet.}{{%
\xxref{erzehlet}{Ferner}}\lemma{erzehlet}\Bfootnote{\textit{(1)}\ Die \textit{(2)}\ Ferner \textit{L}}}
\pend%
\pstart%
Ferner\edlabel{Ferner} %
k\"{o}ndten Proben mit dem Menschen angestellt werden durchs bad, in dem, das von ihm abgespuhlte anatomirt und examinirt w\"{u}rde.
\pend%
\pstart%
So k\"{o}ndte auch der Halitus\protect\index{Sachverzeichnis}{halitus} examinirt werden, dieweil selbiger in ein corpus zu reduciren. 
\pend%
\pstart%
Ein ieder mensch mus achtung auff
\edtext{sich geben}{\lemma{sich}\Bfootnote{\textit{(1)}\ wegen \textit{(2)}\ geben \textit{L}}}
was den schweis\protect\index{Sachverzeichnis}{Schwei{\ss}} betrifft. Der schwei{\ss}\protect\index{Sachverzeichnis}{Schwei{\ss}} kan auff gefangen und de{\ss}en gradus salsedinis\protect\index{Sachverzeichnis}{gradus salsedinis} etc. examinirt werden. 
\pend%
\newpage
\pstart%
Man soll in der Republick gewi{\ss}e Menschen haben die sich gewohnt mit dem geruch\protect\index{Sachverzeichnis}{Geruch}, fuhlen\protect\index{Sachverzeichnis}{F\"{u}hlen}, schmacken\protect\index{Sachverzeichnis}{Schmecken} etc. zu hochster perfection zu kommen, durch die kan man alle res dubias examiniren la{\ss}en. 
\pend%
\pstart%
Ein iedes Amt solte billich einen Medicum, Chirurgum\protect\index{Sachverzeichnis}{chirurgus}, Apotheker\protect\index{Sachverzeichnis}{Apotheker} und mehr andere dazu geh\"{o}rige leute haben. 
\pend%
\pstart%
Ein Koch solte perfect seyn alle dinge ausm geschmack\protect\index{Sachverzeichnis}{Geschmack} und geruch\protect\index{Sachverzeichnis}{Geruch} zu unterscheiden und solte darauff examinirt werden. 
\pend%
\pstart%
Ein Barbierer solte im fuhlen\protect\index{Sachverzeichnis}{F\"{u}hlen} perfect seyn, man m\"{u}ste leute haben, die es per tactum dahin gebracht, wohin der Blinde beym herren
\edtext{Boyle\protect\index{Namensregister}{\textso{Boyle}, Robert 1627-1691}}{\lemma{Boyle}\Cfootnote{\cite{00014}\textsc{R. Boyle}, \textit{Experiments and considerations touching colours}, London 1664, S. 41-49.}}
so alles m\"{u}glich. 
\pend%
\pstart%
Auch
\edtext{von der}{\lemma{von}\Bfootnote{\textit{(1)}\ den \textit{(2)}\ der \textit{L}}}
clarheit, starcke%
% Hier folgt Bl. 2r.