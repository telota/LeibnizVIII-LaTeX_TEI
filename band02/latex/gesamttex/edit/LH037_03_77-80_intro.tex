\footnotesize%
\pstart%
\noindent%
Die vier Unterstücke, aus denen N.~45 besteht, hängen inhaltlich stark zusammen, sind aber unterschied\-lich ausgearbeitet.
Ihr gemeinsamer Gegenstand ist das mechanische Verhalten miteinander gekoppelter Hebel.
Die verschiedenen Wasserzeichen in den Textträgern von N.~45 sind insgesamt für den Zeitraum vom Anfang 1674 bis zum Anfang 1675 belegt.
Das von Leibniz auf Dezember 1674 datierte Stück N.~10 weist jedoch anscheinend auf die Ergebnisse von N.~45 hin.
Im Textträger des Unterstücks N.~45\textsubscript{1}, welches sich aus inhaltlichen Gründen als das früheste in der Vierergruppe erweist,
liegt ferner ein Wasserzeichen vor, das nur für die Monate ab (frühestens) Mitte 1674 belegt ist;
derselbe Typus kommt nämlich in Text\-trägern von N.~8, N.~9, N.~28 und N.~50 vor.
Demgemäß lässt sich die Entstehungszeit von N.~45 ins\-ge\-samt auf die zweite Hälfte 1674 eingrenzen.

\pend
\normalsize
%
%
%
%\begin{Ueberlieferung}% 
%{\textit{L}}Konzept: LH XXXVII 3 Bl. 77-80. 1 Bog. (Bl. 77-78) und 1 Bl. 2\textsuperscript{o}, Bl. 79 unregelmäßig beschnitten (ca. 18x17cm).  Bl. 77~v\textsuperscript{o}, 79~v\textsuperscript{o} leer. Reihenfolge Bl. 77~r\textsuperscript{o}, 78~v\textsuperscript{o} (mit Textverlust durch Papierbruch), 78~r\textsuperscript{o}, 79~r\textsuperscript{o} (Reinschrift mit Verbesserungen), 80~r\textsuperscript{o}, 80~v\textsuperscript{o}.
%\end{Ueberlieferung}
%\vspace*{8mm}
%\begin{Datierungsgruende}%
%??
%\end{Datierungsgruende}
%
%
% knüpft an Leibniz' Exzerpte aus dem dritten Teil von \cite{00301}\textsc{John Wallis}, \textit{Mechanica sive De motu}, 2 Bde., London 1670-1671, bes. Kap. VI an, die in N. ?? % W/1
% ediert sind.
% Diese lassen sich auf den Zeitraum vom Herbst 1674 bis zum Frühjahr 1675 datieren.