\begin{ledgroupsized}[r]{120mm}%
\footnotesize%
\pstart%
\noindent\textbf{\"{U}berlieferung:}%
\pend%
\end{ledgroupsized}%
\begin{ledgroupsized}[r]{114mm}%
\footnotesize%
\pstart%
\parindent -6mm%
\makebox[6mm][l]{\textit{L}}%
Aufzeichnung:
LH III 5 Bl. 56.
1 Bl. 2\textsuperscript{o}. 1 S. auf Bl.~56~r\textsuperscript{o}.
Bl.~56~v\textsuperscript{o} leer.%
\newline%
Cc 2, Nr. 1271%
\pend%
\end{ledgroupsized}%
%
%
\vspace*{8mm}%
\pstart %
\normalsize%
\noindent%
% [56~r\textsuperscript{o}]
[56~r\textsuperscript{o}] 25 janvier 1676
\pend%
\count\Bfootins=1200
\count\Cfootins=1200
\count\Afootins=1200
\pstart%
\noindent%
Estant venu
\edtext{un matin}{\lemma{un matin}\Bfootnote{\textit{erg. L}}}
chez Mons. l'Abb\'{e} de Gravel,\protect\index{Namensregister}{\textso{Gravel}, Jacques Argentan de, 1679}
% \edtext{}{\lemma{Mons.}\Cfootnote{Leibniz hatte de Gravel als franz\"{o}sischen Gesandten am Mainzer Hof unter Johann Philipp 1671 kennen gelernt. Nach Gravels Ausweisung 1674 trafen sich beide wieder in Paris. Die hier festgehaltene Begegnung fand statt, als Leibniz gerade die Stelle als Hofrat in Hannover angenommen hatte. Wegen dieser Ver\"{a}nderung mu{\ss}te er seinen Plan aufgeben, den franz\"{o}sischen Gesandten \"{u}ber den Jahres\-wechsel 1675/76 zu Verhandlungen \"{u}ber die Neutralit\"{a}t des Bistums L\"{u}ttich nach Marchienne au Pont zu begleiten. Cf. Chronik, S. 41.}}
pour aller avec luy \`{a} \edtext{St. Germain,\protect\index{Ortsregister}{Paris, Saint-Germain}
il me dit qu'il}{\lemma{St. Germain, il}\Bfootnote{%
\textit{(1)} venoit de %
\textit{(2)} me dit qu'il \textit{L}}}
venoit de faire une operation de chirurgie sur luy m\^{e}me; contre la goutte, par precaution. Je luy demanda ce que c'estoit. Il me dit que son frere\protect\index{Namensregister}{\textso{Gravel}, Robert de} avoit appris en Allemagne\protect\index{Ortsregister}{Deutschland}, que lorsqu'on commence d'estre attaqu\'{e} de la goutte, ou qu'on l'apprehende, il faut toutes les nouvelles lunes, le plus pr\`{e}s du veritable temps de la nouvelle $\rightmoon$, qu'on peut faire des incisions ou scarifications sur le pouce du pied ou on commence \`{a} ressentir ou apprehender le
\edtext{mal au dessus.}{\lemma{mal}\Bfootnote{\textit{(1)} , un peu au dessus, un peu \textit{(2)} au dessus. Son \textit{L}}}
Son frere commen\c{c}a \`{a} en avoir deux acces icy, \edtext{il y a quelques temps,}{\lemma{il y a quelques}\Bfootnote{\textit{(1)} mois \textit{(2)} temps, \textit{L}}}
il s'en servit; le mal ne revient plus. Luy y a deja ressenti quelques fois des douleurs de goutte bien plus violents encor que son frere\protect\index{Namensregister}{\textso{Gravel}, Robert de}, la dessus il s'est servi de
\edtext{[ce]}{\lemma{se}\Bfootnote{\textit{L \"{a}ndert Hrsg.}}}
m\^{e}me remede, et le mal n'est plus revenu depuis. Il n'y a rien de si raisonnable. Car c'est la plus basse, et la derniere partie du corps, o\`{u} les humeurs les plus grossiers et le plus gluants ou tartareux se rendent peu \`{a} a peu, et enfin s'en durcissent; ou enflent au moins la partie. C'est pourquoy il faut leur donner vent. Cela sert \`{a} y remuer le sang, et \`{a} luy donner de l'air. Il faut apres l'incision appliquer la ventouse, la mode
\edtext{de ventouser}{\lemma{de ventouser}\Bfootnote{\textit{erg. L}}}
des Allemands avec les petites pointes qu'ils donnent est bien plus commode. Pour moy je me souuiens d'avoir entendu la m\^{e}me chose en Allemagne; comme un remede assur\'{e} contre la goutte, quand elle seroit m\^{e}me confirm\'{e}e.
\pend%
\pstart%
Il y a une esp\'{e}ce de Maladie \`{a}
\edtext{Paris, dont}{\lemma{Paris}\Bfootnote{\textit{(1)} que \textit{(2)} dont \textit{L}}}\protect\index{Ortsregister}{Paris, Saint-Germain}
les femmes se plaignent
\edtext{ordinairement, et qu'elles appellent}{\lemma{ordinairement,}\Bfootnote{\textit{(1)} qu'on \textit{(2)} et qu' \textit{(a)}\ ils appel \textit{(b)}\ elles appellent \textit{L}}}
\edtext{vapeurs.}{\lemma{vapeurs}\Cfootnote{Über \glqq Dämpfe\grqq~als Krankheitsursachen siehe \textsc{M. Laxenare} und \textsc{A. Chanson}, \glqq Les va\-peurs: Aperçu historique\grqq, 
\textit{Annales M\'{e}dico-psychologiques} 146 (1988), S. 637-644.\cite{01177}}}
%
Ce sont comme des \'{e}blouissements, et surprises et foiblesses subites qui les prennent et s'en vont tout \`{a} coup et reviennent par intervalles. Et comme cela les eblouit, comme si quelque nu\'{e}e epaisse venoit \`{a} leur obscurcir la veue et l'esprit, elles appellent cecy des
\edtext{vapeurs. Or il est bien manifeste que cecy ne scauroient estre des vapeurs.}{\lemma{vapeurs.}\Bfootnote{\textit{(1)} Or cela n'est qu'une disposition \`{a} la syncope \textit{(2)} Or [...] vapeurs. \textit{L}}}
La comparaison de la teste pour un alembic est fort mal fond\'{e}e; il n'y a point de passages pour la distillation et dans la teste m\^{e}me pour une vapeur il faudroit des places vuides, o\`{u} la vapeur se p\^{u}t \edtext{rassembler.
%
\protect\index{Namensregister}{\textso{Alliot}, Jean-Baptiste ??-??} Or Mons. Alliot le jeune}{\lemma{rassembler.}\Bfootnote{\textit{(1)} Il y a quelqu \textit{(2)}
Or [...] jeune, \textit{L}}}, m'a cont\'{e} que son pere\protect\index{Namensregister}{\textso{Alliot}, Pierre, gest. zwischen 1680-1697} et luy avec Mons. Bourdelot\protect\index{Namensregister}{\textso{Bourdelot}, Pierre Michon 1610-1685} et autres ont assist\'{e} \`{a} l'ouuerture du corps de Mons. le Marechal de Clerambault\protect\index{Namensregister}{\textso{Cl\'{e}rambault}, Philippe, 1620-1665}, on y trouua dans un des passages du
\edtext{sang du}{\lemma{sang}\Bfootnote{\textit{(1)} dans l \textit{(2)} du \textit{L}}}
coeur au poulmon ou co\^{u}tre (et car je ne m'en souviens pas bien) un gros morceau de chair spongieux comme une langue de carpe, qui avoit bouch\'{e} le passage du sang. Car il est probable, que le sang
\edtext{rencontrant ces bouchons}{\lemma{rencontrant}\Bfootnote{\textit{(1)} ses ouvertures \textit{(2)} ces bouchons \textit{L}}}
se reflechit en luy m\^{e}me, et par une espece de revulsion se retire en arri\`{e}re de toutes les extremitez, vers le coeur. Cela doit faire un affaiblissement
\edtext{[subit],}{\lemma{subite}\Bfootnote{\textit{L \"{a}ndert Hrsg.}}}
mais qui cesse incontinent. C'est une disposition \`{a} la syncope, lorsque le sang ne peut plus passer autant qu'il faut pour la vie, on meurt. Ces obstructions causent dans les femmes des desordres dans le bas
\edtext{ventre, ou matrice}{\lemma{ventre,}\Bfootnote{\textit{(1)} ou le sang \textit{(2)} ou matrice \textit{L}}}
comme si on bouchoit viste un alembic pour empecher l'esprit qui veut sortir, tout creveroit. Or les medecins fondent leur indication ridiculement sur le nom de vapeur. Il faut, disent-ils, les condenser, donc il donnent des limonades et autres acides, les quels avancent le mal, parce qu'ils servent \`{a} augmenter la coagulation qui est dans le sang. Et on l'a essay\'{e}, car ayant pris ce morceau qui s'estoit trouu\'{e}, chez Mons. de Clerambault\protect\index{Namensregister}{\textso{Cl\'{e}rambault}, Philippe 1620-1665} et on a tach\'{e} de le dissoudre dans le vinaigre, mais cela n'a servi qu'\`{a} l'endurcir. Par apres on la fort bien dissolu dans un alcali, comme lessive. Donc il faut des
\edtext{[alcalis]}{\lemma{alcali}\Bfootnote{\textit{L \"{a}ndert Hrsg.}}}
pour le dissoudre, et afin, qu'ils penetrent jusque dans le sang, il faut des
\edtext{[alcalis]}{\lemma{alcali}\Bfootnote{\textit{L \"{a}ndert Hrsg.}}}
bien \edtext{[volatiles]}{\lemma{volatile}\Bfootnote{\textit{L \"{a}ndert Hrsg.}}}
\edtext{et [penetrants],}{\lemma{}\Bfootnote{et \textbar\ penetrant \textit{ändert Hrsg.} \textbar\ \textit{erg. L}}} comme l'esprit d'urine, ou salmoniac.
Mons. \edtext{Alliot le pere}{\lemma{Alliot le pere}\Cfootnote{\textsc{P. Alliot}, \textit{Epistola de nuntio profligati sine ferro et igne carcinomatis missus}, Paris 1664.% Dazu: \textsc{M.D. Grmek}, \textit{Leibniz et la m\'{e}dicine pratique}. In: \textit{Leibniz 1646-1716. Aspects de l'homme et de l'oeuvre}, Paris 1968, S. 145-177, hier S. 150 Anm. 1.
}}
\edtext{a fait}{\lemma{a}\Bfootnote{\textit{(1)} so\^{u}tenu \textit{(2)} fait \textit{L}}}
un \'{e}crit, De Cancro sine igne et ferro (per alcalia) curato. Sylvius\protect\index{Namensregister}{\textso{Sylvius}, Franz de la Bo\"{e} 1614-1672} luy \'{e}crit l\`{a} dessus une lettre fort honneste, et luy dit qu'il falloit, qu'ils eussent eu le m\^{e}me maistre (Helmont\protect\index{Namensregister}{\textso{Helmont}, Johann Baptist van 1579-1644} apres la nature) pour avoir des sentiments si conformes.
%
Bartholin\protect\index{Namensregister}{\textso{Bartholin} (Bartholinus), Thomas 1616-1680} in catalogo autorum de son \textit{Anatomia reformata},
\edtext{derniere edition,}{\lemma{derniere edition}\Cfootnote{\cite{01178}\textsc{T. Bartholin}, \textit{Anatomia reformata}, Leiden und Rotterdam 1669.}} \edtext{cite aussi}{\lemma{cite}\Bfootnote{\textit{(1)} le dit \textit{(2)} aussi \textit{L}}}
Petrum Alliot.\protect\index{Namensregister}{\textso{Alliot}, Pierre, gest. zwischen 1680-1697}
%
Le jeune Alliot\protect\index{Namensregister}{\textso{Alliot}, Jean-Baptiste ??-??} a so\^{u}tenu une these; quod natura vitalem exerceat Chymiam. Mons. Alliot le pere\protect\index{Namensregister}{\textso{Alliot}, Pierre, gest. zwischen 1680-1697} a cr\^{u} que Vesicula fellis cum chylo in intestino tenui facit effervescentiam, avant que d'avoir entendu que Mons Alliot\protect\index{Namensregister}{\textso{Alliot}, Jean-Baptiste ??-??} enseigne la m\^{e}me
\edtext{chose. C'est vesiculae felleae liquor, qui}{\lemma{chose.}\Bfootnote{\textit{(1)} Ce qu'o \textit{(2)} C'est [...] qui \textit{L}}}
entretient la fluidit\'{e} et le mouvement dans le sang, par son alcali. Et ce qu'on attribue vulgairement au defaut de la chaleur naturelle, ne vient que du defaut de cette liqueur. Ils engendrent des pierres dans cette vesicule qui diminuent la quantit\'{e} necessaire du fiel. Amarum et acidum reagentia faciunt tertium salsum, quod est urinosum illud sal. Unde fermentatione opus est
\edtext{ad alcali}{\lemma{ad}\Bfootnote{\textit{(1)} spiritum \textit{(2)} alcali \textit{L}}} ex urina recuperandum
(+~mihi videtur Amarum et Acidum facere
\edtext{salsum; proprie et}{\lemma{salsum;}\Bfootnote{\textit{(1)} salsum et \textit{(2)} proprie et \textit{L}}}
gustu talia; seu ex salso non ipsa plane sed nonnihil dissimilia per putrefactionem restitui, sed alias substantias; quibus nomina invenienda~+).
Un nomm\'{e} Jesson\protect\index{Namensregister}{\textso{Jesson}, ???? ??-??} chirurgien ou apothiquaire, que je rencontra chez Mons. Alliot\protect\index{Namensregister}{\textso{Alliot}, Jean-Baptiste ??-??} le
\edtext{jeune, me}{\lemma{}\Bfootnote{jeune,\ \textbar\ en \textit{erg. u. gestr.} \textbar\ me \textit{L}}}
dit, qu'il avoit trouv\'{e} par le raisonnement un moyen de distiller l'esprit d'urine
\edtext{en un instant}{\lemma{en un instant}\Bfootnote{\textit{erg.} \textit{L}}} sans aucune fermentation, en abregeant cette fermentation par l'injection de certaines choses (+ alcalis qui mangent l'acide apparemment afin qu'il quitte l'alcali volatile de l'urine +) apres avoir evapor\'{e} l'urine ad consistentiam mellis.
\pend%
\pstart%
Rien de meilleur contre le rheume
\edtext{schnupfen}{\lemma{schnupfen}\Bfootnote{\textit{erg. L}}}
que de se tenir longtemps droit, sans incliner sa teste.%
\pend%
\count\Bfootins=1500
\count\Cfootins=1500
\count\Afootins=1500
%%%% PR: Das Stück endet hier.