\begin{ledgroupsized}[r]{120mm}%
\footnotesize%
\pstart%
\noindent\textbf{\"{U}berlieferung:}%
\pend%
\end{ledgroupsized}%
\begin{ledgroupsized}[r]{114mm}%
\footnotesize%
\pstart%
\parindent -6mm%
\makebox[6mm][l]{\textit{L}}%
Aufzeichnung:
LH III 5 Bl. 55.
1 Bl. 2\textsuperscript{o}. 1 S. auf Bl.~55~r\textsuperscript{o}.
Bl. 55~v\textsuperscript{o} leer.
Geringf\"{u}giger Textverlust durch Papierbeschädigung am Falz und am Rand.
Schwer erkennbares Wasserzeichen.%
\newline%
KK1, Nr. 977%
\pend%
\end{ledgroupsized}%
%
\vspace*{5mm}%
\begin{ledgroup}%
\footnotesize%
\pstart%
\noindent%
\footnotesize{%
\textbf{Datierungsgr\"{u}nde:}
% Das schwer erkennbare Wasserzeichen im Textträger des vorliegenden Stücks N. ?? weist Ähnlichkeit mit dem Wasserzeichen des eigenhändig auf den 24. Februar 1676 datierten Stücks N. ??. % = LH003,04,03a_001 = Remedia et vires medicamentorum.
% Aus diesem Grund wird N. ?? % = vorliegendes Stück = LH003,05_055
% auf Anfang 1676 datiert.
Im vorliegenden Stück verweist Leibniz auf \cite{00008}D.~Beckhers Abhandlung \textit{De cultrivoro prussiaco}.
Messerschlucker werden auch in N.~70 sowie in den dort zitierten Ausgaben der \textit{Miscellanea curiosa physico-medica}\cite{00001} erwähnt, darunter auch der preußische; siehe insbesondere \textit{Miscellanea curiosa}, 1 (1671), S.~406: Observatio CCLII.
% LH003,01,03_001-008 = Directiones ad rem medicam pertinentes
Da weitere Anhaltspunkte für eine genauere chronologische Einordnung fehlen, wird die Datierung von N.~70 hier übernommen.
% Die ohne Literaturangaben erz\"{a}hlten Ereignisse sind lokalisiert in Mainz und Schwalbach, das kann als Hinweis auf eine Datierung in der Mainzer Zeit gelesen werden. Der Texttr\"{a}ger hat nicht ein Wasserzeichen, aber die Stegbreite (?? mm) deutet auf franz\"{o}sisches Papier hin ??. Damit ergibt sich eine Datierung auf ??.%
}%
\pend%
\end{ledgroup}%
%
%
\vspace*{8mm}%
\count\Bfootins=1200
\count\Cfootins=1200
\count\Afootins=1200
\pstart%
\normalsize%
\noindent%
% [55~r\textsuperscript{o}]%
[55~r\textsuperscript{o}] 
In hizigen Kranckheiten hat man vor diesen den trunck verbothen erat haeresis priorum temporum. Hodie contrarium admissum est. Man giebt dem patienten genug zu trincken contra medicorum sententiam, \edtext{doch nicht starck gebraucke. Einer von Einzing}{\lemma{doch}\Bfootnote{\textit{(1)}\ nur von lin \textit{(2)}\ nicht starck gebraucke. \textit{(a)}\ Der junge Einz \textit{(b)}\ Einer von Einzing \textit{L}}} war todt kranck, begehrte von seinen Medicis da{\ss} er nur eine sach durffe den durst \edtext{leschen. D.~Spina u. andere}{\lemma{leschen.}\Bfootnote{\textit{(1)}\ Es \textit{(2)}\ Sie \textit{(3)}\ D.~Spina u. andere \textit{L}}} schlugens ihm ab. Seine \edtext{Banst}{\lemma{Banst}\Cfootnote{Banz, Frauenzimmer. Siehe \cite{01167}\textsc{J. Müller}, \textit{Rheinisches Wörterbuch}, Bonn 1928, Bd.~1, Sp.~449.}}
die fr. von Boineburg\protect\index{Namensregister}{\textso{Sch\"{u}tz von Holzhausen}, Anna Christin geb. (1630-1689)} Mutter\protect\index{Namensregister}{\textso{Dorfelden}, Maria Eva von (1600-??)}, gab ihm heimlich einen Krug schwalbacher, (es war zu Schwalbach\protect\index{Ortsregister}{Schwalbach}) den tranck er auf einen trunck aus. Legte sich hin und schlieff, welches er viele nachte nicht gethan, den ganzen tag fest, stund auff und war frisch und gesund. Des kaysers bruder Carl Joseph\protect\index{Namensregister}{\textso{\"{O}sterreich}, Karl Jospeh von (1649-1664)} soll verwahrlo{\ss}t worden seyn, da{\ss} man ihm nichts in der hize zu trincken geben, hat verschmachten m\"{u}{\ss}en. Es ist eine vanit\"{a}t daz man sagt, man gie{\ss}e auff einen hizigen stein. Der magen verdauet nicht durch hize, sondern gewi{\ss}e sch\"{a}rffe. Trincken ist guth in hizigen kranckheiten, und \edtext{sonsten wieder}{\lemma{sonsten}\Bfootnote{\textit{(1)}\ contra \textit{(2)}\ wieder \textit{L}}} die Galle, diluit sanguinem, und f\"{u}hret die galle durch den Urin weg. Aber sonsten wenn man sich sehr erhizet, durch e\"{u}serliche hize mu{\ss} man nicht darauf trincken. \edtext{Kein}{\lemma{trincken.}\Bfootnote{\textit{(1)}\ Denn \textit{(2)}\ Kein \textit{L}}} thier ist als ein hund, welches sich nicht \"{u}ber trincket. Der hund s\"{a}ufft nicht sondern lecket nur. Macht die zunge hohl, wie ein l\"{o}ffel da{\ss} er damit fa{\ss}et.%
\pend%
\pstart%
Zu Maynz\protect\index{Ortsregister}{Mainz} hatte eine magd \edtext{ein}{\lemma{magd}\Bfootnote{\textit{(1)}\ bey n \textit{(2)}\ ein \textit{L}}} me{\ss}er eingeschlucket, war ein klein rund me{\ss}ergen etwas spizig, sie kam zur frau Canzler Tasterin\protect\index{Namensregister}{\textso{},} \edtext{in Maynz,\protect\index{Ortsregister}{Mainz}}{\lemma{in Maynz}\Bfootnote{\textit{erg. L}}} da{\ss} \edtext{mensch sagte}{\lemma{mensch}\Bfootnote{\textit{(1)}\ stelet \textit{(2)}\ sagte \textit{L}}} niemand was es war, stellete sich wilt, und \edtext{man}{\lemma{man}\Bfootnote{\textit{erg. L}}} meinte sie wurde von sinnen kommen. Endtlich fand sich ein dicker kneutel in der lincken seite, die frau Canzler Tasterin\protect\index{Namensregister}{\textso{},} riethe man solte pechpflaster auflegen, daz ziehen k\"{o}ndte, \textso{denn daz pechpflaster ziehet gewaltig}, dieses zog daz me{\ss}er so gewaltig, da{\ss} es endtlich als man das pflaster abri{\ss} mit der spize herfur zu gucken began, und man \edtext{fuhlete daz etwas}{\lemma{man}\Bfootnote{\textit{(1)}\ die spize als e \textit{(2)}\ fuhlete daz etwas \textit{L}}} hartes und wie eisen, darinn were. Das mensch hatte unterde{\ss}en weil das pechpflaster draufgewesen sehr getobet. \edtext{Meister Johann medicus\protect\index{Namensregister}{\textso{Medicus}, Johann (Barbier in Mainz; 17. Jahrhundert)}}{\lemma{Meister}\Bfootnote{\textit{(1)}\ hans \textit{(2)}\ Johann medicus \textit{L}}} der Barbier zu Meynz, ward geruffen, der brachte die spize mehr und mehr herfur, und zog endtlich das me{\ss}er mit gewalt heraus. Es roch nach verdauter speise, und waren so gar st\"{u}ckgen fleisch dran, er hat das me{\ss}er noch.
Nota Historia \edtext{Cultrivori Borussii\protect\index{Ortsregister}{Preu{\ss}en}}{\lemma{Cultrivori}\Cfootnote{\cite{00008}\textsc{D. Beckher}, \textit{De cultrivoro prussiaco observatio et curatio singularis}, Leiden 1638 (Erstausgabe: K\"{o}nigsberg 1638).}}.
Dort schnitt man das me{\ss}er heraus. Hier war es selbst heraus gangen. Die ganze frag ist, wie es aus dem Sack, oder Canal, der aus Schlund, r\"{o}hre, magen, kleinen und gro{\ss}en gedarme, als in einem st\"{u}ck bestehet, heraus kommen k\"{o}nnen. Es mu{\ss} die natur in ihrer h\"{o}chsten noth kleine falten oder n\"{a}the, die sonst zu \"{o}fnen, wie auch in der geburth geschieht, und solche als denn wieder schlie{\ss}en. Es kam hernach heraus, daz das mensch wohl gewu{\ss}t, daz sie ein me{\ss}er geschluckt, und es dem juden doctor Salomon\protect\index{Namensregister}{\textso{Oppenheim}, Salomon (1640-1697)} zu Francfurt\protect\index{Ortsregister}{Frankfurt/Main} gesagt, welcher ihr oel gegeben, umb zu heilen, wenn es eine wunde in magen gemacht.%
\pend%
\count\Bfootins=1500
\count\Cfootins=1500
\count\Afootins=1500
% Hier endet das Stück.