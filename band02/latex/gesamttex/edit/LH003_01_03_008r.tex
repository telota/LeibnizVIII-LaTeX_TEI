[8~r\textsuperscript{o}]
\pend%
\pstart%
Man soll mittel finden, immer mehr und mehr in das innerste eines lebendigen corpers kommen zu k\"{o}nnen.
Durch einsprizung\protect\index{Sachverzeichnis}{Einspritzen} der Clistire\protect\index{Sachverzeichnis}{Klistier}, und in die r\"{o}hre und in den hals\protect\index{Sachverzeichnis}{Hals}, hat man bereits einige mittel gefunden, item durch das \edtext{phlegmagogum\protect\index{Sachverzeichnis}{phlegmagogus} des circumforanei\protect\index{Sachverzeichnis}{circumforaneus} davon in \textit{Ephemerid. Med.}}%
{\lemma{phlegmagogum [...] \textit{Ephemerid. Med.}}%
\Cfootnote{\cite{01163}\title{Appendix seu Addenda curiosa omissorum ad annum primum Miscellaneorum medico-physicorum Academiae naturae curiosorum}, Breslau 1671, S.~9:
ad Observationem XXXIV in \cite{00001}\title{Miscellanea curiosa physico-medica}, 1 (1670), S.~110-112.}}
%
item durch schneiden des steins, des bruchs, durch stechung des stahrs,
\edtext{Burrhi\protect\index{Namensregister}{\textso{Borri}, Giuseppe Francesco 1627-1695} restitutionem humorum oculi,%
}{\lemma{Burrhi [...] oculi}\Cfootnote{\cite{00001}\title{Miscellanea curiosa physico-medica}, 1 (1670), S.~39 (Observatio XII).\hspace{20mm}}}
% %
% % Marginalie: Anfang
\edtext{endtlich}{\lemma{}\Afootnote{\textit{Auf der rechten Spalte:}
Adde quae Mericus Casaub.\protect\index{Namensregister}{\textso{Casaubon}, M\'{e}ric 1599-1671}
ad Molinaeum\protect\index{Namensregister}{\textso{Du Moulin}, Pierre, der Jüngere 1601-1684}
ubi de perditis quibusdam veterum pag. 28\textsuperscript{[a]} %
sectio lapidis\textsuperscript{[b]} %
renum tempore \protect\index{Namensregister}{\textso{Hippokrates} um 460-um 370 v.Chr.}Hippocratis.
Disjectio\protect\index{Sachverzeichnis}{disiectio} in Empyematibus,\protect\index{Sachverzeichnis}{empyema}
exustio\protect\index{Sachverzeichnis}{exustio} in Cruoris humoribus:
cranii\protect\index{Sachverzeichnis}{cranium} perforatio in aqua cerebri,\protect\index{Sachverzeichnis}{aqua cerebri}
sectio\protect\index{Sachverzeichnis}{sectio} supra oculum\protect\index{Sachverzeichnis}{oculus} in effusionibus,\protect\index{Sachverzeichnis}{effusio} extractio\protect\index{Sachverzeichnis}{extractio} aquae inter cutis restituta sed non ita feliciter.%
% % Marginalienapparat
\vspace{2mm}%
\newline%
\footnotesize
\textsuperscript{[a]} pag. 28: \cite{00020}\textsc{M. Casaubon}, \textit{A letter to Peter du Moulin, concerning natural experimental philosophy}, Cambridge 1669, S. 28.%
\ \quad%
\textsuperscript{[b]} lapidis\ \textit{(1)}\ in vesica\ \textit{(2)}\ renum\ \textit{L}\vspace{-4mm}%
}}
% % Margiunelie: Ende
% %
ofnung der ader\protect\index{Sachverzeichnis}{Ader\"{o}ffnung} und transfusion\protect\index{Sachverzeichnis}{Transfusion}, von dingen so, per stomachum\protect\index{Sachverzeichnis}{stomachus} eingenommen will ich nicht reden.
Nun soll man weiter finden mittel zu langen in den \edtext{Menschen auch wohl gar aufzuschneiden, wie mit dem Cultrivoro}{\lemma{Menschen [...] Cultrivoro}\Cfootnote{\cite{00001}a.a.O., S.~268 (Observatio CXV).}}
%
geschehen. F\"{u}r allen dingen mu{\ss} man ein mittel finden dem menschen einen \edtext{tieffen schlaff zu geben, so ihm nicht schade, darinnen er nichts f\"{u}ele, und daraus man ihn leicht aufwecken k\"{o}nne, als si opio\protect\index{Sachverzeichnis}{opium} crocus\protect\index{Sachverzeichnis}{crocus} aut fortis odor etc. opponatur.}{\lemma{tieffen [...] opponatur}\Cfootnote{\cite{00001}\textit{Miscellanea curiosa medico-physica}, 2 (1671), S.~128f. (Observatio LXIX).}} Als denn mus man solche kunst zu schneiden suchen, da{\ss} man nur partes facile concreturas verleze, und die wieder heilen k\"{o}nnen, wenn der mensch erwachet, salvo ejus necessario motu. Ob nicht ein mittel zu finden den magen\protect\index{Sachverzeichnis}{Magen} leicht a pituita\protect\index{Sachverzeichnis}{pituita} zu reinigen so wohl per artem vomendi\protect\index{Sachverzeichnis}{ars vomendi} cum velis, als per deglutitionem alicujus cum filo annexi quod postea rursus extrahi possit, wie mit den fadten, sed quod stomachum\protect\index{Sachverzeichnis}{stomachus} expurget.
\pend%
\count\Bfootins=1200
\count\Cfootins=1200
\count\Afootins=1200
\pstart%
Omnia mala corporis vel sunt in liquoribus\protect\index{Sachverzeichnis}{liquor} vel in solidis partibus\protect\index{Sachverzeichnis}{partes solidae}. In liquoribus, scilicet aut spiritibus\protect\index{Sachverzeichnis}{spiritus} si qui sunt aut sanguine\protect\index{Sachverzeichnis}{sanguis}. Spiritibus, odore, sanguini\protect\index{Sachverzeichnis}{sanguis} tum aliis modis tum infusioni\protect\index{Sachverzeichnis}{infusio} succurri potest. Sed et cibo potuque bilis\protect\index{Sachverzeichnis}{bilis}, saliva\protect\index{Sachverzeichnis}{saliva} pituita\protect\index{Sachverzeichnis}{pituita}, succus pancreaticus\protect\index{Sachverzeichnis}{succus pancreaticus} augeri minuique
\edtext{proportione. Sunt in liquore}{\lemma{proportione.}\Bfootnote{\textit{(1)} Sed \textit{(2)} Sunt in \textit{(a)} livoris \textit{(b)} liquore \textit{L}}}
vel defectus in ipso, vel 
\edtext{abundantia in ipso, vel motus}{\lemma{abundantia in ipso,}\Bfootnote{\textit{(1)} vel alteratio, \textit{(2)} vel motus \textit{L}}}
in ipso indebitus, vel locus indebitus,
\edtext{vel pondus indebitum}{\lemma{vel pondus indebitum}\Bfootnote{\textit{erg. L}}}
vel extranei in eum interpositio, vel alteratio. Alteratio est dum vel nimis est liquidus, vel nimis $\langle$de$\rangle$nsus, vel nimis calidus, vel nimis
\edtext{frigidus; coloris, odoris, saporis aliqua}{\lemma{frigidus;}\Bfootnote{\textit{(1)} vel \textit{(2)} ab denso et li \textit{(3)} coloris, [...] aliqua \textit{L}}}
mutatio inest. Hinc decet exactissime explorari%
% Hier folgt Bl. 8v.