\begin{ledgroupsized}[r]{120mm}%
\footnotesize%
\pstart%
\noindent%
\textbf{\"{U}berlieferung:}%
\pend%
\end{ledgroupsized}%
\begin{ledgroupsized}[r]{114mm}%
\footnotesize%
\pstart%
\parindent -6mm%
\makebox[6mm][l]{\textit{L}}%
Abschrift einer unbekannten Vorlage:
LH III, 5 Bl. 86-87.
1 Bog. 4\textsuperscript{o}.
3 S. Bl.~87~v\textsuperscript{o} leer.
Bl.~87 im unteren Bereich beschnitten.%
\newline%
Cc 2, Nr. 430%
\pend%
\end{ledgroupsized}%
%
% \normalsize
\vspace{8mm}%
\begin{ledgroup}%
\footnotesize%
\pstart%
\noindent%
\footnotesize{%
\textbf{Datierungsgr\"{u}nde:}
Im vorliegenden Stück N.~73 %?? = LH003,05_086-087
weist Leibniz eingangs darauf hin,
die Methode zur Herstellung und Anwendung des im Text dargestellten Medikaments von \textit{Mons. Memmin} übernommen zu haben.
Ein \textit{Monsieur Memming} wurde Leibniz in einem auf Anfang September 1674 datierbaren Brief von Günther Christoph Schelhammer vorgestellt (\textit{LSB} III,~1 N.~32, S.~123).
Auf Memmings ersten Besuch geht Leibniz in seiner wohl Mitte September 1674 verfassten Antwort an Schelhammer ein (\textit{LSB} III,~5 N.~I, S.~3).
Ein \textit{Mons. Memmin} wird ferner im Stück N.~77 %?? = LH041,02_009 = Aus einem Gespräch mit R. Boyle
erwähnt, das auf die zweite Hälfte Oktober 1676 datiert werden kann (siehe unten, S.~\pageref{LH041,02_009r_Memmin}).
Ebenda ist auch folgende Bemerkung anzutreffen:
\textit{Le Roy dépuis son rétablissement a eu du parlement plus 975259 liures sterlins et extraordinairement accordées, sans l'argent des cheminées et sur les vins.}
Hierauf dürfte sich wohl der Vermerk \textit{9752597 \Pfund\ sterling} beziehen, welcher sich im vorliegenden Stück N.~73 %?? = LH003,05_086-087
am Rand von Bl.~87~r\textsuperscript{o} findet (siehe unten, S.~\pageref{LH003,05_087r_sterling}).
Handelt es sich beim erwähnten \textit{Mons. Memmin} stets um ein und dieselbe Person, so lässt sich daraus folgern,
dass Leibniz vom September 1674 bis zu seinem Londoner Aufenthalt im Oktober 1676 Gelegenheit hatte,
von dem im vorliegenden Stück dargestellten Medikament Kenntnis zu erlangen.
Da weitere Anhalts\-punk\-te für eine genauere chronologische Einordnung fehlen,
wird diese Zeitspanne als Datierung von N.~73 %?? = vorliegendem Stück = LH003,05_086-087
vorgeschlagen.}%
\pend%
\end{ledgroup}%
%
%
\vspace{8mm}%
\count\Bfootins=1500
\count\Cfootins=1200
\count\Afootins=1200
\pstart%
\normalsize%
\noindent%
[86~r\textsuperscript{o}]
\pend%
\pstart%
\noindent%
\centering%
Essence styptique,
\pend%
\pstart%
\noindent%
qui arreste sang d'une artere ou d'une veine coupp\'{e}e et toute autre sorte d'hemorragie, qui guerit aussi facilement et promtement les playes, les ulceres, la gangrene et presque toutes les maladies externes, elle a est\'{e} invent\'{e}e par Mons. de la Rouuiere\protect\index{Namensregister}{\textso{de la Rouuiere}, ???? ??-??} Medecin d'Aix en Provence\protect\index{Ortsregister}{Aix-en-Provence}, et communiqu\'{e}e par Mons. Motbill\protect\index{Namensregister}{\textso{Motbill}, ???? ??-??} Ecossois \`{a} Mons. \label{????}Memmin.\protect\index{Namensregister}{\textso{Memmin}, ???? ??-??.}%\protect\index{Namensregister}{\textso{Nicht ermittelt}, vgl. III, 3 N. 32; III, 5 N. I und II.}
\pend%
\pstart%
\textrecipe. Une liure de bonne et excellente chaux vive, mettez la dans un plat d'argent ou dans un pot de terre, versez sur icelle environ 5 ou 6 liur. d'eau, couurez le pot ou le plat, et laissez infuser cela environ 1. heure sans y toucher. Puis remuez bien le tout, avec un baston de bois, battant et agitant l'eau, durant un moment; apr\`{e}s laissez encor infuser cela pendant 24 heures, pendant les quelles vous remuerez derechef le tout par 2 ou 3 fois. Finalement, vous laisserez bien rassoir la poudre blanche au fonds du pot, et l'eau estant bien claire et nette par dessus, vous la verserez doucement par inclination, sans la 
\pend
\newpage
\pstart \noindent troubler, puis vous renfermerez dans des bouteilles de verre bien \edtext{[bouch\'{e}es]}{\lemma{bouch\'{e}e}\Bfootnote{\textit{L \"{a}ndert Hrsg.}}} pour l'usage suivant. 
\pend%
%\newpage
\pstart%
\textrecipe. 1. liure de cette eau de chaux, mettez la dans une phiole de verre, et y adjoutez 1. dragme et demie de sublim\'{e} corrosif, pill\'{e} et broy\'{e} en poudre subtile; puis agitez et secouez tres bien la phiole, \`{a} fin que la poudre du sublim\'{e} se dissolve, d'abord l'eau deviendra rougeastre, approchante de l'oranger, puis jaunastre et finalement claire et limpide, parce qu'une poudre rougeastre se precipitera aufonds. Toute la poudre s'estant donc rassise, et l'eau s'estant bien clarifi\'{e}e, il la faut separer de la poudre, en la versant doucement par inclination, dans un autre vaisseau de verre, sans la troubler, et dans cette m\^{e}me eau, vous y ajouterez une dragme et demie de bon esprit de vitriol d\'{e}\pgrk{f}legm\'{e} ou jusqu'\`{a} 2 dragmes, selon que l'esprit de \includegraphics[width=0.02\textwidth]{images/vitriol2.pdf} est fort ou foible. Car s'il estoit bien d\'{e}\pgrk{f}legm\'{e}, 1 dragme suffiroit, et en vaudroit m\^{e}me mieux que 2 dragmes d'autre. Aposez encor avec cela \`{a} la dite eau de chaux, une dragme de sel ou de sucre de \saturn: 
\pend%
\pstart%
Tout cela estant dans la \pgrk{f}iole, il la faut remuer et secouer, pour agiter les matieres, et les bien mesler ensemble, laissez reposer ensuite l'eau, jusqu'\`{a} ce qu'elle soit parfaitement claire, apres quoy vuidez la par inclination dans une autre bouteille, et si elle n'est pas bien claire et limpide, filtrez la au travers du papier gris, pour la bien separer d'une poudre blanche, qui sera au fonds, et pour la clarifier \`{a} perfection, cela fait, vostre essence la sera aussi, gardez la seulement dans une bouteille bien ferm\'{e}e, comme un precieux tresor de la sant\'{e}. 
\pend%
\pstart%
Il faut remarquer qu'on peut employer cette essence \`{a} des usages differens; il est necessaire d'en composer de forte, de foible, et de mediocre. La plus forte sera compos\'{e}e de \Pfund. i. d'eau de chaux \includegraphics[width=0.013\textwidth]{images/drachma.pdf}ij de sublim\'{e} corrosif, \includegraphics[width=0.013\textwidth]{images/drachma.pdf}ij de bon esprit de vitriol bien de\pgrk{f}legm\'{e} et de \includegraphics[width=0.013\textwidth]{images/drachma.pdf}i de sucre saturne. Car pour cette derniere drogue, on ne doit jamais l'augmenter, quoy qu'on augmente les autres ou en peut bien diminuer la dose mais non pas l'augmenter, car une dragme de ce sel de \saturn\ suffit tousjours pour \Pfund. i. d'eau. 
\pend%
\pstart%
La mediocre doit estre faite comme nous avons dit cy devant, c'est \`{a} dire, selon les poids. Ces doses que nous avons \edtext{[marquez]}{\lemma{marequez}\Bfootnote{\textit{L \"{a}ndert Hrsg.}}} \`{a} s\c{c}avoir \Pfund. i. d'eau de chaux, \includegraphics[width=0.013\textwidth]{images/drachma.pdf}i\includegraphics[width=0.02\textwidth]{images/semi.pdf} de sublim\'{e} corrosif, autant d'esprit de vitriol et \includegraphics[width=0.013\textwidth]{images/drachma.pdf}i de sel de \saturn. 
\pend%
\pstart%
 La plus foible sera compos\'{e}e de \raisebox{-0.7mm}{\includegraphics[width=0.03\textwidth]{images/semidrachma.pdf}} de sublim\'{e} corrosif \raisebox{-0.7mm}{\includegraphics[width=0.03\textwidth]{images/semidrachma.pdf}} d'esprit de vitriol et \raisebox{-0.7mm}{\includegraphics[width=0.03\textwidth]{images/semidrachma.pdf}} de sucre de \saturn. On garde ces 3 eaux separ\'{e}es dans des bouteilles bien bouch\'{e}es. On se sert de la plus forte aux tres grandes hemorrhoides quand il y a des plus grands vaisseaux ou vases des cuisses coup\'{e}es ou le foye m\^{e}me \edtext{perc\'{e} ou}{\lemma{}\Bfootnote{perc\'{e}\ \textbar\ ne \textit{streicht Hrsg.}\ \textbar\ ou \textit{ L}}}
le sang de quelque grande artere sort en si grande quantit\'{e}, et avec tant d'impetuosit\'{e}, qu'on ne peut l'arrester par l'application des eaux moins fortes. Car par ex. si un homme avoit receu, quelque coup au%
%% Hier folgt Bl. 86v.