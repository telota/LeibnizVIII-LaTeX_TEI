[68~r\textsuperscript{o}]
\pend%
\pstart%
Opium nunquam exuit amarorem si sal volatile et oleosum conjungantur iuste et vi ignis parum urgeantur artificiale constituent amarum, quod aliquos in balsami sulphuris anisati praeparatione non sine damno expertos scio. Nota fortasse omne, inflammabile est acidum volatile, uti omne ex combustione evolans est alcali volatile.
\pend%
\pstart%
Hinc nullum acidum sulphurea dissolvit, nisi addito sale sive fixo sive volatili.
\pend%
\pstart%
Acidum extrahit ex corporibus eorum sal, ut sal fixum ex \saturn\textsuperscript{ni}, extrahit acetum hinc saccarum \saturn\textsuperscript{ni}, sal volatile ex opio.
\pend%
\pstart%
Volatilia sunt, quae ab igne abeunt. Est ergo et pulvis \edtext{aureus}{\lemma{pulvis}\Bfootnote{\textit{(1)}\ pyrius \textit{(2)}\ aureus \textit{L}}} volatilis etsi ab igne descendat, non ascendat.
\pend%
\pstart%
Spiritus vini est acidum volatile, \edtext{acetum}{\lemma{volatile,}\Bfootnote{\textit{(1)}\ spiritus vini \textit{(2)}\ acetum \textit{L}}} est acidum fixum.
\pend%
\pstart%
Sal volatile oleosum seu aromaticum potens remedium ad flatus discutiendos. Id est acidum volatile. An ergo flatus sunt alcali volatile.
\pend%
\pstart%
Matthias \edtext{Paisenius}{\lemma{Paisenius}\Cfootnote{\cite{01173}\textsc{M. Paisen}, \textit{Disputatio de humorum vitiis eorumque restitutione}, Leiden 1666.}}\protect\index{Namensregister}{\textso{Paisen} (Paisenius), Matthias 1643-1670} Hamburgensis \textit{De Humorum vitiis eorumque restitutione}. Diss. inaug. Lugd. Bat.\protect\index{Ortsregister}{Leiden} 1666. 4\textsuperscript{o} Elsevir.
\pend%
\pstart%
Paisenii\protect\index{Namensregister}{\textso{Paisen} (Paisenius), Matthias 1643-1670} propositiones: omnis sal vel acidus, vel lixiviosus, vel ex utroque mixtus. Salem lixivisiosum vocabo, qui lixivium ex cineribus vegetabilium factum sapore refert. Estque vel volatilis vel fixus. Acidus sal est, qui saporis est acidi. Sal ex utroque mixtus est corpus consistens quod ab aqua ita liquatur, ut eam relinquam pellucidam, nec proprio pondere ab illa
%
\edlabel{auszuege1}%
\edtext{}{{\xxref{auszuege1}{auszuege2}}\lemma{sepa-}\Bfootnote{retur. \textit{(1)}\ Tutius dicemus \textit{(2)}\ \lbrack Mihi videtur commode duo \textit{(a)}\ dici \textit{(b)}\ nominari \textit{L}}}%
%
separetur.
\pend%
\pstart%
\edtext{\lbrack Mihi}{\lemma{\lbrack Mihi}\Cfootnote{Eckige Klammer von Leibniz.}}
videtur commode duo
nominari
%
\edlabel{auszuege2}%
%
posse: Alcali et \edtext{Fermentum.\rbrack}{\lemma{Fermentum.\rbrack }\Cfootnote{Eckige Klammer von Leibniz.}}
\pend%
\pstart%
Ex reactione interdum sequitur calor, interdum frigus. Ex reactione interdum sequitur dissolutio, interdum coagulatio.
\pend%
\pstart%
Ex omni effervescentia aliquid elevatur.
\pend
%\newpage
\pstart%
Ex omni effervescentia aliquid praecipitatur. Si superveniat magis acidum vel lixiviosum,
\edtext{liberatur minus}{\lemma{liberatur}\Bfootnote{\textit{(1)} magis \textit{(2)} minus \textit{L}}}
acidum vel lixiviosum.
\pend%
\pstart%
Si quid detineat acidum volatile liberatur alcali volatile et contra.
\pend%
\pstart%
Quaedam volatilia detinentur per quaedam fixa sed propinqua.
\pend%
\pstart%
Si jungantur partes oleosae salinis effervescentibus fit calor.
(+~seu si jungantur acida volatilia.~+)
\pend%
\pstart%
Bilis mihi in multis videtur opio comparanda, est enim inflammabilis simul et amara et continere videtur
\edtext{alcali volatile}{\lemma{alcali}\Bfootnote{\textit{(1)} fixum, \textit{(2)} fixum \textit{(3)} volatile \textit{L}}}
et acidum volatile.%
%
\edtext{}{\lemma{}\Afootnote{\textit{Neben} volatile: \Denarius}}
%
\pend%
\pstart%
Parari potest sal ex bile ungens cum spiritu acido. Spiritus mannae insipidus dissolvit sulphur, erit ergo alcali fixo vicinum, seu inter volatilia valde inferum.
Succus \edtext{Ribium coralliis}{\lemma{Ribium}\Bfootnote{\textit{(1)} cristallis \textit{(2)} coralliis \textit{L}}}
affusus ex grate acido fit austerus. Humores Corporis:
\edtext{sanguis,}{\lemma{}\Afootnote{\textit{Über} sanguis: oleosus}}
succus
\edtext{pancreaticus%
\edtext{,}{\lemma{}\Afootnote{\textit{Über} pancreaticus: acidus}}
Saliva, Lympha.}{\lemma{pancreaticus,}\Bfootnote{%
\textit{(1)} Lym{\protect\pgrk{f}}a 
\textit{(2)} Saliva, Lympha. \textit{L}}}
\pend%
\pstart%
Effervescentia Corporis nostri fit in Corde et Intestino Tenui.
\pend%
\pstart%
Difficile%
\edtext{}{\lemma{}\Afootnote{\textit{Am Rand:}
\protect\index{Namensregister}{\textso{Van Helmont}, Johan Baptista 1580-1644}Helm.\textsuperscript{[a]}
putat es schlug auf alle seiten unten weil es da ist, und gleichsam ungeschmolzen.\\%
\vspace{2mm}
{\footnotesize \textsuperscript{[a]} Helm.: Stelle bei van Helmont nicht nachgewiesen.\vspace{-4mm}}}}
%
rationem reddere cur pulvis pyrius ascendat aurum fulminans descendat.
\pend%
% \newpage% PR: Rein provisorisch !!!
\pstart%
Alcali et acidum
\edtext{differunt vasorum}{\lemma{differunt}\Bfootnote{\textit{(1)}\ bulla \textit{(2)}\ vasorum \textit{L}}}
contentis; volatile et fixum magnitudine adde crassitiem.
\pend%
\newpage
\pstart%
Puto \edtext{pulverem pyrium aureum}{\lemma{pulverem}\Bfootnote{\textit{(1)}\ pyrium \textit{(2)}\ pyrium aureum \textit{L}}}
esse inter fixissima, ideo conari deorsum, jam semel illuc eunte torrente major conatus quam pro
\edtext{gravitate. Videmus}{\lemma{gravitate.}\Bfootnote{\textit{(1)}\ Omni \textit{(2)}\ Videmus \textit{L}}}
etiam volatilia esse vomitoria, fixa esse dejectoria, ut patet ex antimonio quod nondum satis fixum est.
\pend%
%\newpage
\pstart%
Joh. \edtext{Brunquell}{\lemma{Brunquell}\Cfootnote{Vermutlich Johann Heinrich Brunnquell (1656-1710), Lehrer am Gymnasium illustre in Quedlinburg. Keine Schrift ist unter seinem Namen bekannt; möglicherweise spielt Leibniz hier auf eine m\"{u}ndliche Mitteilung an.}}\protect\index{Namensregister}{\textso{Brunquell}, Johann Heinrich 1656-1710} helt sich beym Farner\protect\index{Namensregister}{\textso{Farnner}, Christoph ??-??} auf, wollen unternander gro{\ss}e dinge laborieren. Farner begehrte vom herzog etlich 200 malter dr\"{u}nckel, versprach so guthen Brandte wein daraus zubrennen als der rheinische wurde aber vom herzog deswegen umb etlich 100 thl. gestrafft, weil es nicht angangen.
\pend%
\pstart%
Nimis fixum etiam purgandi vi amissa fit diureticum, ut Asarum notante jam \edtext{Mesue}{\lemma{Mesue}\Cfootnote{\cite{00523}\textsc{J. Mesu\"{e}}, \textit{De re medica libri tres}, Paris 1542, S.~72f.}}\protect\index{Namensregister}{\textso{Mesu\"{e}} Joannis, von Damascus, 8./9. Jd n. Chr.} in substantia et infusione est vomitorium, acrius coctum primum vim vomitoriam, deinde et purgatoriam amittit et fit diureticum.
\pend%
\pstart%
Medicamenta omnia Aromatica et volatilia (+~forte volatilia acida et alcalia~+) si in largiore quantitate vel longiore tempore usurpentur alvum laxant.
\pend%
\pstart%
Petrus \edtext{Pantelius}{\lemma{Pantelius}\Cfootnote{\cite{00414}\textsc{P. Pantelius}, \textit{Disputatio medica de opio, ejus natura, ac vero usu medico}, Leiden 1670.}}\protect\index{Namensregister}{\textso{Pantelius}, Petrus, um 1670} \textit{De opio} sub praesidio Sylvii\protect\index{Namensregister}{\textso{De le Boe}, Frans 1614-1672} Lugd. B. 1670. 4\textsuperscript{o}. Elsevir.
\pend%
\pstart%
Sub praesidio Sylvii\protect\index{Namensregister}{\textso{De le Boe}, Frans 1614-1672} diss. \textit{Chymico-Medica} Martini
\edtext{Carcei}{\lemma{Carcei}\Cfootnote{\cite{00174}\textsc{M. Karczag-Ujsz\'{a}ll\'{a}si}, \textit{Disputatio chymico-medica, de acido praecipue microcosmi}, Leiden 1670.}}\protect\index{Namensregister}{\textso{Karczag-Ujsz\'{a}ll\'{a}si} (Carceus), Martinus, vor 1672} de Carzays-Miszallaha Cumano Ungari 1671. Elsevir. Lugd. B.\protect\index{Ortsregister}{Leiden}
\pend%
\pstart%
\edtext{Palmer}{\lemma{Palmer}\Cfootnote{Als Autor nicht bekannt.}}\protect\index{Namensregister}{\textso{Palmer}, ????}
 treflicher Chirurgus aniezo zu Uytrecht.\protect\index{Ortsregister}{Utrecht}
\pend%
\pstart%
\textit{De vasis Lym\pgrk{f}aticis et Lym\pgrk{f}a} diss. Sylvii\protect\index{Namensregister}{\textso{De le Boe}, Frans 1614-1672} resp. Christo\pgrk{f} \edtext{Gottwald}{\lemma{Gottwald}\Cfootnote{\cite{01175}\textsc{C. Gottwald}, \textit{Disputatio medica de vasis lymphaticis et lympha, sub praesidio Francisci Deleboe Sylvii}, Leiden 1661.}}\protect\index{Namensregister}{\textso{Gottwald}, Christoph 1636 - 1700} Dantiscano Lugd. B.\protect\index{Ortsregister}{Leiden} 1661. 4\textsuperscript{o}. Elsevir.
\pend%
\pstart%
Putat \edtext{Sylvius}{\lemma{Sylvius}\Cfootnote{\cite{01175}\textsc{C. Gottwald}, a.a.O., \S~42; \S~54; \S~61.}}\protect\index{Namensregister}{\textso{De le Boe}, Frans 1614-1672}
diss. \textit{De Lym\pgrk{f}a}, lym\pgrk{f}am oriri ex spiritibus animalibus quae majore copia partibus alterantur, quam ut omnes consumantur superstites reliquas referri per vasa lym\pgrk{f}atica, quae proinde desinunt in venas cerebro proximas, sanguinem rursus redditura \edtext{[spirituosum]}{\lemma{spiritusosum}\Bfootnote{\textit{\ L \"{a}ndert Hrsg.}}} seu novam datura spiritus animalis regenerationem. Diuturnior est vehementiorque motus lym\pgrk{f}ae quam sanguinis, quid aliquot post mortem horis intumescunt ligata vasa lym\pgrk{f}atica non item venae sanguineae.
\pend%
\pstart%
Nota\edtext{}{\lemma{}\Bfootnote{Nota \textbar\ duo \textit{gestr.}\ \textbar\ : Lixivium \textit{L}}}: Lixivium et Fermentum, sunt nominata maxime naturalia coactis illis: acidum et alcali substituta. Lixivium sorbet, Fermentum inflat. Lixivium a lixando, lixare a laciendo, lacere est trahere. Ignis est acidum volatilissimum. Hinc omnia inflammabilia sunt fermenta volatilia.
\pend%
\count\Bfootins=1500
\count\Cfootins=1500
\count\Afootins=1500
%%%% Hier endet das Stück.