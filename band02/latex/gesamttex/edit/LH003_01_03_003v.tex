[3~v\textsuperscript{o}]
\pend%
\pstart%
Die Examinationes simplicium m\"{u}{\ss}en geschehen in dem wir sie erstlich durch alle qualitates sensibiles\protect\index{Sachverzeichnis}{qualitas sensibilis} durchfuhren und bey einer ieden soviel muglich den gradum determiniren.
Als denn m\"{u}{\ss}en wir sie soviel m\"{u}glich per se tractiren durch pressen, percoliren\protect\index{Sachverzeichnis}{perkolieren} etc., distilliren\protect\index{Sachverzeichnis}{destillieren} mit lufft item mit feuer, und dann vermischen mit solventibus\protect\index{Sachverzeichnis}{solventia}, reagentibus\protect\index{Sachverzeichnis}{reagentia}.
Und als denn ebenma{\ss}ig die qualitates omnes
[combinare]\edtext{}{\lemma{combinati}\Bfootnote{\textit{L \"{a}ndert Hrsg.}}}
und deren gradus annotiren.
\pend%
\pstart%
Sonderlich wird operae pretium seyn aller dinge colores ad lapidem lydium\protect\index{Sachverzeichnis}{lapis lydius} ligni Nephritici\protect\index{Sachverzeichnis}{lignum nephriticum} zu probiren.
\pend%
\pstart%
De saporibus\protect\index{Sachverzeichnis}{sapor} mus vor allen dingen ein mittel und weg gefunden werden.
\pend%
\pstart%
Wir m\"{u}{\ss}en suchen ob wir menstrua\protect\index{Sachverzeichnis}{menstruum} finden nur vor dulcia\protect\index{Sachverzeichnis}{dulcia}, oder acida\protect\index{Sachverzeichnis}{acida}, oder salsa\protect\index{Sachverzeichnis}{salsa}, etc. allein, und dadurch auch die gradus zu finden.
\pend%
\pstart%
So mus man auch achtung haben ob denn etwas wahrhaffts aus den signaturis rerum zu nehmen, wo es wahr, wehre es ein illustre documentum providentiae.
\pend%
\pstart%
Man mus in den thieren unzehliche anatomien\protect\index{Sachverzeichnis}{Anatomie} thun, so wohl lebendig als todt.
\pend%
\pstart%
Man mu{\ss} anfangen auf der thiere kranckheiten\protect\index{Sachverzeichnis}{Tierkrankheiten} be{\ss}er acht zu geben als bishehro geschehen, denn gleichwie
\edtext{Steno\protect\index{Namensregister}{\textso{Stensen}, Niels 1638-1686}%
}{\lemma{Steno}\Cfootnote{\cite{01139}\textsc{N. Stensen}, \textit{Discours sur l'anatomie du cerveau}, Paris 1669. S. 53-58.}}
%
recht sagt, da{\ss} wir aus den thieren die ganze anatomiam\protect\index{Sachverzeichnis}{anatomia} hodiernam gelernet, so k\"{o}nten wir auch aus den thieren vollends die pathologiam\protect\index{Sachverzeichnis}{pathologia} lernen, denn wir k\"{o}nnen sie aufschneiden und examiniren wenn und wie wir wolen. Und w\"{u}rde die Republick dem particulier so seyn thier zu gemeinen nuzen hehrgiebt, es bezahlen.
\pend%
\pstart%
Insgemein geben wir fast nur allein auf der pferde\protect\index{Sachverzeichnis}{Pferd} und wenig ander thiere Kranck\-hei\-ten\protect\index{Sachverzeichnis}{Tierkrankheit} acht.
\pend%
\newpage
\pstart%
Wir k\"{o}nnen auch an den thieren die therapeuticam\protect\index{Sachverzeichnis}{therapeutica} leicht und ohne gefahr versuchen sonderlich wenn wir ihre kranckheiten\protect\index{Sachverzeichnis}{Krankheit} be{\ss}er zu erkennen angefangen. An den thieren konnen wir mit arzneyen\protect\index{Sachverzeichnis}{Arznei} proben thun wenn wir wollen, und daraus proportione vom Menschen
\edtext{schlie{\ss}en, an dem Menschen}{\lemma{schlie{\ss}en,}\Bfootnote{\textit{(1)} von dem thier \textit{(2)} an dem Menschen \textit{L}}}
aber nicht.%
% Hier folgt Bl. 4r.