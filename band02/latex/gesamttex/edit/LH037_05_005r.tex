[5~r\textsuperscript{o}] quantumvis magna, in aqua ita librata, ut facillimo nisu huc illuc impelli queat, ictum\protect\index{Sachverzeichnis}{ictus} fortem non recipiat. Acceleratio\protect\index{Sachverzeichnis}{acceleratio} allegari non potest, pone enim arcum attingere corpus projiciendum fine ictus\protect\index{Sachverzeichnis}{ictus}. Res mira, resistentia\protect\index{Sachverzeichnis}{resistentia} nimia obtrudit, mediocris et proportionata adjuvat vim\protect\index{Sachverzeichnis}{vis} impressionis. Nec puto Elaterium facile advocari posse, dicendo si corpus\textso{ diu }resistat comprimi, et se \edtext{restituere. Primum}{\lemma{restituere.}\Bfootnote{\textit{(1)}\ Pone \textit{(2)}\ Primum \textit{L}}} enim eandem projectionis rationem puto fore in non duris. Deinde ponatur hoc, sane nihil aliud inde orietur, \edtext{quam idem}{\lemma{quam}\Bfootnote{\textit{(1)}\ ut corpus in \textit{(2)}\ idem \textit{L}}} quod provenisset sine Elaterio.
\pend
\count\Bfootins=1200
\count\Afootins=1200
\pstart
(1) Sumatur pila\protect\index{Sachverzeichnis}{pila} lignea plumbo\protect\index{Sachverzeichnis}{plumbum} ita temperata, \edtext{ut aquae}{\lemma{ut}\Bfootnote{\textit{(1)}\ in aqua ubi \textit{(2)}\ aquae \textit{L}}} propemodum aequilibret: inde manu in media aqua projiciatur: quae resistentia\protect\index{Sachverzeichnis}{resistentia} sentietur, non erit utique a pondere pilae\protect\index{Sachverzeichnis}{pila}, sed ab aquae attritu, quaestio est an pila\protect\index{Sachverzeichnis}{pila} ita in aqua manu longe projici, aut alteri corpori etiam in aqua posito, ictum\protect\index{Sachverzeichnis}{ictus} fortem inferre \edtext{possit.}{\lemma{}\Afootnote{\textit{\hspace{1.8mm}Am Rand:} Utile est planum esse in fundo aquae super quo propellatur.}} Si non facit; sequitur in aere ideo eandem pilam\protect\index{Sachverzeichnis}{pila} fortem ictum\protect\index{Sachverzeichnis}{ictus} corpori in quod projicitur imprimere, quod projicienti pondere suo magis restitit. Sin vero ictum\protect\index{Sachverzeichnis}{ictus} nihilominus imprimit fortem, sequitur vim\protect\index{Sachverzeichnis}{vis} ictus\protect\index{Sachverzeichnis}{ictus} impressi ab ipsa corporis soliditate pendere. Prius est \edtext{probabilius.}{\lemma{}\Afootnote{\textit{Am Rand:} Imo forte posterius eveniet.\vspace{-8mm}}}
\pend
\pstart
(2) In Tabula polita \edtext{horizontali}{\lemma{}\Bfootnote{horizontali \textit{erg.} \textit{L}}} pila\protect\index{Sachverzeichnis}{pila} manu, vel chorda a pondere aut elaterio tensa (ut magis [constet]\edtext{}{\Bfootnote{constat \textit{\ L \"{a}ndert Hrsg.}}} quae sit \edtext{vis\protect\index{Sachverzeichnis}{vis} projicientis propellatur,}{\lemma{vis}\Bfootnote{%
\textit{(1)}\ ictus imprime \textit{(2)}\ projicientis \textit{(a)}\ impellatur \textit{(b)}\ propellatur, \textit{L}}} ac primum comparatio instituatur inter ferream, v.g. et ligneam ejusdem molis, deinde inter ferream et ligneam ejusdem \edtext{ponderis; quae scilicet sit ictuum,\protect\index{Sachverzeichnis}{ictus} [quos] sub}%
{\lemma{ponderis;}\Bfootnote{%
\textit{(1)}\ an scilicet ictus %
\textit{(2)}\ quae scilicet sit ictuum, %
\textbar\ quem \textit{ändert Hrsg.} \textbar\ %
\textit{(a)}\ in %
\textit{(b)}\ sub \textit{L}}} tabulae exitum in corpus objectum, (v.g. pondus, aqua extrahendum) exercent ratio. Credibile est, eundem fore ictum\protect\index{Sachverzeichnis}{ictus} a pilis\protect\index{Sachverzeichnis}{pila} ponderis aequalis. Quod si evenit priori experientiae consonat.
\pend
\count\Bfootins=1200
\pstart
(3) Ut rei reddamur certiores, experimentum cum rebus sua natura levibus, ut globo ligneo, sed qui glutine aliquo forti tabulae adhaereat; et nunc chorda eadem, impellatur. Si jam verum est resistentiam\protect\index{Sachverzeichnis}{resistentia} corporis projecti ad projectionem, esse causam ictus\protect\index{Sachverzeichnis}{ictus} a projecto majoris[,] sequitur tantum, imo multo majorem inferri ictum\protect\index{Sachverzeichnis}{ictus} a globo isto [ligneo]\edtext{}{\Bfootnote{ligno \textit{\ L \"{a}ndert Hrsg.}}}, quam a plumbeo libero, quia colla, sive gluten effecit, ut difficilius impelli potuerit globus ligneus, quam impulsus fuisset plumbeus. Et cum credibile sit ictum\protect\index{Sachverzeichnis}{ictus} non fore tantum; \edtext{hinc eo posito sequeretur}{\lemma{hinc}\Bfootnote{\textit{(1)}\ sequitur \textit{(2)}\ eo posito sequeretur \textit{L}}}, nec resistentiam\protect\index{Sachverzeichnis}{resistentia} ad motum esse causam fortis ictus\protect\index{Sachverzeichnis}{ictus} a projectis \edtext{impressi. At}{\lemma{impressi.}\Bfootnote{\textit{(1)}\ At \textit{(2)}\  Quid vero si nihilomin \textit{(3)}\ At \textit{L}}} hoc experimentum difficulter conciliabitur cum primo, \edtext{ubi credo}{\lemma{ubi}\Bfootnote{\textit{(1)}\ suppono \textit{(2)}\ credo \textit{L}}} in aqua ictum\protect\index{Sachverzeichnis}{ictus} fortem non inferri, ab eo quod in aqua parum ponderat. Conciliando utrumque 
\noindent dicendum non resistentiam\protect\index{Sachverzeichnis}{resistentia} ad \edtext{primum impressum}{\lemma{primum}\Bfootnote{\textit{(1)}\ ictum\protect\index{Sachverzeichnis}{ictus} \textit{(2)}\ impressum, \textit{L}}}, \edtext{sed ponderationem}{\lemma{sed}\Bfootnote{\textit{(1)}\ pondus \textit{(2)}\ ponderationem \textit{L}}} seu gravitationem\protect\index{Sachverzeichnis}{gravitatio} 
esse causam ictus\protect\index{Sachverzeichnis}{ictus} fortis. At cur ita ponderatio? Nonne quia resistit ad ictum\protect\index{Sachverzeichnis}{ictus} imprimentis? Aut alia ratio comminiscenda, aut aliter experimenta evenire necesse est. An dicendum \edtext{foret}{\lemma{}\Bfootnote{foret \ \textit{erg.} \textit{L}}} corpus in tantum recipere ictum\protect\index{Sachverzeichnis}{ictus}, in quantum jam \edtext{movetur,}{\lemma{}\Bfootnote{movetur,  \textbar\ et \textit{gestr.}\ \textbar\ resistentiam \textit{L}}} resistentiam\protect\index{Sachverzeichnis}{resistentia} a glutine non esse motum, resistentiam\protect\index{Sachverzeichnis}{resistentia} a pondere esse. Sed hujus hypotheseos difficile foret rationem reddere nisi diceremus unumquodque in tantum agere ac pati in quantum est. Esse autem in quantum agit. Idem glutinis experimentum corpori in aqua librato adhiberi potest, ut appareat an prima ad ictum\protect\index{Sachverzeichnis}{ictus} resistentia\protect\index{Sachverzeichnis}{resistentia} horum phaenomenorum causa sit. Cera in Sclopetis\protect\index{Sachverzeichnis}{sclopetum} hunc habet usum, \edtext{quod longitudinem}{\lemma{quod}\Bfootnote{\textit{(1)}\ idem ad longitu \textit{(2)}\ longitudinem \textit{L}}} Sclopeti\protect\index{Sachverzeichnis}{sclopetum} supplet.
\pend
\count\Bfootins=1200
\pstart
\edtext{(4) Faciendum est experimentum displosionis cum cera in arcu, secund. experim. 3. ictus non foret fortior, si scilicet prima resistentia non esset causa fortis ictus}{\lemma{(4)}\Bfootnote{\textit{(1)}\ Si experimenta docerent vim\protect\index{Sachverzeichnis}{vis} projectionis, a   \textbar\ prima \textit{erg.}\ \textbar\  resistentia\protect\index{Sachverzeichnis}{resistentia} projecti esse, res ita explicari posset, quod scilicet difficultas \textit{(2)}\ Faciendum [...] arcu, \textit{(a)}\ credo \textit{(b)}\ secund. [...] resistentia \textit{(aa)}\ nihil \textit{(bb)}\ non [...] fortis ictus. \textit{L}}}. At idem fiat in Sclopeto\protect\index{Sachverzeichnis}{sclopetum}, ratio% \pend