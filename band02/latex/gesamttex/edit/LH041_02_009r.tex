\begin{ledgroupsized}[r]{120mm}%
\footnotesize%
\pstart%
\noindent\textbf{\"{U}berlieferung:}%
\pend%
\end{ledgroupsized}%
\begin{ledgroupsized}[r]{114mm}%
\footnotesize%
\pstart%
\parindent -6mm%
\makebox[6mm][l]{\textit{L}}%
Aufzeichnung:
LH~XLI~2 Bl.~9.
1~Bl.~2\textsuperscript{o}. 1~S. auf Bl.~9~r\textsuperscript{o}.
Bl.~9 v\textsuperscript{o} leer.%
\newline%
Cc 2, Nr. 00%
\pend%
\end{ledgroupsized}%
%
% \normalsize
\vspace*{5mm}%
\begin{ledgroup}%
\footnotesize%
\pstart%
\noindent%
\footnotesize{%
\textbf{Datierungsgr\"{u}nde:}
Eine Unterredung mit Boyle kann nur w\"{a}hrend eines Aufenthalts Leibniz' in Eng\-land stattgefunden haben.
Das zu Beginn der Aufzeichnung erw\"{a}hnte Modell des Salomon-Tempels von Jacob Judah Leon und Adam Boreel wurde 1675 in London ausgestellt.
(Siehe \cite{01150}\textsc{A.~Balfour}, \textit{Solomon's Tempel: Myth, Conflict, and Faith}, Chichester 2015, S.~199f.)
Daraus ergibt sich eine Datierung der Unterredung sowie der Aufzeichnung auf die zweite Hälfte Oktober 1676,
als Leibniz sich zum letzten Mal in London aufgehalten hat.}%
\pend%
\end{ledgroup}%
%
%
\vspace*{8mm}%
\count\Bfootins=1200
\count\Cfootins=1200
\count\Afootins=1200
\pstart%
\normalsize%
\noindent%
[9~r\textsuperscript{o}]
Templum Salomonis\protect\index{Sachverzeichnis}{templum Salomonis} cura Leonis Judae,\protect\index{Namenverzeichnis}{\textso{Leon}, Jacob Judah 1602-1675} Judaei Amstelodamensis et
\edtext{Petri Borelli}{\lemma{Petri Borelli}\Cfootnote{Siehe eigentlich \cite{01018}\textsc{A. Boreel}, \textit{Ad legem}, o.O. 1645.
Leibniz verwechselt an dieser Stelle Adam Boreel mit Pierre Borel, Mediziner und Naturforscher.}}%
\protect\index{Namenverzeichnis}{\textso{Borel}, Pierre 1620-1689}\protect\index{Namenverzeichnis}{\textso{Boreel}, Adam 1603-1667}
(autoris libri ad legem et testimonium)
exactissime confectum visitur Londini in vico\protect\index{Ortsregister}{London} \dots
\pend%
\pstart%
Mons. Boyle\protect\index{Namensregister}{\textso{Boyle} (Boylius), Robert 1627-1691} m'a cont\'{e} bien des choses de Mons. Greatrick.\protect\index{Namensregister}{\textso{Greatrakes} (Greatrick), Valentine 1629-1683}
Il l'a connu fort particulierement et il a est\'{e} present \`{a} plusieurs de
\edtext{ses impositions de main.}{\lemma{ses}\Bfootnote{\textit{(1)} experiences \textit{(2)} impositions de main. \textit{L}}}
Les \edtext{douleurs fuyoient un}{\lemma{douleurs}\Bfootnote{\textit{(1)} le fuyoient; \textit{(2)} fuyoient \textit{(a)} quand il s \textit{(b)} le \textit{(c)} un \textit{L}}}
attouchement redoubl\'{e} assez leger; bien souuent il touchoit jusqu' \`{a} la 4\textsuperscript{me} fois avant que de chasser le mal. Ayant un mal de teste, il se guerit luy m\^{e}me par l'attouchement. Il a fait en 2 mois une infinit\^{e} d'experiences \`{a} Londres.
Mylord Brounker\protect\index{Namensregister}{\textso{Brouncker} (Brounker, Brunckerus), William 1620-1684}
doutant du succ\`{e}s, et s'en mocquant, fut convaincu lorsque Mons. Boyle\protect\index{Namensregister}{\textso{Boyle} (Boylius), Robert 1627-1691} l'y
\edtext{mena, et}{\lemma{mena,}\Bfootnote{\textit{(1)} une femme de qualit\'{e} vint dans \textit{(2)} et \textit{L}}}
Mons. Brounker\protect\index{Namensregister}{\textso{Brouncker} (Brounker, Brunckerus), William 1620-1684} le prit dans sa maison o\`{u} il resta pendant qu'il fut \`{a} Londres. Une fois estant chez Mons. Boyle\protect\index{Namensregister}{\textso{Boyle} (Boylius), Robert 1627-1691} une femme de qualit\'{e} l'ayant s\c{c}eu y vint et le pria de la toucher. Il la toucha jusqu' \`{a} la quatrieme fois avant que de chasser son mal de teste, qui se retira
\edtext{dans l'epaule droite,}{\lemma{dans}\Bfootnote{\textit{(1)} le bras. \textit{(2)} l'epaule droite, \textit{L}}}
de l\`{a} dans
\edtext{l'eelbogen\protect\index{Sachverzeichnis}{eelbogen}, mais le}{\lemma{l'eelbogen,}\Bfootnote{\textit{(1)} et \textit{(2)} mais \textit{(a)}\ il \textit{(b)}\ le \textit{L}}}
mal ne sortit pas comme il faisoit \`{a} l'ordinaire par les doits. Il luy \'{e}chappa et
\edtext{retourna vers}{\lemma{retourna}\Bfootnote{\textit{(1)} dans \textit{(2)} vers \textit{L}}}
\edtext{l'epaule, la}{\lemma{l'epaule,}\Bfootnote{\textit{(1)} il \textit{(2)} la \textit{L}}}
femme ne sentit plus de mal.
Mais Mons. Greatrik\protect\index{Namensregister}{\textso{Greatrakes} (Greatrick), Valentine 1629-1683} dit \`{a} Mons. Boyle\protect\index{Namensregister}{\textso{Boyle} (Boylius), Robert 1627-1691}, qu'elle n'estoit pas bien guerie, que le mal n'estoit pas sorti, qu'il s' estoit cach\'{e}. Mons. Boyle\protect\index{Namensregister}{\textso{Boyle} (Boylius), Robert 1627-1691} asseure, que la sudeur\protect\index{Sachverzeichnis}{sudeur} de Mons. Greatrick\protect\index{Namensregister}{\textso{Greatrakes} (Greatrick), Valentine 1629-1683} sentoit bien. Mons. Boyle\protect\index{Namensregister}{\textso{Boyle} (Boylius), Robert 1627-1691} est fort en peine pour decider si c'estoit un effect naturel, ou surnaturel. La difficult\'{e} est \`{a} l'egard de la maniere dont il s'est apperceu de sa vertu. Il asseure, que c'estoit par une inspiration. C'est un homme fort croyable, sans interest, car il a de quoy vivre caute. Il ne prend rien. Il est sans vanit\'{e}, et sans ambition. Intra annum abhinc
\edtext{scripsit se}{\lemma{}\Bfootnote{scripsit\ \textbar\ (Boylio) \textit{gestr.}\ \textbar\ se \textit{ L}}}
nuper fecisse curam tanti momenti et difficultatis quanta aliqua priorum.
\edtext{Galaeus}{\lemma{Galaeus}\Cfootnote{\cite{00495}\textsc{T. Gale},\protect\index{Namensregister}{\textso{Gale} (Galaeus), Theophilus 1628–1678} \textit{Court of gentiles}, 4 Bde., London und Oxford 1667-1671.}}
\edtext{Anglus}{\lemma{Anglus}\Bfootnote{\textit{erg. L}}}
in libro \textit{Court of \dots} multis ostendere conatur philoso\pgrk{f}iam pythagoream\protect\index{Sachverzeichnis}{philosophia Pythagorea} et aliam veterem esse ex Moyse\protect\index{Namensregister}{\textso{Moses} (Moyse)} et Judaeis mediate aut immediate.
\pend%
\pstart%
\edtext{Pokokius\edtext{}{\lemma{Pokokius}\Cfootnote{\textsc{\cite{00496}E.~Pocock}, \textit{Commentarius in prophetiam Joelis}, Leipzig 1695, Vorrede.}}\protect\index{Namensregister}{\textso{Pockocke} (Pokokius), Edward 1604-1691}
in quibusdam locis}{\lemma{Pokokius}\Bfootnote{\textit{(1)} sparsim \textit{(2)} in quibusdam locis \textit{L}}}
notionem illam prosecutus est, quod significationes
\edtext{quaedam vocabulorum}{\lemma{quaedam}\Bfootnote{\textit{(1)} locorum \textit{(2)} vocabulorum \textit{L}}}
Scripturae sacrae veteres et
\edtext{perditae sed}{\lemma{perditae}\Bfootnote{\textit{(1)} ad \textit{(2)} et \textit{(3)} sed \textit{L}}}
illo tempore adhuc vigentes quo 70 scribebant adhuc in Arabico extent nonnunquam.
Idque exemplis selectis illustrat.
\pend%
\pstart%
\edtext{Hammondi}{\lemma{Hammondi}\Cfootnote{\cite{00497}\textsc{H. Hammond}, \textit{Paraphrase of the New Testament}, London 1653.}}\protect\index{Namensregister}{\textso{Hammond} (Hammondus), Henry 1605–1660} notae probae
\edlabel{great1}%
in N.T.%
\edtext{}{{\xxref{great1}{great2}}\lemma{N.T.}\Bfootnote{\textit{(1)} Diamant \textit{(2)} Diamas \textit{L}}}%
\pend%
\pstart%
Diamas%
\edlabel{great2}
que Mons. Boyle\protect\index{Namensregister}{\textso{Boyle} (Boylius), Robert 1627-1691} me fit
\edtext{voir qui est comme per}{\lemma{voir}\Bfootnote{\textit{(1)} per \textit{(2)} qui est comme per \textit{L}}}
laminas stratificatas sibi superpositas, quarum commissurae apparent lineolis quibusdam parallelis in ejus superficie dictis.
\pend%
\pstart%
Smaragdus pulcherrima Boylii\protect\index{Namensregister}{\textso{Boyle} (Boylius), Robert 1627-1691}, Electrica\protect\index{Sachverzeichnis}{electrica} est, quicquid contra dicant autores veteres et recentiores, quod smaragdus non sit Electrica\protect\index{Sachverzeichnis}{electrica}.
\pend%
\pstart%
Vossius\protect\index{Namensregister}{\textso{Vossius} (Voss), Isaac 1618-1689} unus omnium mortalium maxime est in Geogra\pgrk{f}ia\protect\index{Sachverzeichnis}{geographia} versatus, perlegit omnes chartas societatis ostindicae et inde ad rem facientia excerpsit. Ejus Bibliotheca tunc tandem allata. Boylius\protect\index{Namensregister}{\textso{Boyle} (Boylius), Robert 1627-1691} eum urgebit ut in Geogra\pgrk{f}ia\protect\index{Sachverzeichnis}{geographia} laboret. Vidit apud eum descriptionem Batava lingua\protect\index{Sachverzeichnis}{lingua Batava} editam regni Matarum\protect\index{Sachverzeichnis}{regnus Matarum}, quod est in aversa Javae\protect\index{Ortsverzeichnis}{Java} parte, catena montium separata ab Hollandia.\protect\index{Ortsverzeichnis}{Holland (Hollandia)}
Rex est potens, idololatra, habet praetorianam \edtext{[militiam]}{\lemma{militeriam}\Bfootnote{\textit{L \"{a}ndert Hrsg.}}} ultra 80,000 hominum. Non antea quisquam id descripsit. Boylius\protect\index{Namensregister}{\textso{Boyle} (Boylius), Robert 1627-1691} habet descriptionem
\edtext{impressam}{\lemma{impressam}\Bfootnote{\textit{erg. L}}}
\edtext{de Iezzo ultra Japonicam.}{\lemma{de}\Bfootnote{\textit{(1)} Ieddo \textit{(2)} Iezzo \textit{(a)}\ Japonensi \textit{(b)}\ ultra Japonicam. \textit{L}}}
Imperator Japoniae\protect\index{Ortsverzeichnis}{Japan (Japonia)} interdixit Hollandis navigatione in Iezzo.
\pend
\newpage
\pstart%
Balsamus sulphuris\protect\index{Sachverzeichnis}{balsamus sulphuris}
\edtext{Boylii}{\lemma{Boylii}\Cfootnote{\cite{00155}\textsc{R. Boyle}, \textit{The usefulnesse of experimental natural philosophy}, Oxford 1671, Teil II, S. 156f.% (\textit{BW} III, ??)
}}\protect\index{Namensregister}{\textso{Boyle} (Boylius), Robert 1627-1691}
in lib. de utilitate descriptus egregius pro omnis generis vulneribus et ulceribus.
Boylius\protect\index{Namensregister}{\textso{Boyle} (Boylius), Robert 1627-1691} semper eum fert secum.
Dolores capitis solo saepe odore sanat Boylius\protect\index{Namensregister}{\textso{Boyle} (Boylius), Robert 1627-1691}, ut spiritus salis. Retinet peregrinator Anglus qui interiora superioris Aegypti\protect\index{Ortsverzeichnis}{Aegypten (Aegyptus)} vidit, Boylio\protect\index{Namensregister}{\textso{Boyle} (Boylius), Robert 1627-1691}, Turcas milites dicere publice se
\edtext{facile iugum}{\lemma{facile}\Bfootnote{\textit{(1)} se \textit{(2)} iugum \textit{L}}}
Turcarum excutere sed metu intestinarum seditionum non fecisse, si quis Bassa amaretur et potens esset, facile se dominum ferret. Sed Bassas et omnino Turcas odere Aegyptii memoriam adhuc Mammeluccorum\protect\index{Sachverzeichnis}{Mamelucci} suorum in veneratione habentes, qui erant longe politiores Turcis.
\pend%
\pstart%
Peregrinator Anglus qui in Siam\protect\index{Ortsverzeichnis}{Siam} fuit, dixit Boylio\protect\index{Namensregister}{\textso{Boyle} (Boylius), Robert 1627-1691} se vidisse ibi tormentum bellicum 600 abhinc annis factum inscriptione Arabica.
\pend%
\pstart%
Mons.\edlabel{LH041,02_009r_Rabel-1}
Rabel\protect\index{Namensregister}{\textso{Rabel} (Rabelius) 17. Jh.} egregia praestitit. Quendam hominem transfossum ipsis pulmonibus laesis (nam sanguinem spuebat), suo liquore ita sanavit ut homo tridui post in Withe\-hal\protect\index{Ortsverzeichnis}{Whitehall} ierit. Nunc alterum dat liquorem internum qui discutit et sanguinem grumosum. Oldenburgius\protect\index{Namensregister}{\textso{Oldenburg}, Heinrich 1618?-1677} egregium sensit usum \protect\index{Namensregister}{\textso{Rabel} (Rabelius) 17. Jh.}Rabeliani liquoris contra Scorbutum\protect\index{Sachverzeichnis}{scorbutus} et dolores gingivarum.\edlabel{LH041,02_009r_Rabel-2}
\pend%
\pstart%
Rex et princeps Robertus\protect\index{Namensregister}{\textso{Pfalz-Simmern: Ruprecht}, Pfalzgraf (princeps Robertus) 1619 - 1682)} habent ejus secretum
%%
\edtext{[qui]}{\lemma{qui}\Bfootnote{\textit{erg. Hrsg.}}}
%%
uno caret oculo, quod perdidit cum in oriente, ut
\edtext{narrat, cuidam}{\lemma{narrat,}\Bfootnote{\textit{(1)} Judaeo \textit{(2)} cuidam \textit{L}}}
homini male habito assisteret, in eo tumultu oculum
\edtext{perdidit. Homo}{\lemma{perdidit.}\Bfootnote{\textit{(1)} In \textit{(2)} Homo \textit{L}}}
qui erat Judaeus vulnus ait se sanare posse[,]
 oculum reddere non posse. Grati animi testem dedit medicinae ipsius praeparationem. Duos habet tantum liquores quibus omnia praestat. Fortissimus est internus, sufficiunt 5 guttae. Alter non ita penetrans. Gustus non malus.
\pend%
\pstart%
Eau \edtext{claire}{\lemma{claire}\Bfootnote{\textit{erg.} \textit{L}}}
de Mons. Boyle\protect\index{Namensregister}{\textso{Boyle} (Boylius), Robert 1627-1691}
qui devient trouble par l'instillation de quelques gouttes, et reprend sa nettet\'{e}, par l'instillation de quelques
\edtext{autres; sans}{\lemma{autres;}\Bfootnote{\textit{(1)} pour p \textit{(2)} sans \textit{L}}}
qu'il tombe aucun sediment au fonds: pour prouuer que le sang pourroit estre \'{e}pur\'{e},
\edtext{sans saign\'{e}e,}{\lemma{sans}\Bfootnote{\textit{(1)} venesectio \textit{(2)} saign\'{e}e, \textit{L}}}
et m\^{e}me sans precipitation.
\pend%
\pstart%
\edtext{Briggs}{\lemma{Briggs}\Cfootnote{\cite{01019}\textsc{W. Briggs}, \textit{Opthalmographia}, Cambridge 1676.}} \textit{ophthalmogra\pgrk{f}ia}.\protect\index{Namensregister}{\textso{Briggs}, William 1642-1704}
\pend%
\pstart%
Willisii\protect\index{Namensregister}{\textso{Willis}, Thomas 1621-1675} praeparatio croci martis\protect\index{Sachverzeichnis}{mars (Eisen)} sine menstruo\protect\index{Sachverzeichnis}{menstruum} lento igne, et clauso.
\pend%
\pstart%
Mons. de\protect\index{Namensregister}{\textso{Verret}, ???? ??-??}
\edtext{Verret ayant prepar\'{e}}{\lemma{Verret}\Bfootnote{\textit{(1)} parla p \textit{(2)} ayant prepar\'{e} \textit{L}}}
les intestins avoit trouv\'{e}, que
\edtext{[leurs]}{\lemma{leur}\Bfootnote{\textit{L \"{a}ndert Hrsg.}}} fibres es\-toi\-ent non circulaires mais spirales[,] ce qu'il pretendoit servir \`{a} l'explication du mouuement peristaltique\protect\index{Sachverzeichnis}{mouvement peristaltique}; depuis un Anglois a donn\'{e} la m\^{e}me chose \`{a} la societ\'{e}.
\pend
\newpage
\pstart%
Servic de monstre des passions singuliere, en ce qu'il revient tousjours la m\^{e}me chose \`{a} une m\^{e}me personne, est que la monstre est sur un gueridon portatif, qu'on a port\'{e} d'une place de la chambre \`{a} l'autre. Un anneau est mis dans du bois sans qu'il paroisse aucune soudure;
la conjuncture de Mons. \label{LH041,02_009r_Memmin}Memmin\protect\index{Namensregister}{\textso{Memmin}, ???? ??-??}
est ingenieuse, s\c{c}avoir que l'anneau a est\'{e} comme ans\'{e} dans du bois d'un arbre, qui en croissant a enferm\'{e} \edtext{[l']}{\lemma{l'}\Bfootnote{\textit{erg. Hrsg.}}}anneau; et puis comme est sech\'{e} s'est retir\'{e}, et l'anneau a la libert\'{e} de couler.
\pend%
\pstart%
Mons. Boyle\protect\index{Namensregister}{\textso{Boyle} (Boylius), Robert 1627-1691} me
\edtext{montra un morceau de bois petrifi\'{e}}{\lemma{montra}\Bfootnote{\textit{(1)} un diamant par \textit{(2)} un [...] petrifi\'{e} \textit{ L}}}
entierement par dedans et par dehors dans le lac d'Irlande\protect\index{Ortsverzeichnis}{Irland (Irlande)}, dont on fait tant de contes[;] ce lac a jusqu' \`{a} 30 milles d'Italie de longueur.
\pend%
\pstart%
Le Roy\protect\index{Namensregister}{\textso{England: Karl II.} 1649-1685} d\'{e}puis son r\'{e}tablissement a eu du parlement
\edtext{plus 975259}{\lemma{plus}\Bfootnote{\textit{(1)} de \textit{(2)} 975259 \textit{L}}}
liures sterlins et extraordinairement \edtext{[accordées],}{\lemma{accordés}\Bfootnote{\textit{L ändert Hrsg.}}} sans l'argent des chemin\'{e}es et sur les vins.
\pend%
\pstart%
Le parlement a est\'{e} tousjours le m\^{e}me. Ce
n'a pas est\'{e} un nouueau parlement mais seulement une prorogation du premier, ainsi les m\^{e}mes personnes sont demeur\'{e}es.
\pend%
\count\Bfootins=1500
\count\Cfootins=1500
\count\Afootins=1500
%%%% PR: Hier endet das Stück.