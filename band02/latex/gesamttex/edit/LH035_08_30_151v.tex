\count\Afootins=1200
\count\Bfootins=1200
\count\Cfootins=1200
\pstart%
\noindent\hangindent=7,5mm%
% [151~v\textsuperscript{o}]
\edtext{}{\lemma{}\Afootnote{\textit{Am oberen Rand:}
De causa circuli Iridis, de Dominis et des-Cartes\protect\index{Namensregister}{\textso{Descartes} (Cartesius, des Cartes), Ren\'{e} 1596-1650}\textsuperscript{[a]}.
De paraheliis et paraselenis, idem\textsuperscript{[b]}
et Hugenius\protect\index{Namensregister}{\textso{Huygens} (Hugenius, Ugenius, Hugens, Huguens), Christiaan 1629-1695}\textsuperscript{[c]}.\vspace{0.5mm}\\
%
\footnotesize{%
\textsuperscript{[a]}\ des-Cartes: \cite{01106}\textsc{R. Descartes}, \textit{Les météores}, Leiden 1637, S. 250-271 (\cite{00120}\textit{DO} VI, S. 325-344).%
\hspace{5mm}%
\textsuperscript{[b]}\ idem: % \textsc{R. Descartes}, \textit{Dissertatio de methodo}, Amsterdam 1672, 
\cite{01106}a.a.O., S. 281-294 (\cite{00120}\textit{DO} VI, S. 354-366).%
\hspace{5mm}%
\textsuperscript{[c]}\ Hugenius: \textsc{C. Huygens}, \cite{01158}\textit{Relation d'une observation faite à la Bibliotheque du Roy}, Paris 1667 (s. \cite{00113}\textit{HO} XVII, S. 498, Anm. 3).
%Hugenius: \textsc{C. Huygens}, \cite{00510}\textit{Dissertatio de coronis et parheliis}, in \textit{Opera reliqua}, Bd. II, Amsterdam 1728, S. 3-54 (\cite{00113}\textit{HO} XVII, S. 351-516).
%Obwohl Huygens' Schrift erst posthum veröffentlicht wurde, ist es durchaus möglich, dass Leibniz sie dank seiner engen Beziehung zu Huygens bereits zur Pariser Zeit kannte.
\vspace{-8mm}}}}%
De figura projectorum parabolica et des jets d'eau.\\
%\pend
%\pstart%
%\noindent%
%\hangindent=7,5mm%
\hspace*{-7.5mm}De linea sexti gradus
\edtext{admonente Mariotto\protect\index{Namensregister}{\textso{Mariotte}, Edme, Seigneur de Chazeuil ca. 1620-1684}}{\lemma{admonente Mariotto}\Cfootnote{Möglicherweise eine m\"{u}ndliche Mitteilung Mariottes.}}
in jactibus aquarum observata.
\edtext{De verticibus. De vento in turbinem eunte.}{\lemma{}\Bfootnote{De verticibus [...] eunte. \textit{erg. L}}}\\
%\pend
%\pstart%
%\noindent%
%\hangindent=7,5mm%
\hspace*{-7.5mm}De Nodis.
Deque forma nodi cujusdam Gordii.
\pend
%\newpage
\pstart%
\noindent%
\hangindent=7,5mm%
De complicatione literarum, deque modo ita complicandi, ut difficile sit aperire ignoranti, sine ullo sigillo.
\edtext{De figura arcuum tensorum, et displosorum.}{\lemma{De figura [...] displosorum.}\Bfootnote{\textit{erg. L}}}\\
%\pend
%\pstart%
%\noindent%
%\hangindent=7,5mm%
\hspace*{-7.5mm}De complicationibus des Serviettes, et des figures produites par ce moyen.\\
%\pend
%\pstart%
%\noindent%
%\hangindent=7,5mm%
\hspace*{-7.5mm}De figuris chartae incisis non inelegantibus quae visuntur in pyxidibus Apothecariorum Germaniae.\\
%\pend
%\pstart%
%\noindent%
%\hangindent=7,5mm%
\hspace*{-7.5mm}Des points de France.
De Venise\protect\index{Ortsregister}{Venice}, de Paris\protect\index{Ortsregister}{Paris}.
De la Reine, de Colbert,\protect\index{Namensregister}{\textso{Colbert}, Jean-Baptiste 1619-1683}
etc.\\
%\pend
%\pstart%
%\noindent%
%\hangindent=7,5mm%
\hspace*{-7.5mm}Des poincts de Londres\protect\index{Ortsregister}{London}, faits de papier, par un instrument.\\
%\pend
%\pstart%
%\noindent%
%\hangindent=7,5mm%
\hspace*{-7.5mm}De Schn\"{u}ren:
toutes sortes de figures, par le moyen des perles, coralles, etc. enfil\'{e}es, usit\'{e}es en Allemagne.\\
%\pend
%\pstart%
%\noindent%
%\hangindent=7,5mm%
\hspace*{-7.5mm}Quibus modis fit ut appareant aquilae in charta aliaeque formae.
\pend
\count\Afootins=1200
\count\Bfootins=1000
\count\Cfootins=1000
\pstart%
\noindent%
\hangindent=7,5mm%
%\hspace*{-7.5mm}
Quomodo in Germania, Noribergae\protect\index{Ortsregister}{N\"{u}rnberg}
inprimis vitris ad bibendum destinatis incidantur figurae, excavatae, quodam torni genere.\\
%\pend
%\pstart%
%\noindent%
%\hangindent=7,5mm%
\hspace*{-7.5mm}De politura Adamantum, et aliorum lapidum, en pointes, etc.\\
%\pend
%\pstart%
%\noindent%
%\hangindent=7,5mm%
\hspace*{-7.5mm}De Sculptoribus Sigillorum et Typorum.\\
%\pend
%\pstart%
%\noindent%
%\hangindent=7,5mm%
\hspace*{-7.5mm}De Eminentiis, bas relief, inverso excavationis.\\
%\pend
%\pstart%
%\noindent%
%\hangindent=7,5mm%
\hspace*{-7.5mm}\edtext{\textit{De Stereometria}}{\lemma{\textit{De}}\Bfootnote{\textit{(1)}\ Geometria \textit{(2)}\ \textit{Stereometria} \textit{L}}}
\edtext{\textit{doliorum},}{\lemma{\textit{doliorum}}\Cfootnote{\textsc{J. Kepler}, \textit{Nova stereometria doliorum vinariorum}, Linz 1615\cite{00346} (\cite{00114}\textit{KGW} IX, S. 5-133).}}
et de l'art de jauger; Visierstab.\\
%\pend
%\pstart%
%\noindent%
%\hangindent=7,5mm%
%\hspace*{-7.5mm}De arte qua pyxidis ope viam inveniunt fossores in terrae cavernis. \edtext{Ferdinandi III.\protect\index{Namensregister}{\textso{Ferdinand III.}, Kaiser des HRR (seit 1637) 1608-1657} et Caroli II. Angli\protect\index{Namensregister}{\textso{Karl II.}, König von England (seit 1660) 1630-1685} muniendi forma.}{\lemma{}\Bfootnote{Ferdinandi III.\protect\index{Namensregister}{\textso{Ferdinand III.}, Kaiser des HRR (seit 1637) 1608-1657} et Caroli II. Angli\protect\index{Namensregister}{\textso{Karl II.}, König von England (seit 1660) 1630-1685}} muniendi forma. \textit{erg. L}}\\
\hspace*{-7.5mm}De arte qua pyxidis ope viam inveniunt fossores in terrae cavernis. \edtext{Ferdinandi III.\protect\index{Namensregister}{\textso{Ferdinand III.}, Kaiser des HRR (seit 1637) 1608-1657}  et Caroli II. Angli\protect\index{Namensregister}{\textso{Karl II.}, König von England (seit 1660) 1630-1685} muniendi forma.}{\lemma{}\Bfootnote{Ferdinandi [...]
forma. \textit{erg. L}}}\\
%\pend
%\pstart%
%\noindent%
%\hangindent=7,5mm%
\hspace*{-7.5mm}De Levini \edtext{Hulsii\protect\index{Namensregister}{\textso{Van Hulse} (Hulsius), Levinus 1546-1606}}{\lemma{Hulsii}\Cfootnote{\cite{01107}\textsc{L. Hulsius}, \textit{Tractatus instrumentorum mechanicorum}, Frankfurt 1605.}} viatorio instrumento.\\
%\pend
%\pstart%
%\noindent%
%\hangindent=7,5mm%
\hspace*{-7.5mm}\textit{De Signatura rerum}.
\edtext{Crollius.\protect\index{Namensregister}{\textso{Croll} (Crollius), Oswald 1560-1609}}{\lemma{Crollius}\Cfootnote{\textsc{O. Croll}, \textit{De signaturis internis rerum}, Frankfurt 1609.\cite{00347}}}
\edtext{Porta,\protect\index{Namensregister}{\textso{Della Porta}, Giovanni Battista 1535-1615}}{\lemma{Porta}\Cfootnote{\textsc{G. B. Della Porta}, \textit{De humana physiognomia}, Oberursel 1601\cite{00348}.}}
alii.\\
%\pend
%\pstart%
%\noindent%
%\hangindent=7,5mm%
\hspace*{-7.5mm}De modis quibus natura formavit lineas in manibus nostris quodam texturae genere.\\
%\pend
%\pstart%
%\noindent%
%\hangindent=7,5mm%
\hspace*{-7.5mm}\textit{De ratione Libellandi} Scipionis \edtext{Claromontii,\protect\index{Namensregister}{\textso{Chiaramonti}, Scipione 1565-1652}}{\lemma{Claromontii}\Cfootnote{\cite{01108}\textsc{S. Chiaramonti} \textit{De usu speculi%
%pro libella et de tota libratione
}, in \textit{Opuscula varia mathematica}, Bologna 1653, S. 151-279;
dazu \cite{00349}\textsc{G.B. Riccioli}, \textit{Geographiae reformatae%
%libri duodecim
}, Venedig 1672, S. 231.}} per speculum, etc.\\
%\pend
%\pstart%
%\noindent%
%\hangindent=7,5mm%
\hspace*{-7.5mm}De divisione \edtext{[instrumentorum]:}{\lemma{instrumentum}\Bfootnote{\textit{L \"{a}ndert Hrsg.}}}
\edtext{Tychonis,\protect\index{Namensregister}{\textso{Brahe}, Tycho 1546-1601}
}{\lemma{Tychonis}\Cfootnote{Vgl. \cite{00327}\textsc{T. Brahe}, \textit{De mundi aetherei phaenomenis}, Prag 1603, S. 458ff. % (\cite{????}\textit{TBO}, ?????).
Siehe hierzu N. 2.% 035,15,06_009-016
}}
\edtext{Nonii,\protect\index{Namensregister}{\textso{Nunes} (Nonius), Pedro 1502-1578}}{\lemma{Nonii}\Cfootnote{\cite{00328}\textsc{P. Nunes}, \textit{De crepusculis}, 2. Ausgabe, Coimbra 1571, S. 20f. %
% ; vgl. \textsc{Hooke}, \textit{Animadversiones}, London 1674, S. 2f.
Siehe hierzu N. 2.% 035,15,06_009-016
}}
$\underbrace{\text{Vernierii, Hedraei,}}_{\displaystyle\text{Clavii,}}$%
\edtext{}{%
{\lemma{Vernierii}\Cfootnote{\cite{00291}\textsc{P. Vernier}, \textit{La construction, l'usage et les proprietez du quadrant nouveau}, Br\"{u}ssel 1631.}}%
{\lemma{Hedraei}\Cfootnote{\cite{00330}\textsc{B. Hedraeus}, \textit{Nova et accurata astrolabii geometrici structura}, Leiden 1643.}}}\advanceline{1}\edtext{}{\lemma{Clavii}\Cfootnote{\cite{01109}\textsc{C. Clavius}, \textit{Geometria practica}, Rom 1604.}}
Florentina,
\edtext{Thevenotiana.\protect\index{Namensregister}{\textso{Th\'{e}venot}, Melchis\'{e}dec 1620-1692}
}{\lemma{Thevenotiana}\Cfootnote{Möglicherweise Anspielung auf \cite{00135}\textsc{M. Thévenot}, \textit{Machine nouvelle}, Paris 1666.}}\\
%\pend
%\pstart%
%\noindent%
%\hangindent=7,5mm%
\hspace*{-7.5mm}Regulae de modo applicandi Theoriam Geometricam ad praxin, ut error sit quam minimus: exempli causa facile in angulis error committitur.
Aliae aliis ad praxin aptiores sunt constructiones.\\
%\pend
%\pstart%
%\noindent%
%\hangindent=7,5mm%
\hspace*{-7.5mm}De \edtext{Robervallii\protect\index{Namensregister}{\textso{Personne de Roberval}, Gilles 1602-1675}}{\lemma{Robervallii}\Cfootnote{Leibniz dürfte sich hier auf unveröffentlichte Vorlesungen über (u.a.) Mechanik, Optik und Landvermessung beziehen, die Roberval 1634 am Coll\`{e}ge Royal gehalten hat.\hspace{-2mm}}}
et \edtext{Cassini\protect\index{Namensregister}{\textso{Cassini}, Giovanni Domenico 1625-1712}}{\lemma{Cassini}\Cfootnote{Auch hier dürfte sich Leibniz auf unveröffentlichtes Material berufen. Denn Cassinis \cite{00511}\textit{Recueil d'observations pour perfectionner l'astronomie et la géographie} erschien erst 1693 (in Paris). Das dort eingefügte Trak\-tat \textit{De l'origine et du progès de l'astronomie et de son usage dans la géographie et dans la navigation} zirkulierte jedoch möglicherweise schon vorher.%
%\textsc{G.D. Cassini}, \textit{De l'origine et du prog\`{e}s de l'astronomie}, in \textit{Revueil d'observations}, Paris 1693.
}}
\edtext{[modo]}{\lemma{}\Bfootnote{modo \textit{erg. Hrsg.}}}
metiendi \edtext{ex stationibus duabus}{\lemma{ex}\Bfootnote{\textit{(1)} una statione \textit{(2)} stationibus duabus \textit{L}}}
quam minimum remotis,
fuit jam \edtext{Casati,\protect\index{Namensregister}{\textso{Casati}, Curzio um 1600}}{\lemma{Casati}\Cfootnote{\cite{00512}\textsc{C. Casati}, \textit{Geometricum problema}, Mailand 1602.}}
videatur \edtext{Schwenter.\protect\index{Namensregister}{\textso{Schwenter}, Daniel 1585-1636}}{\lemma{Schwenter}\Cfootnote{\cite{00513}\textsc{D. Schwenter}, \textit{Geometria practica nova et aucta}, N\"{u}rnberg 1667, Traktat IV, bes. S. 805f.}}\\
%\pend
%\pstart%
%\noindent%
%\hangindent=7,5mm%
\hspace*{-7.5mm}De delineationibus polygonorum qua uti solebat \edtext{Dux Vinariensis,\protect\index{Ortsregister}{Weimar}}{\lemma{Dux Vinariensis}\Cfootnote{Gemeint ist der Herzog von Sachsen-Weimar. Es ist allerdings nicht klar, auf welchen Leibniz hier Bezug nimmt.}}
et nunc \edtext{Weigelius.\protect\index{Namensregister}{\textso{Weigel}, Erhard 1625-1699}}{\lemma{Weigelius}\Cfootnote{\cite{01159}\textsc{E. Weigel}, \textit{Idea matheseos universae}, Jena 1669, S.~67 (§6).}}
\edtext{Pro horologiis solem saepe repraesentat pro numero horae.}{\lemma{}\Bfootnote{Pro [...] horae. \textit{erg. L}}}\\
%\pend
%\pstart%
%\noindent%
%\hangindent=7,5mm%
\hspace*{-7.5mm}Quomodo turritae figurae videantur in vasi aqua pleno, cui \edtext{ovi recentis infusus est liquor.}{\lemma{cui}\Bfootnote{\textit{(1)}\ ovum recens apertum \textit{(2)}\ ovis recentibus \textit{(3)}\ ovi [...] liquor \textit{L}}}
\edtext{De Octagonis aliisque, certa lege implendis. De Ludo Aggerario, et Latrunculorum.}{\lemma{}\Bfootnote{De Octagonis [...] Latrunculorum. \textit{erg. L}}}\\
%\pend%
%\pstart%
%\noindent%
%\hangindent=7,5mm%
\hspace*{-7.5mm}De Mensuris rerum. De pyramidum Aegypti mensura a \edtext{Gravio\protect\index{Namensregister}{\textso{Greaves}, John 1602-1652}}{\lemma{Gravio}\Cfootnote{\textsc{J. Greaves}, \textit{Pyramidographia}, London 1646.\cite{00350}}} relicta.\\
%\pend
%\pstart%
%\noindent%
%\hangindent=7,5mm%
\hspace*{-7.5mm}De Mensura constante per pendulum,\protect\index{Sachverzeichnis}{pendulum}
\edtext{Moutoni,\protect\index{Namensregister}{\textso{Mouton}, Gabriel 1618-1694}}{\lemma{Moutoni}\Cfootnote{\cite{00514}\textsc{G. Mouton}, \textit{Observationes diametrorum solis et lunae apparentium}, Lyon 1670.}}
\edtext{Hugenii,\protect\index{Namensregister}{\textso{Huygens} (Hugenius, Ugenius, Hugens, Huguens), Christiaan 1629-1695}}{\lemma{Hugenii}\Cfootnote{\textsc{C. Hyugens},\cite{00123} \textit{Horologium oscillatorium}, Paris 1673 (\cite{00113}\textit{HO} XVIII, S. 69-365).}}
\edtext{Buratini.\protect\index{Namensregister}{\textso{Burattini}, Tito Livio 1617-1681}}{\lemma{Buratini}\Cfootnote{\cite{00515}\textsc{T. L. Burattini}, \textit{Misura universale}, Vilnius 1675.}}\\
%\pend
%\pstart%
%\noindent%
%\hangindent=7,5mm%
\hspace*{-7.5mm}De modo complicandi chartas planas in globum.\\
%\pend
%\pstart%
%\noindent%
%\hangindent=7,5mm%
\hspace*{-7.5mm}De planisphaerio; Octavii \edtext{Pisanii,\protect\index{Namensregister}{\textso{Pisani}, Ottavio 1575- nach 1637)}}{\lemma{Pisanii}\Cfootnote{\cite{00516}\textsc{O. Pisani}, \textit{Astrologia seu motus et loca siderum}, Antwerpen 1613.}}
\edtext{Pardiesii,\protect\index{Namensregister}{\textso{Pardies}, Ignace Gaston 1636-1673}
}{\lemma{Pardiesii}\Cfootnote{\cite{01110}\textsc{I. G. Pardies}, \textit{Globi coelestis descriptio}, Paris 1673-1674.}}aliorum.\\
%\pend
%\pstart%
%\noindent%
%\hangindent=7,5mm%
\hspace*{-7.5mm}\edtext{De Bartschii\protect\index{Namensregister}{\textso{Bartsch}, Jakob 1600-1633} orbe concavo.}{{\lemma{De}\Bfootnote{\textit{(1)}\ Astrognosia, \textit{(2)}\ Bartschii \textit{(a)}\ , Purbachii\protect\index{Namensregister}{\textso{Peurbach}, Georg von 1423-1461}
\textit{(b)}\ orbe concavo \textit{L}}}{\lemma{Bartschii}\Cfootnote{\cite{00517}\textsc{J. Bartsch}, \textit{Usus astronomicus planisphaerii stellati}, N\"{u}rnberg 1661.}}}\\
%\pend
%\pstart%
%\noindent%
%\hangindent=7,5mm%
\hspace*{-7.5mm}De ovis. Columnis.\\
%\pend
%\pstart%
%\noindent%
%\hangindent=7,5mm%
\hspace*{-7.5mm}De pulvere chalybis magnetis afflatu figuras assumente \edtext{ex Rohalto\protect\index{Namensregister}{\textso{Rohault}, Jacques 1618-1672}}{{\lemma{ex Rohalto}\Bfootnote{\textit{ erg. L}}}{\lemma{ex Rohalto}\Cfootnote{\cite{00087}\textsc{J. Rohault}, \textit{Trait\'{e} de la physique}, Paris 1671, Teil III, S. 210ff.}}}.
\pend
\newpage
\pstart%
\noindent%
\hangindent=7,5mm%
%\hspace*{-7.5mm}
De vi plastica salium, vide \edtext{Quercetanum,\protect\index{Namensregister}{\textso{Du Chesne} (Quercetanus), Joseph 1544-1609}}{\lemma{Quercetanum}\Cfootnote{\cite{00519}\textsc{J. du Chesne}, \textit{La pharmacop\'{e}e des dogmatiques}, Paris 1630.}}
\edtext{Dobrszenski,\protect\index{Namensregister}{\textso{Dobrzensky}, Jacob Johann Wenceslas 1623-1697}}{\lemma{Dobrszenski}\Cfootnote{\cite{00520}\textsc{J.J.W. Dobrszensky}, \textit{Nova et amenior de admirando fontium genio philosophia}, Ferrara 1659.}}
\edtext{Marcum Marci\protect\index{Namensregister}{\textso{Marci}, Johannes Marek 1595-1667}}{\lemma{Marci}\Cfootnote{\cite{00352}\textsc{J. M. Marci}, \textit{Idearum operatricium idea}, Prag 1635.}}
in ideis operatricibus.
\edtext{Davissonium,\protect\index{Namensregister}{\textso{Davison}, William 1593-1669}
}{\lemma{Davissonium}\Cfootnote{\cite{00353}\textsc{W. Davison}, \textit{Oblatio salis}, Paris 1641.}}
et novissime \edtext{Concium.\protect\index{Namensregister}{\textso{Concius}, Andreas 1628-1682}}{\lemma{Concium}\Cfootnote{Möglicherweise \cite{00521}\textsc{A. Concius}, \textit{Physischer Discurs vom Stein der Weisen}, K\"{o}nigsberg 1656.}}
Bartholini\protect\index{Namensregister}{\textso{Bartholin}, Erasmus 1625-1698} \textit{figura} \edtext{\textit{nivis.}}{\lemma{\textit{nivis}}\Cfootnote{\textsc{E. Bartholin}, \textit{De figura nivis}, in \textsc{T. Bartholin}, \textit{De nivis usu medico}, Danzig 1661.\cite{00355}}}\\
%\pend
%\pstart%
%\noindent%
%\hangindent=7,5mm%
\hspace*{-7.5mm}\edtext{Jungii.\protect\index{Namensregister}{\textso{Jungius}, Joachim 1587-1657}}{\lemma{Jungii}\Cfootnote{\cite{00351}\textsc{J. Jungius}, \textit{Geometria empirica}, Rostock 1627.}} \textit{Geometria empirica}\\
%\pend
%\pstart%
%\noindent%
%\hangindent=7,5mm%
\hspace*{-7.5mm}De lineis motus astrorum.
\edtext{Mercatoris\protect\index{Namensregister}{\textso{Mercator}, Nicolaus 1620-1687}
}{\lemma{Mercatoris}\Cfootnote{\cite{00354}\textsc{N. Mercator}, \textit{Hypothesis astronomica nova}, London 1664, S. 2.}}
sectio quam vocat \edtext{[divinam]}{\lemma{divina}\Bfootnote{\textit{L \"{a}ndert Hrsg.}}}.\\
%\pend
%\pstart%
%\noindent%
%\hangindent=7,5mm%
\hspace*{-7.5mm}De homine quem Parisiis vidi liberrimo manuum tractu ex tempore figuras omnis generis formantem, ut quo tenderet, non appareret.\\
%\pend
%\pstart%
%\noindent%
%\hangindent=7,5mm%
\hspace*{-7.5mm}De Scribarum artificiis similibus.
Traits. Z\"{u}ge.
\edtext{De linea unica Claudii Melan,\protect\index{Namensregister}{\textso{Mellan}, Claude 1598-1688}
totam figuram absolvente.}{\lemma{De linea [...] absolvente}\Cfootnote{Claude Mellan entwickelte eine auf parallelen Linien beruhende Technik des Gravierens.}}\\
%\pend
%\pstart%
%\noindent%
%\hangindent=7,5mm%
\hspace*{-7.5mm}De formis monetarum, quibus literae ipsi crassitiei \edtext{circumscribuntur, ut}{\lemma{circumscribuntur,}\Bfootnote{\textit{(1)}\ quales \textit{(2)}\ ut \textit{L}}} \edtext{Blondellus\protect\index{Namensregister}{\textso{Blondel} (Blondellus), Fran\c{c}ois 1618-1686}
}{\lemma{Blondellus}\Cfootnote{Anspielung undeutlich.}}
quidam in Anglia,
\edtext{Firmus% \protect\index{Namensregister}{\textso{??}, ??}
}{\lemma{Firmus}\Cfootnote{Ferme, Pariser Medailleur (\textit{LSB} IV, 6 N. 129, S.~771\cite{01160})\protect\index{Namensregister}{\textso{Ferme} (Firmus), Pariser Medailleur}}}
in \edlabel{amoenior9}Gallia.\\
%\pend
%\pstart%
%\noindent%
%\hangindent=7,5mm%
\hspace*{-7.5mm}\edtext{Kepleri pars \edlabel{amoenior10}harmonica}{{\xxref{amoenior9}{amoenior10}}\lemma{Gallia.}\Bfootnote{\textit{(1)}\ Keplerus de harmonia Mundi \textit{(2)}\ Kepleri pars harmonica \textit{L}}} de \edtext{figuris.}{\lemma{figuris}\Cfootnote{\cite{01111}\textsc{J. Kepler}, \textit{Harmonice mundi}, Linz 1619 (\cite{00114}\textit{KGW} VI).}}
\edtext{Fluddi\protect\index{Namensregister}{\textso{Fludd}, Robert 1574-1637} \textit{Monochordum Mundi.}}{\lemma{Fluddi [...] \textit{Mundi}}\Cfootnote{\cite{01112}\textsc{R. Fludd}, \textit{Utriusque cosmi %
%maioris scilicet et minoris 
metaphysica, physica et technica historia}, Oppenheim 1617, Bd. I, S. 90;
\cite{01113}\textsc{ders.}, \textit{Monochordum mundi symphoniacum}, Frankfurt 1622.}}\\
%\pend
%\pstart%
%\noindent%
%\hangindent=7,5mm%
\hspace*{-7.5mm}De arte Scriptoria: deque invento calami scriptoris:
De figuris der Grabstichel instrumentorium Sculptoriorum.\\
%\pend
%\pstart%
%\noindent%
%\hangindent=7,5mm%
\hspace*{-7.5mm}De penicillo, de Miniaturis.
De lineationibus per puncta.
\pend
\count\Afootins=1500
\count\Bfootins=1500
\count\Cfootins=1500