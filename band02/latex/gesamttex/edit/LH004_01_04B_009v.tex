\pstart\noindent immediate sibi similes producant sed alias quasdam quae postea alias, et tandem hae alias omnino similes iis seminis producant, quod in animalibus videtur potius contingere quam in plantis.
\pend 
\pstart Atque ex his facile intelligitur cur maxima pars animalium et plantarum semen a reliquo corpore diversum excernant, itemque cur nonnulla sint sterilia et alio modo quam ex semine propagentur.
\pend 
\pstart Septem sunt praecipua genera particularum ex quibus corpus humanum conflatur nempe sunt acres amarae, dulces acidae, salsae serosae, aqueae, et pingues; inter acres numero spiritus omnes qui per insensilem transpirationem egrediuntur humoresque illos subtiles, ex quibus pustulae et similia quae ex flava bili oriri dicuntur amarae autem ad fel, et inde ad intestina fere omnes delabuntur; dulces carnem componunt; acidae vehiculum sunt aliarum, itemque salsae, hae punctim illae caesim poros omnes aperientes. Salsae, etiam acribus permixtae ut cera exasperant, serosae pinguibus accurate permistae humores frigidasque fluxiones et pituitam lentam componunt; pingues autem ab acribus compactae humorem melancholicum componunt, et serosas illarum meatus pertranseuntes in acidas mutant.
\pend%
\count\Bfootins=1200
\count\Afootins=1200
%\newpage
%\vspace*{1.0em}%  PR: Diesen leeren Zeilenabstand bitte behalten !!!
\pstart%
\noindent% PR: Neuer Abschnitt.
% \edtext{}{\lemma{}\Afootnote{\textit{Am Rand:} dec. 37.}}%
dec. 37. Non dubium mihi videtur quin animalia generentur primo ex eo quod semina maris et foeminae permista et calore rarescentia excernant ex una parte materiam asperae arteriae et pulmonum ex altera materiam hepatis, deinde ex harum duarum concursu accenditur ignis in corde. Notandumque partes aereas (ex quibus pulmo) terreas et aqueas ex quibus hepar sive ramum cavae in duas partes divisisse, quarum una versus spinam auriculas cordis composuit, alia anterior ventriculum cordis dextrum produxit, se scilicet sursum reflectendo in truncum aortae descendentem. Calor autem cordis effecit ut ex pulmone excerneretur flatus in asperam arteriam qui tandem ad os pervenit quo etiam alius flatus ex cerebro, a naribus et auribus pervenit. Excrementum autem cerebri praecipuum fuit, humor instar pituitae in ejus ventriculis coacervatus ex spiritibus per carotides arterias eo ex corde ascendentibus, qui humor per palatum et gulam delapsus in ventriculum restagnavit, et ex eo etiam itemque in mesenterium arteriae ex coeliaca quicquid crassius continebant expulerunt, unde facta sunt intestina, in quae patentissimi sunt
\edtext{meatus ab intestinis in venas.}{\lemma{meatus ab}\Bfootnote{\textit{(1)}\ arteriis per quas totum corpus eo \textit{(2)}\ intestinis in venas. \textit{L}}}
Lien etiam factum est ex sanguine ab arteriis eo expulso. Videmus enim crasso sanguine expurgato lien minui et aqua fabrorum lienem minuit, agitatio enim partium ferri in ea exstincti siccat quodammodo et indurat ejus \makebox[1.0\textwidth][s]{partes, quae postea melius ramosas partes illius sanguinis in liene coacervati incidunt:}
\pend
\count\Bfootins=1000
\count\Afootins=1000
\newpage
\pstart\noindent nec vero forsitan aquae acidae illas incidunt
\edtext{quia meatus}{\lemma{quia}\Bfootnote{\textit{(1)}\ aquae \textit{(2)}\ meatus \textit{L}}}
lienis ad illas transmittendas magis apti sunt.
\pend%
\vspace{0.7em}%  PR: Diesen leeren Zeilenabstand bitte behalten !!!
\pstart%
\noindent% PR: Neuer Abschnitt.
% \edtext{}{\lemma{}\Afootnote{\textit{Am Rand:} }}%
1631.\textso{ Partes similares et excrementa et morbi. }%
Praeter spiritum animalem constat homo spiritu animali nostro aeri homogeneo humore aquae homogeneo et solidis partibus quae cum terra possunt comparari.
Ex spiritus animalis mixtura cum humore fit spiritus vitalis igni comparabilis.%
%%%%
\edtext{}{\lemma{}\Afootnote{\textit{Am Rand:} (+ haec a juvene scripta +)\newline
spiritus animalis\newline
spiritus vitalis%\newline
\protect\begin{tabbing} 
sanguis  \hspace{18mm} \= dulcis\protect\\
flava bilis \> amara\\
atra bilis \> acida\\
urina \> salsa\\
pituita \>  insipida\\
\> caro\\
\>  cutis\\
\> membranae\\
\> nervi\\
\> ossa
  \protect\end{tabbing}
  %\protect\newline
%sanguis \hspace{14mm} dulcis\protect\\
%flava bilis \hspace{10mm} amara\protect\\
%atra bilis \hspace{12mm}  acida\protect\\
%urina \hspace{16mm} salsa\protect\\
%pituita \hspace{15mm} insipida\protect\\
%\hspace*{15mm} caro\protect\\
%\hspace*{15mm} cutis\protect\\
%\hspace*{15mm} membranae\protect\\
%\hspace*{15mm} nervi\protect\\
%\hspace*{15mm} ossa\protect\newline
\vspace{-2mm}\lbrack subscriptum\textsuperscript{[a]} erat alio atramento\rbrack\textsuperscript{[b]}
atra bilis non est acida, sed quod est atrum est durum et insipidum, liquor vero pellucidus simul mixtus est acidus.%
\vspace{1mm}% PR: Rein provisorisch !!!
\newline%
\footnotesize%
\textsuperscript{[a]} \lbrack subscriptum \textit{eckige Klammer von L}
\quad
\textsuperscript{[b]} atramento\rbrack\ \textit{eckige Klammer von L}\vspace{-8mm}}}
%%%
Ex imperfecta mixtura humoris cum partibus terrenis fit sanguis imperfectior vero contumaciorumque partium mixtura est 
\edtext{[flava bilis]}{\lemma{flammabilis}\Bfootnote{\textit{L \"{a}ndert Hrsg.}}}
perfectior quidem, sed in qua subtilissimum humoris evanuit est atra bilis acida, satis perfecta etiam sed in qua humor redundat est urina; satis perfecta etiam sed in qua desunt extrema tenuitatis et soliditatis est pituita lenta et mucus. Perfecta denique efficit carnes nervos et ossa, prout in ea plus vel minus est solidarum partium. 
\pend%
\pstart%
Ungues et pili sunt ejusdem materiae cum ossibus nec tamen ita durescunt quia nimis cito fluidae partes exhalant. Dentes autem ejusdem profecto materiae atque cornua, durescunt tamen instar aliorum ossium quoniam ore tecti plus humoris habent, lentiusque coalescunt.
\pend
\newpage
\count\Bfootins=1200
\count\Afootins=1200
\pstart Per aures exhalat spiritus excrementitius unde sibili et tinnitus, cum scilicet spiritus ille a sordibus aurium impeditur ne exeat, illisque allisus tunc sonitum edit.
\pend%
\pstart%
Per oculos etiam spiritus exhalat ut patet in menstruatis quarum oculi vaporem emmittere dicuntur; quippe totum corpus mulieris turget humoribus cum emittit menstrua, et quidem crassiore humore per vulvam purgatur, subtiliore vero per altiora, nempe per oculos.
\pend%
\pstart%
Horror omnis et frigus in corpore fit, quod partes fluidae confluunt in unum quemdam focum in quo tunc summus est calor.
Sic post cibum frigent extrema, quod partes calidae confluunt ad stomachum; sic in illis febribus quae a frigore incipiunt est affirmandum illas habere aliquem focum in quo vitiosus humor primum accenditur sive hoc sit in
\edtext{[corde]}{\lemma{corpore}\Bfootnote{\textit{L \"{a}ndert Hrsg.}}}
quod puto, sive alibi.
Sed iste vitiosus humor primo inficit sanguinem; qui sanguis dum ingreditur cor, efficit febrim, hinc accessus febrium nosci possunt.
\pend%
\count\Bfootins=1200
\count\Afootins=1200
%\count\Bfootins=1500
%\count\Cfootins=1500
%\count\Afootins=1500
\pstart%
Tres\edlabel{004_01_04b_009r_pU3a}
\edtext{}{{\xxref{004_01_04b_009r_pU3a}{004_01_04b_009r_pU3b}}\lemma{Tres [...] etc.}\Cfootnote{Für diese Passage aus Descartes' verschollenem Ms. besteht eine parallele Überlieferung in \cite{01144}\textsc{R.~Descartes}, \textit{Opuscula posthuma}, Amsterdam 1701, \glqq Primae co\-gi\-ta\-tio\-nes circa generationem animalium\grqq, S.~23. Siehe \cite{00120}\textit{DO} XI, S.~538.11-18.}}%
foci accenduntur a homine, primus in corde ex aere et sanguine, alius in cerebro ex iisdem sed magis attenuatis, tertius in ventriculo ex cibis et ipsius ventriculi substantia; in corde est quasi ignis ex sicca materia et densa in \edtext{cerebro est quasi ignis}{\lemma{cerebro}\Bfootnote{est\ \textbar\ quasi \textit{erg.}\ \textbar\ ignis \textit{L}}} ex spiritu vini in ventriculo ut ignis ex lignis viridibus;
in hoc \edtext{[cibi]}{\lemma{cibo}\Bfootnote{\textit{L \"{a}ndert Hrsg.}}}
etiam sine ipsius adjumento possunt sponte putrescere et incalescere, ut foenum humidum etc.%
\edlabel{004_01_04b_009r_pU3b} %
Jam in hepate ex consequentia ventriculi accenditur calor per mixturam chyli et sanguinis prius in eo existentis, hepar autem dicitur
% \edtext{}{\Cfootnote{Die Humoralpathologie kombinierte die vier K\"{o}rpers\"{a}fte mit bestimmten Organen und Eigenschaften. Demnach wird gelbe Galle (\pgrk{leuk'h qol'h}) in der Leber produziert, und dies Organ mit den Eigenschaften warm und trocken verbunden.}}
calidum, quando in eo multum est sanguinis jam facti; illud autem cito ad se trahit chylum sive partes maxime calefactas quae continentur in cibis, ideoque reliquiae difficilius corrumpuntur, unde putatur esse frigidus ventriculus. Jam accenduntur alii ignes non naturales in toto corpore, nempe phlegmones erysipelates abscessus, pleuritides etc. his modis: vel fit anastomosis venae et arteriae unde phlegmo, nempe cum sanguis calidior et acrior pervadit venae tunicam; vel idem sanguis acrior non potest quidem penetrare per venae tunicam sed extremitates, simul cum spiritibus sparsis
\edtext{facit erisipelatem}{\lemma{facit}\Bfootnote{\textit{(1)} erysipelatem \textit{(2)} erisipelatem \textit{ L}}} vel materia% Hier endet Bl. 9v und beginnt Bl. 10r.