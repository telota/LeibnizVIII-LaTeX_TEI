[10~r\textsuperscript{o}]
praeter naturam aliquem in locum confluxit, quae
\edtext{ibi ex}{\lemma{ibi}\Bfootnote{\textit{(1)}\ per \textit{(2)}\ ex \textit{L}}}
se ipsa putrescit, ut in \edtext{simplici abscessu;}{\lemma{simplici}\Bfootnote{\textit{(1)}\ abcessu \textit{(2)}\ abscessu; \textit{L}}}
vel ista putredo communicatur cum venis et arteriis propter loci vicinitatem ut in pleuritide in vulneribus etiam ignis accenditur quod ibi aperiuntur fibrae venarum et arteriarum sanguinisque faex ibi corrumpitur. Convulsio fit cum intra nervos flatus continetur, non vero purus spiritus animalis ibi vero flatus generatur, vel si pungatur nervus, vel si forte eo penetret lentus vapor. Convellit autem nervos ille flatus quod quodammodo connectit partes spirituum, efficitque ut plures simul conspirent, atque
\edtext{ita evincant}{\lemma{ita}\Bfootnote{\textit{(1)}\ convellant \textit{(2)}\ evincant \textit{L}}}
vim nervi, seque ipsas disponant ac determinent ad certos motus cum alias a nervis disponi ac determinari consueverint quia singulae nervi particulae sunt potentiores singulis spiritus particulis. Flatus non a solo calore et frigore fieri solent, sed
\edtext{tantum a frigore calori}{\lemma{tantum a}\Bfootnote{\textit{(1)}\ calore \textit{(2)}\ frigore calori \textit{L}}}
superveniente; nam calor quidem attenuat spiritus, sed non ideo flatum facit quia dum illos attenuat, simul et illis meatus aperit, per quos elabantur, et nisi calor tollatur, semper isti meatus in corpore proportione respondent quantitati spirituum qui rarefiunt; si vero superveniat frigus meatus istos intercludens, et spiritus qui rarefieri coeperit, pergit adhuc, tum quia coepit, tum magis etiam ex aliis partibus juvante calore; tunc iste spiritus qui exhalare non potest vertitur in flatum; idem patet in castaneis igni superpositis in ferro perforato; quippe si non moveantur ignis attenuat quidem spiritus intus conclusos, sed tamen attenuat etiam illarum cutem igni proximam per quam spiritus ille in sudorem expirat, si vero moveantur, tunc cutis quae erat igni proxima in aliam partem
\edtext{aeri exponitur}{\lemma{aeri}\Bfootnote{\textit{(1)}\ opponitur \textit{(2)}\ exponitur \textit{L}}} ejusque meatus ideo angustantur, spiritus vero intus nihilominus attenuatur, tum quod coepit, tum quod ignis ex altera parte eum urget, nec vero potest per cutem igni tunc
\edtext{obversam expirare,}{\lemma{obversam}\Bfootnote{\textit{(1)}\ egredi \textit{(2)}\ expirare, \textit{L}}}
tum quod nondum satis rarefacta est tum quod vias suas jam direxit in aliam partem et ita castanea cum impetu frangitur.
\pend%
\pstart%
Quaedam tamen esculenta sunt flatulenta quod cum facile a calore naturali solvantur in crassum spiritum, non tamen illis possunt ab eodem calore meatus aperiri tam facile per quos ex intestinis egrediantur.
\pend%
\pstart%
Brachium alligatur ad venae sectionem, ut copiosior sanguis remaneat in brachio, quod ideo fit quoniam sanguis cum impetu in diastole pellitur ad extremitates corporis, quod quia fit cum impetu, ideo sanguis $\langle$non$\rangle$ impeditur quominus ad brachium etiam perveniat, contra in systole refluit ab extremis corporis sine impetu, quo vinculum potest impedire ne refluat.
\pend%
\pstart%
Si ex morbo cholico fiat paralysis perit tantum motus non sensus, quod scilicet afficiuntur tantum nervorum membranae non medulla.
\pend%
\pstart%
Laesa ... (nervis) medulla perit interdum femoris motus illaeso motu brachii, nec mirum cum nervus ad femur inde perveniens, sit a nervo brachii distinctus; et praeterea illo in loco tenuior.
\pend%
\pstart%
Mucus defluens per nares et palatum in ipsis generatur, non in cerebro, quippe quamdiu materia ex qua gignitur est in cerebro, nihil aliud est quam spiritus, non mucus ut fuligo caminis adhaerens, non est caligo dum ex igne egreditur, sed fumus.
\pend%
\pstart%
Alitur foetus in utero sanguine ex omnibus membris matris defluente, potest\-que sanguis ille imbui formis vel ideis quae sunt in ejus phantasia unde signa in foetus corpore exprimuntur.
\pend%
\pstart%
Tempore somni plures egrediuntur spiritus per nares et palatum quam tempore vigiliae, unde sistitur tunc corpus. Fit\textso{ pandiculatio }post somnum ad replendos musculos spiritibus, qui tempore somni erant evacuati.
\pend%
\pstart%
Crocus\textso{ asthmaticis }prodest, datur ad scrup. 1\textsuperscript{um} cum $\displaystyle\frac{1}{2}$ musci grano et vino optimo.
\pend%
\pstart%
Fabae abstergunt, earumque esu quidam purgatus et a tussi liberatus.
\pend
%\newpage
\pstart%
Phthisicus sanatur utendo duobus vitellis ovorum parum coctis et aspersis pulvere sulphuris et
\edtext{[vino]}{\lemma{vini}\Bfootnote{\textit{L \"{a}ndert Hrsg.}}}
ad fabae majusculae quantitatem cum haustu vini dulcis optimum hora una ante alios cibos.%
\textso{ Antidotum contra pestem }%
\edtext{et venena}{\lemma{et venena}\Bfootnote{\textit{erg. L}}} Regis Mithridatis.\protect\index{Namensregister}{\textso{Mithridates}}
Recipe duas nuces siccas, duas ficus, et rutae folia totidem simul teras addito salis grano, et quolibet mane jejune sumas.
\pend%
\pstart%
Si adsit compunctio taediosa in plantis pedum et volis manuum dum egrediantur morbilli, contineantur tamdiu in aqua calida. Pulsus increbescunt statim a somno, quod sanguis per quietem torpens in quibusdam venis et in carnibus musculorum statim confluit versus cor, propter motum totius corporis et repentinum ingressum spirituum in musculos. Unde tunc oscitatio et pandiculatio simul interdum fiunt.
\pend%
\pstart%
Sternutatio est expurgatio ventriculorum cerebri per nares. Oscitatio est expurgatio
\edtext{[vaporum]}{\lemma{vaporem}\Bfootnote{\textit{L \"{a}ndert Hrsg.}}}
inter utrumque menyngem existentium per palatum. Vapores autem ibi colliguntur ex defectu agitationis in substantia cerebri, vel cum spatio illo inter duas menynges pleno existente, ut est semper, repente contrahitur, quoniam inflatur cerebrum ut cum excitamur a somno olfacimus emittendo spiritum ex pectore per nares, si odor in ore clauso contineatur, et etiam si auri imponatur.
\pend%
\pstart%
Mulier singulis 7 diebus hemicrania laborans.
\textit{Hist. mir.} f. 804.
\pend%
\pstart%
In scorbuto quibusdam 4\textsuperscript{o} aut quinto die, aliis tertio aliis singulis diebus motus aggravativus sine manifesta febri, vel cum levissima observatur noctem.
[10~v\textsuperscript{o}]
\pend%
\newpage
%\count\Bfootins=1500
%\count\Cfootins=1500
%\count\Afootins=1500