[8~v\textsuperscript{o}]
omnes suos per hepar spargit; praecipue versus ejus concavam partem et eo defert omnem sanguinem et succum a radicibus acceptum, ibique iccirco nullis arteriis est comitata. 
\pend%
\pstart%
Emulgentes sunt vasa latissima, quae ex aorta et cava prodeant, videnturque initio illarum finem fuisse, ibique ideo sanguinem restagnasse, atque renes et vesicam produxisse eodem tempore quo arteria ulterius pergens coepit venam conscendere, et ad ilia indeque ad umbilicum per vesicae latera et in 2\textsuperscript{as} divisa tendere.
Hinc fit ut renum situs, et vasorum ad illos valde varient. Et in foetus corpore tanquam natantes, praesertim sinister, reperiantur.
%
Extatque apud Bauhinum,\protect\index{Namensregister}{\textso{Bauhin}, Caspar 1560-1624}%
% \cite{01123}\textsc{C. Bauhin}, \textit{Theatrum anatomicum}, Frankfurt a.M. 1605, S. 140 ??? und 153f. ???
%
 insignis historia cujusdam qui habebat renem sinistrum juxta vesicam locatum et alia vasa miro modo disposita,
quae omnia ex hoc uno videntur contigisse, quod arteria,
\edtext{ut venam}{\lemma{ut}\Bfootnote{\textit{(1)}\ foramen \textit{(2)}\ venam \textit{L}}}
conscenderet per medium venae emulgentis sinistrae transiverit, venit enim semper a parte sinistra:
unde puto omnem rationem petendam, cur hepar in dextro latere, lien in sinistro etc.
item lumbares tum
\edtext{venae}{\lemma{}\Bfootnote{venae\ \textbar\ tum venae \textit{gestr.}\ \textbar\ tum \textit{L}}}
tum arteriae, quae infra emulgentes producuntur, postquam ad spinae medullam interius penetrarunt ramos habent qui sursum versus cerebrum reflectuntur. Quod indicat arteriam ulterius pergere conatam in omnes partes sibi
\edtext{viam quaesiisse:}{\lemma{viam}\Bfootnote{\textit{(1)}\ fecisse \textit{(2)}\ quaesiisse: \textit{L}}}
tunc autem umbilicus totam ventris capacitatem a nothis costis ad inguina occupabat.
Valvulas in venis emulgentibus
%
dicit esse Bauhinus\protect\index{Namensregister}{\textso{Bauhin}, Caspar 1560-1624} %
 quae seri refluxum impediant;
de qua re dubito, contra enim potius sanguinis in renes a venis illapsum deberent impedire.
\pend%
\pstart%
Ureteres autem ita ex renibus prodeunt, ut in quoque rene sint 8 vel 9 infundibula carne renum instar glandularum occlusa, quorum deinde 2 vel 3 in unum coeunt, et denique tres in unum canalem, qui est ureter quique nervulum a sexto pari recipit, et vesicae ita implantatur, ut ab ea sine fractione separari non possit.
\pend%
\pstart%
Mihi videtur in embryone lienem versus spinam in medio corporis, et hepar versus umbilicum fuisse sita, venamque umbilicalem medio hepatis fuisse implantatam; sed postea dum inflaretur ventriculus et aorta a sinistris cavae truncum in lumbis conscenderet, secessit hepar in dextrum latus, et lien in sinistrum.
\pend%
%\newpage
\pstart%
Ex venis et arteriis per lienem transeuntibus; unae sunt vas breve dictae quae ad fundum ventriculi transeunt, et aliae ad rectum intestinum, ubi haemorrhoidales internas constituunt. Est autem canalis patentissimus a venis lienis per truncum portae ad hepar, et in ipso hepate a porta in cavam, et deinde a cava in cor, a corde in cerebrum; unde fit
\pend
\newpage
\pstart\noindent ut nocte liene compresso vel manu, vel ob decubitum in sinistrum latus, gravia occurrant insomnia: tetri enim vapores a liene expressi in cerebrum statim ascendunt.
\pend%
\pstart%
Flava bilis in embrione videtur medium hepatis infima ejus parte occupasse, nempe partes sanguinis amarescentes eo fuisse sponte delapsas: postea vero crescente hepate et recedente versus dextrum latus, ejus flavae bilis receptaculum in duas partes fuisse divisum, nempe in porum bilarium qui recipit
\edtext{[fel]}{\lemma{vel}\Bfootnote{\textit{L \"{a}ndert Hrsg.}}}
a sinistra hepatis parte, et vesicam bilariam quae recipit a parte dextra, quaeque ideo major est poro bilario.
\pend%
\pstart%
In hepate notandum quasdam venae portae extremitates
(ut ajunt libri) medias venae cavae radices subire, et contra quasdam cavae medias portae radices subire.
Patet autem cavam ex hepate omnino prodire, non tantum enim ejus pars ascendens ex summa ejus parte egreditur sed etiam descendens quae statim reflectitur, et secundum ejus posteriorem partem descendit, atque it comitatum aortam descendentem.
\pend%
\pstart%
In ventriculo observo intus illum habere
\edtext{fibras rectas,}{\lemma{habere}\Bfootnote{\textit{(1)}\ vesicas rectas \textit{(2)}\ fibras rectas, \textit{L}}}
quae ab ore per oesophagum eo pertingunt, intestina autem transversas. Item illum habere multos nervos, et duos etiam
\edtext{esse}{\lemma{esse}\Bfootnote{\textit{erg. L}}}
recurrentes: item noto historiam illius qui hepate carebat, sed omnia intestina magis carnosa: item in pueris multa excrementa a cerebro in ventriculum delabi. Ex quibus conjicio totum ductum ab ore ad podicem, ortum habere ab excrementis e cerebro delabentibus; ipsamque oris aperturam ab iisdem excrementis eo regurgitantibus. Restagnasse autem ista excrementa infra hepar, ibique ideo capacitatem ventriculi excavasse, dum sanguis in emulgentibus etiam restagnabat.
Ex hoc autem quod ex ore in jugulum laberentur ista excrementa, viamque aeri ex aspera arteria egredi tentanti clauderent, fit, ut nares sint geminae, quia per gulae latera iste aer sursum ascendit.
(+ ingeniose +)
[9~r\textsuperscript{o}]
\pend%
%\count\Bfootins=1500
%\count\Cfootins=1500
%\count\Afootins=1500