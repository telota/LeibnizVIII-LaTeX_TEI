      
                \begin{Ueberlieferung}%
               {\textit{L}}Notiz: LH XXXV 13, 2c Bl. 144. Papierstreifen (23 x 4 cm). 4\,\nicefrac{1}{2} Z. auf Bl.~144~r\textsuperscript{o}. Bl.~144~v\textsuperscript{o} leer. Ein
Wasserzeichen.\\Cc 2, Nr. 00
\end{Ueberlieferung}
\begin{Datierungsgruende}% 
Das Wasserzeichen ist für die Zeit von Frühjahr 1672 bis Anfang 1673 belegt (derselbe Typus ist in den Textträgern von \textit{LSB} VI, 3 N.~2 und N.~4 anzutreffen).
                \end{Datierungsgruende}
\pstartfirst
            [144~r\textsuperscript{o}] Scientia de progressionibus potest perficere Geometriam: Nam si ratio invenietur, datis duobus altero decrescente altero crescente, diversa proportione, invenire punctum aequalitatis, habebimus \edtext{circumferentiae aequalem  rectam}{\lemma{habebimus}\Bfootnote{\textit{(1)}\ Circulum \textit{(2)}\ circumferentiae aequalem rectam. \textit{L}\ }}. Finge Tibi corpora duo se accedere in linea recta, diversis celeritatibus, in certo quodque  proportionum in genere, invenire punctum concursu seu \textso{quasi }\textso{centrum gravitatis}\protect\index{Sachverzeichnis}{centrum gravitatis}. Potest enim tale punctum concursus\protect\index{Sachverzeichnis}{concursus} jure appellari centrum gravitatis motuum.\pend 
 


 

