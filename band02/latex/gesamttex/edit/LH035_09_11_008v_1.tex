[8~v\textsuperscript{o}]
addatur $\displaystyle g1c$, et auferatur\rule[-4mm]{0mm}{10mm}
\edtext{$\displaystyle \frac{d}{a}g1c$ sitque $\displaystyle g \; \groesser \displaystyle \frac{d}{a}g$ 
erit et $\displaystyle g1c \ \groesser \, \displaystyle \frac{d}{a}g1c$ 
adeoque plus addetur}{\lemma{auferatur $\displaystyle \frac{d}{a}g1c$}\Bfootnote{ \textit{(1)} seu cum adda \textit{(2)} sitque $\displaystyle g1c \ \groesser \, \displaystyle \frac{d}{a}g1c$,  
ac proinde et \textit{(3)} sitque [...] erit et \textit{(a)} $\displaystyle \frac{d}{a}g \ \groesser \ 0$  
 \textit{(b)} $\displaystyle g1c \ \groesser \, $
[...] addetur, \textit{L}}},
quam auferetur celeritatis\protect\index{Sachverzeichnis}{celeritas}.
Ideoque nulla erit retardatio\protect\index{Sachverzeichnis}{retardatio}, si quidem motus adhuc acceleratur, sed si motus non amplius acceleretur, erunt decrementa uniformia.
\pend
\count\Bfootins=1200
\pstart
Nota si causa illa ipsa quae est gravitatis\protect\index{Sachverzeichnis}{gravitas}, esset etiam gravitationis acceleratione\protect\index{Sachverzeichnis}{acceleratio} quaesitae,
tunc sequeretur corpus a quantacunque lapsum altitudine non posse levare sibi aequale; nam
\edtext{ea gravitas celeriter agens}{\lemma{ea}\Bfootnote{%
\textit{(1)} ipsa ea %
\textit{(2)} gravitas %
\textit{(a)} sibi ips %
\textit{(b)} celeriter agens \textit{L}}}
ad deprimendum unum corpus; eadem ageret celeritate ad deprimendum tantundem; ac proinde nihil ageret, ob compensationem.
Necesse est ergo rem fieri
\edtext{undulatione\protect\index{Sachverzeichnis}{undulatio} seu}{\lemma{undulatione}\Bfootnote{\textit{(1)} . Contra hinc videtur causa eadem esse elaterii et gravitatis \textit{(2)}  seu \textit{L}}}
motu in medio circumfuso relicto, vel etiam in ipso corpore existente.
\pend
\pstart
Si obstaculum aliquod tale sit, ut quo fortius impingis, hoc fortius
\edtext{repellat, tunc}{\lemma{repellat,}\Bfootnote{\textit{(1)} seu \textit{(2)} tunc \textit{L}}}
habebit locum calculus noster de
\edtext{frictione\protect\index{Sachverzeichnis}{frictio}.\\\hspace*{7,5mm}\textso{Experience}}{\lemma{frictione.}\Bfootnote{\textit{(1)} Experimentum \textit{(2)} \textso{Experience} \textit{L}}}
\textso{sur le frottement.}\protect\index{Sachverzeichnis}{frottement}
Un corps qui est m\^{u} avec
\edtext{difficult\'{e} le long d'un autre, ou sur}{\lemma{difficult\'{e}}\Bfootnote{\textit{(1)} sur \textit{(2)} le long [...] sur \textit{L}}}
un autre, est m\^{u} avec d'autant plus de difficult\'{e}, qu'il est m\^{u} plus viste.
Par exemple un moulin qui bat l'air ou l'eau avec ses ailes, trouuera d'autant plus de resistence\protect\index{Sachverzeichnis}{r\'{e}sistance}, qu'il doit aller plus viste.
Une surface \^{a}pre, comme par exemple une poutre entour\'{e}e d'une
\edtext{corde, nous fera perdre d'autant plus de mouuement que la corde va plus viste.}{\lemma{corde,}\Bfootnote{\textit{(1)} resistera \`{a} la \textit{(2)} nous [...] viste. \textit{L}}}
\pend
\newpage
\count\Afootins=1500
\count\Bfootins=1500
\count\Cfootins=1500