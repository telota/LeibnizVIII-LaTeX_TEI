\begin{ledgroupsized}[r]{120mm}%
\footnotesize%
\pstart%
\noindent\textbf{\"{U}berlieferung:}%
\pend%
\end{ledgroupsized}%
\begin{ledgroupsized}[r]{114mm}%
\footnotesize%
\pstart%
\parindent -6mm%
\makebox[6mm][l]{\textit{L}}%
Aufzeichnung:
LH XXXV 5, 23 Bl. 14.
Zwei unregelm\"{a}{\ss}ig beschnittene,
durch einen d\"{u}nnen Papierstreifen verbundene rechteckige Schnipsel
(cm 17,5 x 6 und 10 x 4,5),
beide auf der Vorderseite beschrieben, der obere auch auf der Rückseite.
% Auf Bl.~14~r\textsuperscript{o} Text mit Zeichnung als Markierungszeichen und zwei Marginalien, die inhaltlich mit dem Text auf der R\"{u}ckseite verweisen.
% Auf 14~v\textsuperscript{o} Text vom Mai 1678, mit dem Leibniz seine Ausführungen von 1675 richtigstellt.
%Textverlust durch die Beschneidung der Ränder.%
Vom unteren Schnipsel ist die linke Hälfte abgerissen (mit beträchtlichem Textverlust).
\newline%
Cc 2, Nr. 00%
\pend%
 \end{ledgroupsized}%
% \normalsize
% \vspace*{5mm}
% \begin{ledgroup}
% \footnotesize 
% \pstart
% \noindent\footnotesize{\textbf{Datierungsgr\"{u}nde}: Von Leibniz eigenh\"{a}ndig datiert.}
% \pend
% \end{ledgroup}
%
%
\vspace*{8mm}%
\pstart
\noindent
[14~r\textsuperscript{o}]
\pend
\count\Afootins=1200
\count\Bfootins=1200
\count\Cfootins=1200
\pstart%
\normalsize%
\noindent%
\centering%
% [14~r\textsuperscript{o}]
\textso{Calculus Logarithmicus memorabilis}%
\edtext{}{\lemma{}\Afootnote{\hspace{1.8mm}\textit{Am Rand:}
Hoc scripsi anno 1675 sed nunc reperi anno 1678 esse paralogismum.}}\protect\index{Sachverzeichnis}{logarithmus}
\pend%
\vspace{1,0em}% PR: Rein provisorisch !!!
\pstart%
\noindent%
% \centering%
\begin{wrapfigure}[4]{l}{0.15\textwidth}%              
\vspace{-4mm}\includegraphics[trim = 0mm -4mm -5mm 0mm, clip,width=0.15\textwidth]{images/lh0350523_14r-d.pdf}\\%
%\vspace*{2mm}%
\centering%
[\textit{Fig. 1}]%
\end{wrapfigure}%
% \pend%
% \vspace*{1.5em}% PR: Rein provisorisch !!!
% \pstart%
% \noindent%
\textso{Recipiens magdeburgicus}\protect\index{Sachverzeichnis}{magdeburgicus} contineat aerem\protect\index{Sachverzeichnis}{aer} ut  $\displaystyle a$.
exhauriatur una\rule[0mm]{0mm}{5mm}
\edtext{vice $\displaystyle\frac{a}{b}.$ quantum}{\lemma{vice $\displaystyle\frac{a}{b}.$}\Bfootnote{\textit{(1)} restabit $\displaystyle a-\frac{a}{b}.$ \textit{(2)} quantum \textit{L}}}
scilicet implet Vas minus unde Embolus vi extrahendus est.
Quod ponatur esse \rule[-4mm]{0mm}{6mm}ad $\displaystyle a$
\edtext{ut $\displaystyle\frac{a}{b}$. Secunda}{\lemma{ut $\displaystyle\frac{a}{b}.$}\Bfootnote{\textit{(1)} Cum \textit{(2)} Secunda \textit{L}}}
vice exhaurietur $\displaystyle\frac{a-\displaystyle\frac{a}{b}}{b}$ seu $\displaystyle\frac{a}{b}-\frac{a}{bb}.$
et tertia vice exhaurietur $\displaystyle\frac{\displaystyle\frac{a}{b}-\frac{a}{bb}}{b}$
seu $\displaystyle \frac{a}{bb}-\frac{a}{bbb}.$
et quarta vice $\displaystyle \frac{a}{bbb}-\frac{a}{bbbb}.$
Summa exhaustorum 4 vicibus:\rule[-4mm]{0mm}{10mm}
$\displaystyle \frac{a}{b},+\frac{a}{b}-\frac{a}{bb},+\frac{a}{bb}-\frac{a}{bbb},+\frac{a}{bbb}-\frac{a}{bbbb}.$%
\pend%
\pstart%
% \newpage% PR: Rein provisorisch !!!
\vspace*{1.0em}% PR: Rein provisorisch !!!
\normalsize%
\centering%
Seu[:]%
\pend%
\pstart%
\vspace*{1.0em}% PR: Rein provisorisch !!!
\normalsize%
\noindent%
A primo duplicato ultimus \edtext{detrahatur,}{\lemma{}\Afootnote{\textit{Am Rand:} Error\vspace{-4mm}}}
$\displaystyle \frac{a}{b}+\frac{a}{b}-\frac{a}{bbbb}$
productum erit summa.
Et summa detractorum \rule[-4mm]{0mm}{10mm}$\displaystyle z$ vicibus erit $\displaystyle\frac{za}{b}-\frac{a}{b^{z+2}}.$%
\pend%
\newpage
\count\Afootins=1200
\count\Bfootins=1200
\count\Cfootins=1200
\pstart%
%\vspace*{1.0em}% PR: Rein provisorisch !!!
\normalsize%
\noindent%
\edtext{%
\textlangle\textendash\,\textendash\,\textendash\textrangle
s, ponatur aer millies levior corpore labente. Corpus labens
%
\textlangle\textendash\,\textendash\,\textendash\,c\textrangle
eleritate \edtext{ut $\displaystyle1000.$}{\lemma{ut}\Bfootnote{%
\textit{(1)} $1.$ \textit{(2)} $1000.$ \textit{ L}}}
resistentem \edtext{ut $\displaystyle1.$
%
\textlangle\textendash\,\textendash\,\textendash\,ad\textrangle
imet}{\lemma{ut $\displaystyle1.$}\Bfootnote{%
\textbar\ ex secunda celeritate ut \textit{gestr.} \textbar\ \textlangle\textendash\,\textendash\,\textendash\,ad\textrangle imet \textit{ L}}}
$\displaystyle1.$ remanet $\displaystyle99.$
Hic impetus ut \edtext{$\displaystyle99$ remanet, acceditque}{\lemma{$\displaystyle99$ remanet,}\Bfootnote{%
\textit{(1)} sed iam \textit{(2)} acceditque \textit{ L}}}
%
\textlangle\textendash\,\textendash\,\textendash\,corpor\textrangle
is. Non erit ergo secundo momento
\edtext{vis ut $4.$}{\lemma{vis}\Bfootnote{\textit{(1)} ut $99\smallfrown99.$ \textit{(2)} ut $4.$ \textit{ L}}}
Sed po%
%
\textlangle\textendash\,\textendash\,\textendash.\textrangle\
Erit $\displaystyle4\smallfrown999.$ et aer erit $\displaystyle4.$ restabit $\displaystyle4\smallfrown 999,\ -\ 4.$ Patet
%
\textlangle\textendash\,\textendash\,\textendash\,d\textrangle
ecursum aerem aequiponderaturum.%
}{\lemma{\textlangle\textendash\,\textendash\,\textendash\textrangle s, [...] aequiponderaturum}\Cfootnote{%
Die Anzahl der durch die Beschädigung des Textträgers ausgefallenen Wörter lässt sich nicht ermitteln.}}
[14~v\textsuperscript{o}]
\pend%
\vspace*{2.0em}% PR: Rein provisorisch !!!
\pstart%
\normalsize%
\noindent%
%Maii 1678%
%\newline%
%Aer\protect\index{Sachverzeichnis}{aer}
%\edtext{recipientis $\displaystyle a$.}{\lemma{}\Bfootnote{recipientis\ \textbar\ majoris \textit{gestr.}\ \textbar\ $a$. \textit{L}}}
%\edtext{unde detrahendus}{\lemma{unde detrahendus}\Bfootnote{\textit{erg. L}}}
%aer inde detractus $\displaystyle b$. residuus in majore $\displaystyle a-b$.
%Inde rursus detrahetur $\displaystyle d$ qui sit ad residuum, $\displaystyle r$,
%ut $\displaystyle b$ ad $\displaystyle a\frac{d}{r}$ aequ. $\displaystyle\frac{b}{a}$.
%Ergo $d$ aequ. $\displaystyle\frac{b}{a}r$. et sequens residuum $\displaystyle(r)$. est $\displaystyle r-d$.
%Ergo \rule[-4mm]{0mm}{10mm}$\displaystyle(r)$ aequ. $\displaystyle r-\frac{b}{a}r$, seu $\displaystyle(r)$ aequ. $\displaystyle\frac{a-b}{a}r$.
%Hinc series residuorum erit:\rule[-4mm]{0mm}{10mm} $\displaystyle a. a-b. \frac{\overline{a-b^{2}}}{a}$. $\displaystyle \frac{\overline{a-b^{3}}}{a^{2}}$ etc.
%Sunt ergo progressionis geometricae.\protect\index{Sachverzeichnis}{progressio geometrica}
%Hinc si aeres residui sint ut numeri, erunt
%\edtext{exhaustionum multitudines}{\lemma{exhaustionum}\Bfootnote{\textit{(1)} numeri ut \textit{(2)} multitudines \textit{L}}}
%ut Logarithmi.\protect\index{Sachverzeichnis}{logarithmus}
%Videndum autem
%\edtext{an pro dato}{\lemma{an}\Bfootnote{\textit{(1)} possit dete \textit{(2)} pro dato\textit{L}}}
%quovis numero qui progressiones Geometricae non sit, possit definiri hoc modo Logarithmus.
%Nempe si praeter tres exhaustiones subjiciatur adhuc una dimidia extractio.
%Sed ea ut video rem non efficit. Interim hoc methodo illud efficietur pulcherrimum,
%ut possit excitari potentia altissima vel extrahi radix pura cuiuscunque gradus solo extructionum vel admissionum aeris certo
%\edtext{numero. Imo}{\lemma{numero}\Bfootnote{\textit{(1)} ex.g. si ex numero aliquo \textit{(2)} Imo video \textit{L}}}
\noindent%
Maii 1678%
\newline%
Aer\protect\index{Sachverzeichnis}{aer}
\edtext{recipientis unde detrahendus $\displaystyle a$.}{\lemma{recipientis}\Bfootnote{%
\textit{(1)} majoris $a$. \textit{(2)} unde detrahendus $a$. \textit{L}}}
aer inde detractus $\displaystyle b$.
residuus in majore $\displaystyle a-b$.
inde rursus detrahetur $\displaystyle d$ qui sit ad residuum, $\displaystyle r$,
ut $\displaystyle b$ ad $\displaystyle a.$
$\displaystyle\frac{d}{r}$ aequ. $\displaystyle\frac{b}{a}$.
Ergo $d$ aequ. $\displaystyle\frac{b}{a}r$. et sequens residuum $\displaystyle(r)$ est $\displaystyle r-d$.
Ergo \rule[-4mm]{0mm}{10mm}$\displaystyle(r)$ aequ. $\displaystyle r-\frac{b}{a}r$, seu $\displaystyle(r)$ aequ. $\displaystyle\frac{a-b}{a}r$.
Hinc series residuorum erit:\quad%
\rule[-4mm]{0mm}{10mm}%
$\displaystyle a.$\quad%
$\displaystyle a-b.$\quad%
$\displaystyle\frac{\overline{a-b\ }^{2}}{a}.$\quad%
$\displaystyle \frac{\overline{a-b\ }^{3}}{a^{2}}$
etc. Sunt ergo progressionis geometricae.\protect\index{Sachverzeichnis}{progressio geometrica}
Hinc si aeres residui sint ut numeri, erunt
\edtext{exhaustionum multitudines}{\lemma{exhaustionum}\Bfootnote{\textit{(1)} numeri ut \textit{(2)} multitudines \textit{L}}}
ut Logarithmi.\protect\index{Sachverzeichnis}{logarithmus}
Videndum autem
\edtext{an pro dato quovis numero qui progressiones Geometricae non sit, possit definiri}%
{\lemma{an}\Bfootnote{\textit{(1)} possit dete \textit{(2)} pro [...] definiri \textit{ L}}}
hoc modo Logarithmus.
Nempe si praeter tres exhaustiones subjiciatur adhuc una dimidia extractio.
Sed ea ut video rem non efficit. Interim hoc methodo illud efficietur pulcherrimum,
ut possit excitari potentia altissima vel extrahi radix pura cuiuscunque gradus solo extructionum vel admissionum aeris certo
\edtext{numero. Imo}{\lemma{numero}\Bfootnote{\textit{(1)} ex.\,g. si ex numero aliquo \textit{(2)} . Imo \textit{L}}}
video sic solum procedere tantum excitationem potentiae seu multiplicationem non vero extractionem, quia $\displaystyle a$ et $\displaystyle b$. adeoque et $\displaystyle a-b$. iam dantur adeoque frustra per extractionem quaeruntur.
\edtext{Si fiat}{\lemma{Si}\Bfootnote{\textit{(1)} posset \textit{(2)} fiat \textit{L}}}
$\displaystyle a$ valde magna, $\displaystyle b$ valde parva,
v.g. vix $\displaystyle 1000000\textsuperscript{ma}$ prioris quod facile est ob spatiorum solidorum proportionem triplicatam et
\edtext{adhibeamus intus tubum}{\lemma{adhibeamus}\Bfootnote{\textit{(1)} instrumentum \textit{(2)} intus tubum \textit{ L}}}\protect\index{Sachverzeichnis}{tubus}
monstrantem aeris quantitatem accidente indice maioris exactitudinis causa,
et conversionibus emboli instrumento\protect\index{Sachverzeichnis}{instrumentum} dentato denumerantibus pulcherrime Logarithmos sine calculo inveniemus.
Cavendum ne aeris mutatio noceat.
\pend%
\count\Afootins=1500
\count\Bfootins=1500
\count\Cfootins=1500
%%%%  PR: Hier endet das Stück.