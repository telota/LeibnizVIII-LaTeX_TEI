\begin{ledgroupsized}[r]{120mm}
\footnotesize
\pstart
\noindent\textbf{\"{U}berlieferung:}
\pend
\end{ledgroupsized}
%
\begin{ledgroupsized}[r]{114mm}
\footnotesize
\pstart
\parindent -6mm
\makebox[6mm][l]{\textit{L}}Konzept: LH XXXV 9, 11 Bl. 7-10. 2 Bog. 2\textsuperscript{o}. Etwas mehr als 2 \nicefrac{1}{2} S. auf Bl.~8~v\textsuperscript{o} bis Bl.~10~r\textsuperscript{o}.
Auf Bl.~7~r\textsuperscript{o} bis Bl.~8~v\textsuperscript{o} (mittig) ist N.~34\textsubscript{1} überliefert.
Auf Bl.~10~r\textsuperscript{o} beginnt nach elf Zeilen N.~34\textsubscript{3}.
Leibniz' eigenh\"{a}ndige Datierung und Nummerierung der Bogen:
\textit{May 1675. Frottement part. (2)} am oberen Rand von Bl. 7~r\textsuperscript{o};
\textit{May 1675. Frottement. Part. (3)} am oberen Rand von Bl. 9~r\textsuperscript{o}.
Gleicher Wasserzeichentypus auf Bl.~8 und Bl.~10.
Der Text wird editorisch in drei Teile unterteilt,
die auf verschiedene Redaktionsstufen zur\"{u}ckgehen k\"{o}nnten.%
\\Cc 2, Nr. 965 D, K
\pend
\end{ledgroupsized}
\vspace{8mm}
%
%
%\newpage
\count\Bfootins=1200
\count\Afootins=1200
\pstart\noindent
[8~v\textsuperscript{o}]
\pend
\pstart
\centering
De la Retardation\protect\index{Sachverzeichnis}{retardation} du mouuement par le frottement.
\pend
\vspace*{0.5em}
\pstart
\centering
[\textit{Teil 1}]
\pend
\begin{Geometrico}
\textso{Frottement,} est un attouchement continuel d'un corps qui est en mouuement, \`{a} un autre qui ne l'est pas, ou qui l'est autrement
\end{Geometrico}
\begin{Geometrico}
\textso{Observation[:]}
\edtext{(1)}{\lemma{}\Bfootnote{(1) \textit{erg. L}}}
Tout frottement des corps sensibles retarde leur mouuement.\\
(2) Tout frottement des corps
\edtext{sensibles fait}{\lemma{sensibles}\Bfootnote{\textit{(1)}\ cause \textit{(2)}\ fait \textit{L}}}
quelque bruit ou produit quelque son\protect\index{Sachverzeichnis}{son}.
\end{Geometrico}
\begin{Geometrico}
\textso{Consequences[:]}
\edtext{(1) Les corps sensibles ont les surfaces \^{a}pres ou in\'{e}gales}{\lemma{(1)}\Bfootnote{\textit{(1)}\ Les corps qui se touchent, pendant qu'ils se touchent sont \^{a}pres ou in\'{e}gaux. \textit{(2)}\  Les corps sensibles  \textit{(a)}\ sont \^{a}pres ou in\'{e}gaux \textit{(b)}\ ont [...] in\'{e}gales \textit{L}}}
car sans cela la surface de l'un ne resisteroit pas au mouuement de l'autre, contre la premiere observation.\\
(2) Les in\'{e}galitez des surfaces
\edtext{sont flexibles}{\lemma{sont}\Bfootnote{\textit{(1)}\ sensibles \textit{(2)}\ flexibles \textit{L}}}
mais elles font ressort, et se remettent[,] t\'{e}moin le bruit qui est caus\'{e} par le frottement (2 obs.) qui ne se fait que par des \edtext{corps qui cedent,}{\lemma{corps qui}\Bfootnote{\textit{(1)}\ font ressort \textit{(2)}\ cedent, \textit{L}}} et qui se remettent subitement par leur
\edtext{ressort\protect\index{Sachverzeichnis}{ressort}.}{\lemma{}\Afootnote{\textit{Am Rand:} Se plieront\textsuperscript{[a]} par le choc ou par l'appropinquation du mobile \`{a} l'obstacle, c'est \`{a} dire\textsuperscript{[b]} par le mouuement du 
mobile $\displaystyle BC$ puisque l'obstacle $\displaystyle EF$ est en repos\textsuperscript{[c]}.\vspace{2mm}\\% PR: Marginalienapparat:
\footnotesize
\textsuperscript{[a]} plieront\ \textit{(1)}\ \`{a} proportion du choc \textit{(2)}\ par le choc ou \textit{(a)} \`{a} proportion de \textit{(b)} par l'appropinquation \textit{L}\quad
\footnotesize \textsuperscript{[b]} dire\ \textit{(1)}\ \`{a} proportion du \textit{(2)}\ par le mouuement du \textit{L} \quad
\footnotesize \textsuperscript{[c]} en repos:\ \ Siehe zum beschriebenen Sachverhalt die Zeichnung [\textit{Fig. 1}].\vspace{-8mm}}}
Outre que l'experience
\edtext{fait voir}{\lemma{fait}\Bfootnote{\textit{(1)}\ son \textit{(2)}\ voir \textit{L}}},
que tout corps a quelque duret\'{e}, et quelque
\edtext{flexibilit\'{e}, et que tout corps fait ressort}{\lemma{flexibilit\'{e},}\Bfootnote{\textit{(1)}\ aussi bien que ressort o \textit{(2)}\ qu' \textit{(3)}\ et que [...] ressort, \textit{L}}},
parce que tout corps reflechit.
Or toute la reflexion se fait par le moyen du ressort.\edlabel{35.09.11_008v_01}
\end{Geometrico}
\pstart
\noindent Un corps qui fait ressort estant en mouuement sur un plan horizontal in\'{e}branslable et rencontrant un obstacle qui fait ressort\edlabel{35.09.11_008v_02}\edtext{}{{\xxref{35.09.11_008v_01}{35.09.11_008v_02}}\lemma{du ressort.}\Bfootnote{\textit{(1)}\ Si un corps qui est  \textit{(a)}\ un \textit{(b)}\ en mouuement sur un plan horizontal, et dont la masse fait ressort rencontre un obstacle joint au plan par le moyen d'une ligature qui fait ressort \textit{(2)}\  Un corps  \textit{(a)}\ dont la masse fait ressort \textit{(b)}\ qui fait [...] in\'{e}branslable et \textit{(aa)}\ rencontre \textit{(bb)}\ rencontrant  \textit{(aaa)}\ une \'{e}minence \textit{(bbb)}\ un obstacle qui fait ressort, \textit{L}}},
et qui peut se
\edtext{plier, ou soubsmettre et remettre}{\lemma{plier,}\Bfootnote{\textit{(1)}\ et rem \textit{(2)}\ ou [...] remettre, \textit{L}}},
le retardement
\edtext{du mouuement}{\lemma{}\Bfootnote{du mouuement \textit{erg. L}}}
\edtext{sera proportionel}{\lemma{sera}\Bfootnote{\textit{(1)}\ proportional \textit{(2)}\ proportionel \textit{L}}}
\`{a} la vitesse\protect\index{Sachverzeichnis}{vitesse}.
\pend
\count\Bfootins=1000
\count\Afootins=1200
\pstart
Sur le plan $\displaystyle AB$, glisse le
\edtext{corps $\displaystyle BC$. Pour exprimer mieux dans la figure que ce corps fait ressort,}{\lemma{corps $\displaystyle BC$.}\Bfootnote{\textit{(1)}\ dans lequel \textit{(2)}\ Pour [...] ressort, \textit{L}}}
conceuuons que la
\edtext{cheville $\displaystyle CD$ est fich\'{e}e l\`{a} dedans}{\lemma{cheville $\displaystyle CD$}\Bfootnote{\textit{(1)}\ y \textit{ (2) }\ est [...] dedans, \textit{L}}},
et \edtext{mobile \`{a} l'entour du centre $\displaystyle C$ et qu'elle se remet %
par le moyen d'un ressort qui y est appliqu\'{e}. Cette}{\lemma{mobile}\Bfootnote{%
\textit{(1)}\ en $\displaystyle C$ par le moyen d'une charniere, mais %
\textit{(a)}\ qu'elle se %
\textit{(b)}\ que la dite cheville se remet apr\`{e}s avoir ced\'{e} par le moyen d'un ressort appliqu\'{e} \`{a} la dite charniere %
\textit{(2)}\ \`{a} l'entour [...] appliqu\'{e}. %
\textit{(a)}\ Et %
\textit{(b)}\ Cette \textit{L}}}
cheville rencontre un obstacle $\displaystyle EF$, qui est mobile de
\edtext{m\^{e}me [9~r\textsuperscript{o}] \`{a} l'entour du point $\displaystyle E$ et capable de se remettre par le moyen du ressort $\displaystyle EG$.}{\lemma{m\^{e}me}\Bfootnote{\textit{(1)}\   \textbar\ \`{a} l'entour du point $\displaystyle E$ \textit{erg.}\ \textbar\  et capable de se remettre par le moyen du ressort $\displaystyle EG$. Apr\`{e}s le choc les deux ressorts \textit{(2)}\ \`{a} l'entour [...] ressort $\displaystyle EG$. \textit{L}}}
\count\Bfootins=1200
\count\Afootins=1500
% \pend