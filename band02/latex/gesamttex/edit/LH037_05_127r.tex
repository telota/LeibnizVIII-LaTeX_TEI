\begin{ledgroupsized}[r]{120mm}
\footnotesize 
\pstart 
\noindent\textbf{\"{U}berlieferung:}
\pend
\end{ledgroupsized}

\begin{ledgroupsized}[r]{114mm}
\footnotesize 
\pstart \parindent -6mm
\makebox[6mm][l]{\textit{L}}Reinschrift mit Verbesserungen:
LH XXXVII 5 Bl. 127.
1 Bl. 4\textsuperscript{o}. 2 S.
Papierabbr\"{u}che an den R\"{a}ndern ohne Textverlust.
Blatt durch Papiererhaltungsma{\ss}nahmen stabilisiert.%
\\Cc 2, Nr. 965 L
\pend
\end{ledgroupsized}
%\normalsize
\vspace*{5mm}
\begin{ledgroup}
\footnotesize 
\pstart
\noindent\footnotesize{\textbf{Datierungsgr\"{u}nde}: Das vorliegende St\"{u}ck N.~37 ist eine nahezu w\"{o}rtliche Abschrift des zweiten Teils von N.~36\textsubscript{1} (S.~\refpassage{035,09,11_002r_Abschr.}{035,09,11_002v_Abschr.}).
Der abgeschriebene Text entf\"{a}llt g\"{a}nzlich in N.~36\textsubscript{2}.
Es ist daher anzunehmen, dass N.~37 in der Zeit zwischen N.~36\textsubscript{1} und N.~36\textsubscript{2} verfasst wurde.}%
\pend
\end{ledgroup}

\vspace*{8mm}
\pstart 
\normalsize
\noindent
[127~r\textsuperscript{o}] \`{A} fin de faire voir en peu de mots que le retardement uniforme selon les lieux peut avoir lieu dans le calcul du frottement:\edtext{}{\lemma{}\Afootnote{\textit{Am Rand:} Error\vspace{-6mm}}} Conceuuons que le frottement dans les corps durs, vient de l'inegalit\'{e} de leur surface \textit{AB}, c'est \`{a} dire de quelques eminences ou pointes \textit{P}, \textit{(P)} qui se peuuent plier jusqu'\`{a} \textit{p}, \textit{(p)} pour donner passage au mobile \textit{M}, quoqu'ils se remettent par leur propre ressort, quand le mobile est pass\'{e}.
\pend
\count\Bfootins=1200
\count\Cfootins=1200
\count\Afootins=1200
\pstart
Cela pos\'{e}, il est manifeste, que le mobile perd autant de sa force, qu'il en a communiqu\'{e} au ressort ou \`{a} la pointe \textit{P}. Et comme la force est compos\'{e}e de la pesanteur du corps, et de sa vitesse, il est manifeste, que le corps \textit{M} demeurant le m\^{e}me, la diminution de sa force, ne sera que celle de la vitesse.
\pend
\pstart
Or supposons \`{a} present, que le mobile \textit{M} continue son mouuement sur la m\^{e}me surface, quoyqu'avec une vitesse diminu\'{e}e et qu'il rencontre une autre pointe \edtext{\textit{(P)}}{\lemma{}\Bfootnote{\textit{(P)} \textit{erg. L}}}, semblable en tout \`{a} la premiere par ce que nous supposons la dite surface \'{e}galement \^{a}pre par tout; alors le mobile pourveu qu'il ait encor assez de force ne laissera pas de plier encor de m\^{e}me la seconde \edtext{pointe \textit{(P)}}{\lemma{pointe}\Bfootnote{\textit{(1)}\ \textit{(C)} \textit{(2)}\ \textit{(P)} \textit{L}}} pour se faire passage.
\pend
\pstart
Pour faire passage au mobile, (:~que nous supposons bien uni pour la facilit\'{e} de l'imagination ne donnant l'in\'{e}galit\'{e} qu'\`{a} la surface du corps sur le quel il marche~:) il suffit que la pointe \edtext{\textit{P} ou \textit{(P)}}{\lemma{}\Bfootnote{\textit{P} ou \textit{(P)} \textit{erg. L}}} soit pli\'{e}e jusqu'\`{a} ce qu'elle devienne parallele \`{a} la surface, (ou si elle est courbe au plan touchant) \textit{AB}.
\pend
\newpage
\pstart
C'est donc \`{a} cet \'{e}gard, que les pointes receuuront tousjours le m\^{e}me \edtext{pli, pour}{\lemma{pli,}\Bfootnote{\textit{(1)}\ par \textit{(2)}\ pour \textit{L}}} le mouuement ou passage du mobile. Or un m\^{e}me ressort (:~ou \'{e}gal~:) receuuant tousjours un m\^{e}me pli, re\c{c}oit tousjours une m\^{e}me force. Donc le mobile \textit{M} quelle vistesse qu'il puisse avoir perdra tousjours une m\^{e}me quantit\'{e} de force,% \pend