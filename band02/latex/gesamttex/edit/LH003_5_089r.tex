\begin{ledgroupsized}[r]{120mm}%
\footnotesize%
\pstart%
\noindent%
\textbf{\"{U}berlieferung:}%
\pend%
\end{ledgroupsized}%
\begin{ledgroupsized}[r]{114mm}%
\footnotesize%
\pstart%
\parindent -6mm%
\makebox[6mm][l]{\textit{L}}%
Aufzeichnung:
LH III 5 Bl. 89.
1 Bl. 4\textsuperscript{o}, an den Rändern beschnitten.
1 S. auf Bl.~89~r\textsuperscript{o}.
Bl.~89~v\textsuperscript{o} leer.
Ein Wasserzeichen.%
\newline%
Cc 2, Nr. 869%
\pend%
\end{ledgroupsized}%
%
\vspace*{5mm}%
\begin{ledgroup}%
\footnotesize%
\pstart%
\noindent%
\footnotesize{%
\textbf{Datierungsgr\"{u}nde:}
Das Wasserzeichen im Textträger des vorliegenden Stücks ist für die zweite Hälfte 1674 und den Anfang 1675 belegt.
Das gleiche Wasserzeichen kommt unter anderem im Stück N.~10 %?? = ex 8 = LH037,05_215 = De vitandis erroris geometricis in re mechanica
vor, welches von Leibniz eigenhändig auf Dezember 1674 datiert wurde.
In demselben Zeitraum hat sich Leibniz intensiv mit Mariottes Abhandlung \cite{00311}\textit{De la percussion ou chocq des corps} befasst
(siehe die Datierungsgründe von N.~50).%?? = LH035,14,02_112-115 = Excerpta ex libro Du choc des corps
 % = LH035,14,02_112-115 = Excerpta ex libro Du choc des corps
}%
\pend%
\end{ledgroup}%
%
%
\vspace*{8mm}%
\count\Bfootins=1200
\count\Cfootins=1200
\count\Afootins=1200
\pstart%
\normalsize%
\noindent%
% [89~r\textsuperscript{o}]
[89~r\textsuperscript{o}] Mons. l'Abb\'{e} de Mariotte\protect\index{Namensregister}{\textso{Mariotte}, Edme 1620-1684} a des pens\'{e}es \edtext{tres belles}{\lemma{tres}\Bfootnote{\textit{(1)} importantes \textit{(2)} belles \textit{L}}} pour la perfection de la physique. Il me dit que Mons. Du Vernette\protect\index{Namensregister}{\textso{du Vernette}, ???? ??-??} s'offre d'assister \`{a} la dissection des malades de l'hostel \edtext{Dieu, si on l'ordonne}{\lemma{Dieu,}\Bfootnote{\textit{(1)} s'il plaira l'ord \textit{(2)} si on l'ordonne \textit{L}}} qu'on en fasse. 
\pend%
\pstart%
Mons. Sanguien\protect\index{Namensregister}{\textso{Sanguien}, ????, ??-??} a peur d'estre empoisonn\'{e}, il guerit une femme abandonn\'{e}e des Medecins, par des remedes qui estoient effectivement trop violents.
\pend%
\pstart%
Mons. De Mariotte\protect\index{Namensregister}{\textso{Mariotte}, Edme 1620-1684} m'a \edtext{appris un}{\lemma{appris}\Bfootnote{\textit{(1)} une \textit{(2)} un \textit{L}}} remede fort important, qu'il a appris d'une paisane. Il avoit mal \`{a} la gorge, il y avoit comme de petits ulceres et inflammations; on luy donna quantit\'{e} de remedes inutilement, et il apprehendoit pour la suite. Une paisane par hazard dit qu'elle l'en gueriroit en moins de rien. \edtext{[Effectivement]}{\lemma{Effectifement}\Bfootnote{\textit{L \"{a}ndert Hrsg.}}} elle luy apprit un remede, \edtext{dont je}{\lemma{}\Bfootnote{dont \textbar\ que \textit{streicht Hrsg.}\ \textbar\ je \textit{L}}} diray par apres, c'estoit un \'{e}spece de Gargarisme. En se couchant il en prit; il en gargarisa la gorge, il en avalla m\^{e}me quelque chose. A minuit, il en prit quelques autres cuiller\'{e}es de m\^{e}me, et \edtext{il s'en}{\lemma{il}\Bfootnote{\textit{(1)} se \textit{(2)} s'en \textit{L}}} trouva encor mieux ayant fait cela la m\^{e}me nuit, la troisiesme \edtext{fois; le}{\lemma{fois;}\Bfootnote{\textit{(1)} il \textit{(2)} le \textit{L}}} lendemain le mal avoit cess\'{e} entierement, par une espece de merveille. La m\^{e}me \edtext{cure luy a}{\lemma{cure luy}\Bfootnote{\textit{(1)} avoit \textit{(2)} a \textit{L}}} reussi trois fois. Estant incommod\'{e} \`{a} Dijon\protect\index{Ortsregister}{Dijon}
 du m\^{e}me mal, on luy donna quantit\'{e} de choses; comme sirop de meures, etc. mais cela ne faisoit rien; on le voulut saigner et se servir d'autres remedes. Cela l'obligea d'avoir recours \`{a} son remede. \edtext{Il en fit}{\lemma{Il}\Bfootnote{\textit{(1)} fit \textit{(2)} en fit \textit{L}}} chercher. Et il fut gueri tout de m\^{e}me. La troisiesme fois \`{a} \edtext{Paris,\protect\index{Ortsregister}{Paris} o\`{u}}{\lemma{Paris,}\Bfootnote{\textit{(1)} et \textit{(2)} il \textit{(3)} o\`{u} \textit{ L}}} eut de la peine \`{a} trouver du \textso{ronce}, dont je parleray par apres; et on luy porta au lieu de cela des feuilles de rosier sauvage mais enfin on trouua du dit ronce.
\pend%
\pstart%
Le remede est tel: prenez cinq ou six feuilles de \textso{sauge} (: salvia :) et deux mains des bouts ou extremitez ou pointes des feuilles de \textso{ronce} (rubus) faites bouillir cela dans de l'eau d'\textso{orge}, ou si vous voulez avec de l'orge. Mettez dedans du bon \textso{miel de Narbonne}\protect\index{Ortsregister}{Narbonne} (:~peut estre que le succre feroit le m\^{e}me effect, car il y a des gens qui ont de l'aversion pour le miel \`{a} cause qu'on en met dans les lavements~:) et vous aurez un gargarisme, dont vous vous servirez comme j'ay dit cy dessus.
\pend%
\count\Bfootins=1500
\count\Cfootins=1500
\count\Afootins=1500
%%%% Hier endet das Stück.