\vspace*{8mm}
\pstart 
\footnotesize
\noindent In den folgenden zwei Texten diskutiert Leibniz das Problem, wie von der Schwingungs\-zahl zweier oder mehrerer Pendel auf deren L\"{a}nge geschlossen werden kann. Die St\"{u}cke N. 17\textsubscript{1} und N. 17\textsubscript{2} geben daf\"{u}r Regeln an, die f\"{u}r unter\-schiedliche Ausgangsbedingungen gelten. Ein vergleichba\-res Problem behandelt auch N.~16.
%=LH XXXV 12, 2 Bl. 62 r = De pendulorum longitudinibus
Dass darin mit denselben Rechenbeispielen operiert wird, spricht f\"{u}r eine gemeinsame Ent\-stehungs\-zeit. Dieser Befund kann sich zudem auf \"{u}bereinstimmende Wasserzeichen  st\"{u}tzen, die für das Frühjahr 1675 belegt sind. 
Zusammen mit N. 16
%=LH XXXV 12, 2 Bl. 62 r = De pendulorum longitudinibus
ist ein in \cite{00115}\title{LSB} VII, 5 N. 9 erschienenes Stück überliefert, das auf Oktober 1674 datiert wird. 
\pend
 
 

