\begin{ledgroupsized}[r]{120mm}
\footnotesize
\pstart
\noindent\textbf{\"{U}berlieferung:}
\pend
\end{ledgroupsized}
%
\begin{ledgroupsized}[r]{114mm}
\footnotesize
\pstart
\parindent -6mm
\makebox[6mm][l]{\textit{L}}Konzept: LH XXXV 9, 11 Bl. 9-12. 2 Bog. 2\textsuperscript{o}. Etwas weniger als 5 \nicefrac{1}{2} S. auf Bl.~10~r\textsuperscript{o} bis Bl.~12~v\textsuperscript{o}.
Auf Bl.~9~r\textsuperscript{o} bis Bl. 10~r\textsuperscript{o} (Z.~11)
 ist N.~34\textsubscript{2} überliefert.
Auf Bl.~12~v\textsuperscript{o} (mittig) beginnt N.~34\textsubscript{4}.
Leibniz' eigenh\"{a}ndige Datierung und Nummerierung der Bogen:
\textit{May 1675. Frottement. Part. (3)} am oberen Rand von Bl. 9~r\textsuperscript{o};
\textit{Frottement part. (4) May 1675} am oberen Rand von Bl. 11~r\textsuperscript{o}.
Gleicher Wasserzeichentypus auf Bl.~10 und Bl.~11.\\
Cc 2, Nr. 965 E, J
\pend
\end{ledgroupsized}
%
%
\vspace*{8mm}
\count\Bfootins=1200
%\pstart
%\normalsize
%\centering% PR: Das Zentrierte bitte al Überschrift gestalten.
%De Motu \edtext{uniformiter in singulis spatii punctis}{\lemma{uniformiter}\Bfootnote{\textit{(1)}\ per spatia \textit{(2)}\ in singulis spatii punctis \textit{L}}}
%mutato,\\
%qualis in\textso{ frictione\protect\index{Sachverzeichnis}{frictio} }corporis in alio corpore
%uniformiter aspero aut resistente decurrentis
%intelligi potest,
%demonstrationes Geometricae.\edlabel{35.09.11_010r.2_01}
%\pend
%\pstart
%\centering
%Essay de quelques Demonstrations Geometriques 
%De l'alteration uniforme du mouuement dans chaque point 
%de l'espace par lequel le mobile passe, comme il arrive 
%par le\textso{ frottement }du corps mobile \`{a} un autre qui est homogene ou egalement \^{a}pre\edlabel{35.09.11_010r.2_02}\edtext{}{{\xxref{35.09.11_010r.2_01}{35.09.11_010r.2_02}}
%\lemma{Geometricae.}\Bfootnote{\textit{(1)}\ Essay de quelques Demonstrations Geometriques, sur le \textso{frottement}  \textit{(a)}\ Sur le mouuement uniformement retard\'{e} dans chaque point de l'espace, par le frottement d'un corps  \textit{(aa)}\ \`{a} un autre   \textbar\ corps \textit{erg.}\ \textbar\  homogene, sur lequel il est m\^{u}.  \textit{(bb)}\ m\^{u}  \textit{(cc)}\ qui est en mouuement \`{a} un autre corps homogene.  \textit{(b)}\  Sur le \textso{Retardement} uniforme  \textit{(2)}\  Essay de [...] qui est  \textit{(a)}\ \'{e}galement \^{a}pre  \textit{(b)}\ homogenement  \textit{(c)}\ homogene [...] \^{a}pre \textit{L}}}
%tout \edtext{par tout avec l'admonition de ce}{\lemma{par tout}\Bfootnote{\textit{(1)}\ avec la proposition d'un \textit{(2)}\ avec l'admonition de ce \textit{L}}}
%qu'il y a lieu de douter de quelques
%[suppositions]\edtext{}{\Bfootnote{supposition\textit{\ L \"{a}ndert Hrsg.}}} de Galilaei\protect\index{Namensregister}{\textso{Galilei} (Galilaeus, Galileus), Galileo 1564-1642} de la descente des corps pesans.
%\pend
%\vspace{0,5em}
\pstart%
\normalsize%
\noindent%
[10~r\textsuperscript{o}]
\pend%
\count\Bfootins=1000
\count\Cfootins=1000
\count\Afootins=1000
\pstart%
\centering%
De Motu \edtext{uniformiter in singulis spatii punctis}{\lemma{uniformiter}\Bfootnote{\textit{(1)}\ per spatia \textit{(2)}\ in singulis spatii punctis \textit{L}}}
mutato,\\
qualis in\textso{ frictione\protect\index{Sachverzeichnis}{frictio} }corporis in alio corpore uniformiter aspero\\
aut resistente decurrentis intelligi potest, demonstrationes Geometricae.\edlabel{35.09.11_010r.2_01}
\pend
\vspace*{0.5em}%
\pstart
\centering
Essay de quelques Demonstrations Geometriques\\
De l'alteration uniforme du mouuement dans chaque point de l'espace\\
par lequel le mobile passe, comme il arrive par le\textso{ frottement }du corps mobile\\
\`{a} un autre qui est homogene ou egalement \^{a}pre\edlabel{35.09.11_010r.2_02}\edtext{}{{\xxref{35.09.11_010r.2_01}{35.09.11_010r.2_02}}%
\lemma{Geometricae.}\Bfootnote{\textit{(1)}\ Essay de quelques Demonstrations Geometriques, sur le \textso{frottement}  \textit{(a)}\ Sur le mouuement uniformement retard\'{e} dans chaque point de l'espace, par le frottement d'un corps  \textit{(aa)}\ \`{a} un autre   \textbar\ corps \textit{erg.}\ \textbar\  homogene, sur lequel il est m\^{u}.  \textit{(bb)}\ m\^{u}  \textit{(cc)}\ qui est en mouuement \`{a} un autre corps homogene.  \textit{(b)}\  Sur le \textso{Retardement} uniforme  \textit{(2)}\  Essay de [...] qui est  \textit{(a)}\ \'{e}galement \^{a}pre  \textit{(b)}\ homogenement  \textit{(c)}\ homogene [...] \^{a}pre \textit{L}}}
tout \edtext{par tout[,]
avec l'admonition de ce}{\lemma{par tout[,]}\Bfootnote{\textit{(1)}\ avec la proposition d'un \textit{(2)}\ avec l'admonition de ce \textit{L}}}
qu'il y a lieu de douter de quelques
[suppositions]\edtext{}{\Bfootnote{supposition\textit{\ L \"{a}ndert Hrsg.}}} de Galilaei\protect\index{Namensregister}{\textso{Galilei} (Galilaeus, Galileus), Galileo 1564-1642}\\
de la descente des corps pesans.
\pend
\vspace{0,5em}
\pstart
\noindent
\edtext{L'incomparable}{\lemma{}\Bfootnote{L'  \textit{(1)}\ illustre \textit{(2)}\ incomparable \textit{L}}}
\edtext{Galilaei\protect\index{Namensregister}{\textso{Galilei} (Galilaeus, Galileus), Galileo 1564-1642}}{\lemma{Galilaei}\Cfootnote{\cite{00048}\cite{00050}\textit{Discorsi}, Leiden 1638, S. 157f. und 163-165 (\textit{GO} VIII, S. 197f. und 202-204).}}
a raisonn\'{e} sur l'acceleration\protect\index{Sachverzeichnis}{acc\'{e}l\'{e}ration} ou retardation uniforme du mouuement dans chaque moment du temps.
Car il suppose qu'un corps pesant re\c{c}oit une nouuelle impression \'{e}gale \`{a} la premiere chaque moment du temps de sa descente.
Et il en tire des consequences tr\`{e}s belles et tr\`{e}s importantes.
\edtext{Mais il seroit \`{a} souhaiter que cette supposition se p\^{u}t demonstrer \`{a} priori,  car}{\lemma{Mais}\Bfootnote{\textit{(1)}\ bien des gens ont dout\'{e} de sa supposition; \textit{(2)}\ il seroit [...] \`{a} priori,  \textit{(a)}\ comme  \textit{(b)}\  car \textit{L}}}
si nous posons que le corps re\c{c}oit une nouuelle impression,
non pas chaque moment du temps qu'il employe \`{a} descendre,
\edtext{mais dans chaque}{\lemma{mais}\Bfootnote{\textit{(1)}\ chaque \textit{(2)}\ dans chaque \textit{L}}}
point de l'espace qu'il doit parcourir,
les consequences en seront tout autres,
s\c{c}avoir telles, que je proposeray icy.
Et j'apprehende que Galilaei\protect\index{Namensregister}{\textso{Galilei} (Galilaeus, Galileus), Galileo 1564-1642} n'ait est\'{e} forc\'{e} de preferer la premiere supposition \`{a} la seconde,
\edtext{[que]}{\lemma{}\Bfootnote{que \textit{erg. Hrsg.}}}
parce qu'il pouuoit
[assujettir]\edtext{}{\Bfootnote{assejuttir\textit{\ L \"{a}ndert Hrsg.}}}
la premiere au calcul, et que la seconde en paroissoit incapable.
\edtext{Car on ne s\c{c}avoit pas encor du temps de Galilaei certaines}%
{\lemma{Car}\Bfootnote{%
\textit{(1)}\ du temps de Galilaei %
\textit{(2)}\ on ne [...] Galilaei %
\textit{(a)}\ quelques %
\textit{(b)}\ certaines \textit{L}}}
%
\edtext{propositions de Geometrie, qui ont est\'{e} trouu\'{e}es}{\lemma{propositions}\Bfootnote{\textit{(1)}\ qui ont est\'{e} trou \textit{(2)}\ de Geometrie [...] trouu\'{e}es \textit{L}}}
%
\edtext{depuis, et sans lesquelles ceux qui voudroient raisonner sur cette seconde supposition seroient arrest\'{e}s tout court, d'abord}%
{\lemma{depuis,}\Bfootnote{%
\textit{(1)}\ et qui estoit absolument necessaires pour ne pas estre arrest\'{e} %
\textit{(2)}\ et sans [...] supposition %
\textit{(a)}\ seroit %
\textit{(b)}\ seroient arrest\'{e}s %
\textit{(aa)}\ d'abord %
\textit{(bb)}\ tout court, d'abord. \textit{L}}}%
\edtext{. Mais il est vray  qu'on dit que les experiences s'accordent passablement bien avec la supposition de Galilaei: mais la matiere meriteroit peut estre une discussion un peu plus severe par des experiences de toute sorte.}{\lemma{d'abord.}\Bfootnote{\textit{(1)}\ Je  \textit{(a)}\ veux  \textit{(aa)}\ croire  \textit{(bb)}\ bien croire  \textit{(b)}\  ne s\c{c}ay pas que les experiences s'accordent  \textit{(aa)}\ plus  \textit{(bb)}\  d'avantage avec  \textit{(aaa)}\ les  \textit{(bbb)}\ la premiere qu'avec la seconde supposition; quoyque je souhaitterois que la matiere fust un peu plus severement examin\'{e}e; \textit{(2)}\ Mais il est vray  \textit{(a)}\ que  \textit{(aa)}\ Galilaei\protect\index{Namensregister}{\textso{Galilei} (Galilaeus, Galileus), Galileo 1564-1642} asseure que les experiences se sont assez aco  \textit{(bb)}\ les experiences  \textit{(b)}\ qu'on dit [...] par des experiences  \textit{(aa)}\ differentes et faites  \textit{(bb)}\ de toute sorte. \textit{L}}}
Et \`{a} fin qu'on comprenne plus ais\'{e}ment la difference entre ces deux suppositions, je me serviray de l'exemple d'un bateau qui
\edtext{va par la repercussion de l'eau battue du plat de la rame}{\lemma{va}\Bfootnote{\textit{(1)}\ par fo \textit{(2)}\ a force de ram \textit{(3)}\ par la force\protect\index{Sachverzeichnis}{force} des bras \textit{(4)}\ par la [...] battue  \textit{(a)}\ par la rame  \textit{(b)}\ du plat de la rame. \textit{L}}}.
Imaginons nous
\edtext{que le frottement de l'eau ne deminue pas le mouuement imprim\'{e} au bateau par la rame;}{\lemma{que}\Bfootnote{\textit{(1)}\ le rameur \textit{(2)}\ celuy qui rame puisse moderer le mou \textit{(3)}\  le frottement de l'eau  \textit{(a)}\ n'empeche pas  \textit{(b)}\ ne deminue [...] imprim\'{e}  \textbar\ au bateau \textit{ erg.}\ \textbar\  par la rame; \textit{L}}}
\count\Bfootins=1200
% [10~v\textsuperscript{o}]
% \pend