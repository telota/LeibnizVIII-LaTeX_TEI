%  V E R Z E I C H N I S S E


\clearpage{\pagestyle{empty}\cleardoublepage}

\addtocontents{toc}{\vspace{5mm}}
\addpart[\large\uppercase{Verzeichnisse}]{\protect\textso{VERZEICHNISSE}}
\clearpage{\pagestyle{empty}\cleardoublepage}
\renewcommand*{\chapterpagestyle}{empty}
 	
 				    			    
% PERSONENVERZEICHNIS
\small
\setindexpreamble{\vskip4ex%
Kaiser werden unter dem Stichwort Kaiser mit nachfolgendem Namen, 
P\"{a}pste unter dem Stichwort Papst mit nachfolgendem Namen aufgef\"{u}hrt. 
Andere Regenten werden unter dem Namen des von ihnen regierten Staates gelistet. 
Bei diesen Personengruppen sind die Jahreszahlen Regierungszeiten, bei allen anderen Lebensdaten. 
Bei Autoren ist zus\"{a}tzlich das Schriftenverzeichnis heranzuziehen. 
Es wird nach Seiten zitiert. Kursive Seitenzahlen verweisen auf den Apparat-Teil.\vskip4ex}
\addcontentsline{toc}{chapter}{\enskip Personen}
\markleft{\scriptsize\uppercase{Verzeichnisse}}
\printindex{Namensregister}{PERSONEN}


% SCRIFTENVERZEICHNIS
\clearscrheadfoot
\chead{\sc{Schriften}}
\ohead[\pagemark]{\pagemark}
\begin{multicols}{2}[\footnotesize\addchap*{\hfill SCHRIFTEN\hfill}
\addcontentsline{toc}{chapter}{\enskip Schriften}
\vspace{15pt}
Das Schriftenverzeichnis enth\"{a}lt die Literatur der Leibnizzeit und die in den Erl\"{a}uterungen 
benutzte Literatur. Es wird nach Seiten zitiert. Kursive Seitenzahlen verweisen auf den Apparat-Teil.\\]
\setlength{\columnseprule}{0.4pt}					
\renewcommand*{\chapter}{\OrigChapter}
\bibliographystyle{leibniz8bib}
\footnotesize
\bibliography{l8bib}{}
\end{multicols}    


% SACHVERZEICHNIS
\setindexpreamble{\vskip4ex%
Eintr\"{a}ge in dieses Verzeichnis erfolgen in der jeweils von Leibniz verwendeten Sprache. 
Die Reihenfolge der Eintr\"{a}ge ist rein alphabetisch bestimmt, 
eine systematische Gliederung findet nicht statt. 
Es wird nach Seiten zitiert. Kursive Seitenzahlen verweisen auf den Apparat-Teil. \vskip4ex}
\addcontentsline{toc}{chapter}{\enskip Sachen}
\printindex{Sachverzeichnis}{SACHEN} 


% ORTSVERZEICHNIS
\setindexpreamble{\vskip4ex%
Dieses Verzeichnis listet alle von Leibniz genannten Ortsnamen in ihrer deutschen Version auf. 
Es wird nach Seiten zitiert. Kursive Seitenzahlen verweisen auf den Apparat-Teil.\vskip4ex}
\addcontentsline{toc}{chapter}{\enskip Orte}
\printindex{Ortsregister}{ORTE}  
       

% FUNDSTELLENVERZEICHNIS
\clearscrheadfoot
\chead{\sc{Fundstellen}}
\ohead[\pagemark]{\pagemark}
\clearpage{\pagestyle{empty}\cleardoublepage}
\setindexpreamble{\vskip4ex%
% ???? 
\vskip4ex}
\addcontentsline{toc}{chapter}{\enskip Fundstellen}
    \vspace{3.0ex}
\begin{center} \uppercase{Fundstellen}\end{center}
\footnotesize
Verzeichnet sind hier die im vorliegenden Band edierten Hand- und Druckschriften, 
geordnet nach Fundort und Signaturen.\\[2.0ex]
%\vspace{2.0ex}
\textsc{Hannover} \textit{Gottfried Wilhelm Leibniz Bibliothek}\\
%\setlength{\columnseprule}{0.4pt}
\renewcommand*{\chapter}{\OrigChapter}
\setlength\LTleft{\fill} \setlength\LTright{\fill}
\begin{longtable}{llll}
\footnotesize
LH IV & 8, 22 & Bl. 72\textendash 73 & N. 1\\
LH XXXV & 1, 12 & Bl. 12 & N. 28\\
LH XXXV & 3A, 8 & Bl. 27 & N. 64\\
LH XXXV & 3A, 16 & Bl. 19\textendash 20 & N. 52\\
LH XXXV & 5, 2 & Bl. 5\textendash 6 & N. 53\\
LH XXXV & 5, 2 & Bl. 7\textendash 9 & N. 54\\
LH XXXV & 12, 1 & Bl. 327 & N. 67\\
LH XXXV & 12, 2 & Bl. 156 & N. 9\\
LH XXXV & 14, 2 & Bl. 91\textendash 101 & N. 36\\
LH XXXV & 14, 2 & Bl. 129\textendash 134 & N. 37\\
LH XXXV & 15, 1 & Bl. 14+17 & N. 38\\
LH XXXV & 15, 6 & Bl. 20 & N. 7\\
LH XXXV & 15, 6 & Bl. 21 & N. 8\\
LH XXXV & 15, 6 & Bl. 24 & N. 10\\
LH XXXV & 15, 6 & Bl. 46+74 & N. 2\textsubscript{1}\\
LH XXXV & 15, 6 & Bl. 47\textendash 48 & N. 2\textsubscript{3}\\
LH XXXV & 15, 6 & Bl. 49\textendash 50 & N. 2\textsubscript{4}\\
LH XXXV & 15, 6 & Bl. 51\textendash 52 & N. 2\textsubscript{5}\\
LH XXXV & 15, 6 & Bl. 54\textendash 55+63 & N. 3\\
LH XXXV & 15, 6 & Bl. 56 & N. 5\\
LH XXXV & 15, 6 & Bl. 57 & N. 4\\
LH XXXV & 15, 6 & Bl. 64\textendash 65 & N. 6\textsubscript{1}\\
LH XXXV & 15, 6 & Bl. 66\textendash 73 & N. 6\textsubscript{2}\\
LH XXXV & 15, 6 & Bl. 74 & N. 2\textsubscript{2}\\
LH XXXVII & 2 & Bl. 1 & N. 15\\
LH XXXVII & 2 & Bl. 1 & N. 18\\
LH XXXVII & 2 & Bl. 2 & N. 16\\
LH XXXVII & 2 & Bl. 6 & N. 24\\
LH XXXVII & 2 & Bl. 7 & N. 23\\
LH XXXVII & 2 & Bl. 8 & N. 34\\
LH XXXVII & 2 & Bl. 9 & N. 32\\
LH XXXVII & 2 & Bl. 10 & N. 35\\
LH XXXVII & 2 & Bl. 11 & N. 20\\
LH XXXVII & 2 & Bl. 13 & N. 31\\
LH XXXVII & 2 & Bl. 16 & N. 14\textsubscript{1}\\
LH XXXVII & 2 & Bl. 17\textendash 18 & N. 14\textsubscript{2}\\
LH XXXVII & 2 & Bl. 83\textendash 91 & N. 19\\
LH XXXVII & 2 & Bl. 97 & N. 21\textsubscript{1}\\
LH XXXVII & 2 & Bl. 99\textendash 100 & N. 21\textsubscript{2}\\
LH XXXVII & 2 & Bl. 101 & N. 22\\
LH XXXVII & 3 & Bl. 90 & N. 44\\
LH XXXVII & 3 & Bl. 91\textendash 96 & N. 39\\
LH XXXVII & 3 & Bl. 96 & N. 63\\
LH XXXVII & 3 & Bl. 97\textendash 98 & N. 40\\
LH XXXVII & 3 & Bl. 99\textendash 104 & N. 41\\
LH XXXVII & 3 & Bl. 105\textendash 106 & N. 45\\
LH XXXVII & 3 & Bl. 107\textendash 112 & N. 46\\
LH XXXVII & 3 & Bl. 113\textendash 114 & N. 43\\
LH XXXVII & 3 & Bl. 128 & N. 48\textsubscript{5}\\
LH XXXVII & 3 & Bl. 129\textendash 131 & N. 48\textsubscript{4}\\
LH XXXVII & 3 & Bl. 132\textendash 134 & N. 48\textsubscript{3}\\
LH XXXVII & 3 & Bl. 135 & N. 48\textsubscript{2}\\
LH XXXVII & 3 & Bl. 136\textendash 143 & N. 50\\
LH XXXVII & 3 & Bl. 144\textendash 145 & N. 51\\
LH XXXVII & 3 & Bl. 146\textendash 147 & N. 49\textsubscript{1}\\
LH XXXVII & 3 & Bl. 148\textendash 149 & N. 49\textsubscript{2}\\
LH XXXVII & 3 & Bl. 150\textendash 151 & N. 48\textsubscript{1}\\
LH XXXVII & 4 & Bl. 71 & N. 42\\
LH XXXVIII & & Bl. 17\textendash 18 & N. 13\textsubscript{5}\\
LH XXXVIII & & Bl. 19\textendash 20 & N. 13\textsubscript{1}\\
LH XXXVIII & & Bl. 19\textendash 20 & N. 13\textsubscript{3}\\
LH XXXVIII & & Bl. 20 & N. 13\textsubscript{2}\\
LH XXXVIII & & Bl. 21 & N. 13\textsubscript{4}\\
LH XXXVIII & & Bl. 22 & N. 11\\
LH XXXVIII & & Bl. 87 & N. 66\\
LH XXXVIII & & Bl. 139 & N. 68\textsubscript{1}\\
LH XXXVIII & & Bl. 138 & N. 68\textsubscript{2}\\
LH XXXVIII & & Bl. 144 & N. 55\\
LH XXXVIII & & Bl. 158\textendash 161 & N. 70\\
LH XXXVIII & & Bl. 172 & N. 71\\
LH XXXVIII & & Bl. 188 & N. 69\\
LH XXXVIII & & Bl. 197 & N. 56\\
LH XXXVIII & & Bl. 198\textendash 199 & N. 60\\
LH XXXVIII & & Bl. 200 & N. 61\\
LH XXXVIII & & Bl. 201 & N. 57\\
LH XXXVIII & & Bl. 202 & N. 59\\
LH XXXVIII & & Bl. 226\textendash 227 & N. 65\\
\multicolumn{3}{l}{\textsc{Gerland} 1906} & N. 58\\
\multicolumn{3}{l}{Leibn. Marg. 0} & N. 26\\
\multicolumn{3}{l}{Leibn. Marg. 94} & N. 33\\
\multicolumn{3}{l}{Leibn. Marg. 105} & N. 47\\
\multicolumn{3}{l}{Leibn. Marg. 163} & N. 17\\
\multicolumn{3}{l}{Leibn. Marg. 175} & N. 27\\
\multicolumn{3}{l}{N \textendash\ A 7073} & N. 62\\
\multicolumn{3}{l}{Nm \textendash\ A 251} & N. 25\\
\multicolumn{3}{l}{Nm \textendash\ A 428} & N. 12\\
\multicolumn{3}{l}{Nm \textendash\ A 7036} & N. 29\\
\multicolumn{3}{l}{Nm \textendash\ A 7036} & N. 30
\end{longtable}
\noindent Die letzten f\"{u}nf Zeilen enthalten die Signaturen von Titeln mit Leibniz-Marginalien, die nicht durch die Signatur Leibn. Marg. als solche ausgewiesen sind.\\[1.0ex]

\begin{center} \uppercase{Erw\"{a}hnte Leibniz-Handschriften}\end{center}
Dieses Verzeichnis umfa{\ss}t die in den \"{U}berlieferungen und Erl\"{a}uterungen erw\"{a}hnten, nicht edierten Handschriften. Es ist nach Cc 2-Nummern und Handschriftensignaturen geordnet und verweist auf die Nummern des vorliegenden Bandes.\\
\begin{center}
\begin{tabular}{llll}
Cc 2, Nr. & LH, Nr. & & N.\\[0.5ex]
823 & XXXV 5,2 & Bl. 1 & 53\\
823 & XXXV 5,2 & Bl. 1 & 54\\
836 & XXXVII 5 & Bl. 57 & 67\\
836 & XXXVII 5 & Bl. 58\textendash 59& 67\\
836 & XXXVII 5 & Bl. 92\textendash 93 & 67\\
k.E. & XXXVIII & Bl. 104 & 70
\end{tabular}
\end{center}
\clearpage

\begin{center} \uppercase{Konkordanzen}\end{center}
Verzeichnet sind hier die Nummern der im \textit{Kritischen Katalog 1} bzw. \textit{Catalogue critique 2} erfassten St\"{u}cke mit Angabe der ihnen entsprechenden St\"{u}cke des vorliegenden Bandes. Zu den St\"{u}cken der Nummern 14, 19, 20, 23, 31, 32, 34, 35, 65 und 66 gibt es weder im KK 1 noch im Cc 2 einen Eintrag. Der Zusatz tlw. hinter einer der Katalognummern bedeutet, dass sich auf dem entsprechenden Texttr\"{a}ger mehrere St\"{u}cke befinden. F\"{u}r die Marginalien (N. 12, 17, 25, 26, 27, 29, 30, 33, 47 und 62) ist diese Zuordnung gegenstandslos.\\[1.0ex]
\begin{multicols}{4}[\centering\footnotesize{\uppercase{KK 1-Konkordanz}}]
\setlength{\columnseprule}{0.4pt}
\begin{tabular}{ll}
193 A & N. 2\textsubscript{1}\\
193 B & N. 2\textsubscript{2}\\
193 C & N. 2\textsubscript{3}\\
193 D & N. 2\textsubscript{4}
\end{tabular}
\columnbreak
\begin{tabular}{ll}
193 E & N. 2\textsubscript{5}\\
193 F, G, H & N. 3\\
193 J & N. 4\\
193 K & N. 5
\end{tabular}
\columnbreak
\begin{tabular}{ll}
725\textsuperscript{a,b} & N. 59\\
971\textsuperscript{b} & N. 56\\
973 A & N. 18\\
973 B & N. 15\\
\end{tabular}
\columnbreak
\begin{tabular}{ll}
973 C & N. 16\\
974\textsuperscript{a} & N. 57\\
1163 & N. 61\\
1164 & N. 60
\end{tabular}
\end{multicols}
\vspace{1.0ex}
\begin{multicols}{4}[\centering\footnotesize{\uppercase{Cc 2-Konkordanz}}]
\setlength{\columnseprule}{0.4pt}
\begin{tabular}{ll}
28 & N. 42\\
282 & N. 64\\
344 & N. 1\\
474 A, B & N. 36\\
475 & N. 37\\
476 & N. 13\textsubscript{5}\\
477 tlw. & N. 13\textsubscript{1}\\
477 tlw. & N. 13\textsubscript{2}\\
477 tlw. & N. 13\textsubscript{3}\\
478 & N. 13\textsubscript{4}\\
479 & N. 45\\
483 & N. 44\\
484 A & N. 6\textsubscript{1}
\end{tabular}
\columnbreak
\begin{tabular}{ll}
484 B & N. 6\textsubscript{2}\\
486 A tlw. & N. 39\\
486 A tlw. & N. 63\\
486 B & N. 46\\
486 C & N. 40\\
486 D & N. 41\\
487 & N. 43\\
491 A tlw. & N. 48\textsubscript{4}\\
491 A tlw. & N. 48\textsubscript{5}\\
491 B & N. 48\textsubscript{1}\\
491 C & N. 50\\
491 D & N. 51\\
491 E & N. 49\textsubscript{1}
\end{tabular}
\columnbreak
\begin{tabular}{ll}
491 F & N. 49\textsubscript{2}\\
491 G tlw. & N. 48\textsubscript{2}\\
491 G tlw. & N. 48\textsubscript{3}\\
492 A & N. 21\textsubscript{1}\\
492 B & N. 21\textsubscript{2}\\
493 & N. 22\\
494 A, B & N. 38\\
507 & N. 10\\
632 tlw. & N. 9\\
718 & N. 52\\
822 & N. 53\\
825, 826 & N. 54\\
936 & N. 28
\end{tabular}
\columnbreak
\begin{tabular}{ll}
949& N. 55\\
966 A, B & N. 68\textsubscript{1}\\
966 C & N. 68\textsubscript{2}\\
1188 & N. 71\\
1551 & N. 67\\
1554 & N. 69\\
1556 A, B & N. 7\\
1556 C & N. 8\\
1557 & N. 11\\
1558 A, B & N. 70
\end{tabular}
\vfill
\end{multicols}
\noindent Die Entsprechung von St\"{u}cknummern und KK- bzw. Cc-Nummern ist in der \"{U}berlieferung des jeweiligen St\"{u}ckes vermerkt.\\

       

% SIGLENVERZEICHNIS
\clearscrheadfoot
\chead{\sc{Siglen, Abkürzungen, Zeichen}}
\ohead[\pagemark]{\pagemark}
\clearpage{\pagestyle{empty}\cleardoublepage}
\setindexpreamble{\vskip4ex%
% ???? 
\vskip4ex}
\addcontentsline{toc}{chapter}{\enskip Siglen, Abkürzungen, Zeichen}
    \renewcommand*{\chapter}{\OrigChapter}
\vspace{3.0ex}
\footnotesize
\uppercase{1. Siglen und editorische Zeichen}\\[1.0ex]
\begin{tabular}{lp{110mm}}
\textit{E} & Erstdruck\\
\textit{E}\textsuperscript{1}, \textit{E}\textsuperscript{2} & weitere Drucke\\
\textit{L} & Leibniz, eigenh\"{a}ndig\\
\textit{l} & Leibniz, Abschrift von Schreiberhand\\
\textit{A} & Abschrift eines fremden Textes\\
\textit{LiH} & Leibniz' eigenh\"{a}ndige Bemerkungen in einem Handexemplar\\
\textit{Lil} & Leibniz' Korrekturen zu einer Abschrift von Schreiberhand\\
\textit{LiA} & Leibniz' Bemerkungen und Korrekturen in der Abschrift eines fremden Textes\\
$[~]$ & in der Datierung: erschlossenes Datum\newline im Text: Erg\"{a}nzungen des Herausgebers\newline von Leibniz gelegentlich benutzte eckige Klammern werden im Erl\"{a}uterungs\-apparat angezeigt\\
\textit{[~]} & vom Herausgeber hinzugef\"{u}gte Figurenbezeichnungen\\
\textit{(~)} & von Leibniz in Figuren seines Handexemplars hinzugef\"{u}gte Bezeichnungen\\
$\langle$\ $\rangle$ & Konjektur schwer lesbarer oder durch Besch\"{a}digung des Textzeugen ausgefallener W\"{o}rter bzw. Wortteile\\
$\langle$\textendash $\rangle$~$\langle$\textendash \textendash$\rangle$ & nicht entziffertes bzw. durch Besch\"{a}digung aus\-gefallenes Wort; die Anzahl der Striche entspricht der Anzahl der vermuteten W\"{o}rter.\\
\textit{Kursivierung} & w\"{o}rtliche oder fast w\"{o}rtliche Zitate, Buchtitel, vom Herausgeber hinzugef\"{u}gter Text. Fast w\"{o}rtlich meint geringf\"{u}gige Abweichungen ohne Signifikanz wie fl\"{u}chtige Wiedergaben der Wortfolge oder Kasus\"{a}nderungen durch Leibniz.\\
\textso{Sperrung} & Hervorhebungen von Leibniz. Die Art der Hervorhebung wird im Er\-l\"{a}uterungsapparat angezeigt.
\end{tabular}
\vspace{2.0ex}

\noindent\footnotesize{\uppercase{2. Abk\"{u}rzungen} (allgemein)}
\setlength\LTleft{0pt} \setlength\LTright{0pt}
\begin{longtable}{ll}
\footnotesize
a.a.O. & am angegebenen Ort\\
Anm. & Anmerkung\\
Aufl. & Auflage\\
Bd(e) & Band (B\"{a}nde)\\
Bl. & Blatt\\
Bog. & Bogen\\
bzw.  & beziehungsweise\\
ca & circa\\
ebd. & ebenda\\
erg. & erg\"{a}nzt\\
Fig. & Figur\\
f. & folgend\\
ff. & folgende (pl.)\\
gestr. & gestrichen\\
Hrsg. (hrsg.) & Herausgeber (herausgegeben)\\
Jh. & Jahrhundert\\
k.E. & kein Eintrag\\
LBr & HANNOVER, \textit{Gottfried Wilhelm Leibniz-Bibliothek}, Leibniz-Briefwechsel\\
LH & HANNOVER, \textit{Gottfried Wilhelm Leibniz-Bibliothek}, Leibniz-Handschriften\\
Marg. & Marginalie(n)\\
Ms. & Manuskript\\
N., Nr. & Nummer\\
Nachdr. & Nachdruck\\
o. S. & ohne Seitenangabe\\
r\textsuperscript{o} & recto\\
RS & Royal Society\\
S. & Seite\\
s.a. & siehe auch\\
s.o. & siehe oben\\
s.u. & siehe unten\\
Sp. & Spalte\\
SV & Schriftenverzeichnis\\
TD & Teildruck\\
tlw. & teilweise\\
u.a. & und andere, unter anderem\\
v. & van, von\\
Var. & Variante\\
vgl. & vergleiche\\
vermutl. & vermutlich\\
v\textsuperscript{o} & verso\\
Z. & Zeile\\
\Denarius & destilletur, distilletur (noch zu bedenken)
\end{longtable}
\vspace{2.0ex}
%\clearpage
\noindent\footnotesize{\uppercase{3. Abk\"{u}rzungen} (Schriften)}\par
\vspace{1.0ex}
%\setlength\LTleft{0pt} \setlength\LTright{0pt}
%\begin{longtable}{lp{105mm}}
\noindent\hangindent=10mm\textit{BH}: \textsc{Birch, Th.}, \textit{The History of the Royal Society of London for improving of natural knowledge: from its first rise}, London 1757.\par
\noindent\hangindent=10mm\textit{BW}: \textsc{Boyle, R.}, \textit{The Works}, hrsg. von M. Hunter und E. B. Davis, London 1999ff.\par
\noindent\hangindent=10mm Cc 2: \textit{Catalogue critique des manuscrits de Leibniz, Fascicule  II (Mars 1672\textendash Novembre 1676)}, hrsg. von A. Rivaud u.a., Poitiers 1914\textendash 1924.\par
\noindent\hangindent=10mm\textit{DO}: \textsc{Descartes, R.}, \textit{Oeuvres}, hrsg. von Ch. Adam u. P. Tannery, 12 Bde, Paris 1879\textendash 1910, 2. Aufl. ebd. 1964\textendash 1972.\par
\noindent\hangindent=10mm\textsc{Dutens}: \textsc{Leibniz, G. W.}, \textit{Opera omnia, nunc primum collecta, in classes distributa, praefationibus et indicibus exornata}, hrsg. von L. Dutens, 6 Bde, Genf 1768, Nachdr. Hildesheim 1989.\par
\noindent\hangindent=10mm\textsc{Gerland} 1906: \textsc{Leibniz, G. W.}, \textit{Nachgelassene Schriften physikalischen, mechanischen und technischen Inhalts}, hrsg. von E. Gerland, Leipzig 1906, Nachdr.: Hildesheim, New York 1995.\par
\noindent\hangindent=10mm\textit{GO}: \textsc{Galilei, G.}, \textit{Le Opere}, Edizione Nazionale, hrsg. von A. Favaro u.a., 20 Bde, Florenz 1890\textendash 1909. Neuausgabe von S. Garbasso u. Mitarbeiter, Florenz 1929\textendash 1932.\par
\noindent\hangindent=10mm\textit{GOO}: \textsc{Gassendi, P.}, \textit{Opera omnia}, 6 Bde, Lyon 1658, Nachdr.: Stuttgart-Bad Cannstatt 1964.\par
\noindent\hangindent=10mm\textit{HO}: \textsc{Huygens, Chr.}, \textit{Oeuvres compl\`{e}tes}, hrsg. von D. Bierens de Haan, J. Bosscha u.a., 22 Bde, Den Haag 1888\textendash 1950.\par
\noindent\hangindent=10mm\textit{JS}: \textit{Journal des S\c{c}avans}, Paris 1665ff.\par
\noindent\hangindent=10mm\textit{KGW}: \textsc{Kepler, J.}, \textit{Gesammelte Werke}, hrsg. von der Bayerischen Akademie der Wissenschaften, M\"{u}nchen 1923ff.\par
\noindent\hangindent=10mm KK 1: \textit{Kritischer Katalog der Leibniz-Handschriften, 1. Heft 1646\textendash 1672}, hrsg. von P. Ritter, als Manuskript ver\"{o}ffentlicht Berlin 1908.\par
\noindent\hangindent=10mm\textit{LSB}: \textsc{Leibniz, G. W.}, \textit{S\"{a}mtliche Schriften und Briefe}, Akademie Ausgabe, Darmstadt 1923ff. (seit 1954: Berlin).\par
\noindent\hangindent=10mm\textit{PO}: \textsc{Pascal, B.},	\textit{Oeuvres}, hrsg. von P. Boutroux, L. Brunschvicg, F. Gazier, 14~Bde, Paris 1904\textendash 1914, Nachdr.: Vaduz 1965.\par
\noindent\hangindent=10mm\textit{PT}: \textit{Philosophical Transactions}, London 1665ff.\par
\noindent\hangindent=10mm\textit{SPW}: \textsc{Stevin, S.}, \textit{The principal works}, hrsg. von E. J. Dijksterhuis u.a., Amsterdam 1955\textendash 1966.\par
\noindent\hangindent=10mm\textit{TO}: \textsc{Torricelli, E.}, \textit{Opere}, hrsg. von G. Loria, G. Vassura, 4 Bde, Faenza 1919\textendash 1944.\par
\noindent\hangindent=10mm\textit{WO}: \textsc{Wallis, J.}, \textit{Opera mathematica}, 3 Bde, Oxford 1693\textendash 1699, Nachdr.: Hildesheim 1972.\par
%\end{longtable}
\vspace{3.0ex}

\noindent\footnotesize{\uppercase{4. Symbole und Zeichen}}
\setlength\LTleft{0pt} \setlength\LTright{0pt}
\begin{longtable}{lp{100mm}}
\footnotesize
\earth & Antimon\\
\saturn & Blei (Saturn)\\
\mars & Eisen (Mars)\\
\astrosun & Gold (Sonne)\\
\mercury & Quecksilber (Merkur)\\
\Pfund & Pfund\\
\rightmoon & Silber (Mond)\\
\jupiter & Zinn (Jupiter)\\
\protect\includegraphics[width=0.02\textwidth]{images/vitriol.pdf} & Vitriol\\
\protect\includegraphics[width=0.02\textwidth]{images/salpeter.pdf} & Salpeters\"{a}ure\\
\protect\includegraphics[width=0.02\textwidth]{images/taros.pdf} & Weinstein\\
$\bigtriangleup$ & Wasser\\
\protect\includegraphics[width=0.02\textwidth]{images/oleum.pdf} & \"{O}l\\
$\smallfrown$ & Multiplikation\\
$\smallsmile$ & Division\\
$\bigtriangledown$ & Dreieck\\
\rule{1pt}{3mm} & K\"{u}rzung eines Bruchs\\
$f$ & facit\\
$\square$ \fbox{2} & Quadrat\\[0.5ex]
\fbox{3}~~cub. & Kubus\\[0.5ex]
$\surd~~~\sqrt{~~~}$ & Quadratwurzel\\
 =, aequ., aeq., $\sqcap$ & gleich\\
$\raisebox{0.8pt}{$\urcorner$} \hspace{-5.5pt}|\hspace{3pt}$ & gr\"{o}{\ss}er als\\
, ,, ,,, $\llcorner \lrcorner$ & Klammerausdr\"{u}cke\\
\ovalbox{\makebox[15mm][l]{~~~}} & Umrahmungen zur Bezeichnung wegfallender Terme\\
\leibdashv & Vorzeichen plus minus\\
\leibvdash & Vorzeichen minus plus\\
... & Platzhalter f\"{u}r Terme
\end{longtable}


% ENDE DER VERZEICHNISSE