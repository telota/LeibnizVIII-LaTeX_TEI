\begin{ledgroupsized}[r]{120mm}%
\footnotesize%
\pstart%
\noindent\textbf{\"{U}berlieferung:}%
\pend%
\end{ledgroupsized}%
\begin{ledgroupsized}[r]{114mm}%
\footnotesize%
\pstart%
\parindent -6mm%
\makebox[6mm][l]{\textit{L}}%
Notiz: LH XXXVII 4 Bl. 49-50.
1 Bog. 2\textsuperscript{o}.
4 Z. am unteren Rand von Bl.~50~v\textsuperscript{o}.
Der übrige Teil von Bl.~50~v\textsuperscript{o} überliefert N.~43. % LH037_04_050v_text1 = Omne flexile naturale
Bl.~50~r\textsuperscript{o} ist leer.
Bl.~49 überliefert N.~42. % LH037_04_49rv = Cc2 972 = Demonstratio de trabis aequilibrio
Ein Wasserzeichen auf Bl.~49.%
\newline%
Cc 2, Nr. 973 (tlw.)%
\pend%
\end{ledgroupsized}%
%
\vspace*{5mm}%
\begin{ledgroup}%
\footnotesize%
\pstart%
\noindent%
\footnotesize{\textbf{Datierungsgr\"{u}nde:}
Das Wasserzeichen im Textträger des vorliegenden Stücks ist für die Zeitspanne vom September 1672 bis zum März 1673 belegt
(siehe die Datierungsgründe von N.~42). % LH037_04_49rv = Cc2 972 = Demonstratio de trabis aequilibrio
}%
\pend%
\end{ledgroup}%
%
%
\vspace*{8mm}%
\pstart%
\normalsize%
\noindent%
% [50~v\textsuperscript{o}]
[50~v\textsuperscript{o}]
\edtext{Mittel\edtext{ einen}{\lemma{Mittel}\Bfootnote{\textit{(1)} die \textit{(2)} einen \textit{ L}}}
warmen wind zu machen.
Man m\"{u}ste ein gef\"{a}{\ss} ubern Ofen hengen, so ledig, darein kalte lufft\protect\index{Sachverzeichnis}{kalte Luft} von aussen gehet, so die warme\protect\index{Sachverzeichnis}{warme Luft} darinn forttreibt. Wo aber die kalte\protect\index{Sachverzeichnis}{K\"{a}lte} von rarefaction der warmen zur\"{u}ckgestossen wird, und es nicht angehet, so nehme man so einen hafen voll wasser, so sich rarefaciret, la{\ss} es durch gewichst oder fett gemacht.%
}{\lemma{Mittel [...] gemacht}\Cfootnote{Leibniz bezieht sich vermutlich auf \cite{00041}\textsc{C.~Drebbel}, \textit{Ein kurzer Tractat von der Natur der Elementen}, Leiden 1608, S.~8.}}
\pend%
%%%%  PR: Hier endet das Stück.