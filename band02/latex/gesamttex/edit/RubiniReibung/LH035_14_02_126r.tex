\edtext{$\displaystyle \frac{m^2 + lm + l^2}{m + l, \, \protect\boxed{\protect\phantom{\scriptstyle{2}}}} vl$. [126~r\textsuperscript{o}] quod momentum rursus dividendum}{\lemma{$\displaystyle \frac{m^2 + lm + l^2}{m + l, \, \protect\boxed{\protect\phantom{\scriptstyle{2}}}} vl$.}\Bfootnote{\textit{(1)}\ quod rursus dividendo \textit{(2)}\ quod momentum rursus  \textit{(a)}\ multiplicandum \textit{(b)}\ dividendum \textit{L}}}
est per $\displaystyle l + m$. fiet: \rule[-4mm]{0mm}{10mm}$\displaystyle \frac{m^2 + lm + l^2}{m + l, \, \boxed{3}} lv$.
\protect\renewcommand{\arraystretch}{1.2} 
\edtext{}{
	\lemma{}\Afootnote{
		\textit{Nebenrechnung am Rand:}
		$\displaystyle 
			\protect\begin{array}[t]{r} 
				2ml \, + \, l^2 
			\\
				\ m \, + \, l^{\protect \phantom{2}}
			\\
				\cline{1-1} \vspace{1.3mm} $\protect\raisebox{-1mm}{$+ \ 2m^2l \, + \, 3ml^2 \ [+] \, l^3$}$
				% K.Zeitz: Die folgende Zeile bewirkt nichts. Fußnote in Fußnote ist nicht möglich, da sich die Fußnoten ja immer auf Zeilennummern beziehen. Vorschlag: In Extrazeile hinter die Tabelle in die AFootnote.
%\edtext{}{\lemma{$\displaystyle [+]$}\Bfootnote{ \textit{erg. Hrsg.}}}
			\\
				\cline{1-1} + \ 1\protect \phantom{\ m^2l \, + \, 3ml^2 \, [+] l^3} 
			\protect \end{array}
		$
		\\
		$\displaystyle [+]$ \textit{erg. Hrsg.}
	}
}
auferendo ab $\displaystyle v$. restabit:
$\displaystyle \frac{{\displaystyle{m^3} + \atop} \overset{2}{\raisebox{-4.5mm}{{\def\firstcircle{(0,0) circle (0.45cm)}\begin{tikzpicture}\draw \firstcircle node {$\displaystyle {3 \atop -1}$};\end{tikzpicture}}}} {\displaystyle{m^2l} \atop \displaystyle{\dots}} \overset{2}{\raisebox{-4.5mm}{{\def\firstcircle{(0,0) circle (0.45cm)}\begin{tikzpicture}\draw \firstcircle node {$\displaystyle {+3 \atop -1}$};\end{tikzpicture}}}} {\displaystyle{l^2m} \atop} \raisebox{-4.5mm}{{\def\firstcircle{(0,0) circle (0.45cm)}\begin{tikzpicture}\draw \firstcircle node {$\displaystyle {+l^3 \atop -}$};\end{tikzpicture}}}}{m + l, \, \boxed{3}} [v]$
\edtext{.}{\lemma{$[v]$}\Bfootnote{\textit{erg. Hrsg.}}}
%
Imo ne erremus celeritates\protect\index{Sachverzeichnis}{celeritas} incussae simul addendae sunt, non singulatim 
considerandae, nempe prima erat \rule[-4mm]{0mm}{10mm}$\displaystyle \frac{l}{l + m} v$.
secunda est 
%
%neue Seite
%
\rule[-4mm]{0mm}{10mm}$\displaystyle \frac{ml}{l + m, \, \boxed{\phantom{\scriptstyle{2}}}} v$.
summa utriusque: 
\rule[-4mm]{0mm}{10mm}$\displaystyle \frac{2ml + l^2}{l + m, \, \boxed{\phantom{\scriptstyle{2}}}} v$ 
\edtext{
	seu 
	\edtext{$\displaystyle \frac{2l}{l + m}$}{
		\lemma{$\displaystyle \frac{2l}{l + m}$}
		\Cfootnote{
			Die vorgenommene K\"{u}rzung des Bruches ist nicht m\"{o}glich. Richtig hei{\ss}t es: 
			$\displaystyle \frac{(2ml + l)l}{(l+m)^2}$
		}
	}
	summa ictuum\protect\index{Sachverzeichnis}{ictus}
}{
	\lemma{}
	\Bfootnote{seu [...] ictuum \textit{erg.} \textit{L}}
} 
auferenda ab $\displaystyle v$, dabit:
\rule[-4mm]{0mm}{10mm}$\displaystyle \frac{l^2 + 2ml + m^2 - 2ml - l^2}{l + m, \, \boxed{\phantom{\scriptstyle{2}}}}[, v]
\edtext{}{
	\lemma{$[, v]$}
	\Bfootnote{\textit{erg. Hrsg.}}
} 
\ \sqcap \ \frac{m}{l + m} \ \boxed{\phantom{\scriptstyle{2}}} \,, \, v$.
Multiplicetur per \rule[-4mm]{0mm}{10mm}$\displaystyle \frac{l}{l + m}$
\edtext{fiet $\displaystyle \frac{m^2 l}{\protect\boxed{3} \ l + m}[v]$}{
	\lemma{fiet}
	\Bfootnote{ 
		\textit{(1)}\ cubus ab \textit{(2)}\ $\displaystyle \frac{m^2 l}{\protect\boxed{3} \ l + m}[v]$ \textit{L}
	}
}
\edtext{}{
	\lemma{$[v]$}\Bfootnote{\textit{erg. Hrsg.}}
}
%
% RUBINI: Um den richtigen Zeilenzähler (siehe unten, Z. 48) zu bekommen, sollte man hier den \edtext-Befehl 
% einsetzen, der sich jetzt in Z. 42 befindet. Das klappt aber nicht. 
% Alternativ könnte man mit folgendem Befehl operieren: \edlabel{one}
% Dieser Befehl verweist auf \edlabel{two} etc. in Z. 46 (auch damit funktioniert es aber nicht)
%
qui est ictus tertii liquidi. Addatur ille ad
\rule[-4mm]{0mm}{10mm}$\displaystyle \frac{2ml + l^2}{l + m, \, \boxed{\phantom{\scriptstyle{2}}}}[v]$
\edtext{}{
	\lemma{$[v]$}\Bfootnote{\textit{erg. Hrsg.}}
}
fiet: $\displaystyle \frac{3ml^2 + 3m^2 l + l^3}{l + m, \, \boxed{3}} v$
\edtext{summa ictuum}{
	\lemma{}\Bfootnote{summa ictuum \textit{erg.} \textit{L}}
}
. Auferatur ab $\displaystyle v$,\rule[-4mm]{0mm}{10mm} restabit: 
\rule[-4mm]{0mm}{10mm}$\displaystyle 
\frac{
	\underset{\scriptscriptstyle{I\!\!I}}{\smash[b]{\ovalbox{$l^3\!$}}} \ \ovalbox{$\!+\,3l^2m$} + 
	\underset{\scriptscriptstyle{I}}{\smash[b]{\ovalbox{\vphantom{$3^2$}$3$}}} 
	lm^2 + m^3 \, \ovalbox{$\!-\,3ml^2$} \ 
	\underset{\scriptscriptstyle{I}}{\smash[b]{\ovalbox{$\!-\,[3]m^2l$}}} \ 
	\underset{\scriptscriptstyle{I\!\!I}}{\smash[b]{\ovalbox{$\!-\,l^3\!$}}}
}{
	l + m, \, \boxed{3}}
v$
\edtext{,}{
	\lemma{}
	\Bfootnote{$-\,2m^2l$ \textit{\ L \"{a}ndert Hrsg.\ }}
}
%
%Zeile 5
%
\rule[-4mm]{0mm}{10mm}\edtext{ cubus ab $\displaystyle \frac{m}{l + m}$}{
	\lemma{
		$\displaystyle 
		\frac{
			\protect\underset{\scriptscriptstyle{I\!\!I}}{\protect \smash[b]{\protect \ovalbox{$l^3\!$}}} \ 
			\protect \ovalbox{$\!+\,3l^2m$} + \protect \underset{\scriptscriptstyle{I}}
			{\protect \smash[b]{\protect \ovalbox{\protect \vphantom{$3^2$}$3$}}} 
			lm^2 + m^3 \, \protect \ovalbox{$\!-\,3ml^2$} \ 
			\protect \underset{\scriptscriptstyle{I}}{\protect \smash[b]{\protect \ovalbox{$\!-\,[3]m^2l$}}} \ 
			\protect \underset{\scriptscriptstyle{I\!\!I}}{\protect \smash[b]{\protect \ovalbox{$\!-\,l^3\!$}}}
		}{
			l + m, \, \protect \boxed{3}
		}
	v$,
	}
	\Bfootnote{\textit{(1)}\ sive $\displaystyle \frac{m^2}{l + m, \, \protect \boxed{2}}$ 
		\textit{(2)}\ cubus ab $\displaystyle \frac{m}{l + m}$ \textit{L}
	}
}
. Multiplicetur per $\displaystyle \frac{l}{l + m}$ fiet $\displaystyle \frac{m^3 l}{l + m, \, \boxed{4}}[v]$
\edtext{}{
	\lemma{$[v]$}
	\Bfootnote{\textit{erg. Hrsg.}}
}
qui est ictus quarti liquidi. 
%Zeile 6
Addatur ad 
\rule[-4mm]{0mm}{10mm}$\displaystyle \frac{3ml^2 + 3m^2 l + l^3}{l + m, \, \protect \boxed{3}}$, fiet:
\\
%Zeile 7
\rule[-4mm]{0mm}{10mm}$\displaystyle \renewcommand{\arraystretch}{1.2} 
\begin{array}{r} 
	\hspace{70pt} 3ml^2 \, + \ 3m^2 l \, + \ l^3 
	\\
	m \ + \ l^{\phantom{3}} 
\end{array} 
\\
%Zeile 8
\renewcommand{\arraystretch}{1.4} 
\begin{array}{llll} 
		\hline + \ 3m^3l 
		& 
		+ \ 3m^2l^2 
		& 
		+ \ 3ml^3 & \! 
\edtext{[+]}{\lemma{???$\displaystyle [+]$}\Bfootnote{\textit{erg. Hrsg.}}} % RUBINI: Falscher Zeilenzähler
%ZEITZ technisch zählt LaTex hier korrekt, diese Fußnote befindet sich in Zeile 8. 
		\ l^4 
	\\
		& 
		+ \ 3 \dots 
		& 
		+ \ 1 \dots 
		& 
	\\
		\hline + \ 1m^3l 
		& \ 
		& \ 
		& 
	\\
		\hline \hline 
\end{array}
\Bigg\} 
\begin{array}{r} 
	$summa ictuum
	\edtext{}{
		\lemma{}
		\Afootnote{\textit{Neben der Rechnung:} NB}
	}$ 
\end{array} 
\\ %leere Zeilen werden nicht gezählt
\\
%Zeile 9
\protect \mbox{quadrato-quadratus ab} \ m + l \ \protect \mbox{demto quadratoquadrato ab} \ m$
% RUBINI: Hier sollte man möglicherweise mit folgendem Befehl operieren, der an Z. 21 anknüpft:
% \edlabel{two}{\xxref{one}{two}}{\lemma{liquidi.}\Bfootnote{ \textit{ (1) }\ Addatur ad $\displaystyle \frac{3ml^2 + 2m^2 l + l^3}{l + m, \, \protect \boxed{3}}$,
% fiet $\displaystyle \frac{3ml^2 + 3m^2 l + l^3}{l + m, \, \protect \boxed{3}}$.
% auferatur ab $\displaystyle v$,
% restabit $\displaystyle \frac{\protect \begin{array}{l}\ \ l^3 + 3ml^2 + 3m^2 l + m^3 \\ - l^3 - 3ml^2 - 3m^2 l \quad \ \quad \protect \end{array}}{l + m, \, \protect \boxed{3}} \ \sqcap \ \protect \mbox{cubus ab} \ \frac{m}{\protect \phantom{l + m}}
% $ \textit{ (2) }\ Addatur ad [...] quadratoquadrato ab $\displaystyle m$ \textit{ L}}}
\edtext{.}{\lemma{???
% RUBINI: Falscher Zeilenzähler
%ZEITZ auch diese Fußnote ist meiner Ansicht nach richtig gezählt. Sie bezieht sich auf das "m." in Zeile 9
liquidi.}\Bfootnote{\textit{(1)}\ Addatur ad $\displaystyle \frac{3ml^2 + 2m^2 l + l^3}{l + m, \, \protect \boxed{3}}$,
fiet $\displaystyle \frac{3ml^2 + 3m^2 l + l^3}{l + m, \, \protect \boxed{3}}$.
auferatur ab $\displaystyle v$,
restabit $\displaystyle \frac{\protect \begin{array}{l}\ \ l^3 + 3ml^2 + 3m^2 l + m^3 \\ - l^3 - 3ml^2 - 3m^2 l \quad \ \quad \protect \end{array}}{l + m, \, \protect \boxed{3}} \ \sqcap \ \protect \mbox{cubus ab} \ \frac{m}{\protect \phantom{l + m}}
$ \textit{(2)}\ Addatur ad [...] quadratoquadrato ab $\displaystyle m$ \textit{L}}}\edtext{}{\lemma{quadrato-quadratus [...] ab $m$}\Cfootnote{Das richtige Ergebnis lautet: $\displaystyle \frac{(m + l)^4 - m^4}{(m + l)^4} v$.}}
Haec summa ictuum 
%
%Zeile 10
%
auferatur ab $\displaystyle v.$
restabit quadratoquadratus ab \rule[-4mm]{0mm}{10mm}$\displaystyle \frac{m}{m + l}[v]$\edtext{.}{\lemma{$[v]$}\Bfootnote{\textit{erg. Hrsg.}}}
qui si per $\displaystyle \frac{l}{m + l}$ multiplicetur, ictus quinti liquidi \edtext{habebitur. Nec opus est calculum}{\lemma{habebitur.}\Bfootnote{\textit{(1)}\ Ex hoc jam ca \textit{(2)}\ Nec [...] calculum \textit{L}}}
ultra continuari. Praeclare enim 
%
%neue Seite
%
\edtext{demonstratur ictus, ac proinde et ictuum summas progressionis}{\lemma{demonstratur}\Bfootnote{\textit{(1)}\ tum ictus, tum ictuum summas progressionis \textit{(2)}\ ictus [...] progressionis \textit{L}}}
Geometricae esse.
\pend
\pstart
Tota haec demonstratio multo fit clarior si mobili soli motum demus, hoc enim et ab ictibus retardatur. Sed inde statim alia praeclarissima sequitur demonstratio, quod quae vento \edtext{secundo}{\lemma{}\Bfootnote{secundo \textit{erg.} \textit{L}}}
ferantur, eorum motus crescant progressione \edtext{geometrica quorum logarithmi 
%
%Zeile 5
%
sint spatia;}{\lemma{geometrica}\Bfootnote{\textit{(1)}\ ; secundum \textit{(2)}\ quorum [...] spatia; \textit{L}}}
et quod \edtext{quae contra ventum eant}{\lemma{quae}\Bfootnote{\textit{(1)}\ vento feruntur \textit{(2)}\ contra ventum eant \textit{L}}};
aut fluvium eadem ratione retardentur. Secus ac \edtext{creditum est a doctissimis viris,}{\lemma{creditum [...] viris}\Cfootnote{???? Quelle nicht nachgewiesen. ????}} qui vento flante eadem progressione accelerari credidere \edtext{motum, eorum quae feruntur vento, qua}{\lemma{motum,}\Bfootnote{\textit{(1)}\ quo \textit{(2)}\ eorum [...] vento,  \textit{(a)}\ quo \textit{(b)}\ qua \textit{L}}}
gravium\protect\index{Sachverzeichnis}{grave} celeritas descensu augetur. Caeterum sciendum \edtext{est si}{\lemma{est}\Bfootnote{\textit{(1)}\ eousque \textit{(2)}\ si \textit{L}}}
hanc sequamur demonstrationem, \edtext{nunquam}{\lemma{}\Bfootnote{nunquam  \textbar\ absolute \textit{gestr.}\ \textbar\ perveniri \textit{L}}}
perveniri a corpore mobili ad eam celeritatem quae est ipsius venti, seu ad celeritatem $\displaystyle v$, quia \edtext{semper summa}{\lemma{semper}\Bfootnote{\textit{(1)}\ summae \textit{(2)}\ summa \textit{L}}}
%
%Zeile 10
%
ictuum est ad $\displaystyle v$, ut potestas quaedam ab $\displaystyle m + l$. demta ejusdem gradus potestate ab $\displaystyle m$, est ad integram potestatem quam dixi ab $\displaystyle m + l$.
Unde illud quoque elegantissimum sequitur, nunquam finiri motum corporis a liquidi resistentia\protect\index{Sachverzeichnis}{resistentia}, etsi continue retardetur. Idque verum est, sive liquidum perfecte fluidum esse, sive ex ramentis ac quadam pulverea congerie constare ponamus.
\pend
\pstart
%
%Zeile15
%
Sed si ponamus motum ipsius \edtext{mobilis (vel quod idem est liquidi, mobili quiescente) secundum tempora accelerari}{\lemma{mobilis}\Bfootnote{\textit{(1)}\ secundum tempora acceleratum \textit{(2)}\ (vel [...] accelerari \textit{L}}},
tunc si secundum spatia sumatur, crescet in subduplicata spatiorum ratione. Contra si sumamus secundum \edtext{tempora, tunc celeritatibus in eodem tempore}{\lemma{tempora,}\Bfootnote{\textit{(1)}\ in quo \textit{(2)}\ tunc [...] tempore \textit{L}}}
uniformiter crescentibus, spatia erunt in duplicata ratione temporum seu celeritatum quaesitarum. Jam \edtext{si sit constans quaedam celeritas uniformis}{\lemma{si}\Bfootnote{\textit{(1)}\ celeritates sint uniformes diminution \textit{(2)}\ sit [...] uniformis \textit{L}}},
ostensum est 
%
%Zeile 20
%
si decrementa celeritatis sint ut numeri, spatia percursa fore ut logarithmos. Cum ergo celeritas accelerata composita intelligi possit ex \edtext{uniformibus, quarum unam notavi signo 1. secundam signo 2, tertiam 3. etsi singulae inter se sint aequales[,] intelligi potest}{\lemma{uniformibus,}\Bfootnote{\textit{(1)}\ videamus ecce: \textit{(2)}\ quarum  \textit{(a)}\ una sit  \textit{(b)}\ unam [...] aequales  \textbar\ itaque \textit{gestr.}\ \textbar\  intelligi potest \textit{L}}}
quamlibet ex illis separatim diminui in ratione secundum spatia multiplicata.\\
$\displaystyle 1\\
1 \quad 2\\
1 \quad 2 \quad 3\\
1 \quad 2 \quad 3 \quad 4\\
1 \quad 2 \quad 3 \quad 4 \quad 5$
\pend
\pstart
%
%Neue Seite
%
Nota si decrementa celeritatis sunt progressionis Geometricae non ideo residuae progressionis Geometricae erunt, summae tamen diminutionum erunt progressionis Geometricae, ergo et residuae celeritates erunt progressionis Geometricae terminis a certa quantitate ademtis residui. Ergo si residuae celeritates sint progressionis Arithmeticae, 
%
%Zeile 5
%
erunt spatia ut Logarithmi non residuarum quidem celeritatum sed perditarum, seu differentiae inter quantitatem constantem et residuas celeritates. Temporum ergo incrementa tunc sunt ut applicatae hyperbolae, \edtext{ergo tempora}{\lemma{ergo}\Bfootnote{\textit{(1)}\ temporum incrementa \textit{(2)}\ tempora \textit{L}}}
ipsa erunt ut Logarithmi. Ergo tempora certa quadam quantitate excepta, erunt spatiis proportionalia. Nisi forte aliud prodeat hyperbolae et logarithmorum genus, quod necessarium credo, ne motus fiat uniformis, quod foret absurdum.
\pend