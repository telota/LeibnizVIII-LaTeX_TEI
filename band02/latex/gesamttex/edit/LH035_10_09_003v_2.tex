\begin{ledgroupsized}[r]{120mm}
\footnotesize 
\pstart 
\noindent\textbf{\"{U}berlieferung:}
\pend
\end{ledgroupsized}
\begin{ledgroupsized}[r]{114mm}
\footnotesize 
\pstart \parindent -6mm
\makebox[6mm][l]{\textit{L}}%
Reinschrift mit Verbesserungen:
LH XXXV 10, 9 Bl. 3-4. 1 Bog. 2\textsuperscript{o}. \nicefrac{1}{2} S. auf Bl.~3~v\textsuperscript{o}. Der Bogen überliefert ferner N.~28\textsubscript{1}, N.~28\textsubscript{5}, N.~28\textsubscript{6} und N.~5.
%LH35,10,09 Bl. 3r = Demonstratio geometrica de magnetis sphaera
\\
Cc 2, Nr. 1190 C
\pend
\end{ledgroupsized}

%%\normalsize
%\vspace*{5mm}
%\begin{ledgroup}
%\footnotesize 
%\pstart
%\noindent\footnotesize{\textbf{Datierungsgr\"{u}nde}: bislang nur relative Chronologie etabliert, absolute Datierung noch ausstehend.}
%\pend
%\end{ledgroup}
\vspace*{8mm}
\count\Bfootins=1200
\count\Afootins=1200
\pstart 
\normalsize
\noindent
[3~v\textsuperscript{o}]
\pend
\pstart
\centering
\noindent
Regle \edtext{}{\lemma{}\Afootnote{\textit{Am Rand}: Mieux conceu.\vspace{-4mm}}} pour calculer la force d'une Machine\protect\index{Sachverzeichnis}{machine}\protect\index{Sachverzeichnis}{force}\\%
\edtext{dont voicy la figure}{\lemma{dont voicy la figure}\Cfootnote{Siehe [\textit{Fig. 1}] in N.~28\textsubscript{1}.}}
\pend
\vspace*{1.0em}
\pstart 
%\normalsize
%{\centering[3~v\textsuperscript{o}] Regle\edtext{}{\lemma{}\Afootnote{\textit{Daneben am linken Rand}: Mieux conceu.}}\\ pour calculer la force\protect\index{Sachverzeichnis}{force} d'une\\Machine dont voicy la figure\\} 
\noindent
Soit la roue \textit{ABCD}, mobile \`{a} l'entour du Centre \textit{L}, entrecoup\'{e}e \`{a} angles droits de deux diametres solides \textit{AC}, et \textit{DB}, lesquels seront transferez par le mouuement \`{a} l'entour du Centre de la situation perpendiculaire ou horizontale \textit{ABCD} \`{a} l'inclin\'{e}e \textit{EFGH}, dans un angle \textso{donn\'{e}} \textit{ALE}.
\pend 
\pstart Conceuuons la dite roue charg\'{e}e dans les points \textit{E, F, K, I} de quatre poids egaux entre eux.
\pend 
\pstart Soit \textso{donn\'{e}e} la longueur de \textit{AC} diametre de la roue item la longeur des droites \textit{ELK}, et \textit{FLI} egales entre elles. 
\pend 
\pstart Et enfin la force absolue\protect\index{Sachverzeichnis}{force absolue} d'un de ces poids\protect\index{Sachverzeichnis}{poids}, c'est \`{a} dire avec laquelle il agit librement ou sur un plan parallele \`{a} l'horison, s'il en estoit so\^{u}tenu.\pend \pstart On demande la force de la machine\protect\index{Sachverzeichnis}{machine}, qu'elle auroit dans l'Estat \textit{EFGH} si elle y commenceroit le mouuement, car il faut adjouter cette \edtext{condition}{\lemma{cette}\Bfootnote{\textit{(1)}\ copie \textit{(2)}\ condition \textit{L}}}, afin de ne pas embarasser le calcul de la force\protect\index{Sachverzeichnis}{force} simple par celuy de la force gagn\'{e}e par l'acceleration, dont le calcul\protect\index{Sachverzeichnis}{calcul} se doit faire \`{a} part.
\pend 
\pstart Des points \textit{E.F} menez les perpendiculaires, \textit{EM}, \textit{FN} sur le diametre vertical \textit{AC}, lesquelles \edtext{seront donn\'{e}es}{\lemma{seront}\Bfootnote{\textit{(1)}\ tout autres \textit{(2)}\ seront \textit{(3)}\ donn\'{e}es \textit{L}}}, par ce \edtext{que les}{\lemma{que}\Bfootnote{\textit{(1)}\ dit \textit{(2)}\ les \textit{L}}} Angles \textit{ALE}, \textit{CLF} sont donn\'{e}s, dont elles sont les sinus droits.
\pend 
\newpage
\pstart Cela estant pos\'{e}, je dis que \textso{la force absolue} d'un des poids\protect\index{Sachverzeichnis}{poids} susdits est \textso{\`{a} la force de la machine\protect\index{Sachverzeichnis}{machine}}, comme \edtext{le rectangle}{\lemma{}\Bfootnote{le \textbar\ le \textit{streicht Hrsg.} \textbar\ rectangle \textit{L}}} \textit{ELK} (: ou compris soubs \textit{EL}, \textit{LK}~:) est au rectangle compris soubs \textit{HI} et \textit{MN}. Theoreme\protect\index{Sachverzeichnis}{theoreme} assez beau et d'un grand usage pour le calcul des mouuements circulaires.
\pend 
\pstart Pour donner cette raison en nombres, il faut se servir des lettres de l'Analyse\protect\index{Sachverzeichnis}{analyse}, qui signifient des nombres indefinis. 
\pend 
\count\Bfootins=1500
\count\Afootins=1500
\pstart%
Soit
\begin{tabular}[t]{lr}
le sinus droit \textit{EM}&appell\'{e} \textit{y}\hspace{0.1mm}\\
le Rayon \textit{AL}&\dotfill \textit{a}\\
le petit Rayon \textit{LI}&\dotfill \textit{b}\\
\end{tabular}
\pend
\vspace{1mm}
\pstart\noindent%
et\setline{10} la force absolue\protect\index{Sachverzeichnis}{force absolue} d'un des poids\protect\index{Sachverzeichnis}{poids},
sera \`{a} la force de la Machine\protect\index{Sachverzeichnis}{Machine}, comme est 1, ou l'unit\'{e}, \`{a} \rule[-4mm]{0mm}{12mm}$\displaystyle \frac{y + \sqrt{a^2 - y^2}, \smallfrown a - b}{ba}$ ou comme 1 \`{a} \rule[-4mm]{0mm}{12mm}$\displaystyle \frac{ay - yb + a\sqrt{a^2 - y^2} - b\sqrt{a^2 - y^2}}{ba}$. Enfin si l'on demande la raison de la \edtext{force de la}{\lemma{}\Bfootnote{force de la \textit{erg.} \textit{L}}} machine dans l'inclination ou angle \edtext{\textit{ALE}, \`{a}}{\lemma{\textit{ALE},}\Bfootnote{\textit{(1)}\ dans \textit{(2)}\ \`{a} \textit{L}}} celle qu'elle aura dans un autre angle \edtext{\textit{ALP}, la regle deviendra}{\lemma{\textit{ALP},}\Bfootnote{\textit{(1)}\ je dis que l \textit{(2)}\ la regle deviendra \textit{L}}} encor plus simple, car les \edtext{forces seront}{\lemma{forces}\Bfootnote{\textit{(1)}\ des \textit{(2)}\ seront \textit{L}}} entre elles, comme les sommes des sinus droits et des sinus de complement, des angles. C'est \`{a} dire la force de la machine dans l'Estat \textit{E} sera \`{a} la force de la machine\protect\index{Sachverzeichnis}{machine} dans l'Estat \textit{P}, comme $EM + ML$ \`{a} $PQ + QL$.
\pend