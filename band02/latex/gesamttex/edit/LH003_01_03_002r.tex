[2~r\textsuperscript{o}]
reinligkeit, etc. der sprache eines menschen la{\ss}en sich consequentiae Medicae ziehen.
\pend%
\pstart%
Man mus sich gebrauchen aller bereits gefundener Experimentorum und observationum Medico-physicarum.
\pend%
\pstart%
Die mus man aus allen autoribus zusammen tragen und in eine ordnung bringen la{\ss}en cum gradibus verisimilitudinis. 
\pend%
\pstart%
 Alsdann mus man sie alle sobald muglich probiren la{\ss}en.
\pend%
\pstart%
Etliche kan man probiren wenn man will und dann mus es gleich geschehen.
\pend%
\pstart%
Etliche zum exempel remedia certorum morborum kan man nur probiren wenn die occasiones vorhanden. Und dahehr mus anstalt gemacht werden da{\ss} man allen orthen catalogum der patienten\protect\index{Sachverzeichnis}{Patient} des orths mit allen umbst\"{a}nden habe.
\pend%
\pstart%
Wenn man nun dabey hat directiones probandorum so kan man alsdenn proben thun. Doch da{\ss} solche ohngefehrlich seyen, es were denn der patient\protect\index{Sachverzeichnis}{Patient} damnatus.
\pend%
\pstart%
Man mus uberall die Leute zusammenfodern, und ihnen andeuten wer eine n\"{u}zliche cur\protect\index{Sachverzeichnis}{Kur} wi{\ss}e \"{u}ber lang oder kurz mit umbstanden zu erzehlen und glaubhafft zu machen solle Verehrungen haben. Der Medicus\protect\index{Sachverzeichnis}{medicus} des Amts so viel hubsche dinge zusammen bringen wird, soll auch Verehrungen haben.
\pend%
\pstart%
Ein ieder Medicus\protect\index{Sachverzeichnis}{medicus} und Chymicus\protect\index{Sachverzeichnis}{chymicus} soll ein stets wehrendes journal aller seiner laborum halten.
\pend%
\pstart%
Man mus f\"{u}r allen dingen der alten weiber und Marcktschreyer tradita circa simplicia zusammen bringen.
\pend%
\pstart%
Alle patienten\protect\index{Sachverzeichnis}{Patient} die in einem Hospital\protect\index{Sachverzeichnis}{Hospital} sterben, sollen anatomirt werden.
\pend%
\pstart%
Were guth da{\ss} die meisten Menschen anatomirt w\"{u}rden. 
\pend%
\newpage
\count\Bfootins=1500
\count\Cfootins=1500
\count\Afootins=1200
\pstart%
Alle anatomien\protect\index{Sachverzeichnis}{Anatomie} sollen modo diverso geschehen, wie M\textsuperscript{r} Stenonis\protect\index{Namensregister}{\textso{Stensen}, Niels 1638-1686} vorgeschrieben in
\edtext{\textit{Anatomia cerebri}.}{\lemma{\textit{Anatomia cerebri}}\Cfootnote{\cite{00100}\textsc{N. Stensen}, \textit{De cerebri anatome}, Leiden 1671, S. 51-58.}}
\pend%
\pstart%
Des Menschen den man anatomirt Historiam naturalem\protect\index{Sachverzeichnis}{historia naturalis} soll man soviel m\"{u}glich wi{\ss}en und denn alle seine humores\protect\index{Sachverzeichnis}{humor} etc. examiniren, succum pancreaticum\protect\index{Sachverzeichnis}{succus pancreaticus}, bilem\protect\index{Sachverzeichnis}{bilis}, etc. ob succus mehr acidus oder salsus sey. Was bilis\protect\index{Sachverzeichnis}{bilis} oder andere theile mit ligno nephritico\protect\index{Sachverzeichnis}{lignum nephriticum} etc. f\"{u}r farben geben.
\pend%
\pstart%
Man soll in der anatomi\protect\index{Sachverzeichnis}{Anatomie} alle minima auffzeichnen, alle ductus\protect\index{Sachverzeichnis}{ductus} und passagen\protect\index{Sachverzeichnis}{Passage} affusis coloratis $\langle$li$\rangle$quoribus probiren, allerhand ligaturas brauchen.%
% Hier folgt Bl. 2v.