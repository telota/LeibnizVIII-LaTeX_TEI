      
               
                \begin{ledgroupsized}[r]{120mm}
                \footnotesize 
                \pstart                
                \noindent\textbf{\"{U}berlieferung:}   
                \pend
                \end{ledgroupsized}
            
              
                            \begin{ledgroupsized}[r]{114mm}
                            \footnotesize 
                            \pstart \parindent -6mm
                           \makebox[6mm][l]{\textit{L}}Konzept:
LH XXXV 11, 13 Bl. 9. 1 Bl. 4\textsuperscript{o}. 1\,\nicefrac{1}{2} S.
\\Cc 2, Nr. 914 \pend
                            \end{ledgroupsized}
                %\normalsize
                \vspace*{5mm}
                \begin{ledgroup}
                \footnotesize 
                \pstart
            \noindent\footnotesize{\textbf{Datierungsgr\"{u}nde}: Die für den Gang von Uhren bereits in N.~85 angeführte Elastizität wird hier in ihrer Wirkungsweise systematisch dargelegt und begrifflich definiert, was sich als fortgesetzte und vertiefende Auseinandersetzung deuten ließe, die im Anschluss an N. 85 etwas später erfolgt sein könnte.}
                \pend
                \end{ledgroup}
            
                \vspace*{8mm}
                \pstart%
\normalsize
\noindent
[9~r\textsuperscript{o}]
\pend%
\pstart%
\centering%
De HOROLOGIO ABSOLUTO\\
sive De Motu aequabili pure mechanico\\
demonstratio Geometrica:
\pend%
            \count\Bfootins=1200 
\pstart 
\vspace*{0.5em} \noindent Axioma:
\edtext{si idem sit agentium\protect\index{Sachverzeichnis}{agens} patientiumque\protect\index{Sachverzeichnis}{patiens} status;
tempora\protect\index{Sachverzeichnis}{tempus} operationum eundem effectum\protect\index{Sachverzeichnis}{effectus} producentium, erunt aequalia.}%
{\lemma{si}\Bfootnote{%
\textit{(1)}\ eaedem sint vires, eademque impedimenta\protect\index{Sachverzeichnis}{impedimentum}
\textit{(2)}\ idem [...] status;
\textit{(a)}\ effectus aequali tempore sequetur 
\textit{(b)}\ tempora [...] aequalia. \textit{L}}}
\pend 
\count\Bfootins=800
\count\Cfootins=1200
\pstart  Corollarium est hoc, axiomatis generalissimi,
quod ejusdem positionis eaedem sunt consequentiae;
unde ejusdem causae, caeteris paribus iidem effectus:
et corpus
\edtext{idem situm eodem modo}{\lemma{idem [...] modo}\Bfootnote{\textit{erg. L}}}
ex eadem altitudine per idem medium eodem tempore sua gravitate\protect\index{Sachverzeichnis}{gravitas}
\edtext{libere}{\lemma{libere}\Bfootnote{\textit{erg. L}}}
descendet; et Elaterium\protect\index{Sachverzeichnis}{elaterium}
\edtext{idem eodem}{\lemma{idem}\Bfootnote{\textit{(1)}\ ad \textit{(2)}\ eodem \textit{L}}}
semper modo
\edtext{tensum, atque mox sibi permissum eodem tempore inde a momento recuperati sui juris ad eundem perveniet statum libertatis.}%
{\lemma{tensum,}\Bfootnote{%
\textit{(1)}\ eodem tempore se liberabit
\textit{(2)}\ atque
\textit{(a)}\ dimissum
\textit{(b)}\ mox libertat
\textit{(c)}\ inde
\textit{(d)}\ mox [...] tempore
\textit{(aa)}\ ad eundem perveniet statum libertatis.
\textit{(bb)}\ inde [...] libertatis. \textit{L}}}
\pend 
%\newpage
\pstart 
\textso{Elaterium }\protect\index{Sachverzeichnis}{elaterium}voco corpus quod figuram vi amissam vi cessante repetit;
quam\textso{ figuram }voco%
\edtext{\textso{ naturalem.}
\newline\indent%
\textso{Tensio }est actio causae figuram corporis Elaterio praediti naturalem mutantis.
\newline\indent%
\textso{Displosio }est actio elaterii figuram naturalem repetentis.
\newline\indent%
\textso{Liberatio }est ablatio impedimenti displosionis.
\newline\indent%
\textso{Detentacula }sunt machinamenta ad elateria detinenda atque detendenda, sive liberanda, apta.\textso{ Detentes }Gallis.
\newline\indent%
Si sint quotcunque Elateria, tensa \textit{A}. \textit{B}. \textit{C}. etc.
ita adhibitis\textso{ detentaculis }disposita,
ut primo \textit{A} liberato et ad certum displosionis statum perveniente, liberetur secundum \textit{B},
et secundo \textit{B} liberato et ad certum displosionis statum perveniente,}%
{\lemma{\textso{naturalem.}}\Bfootnote{%
\textit{(1)}\ Si sint %
\textit{(a)}\ aliquot elateria \textit{A}. \textit{B}. \textit{C}. etc. ita disposita, %
\textit{(aa)}\ ut ipse \textit{A} %
\textit{(bb)}\ atque tensa %
\textit{(b)}\ quotcunque elateria tensa %
\textit{(2)}\ \textso{Tensio} %
\textit{(a)}\ est motus partium corporis Elaterio praediti, quo per vim a figura sua demovetur %
\textit{(b)}\ est motus quo corpus elaterio praeditum mutat %
\textit{(c)}\ est actio [...] displosionis.\,
\textbar\ \textso{Detentacula} [...] Gallis. \textit{erg.} \textbar\
Si sint quotcunque Elateria,
\textbar\ tensa \textit{erg.} \textbar\
\textit{A}. \textit{B}. \textit{C}. etc. ita
\textbar\ adhibitis \textso{detentaculis} \textit{erg.} \textbar\
disposita, ut %
\textit{(aa)}\ sub finem displosionis primi \textit{A}, liberetur secundum \textit{B}, et sub finem displosionis secundi \textit{B} %
\textit{(bb)}\ primo \textit{A} [...] perveniente \textit{L}}}
liberetur tertium \textit{C}, etc.
\edtext{et ita pergatur}{\lemma{et ita pergatur}\Bfootnote{\textit{erg. L}}}
usque dum ultimum
\edtext{quoque faciat}{\lemma{ultimum}\Bfootnote{%
\textit{(1)}\ absolvat %
\textit{(2)}\ quoque faciat \textit{L}}}
displosionem
\edtext{suam; rursusque ultimo ad certum displosionis statum perveniente, liberetur denuo primum \textit{A}, tempus ab}%
{\lemma{suam;}\Bfootnote{%
\textit{(1)}\ tempus a %
\textit{(2)}\ eodemque momento quo ultimum absolvit displosionem suam, %
\textit{(3)}\ rursusque %
\textit{(a)}\ sub finem displosionis ultimi, liberetur iterum primum \textit{A},
(quod interea jam rursus tensum fuisse, et displosionem ultimi in eo statu expectare, supponitur.) %
\textit{(b)}\ ultimo [...] tempus ab \textit{L}}}
uno momento liberationis Elaterii\protect\index{Sachverzeichnis}{elaterium} \textit{A},
\edtext{ad momentum liberationis alterius}%
{\lemma{ad}\Bfootnote{%
\textit{(1)}\ aliud %
\textit{(2)}\ momentum liberationis alterius \textit{L}}} 
proximae sequentis, ejusdem Elaterii\protect\index{Sachverzeichnis}{elaterium} \textit{A},
vocabo\textso{ Periodum }\edtext{\textso{Elateriorum.}
% \newpage%
\newline%
\indent%
Si duae sint Periodi Elateriorum eademque sint Elateria ac detentacula eodem modo sita,
eademque cuilibet Elaterio, quae ante, vis, gradusque tensionis, erunt eae duae periodi inter se aequidiuturnae,}%
{\lemma{\textso{Elateriorum.}}\Bfootnote{%
\textit{(1)}\ Si quodlibet ex Elateriis %
\textit{(a)}\ eodem sem %
\textit{(b)}\ ad eundem semper \textso{tensionis statum} redactum intelligatur, itemque a praecedente ad eundem %
\textit{(aa)}\ tensionis %
\textit{(bb)}\ semper displosionis statum perveniente liberetur;
\textbar\ neque quicquam extrinsecus variationem inducere intelligatur \textit{erg.} \textbar\
Periodi Elateriorum erunt aequidiuturnae. Nam %
\textit{(aaa)}\ ad eu %
\textit{(bbb)}\ eadem sunt %
\textit{(ccc)}\ agentia patientiaque, elateria scilicet numero %
\textit{(aaaa)}\ situque eo %
\textit{(bbbb)}\ magnitudine materia, figura situ; denique ut verbo %
\textbar\ scholarum \textit{erg.}\ \textbar\
dicam, ipso individuo, eadem sunt. %
\textit{(aaaaa)}\ Vi %
\textit{(bbbbb)}\ Status quoque agentium semper idem,
quia eodem semper modo sunt tensa, ad idem scilicet ut ita dicam, punctum, ac tensionis gradum.
Patientium quoque eadem plane dispositio, quoniam ad eundem sem %
\textit{(2)}\ Si duae sint Periodi Elateriorum %
\textit{(a)}\ inter se %
\textit{(b)}\ simillimae, neque quicquam extrinsecus %
\textit{(c)}\ per omnia similes, neque quicquam extrinsecus variationem inducere ponatur, erunt tempora periodorum aequidiuturna %
\textit{(d)}\ eademque sint Elateria
\textbar\ ac detentacula \textit{erg.}\ \textbar\
eodem modo sita, %
\textit{(aa)}\ idemque cujuslibet Elaterii, qui ante, %
\textit{(bb)}\ eademque [...] aequidiuturnae, \textit{L}}}
modo nihil extrinsecus superveniens varietatem inducere ponatur.
\pend 
\count\Bfootins=1000
\count\Cfootins=1200
\pstart  
Hoc enim posito sequetur, duas illas periodos esse per omnia similes,
sive eundem utrobique agentium patientiumque statum esse,
nam per definitionem\textso{ periodi Elateriorum,\protect\index{Sachverzeichnis}{elaterium}}
nulla alia in ipsis agentia atque patientia sunt quam
\edtext{Elateria ac\textso{ detentacula.}}%
{\lemma{Elateria}\Bfootnote{%
\textit{(1)}\ et idem Elaterium quod praecedenti est agens, sequenti est patiens, idem autem qui ante est Elateriorum et instrumenta %
\textit{(2)}\ ac \textso{detentacula.} \textit{L}}}
\edtext{Porro detentacula eadem}%
{\lemma{Porro}\Bfootnote{%
\textit{(1)}\ in Elateriis nihil nisi numerus, et magnitudo, et ma %
\textit{(2)}\ Elateria et %
\textit{(3)}\ detentacula. \textit{(a)}\ et numero {(b)}\ eadem \textit{L}}}
ipsa sunt quae ante, eodemque modo sita:
neque alia in Detentaculis quae rigida intelliguntur,
\edtext{varietas potest concipi.}{\lemma{varietas}%
\Bfootnote{\textit{(1)}\ intelligi \textit{(2)}\ potest concipi. \textit{L}}} Elateria\protect\index{Sachverzeichnis}{elaterium} vero non tantum
[9~v\textsuperscript{o}]
eadem ipsa sunt, eodemque modo sita;
\edtext{sed et eandem restituendi sese vim, et flexionis sive tensionis gradum}%
{\lemma{sed et}\Bfootnote{%
\textit{(1)}\ eundem flexionis sive tensionis gradum \textit{(2)}\ eandem [...] gradum \textit{L}}}
habere intelliguntur: neque aliud quicquam in
\edtext{illis fingi potest per Elaterii definitionem.
Neque varietas ulla praeterea extrinsecus accedit;
ex hypothesi erunt ergo paria omnia,
idemque agentium patientiumque status,}%
{\lemma{illis}\Bfootnote{%
\textit{(1)}\ intelligi %
\textit{(2)}\ fingi potest %
\textit{(a)}\ . Cum ergo idem sit %
\textit{(b)}\ per Elaterii definitionem. %
\textit{(aa)}\ Eadem e %
\textit{(bb)}\ Idem ergo erit, qui ante agentium patientiumque status; neque ex hypothesi %
\textit{(cc)}\ Neque ex hy %
\textit{(dd)}\ Neque [...] accedit %
\textit{(aaa)}\ ; erunt ergo %
\textit{(bbb)}\ ; ex hypothesi [...] status, \textit{L}}}
ac proinde per\textso{ axioma }supradictum: tempora operationum eadem, tempora 
\edtext{autem operationum, a primo ad ultimum}%
{\lemma{autem operationum,}\Bfootnote{%
\textit{(1)}\ ab ultimo %
\textit{(2)}\ a primo ad ultimum \textit{L}}}
usque in primum denuo agens, periodi sunt, erunt ergo periodi
\edtext{Elateriorum aequidiuturnae.%
\newline%
\indent%
\textso{Schol. }satis ex his patet, nihil referre an Elateriorum diversorum vires sint aequales inter se;
neque enim ideo minus periodi omnium simul sumtorum erunt inter se aequales.
Si Elateria in rotundum}%
{\lemma{Elateriorum}\Bfootnote{%
\textit{(1)}\ eaedem %
\textit{(2)}\ aequidiuturnae. %
\textit{(a)}\ Si Elateria in rotundu %
\textit{(b)}\ Satis ex his patet nih %
\textit{(c)}\ \textso{Schol}. [...] Elateriorum %
\textit{(aa)}\ duorum %
\textit{(bb)}\ diversorum [...] Elateria \textbar\ ita \textit{gestr.}\ \textbar\ in rotundum \textit{L}}}
disposita sint ut ultimum ad primi viciniam redeat,
ac proinde nihil referat quod ultimum quod primum habeatur;
sitque vis quaedam separata, tantae celeritatis\protect\index{Sachverzeichnis}{celeritas},
ut ejus ope quodlibet Elaterium\protect\index{Sachverzeichnis}{elaterium}
\edtext{displosum}{\lemma{displosum}\Bfootnote{\textit{erg. L}}}
ad priorem tensionis statum redigi intelligatur,
\edtext{antequam absoluta periodo ordo displosionis ipsum denuo attingat;}%
{\lemma{antequam}\Bfootnote{%
\textit{(1)}\ periodus %
\textit{(2)}\ ad ipsum redeat %
\textit{(3)}\ absoluta [...] attingat; \textit{L}}}
periodi
\edtext{uniformes}{\lemma{uniformes}\Bfootnote{\textit{erg. L}}}
sine interruptione continuabuntur,
quamdiu ad omnia displosa re-tendenda sufficiet vis illa separata.
\pend 
\pstart  
Nam vi illa separata hoc efficitur, ut
\edtext{Elateria\protect\index{Sachverzeichnis}{elaterium} singula ad eum}%
{\lemma{Elateria}\Bfootnote{%
\textit{(1)}\ omnia %
\textit{(2)}\ singula %
\textit{(a)}\ in eundem %
\textit{(b)}\ ad eum \textit{L}}}
quem ante tensionis statum redigantur, antequam
\edtext{ea}{\lemma{ea}\Bfootnote{\textit{erg. L}}}
displosionis ordo rursus attingat;
itaque idem, qui ante redibit semper agentium patientiumque
\edtext{status.
Quae jam parata stabunt ad agendum patiendumve
cum volabit ordo itaque semper ex ordine agent,
semperque eodem modo agent.}%
{\lemma{status}\Bfootnote{%
\textit{(1)}\ , tunc scilicet cum agendi, ordo ea attigit %
\textbar\ jam parata s \textit{erg. u. gestr.} \textbar\ itaque semper %
\textit{(a)}\ agent, semperque eodem modo agent %
\textit{(b)}\ uniformes %
\textit{(2)}\ . Quae [...] agent. \textit{L}}}
Periodi
\edtext{ergo semper durabunt; et eodem modo durabunt,
et sese sine interruptione consequentur.}%
{\lemma{ergo}\Bfootnote{%
\textit{(1)}\ neque %
\textit{(2)}\ nunquam interrumpentur; %
\textit{(3)}\ sese sine ullo intervallo, %
\textit{(4)}\ sine interruptione, sine intervallo excipient, %
\textit{(5)}\ semper [...] consequentur. \textit{L}}}
\pend 
 \count\Bfootins=1500
 \count\Cfootins=1500

