\begin{ledgroupsized}[r]{120mm}%
\footnotesize%
\pstart%
\noindent\textbf{\"{U}berlieferung:}%
\pend%
\end{ledgroupsized}%
\begin{ledgroupsized}[r]{112mm}%
\footnotesize%
\pstart%
\parindent -8mm%
\makebox[8mm][l]{\textit{LiMs}}%
Notiz zu fremdhändigem Text: LH XLII 1 Bl. 21. 1 Zettel (16,5 x 13 cm).
Insgesamt 12~Z. auf Bl.~21~r\textsuperscript{o}.
Bl.~21~v\textsuperscript{o} leer.
Die meiste Fl\"{a}che durch die Schreiberin genutzt.
\newline%
Cc 2, Nr. 00%
\pend%
\end{ledgroupsized}%
%
% \normalsize%
% \vspace*{5mm}%
% \begin{ledgroup}%
% \footnotesize%
% \pstart%
% \noindent%
% \footnotesize{%
% \textbf{Datierungsgr\"{u}nde:}
% Der Kommentar dürfte zeitnah zu dem im Text erwähnten Datum der Schriftproben entstanden sein.}%
% \pend
% \end{ledgroup}
%
%
\vspace*{8mm}
\pstart%
\normalsize%
\noindent%
% [21~r\textsuperscript{o}]
[21~r\textsuperscript{o}] Haec a foemina brachiis carente, pedibus\protect\index{Sachverzeichnis}{pes} scripta
vidi Parisiis\protect\index{Ortsregister}{Paris} Martio 1674.
Dantisco\protect\index{Ortsregister}{Danzig} se oriundam dicebat.
Sinistri brachii\protect\index{Sachverzeichnis}{brachium} ne vestigium quidem[,] dextri initia quaedam atque rudimenta.
Vidi nentem, et chordas\protect\index{Sachverzeichnis}{chorda} pulsantem,
et sclopetum\protect\index{Sachverzeichnis}{sclopetum} portabile displodentem.%
\pend%
\vspace*{1.0em}%
\pstart%
\noindent%
[\textit{Nicht von Leibniz' Hand:}]%
\newline%
+
\newline%
Al mein Thunt zu jeder frist geschehe Jm namen Jesu Christ.%
\newline%
Dieu nous forme selon sa volonté.%
\newline%
Dio vi dia la bona sera a Tuti lor e altri signori.%
\pend%
%%%%  PR: Hier endet das Stück.