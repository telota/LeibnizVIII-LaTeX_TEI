\begin{ledgroupsized}[r]{120mm}%
\footnotesize%
\pstart%           
\noindent\textbf{\"{U}berlieferung:}%
\pend%
\end{ledgroupsized}%
\begin{ledgroupsized}[r]{114mm}%
\footnotesize%
\pstart%
\parindent -6mm%
\makebox[6mm][l]{\textit{L}}%
Aufzeichnung:
LH III 1, 3 Bl. 9.
1 Bl. 2\textsuperscript{o}, oben und unten beschnitten (21 x 19 cm).
2~S. Textfolge: Bl. 9~v\textsuperscript{o}, 9~r\textsuperscript{o}.
Ein Wasserzeichen.%
\newline%
KK1, Nr. 976%
\pend%
\end{ledgroupsized}%
%
\begin{ledgroupsized}[r]{114mm}%
\footnotesize%
\pstart%
\parindent -6mm%
\makebox[6mm][l]{\textit{E}}%
\cite{01131}\textsc{G.W. Leibniz}, \glqq Directiones ad rem medicam pertinentes\grqq, hrsg. von \textsc{F. Hartmann} und \textsc{M. Kr\"{u}ger}, \textit{Studia Leibnitiana} VIII, 1 (1976), S. 40-68: S. 66-68.
\pend%
\end{ledgroupsized}%
%
%
\vspace*{5mm}%
\begin{ledgroup}%
\footnotesize%
\pstart%
\noindent%
\footnotesize{%
\textbf{Datierungsgr\"{u}nde:}
Das gleiche Wasserzeichen wie im Textträger des vorliegenden Stücks ist in sämtlichen Bogen vorhanden,
die N.~70
 % = LH 03,01,03_001-008 = Directiones ad rem medicam pertinentes
überliefern. Die thematische Verwandtschaft beider Stücke legt ebenfalls nahe,
die Datierung von N.~70
 % = LH 03,01,03_001-008 = Directiones ad rem medicam pertinentes
auch für N.~69
 % = vorliegendes Stück = LH 03,01,03_009
zu übernehmen.
Dass \glqq medizinische Richtlinien\grqq~(\textit{directiones medicae}) im vorliegenden Stück
als Desiderat dargestellt werden (siehe unten, S.~\refpassage{003,01,03_009_z1}{003,01,03_009_z2}),
könnte man auch als Hinweis darauf deuten, dass N.~69
 % = vorliegendes Stück = LH 03,01,03_009
im Vorfeld von N.~70
% = LH 03,01,03_001-008 = Directiones ad rem medicam pertinentes
verfasst wurde.%
}%
\pend%
\end{ledgroup}%
%
%
\vspace*{8mm}%
\count\Bfootins=1200
\count\Cfootins=1200
\count\Afootins=1200
\pstart%
\normalsize%
\noindent%
[9~v\textsuperscript{o}]
Mirifice mihi placet Henrici\protect\index{Namensregister}{\textso{Stubbe}, Henry 1632-1676}
\edtext{Stubbii institutum.}{\lemma{Stubbii institutum}\Cfootnote{\cite{00105}\textsc{H. Stubbe}, \textit{The plus ultra reduced to a non plus}, London 1670.}}
%
Est velut
\edtext{Billichius\protect\index{Namensregister}{\textso{Billich}, Anton G\"{u}nther 1598-1640}%
}{\lemma{Billichius}\Cfootnote{Vermutlich \cite{00011}\textsc{A.G. Billich}, \textit{Thessalus in chymicis redivivus}, Frankfurt a.M. 1643.}}
%
quidam, et
\edtext{Claramontius\protect\index{Namensregister}{\textso{Chiaramonti}, Scipione 1565-1652}%
}{\lemma{Claramontius}\Cfootnote{Etwa
\cite{00026}\textsc{S. Chiaramonti}, \textit{In Aristotelem De iride}, Venedig 1668;
\cite{01146}\textsc{Ders.}, \textit{Philosophia naturalis}, Venedig 1652;
\cite{01147}\textsc{Ders.}, \textit{De atra bile quoad mores attinet}, Paris 1641.}}
%
et \edtext{Lindanus\protect\index{Namensregister}{\textso{Linden}, Jan Antonides van der 1609-1664}%
}{\lemma{Lindanus}\Cfootnote{J.A. van der Linden hat zwei für die humoralpathologische Medizin wichtige Schriften herausgegeben:
\cite{00060}\textsc{Hippocrates}, \textit{Opera Omnia}, Leiden 1660,
und \cite{00022}\textsc{Celsus}, \textit{De medicina}, Leiden 1657.}}
et \edtext{Conringius\protect\index{Namensregister}{\textso{Conring}, Hermann 1606-1681}%
}{\lemma{Conringius}\Cfootnote{\cite{00029}\textsc{H. Conring}, \textit{De hermetica}, Helmstedt 1669.}}
%
qui nostri temporis jactatores ad veteres revocant.
Si contumeliosa demas caetera praeclara sunt.
Ego ita \edtext{sentio: inquirendum}{\lemma{sentio:}\Bfootnote{\textit{(1)}\ quaerendum \textit{(2)}\ inquirendum \textit{ L}}}
esse, quasi nihil ante nos inventum esset; concludendum,
\edtext{definiendum, quasi omnia veteribus}{\lemma{definiendum,}\Bfootnote{\textit{(1)}\ jactandum \textit{(2)}\ quasi omnia veteribus \textit{L}}}
constitissent, quamdiu non effecimus, ut usum ipso medico fructum inventorum ostendimus.
Sed hoc non fiet etsi mille inventis novis productis, antequam Respublica manum admoliatur,
et ordinem quendam methodum, applicandi rationem et communitionem experimentorum constituat.
Quid prodest detegi aliquid quod vix ad millesimum quemque pervenit et forte ante fructum rursus obliteratur.
Quam multa praeclara tum in libris ante nos, tum in memoria schedisque medicorum nostri temporis, imo et vulgi sermonibus sunt, quae si collecta in unum, digesta in ordinem, et vera ratiocinandi atque inducendi arte adhibita essent in
\edtext{usum essemus}{\lemma{usum}\Bfootnote{\textit{(1)}\ possemus de \textit{(2)}\ essemus \textit{L}}}
paulo propius perfectioni medicinae, certae, vitam multorum proferremus.
Neque enim dubitandum est multos homines facile negotio servari posse quales ego omnes esse arbitror, qui febri, qui peste, qui hydrope, qui calculo, qui phthisi, qui vitio aliquo humorum moriuntur.
Nam in quibus ruptum est aliquid, aut quibus viscera vivendi tractu velut detrita
\edtext{sunt, eorum ratio}{\lemma{sunt,}\Bfootnote{\textit{(1)}\ in quibus \textit{(2)}\ eorum ratio \textit{L}}}
alia est.
Denique putem brevi temporis spatio per medicinam effici posse, ut homines fere non nisi morte naturali, aut saltem infortuniis inevitabilibus (qualia sunt vulnera letalia, lapsus, rupturae viscerum aliaque id genus) moriantur.
Quod si fiat, non ideo peius res humanae habebunt dum multae sint incultae terrae, habitabiles tamen, ut non debeamus de hominum multitudine conqueri.
Quid prohibet aliquando hominibus omnibus a juventute certas vivendi regulas praescriptas esse, quas profecto tam religiose servarent,
%
[9~r\textsuperscript{o}]
%
quam cibi tempore, et preces matutinas atque ante coenam.
Possunt accedere Ecclesiae praecepta, possunt infantes a parentibus institui certa proscripta forma.
Possunt ipsae scholae publicae emendari, ut parentes distracti ea cura liberentur.
Ego caetera omnia nullius pretii habeo, si comparentur medicinae tum corporum tum animorum, id est curae sanitatis et justitiae seu pietatis.
Caetera inventa mechanica quibus astronomia, geogra$\phi$ia, res nautica, statica, Belopoeetica, agricultura, metallica, botanica excoluntur,
quatenus his non inserviunt parvi facio.
Moralia et medicinam haec sunt quae unice aestimari debent.
Quare Microscopia longe magis quam Telescopia aestimo,
et si quis morbi cujuscunque certam exploratamque curationem invenerit, eum ego majoris faciendum arbitror, quam si quadraturam circuli invenisset.
Alia res est motus perennis, nam qui hunc invenit is quantum ad usum mechanicum totidem nova flumina novosque homines vel animalia saltem produxisse in effectu dicendus est, quia ita parci laboribus, et homines alio vertere curas possunt.
Nauticae finis \edtext{verus}{\lemma{verus}\Bfootnote{\textit{erg. L}}}
est mea sententia detegere et colere novas terras, et earum homines ad cultum vitae veramque sapientiam traducere.
Finis Astronomiae mea sententia est ut caeterarum curiosarum scientiarum omnium admiratio harmoniae rerum, seu DEI.
Sed Astronomiae finis peculiaris est, investigare, an non possint aliquae rationes inveniri circa originemque finem connexionemque mundi.
Artis militaris finis est tum defendere sese, tum vero posse cogere populos barbaros ad leges meliores, ubi omnis crudelitas abesse debet.
Caeterae artes diriguntur ad voluptatem et commoda vitae.
\pend%
\pstart%
Admonendi omnes medici ut quisque suam sententiam dicat de modo perficiendae medicinae, inprimis universitates et collegia, tum qui volent, particulares:
inserant exemplis observata sua; quanto magis illustrabunt praemia constituentur meliora afferentibus.
\edlabel{003,01,03_009_z1}%
Logica quasi Medica scribenda, ut habeamus Logicas juridicas
\edtext{interrogatoria ut}{\lemma{interrogatoria}\Bfootnote{\textit{(1)}\ seu dire \textit{(2)}\ ut \textit{ L}}}
juridica, directiones medicas.%
\edlabel{003,01,03_009_z2}
Adhibendi libri nonnulli qui de talibus jam scripsere,
ut Claudinus\protect\index{Namensregister}{\textso{Claudini}, Giulio Cesare ca. 1550-1618} \textit{de ingressu ad}
\edtext{\textit{infirmos}}{\lemma{\textit{infirmos}}\Cfootnote{\cite{00027}\textsc{G.C. Claudini}, \textit{De ingressu ad infirmos}, Bologna 1612.}}.
%
Colligendum ex omnibus medicis, quicquid huc pertinet, utilissimus ad eam rem
\edtext{Sachsius\protect\index{Namensregister}{\textso{Sachs von L\"{o}wenheim}, Philipp Jacob 1627-1672}%
}{\lemma{Sachsius}\Cfootnote{P.J. \mbox{Sachs} von L\"{o}wenheim gr\"{u}ndete als Stadtarzt in Breslau 1670 die \cite{00001}\textit{Miscellanea curiosa medico-physica Academiae naturae curiosorum sive Ephemeridum medico-physicarum Germanicarum curiosarum.}%
% Ebenfalls erw\"{a}hnt in Brief vom 1. Juni 1672 (I,1 N. 182).
}}
et similes. Stenonis\protect\index{Namensregister}{\textso{Stensen}, Niels 1638-1686} de modo perficiendae
\edtext{Anatomiae.}{\lemma{Anatomiae}\Cfootnote{\cite{00101}\textsc{N. Stensen}, \textit{Observationes anatomicae}, Leiden 1662.}}
%
Item petendum ab omnibus ut scribant de diaeta et morborum praecautionibus non tantum tractatibus,
sed proponant sibi distinguantque summas hominum varietates quae sint v.g. 10.
et cuilibet praescribant breve consilium quomodo optime vivere possit. NB.
\pend%
\count\Bfootins=1500
\count\Cfootins=1500
\count\Afootins=1500
%
% Hier endet das Stück.