\begin{ledgroupsized}[r]{120mm}%
\footnotesize%
\pstart%
\noindent\textbf{\"{U}berlieferung:}%
\pend%
\end{ledgroupsized}%
\begin{ledgroupsized}[r]{114mm}%
\footnotesize%
\pstart%
\parindent -6mm%
\makebox[6mm][l]{\textit{L}}%
Konzept:
LH XXXVI Bl. 130-131.
2 Bl. 2\textsuperscript{o}, jeweils um etwa eine Spalte beschnitten.
2 S. auf Bl.~130~r\textsuperscript{o} und 131~r\textsuperscript{o}.
Bl.~130~v\textsuperscript{o} und 131~v\textsuperscript{o} leer.
Auf jedem Blatt ein Wasserzeichen.
Der Text wird editorisch in zwei Teile untergliedert.%
\newline%
Cc 2, Nr. 508 (irrt\"{u}mlich als LH XXXV 15, 6 Bl. 130-131 bezeichnet)
\pend%
\end{ledgroupsized}%
%
% \normalsize
\vspace*{5mm}%
\begin{ledgroup}%
\footnotesize%
\pstart%
\noindent%
\footnotesize{%
\textbf{Datierungsgr\"{u}nde:}
Die Wasserzeichen in den Textträgern des vorliegenden Stücks sind für den Zeit\-raum von Anfang 1674 bis zum Anfang
1675 belegt.%
}%
\pend%
\end{ledgroup}%
%
%
\vspace*{8mm}%
\pstart
\noindent
[130~r\textsuperscript{o}]
\pend
\pstart%
\normalsize%
\noindent%
\centering%
\lbrack\textit{Teil 1}\rbrack%
\pend%
\vspace*{0.5em}% PR: Rein provisorisch !!!
\count\Bfootins=1200
\count\Cfootins=1200
\count\Afootins=1200
\pstart%
\noindent%
\centering
% [130~r\textsuperscript{o}]
Onomasticon vocum militarium ad hodiernam consuetudinem,\\
\edtext{ex Hermanno Hugone,\protect\index{Namensregister}{\textso{Hugo}, Hermann 1588-1639}
Grotio,\protect\index{Namensregister}{\textso{Grotius}, Hugo 1583-1654}
Heinsio,\protect\index{Namensregister}{\textso{Heinsius}, Daniel 1580-1655}
Strada.\protect\index{Namensregister}{\textso{Strada}, Famiano 1572-1649}%
}{\lemma{ex [...] Strada}\Cfootnote{Mögliche Quellen sind
\cite{01181}\textsc{H. Hugo}, \textit{Obsidio Bredana armis Philippi IV, auspiciis Isabellae, ductu Ambrosii Spinolae perfecta}, Antwerpen 1626;
\cite{01182}\textsc{Ders.}, \textit{De militia equestri antiqua et nova}, Antwerpen 1630;
\cite{01183}\textsc{H. Grotius}, \textit{Grollae obsidio cum annexis anni 1627}, Amsterdam 1629;
\cite{01184}\textsc{Ders.}, \textit{Annales et historiae de rebus Belgicis}, Amsterdam 1657;
\cite{01185}\textsc{D. Heinsius}, \textit{Rerum ad Sylvam-Ducis atque alibi in Belgio aut a Belgis anno 1629 gestarum historia}, Leiden 1631;
\cite{01186}\textsc{F. Strada}, \textit{De bello Belgico decades duae}, 2 Teile, Rom 1632-1647.%
}}\\
\edtext{Adde Boxhornii\protect\index{Namensregister}{\textso{Boxhorn} (Boxhornius), Marcus Zuerius van, 1612(?)-1653}
\edtext{\textit{Bredanam obsidionem}}{\lemma{\textit{obsidionem}}\Cfootnote{\cite{01151}\textsc{M. Boxhorn}, \textit{Historia obsidionis Bredanae et rerum anno 1637 gestarum}, Leiden 1640.}}%
}{\lemma{Adde [...] \textit{obsidionem}}\Bfootnote{\textit{erg. L}}}%
\pend
\vspace*{0.5em}%
\pstart%
\textso{Approches}: \textso{accessus}, propinquationes, aperire solum, ut tecto ad latera itinere
\edtext{et per artem}{\lemma{et per artem}\Bfootnote{\textit{erg. L}}}
situato, quo magis oppidanorum tela per obliquum vitarentur, tutius arreperetur oppido.
\pend%
\pstart%
\textso{Batteries}. aggeres, suggestus tormentarius, suggesta tormentaria, sedes tormentorum.
\pend%
\pstart%
\textso{Bastion}. propugnaculum.
\pend%
\pstart%
\textso{Flanc du bastion}. latus propugnaculi.
\pend%
\pstart%
\textso{Baracche}. castrensium tuguria.
\pend%
\pstart%
\textso{Breche}. ruina.
\pend%
\pstart%
\textso{Bolwerck}. munimenta.
\pend%
\pstart%
 Binte van de Gallerie, colomnae vineae.
\pend%
\pstart%
\textso{Cavallieri}. \textso{Catten}. colles, e quibus longius machinae suas pilas permittunt, tumulus moenibus impositus.
\pend%
\pstart%
\textso{Canoni da batteria}, tormenta obsidionalia, oppon. \textso{campestria}. Veld\-st\"{u}cken.
\pend%
\pstart%
\textso{Casematte}. caecae cryptae. imae cryptae ad latera propugnaculorum.
\pend%
\pstart%
\textso{Contramina}. contrarius cuniculus, transversus meatus.
\pend%
%\newpage% PR: Rein provisorisch !!!
\pstart%
\textso{Contrascarpa}. fossae pars moenibus adversa: Lorica quae fossarum ripam ex\-timam circumvenit. Via cooperta. Lorica viae coopertae.
\pend%
\pstart%
\textso{Corps de Garde}. statio stationarii.
\pend%
\pstart%
\textso{Cortina}. interjecti muri lorica.
\pend%
\pstart%
\textso{Schanzcorben}. corbes loricales. Corbium seu cistarum objectus. Loricae vimineae.
\pend%
\pstart%
\textso{Circonvallation}. agger ambitus.
\pend%
\pstart%
\textso{Dam}. obex. repagula et objices versatiles.
\pend%
\pstart%
\textso{Demilune}. demidiata Luna. Lunata species. semilunare munimentum.
\pend%
\pstart%
\textso{Dyck}. agger.
\pend%
\pstart%
\textso{Faussebraye}. succinctum valli. lorica succinctus. Lorica horizontalis.
\pend%
\pstart%
\textso{Forts}. castella.
\pend%
\pstart%
\textso{Gallerie}. vinea.
\pend%
\pstart%
\textso{Grenades}. Mala punica. Punica mala militibus appellantur, eo quod simul cecidere rupti in multos quasi acinos sparguntur, facili ut quidque attigerint incendio.
\pend%
\pstart%
\textso{Hornwerck}. opera cornuta. praeducta aggeri et alia monumenta, partim cunei, partim jugi jacentis in modum, quaedam et fronte in forficem recedente, quorum multus nunc usus: Cornuta vocant. Forficato opere muniri.
\pend%
\pstart%
 \textso{Ligne}. linea.
\pend%
\pstart%
\textso{Ligne de communion}. Linea communis, vulgo communionis.
\pend%
\pstart%
\textso{Mortier}. Mortariolum.
\pend%
\newpage
\pstart%
\textso{Palissaden}. sudes praepilatae. densa seps sudium duabus pinnis superne instar ericii armatarum; ne quis varicando transiret.
\pend%
\pstart%
\textso{Parapet}. lorica. Lorica post circitorum viam (vielleicht circitorum via scilicet zwinger).%
\edtext{}{\lemma{}\Afootnote{\textit{Am Rand:}
Imo circitorum via est le chemin des rondes.}}
\pend%
\pstart%
\textso{Petard}. Pyloclastrum.
\pend%
\pstart%
\textso{Pialtaforma}. planiforme propugnaculum.
\pend%
\pstart%
\textso{Ravelins}. moles portae praestructa. propugnaculum. moles. portae munimentum. jaculum.%
\edtext{}{\lemma{}\Afootnote{\textit{Über} jaculum: \Denarius}}
\pend%
%\newpage% PR: Rein proviosorisch !!!
\pstart%
\textso{Redoutes}. turres. Reductus. Turribus humilioribus terrae aggestu substructis.
\edtext{Receptacula.
\newline%
\hspace*{7,5mm}%
\textso{Rempart}.}{\lemma{}\Bfootnote{Receptacula.\ \textbar\ \textso{Retrenchement}. agger continuus. castrorum aggeres. \textit{gestr.}\ \textbar\ \textso{Rempart}. \textit{L}}}
vallum.
\pend%
\pstart%
\textso{Retranchement}. agger continuus. castrorum aggeres.
\pend%
\pstart%
\textso{Retirate}. interni receptus. perfugia secundaria.
\pend%
\pstart%
\textso{Sluys}. \textso{Ecluse} Catarracta.
\pend%
\pstart%
\textso{Trench\'{e}e}. agger continuus; cum de obsidentibus dicitur. de obsessis, recessus. seps castrorum. agger castrensis. continuatus.
\pend%
\pstart%
\textso{Tenailles}. forcipes.
\pend%
\pstart%
\textso{Une Traverse}. lorica transversa. Transversa sepimenta.
\pend%
\pstart%
\textso{Venute}. itinerum aditus.%
% \pend%
% \pstart%
%%  PR: Hier folgt Bl. 131r