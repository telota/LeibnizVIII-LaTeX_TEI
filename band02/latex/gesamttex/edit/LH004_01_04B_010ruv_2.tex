\begin{ledgroupsized}[r]{120mm}
\footnotesize 
\pstart                
\noindent\textbf{\"{U}berlieferung:}   
\pend
\end{ledgroupsized}
\begin{ledgroupsized}[r]{114mm}
\footnotesize 
\pstart \parindent -6mm
\makebox[6mm][l]{\textit{L}}%
Auszüge mit Bemerkungen aus einem verschollenen Manuskript von R. Descartes, nach \cite{01060}A. \textsc{Kircher}, \title{Magnes}, Rom 1641:
LH IV 1, 4b Bl. 9-10. 1 Bog. 2\textsuperscript{o}.
Insgesamt 17 Z. am unteren Ende von Bl. 10~r\textsuperscript{o} und von Bl. 10~v\textsuperscript{o}. Der übrige Text auf Bl. 9-10 gehört zu N. 58.% = LH004,1,4B Bl. 3r-14v = Anatomica quaedam ex manuscripto Cartesii
\newline%
Cc 2, Nr. 1322 D (tlw.)
\pend
\end{ledgroupsized}
%
\begin{ledgroupsized}[r]{114mm}
\footnotesize 
\pstart \parindent -6mm
\makebox[6mm][l]{\textit{E}}%
\cite{00120}R. \textsc{Descartes}, \textit{{\OE}uvres}, hrsg. von \textsc{C. Adam} und \textsc{P. Tannery}, Bd. XI, Paris 1909, S.~635-639.
\pend
\end{ledgroupsized}
%
%\normalsize
\vspace*{5mm}%
\begin{ledgroup}
\footnotesize%
\pstart%
\noindent%
\footnotesize{\textbf{Datierungsgr\"{u}nde:}
Das vorliegende Stück befindet sich auf einem Textträger von N. 58 
% = LH004,1,4B Bl. 3r-14v = Anatomica quaedam ex manuscripto Cartesii
und dürfte daher zeitnah entstanden sein.
Die Datierung von N. 58 % = LH004,1,4B Bl. 3r-14v = Anatomica quaedam ex manuscripto Cartesii
-- Februar bis September 1676 -- wird demgemäß auch für das vorliegende Stück übernommen.}
\pend
\end{ledgroup}
%
\count\Afootins=1500
\count\Bfootins=1500
\count\Cfootins=1300
\vspace*{8mm}
\pstart%
\normalsize%
\noindent%
% [10~r\textsuperscript{o}]
[10~r\textsuperscript{o}] Excerpta ex P. Kircheri\protect\index{Namensregister}{\textso{Kircher}, Athanasius 1602-1680} \textit{De Magnete}\cite{01060} ut quod \edtext{is}{\lemma{is}\Bfootnote{\textit{erg. L}}}\protect\index{Sachverzeichnis}{magnes} ait pag. 7. cristallum\protect\index{Sachverzeichnis}{cristallus} combustum tantum ponderis\protect\index{Sachverzeichnis}{magnes} cinerum\protect\index{Sachverzeichnis}{cinis} dare, quantum erat prius\edtext{}{\lemma{prius}\Cfootnote{\textsc{A. Kircher}, \textit{Magnes},\cite{01060} Rom 1641, S. 7.}}.
Pag. 14. Quaenam chalybem\protect\index{Sachverzeichnis}{chalybs} \edtext{durent.}{\lemma{durent}\Cfootnote{\cite{01060}a.a.O., S. 14.}}
De venis terrae. \edtext{Pag. 45.50}{\lemma{Pag. 45.50}\Cfootnote{\cite{01060}a.a.O., S. 45-50 u.a.}} Quod polus borealis hic plus ferri\protect\index{Sachverzeichnis}{ferrum} trahat, quia juvatur a terra, alio \edtext{magnete.}{\lemma{magnete}\Cfootnote{\cite{01060}a.a.O., S. 158.}}
Vitrarii\protect\index{Sachverzeichnis}{vitrarius} liquorem vitri\protect\index{Sachverzeichnis}{vitrum} a terrestreitate purgant injecto magnete, qui eam attrahit et post cum ea igne \edtext{absumitur;}{\lemma{absumitur}\Cfootnote{\cite{01060}a.a.O., S. 113, bzw. \cite{01114}\textsc{A. Kircher}, \textit{Magnes}, K\"{o}ln 1643, S. 110.}} ferrum vel magnes debilior a potentiore ferrum subducit.
Cujus rationem male reddit P. Kircher; ea autem est, quod \edtext{etc.}{\lemma{etc.}\Cfootnote{\cite{01060}\textsc{A. Kircher}, \textit{Magnes}, Rom 1641, S. 150.}}
(+ plura non ascripta +)
Magnes cujus anguli detrahuntur si \edtext{detrahantur vis \edtext{augetur.}{\lemma{augetur}\Cfootnote{\cite{01060}a.a.O., S. 136.}}}{\lemma{detrahantur}\Bfootnote{\textit{(1)}\ anguli vis\ \textit{(2)}\ a vi movetur\ \textit{(3)}\ a\ \textit{(4)}\  vis augetur \textit{L}}}
\pend
\pstart
Ferrum candens attrahitur a \edtext{magnete.}{\lemma{magnete}\Cfootnote{\cite{01060}a.a.O., S. 157.}}
177 Magnes ingens vix trahens aciculam\protect\index{Sachverzeichnis}{acicula} sibi conjunctam movet versoriam\protect\index{Sachverzeichnis}{versoria} \edtext{ad}{\lemma{ad}\Bfootnote{\textit{(1)}\ unam partem \textit{(2)}\ [unum] pedem \textit{L}}} \edtext{[unum]}{\lemma{}\Bfootnote{unam\ \textit{L \"{a}ndert Hrsg.}}} pedem, quod minores longe fortiores non \edtext{faciunt.}{\lemma{faciunt}\Cfootnote{\cite{01060}a.a.O., S. 177.}}
\pend
\pstart
617 Modum excitandi venti describit lapsu\protect\index{Sachverzeichnis}{lapsus} aquae\protect\index{Sachverzeichnis}{aqua} per longum canalem\protect\index{Sachverzeichnis}{canalis} supra latiorem quam infra, in aliquod vas clausum, in quo ait aerem\protect\index{Sachverzeichnis}{aer} ipso lapsu sic agitari et reproduci aqua scilicet per foramen in inferiore vasis parte, elabente aere vero ex vaporibus aquae generato et flante per foramen in parte vasis\protect\index{Sachverzeichnis}{vas} superiore, ut viderit malleatores\protect\index{Sachverzeichnis}{malleator} ferrum\protect\index{Sachverzeichnis}{ferrum} in virgas\protect\index{Sachverzeichnis}{virga} ducentes ad ignem\protect\index{Sachverzeichnis}{ignis} continuo sufflandum ea machina\protect\index{Sachverzeichnis}{machina} uti. Hinc rationem reddit, cur ventus ex quibusdam cavernis\protect\index{Sachverzeichnis}{caverna} perpetuo exeat et \edtext{recte.}{\lemma{recte}\Cfootnote{\cite{01060}a.a.O., S. 615-618.}}
[10~v\textsuperscript{o}]
\pend
\pstart%
Electrica frictu\protect\index{Sachverzeichnis}{frictus} calefacta trahunt, igni admota non trahunt, quippe ut trahant debet aliquid egredi
%\edtext{}{\lemma{}\Bfootnote{egredi  \textbar\ egredi \textit{ gestr.}\ \textbar\ quod \textit{ L}}}
quod \edtext{redeat.}{\lemma{redeat}\Cfootnote{\cite{01060}a.a.O., S. 618f.}}
Ait in magno coenaculo\protect\index{Sachverzeichnis}{coenaculum} rotundo et alibi se observasse voces ab una parte ad aliam transferri etiam musica\protect\index{Sachverzeichnis}{musica} obstrepente ita ut quod ex una parte summisse dicitur aure\protect\index{Sachverzeichnis}{auris} apposita muro diametraliter opposito possit audiri, non autem in aliis locis. Cujus rationem ait, quod aer\protect\index{Sachverzeichnis}{aer} utrinque motus in semicirculo ibi \edtext{concurrat.}{\lemma{concurrat}\Cfootnote{\cite{01060}a.a.O., S. 863.}}
Recte.
\pend
\count\Afootins=1500
\count\Bfootins=1500
\count\Cfootins=1500