[4~r\textsuperscript{o}]
\pend%
%\newpage
\pstart%
\edtext{Man mus}{\lemma{Man}\Bfootnote{\textit{(1)} mu{\ss}en \textit{(2)} mus \textit{L}}}
general visitationes rei Apothecariae\protect\index{Sachverzeichnis}{res apothecaria} anstellen,
und dabey in acht nehmen sowohl
was Bartholinus\protect\index{Namensregister}{\textso{Bartholin}, Thomas 1616-1680}
\edtext{gegen die apothecer}{\lemma{gegen die apothecer}\Bfootnote{\textit{erg. L}}}
%
\edtext{edirt,}{\lemma{edirt}\Cfootnote{Thomas Bartholin hatte wenige Jahre zuvor eine von Sebastien Colin 1557 unter dem Pseudonym \glqq Lisset Benancio\grqq~ver\"{o}ffentlichte Streitschrift gegen die Apotheker 
ins Lateinische übersetzt und herausgegeben; siehe \cite{00010}\textsc{L. Benancio}, \textit{Declaratio fraudum}, Frankfurt 1667.}}
%
als neulich bey den
\edtext{Englischen disputen}{\lemma{Englischen disputen}\Cfootnote{Zu dem Streit zwischen englischen \"{A}rzten und Apothekern, auf den Leibniz hier anspielt, siehe \cite{00079}\textsc{C. Merret}, \textit{Short view of the frauds}, London 1669; \cite{00103}\textsc{H. Stubbe} [?], \textit{Lex talonis}, London 1670; \cite{00078}\textsc{C. Merret}, \textit{A short reply}, London 1670; \cite{00104}\textsc{H. Stubbe}, \textit{Medice cura teipsum!}, London 1671.}}
%
zwischen apothekern\protect\index{Sachverzeichnis}{Apotheker} und Medicis\protect\index{Sachverzeichnis}{medicus} vorkommen.%
\pend%
\pstart%
Man m\"{u}{\ss}te exacte
\edtext{observiren die zeiten}{\lemma{observiren die}\Bfootnote{\textit{(1)} tempora \textit{(2)} zeiten \textit{L}}}
\edtext{das der}{\lemma{das}\Bfootnote{\textit{(1)} die \textit{(2)} der \textit{L}}}
tranck einen urin\protect\index{Sachverzeichnis}{Urin} und die Speise ein excrement\protect\index{Sachverzeichnis}{Exkrement} giebt, welches denn bey einem menschen ehe geschehen wird als beym andern.
\pend%
\pstart%
Man m\"{u}{\ss}te auch achthaben wieviel den stimulis naturalibus und indicationibus zu trauen als wenn die natur einen vomitum per conatum curtum, eine venae sectionem\protect\index{Sachverzeichnis}{sectio venae} per sanguinis emissionem\protect\index{Sachverzeichnis}{emissio sanguinis} etc. indicirt. Item umb wieviel dem naturlichen appetit\protect\index{Sachverzeichnis}{Appetit} zu e{\ss}en dieses oder jenes, schlaffen etc. zu folgen oder nicht zu folgen.
\pend%
\pstart%
Und weil bekand so ziemlich eine symmetria partium in corpore humano befunden, solche aber bey keinen
\edtext{Menschen in}{\lemma{Menschen}\Bfootnote{\textit{(1)} uber \textit{(2)} in \textit{L}}}
allen just seyn wird, so muste man solche evagationes\protect\index{Sachverzeichnis}{evagatio} annotiren, und versuchen ob daraus etwas de constitutione corporis\protect\index{Sachverzeichnis}{constitutio corporis} zu schlie{\ss}en.
\pend%
\pstart%
Und wenn nach Herren Wrenni\protect\index{Namensregister}{\textso{Wren}, Christopher 1632-1723}, Hook\protect\index{Namensregister}{\textso{Hooke}, Robert 1635-1703} und anderer Gedancken eine Historia temporum\protect\index{Sachverzeichnis}{historia temporum} formirt, oder wie ich offt gedacht, Calender\protect\index{Sachverzeichnis}{Kalender} von vergangenen jahren gemacht wurden, so m\"{u}ste man minutim einen ieden annotiren la{\ss}en was
\edtext{er fur}{\lemma{er}\Bfootnote{\textit{(1)} bey \textit{(2)} fur \textit{L}}}
\edtext{veranderung dabey}{\lemma{veranderung}\Bfootnote{\textit{(1)} damit \textit{(2)} dabey \textit{L}}}
an sich empfunden.
Und sonderlich kondten hier die beste annotationes machen die jenigen so immer einerley art zu leben brauchen, als bauern ordens personen.
\pend%
\pstart%
Man mus
\edtext{versuchen was}{\lemma{versuchen}\Bfootnote{\textit{(1)} umb \textit{(2)} was \textit{L}}}
es thate wenn ein gewi{\ss}er Mensch mit wa{\ss}er, item wa{\ss}er und brot etc. allzeit unterhalten wurde,
und was f\"{u}r nuzen bey einer allezeit simplen und einerley kost.
\pend%
\pstart%
Aus der
\edtext{figur der}{\lemma{figur}\Bfootnote{\textit{(1)} ein \textit{(2)} der \textit{L}}}
hahre\protect\index{Sachverzeichnis}{Haar} eines menschen la{\ss}en sich au{\ss}er Zweifel allerhand n\"{u}zliche consequenzen machen.
Von Nase\protect\index{Sachverzeichnis}{Nase} und anderen will ich nicht sagen.
\pend%
\pstart%
Zu versuchen ob nicht das Antimonachale Antimonium crudum\protect\index{Sachverzeichnis}{antimonium crudum} auch menschen sowohl als pferden\protect\index{Sachverzeichnis}{Pferd} und schweinen\protect\index{Sachverzeichnis}{Schwein} guth sey, wenn man es wie eine cur\protect\index{Sachverzeichnis}{Kur} per gradus anfienge. NB.%
% Hier folgt Bl. 4v.