\begin{ledgroupsized}[r]{120mm}%
\footnotesize%
\pstart%
\noindent%
\textbf{\"{U}berlieferung:}%
\pend%
\end{ledgroupsized}%
\begin{ledgroupsized}[r]{114mm}%
\footnotesize%
\pstart%
\parindent -6mm%
\makebox[6mm][l]{\textit{L}}%
Aufzeichnung:
LH XXXVII 3 Bl. 89.
Zettel (etwa 20 x 6 cm), unregelm\"{a}{\ss}ig beschnitten.
12~Z. auf Bl.~89~r\textsuperscript{o}, 2~Z. auf Bl.~89~v\textsuperscript{o}.
Ursprünglich dürfte der Zettel Teil des Textträgers von N.~97\textsubscript{4}
% LH037,03_084 = Cc 2, Nr. 1142 = Barrages des eaux courantes
gewesen sein.
\newline%
Cc 2, Nr. 1179%
\pend%
\end{ledgroupsized}%
% \normalsize%
%
%
\vspace*{8mm}%
\pstart%
\normalsize%
\noindent%
% [89~r\textsuperscript{o}]
[89~r\textsuperscript{o}] 
J'ay parl\'{e} aujourdhuy (ultimo anni 1675) \`{a} Mons. le duc de Roanez\protect\index{Namensregister}{\textso{de Roannez}, Artus Gouffier 1627-1699} en presence de Mons. de S. Martin, de ma \edtext{pens\'{e}e de dessaler}{\lemma{pens\'{e}e de}\Bfootnote{\textit{ (1) } faire \textit{ (2) } dessaler \textit{L}}} l'eau de mer, sans feu. Car ayant \edtext{avou\'{e}, que}{\lemma{avou\'{e},}\Bfootnote{\textit{ (1) } qu'on \textit{ (2) } que \textit{ L}}} la distillation\protect\index{Sachverzeichnis}{distillation} ne sert de rien, \`{a} cause qu'on peut remplir d'eau la place qui seroit necessaire pour les charbons; je luy dis que j'avois un moyen \edtext{de dessaler}{\lemma{de}\Bfootnote{\textit{ (1) } faire \textit{ (2) } dessaler \textit{ L}}} l'eau sans feu, en la pressant et \edtext{[l'obligeant]}{\lemma{}\Bfootnote{l'obligant\textit{\ L \"{a}ndert Hrsg.}}} de passer par quelque chose, comme pierre, sable, plomb. Il me dit qu'il trouvoit cela fort considerable, d'autant qu'il s\c{c}avoit que l'eau passoit m\^{e}me par le fer; \edtext{et qu'un}{\lemma{et}\Bfootnote{\textit{ (1) } qu'une \textit{ (2) } qu'un \textit{L}}} millier pesant sur piston de demy pouce de diametre a fait pisser l'eau par le fer de deux lignes de largeur. Il me dit de n'en pas parler, afin qu'on le p\^{u}t essayer.
\pend%
\pstart%
Mons. de Galin\'{e}e\protect\index{Namensregister}{\textso{Galin\'{e}e}, Abbé de 1678} a port\'{e} une pierre blanche de Bretagne\protect\index{Ortsregister}{Bretagne}, \`{a} travers de la quelle les choses distill\'{e}es\protect\index{Sachverzeichnis}{distiller} perdent le goust\protect\index{Sachverzeichnis}{go\^{u}t}. L'urine\protect\index{Sachverzeichnis}{urine} m\^{e}me. Mais il faut tousjours. Il y a une difficult\'{e} icy: comment oster la sal\^{u}re au \protect\index{Sachverzeichnis}{plomb}plomb? Car il y restera du \protect\index{Sachverzeichnis}{sel}sel, il en seroit tout chang\'{e}. Je croy qu'en le remettant dans \edtext{l'eau et l'y}{\lemma{l'eau et}\Bfootnote{\textit{ (1) } luy \textit{ (2) } l'y \textit{ L}}} remuant, le sel se dissoudroit et sortiroit.
[89~v\textsuperscript{o}]
\pend%
\pstart%
On pourroit faire passer l'eau \`{a} travers de la poudre de plomb au lieu de sable, et quand l'eau ne voudroit plus passer, ny le plomb retenir du sel d'avantage on le pourroit fondre, et pulveriser derechef.
\pend%
%%%% PR: Das Stück endet hier.