[6~v\textsuperscript{o}]
\pend%
\pstart%
Und da{\ss} wir Menschen mit hochsten unverstand unsere seeligkeit nicht allein, damit es zwar kein wunder dieweil wir noch nie einen seeligen oder verdamten gesehen, sondern auch gesundheit\protect\index{Sachverzeichnis}{Gesundheit} nicht achten. Da wir t\"{a}glich sehen was gleichsam fur hollische marter schohn in diesem Leben, denen
\edtext{so mehr f\"{u}r g\"{u}ther als ihren leib sorgen}{\lemma{so}\Bfootnote{\textit{(1)} daf\"{u}r nicht sor \textit{(2)} mehr [...] sorgen \textit{L}}}
(von der Seele will nicht sagen) angethan werden.
\pend%
\count\Bfootins=1200
\pstart%
Es thate noth da{\ss} ich alle exclamationes exhortationes, paraeneses, und was nur zu excitirung der affecten cr\"{a}fftig gnug bey Predigern, und Oratoren gefunden wird zusammen br\"{a}chte uns unseren unverstand vorzumahlen.
\pend%
\pstart%
Aber ich hoffe mit personen zu thun haben, die gnugsam alle solche dinge fa{\ss}en wenn man sie ihnen schohn auch mit wenig worten sagt. Und dazu giebt mir eine gro{\ss}e hofnung sowohl das dessein der Englischen societat in Mechanicis, als die Instructio in politicis
\edtext{so den Magistris}{\lemma{so}\Bfootnote{\textit{(1)} dem Magistro \textit{(2)} den Magistris \textit{L}}}
supplicationum gegeben worden, damit man erkenne da{\ss} ein ebenm\"{a}"siges in Medicis\protect\index{Sachverzeichnis}{medicus} hochn\"{o}thig sey.
\pend%
\pstart%
Man muss aller orthen apotheker taxen\protect\index{Sachverzeichnis}{Apothekertaxen}, pistilents- und gesundheits-ordnungen\protect\index{Sachverzeichnis}{Pestilenz– und Gesundheitsordnungen} zusammen bringen la{\ss}en.
Legenda \edtext{Verulamii\protect\index{Namensregister}{\textso{Bacon}, Francis 1561-1626} incrementa scientiarum,
% }{\lemma{Verulamii [...] scientiarum}\Cfootnote{\cite{00004}\textsc{F. Bacon}, \textit{De dignitate et augmentis scientiarum}, London 1623.}}
%
%\edtext{
\textit{Historia vitae et} \textit{mortis},%
}{\lemma{Verulamii [...] \textit{mortis}}\Cfootnote{\cite{00004}\textsc{F. Bacon}, \textit{De dignitate et augmentis scientiarum}, London 1623;
\cite{00005}\textsc{Ders.}, \textit{Historia vitae et mortis}, London 1623.}}
%
\edtext{Sanctorii\protect\index{Namensregister}{\textso{Santorio}, Santorio 1561-1636} \textit{Methodus vitandorum errorum omnium in Medicina.}%
}{\lemma{Sanctorii [...] \textit{Medicina}}\Cfootnote{\cite{01127}\textsc{S. Santorio}, \textit{Methodi vitandorum errorum omnium, qui in arte medica contingunt}, Venedig 1630.}}
%
\pend%
\pstart%
Ob ein diaet\protect\index{Sachverzeichnis}{Di\"{a}t} so anzustellen,
da{\ss} es auf viel zugleich das absehen hab, und etliche zu einer Zeit zu conjungiren,
zum exempel music\protect\index{Sachverzeichnis}{Musik} und geruch,\protect\index{Sachverzeichnis}{Geruch}
andere separaten zum exempel music und schlaf.\protect\index{Sachverzeichnis}{Schlaf}
\pend%
\pstart%
Ob ein mittel zu finden daraus mechanice zu judiciren ob der Mensch krafftiger oder schw\"{a}cher, v.g. ponderatio, usus purgantis etc.%
% Hier folgt Bl. 7r.