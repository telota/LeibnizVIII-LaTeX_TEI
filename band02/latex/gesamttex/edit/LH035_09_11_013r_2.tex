\begin{ledgroupsized}[r]{120mm}
\footnotesize 
\pstart                
\noindent\textbf{\"{U}berlieferung:}   
\pend
\end{ledgroupsized}
%           
\begin{ledgroupsized}[r]{114mm}
\footnotesize 
\pstart
\parindent -6mm
\makebox[6mm][l]{\textit{L}}Notiz:
LH XXXV 9, 11 Bl. 13-14.
1 Bog. 2\textsuperscript{o}.
16 Z. in der oberen rechten Spalte von Bl.~13~r\textsuperscript{o}.
Auf B.~13~r\textsuperscript{o} und Bl.~13~v\textsuperscript{o} ist N.~34\textsubscript{4} überliefert.
Die untere H\"{a}lfte von Bl.~13~v\textsuperscript{o} sowie Bl.~14 sind leer.
Leibniz' eigenh\"{a}ndige Datierung und Nummerierung des Bogens am oberen Rand von Bl.~13~r\textsuperscript{o}:
\textit{Frottement (5) part. May. 1675}.
Der Text stimmt nahezu w\"{o}rtlich mit einer Passage von N.~36\textsubscript{1} (S.~\refpassage{LH035,09,11_001r_z1}{LH035,09,11_001r_z2}) überein.%
\\Cc 2, Nr. 00
\pend
\end{ledgroupsized}
%%
\vspace*{8mm}
\pstart
\noindent
[13~r\textsuperscript{o}]
\pend
\pstart 
\normalsize
%
\centering% PR: Bitte als Überschrift gestalten.
Demonstrations Geometriques de quelques
\edtext{propositions fondamentales}{\lemma{}\Bfootnote{propositions fondamentales \textit{erg.} \textit{L}}}\\
d'une partie nouuelle des Mechaniques,
qui traite
du\\
\textso{ frottement.}
\pend
\vspace*{0,5em} 
\pstart
\noindent
Le\textso{ Frottement }\edtext{est la}{\lemma{est}\Bfootnote{\textit{(1)}\ une \textit{(2)}\ la \textit{L}}}
resistence du Lieu par o\`{u} le mobile passe.
\pend
\pstart
J'entends par le\textso{ lieu }la surface du corps ambient (entiere, ou en partie) comme le definit
\edtext{\edtext{Aristote\protect\index{Namensregister}{\textso{Aristoteles}, 384-322 v. Chr.}.}{\lemma{Aristote}\Cfootnote{\cite{00235}\textit{Phys.} IV 4, 212a2-30.}}\protect\\
\hspace*{7,5mm}Cette resistence se fait par la complication}{\lemma{Aristote.}\Bfootnote{\textit{(1)}\ Cette Resistence est de \textit{(2)}\ Cette resistence  \textit{(a)}\ est compli  \textit{(b)}\ se [...] complication \textit{L}}}
de deux causes;
\edtext{et c'est pourquoy elle}{\lemma{et}\Bfootnote{\textit{(1)}\ d'o\`{u} vient qu'elle \textit{(2)}\ c'est pourquoy elle \textit{L}}}
est aussi de deux especes, absolue, et respective.
Je veux traiter icy de la resistence absolue,
et je me reserve de parler de la respective dans un \edtext{autre cahier.}{\lemma{autre cahier}\Cfootnote{Vermutlich N. 35.}}
O\`{u} j'expliqueray la difference qu'il y a entre ces deux Resistences, et leurs origines.
\pend