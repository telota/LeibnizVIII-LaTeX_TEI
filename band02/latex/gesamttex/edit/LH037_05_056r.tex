	\begin{ledgroupsized}[r]{120mm}
	\footnotesize 
	\pstart 
	\noindent\textbf{\"{U}berlieferung:}
	\pend
	\end{ledgroupsized}
	\begin{ledgroupsized}[r]{114mm}
	\footnotesize 
	\pstart \parindent -6mm
	\makebox[6mm][l]{\textit{L}}Aufzeichnung:
LH XXXVII 5 Bl. 56.
1 Bl. 4\textsuperscript{o}.
1\,\nicefrac{1}{5} S. auf Bl.~56~r\textsuperscript{o} und im untersten Teil von Bl.~56~v\textsuperscript{o}.
Bl.~56~v\textsuperscript{o} überliefert zudem N.~17\textsubscript{2}.
Ein Wasserzeichen, beschnitten.
Papier durch Erhaltungsma{\ss}nahmen stabilisiert.%
%	Konzept: LH XXXVII 5 Bl. 56. 1 Bl. 4\textsuperscript{o}. Papier durch Erhaltungsma{\ss}nahmen stabilisiert. 1 \nicefrac{1}{5} S.
	\\Cc 2, Nr. 975 A\pend
	\end{ledgroupsized}
	
	\count\Bfootins=1200
	\count\Afootins=1200

	
	\vspace*{8mm}
	\pstart 
	\normalsize
	\noindent[56~r\textsuperscript{o}] Deux pendules\protect\index{Sachverzeichnis}{pendule} \edtext{inegales}{\lemma{}\Bfootnote{inegales \textit{erg.} \textit{L}}} estant donn\'{e}es, et le nombre \edtext{des battements}{\lemma{nombre}\Bfootnote{\textit{(1)}\ du battement \textit{(2)}\ des battements \textit{L}}} de chacune dans un m\^{e}me temps, (: comme par exemple dans une heure :) estant connu; il faut diviser le plus grand nombre \edtext{par le moindre;}{\lemma{par le}\Bfootnote{\textit{(1)}\ nombre \textit{(2)}\ moindre; \textit{L}}} et prendre par apres le nombre quarr\'{e} du produit ou du quotient: \edtext{et autant de fois que le dit nombre}{\lemma{quotient:}\Bfootnote{\textit{(1)}\ et comme a le dit nombr \textit{(2)}\ et [...] nombre \textit{L}}} quarr\'{e} contient l'unit\'{e} autant de fois la longueur de la plus grande des deux pendules contiendra celle de la petite.
	\pend 
\pstart Par exemple si de deux pendules la plus grande fait 333 vibrations\protect\index{Sachverzeichnis}{vibration} dans \edtext{un}{\lemma{dans}\Bfootnote{\textit{(1)}\ une \textit{(2)}\ un \textit{L}}} certain espace de temps, et la moindre en \edtext{m\^{e}me}{\lemma{}\Bfootnote{m\^{e}me \textit{erg.} \textit{L}}} temps 999, divisant 999 vous \edtext{aurez 3.%
\edtext{}{\lemma{}\Afootnote{\textit{Am Rand:} $\frac{999}{333}$\vspace{-4mm}}}
dont le quarr\'{e} est 9 et par consequent}{\lemma{aurez 3.}\Bfootnote{\textit{(1)}\ et par consequent \textit{(2)}\ dont [...] consequent \textit{L}}} la raison des longueurs sera comme d'un \`{a} 9.
\pend 
\count\Bfootins=1100
\pstart
De m\^{e}me, si \edtext{la moindre}{\lemma{si}\Bfootnote{\textit{(1)}\ l'une \textit{(2)}\ la moindre \textit{L}}} fait $1500$ battements, pendant que la plus grande fait $\displaystyle1000$; divisant $\displaystyle1500$ par $\displaystyle1000$, nous aurons $1+\rule[-4mm]{0mm}{9mm}\displaystyle\frac{1}{2}$, ou reduisant tout \`{a} une fraction, nous aurons $\displaystyle\frac{3}{2}$, dont le quarr\'{e} est $\displaystyle\frac{9}{4}$, par consequent \edtext{la}{\lemma{consequent}\Bfootnote{\textit{(1)}\ une \textit{(2)}\ la \textit{L}}} moindre par exemple ayant quatre pouces la plus grande\rule[-4mm]{0mm}{9mm} en aura 9.
\pend
%\vspace{2mm}
\pstart
$\displaystyle\protect\begin{array}{lllllllll}
% Line 1
% Column 1
\protect\vspace{1mm} \displaystyle\protect\frac{34}{21} \ \protect\mbox{\protect\large f} \ \displaystyle1 +  \displaystyle\protect\frac{13}{21}\  
& % Column 2
\displaystyle\protect\frac{35}{21}\ 
& % Column 3
\displaystyle\protect\frac{\protect\overset{\protect\scriptstyle 14}{\protect\cancel{3}\protect\cancel{4}}}{\protect\cancel{2}\protect\cancel{1}} \ \protect\mbox{\protect\large f} 
& % Column 4
\displaystyle1 \displaystyle\protect\frac{14}{21}\ \lbrack \textit{sic!}\rbrack \ 
& % Column 5
& % Column 6
\hspace{9pt} \displaystyle\efrac{999}{999} 
& % Column 7
& % Column 7
& % Column 9
\hspace{6pt} \displaystyle\efrac{333}{\uline{333}} 
\\
% Line 2 
% Column 1
& % Column 2
& % Column 3
& % Column 4
1\hspace{3pt}\displaystyle\protect\frac{2}{3}\hspace{1pt} 
& % Column 5
& % Column 6
\ \displaystyle\efrac{999}{333} \ \protect\mbox{\protect\large f}\ \displaystyle\protect\frac{3}{1}
& % Column 7
&  % Column 8
\displaystyle\frac{9}{1}
& % Column 9
\hspace{3pt}\displaystyle\efrac{\ 999}{999} 
\\
% Line 3
% Column 1
& % Column 2
& % Column 3
& % Column 4
\displaystyle\protect\overline{\protect\frac{5}{3} \hspace{1pt} \protect\frac{25}{9}} 
& % Column 5
\hspace{-7pt} \displaystyle\protect\mbox{\protect\large f} \ \displaystyle2 \protect\frac{7}{9} \ \protect\frac{\protect\raisebox{0ex}{A}^2}{\protect\raisebox{0ex}{B}^2}\protect
& % Column 6
& % Column 7
& % Column 8
& % Column 9
\hspace{-3pt}\displaystyle\genfrac{}{}{}{}{999}{110889}
\end{array}$
\pend
%
%
%
\vspace*{2mm}
\pstart
\hspace{3pt}$\displaystyle1+\frac{13}{21}$ \hspace{3pt} $\displaystyle1+\frac{13}{21}$ \hspace{3pt} $\displaystyle\frac{34}{21}$ \hspace{3pt} $\displaystyle\frac{\displaystyle\genfrac{}{}{0pt}{}{13}{13}}{9}$
\pend
%
%
%
\vspace{4mm}
\pstart
$\displaystyle\protect\begin{array}[t]{llllll}
% Line 1
% Column 1
\displaystyle1+\frac{13}{21}
& % Column 1
& % Column 2
\displaystyle1\hspace{9pt}\frac{13}{21}
& % Column 3
\displaystyle\efrac{13}{\uline{13}}
\\
% Line 2
% Column 1
& % Column 2
& % Column 3
\uline{\displaystyle1\hspace{9pt}\frac{13}{21}\hspace{2pt}}
& % Column 4
\displaystyle\efrac{39}{}
\\
% Line 3
% Column 1
\displaystyle\frac{26}{42}
& % Column 2
\displaystyle\frac{169}{441}
& % Column 3
\displaystyle\frac{13}{21}\hspace{3pt}\frac{169}{441}
& % Column 4
\displaystyle\frac{13}{\uline{169}}
& % Column 5
\hspace{3pt}\displaystyle\efrac{21}{\uline{21}}
\\
% Line 4
% Column 1
& % Column 2
& % Column 3
\uline{\displaystyle\frac{13}{21}\hspace{12pt}}
& % Column 4
& % Column 5
\displaystyle\efrac{\hspace{3pt}21}{\hspace{-3pt}\uline{42}}
\\
% Line 5
% Column 1
& % Column 2
& % Column 3
& % Column 4
& % Column 5
\displaystyle\efrac{441}{}
\end{array}$
\pend
%
%
%
%\count\Bfootins=1500
%\vspace{4mm}
%\newpage
\pstart\noindent
[56~v\textsuperscript{o}] \hspace*{4mm}\lbrack \textit{Quer zur Schreibrichtung:}\rbrack 
\pend 
%
%
%
\vspace{2mm}
\pstart\noindent
La moindre $1500$:
\pend
%
%
%
\pstart\noindent
La plus grande $1000$
\pend
%
%
%
\vspace*{2mm}
\pstart\noindent
\begin{tabular}{c}
\cancel{1500}
\\
\cancel{1000}
\\
\end{tabular}
$\displaystyle \bigg \vert \frac{3}{2}$
\pend
%
%
%
\vspace{2mm}
\pstart\noindent
\hspace*{1mm}
$\displaystyle \frac{3}{2}$
\begin{tabular}{c}
\textemdash
\\
\textemdash
\\
\end{tabular}
$\displaystyle \frac{3}{2}$ \hspace*{5mm} $\displaystyle \frac{9}{4}$
\pend
%
%
%
\pstart\noindent
\vspace{2mm}
\hspace{4mm}
$\displaystyle 2\ \frac{1}{4}$
\pend
\count\Bfootins=1500
\count\Afootins=1500
\newpage


	 
	
