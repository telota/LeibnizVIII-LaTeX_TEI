\hspace{-1.2mm}[15~v\textsuperscript{o}] Proximo jam loco ostensa ratione ita regendi custodiendique instrumentum semper ut maneat in eodem plano cum duobus objectis observandis; ostendam qua ratione fieri possit, ut quadrans sit semper conservatus perpendicularis et in Azimutho\protect\index{Sachverzeichnis}{azimuth} coelestis objecti hoc fieri potest additione satis facili ad superiorem Machinam, ita ut fixae dioptrae\protect\index{Sachverzeichnis}{dioptra} quadrantis\protect\index{Sachverzeichnis}{quadrans} observent exactam horizontalitatem, et planum quadrantis\protect\index{Sachverzeichnis}{quadrans} semel accommodatum plano objecti coelestis, motu circa axem aequali cum motu objecti circa axem terrae semper conservabitur in eodem plano objecti, cujus Azimuth\protect\index{Sachverzeichnis}{azimuth} et altitudo observari debet. Motus enim axis inclinati ad perpendicularem est semper in Geometrica proportione, et stricte qualis esse debet, ut servet planum quadrantis\protect\index{Sachverzeichnis}{quadrans} exacte in Azimutho\protect\index{Sachverzeichnis}{azimuth} coelestium objectorum, ut is facile inveniet qui tantillum in Geometria sit versatus, et postea amplius demonstrabo, quando ostendam, \textit{what use i have of this joynt,}\edtext{}{\lemma{\textit{joynt},}\Cfootnote{a.a.O., S. 73.}} pro Instrumento universali, Gnomonico, pro aequando tempore et pro \edtext{efficiendo \textit{the hand of a clock}}{\lemma{efficiendo}\Bfootnote{\textit{(1)}\ manus lam \textit{(2)}\ \textit{the hand of a clock} \textit{L}}}\edtext{}{\lemma{\textit{clock}}\Cfootnote{a.a.O., S. 73.}} \textit{move in the shadow of a style,}\edtext{}{\lemma{\textit{style},}\Cfootnote{a.a.O., S. 73.}} (+ ut index horologii moveatur semper in umbra styli +) aliisque multis mechanicis operationibus. 
\pend 
\count\Afootins=1200
\count\Bfootins=1500
\count\Cfootins=1500 
\pstart Superest explicem quot revolutiones cochleae, et quot unius revolutionis partes faciant rectum angulum, aut 90 gradus in quadranti\protect\index{Sachverzeichnis}{quadrans}. Hoc ut inveniatur, in loco ubi facilis sit prospectus pro semicirculo primo dirigantur ambae dioptrae\protect\index{Sachverzeichnis}{dioptra} telescopiorum\protect\index{Sachverzeichnis}{telescopium} directe ad idem objectum idemque ejus punctum, et rectificentur inde Indices ad 0 vel initium divisionum. Inde circumagatur cochlea donec quanta exactitudine \edtext{[mensurari]}{\lemma{}\Bfootnote{mesurari \textit{L \"{a}ndert Hrsg.}}} potest circino mobile telescopium\protect\index{Sachverzeichnis}{telescopium} percurrisse videbitur quadrantem\protect\index{Sachverzeichnis}{quadrans}; et per tria haec \edtext{telescopia\protect\index{Sachverzeichnis}{telescopium} notentur tria puncta}{\lemma{telescopia}\Bfootnote{\textit{(1)}\ sumatur \textit{(2)}\ notentur tria puncta \textit{L}}} horizontis, hoc est duo puncta exacte opposita invicem, in respectu centri quadrantis\protect\index{Sachverzeichnis}{quadrans}, et tertium eodem respectu fere medium, (+ fere inquam quia et Telescopium\protect\index{Sachverzeichnis}{telescopium} quo observatur nondum exacte sed circiter medium habemus +) ostendi supra quomodo rectificandae dioptrae\protect\index{Sachverzeichnis}{dioptra} fixae, ita ut prorsum retrorsumque respiciant, quo consequenter facto, observo suppositum angulum rectum cum mobili dioptra\protect\index{Sachverzeichnis}{dioptra} in quadrante\protect\index{Sachverzeichnis}{quadrans} et cum dioptra\protect\index{Sachverzeichnis}{dioptra} in quadrante\protect\index{Sachverzeichnis}{quadrans} fixa respiciens antrorsum, et noto diligenter duo objectorum puncta. Inde neque cochlea neque mobili brachio quadrantis\protect\index{Sachverzeichnis}{quadrans} motis; eadem objecta invenio per dioptras\protect\index{Sachverzeichnis}{dioptra} mobiles et fixas, respiciens retrorsum, et dirigens unam dioptrarum\protect\index{Sachverzeichnis}{dioptra} exacte ad unum punctum, exacte observo quantum variet ab altero objecto, intra vel extra. Inde dimidietur differentia aestimatione et moveatur mobilis dioptra\protect\index{Sachverzeichnis}{dioptra} ope cochlearum, ita ut respiciat medium punctum. Atque hoc examen aliquoties repeto donec non amplius appareat differentia, et ita certus sum angulos a quolibet latere mobilis tubi inter ipsum et dioptras\protect\index{Sachverzeichnis}{dioptra}, prorsum ac retrorsum introspiciendo esse inter se aequales, atque ideo ambos esse rectos. Quo invento observo per Indices in cochleari lamina et limbo, quot revolutiones et quot revolutionis partes cochlea fuit acta ad aperiendum hunc angulum, is numerus respondebit gradibus 90 quo diviso in partes 90 habentur numeri pro quolibet gradu, et dividendo communem differentiam inter eos in 60 partes \edtext{habebis minutorum numerum}{\lemma{habebis}\Bfootnote{\textit{(1)}\ minutum et \textit{(2)}\ minutorum numerum \textit{L}}}; eodem modo et secunda habebis subdividendo per 60 sed hoc non necesse, nam subducendo numerum proxime minorem in Tabula quam ratione revolutionum condendam supra diximus pag. 55, habebis gradum minutum, et aliquos forte numeros praeterea, quos facile invenies \edtext{per}{\lemma{}\Bfootnote{per \textit{erg. L}}} brevem tabulam communium differentiarum secundarum. Objiciet aliquis forte divisiones  in quadrante\protect\index{Sachverzeichnis}{quadrans} non respondere divisionibus supra factis \textit{in the plate}\edtext{}{\lemma{\textit{in the plate}}\Cfootnote{a.a.O., S. 74.}} (+ credo in the screw plate +) respondeo partim respondent partim non. Respondent in eo quod omnes divisiones factae integris revolutionibus monstrant exacte idem per Indices id quod faciunt in quadrante\protect\index{Sachverzeichnis}{quadrans}; sed partes revolutionum non sunt exacte et \edtext{geometrice respondentes}{\lemma{geometrice}\Bfootnote{\textit{(1)}\ aequales \textit{(2)}\ respondentes \textit{ L}}} \textit{are not exactly and mathematically the same pointed out by the Index, upon a ring equally divided, that are made upon a limb of a quadrant\protect\index{Sachverzeichnis}{quadrans}.}\edtext{}{\lemma{\textit{quadrant}.}\Cfootnote{a.a.O., S. 75.}} Sed hoc sensu etiam 60 pedum telescopio\protect\index{Sachverzeichnis}{telescopium} adjuto, apparent aequalia, ideoque sufficientia nec opus rectificatione, si quis velut summa subtilitate faciet divisiones [\textit{on}]\edtext{}{\Bfootnote{\textit{at}\textit{\ L \"{a}ndert Hrsg.}}} \textit{the Index Ring}\edtext{}{\lemma{\textit{Ring}}\Cfootnote{a.a.O., S. 75.}} secundum proportiones differentiarum Tangentium, \textit{that are subtended within half the Compass of the distance of the two next Threads}\edtext{}{\lemma{\textit{Threads}}\Cfootnote{a.a.O., S. 75.}}, et certe in ipsis minutis inveniemus differentiam in 6 minutis etiam non differe nisi \rule[-4mm]{0mm}{10mm}$\displaystyle\frac{2}{1000,000}$ partibus, quod est millies subtilius quam sensus etiam armatus dare possit. 
\pend 
\vspace{1mm}
\count\Afootins=1500
\count\Bfootins=1500
\count\Cfootins=1500 
\pstart