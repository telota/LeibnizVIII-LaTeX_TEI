\begin{ledgroupsized}[r]{120mm}
\footnotesize 
\pstart    
\noindent\textbf{\"{U}berlieferung:} 
\pend
\end{ledgroupsized}
\begin{ledgroupsized}[r]{114mm}
\footnotesize 
\pstart \parindent -6mm
\makebox[6mm][l]{\textit{L}}%
Reinschrift mit Verbesserungen: LH XXXV 10, 9 Bl. 2. 1 Bl. beschnitten (11 x 17 cm). 1 S. auf Bl.~2~r\textsuperscript{o}. Bl.~2~v\textsuperscript{o}
leer. Auszug aus N. 28\textsubscript{1} mit Änderungen.\\
Cc 2, Nr. 1190 B
\pend
\end{ledgroupsized}

\vspace{8mm}
\pstart
\noindent
[2~r\textsuperscript{o}]
\pend
\pstart\noindent
\normalsize
Appellons \begin{tabular}[t]{lr}
le sinus droit \textit{EM},&\textit{y}\\
le Rayon \textit{AL},&\textit{a}\\
le petit Rayon \textit{LI} ou \textit{LK},&\textit{b}\\
la force absolue du poids,&\textit{g}\\
\end{tabular}
\pend
\vspace{0.5em}
\pstart \noindent et \setline{10}\textso{la force de la machine} sera $\rule[-4mm]{0mm}{10mm}\displaystyle\frac{yag + ag\sqrt{a^2 - y^2} - gby - gb\sqrt{a^2 - y^2}}{ba}$, ou $\rule[-4mm]{0mm}{10mm}\displaystyle y + \sqrt{a^2 - y^2}, \smallfrown \frac{a}{b} - 1, \smallfrown \frac{g}{a}$. C'est \`{a} dire, prenez la somme de \textit{ME} et \textit{ML}; et la multipliez par $\rule[-4mm]{0mm}{10mm}\displaystyle\frac{a}{b}-1$; et le produit, par $\rule[-4mm]{0mm}{10mm}\displaystyle\frac{g}{a}$. Ce qui en proviendra, sera la force de la machine\protect\index{Sachverzeichnis}{machine}, en quelque situation qu'elle puisse estre. Ou si vous voulez la force absolue du poids\protect\index{Sachverzeichnis}{poids}, sera \`{a} la force de la machine, dans l'estat donn\'{e}, comme $ba$ \`{a} $\rule[-4mm]{0mm}{10mm}\displaystyle y + \sqrt{a^2 - y^2}, \smallfrown a - b$ \textso{et en termes de Geometrie\protect\index{Sachverzeichnis}{geometrie}, comme le rectangle \textit{ELK}, au rectangle compris soubs }\edtext{\textso{\textit{HI} et \textit{MN}}}{\lemma{\textit{HI} et}\Bfootnote{\textit{(1)}\ $LM+ME$ \textit{(2)}\ \textit{MN}: \textit{L}}}: Theoreme\protect\index{Sachverzeichnis}{theoreme} assez beau, et d'un tres grand usage pour \edtext{calculer}{\lemma{}\Bfootnote{calculer \textit{erg. L}}} toutes sortes des mouuements circulaires\protect\index{Sachverzeichnis}{mouvement circulaire}.
\pend