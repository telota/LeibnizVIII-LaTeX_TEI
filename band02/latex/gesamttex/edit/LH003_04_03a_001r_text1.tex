\begin{ledgroupsized}[r]{120mm}%
\footnotesize%
\pstart%
\noindent\textbf{\"{U}berlieferung:}%
\pend%
\end{ledgroupsized}%
\begin{ledgroupsized}[r]{114mm}%
\footnotesize%
\pstart%
\parindent -6mm%
\makebox[6mm][l]{\textit{L\textsuperscript{1}}}%
Auszüge mit Bemerkungen aus einem verschollenen Manuskript von René Descartes:
LH~III~4,~3a Bl.~1.
1~Bl. 2\textsuperscript{o}. \unitfrac{3}{4} S. auf Bl.~1~r\textsuperscript{o}
(unsere Druckvorlage).
Das untere Viertel von Bl.~1~r\textsuperscript{o} sowie Bl.~1~v\textsuperscript{o} überliefern N.~82.% = LH003,04,03a_001 = De refract.
\newline%
Cc 2, Nr. 1323 A (tlw.)
%
\pend%
\end{ledgroupsized}%
%
\begin{ledgroupsized}[r]{114mm}%
\footnotesize%
\pstart%
\parindent -6mm%
\makebox[6mm][l]{\textit{L\textsuperscript{2}}}%
Abschrift von \textit{L\textsuperscript{1}} mit Auslassungen und Verbesserungen:
LH~III~5 Bl.~49.
1~Bl. 4\textsuperscript{o}. 1\,\unitfrac{2}{3}~S.
R\"{a}nder beschnitten.
Spuren eines Wasserzeichens.%
\newline%
Cc 2, Nr. 1323 B%
\pend%
\end{ledgroupsized}%
%
\begin{ledgroupsized}[r]{114mm}%
\footnotesize%
\pstart%
\parindent -6mm%
\makebox[6mm][l]{\textit{E\textsuperscript{1}}}%
\cite{01121}\textsc{R.~Descartes}, \textit{{\OE}uvres in\'{e}dites}, hrsg. von \textsc{L.A.~Foucher de Careil}, Bd.~II, Paris 1860, S.~210-213 (mit franz\"{o}sischer \"{U}bersetzung; nur \textit{L\textsuperscript{1}}).
\pend%
\end{ledgroupsized}%
%
\begin{ledgroupsized}[r]{114mm}%
\footnotesize%
\pstart%
\parindent -6mm%
\makebox[6mm][l]{\textit{E\textsuperscript{2}}}%
\cite{00120}\textsc{R.~Descartes}, \textit{{\OE}uvres}, hrsg. von \textsc{C.~Adam} und \textsc{P.~Tannery}, Bd.~XI, Paris 1909, S.~641-644 (\textit{L\textsuperscript{1}} und \textit{L\textsuperscript{2}}).%
\pend
\end{ledgroupsized}
%
%
% \normalsize
\vspace*{5mm}%
\begin{ledgroup}%
\footnotesize%
\pstart%
\noindent%
\footnotesize{%
\textbf{Datierungsgr\"{u}nde:}
\textit{L\textsuperscript{1}} ist von Leibniz datiert.
\textit{L\textsuperscript{2}} ist nicht datiert:
als Abschrift von \textit{L\textsuperscript{1}} muss \textit{L\textsuperscript{2}} aber zu einem sp\"{a}teren Zeitpunkt angefertigt worden sein.
Die schwer erkennbaren Spuren des Wasserzeichens im Träger von \textit{L\textsuperscript{2}} ermöglichen keine weitere zeitliche Eingrenzung.}%
\pend%
\end{ledgroup}%
%
%
\vspace*{8mm}%
\pstart%
\normalsize%
\noindent%
% [1~r\textsuperscript{o}]%
[1~r\textsuperscript{o}] 
\edtext{Descripsi 24 Febr. 1676.}{\lemma{Descripsi [...] 1676.}\Bfootnote{\textit{nicht in L\textsuperscript{2}}\hspace{-3mm}}} 
\edtext{Excerptum ex Autogra\pgrk{f}o Cartesii\protect\index{Namensregister}{\textso{Descartes}, Ren\'{e} 1596-1650}}{\lemma{}\Bfootnote{Excerptum % ex Autogra{\protect\pgrk{f}}o 
[...] Cartesii \textit{L\textsuperscript{1}}\quad Excerptum ex Cartesii Autogra\-pho de Purgantibus, et aliis \textit{L\textsuperscript{2} am Rand}\hspace{-3.5mm}}}
\pend%
\vspace*{1.0em}% PR: Rein provisorisch !!!
\pstart%
\centering%
\textso{Remedia,}%
\edlabel{remedia1}%
\edtext{}{{\xxref{remedia1}{remedia2}}\lemma{}\Bfootnote{\textso{Remedia} [...] transire. \textit{nicht in L\textsuperscript{2}}\hspace{-3.5mm}}}%
\textso{ et vires Medicamentorum}
\pend%
\vspace*{0.5em}% PR: Rein provisorisch !!!
\count\Bfootins=1200
\count\Cfootins=1200
\count\Afootins=1200
\pstart%
\noindent%
Lac in visceribus coagulatum; et vinum, et aqua frigida, nimis calentibus hausta \edtext{inter venena numerantur}{\lemma{}\Bfootnote{in\-ter venena numerantur \textit{erg. L\textsuperscript{1}}}}. Unde patet, facile etiam maxime communia alimenta in noxiam vim transire.\edlabel{remedia2}
\pend%
\pstart%
Crediderim
\edtext{ventriculi cutem esse laxam et porosam, et per quam serosus humor e toto corpore in eum illabitur.}{\lemma{ventriculi [...] illabitur}\Bfootnote{\textit{unterstrichen in L\textsuperscript{2}}}}
Hoc patet ex eo quod famelicis
\edtext{cibum videntibus}{\lemma{cibum}\Bfootnote{\textit{(1)} videntes \textit{(2)} videntibus \textit{L\textsuperscript{1}}}}
humor iste usque in palatum redundet, istis nempe meatibus imaginationis vi laxatis. Quia scilicet humor iste ad digerendos cibos est utilis, ut foenum\protect\index{Sachverzeichnis}{foenum} si aqua aspersum recondatur,
\edtext{incalescet et putrefiet.}{\lemma{}\Bfootnote{incalescet et putrefiet \textit{L\textsuperscript{1}}\quad incalescit et putrefit \textit{L\textsuperscript{2}}}}
\pend%
\pstart%
\edtext{Hinc et facile reddi ratio poterit multorum astringentium}{\lemma{}\Bfootnote{\hspace{-1.5mm}Hinc [...] astringentium \textit{unterstrichen in L\textsuperscript{2}}\hspace{-3.5mm}}}
ut \edtext{vertdegris,}{{\lemma{}\Afootnote{\textit{Über} vertdegris: \Denarius}}{\lemma{vertdegris,}\Bfootnote{%\hspace{-1.5mm}
\textit{L\textsuperscript{1}}\quad viride aeris, item \textit{L\textsuperscript{2}}\hspace{-3.5mm}}}}
acerbi omnes fructus, \edtext{sorba,\protect\index{Sachverzeichnis}{sorbum}}{\lemma{sorba,}\Bfootnote{%\hspace{-1.5mm}
\textit{L\textsuperscript{1}}\quad ut sorba \textit{L\textsuperscript{2}}}}
mespili,\protect\index{Sachverzeichnis}{mespilus} etc.
\edtext{Certum}{\lemma{}\Bfootnote{Certum \textit{L\textsuperscript{1}}\quad quos certum \textit{L\textsuperscript{2}}}}
est meatus istos occludere, contra vero
\edtext{\mercury. \earth.\protect\index{Sachverzeichnis}{antimonium}
\edtext{quae frigida, atque}{\lemma{quae}\Bfootnote{\textit{(1)} figi, atque \textit{(2)} frigida, atque \textit{L\textsuperscript{1}}}}
humida ut pruna,\protect\index{Sachverzeichnis}{prunum}
\edtext{[cassiam],\protect\index{Sachverzeichnis}{cassia}}{\lemma{cassia}\Bfootnote{\textit{L\textsuperscript{1} \"{a}ndert Hrsg.}}}
poma,\protect\index{Sachverzeichnis}{pomum}}{\lemma{}\Bfootnote{\mercury. \earth. [...] poma, \textit{L\textsuperscript{1}}\quad Mercurium et Antimonium, item quae frigida simul et humida sunt, ut pruna, poma, cassiam \textit{L\textsuperscript{2}}}}
etc. illos laxare; ideoque esse
\edtext{purgantia}{\lemma{}\Bfootnote{purgantia \textit{unterstrichen in L\textsuperscript{2}}}}.
Possunt vero alia esse purgantia vel astringentia, alias ob causas; sed hanc puto
\edtext{praecipuam, quae}{\lemma{praecipuam, quae}\Bfootnote{\textit{L\textsuperscript{1}}\quad esse praecipuam. Quae \textit{L\textsuperscript{2}}}}
enim cito corrumpuntur in ventriculo, ut cibi
\edtext{delicatiores solitis etc.}{\lemma{}\Bfootnote{delicatiores solitis etc. \textit{L\textsuperscript{1}}\quad solitis delicatiores \textit{L\textsuperscript{2}}}}
fructus horarii, etc. faeces quidem molles reddunt, sed non ideo purgant ex reliquo corpore,
\edtext{item quae astringunt, sed tantum ex accidenti}{\lemma{}\Bfootnote{, item [...] accidenti \textit{nicht in L\textsuperscript{2}}}}.
\pend%
\pstart%
Notandum \edtext{astringentia fere omnia juvare concoctionem}{\lemma{}\Bfootnote{astringentia [...] concoctionem \textit{unterstrichen in L\textsuperscript{2}}}}. \edtext{Quo minus enim est humoris serosi in stomacho}{\lemma{}\Bfootnote{est humoris serosi in stomacho, \textit{L\textsuperscript{1}}\quad humoris serosi in ventriculo est, \textit{L\textsuperscript{2}}}}, eo magis calor accenditur. Unde fit, ut quaedam astringentia post cibum sumta laxent ventriculum ex accidenti, quoniam accelerant concoctionem, ut Cydoniacum\protect\index{Sachverzeichnis}{cydoniacum}.
\pend%
\pstart%
\edtext{Ventriculus premit cibos intus conclusos,}{\lemma{}\Bfootnote{Ventriculus [...] conclusos \textit{unterstrichen in L\textsuperscript{2}}}}
et se ad eorum quantitatem accommodat.
Hinc famelici videntes cibum vi imaginationis stomachum comprimunt, antequam cibus eo ingressus, unde aquae ad os ascendunt.
\edtext{Purgantia vero fortasse quaedam sunt, quae obstant ne comprimatur,}{\lemma{}\Bfootnote{Purgantia [...] comprimatur \textit{unterstrichen in L\textsuperscript{2}}}}
ut \edtext{\mercury\protect\index{Sachverzeichnis}{mercurius} qui forte}{\lemma{}\Bfootnote{\mercury\ qui forte \textit{L\textsuperscript{1}}\quad Mercurius, qui fortasse \textit{L\textsuperscript{2}}}}
resolvit ejus nervos quod esset periculosissimum.
\pend%
%\newpage% PR: Rein provisorisch !!!
\pstart%
\edtext{\edtext{Virgae}{\lemma{}\Afootnote{\hspace{-1.8mm}\textit{Über} Virgae: \Denarius}}
aureae totius plantae pulvis drachmae pondere potus, item semen genistae
\edtext{calculum}{\lemma{}\Afootnote{\hspace{-1.8mm}\textit{Über} calculum: \Denarius\vspace{-8mm}}}
in vesicis renibusque comminuit}{\lemma{}\Bfootnote{Virgae [...] comminuit. \textit{nicht in L\textsuperscript{2}}}}.
\pend%
% \newpage% PR: Rein provisorisch !!!
\pstart%
Purgant quaedam molliendo faeces ut malva, alia lubricando intestina, ut butyrum alia comprimendo faeces, ut cydonia post pastum, alia
\edtext{abstergendo}{\lemma{abstergendo}\Bfootnote{\textit{L\textsuperscript{1}}\quad%
abstirgendo \textit{L\textsuperscript{2}}}}
intestina, ut aqua salsa, vel etiam dulcis; alia incidendo et aperiendo poros,
ut \protect\index{Sachverzeichnis}{cremor tartari}cremor
\edlabel{LH003,04,03a_001}%
\edtext{}{{\xxref{LH003,04,03a_001}{LH003,04,03a_002}}\lemma{}\Bfootnote{%
{\protect\raisebox{-0.5mm}{\protect\includegraphics[width=0.02\textwidth]{images/taros.pdf}}}%
\textsuperscript{ri}, alii \textit{L\textsuperscript{1} ändert Hrsg.}\quad tartari: alia \textit{L\textsuperscript{2}}\hspace{-2mm}}}%
\protect\raisebox{-0.5mm}{\protect\includegraphics[width=0.02\textwidth]{images/taros.pdf}}\textsuperscript{ri}, [alia]%
\edlabel{LH003,04,03a_002}
%%
nervos retentrici inservientes
\edtext{resolvendo}{\lemma{resolvendo}\Bfootnote{\textit{erg. L\textsuperscript{1}}\hspace{-2mm}}}
ut \edtext{\mercury.\protect\index{Sachverzeichnis}{mercurius}}{\lemma{}\Bfootnote{\mercury.\ \textit{L\textsuperscript{1}}\quad Mercurius. \textit{L\textsuperscript{2}}\hspace{-2mm}}}
Sed et mille aliis modis alia possunt
\edtext{purgare, ut}{\lemma{}\Bfootnote{purgare\ \textbar\ . Sed et mille aliis modiis alia possunt purgare \textit{streicht Hrsg.}\ \textbar\ , ut \textit{L\textsuperscript{1}}}}
venarum \edtext{orificia obturando,}{\lemma{orificia}\Bfootnote{\textit{(1)} obturendo, \textit{(2)} obturando, \textit{L\textsuperscript{1}}}}
coctionem impediendo, etc.
Quin etiam sum expertus aliquando, vini Hispanici potum me purgasse, calefaciendo scilicet sanguinis massam ita ut ex eo multi vapores in ventriculum delabantur, atque instar aquae dulcis copiose faecibus misceantur. Quod mihi manifestum fuit,
\edtext{quoniam}{\lemma{quoniam}\Bfootnote{\textit{L\textsuperscript{1}}\quad quia \textit{L\textsuperscript{2}}}}
alia vice, eodem vino mane sumto multas urinas instar mellis pellucidas et insipidas promoverit, tunc scilicet magis apertis meatibus in vesicam quam in alvum.
\pend%
\pstart%
\textso{Alvi}\edlabel{remedia3}%
\edtext{}{{\xxref{remedia3}{remedia4}}\lemma{}\Bfootnote{\textso{Alvi} [...] solicitaretur. \textit{nicht in L\textsuperscript{2}}}}%
\textso{ egestio difficillima post menses }%
\edtext{\lbrack+}{\lemma{\lbrack+}\Cfootnote{Eckige Klammer von Leibniz.}}%
~credo post menses aliquot~%
\edtext{+\rbrack}{\lemma{+\rbrack}\Cfootnote{Eckige Klammer von Leibniz.}}%
\textso{ sic provocata }728.
Fellis taurini recentis, butyri insulsi, hellebori nigri, extracti diacolocynthidis diagridii et croci partes aequales, in unam massam redactae, et igni ad mellis consistentiam decoctae.
\edtext{Italicae nucis}{\lemma{}\Afootnote{\textit{Über} Italicae nucis: welsche nu{\ss}}}
testae inditae, umbilico impositae sunt.
Ligataque fuit mox ne caderet, et binae
\edtext{(+~credo testae~+),}{\lemma{(+~credo testae~+)}\Bfootnote{\textit{erg. L\textsuperscript{1}}}}
diebus singulis, potionibus
\edtext{intus absumtis}{\lemma{intus}\Bfootnote{\textit{(1)} assumtis \textit{(2)} absumtis  \textit{L\textsuperscript{1}}}}
(+~\Denarius\ puto assumtis~+)
sic repleta impositae sunt
(\phantom)\hspace{-1.2mm}+~binae credo testae sic repletae impositae
\edtext{sunt~+\phantom(\hspace{-1.2mm}).
Primis diebus}{\lemma{sunt \phantom(\hspace{-1.2mm}+).}\Bfootnote{\textit{(1)} Primo die \textit{(2)} Primis diebus \textit{L\textsuperscript{1}}}}
nihil praeter fluctuationes et murmura a patiente sentiebantur; tertia die cum immensis doloribus supervenit egerendi desiderium, at
\edtext{induratis excrementis}{\lemma{induratis}\Bfootnote{\textit{(1)} intestinis \textit{(2)} excrementis \textit{L\textsuperscript{1}}}}
non successit excretio, donec vituli abdomen recens, cum oleo antiquo ad ignem cribratum et calens ventriculo induceretur, digitisque felle et butyro inunctis anus solicitaretur.\edlabel{remedia4}
\pend%
\count\Bfootins=1500
\count\Cfootins=1500
\count\Afootins=1500
%%%% PR: Hier endet das Stück.