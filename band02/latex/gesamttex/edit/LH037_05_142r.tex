\begin{ledgroupsized}[r]{120mm}
\footnotesize 
\pstart 
\noindent\textbf{\"{U}berlieferung:}
\pend
\end{ledgroupsized}
\begin{ledgroupsized}[r]{114mm}
\footnotesize 
\pstart \parindent -6mm
\makebox[6mm][l]{\textit{L}}Reinschrift mit Verbesserungen:
LH XXXVII 5 Bl. 142.
1 Bl. 4\textsuperscript{o}.
1 S. auf Bl.~142~r\textsuperscript{o}. Bl.~142~v\textsuperscript{o} leer.
Blatt durch Papiererhaltungsma{\ss}nahmen stabilisiert.%
\\Cc 2, Nr. 948 
\pend
\end{ledgroupsized}

\vspace*{5mm}
\begin{ledgroup}
\footnotesize 
\pstart
\noindent\footnotesize{\textbf{Datierungsgr\"{u}nde}: Im vorliegenden St\"{u}ck N.~38 werden die relevantesten Ergebnisse der Untersuchung \"{u}ber die Reibung als Ursache der gleichförmigen Verz\"{o}gerung eines sich in einem widerstehenden Medium bewegenden K\"{o}rpers nahezu stichwortartig zusammenfasst.
S\"{a}mtliche in N.~36 formulierten einschl\"{a}gigen Thesen sind auch in N.~38 anzutreffen.
Demgemäß d\"{u}rfte N.~38 zu etwa der gleichen Zeit wie N.~36 entstanden sein.}
\pend
\end{ledgroup}

\vspace*{8mm}
\pstart 
\normalsize
\noindent
[142~r\textsuperscript{o}] Personne ayant reduit sous des Loix Geometriques la perte du mouuement qui se fait par le\textso{ frottement }du mobile contre son support, ou contre le milieu par le quel il passe; j'y ay travaill\'{e} depuis quelques jours, et je trouue la recherche assez considerable.
\pend
\count\Bfootins=1200
\count\Cfootins=1200
\count\Afootins=1200
\pstart
Car les corps jett\'{e}s, les pendules, les balances, \edtext{les pompes,}{\lemma{}\Bfootnote{les pompes, \textit{erg.} \textit{L}}} les machines \`{a} lever des fardeaux, \edtext{les vaisseaux,}{\lemma{}\Bfootnote{vaisseaux  \textbar\ qui ont de la peine \`{a} percer l'eau \textit{ gestr.}\ \textbar\ , \textit{L}}} y sont interessez, et je m'\'{e}tonne qu'une partie si necessaire de la mechanique \edtext{n'a}{\lemma{n'}\Bfootnote{\textit{(1)}\ ait \textit{(2)}\ a \textit{L}}} pas encor est\'{e} \edtext{cultiv\'{e}e. Souuent des beaux projets n'ont pas reussi, \`{a} cause de l'imperfection de la matiere, ou plustost \`{a} cause du frottement, dont les Mathematiciens ne parlent quasi point, comme si c'estoit une chose purement materielle et sujette au hazard et incapable de calcul.}{\lemma{cultiv\'{e}e}\Bfootnote{\textit{(1)}\ . \ \textit{(2)}\  dont la faute a fait que souuent \textit{(3)}\ . Souuent [...] calcul. \textit{L}}}
\pend
\pstart
Galilaei\protect\index{Namensregister}{\textso{Galilei} (Galilaeus, Galileus), Galileo 1564-1642} a trait\'{e} de l'acceleration et de la diminution du mouuement qui vient de la pesanteur; et il a demonstr\'{e} geometriquement certaines \edtext{propositions,}{\lemma{propositions}\Cfootnote{\cite{00050}\cite{00048}Vgl. den Abschnitt \textit{De motu naturaliter accelerato} in G. \textsc{Galilei}, \textit{Discorsi}, Leiden 1638, S. 156ff. (\textit{GO} VIII, S. 197ff.).}} [qui]\edtext{}{\Bfootnote{que \textit{\ L \"{a}ndert Hrsg.}}} sont toutes differentes de celles que j'ay \'{e}tablies \`{a} l'\'{e}gard du retardement qui vient du frottement. En voicy quelques unes des miennes[:]
\pend
\pstart
% PR: Folgenden Absatz bitte ganz einrücken.
Dans un mouuement uniforme en soy m\^{e}me, mais retard\'{e} \edtext{continuellement}{\lemma{}\Bfootnote{continuellement \textit{ erg.} \textit{ L}}} par le\\
\hspace*{7,5mm}frottement d'un milieu homogene;
\pend
\newpage
\begin{Geometrico}
Les\textso{ temps }employ\'{e}s estant comme des\textso{ nombres,}\textso{ les espaces }parcourus seront comme leurs\textso{ Logarithmes.}
\end{Geometrico}
\pstart
% PR: Folgenden Absatz bitte ganz einrücken.
NB. Voicy la representation des Logarithmes par la Physique, quoyque on n'ait p\^{u}\\
\hspace*{7,5mm}encor les representer dans la Geometrie, par une ligne exactement descriptible.
\pend
\begin{Geometrico}
Les espaces parcourus estant en progression Arithmetique, les temps employ\'{e}s seront en progression geometrique.
\end{Geometrico}
\begin{Geometrico}
Les \edtext{forces}{\lemma{}\Afootnote{\textit{\"{U}ber} forces: vitesses\vspace{-6mm}}} du corps dans chaque moment du mouuement, sont en raison reciproque des temps employ\'{e}s.
\end{Geometrico}
\begin{Geometrico}
Les retardations ne sont pas uniformes (:~\edtext{comme celles}{\lemma{comme}\Bfootnote{\textit{(1)}\ dans les co \textit{(2)}\ celles \textit{L}}} qui viennent de la pesanteur~:) mais en raison reciproque doubl\'{e}e, des
temps.\edlabel{37.05_142r_01}
%
% PR: Die Lösung für \edtext bei More-geometrico-Gestaltung besteht in der Anwendung von \edlabel + \xxref.
%
\end{Geometrico}
\begin{Geometrico}
Un corps\edlabel{37.05_142r_02}\edtext{}{{\xxref{37.05_142r_01}{37.05_142r_02}}\lemma{temps.}\Bfootnote{\textit{(1)}\ Les memes  \textit{(a)}\ raiso \textit{(b)}\ retardations sont co \textit{(2)}\  Un corps \textit{L}}} port\'{e} en m\^{e}me temps par deux mouuements, l'un uniforme, l'autre decroissant \`{a} cause du frottement, d\'{e}crira cette ligne admirable, qu'on appelle communement\textso{ Logarithmique }la quelle si elle pouuoit estre d\'{e}crite exactement par un certain mou\-uement continu, seroit d'un \edtext{usage incomparable}{\lemma{usage}\Bfootnote{\textit{(1)}\ admirable \textit{(2)}\ incomparable \textit{L}}} dans les mathematiques. Mais jusqu'icy on n'en a p\^{u} donner, qu'un certain nombre fini des points.
\end{Geometrico}
\count\Bfootins=1500
\count\Cfootins=1500
\count\Afootins=1500

