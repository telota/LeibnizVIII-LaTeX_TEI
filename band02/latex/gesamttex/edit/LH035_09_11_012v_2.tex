\begin{ledgroupsized}[r]{120mm}
\footnotesize
\pstart
\noindent\textbf{\"{U}berlieferung:}
\pend
\end{ledgroupsized}
%
\begin{ledgroupsized}[r]{114mm}
\footnotesize
\pstart
\parindent -6mm
\makebox[6mm][l]{\textit{L}}Konzept: LH XXXV 9, 11 Bl. 11-14. 2 Bog. 2\textsuperscript{o}. Etwa 2 S. auf Bl.~12~v\textsuperscript{o} bis Bl.~13~v\textsuperscript{o}.
Auf Bl.~11~r\textsuperscript{o} bis Bl.~12~v\textsuperscript{o} (mittig) ist N.~34\textsubscript{3} überliefert.
Die untere H\"{a}lfte von Bl.~13~v\textsuperscript{o} sowie Bl.~14 sind leer.
Auf Bl.~13~r\textsuperscript{o} ist ferner N.~34\textsubscript{5} überliefert.
Leibniz' eigenh\"{a}ndige Datierung und Nummerierung der Bogen:
\textit{Frottement part. (4) May 1675} am oberen Rand von Bl. 11~r\textsuperscript{o};
\textit{Frottement (5) part. May. 1675} am oberen Rand von Bl. 13~r\textsuperscript{o}.
Gleicher Wasserzeichentypus auf Bl.~11 und Bl.~14.%
\\Cc 2, Nr. 965 F
\pend
\end{ledgroupsized}
%%
\vspace*{8mm}
\count\Bfootins=1200
\count\Cfootins=1200
\count\Afootins=1200
\pstart
\noindent
[12~v\textsuperscript{o}] 
On feroit peut estre mieux de former ce petit discours ainsi[:] 
\pend
\pstart
\vspace*{3mm}% PR: Provisorisch!
\centering
Essay de quelques Demonstrations
\edtext{Mechaniques, DU FROTTEMENT}{\lemma{Mechaniques,}\Bfootnote{\textit{(1)}\  sur le\textso{ frottement }\protect\index{Sachverzeichnis}{frottement}\textit{(2)}\ DU FROTTEMENT \textit{L}}}
\pend
\vspace*{0.5em}% PR: Provisorisch!
\begin{Geometrico}% PR: Erste Zeile bitte hängend (more geometrico)
Le\textso{ Frottement }est la resistence\protect\index{Sachverzeichnis}{r\'{e}sistance} du lieu par o\`{u} le mobile\protect\index{Sachverzeichnis}{mobile} \edtext{passe.\protect\\
J'entends par le\textso{ lieu,} la surface du corps ambient toute entiere, ou en partie, comme le definit \edtext{Aristote.}{\lemma{Aristote}\Cfootnote{\cite{00235}\textit{Phys.} IV 4, 212a2-30.}}
\end{Geometrico}
\pstart% PR: Erste Zeile bitte hängend (more geometrico)
\noindent La\textso{ Resistence }est absolue ou respective.}{\lemma{passe.}\Bfootnote{\textit{(1)}\ Cette\textso{ Resistence }est absolue ou respective en ent \textit{(2)}\ J'entends [...] ambient  \textbar\ toute [...] partie \textit{erg.}\ \textbar\ , comme [...] respective. \textit{L}}}
\pend
\begin{Geometrico}
\hspace{7,5mm}Car je remarque qu'il y a deux especes de resistence dans les corps sensibles,
dont \edtext{les origines ou principes}{\lemma{les}\Bfootnote{\textit{(1)}\ principes \textit{(2)}\ origines ou principes \textit{L}}}
sont fort differens.
\end{Geometrico}
\begin{Geometrico}% PR: Erste Zeile bitte hängend (more geometrico)
La\textso{ Resistence absolue,}\protect\index{Sachverzeichnis}{r\'{e}sistance absolue}
celle qui est tousjours la m\^{e}me quelque vistesse\protect\index{Sachverzeichnis}{vitesse} que le mobile puisse avoir,
et qui deminue tousjours
\edtext{\'{e}galement}{\lemma{}\Bfootnote{\'{e}galement \textit{erg.} \textit{L}}}
cette vitesse, d'un certain degrez
\edtext{determin\'{e}. Par}{\lemma{}\Bfootnote{determin\'{e}  \textbar\ et \'{e}gal \textit{gestr.}\ \textbar\ . Par \textit{L}}}
\edtext{exemple si}{\lemma{exemple}\Bfootnote{\textit{(1)}\ soit \textit{(2)}\ si \textit{L}}}
la vistesse du mobile, $\displaystyle b$ ou $\displaystyle (b)$,
\edtext{et}{\lemma{}\Bfootnote{et \textit{erg.} \textit{L}}}
la resistence, $\displaystyle c.$ la vitesse residue apr\`{e}s la
\edtext{resistence est}{\lemma{resistence}\Bfootnote{\textit{(1)}\ sera \textit{(2)}\ est \textit{L}}}
$\displaystyle b - c$, ou $\displaystyle (b) - c$.
\end{Geometrico}
\count\Bfootins=1200
\count\Cfootins=1200
\count\Afootins=1200
\begin{Geometrico}
% PR: Erste Zeile bitte hängend (more geometrico)
La\textso{ Resistence respective }\protect\index{Sachverzeichnis}{r\'{e}sistance respective}est celle qui est proportionnelle \`{a} la vitesse du
\edtext{mobile. Dans le m\^{e}me exemple}{\lemma{mobile.}\Bfootnote{\textit{(1)}\ Par exemple \textit{(2)}\ Dans [...] exemple, \textit{L}}},
\edtext{si la force qui resiste}{\lemma{si la}\Bfootnote{\textit{(1)}\ resistence \textit{(2)}\ force qui resiste \textit{L}}}
\`{a} la vitesse du \edtext{mobile $\displaystyle b$, est}{\lemma{mobile $\displaystyle b$,}\Bfootnote{\textit{(1)}\ sera \textit{(2)}\ est \textit{L}}}
\edtext{\`{a} la [force] qui}{\lemma{\`{a} la}\Bfootnote{\textbar\ vitesse \textit{ändert Hrsg.} \textbar\ \textit{(1)}\ du \textit{(2)}\ qui \textit{L}}}
resiste au mobile $\displaystyle (b)$ comme $\displaystyle b$ \`{a} $\displaystyle (b)$:
\edtext{et la resistence absolue, ou la force qui resiste en elle m\^{e}me}{\lemma{et la}\Bfootnote{%
\textit{(1)}\ force absolue \textit{(a)}\ du \textit{(b)}\ qui resiste \textit{(2)}\ resistence [...] m\^{e}me \textit{L}}}
estant tousjours, $\displaystyle c.$ la vistesse residue apr\`{e}s la
\edtext{resistence respective est}{\lemma{resistence}\Bfootnote{\textit{(1)}\ au mob \textit{(2)}\ respective   \textit{(a)}\ sera   \textit{(b)}\ est \textit{L}}}
\rule[-4mm]{0mm}{10mm}$\displaystyle b - \frac{b}{a}c$ ou $\displaystyle (b) - \frac{(b)}{a}c$.
Car nous voyons
\edtext{souuent}{\lemma{}\Bfootnote{souuent \textit{erg.} \textit{L}}}
\edtext{que la resistence du milieu est d'autant plus grande, que la vistesse du mobile est plus rapide}{\lemma{que}\Bfootnote{\textit{(1)}\ la vitesse est d'autant plus grande, que  \textit{(2)}\ la  \textit{(a)}\ vistesse rapi  \textit{(b)}\ resistence [...] rapide. \textit{L}}}.
\end{Geometrico}
\pstart
\noindent J'expliqueray \edtext{ailleurs}{\lemma{ailleurs}\Cfootnote{Vermutlich N. 35.}}
l'origine de ces deux resistences, et comment et pourquoy elles se trouuent dans les corps sensibles.
\edtext{Elles sont}{\lemma{Elles}\Bfootnote{\textit{(1)}\ se trouuent \textit{(2)}\ sont \textit{L}}}
compliqu\'{e}es dans le \edtext{frottement et c'est ce}{\lemma{frottement}\Bfootnote{\textit{(1)}\ : elle \textit{(2)}\  et c'est ce \textit{L}}}
qui en a rendu le calcul difficile;
outre \edtext{que l'analyse qui}{\lemma{que}\Bfootnote{\textit{(1)}\ le calcul qui \textit{(2)}\ l'analyse qui \textit{L}}}
est fond\'{e}e l\`{a} dessus nous mene \`{a} de certaines parties de la Geometrie, qui ne sont pas connues de tout le monde, et qui estoient encor ignor\'{e}es du temps m\^{e}me de Galilaei\protect\index{Namensregister}{\textso{Galilei} (Galilaeus, Galileus), Galileo 1564-1642}[,]
sans parler de Mons. des Cartes\protect\index{Namensregister}{\textso{Descartes} (Cartesius, des Cartes), Ren\'{e} 1596-1650},
qui ne temoigne pas \edtext{d'avoir eu de l'habitude avec}{\lemma{d'avoir}\Bfootnote{\textit{(1)}\ raisonn\'{e} sur \textit{(2)}\ travail \textit{(3)}\ eu [...] avec \textit{L}}}
ces parties de Geometrie ou de Mechanique.
Elles ne laissent pas pourtant d'estre importantes, et
\edtext{particulierement le frottement}{\lemma{particulierement}\Bfootnote{\textit{(1)}\ le mouuem \textit{(2)}\ le frottement \textit{L}}}
ou la resistence de \edtext{l'air [a] beaucoup d'influence sur}{\lemma{l'air}\Bfootnote{%
\textbar\ \`{a} \textit{\"{a}ndert Hrsg.} \textbar\ %
\textit{(1)}\ grande %
\textit{(2)}\ beaucoup %
\textit{(a)}\ d'influence sur %
\textit{(b)}\ de part %
\textit{(c)}\ d'influence sur \textit{L}}} le mouuement des corps jettez,
dont les hommes pourront peut estre trouuer un jour la regle,
pour \edtext{[la]}{\lemma{}\Bfootnote{la \textit{erg.} \textit{Hrsg.}}}
donner sans faute dans un point propos\'{e}.
[13~r\textsuperscript{o}]
\pend
\vspace{1.5em}
\pstart
