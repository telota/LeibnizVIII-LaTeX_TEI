% [5~v\textsuperscript{o}]
\count\Bfootins=1000
\pstart
Si deux corps egaux se rencontrent sur une m\^{e}me ligne de mouuement oppos\'{e} et egal, ils demeurent en repos.
\pend
\pstart
Si les corps sont inegaux le plus grand l'emportera et tous deux seront m\^{e}us avec
\edtext{une m\^{e}me vitesse}{\lemma{une}\Bfootnote{\textit{(1)}\ vitesse \textit{(2)}\ m\^{e}me vitesse \textit{L}}},
mais avec quelle vitesse? Avec celle qui seroit,
\edtext{si la difference des corps}{\lemma{si}\Bfootnote{\textit{(1)}\ le plus gran \textit{(2)}\ la difference des corps \textit{L}}}
\edtext{avec le mouuement de l'un}{\lemma{}\Bfootnote{avec [...] l'un \textit{erg.} \textit{L}}}
venoit rencontrer le double du moindre, en
\edtext{repos. Et ainsi}{\lemma{repos.}\Bfootnote{\textit{(1)}\ Car alors \textit{(2)}\ Et ainsi \textit{L}}}
\edtext{comme la somme de tous deux est \`{a} leur difference}{\lemma{comme}\Bfootnote{\textit{(1)}\ le double du moindre est \`{a} leur difference, \textit{(2)}\ la somme [...] difference \textit{L}}}
ainsi le \edtext{mouuement avant la rencontre}{\lemma{mouuement}\Bfootnote{\textit{(1)}\ apr\`{e}s la renco \textit{(2)}\ avant la rencontre \textit{L}}}
sera au mouuement apr\`{e}s la rencontre.
\pend
\pstart 
Si deux \edtext{corps egaux}{\lemma{corps}\Bfootnote{\textit{(1)}\ inegaux \textit{(2)}\ egaux \textit{L}}}
se rencontrent en m\^{e}me ligne droite avec des mouuemens in\'{e}gaux opposez[,] prenons de celuy qui est le plus fort, autant qu'il en faut pour arrester
\edtext{l'autre, le reste du corps et de la vitesse}{\lemma{l'autre,}\Bfootnote{\textit{(1)}\ le reste pris avec la m\^{e}me vitesse \textit{(2)}\ le reste [...] vitesse, \textit{L}}},
soit pos\'{e} pousser le plus foyble, comme suppos\'{e} en repos.
\pend
\pstart
Si les mouuemens ne sont pas opposez, la vitesse du plus
\edtext{tard attrap\'{e}}{\lemma{tard}\Bfootnote{\textit{(1)}\ , et \textit{(2)}\  attrap\'{e} \textit{L}}}
\edtext{par celuy}{\lemma{par}\Bfootnote{\textit{(1)}\ le \textit{(2)}\ celuy \textit{L}}}
qui est plus viste doit estre augment\'{e}e.
\pend
\count\Bfootins=1000
\pstart
Regle generale de la \edtext{nature:\\
La m\^{e}me quantit\'{e} d'effort pour un m\^{e}me mouuement, demeure tousjours}{\lemma{nature:}\Bfootnote{\textit{(1)}\ l'eff \textit{(2)}\ autant \textit{(3)}\ deux corps se choquans, il se fait autant d'effort pour aller en m\^{e}me sens avant le choc qu'apr\`{e}s le  \textit{(a)}\ choq  \textit{(b)}\ choc \textit{(4)}\ La m\^{e}me [...] tousjours. \textit{L}}}.
\pend
\pstart
Dans le \edtext{plein\protect\index{Sachverzeichnis}{plein} les quantit\'{e}s des efforts sont [compos\'{e}es] de celles de la quantit\'{e} de la matiere qui le fait,}{\lemma{plein}\Bfootnote{%
\textit{(1)}\ la quantit\'{e} d'un effort se doit estimer par la quantit\'{e} de la matiere m\^{u}e, %
\textit{(2)}\ les quantit\'{e}s [...] sont  %
\textit{(a)}\ en raisons composez des %
\textit{(b)}\ compos\'{e}e [...] fait, \textit{L ändert Hrsg.}}}
et de la vitesse.
\pend
\pstart
Dans le plein soit qu'un corps rencontre un
\edtext{autre avec toutes ses parties tout \`{a} la fois}{\lemma{autre}\Bfootnote{\textit{(1)}\ tout \`{a} la fois, \textit{(2)}\ avec [...] fois; \textit{L}}};
ou qu'une partie survienne apr\`{e}s que le choc a est\'{e} fait par les autres; il en doit arriver le m\^{e}me effect. Ce qui n'arrivera pas dans le vuide\protect\index{Sachverzeichnis}{vuide}.
\pend
\pstart
Dans la verit\'{e} l'effort ne se fait pas dans les corps qui se remuent, mais dans les corps qui les poussent ou menent. Comme dans l'eau qui porte une poutre, ce n'est pas la poutre, mais c'est l'eau qui fait l'effort, de m\^{e}me dans le monde en le supposant
\edtext{plein, toute}{\lemma{plein,}\Bfootnote{\textit{(1)}\ l'effort ne se f \textit{(2)}\ toute \textit{L}}}
la matiere qui fait l'effort, est tout ce qui
\edtext{se remueroit aussi en temps, et qui sans cela ne se remueroit pas ainsi.}{\lemma{se}\Bfootnote{\textit{(1)}\ remue en suite et en consequence du mouuement d'un corps donn\'{e}. \textit{(2)}\ remueroit [...] ainsi. \textit{L}}}
De sorte que dans le plein, la matiere
\edtext{qui agit}{\lemma{qui}\Bfootnote{\textit{(1)}\ fait \textit{(2)}\ agit \textit{L}}}
correspondemment au corps que nous voyons agir
\edtext{est repandue par tout le monde. Quand je jette une pierre c'est par quelques ressorts qui se d\'{e}bandent}{\lemma{est}\Bfootnote{\textit{(1)}\ infinie.  \textit{(a)}\ Im  \textit{(b)}\ Quand \textit{(2)}\ repandue [...] monde.  \textit{(a)}\ Quand je remue le bras  \textit{(aa)}\ c'est par la decha \textit{(bb)}\ ou \textit{(b)}\ Quand [...] d\'{e}bandent, \textit{L}}},
soit dans mon bras soit dans un arc.
\edtext{Or ces}{\lemma{Or}\Bfootnote{\textit{(1)}\ ses \textit{(2)}\ ces \textit{L}}}
ressorts sont poussez par le mouuement general; et ce mouuement
\edtext{general de nostre atmosph\`{e}re}{\lemma{general}\Bfootnote{\textit{(1)}\ de nostre tourbi \textit{(2)}\ de nostre atmosph\`{e}re \textit{L}}}
a communication apparemment avec ceux de tous les autres
\edtext{corps. Mais}{\lemma{corps.}\Bfootnote{ \textit{ (1) }\ D'o\`{u} vient, qu'un corps \textit{ (2) }\ Mais \textit{ L}}}
[quoique]\edtext{}{\Bfootnote{quoque\textit{\ L \"{a}ndert Hrsg.}}}
cette matiere soit indefinie, elle s'estime neantmoins par
\edtext{la solidit\'{e}}{\lemma{la}\Bfootnote{\textit{(1)}\ quantit\'{e} \textit{(2)}\  solidit\'{e} \textit{L}}}
du corps pouss\'{e}. Par ce qu'un corps d'autant plus qu'il est solide, d'autant plus
[a-t-il]\edtext{}{\Bfootnote{a-il\textit{\ L \"{a}ndert Hrsg.}}}
de matiere, qui se remue
\edtext{effectivement avec luy,}{\lemma{effectivement}\Bfootnote{\textit{(1)}\ , par exemple de petits \textit{(2)}\  avec luy, \textit{L}}}
car la matiere liquide qui y passe
\edtext{comme le vent entre les arbres d'un bois, ou branches d'un [6~r\textsuperscript{o}] arbre n'est pas}{\lemma{comme}\Bfootnote{\textit{(1)}\ dans une eponge ne fait \textit{(2)}\ l'air \textit{(3)}\ le vent [...] n'est pas \textit{L}}}
ce qui est pouss\'{e} quand on pousse le corps, ne luy estant pas continu.
\count\Bfootins=1000%\pend