\begin{ledgroupsized}[r]{120mm}
\footnotesize 
\pstart 
\noindent\textbf{\"{U}berlieferung:}
\pend
\end{ledgroupsized}

\begin{ledgroupsized}[r]{114mm}
\footnotesize 
\pstart \parindent -6mm
\makebox[6mm][l]{\textit{L}}Konzept: LH XXXVII 5 Bl. 56. 1 Bl. 8\textsuperscript{o}. \unitfrac{4}{5} S. auf Bl. 56~v\textsuperscript{o} mit vorliegendem St\"{u}ck. Am unteren Rand quer zur Schreibrichtung eine Beispielrechnung, die zu N. ??2 auf der Vorderseite geh\"{o}rt. Papier durch Erhaltungsma{\ss}nahmen stabilisiert. \pend
\end{ledgroupsized}
%\normalsize
\vspace*{5mm}
\begin{ledgroup}
\footnotesize
\pstart
\noindent\footnotesize{\textbf{Datierungsgr\"{u}nde}: Siehe oben die Einleitung zu N. ??Intro??.}
\pend
\end{ledgroup}

\vspace*{8mm}
\pstart 
\normalsize
\noindent[56~v\textsuperscript{o}] \selectlanguage{french}Si vous demandez la longueur d'un pendule\protect\index{Sachverzeichnis}{ pendule}, qui fasse un certain nombre de battements  dans un certain temps \edtext{par exemple dans un quart d'heure}{\lemma{}\Bfootnote{par [...] d'heure \textit{ erg.} \textit{ L }\ }}; vous la pourrez trouver ainsi:\pend \pstart  Prenez une \edtext{pendule,}{\lemma{une}\Bfootnote{ \textit{ (1) }\ autre pendule, dont \textit{ (2) }\ pendule, \textit{ L }\ }} \`{a} discretion, mesurez sa longueur; et contez combien de battements elle fait dans le m\^{e}me temps susdit, par exemple dans un quart  d'heure.\pend \pstart  A present pour s'expliquer plus aisement, appellons le nombre des battements de  la pendule, prise \`{a} discretion, \textit{(A)} et le nombre des battements  demand\'{e}, de la pendule dont nous cherchons la longueur, \textit{(B)}  et la longueur de la pendule prise \`{a} discretion, \textit{(C)}  et enfin la longueur de la pendule demand\'{e}e, \textit{(D)}.  Cela estant pos\'{e}, l'operation sera telle.\pend
\vspace*{1mm} \pstart
\lbrack \textit{Nachfolgend klein gedruckter Text gestrichen:}\rbrack \pend \pstart 
\footnotesize
Des deux nombres, $(A)$ et $(B)$ divisez le plus grand par le moindre; et multipliez le quotient par luy m\^{e}me, ou (ce qui est  la m\^{e}me chose) prenez le quarr\'{e} du dit quotient: appellons le dit quarr\'{e}, $(E)$.\pend \pstart
\footnotesize
Enfin faites l'operation suivante de la regle des trois;\\
\edtext{}{\lemma{}\Bfootnote{ \textit{ (1) }\ Comme le nombre quarr\'{e} $F$,\textit{ (2) }\ Si le [...] $(E)$ \textit{ L }\ }}
Si le nombre quarr\'{e} $(E)$, donne l'Unit\'{e}; combien\pend
\vspace*{1mm}\pstart
\footnotesize
$\displaystyle \frac{A}{B}$\hspace*{2mm}$\displaystyle\frac{A^{2}}{\uuline{{B}^{2}}}$\hspace*{2mm}$\displaystyle\frac{C}{D}$\pend \pstart 
\hspace*{6mm}$r$\pend \vspace{1mm} \pstart 
 \normalsize
Multipliez le nombre $A$ par luy m\^{e}me, ou (: ce qui est la m\^{e}me chose :) prenez son quarr\'{e}; de m\^{e}me multipliez le nombre $B$ par luy m\^{e}me, ou prenez son quarr\'{e};  et enfin faites une telle operation de la regle des trois:\pend\pstart
Si le \edtext{quarr\'{e} du nombre}{\lemma{le}\Bfootnote{ \textit{ (1) }\ nombre \textit{ (2) }\ quarr\'{e} du nombre \textit{ L }\ }} \edtext{$A$}{\lemma{}\Bfootnote{$A$\textit{ erg.} \textit{ L}}} des battements\edtext{}{\lemma{}\Bfootnote{battements \textbar\ $A$ \textit{ gestr.}\ \textbar\ de \textit{ L }\ }} de la pendule prise \`{a} discretion, donne la longueur $C$, de sa pendule.\pend\pstart
Combien donnera le quarr\'{e} du Nombre \edtext{donn\'{e}}{\lemma{}\Bfootnote{donn\'{e}\textit{ erg.} \textit{ L }\ }} $B$ des battements de la pendule demand\'{e}e,  pour la longueur $D$, de la dite pendule.\pend \pstart 
Et le produit de cette operation, vous donnera la dite longueur $D$, que vous aviez demand\'{e}e.\selectlanguage{latin}\pend

