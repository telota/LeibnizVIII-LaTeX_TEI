\vspace*{8mm}
\pstart 
\normalsize
\noindent In den folgenden zwei Texten diskutiert Leibniz das Problem, wie aus der Schwingungszahl zweier oder mehrerer Pendel auf deren L\"{a}nge geschlossen werden kann. Die St\"{u}cke N. ??1 und N. ??2 geben daf\"{u}r Regeln an, die f\"{u}r unter\-schiedliche Ausgangsbedingungen gelten. Ein vergleichba\-res Problem behandelt N. ?? (LH XXXV 12, 2 Bl. 62 r\textsuperscript{o}). Dass darin mit denselben Rechenbeispielen operiert wird, spricht f\"{u}r eine gemeinsame Entstehungszeit. Dieser Befund kann sich zudem auf \"{u}bereinstimmende Wasserzeichen  st\"{u}tzen. Das in \cite{00115}\title{LSB} VII, 5A N. 9 erschienene St\"{u}ck auf der R\"{u}ckseite des Blattes (LH XXXV 12, 2 Bl. 62 v\textsuperscript{o}) von N. ?? l\"{a}sst annehmen, dass N. ??1 und ??2 fr\"{u}hestens im Oktober 1674 entstanden sind.\pend
 

