      
               
                \begin{ledgroupsized}[r]{120mm}
                \footnotesize 
                \pstart                
                \noindent\textbf{\"{U}berlieferung:}   
                \pend
                \end{ledgroupsized}
            
              
                            \begin{ledgroupsized}[r]{114mm}
                            \footnotesize 
                            \pstart \parindent -6mm
                            \makebox[6mm][l]{\textit{L}}Notiz: LH XXXV 13, 2c Bl. 144. Papierstreifen 23 x 4 cm, in Richtung des Zeilenverlaufs leicht konisch zugeschnitten. 4 \unitfrac{1}{2} Z. R\"{u}ckseite leer. \pend
                            \end{ledgroupsized}
                %\normalsize
                \vspace*{5mm}
                \begin{ledgroup}
                \footnotesize 
                \pstart
            \noindent\footnotesize{\textbf{Datierungsgr\"{u}nde}: Die Datierung erfolgt aufgrund des Wasserzeichens, das in der Zeit zwischen Fr\"{u}hjahr und Sp\"{a}therbst 1672 nachgewiesen ist.}
                \pend
                \end{ledgroup}
            
                \vspace*{8mm}
                \pstart 
                \normalsize
            [144~r\textsuperscript{o}] Scientia de progressionibus potest perficere Geometriam: Nam si ratio invenietur, datis duobus altero decrescente altero crescente, diversa proportione, invenire punctum aequalitatis, habebimus \edtext{circumferentiae aequalem  rectam}{\lemma{habebimus}\Bfootnote{ \textit{ (1) }\ Circulum \textit{ (2) }\ circumferentiae aequalem rectam \textit{ L }\ }}. Finge Tibi corpora duo se accedere in linea recta, diversis celeritatibus, in certo quodque  proportionum in genere, invenire punctum concursu seu \textso{quasi }\textso{centrum gravitatis}\protect\index{Sachverzeichnis}{centrum gravitatis}. Potest enim tale punctum concursus\protect\index{Sachverzeichnis}{concursus} jure appellari centrum gravitatis motuum.\pend 
 


 

