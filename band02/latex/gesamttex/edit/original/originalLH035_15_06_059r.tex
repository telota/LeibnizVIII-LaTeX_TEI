      
               
                \begin{ledgroupsized}[r]{120mm}
                \footnotesize 
                \pstart                
                \noindent\textbf{\"{U}berlieferung:}   
                \pend
                \end{ledgroupsized}
            
              
                            \begin{ledgroupsized}[r]{114mm}
                            \footnotesize 
                            \pstart \parindent -6mm
                            \makebox[6mm][l]{\textit{L}}Konzept: LH XXXV 15, 6 Bl. 59, 10 x 15 cm. 2 S. Linke und obere Seite ungleichm\"{a}{\ss}ig beschnitten. Auf Bl. 59 v\textsuperscript{o} die letzten beiden Zeilen parallel zum linken Seitenrand.\\Cc 1, Nr. 194 C \pend
                            \end{ledgroupsized}
                %\normalsize
                \vspace*{5mm}
                \begin{ledgroup}
                \footnotesize 
                \pstart
            \noindent\footnotesize{\textbf{Datierungsgr\"{u}nde}: Leibniz beschreibt verschiedene Mechanismen zur Erzeugung einer gleichf\"{o}rmigen Bewegung und beruft sich dabei vor allem auf Francesco Lana, dessen \title[115]{Prodromo} (1670) er vermutlich in der zweiten H\"{a}lfte 1671 zur Optik exzerpiert hat (\title[115]{LSB} VIII,1 N. 16). In dem vorliegenden St\"{u}ck erw\"{a}hnt Leibniz sein Perpetuum mobile, dessen Konstruktion er im Juni 1671 beschreibt (\title[115]{LSB} VIII,1 N. 59). Auf dasselbe Jahr geht die Arbeit an seiner hier ebenfalls erw\"{a}hnten Rechenmaschine für die vier Grundrechenarten (\title[115]{Panarithmicum}) zurück. Über das Heliotropium Kirchers hat sich Leibniz bei dem Jesuiten selbst im Mai und Juni 1670  informiert (\title[115]{LSB} II,1.2 N.20a u. N. 23). Das Stück mag daher zeitnah zu den Lana-Exzerpten oder nicht viel später entstanden sein.}
                \pend
                \end{ledgroup}
            
                \vspace*{8mm}
                \pstart 
                \normalsize
            [59 r\textsuperscript{o}] \edtext{Motum uniformem praestitit Galil.\protect\index{Namensregister}{\textso{Galilei}, Galileo (1564-1642)}\edtext{}{\lemma{Galil.}\Cfootnote{\textsc{G. Galilei, }\cite{00050}\textit{Discorsi}, Leiden 1638, S. 97f. (\textit{GO} VIII, S. 140f.)}}, perp. L.\edtext{}{\lemma{L.}\Cfootnote{\textsc{F. Lana, }\cite{00069}\textit{Prodromo}, Brescia 1670, S. 80-85.}}}{\lemma{}\Bfootnote{Motum uniformem praestitit Galil.\protect\index{Namensregister}{\textso{Galilei}, Galileo (1564-1642)}, perp. L. \textit{ erg.} \textit{ L }\ }} 
\\ \textso{Motum uniformem}\protect\index{Sachverzeichnis}{motus uniformis} pendulo\protect\index{Sachverzeichnis}{pendulum} praestari primus orbi aperuit Galilaeus\protect\index{Namensregister}{\textso  {Galilei}, Galileo (1564-1642)}, eundem praestari restitutione elateris\protect\index{Sachverzeichnis}{elater} alii addidere, et huic principio innititur Horologiorum\protect\index{Sachverzeichnis}{horologium} portatilium Elasticorum constructio, quae non valde antiqua sunt.  Posset ergo vibratione chordarum, restitutione arcuum\protect\index{Sachverzeichnis}{arcus}, pulsatione campanarum\protect\index{Sachverzeichnis}{campana} vel tympanorum\protect\index{Sachverzeichnis}{tympanum} haberi motus uniformis\protect\index{Sachverzeichnis}{motus uniformis}. Sed chordarum tympanorumque\protect\index{Sachverzeichnis}{tympanum} vibratio invisibilis in parvo visibilis futura si quid iis longi radii applicetur. Sed cum constet chordas in tensione sua mirifice variare (quanquam appenso pondere mederi liceat, quod remittentem attrahat magis) sit vero etiam applicatio futura difficillima, nunc quidem de hac re non dicemus. Etsi [applicatione]\edtext{}{\Bfootnote{applicationem\textit{\ L \"{a}ndert Hrsg. } }} commoda reperta res mirae perfectionis futura sit machinula, cum non dubitem unam chordae vibrationem centesima millesima horae parte  et fortasse minore absolvi. Ergo si vibratio quaelibet moveret rotam subtilissimam, ita constructam ut reciprocatione tamen eodem  semper iret, et redeundo etiam prorsum ageret, uti Lana\protect\index{Namensregister}{\textso  {Lana Terzi}, Francesco (1631-1687)} habet. \edtext{}{\lemma{habet.}\Cfootnote{\textsc{F. Lana, }\cite{00069}a.a.O., S. 80-85.}} Mirabilis subtilitatis esset haec temporis divisio, pone semper finem vibrantis unius facere \edtext{pulsationem alterius}{\lemma{facere}\Bfootnote{ \textit{ (1) } vibrationem alterius \textit{ (2) } pulsationem alterius \textit{ L }}},  et rem in circulum redire, applicata nostra m. p.\edtext{}{\lemma{m. p.}\Cfootnote{motus perpetui}} machina.  Haberet ea res haec commoda, quod \edtext{chorda valide}{\lemma{quod}\Bfootnote{ \textit{ (1) } valde \textit{ (2) }chorda valide \textit{ L }\ }} an leniter  pulsaretur, nihil interesset, ut constat ex sono\protect\index{Sachverzeichnis}{sonus} chordarum  in musicis. Nam etsi vehementia mutetur, sonus\protect\index{Sachverzeichnis}{sonus} tamen id est vibrationum isochronismus idem est. Applicatione tantum vel ponderis vel alterius \edtext{moventis ita  facta}{\lemma{alterius}\Bfootnote{ \textit{ (1) }\ rei ita fac \textit{ (2) }\ moventis ita  facta \textit{ L }\ }}, \edtext{daß die Seite sich selbst stimme.}{\lemma{facta,}\Bfootnote{ \textit{ (1) }\ ut chorda \textit{ (2) }\ daß [ ... ] stimme. \textit{ L }\ }} Ita  tamen ut a pulsante non possit retrahi illud, quia scil.  in alterum forte latus aliqua cum applicatione trahit,  aut lente per circumvolutiones trochleares\protect\index{Sachverzeichnis}{trochlea}. Ut autem [commodissime]\edtext{}{\Bfootnote{commodisse\textit{\ L \"{a}ndert Hrsg. } }} numerari possint minuta, adhibenda Logistica decimalis,  atque ea applicatio \edtext{quam meo}{\lemma{quam}\Bfootnote{ \textit{ (1) }\ ad \textit{ (2) }\ meo \textit{ L }\ }} \textso{Panarithmico} destinavi rotarum ita sibi applicatarum, ut una semel circumacta alterius  decimam tantum partem circumagat, adde Lanam\protect\index{Namensregister}{\textso  {Lana Terzi}, Francesco (1631-1687)}\edtext{}{\lemma{Lanam}\Cfootnote{\textsc{F. Lana, }\cite{00069}a.a.O., Tafel XVI.}}. Haec  de chordis, solida: campanae\protect\index{Sachverzeichnis}{campana}, tympana\protect\index{Sachverzeichnis}{tympanum} non sunt commodi usus,  nisi construatur machina tantae subtilitatis, ut solo sono\protect\index{Sachverzeichnis}{sonus} moveatur  uti chorda tensa aliter similiter tensa sonante resonat. Ergo  si campana\protect\index{Sachverzeichnis}{campana} sonans moveat chordam solo sono\protect\index{Sachverzeichnis}{sonus}, chorda mota circumagat aliquid quod vicissim eundem sonum\protect\index{Sachverzeichnis}{sonus} rursus imprimat campanae\protect\index{Sachverzeichnis}{campana}habebitur circulatio et uniformitas summa [59 v\textsuperscript{o}] \edtext{Arcuum}{\lemma{summa}\Bfootnote{ \textit{ (1) }\ Arcubus ita forsan, uti licebit, atque his \textit{ (2) }\ Arcuum \textit{ L }\ }} [restitutione]\edtext{}{\Bfootnote{restitutio\textit{\ L \"{a}ndert Hrsg. } }} quia sensibilissima, celerrima tamen facilius nos uti posse censeo. Sit circulus aliquis vel annulus  meris arcubus circumdatus qui contrahente se seu minuente  circulo tendantur red. aperiente restituantur. \edtext{Possint}{\lemma{restituantur.}\Bfootnote{ \textit{ (1) }\ Applicat \textit{ (2) }\ Possint \textit{ L }\ }} tamen et sine circuli dilatatione aperiri a motis  tantum quibusdam impedimentulis, ita ut arcus unus restitutus, ubi primum ad statum naturalem\protect\index{Sachverzeichnis}{status naturalis} rediit tangat  alium eumque similiter liberet. Ita in toto dato spatio  liberabuntur: et quidem uniformiter, quia restitutiones ejusdem arcus, etiam inaequaliter tensi, sunt Isochronae. Nec erit hic quae in Elatere\protect\index{Sachverzeichnis}{elater} communi se restituente irregularitas, quia eum multa morantur, hic  cum restitutio sit pene momentanea, non potest esse  sensibilis irregularitas: restitutione chordae ultimae aperiatur circulus. Interea apertura sua chordae istae contraxere seu retendere circulum alium vicinum vel si  lubet ejusdem partem oppositam. Atque ita motus continuabitur uniformiter. Non dubito decem millia  restitutionum ejusmodi una hora fieri posse. Habebitur  ergo horae pars \edtext{decies}{\lemma{pars}\Bfootnote{ \textit{ (1) }\ 10 \textit{ (2) }\ decies \textit{ L }\ }} millesima, forte non minore  quam in pendulis\protect\index{Sachverzeichnis}{pendulum} regularitate, nullo autem jactationis maritimae metu. Ita ut applicari quoque horologiis portatilibus\protect\index{Sachverzeichnis}{horologium portabilium} possit. Porro ope Logisticae decimalis rotis pluribus applicatae, facile poterit etiam millesies millesima pars anni, si scil. ponamus integro anno currere posse horologium\protect\index{Sachverzeichnis}{horologium}, vel septimanae  saltem (quanquam applicatione possit esse perpetuum)  nullo negotio exhiberi. Una superest ratio procurandi motus uniformis\protect\index{Sachverzeichnis}{motus uniformis}. Nimirum per magnetem\protect\index{Sachverzeichnis}{magnes}, constat acum  libratam diu vacillare antequam requiescat. Quaeritur ergo an vacillationes esse possint isochronae. Item an ipse motus tendendi ad polum isochronus quacunque posita distantia, ita scil.  ut si remota sit longius, moveatur celerius. Sed hic subest  ea difficultas, quod perpendicularis situs requiritur. Ergo  videndum an attractione magnetica\protect\index{Sachverzeichnis}{attractio magnetica} quicquam agi possit, ita  scil. ut acus attracta accessu aperiat aliquid quo repercutiatur, idque nova attractione\protect\index{Sachverzeichnis}{attractio} rursus claudat, novo tactu rursus aperiat\edtext{. Rursus claudet, si in attrahendo}{\lemma{aperiat}\Bfootnote{ \textit{ (1) }\ , attrahet \textit{ (2) }\ . Rursus claudet, si in attrahendo \textit{ L }\ }}  applicetur spirae\protect\index{Sachverzeichnis}{spira} alicui vel vecti, sed non erit celeritas\protect\index{Sachverzeichnis}{celeritas} tanta  puto tamen absolvi posse intra minutum secundum. Machina pure magnetica sine omni elatere. A pondere si ab amico attrahatur, semel allapsa inimicum inveniat, et  ab eo repellatur. Heliotropium Kircheri\protect\index{Namensregister}{\textso  {Kircher}, Athanasius (1602-1680)}\edtext{}{\lemma{Heliotropium}\Cfootnote{\textsc{A. Kircher, }\cite{00067}\textit{Magnes}, Rom 1654, S. 508.}} habet etiam motum uniformem\protect\index{Sachverzeichnis}{motus uniformis}.\pend
 


 


 

 

 


 


 


 


 

