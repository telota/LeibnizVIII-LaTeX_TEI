      
               
                \begin{ledgroupsized}[r]{120mm}
                \footnotesize 
                \pstart                
                \noindent\textbf{\"{U}berlieferung:}   
                \pend
                \end{ledgroupsized}
            
              
                            \begin{ledgroupsized}[r]{114mm}
                            \footnotesize 
                            \pstart \parindent -6mm
                            \makebox[6mm][l]{\textit{L}}Konzept: LH XXXVIII Bl. 170-171. 1 Bog. 18 x 9 cm. \nicefrac{1}{3} S. auf Bl. 171~r\textsuperscript{o} unser St\"{u}ck; Bl. 171~v\textsuperscript{o} leer. Auf Bl. 170 befindet sich N.???. Die unteren zwei Drittel von Bl. 171 wurden abgetrennt. Auf Vorderseite von Bl. 171 oben links Teile eines Wasserzeichens. \\Kein Eintrag in KK 1 oder Cc 2 \pend
                            \end{ledgroupsized}
                %\normalsize
                \vspace*{5mm}
                \begin{ledgroup}
                \footnotesize 
                \pstart
            \noindent\footnotesize{\textbf{Datierungsgr\"{u}nde}: Der Text des vorliegenden St\"{u}cks befindet sich mit N.??? auf demselben Bogen. Nachdem Leibniz erkennen musste, dass Pendeluhren nicht so gleichm\"{a}\ss{}ig gehen, sieht er sich offenbar nach neuen Stabilisierungsm\"{o}glichkeiten f\"{u}r deren Ganggenauigkeit um. Diese entdeckt er nun in der Unruhe. Der inhaltliche Zusammenhang sowie der gemeinsame Texttr\"{a}ger lassen auf eine zeitnahe Entstehung zusammen mit  N.??? schlie{\ss}en.}
                \pend
                \end{ledgroup}
            
                \vspace*{8mm}
                \pstart 
                \normalsize
            \noindent[171~r\textsuperscript{o}] In Horologio\protect\index{Sachverzeichnis}{horologium} communi, pars quae Germanis vocatur inquies\protect\index{Sachverzeichnis}{inquies} impetum rotae moratur, dum simul contrariis \edtext{dentibus illiditur; sed quia impetus concepti pars}{\lemma{contrariis}\Bfootnote{ \textit{ (1) }\ parte \textit{ (2) }\ dentibus [ ... ] pars \textit{ L }\ }} magna perditur ea ratione; cogitavi an non satius sit, rotam vi sua elateriolum\protect\index{Sachverzeichnis}{elateriolum} tendere, atque ita ubi ipsum ad certum perduxit terminum, fracta vi sua impeditum teneri, \edtext{maxima vis}{\lemma{teneri,}\Bfootnote{ \textit{ (1) }\ magna vis \textit{ (2) }\ maxima vis \textit{ L }\ }} parte hoc modo conservata. Et posset hoc Elastrum esse additum ipsi illi inquieti, cujus dum extrema in diversa pelluntur, posset tendi Elastrum in medio, ad certum usque terminum. Sed pendulum staticum vel Elasticum mox reversum liberabit hoc Elaterium, et ab eo ictum accipiet. Quo peracto rota quoque horologii Elateriolum inquietis denuo tendet.\pend \pstart \edtext{Pendulum}{\lemma{}\Bfootnote{ \textit{ (1) }\ In \textit{ (2) }\ Pendulum \textit{ L }\ }} staticum affigi solet libramento, at Elasticum tenet \edtext{arborem}{\lemma{tenet}\Bfootnote{ \textit{ (1) }\ arb \textit{ (2) }\ in medio \textit{ (3) }\ arborem \textit{ L }\ }} rotae serratae. \pend
 


 

