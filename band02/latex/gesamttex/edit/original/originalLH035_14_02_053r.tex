\begin{ledgroupsized}[r]{120mm}
\footnotesize 
\pstart  
\noindent\textbf{\"{U}berlieferung:}  
\pend
\end{ledgroupsized}
\begin{ledgroupsized}[r]{114mm}
\footnotesize 
\pstart \parindent -6mm
\makebox[6mm][l]{\textit{L}}Konzept: LH XXXV 14, 2 Bl. 53. 1 Bl. 2\textsuperscript{o}, \nicefrac{4}{5} S. R\"{u}ckseite leer.\\Cc 2, Nr. 509 \pend
\end{ledgroupsized}

%\normalsize
\vspace*{5mm}
\begin{ledgroup}
\footnotesize \pstart
\noindent\footnotesize{\textbf{Datierungsgr\"{u}nde}: Leibniz erw\"{a}hnt neben der Leidener Ausgabe der \textit{Disquisitio physica} \textsc{Travagini}s 1669 auch die in Venedig 1673 publizierte Edition. Damit gibt es einen Terminus post quem f\"{u}r die Entstehungszeit des Textes. Der Zeitraum l\"{a}sst sich weiter durch das Wasserzeichen eingrenzen, das sich in dem Konzeptpapier eines Briefes an Oldenburg findet. Die Bearbeiter dieses Briefes, der als N. 117 in \textit{LSB} II, 1 gedruckt wurde, betonen die Schwierigkeit der Datierung und geben als m\"{o}gliches Entstehungsdatum die Jahre 1673 bis 1676 an. Das entspricht der durch die erw\"{a}hnten Ausgaben nahe gelegten Entstehungszeit der Exzerpte.}
\pend
\end{ledgroup}

\vspace*{8mm}
\pstart 
\normalsize
[53 r\textsuperscript{o}] \textit{Francesci Travagini}\protect\index{Namensregister}{\textso {Travagini}, Francesco (? - nach 1667)}\textit{ super observationibus a se factis tempore ultimorum terrae motuum ac potissimum }[\textit{Ragusiani}]\edtext{}{\Bfootnote{\textit{Ragusini}\textit{\ L \"{a}ndert Hrsg. } }}\textit{, physica disquisitio, seu gyri terrae diurni indicium.}\edtext{}{\lemma{\textit{indicium.}}\Cfootnote{\textsc{F. Travagini, }\cite{00112}\textit{Disquisitio physica}, Leiden 1669, S. 1.}} Juxta exemplar Venetiis\protect\index{Ortsregister}{Venedig} impressum. 1673. 4\textsuperscript{o}. Inscripsit Haberto Mommor\protect\index{Namensregister}{\textso {Mommor}, Henri Louis Habert de (??)} tempore novissimo terrae motus omnes erant persuasi aliqua vertigine et credebant motum esse in suo capite, qui erat in terra ipsa. \pend \pstart  Anno 1667. 6 April. hora 13. circiter contigit terrae motus qui Ragusium\protect\index{Ortsregister}{Dubrovnik} diruit. \textit{Versabar tunc }\textit{Venetiis}\protect\index{Ortsregister}{Venedig}\textit{ ac forte illo momento quiete agebam in Musaeo meo, quo factum, ut nec primus ejus impetus me latuerit.}\edtext{}{\lemma{\textit{latuerit.}}\Cfootnote{\textsc{F. Travagini, }\cite{00112}a.a.O., S. 1.}} Quo facto statim aperui fenestras, ne quae circumstantiae \textit{fugerent diligentiam meam, ac praecipue earum,}\edtext{}{\lemma{\textit{earum,}}\Cfootnote{\textsc{F. Travagini, }\cite{00112}a.a.O., S. 1.}} quae mihi magnum veritatis indicium fecerant. Fuit is ille qui contigit in Aemilia\protect\index{Ortsregister}{Romagnese} vulgo Romagna\protect\index{Ortsregister}{Romagnese} Anno 1661. 22 April. post meridiem. \textit{Observavi primo terram tunc moveri ac ferri tranquillissime multiplicatis vibrationibus ab occidente ad orientem ac reciproce ab oriente ad occidentem, ita ut vix bene me regerem}\edtext{}{\lemma{\textit{regerem}}\Cfootnote{\textsc{F. Travagini, }\cite{00112}a.a.O., S. 2.}} sed quasi titubarem prorsus ut qui in cymba stans \textit{subito improviso aliquo motu dejicitur a pacifico tenore, quo antea immotae similis dilabebatur. Caeterum nullam tunc sensi terrae successionem, quia scilicet ipsa recedens a suo centro vel me, vel domos circumstantes in altum succuteret, nec }\edtext{\textit{ullus}}{\lemma{\textit{nec}}\Bfootnote{ \textit{ (1) }\ \textit{alius} \textit{ (2) }\ \textit{ullus} \textit{ L }\ }}\textit{ fuit hic }\textit{Venetiis}\protect\index{Ortsregister}{Venedig}\edtext{}{\lemma{\textit{Venetiis}}\Cfootnote{\textsc{F. Travagini, }\cite{00112}a.a.O., S. 2.}} qui talem motum notaverit. \textit{Secundo observavi, atque ab aliis omnibus a quibus inquisivi, video confirmatum, canales omnes, quos hic }\textit{Venetiis}\protect\index{Ortsregister}{Venedig}\textit{ plurimos habemus, ab oriente ad occidentem recta linea deductos, tunc undas suas, (quae tunc maximae factae sunt, cum antea prorsus nullae essent), secundum eandem lineam refluas ac reciprocas habuisse. Atque enim contra in iis qui septentrione ad meridiem deducuntur eas undas fuisse laterales ab una ripa ad aliam. Hoc est et ipsas quoque ab oriente ad occidentem atque ab occidente ad orientem invicem reciprocantes. Tertio quod campanilia atque aliae ejusmodi fabricae erectiores, eodem modo hinc inde lateraliter vibrarentur, ac tantummodo orientalibus atque occidentalibus aedibus quas habebant vicinas, suae molis ruinam interminarentur. Quarto quod omnia quae ex domorum laquearibus aliquo fune ligata pendebant, cujusmodi sunt omnes lampades, Ecclesiarum, tunc etiam ab occidente ad orientem vibrarentur. Porro haec }\edtext{\textit{eadem quatuor}}{\lemma{\textit{haec}}\Bfootnote{ \textit{ (1) }\ \textit{omnia qu} \textit{ (2) }\ \textit{eadem quatuor} \textit{ L }\ }}\textit{ in superiori jam dicto terrae motu olim observaveram, prout lego in observationum mearum diario.}\edtext{}{\lemma{\textit{diario.}}\Cfootnote{\textsc{F. Travagini, }\cite{00112}a.a.O., S. 2f.}} Interrogavi alios quotquot novi talium non indiligentes observatores, qui \edtext{unanimi confessione me confirmavere, ne uno quidem contrarium asserente}{\lemma{qui}\Bfootnote{ \textit{ (1) }\ inde \textit{ (2) }\ unanimi [ ... ] asserente \textit{ L }\ }}; sed et seniores non pauci qui aliis interfuerant terrae motibus, interrogati \textso{ex arte} motum semper talem fuisse asseruere. Ragusii motus fuit mixtus ex laterali et succustatione in altum, et ex laterali motu seu vibratione Ragusium\protect\index{Ortsregister}{Dubrovnik} inter et Venetias\protect\index{Ortsregister}{Venedig} quasi medio intervallo, mixtus etiam motus, sed lateralis vibratio videbatur major visa tamen semper reciproca ab oriente in occidentem. Ratio succussationis non est hujus loci, et habet multas causas possibiles, quas hic inutile discutere. Quod attinet motum lateralem, is \textso{a} motu succussationis produci non potest, quia inde non  potest oriri motus ab oriente ad occidentem, res constans ab inconstante et irregulari. De causa igitur hujus motus ita ratiocinatur. Si quis in cymbam translatus dormiens, \edtext{ secundo }{\lemma{dormiens,}\Bfootnote{ \textit{ (1) }\ pleno \textit{ (2) }\ secundo \textit{ L }\ }} amne placidissime labatur, exporrectus ne somniabit quidem se moveri. At si cymba forte impingat in saxum, duos sentiet motus, unum succussationis, alterum vibrationis \edtext{seu progressivum}{\lemma{vibrationis}\Bfootnote{ \textit{ (1) }\ motum \textit{ (2) }\ seu progressivum \textit{ L }\ }}. \textso{Nimirum cum cymba incidit in impedimentum vel retinaculum, quod brevi licet tempore ejus }\textit{\textso{cursum moretur, vel a placidissimo suo tenore dejiciat certissimum est, }}\edtext{\textit{\textso{quod}}}{\lemma{}\Bfootnote{\textit{\textso{quod}} \textit{ erg.} \textit{ L }\ }}\textit{\textso{ cessante illo impedimento, ubi cymba cursus sui tenori restituetur: quod ipse tunc sentiet, et quod locus ubi est movetur,}}\edtext{}{\lemma{\textit{\textso{movetur,}}}\Cfootnote{\textsc{F. Travagini, }\cite{00112}a.a.O., S. 18.}}\textso{ et in quam partem movetur sed ubi cymba restituetur insensibili placiditati, iterum eam stare arbitrabitur, donec rursus incidat in impedimentum. }\textit{\textso{Tellus quo Tempore succutitur sensibiliter videtur vibrari versus orientem et immediati post retrocedere versus occidentem ad punctum a quo retrocesserat,}}\edtext{}{\lemma{\textit{retrocesserat,}}\Cfootnote{\textsc{F. Travagini, }\cite{00112}a.a.O., S. 20.}}\textso{ idque fit quoties actio iteratur.} \edtext{[+ Videtur explicare sed non explicat unde fiat vibratio seu itio et reditio etiam in cymba. +]}{\lemma{[+ Videtur [...] in cymba. +]}\Bfootnote{Eckige Klammern durch \textit{ L}}} Ait succussationem motum placidum toti communem in partibus succussis retardare. Nimirum si contingat motum cymbae subito accelerari vel retardari, statim pendula malo appensa, et aquae in catinis lateraliter \edtext{vibrabuntur}{\lemma{lateraliter}\Bfootnote{ \textit{ (1) }\ fluent \textit{ (2) }\ vibrabuntur \textit{ L}}}. Et quidem ab occidente ad orientem seu in eam partem in quam est motus. \edtext{[Hoc etiam non explicat in quam primum partem debeat esse lateralis vibratio, et quomodo revibretur. Sed nec causam satis distructe explicat eorum quae contingunt in cymba.[\thinspace]\thinspace]}{\lemma{[Hoc etiam [...] in cymba.]}\Bfootnote{Eckige Klammern durch \textit{ L}, schlie{\ss}ende Klammer \textit{erg. Hrsg.}}} Variatio celeritatis non est \edtext{sola causa sed}{\lemma{est}\Bfootnote{ \textit{ (1) }\ causa, sed imp \textit{ (2) }\  quod res mota \textit{ (3) }\ sola causa sed \textit{ L }\ }} fluidum circumstans, quod turbatur ab hac variatione; videndum et quomodo succussatio retardet; \edtext{non video enim}{\lemma{retardet;}\Bfootnote{ \textit{ (1) }\ an quod \textit{ (2) }\ non video enim \textit{ L }\ }}, quomodo non corpus succussum observatam continuitatem simul procedat cum toto, nec duorum motuum compositio imminuit priorem. Item aliud atque aliud oriretur, prout succussatio fieret magis vel minus perpendicularis ut si rem succussam in eam partem oblique pelleret, in quam jam a motu fertur, contraria omnia deberent evenire itaque, satis manifestum arbitror phaenomenon hoc debere oriri a motu terrae diurno, unde enim oriatur, si non ab illo, sed quomodo ab illo oriatur nondum satis video explicatum.\pend
 

