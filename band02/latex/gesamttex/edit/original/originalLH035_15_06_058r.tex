      
               
                \begin{ledgroupsized}[r]{120mm}
                \footnotesize 
                \pstart                
                \noindent\textbf{\"{U}berlieferung:}   
                \pend
                \end{ledgroupsized}
            
              
                            \begin{ledgroupsized}[r]{114mm}
                            \footnotesize 
                            \pstart \parindent -6mm
                            \makebox[6mm][l]{\textit{L}}Konzept: LH XXXV 15,  6 Bl. 58. 1 Bl. 20 x 20 cm. 1 S. Oberes rechtes Viertel ausgeschnitten. Bl. 58 r\textsuperscript{o} zweispaltig beschrieben, Bl. 58 v\textsuperscript{o} leer. Sprachwechsel zum Deutschen gegen Ende des St\"{u}cks. Der urspr\"{u}ngliche Titel lautete \textit{Efficere Horologia accurata}. Sp\"{a}ter wurde \textit{Chronologia} als Haupttitel dar\"{u}ber gesetzt. \\Cc 1, Nr. 194 B \pend
                            \end{ledgroupsized}
                %\normalsize
                \vspace*{5mm}
                \begin{ledgroup}
                \footnotesize 
                \pstart
            \noindent\footnotesize{\textbf{Datierungsgr\"{u}nde}: In der von Leibniz erw\"{a}hnten Stelle aus Monconys' \title[134]{Journal des voyages} wird \"{u}ber eine Methode zur Regulierung der Pendelbewegung berichtet, die auf Isaac Vossius zur\"{u}ckgeht. Zentrales Element dieser Methode ist ein Siphon, dessen Konstruktion einen kontinuierlichen Ausfluss erm\"{o}glicht. Leibniz hat diese Idee in \title[115]{LSB} VIII, 1 N. 63 aufgegriffen und zu einer Clepsydra fortentwickelt. Das daf\"{u}r entworfene Z\"{a}hlwerk entspricht in seiner Grundkonstruktion dem des vorliegenden Textes. Aufgrund dieser \"{U}bereinstimmung kann eine zeitnahe Entstehung der beiden Texte angenommen werden. Diese Datierung lässt sich weiter dadurch st\"{u}tzen, dass Leibniz in dem vorliegenden St\"{u}ck einen Vergleich mit Vakuumph\"{a}nomenen liefert, mit denen er sich gleichfalls in der zweiten Jahresh\"{a}lfte 1672 beschäftigt hat (\title[115]{LSB} VIII, 1 N. 41}).
                \pend
                \end{ledgroup}
            
                \vspace*{8mm}
               \pstart 
                \normalsize
           \begin{center}[58~r\textsuperscript{o}] \edtext{Chronologia}{\lemma{} \Bfootnote{Chronologia \textit{ erg.} \textit{ L}}} \end{center} \pend 
            \pstart \textso{Efficere Horologia accurata}. Vid. Isaaci Vossii\protect\index{Namensregister}{\textso  {Vossius}, Isaac (1618-1689)} consilium apud \edtext{Monconisium}{\lemma{Monconisium}\Cfootnote{\textsc{B. de Monconys,}\cite{00118}\textit{ Journal des voyages}, 2. Teil, Paris 1666, S. 154. }}.\protect\index{Namensregister}{\textso  {Monconys}, Balthasar de (1611-1665)} Iniri possunt rationes variae, si quolibet ictu  penduli aperiretur superius foramen, unde excideret  sive liquor sive granum aliquod granum grano aequale, aut gutta\edtext{ guttae, quae lapsu suo priorem semper impetum de novo  imprimerent pendulo et praeterea laberentur in vas quod implendo signarent numerum in eo notatum}{\lemma{guttae,}\Bfootnote{ \textit{ (1) }\ quibus notarentur \textit{ (2) }\ quae [ ... ] notatum \textit{ L }\ }} vibrationum. Deberet vitreum seu perspicuum esse.  Posset esse in cochleam\protect\index{Sachverzeichnis}{cochlea} contortum, notandis exactius  gradibus. Aut si rectilineum esset, gradus designati  paralleliter deberent transversis subdividi  si cochleare, eousque contortura opus esset, quousque  adhuc sive gutta, sive granum per obliquitatem  descendere possent; nam in cochlea\protect\index{Sachverzeichnis}{cochlea} nimis obliqui  siccum non descenderet, liquidum descenderet sed  tarde. Quid si ut in experientia vacui\protect\index{Sachverzeichnis}{experientia vacui} qualibet  apertura immitteretur aer, qui sive hydrargyrum\protect\index{Sachverzeichnis}{hydrargyrus}  sive aquam pendentem magis descendere faceret  sed tunc deesset causa perpetuo percutiens pendulum\protect\index{Sachverzeichnis}{pendulum}. Liquidis non facile designari potest  numerus vibrationum, etsi alia divisio in minuta  forte tertia signari possit. Siccis potest. Caeterum  ita instituto instrumento nescio an obesse possit  situs \edtext{non perpendicularis}{\lemma{situs}\Bfootnote{ \textit{ (1) }\ perpendicularis \textit{ (2) }\ non perpendicularis \textit{ L }\ }}. Forte  rectius res geretur, si qualibet vibratione rotulae  alicujus aculeus novus descendat descensuque pendulum\protect\index{Sachverzeichnis}{pendulum}  rursus, aequaliter semper, impellat. In rotae aculeis  designari numerus potest, sit \edtext{v.g.}{\lemma{sit}\Bfootnote{ \textit{ (1) }\ alia rota major p \textit{ (2) }\ v.g. \textit{ L }\ }}  quaelibet vibratio unius minuti \edtext{secundi}{\lemma{minuti}\Bfootnote{ \textit{ (1) }\ tertii \textit{ (2) }\ secundi \textit{ L }\ }}. 
            Erunt  in rota aculei $60 \smallfrown 60$ seu 3600.  
            Sit alia rota major quae una gyratione minoris \edtext{unum sui gradum absolvat}{\lemma{minoris}\Bfootnote{ \textit{ (1) }\ absolvatur \textit{ (2) }\ unum sui gradum absolvat \textit{ L }\ }}, divisa in gradus 60, notandis minutis primis, denique  sit rota pro horis, cujus unus gradus qualibet  secundae rotae gyratione absolvatur, tot quot  sunt horae.\pend 
            \pstart  Damit dem pendulo\protect\index{Sachverzeichnis}{pendulum} keine jactation  schade, auch da \edtext{es}{\lemma{da}\Bfootnote{ \textit{ (1) }\ $\langle$ man $\rangle$ \textit{ (2) }\ es \textit{ L }\ }} par un ressort  ge\-macht werden solte, dessen unvermeidliche irregularitaten durch luftspannung\protect\index{Sachverzeichnis}{Elastizit\"{a}t der Luft}  und sonsten, nicht in das pendulum\protect\index{Sachverzeichnis}{pendulum} transferirt, konte also geschehen, wenn \edtext{ein rad}{\lemma{wenn}\Bfootnote{ \textit{ (1) }\ durch \textit{ (2) }\  ein  \textit{(a)}\ rath \textit{(b)}\ rad \textit{ L }\ }} so soviel zacken oder abtheilung  hatte, als das pendulum in einer stunde vibrationes thut  pone 3600. Wenn alle minuta 2\textsuperscript{da} eine vibration  geschehe, herunter iedes mahl mit einem zacken stiege so oft das pendulum\protect\index{Sachverzeichnis}{pendulum} mit einer vibration ihm den  weg offnet, aber nicht mehr als ein zacken dieweil  mit dem anderen das pendulum\protect\index{Sachverzeichnis}{pendulum} wieder her\"{u}bergestossen, und von neuen vibrirt w\"{u}rde. Damit aber  die vibration allezeit egal bleiben soll der  ictus alle mahl fortior seyn, als pro vibratione  n\"{o}thig, und doch die vibratio wegen des anstosses  auf beyden seiten allezeit klein \edtext{bleiben. Es}{\lemma{bleiben.}\Bfootnote{ \textit{ (1) }\ Der \textit{ (2) }\  Es \textit{ L }\ }} werde also der ictus bald st\"{a}rcker bald  schw\"{a}cher, dennoch solange er nicht gar  zu sehr heruntersteiget, so doch per constructionem nicht geschehen kan wird die vibration, auch ungeachtet der obliquit\"{a}t gleich bleiben.  \pend 
 


 


 


 


 


 


 


 


 


 

