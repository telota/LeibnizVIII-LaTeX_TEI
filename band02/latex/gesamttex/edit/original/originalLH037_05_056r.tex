	\begin{ledgroupsized}[r]{120mm}
	\footnotesize 
	\pstart 
	\noindent\textbf{\"{U}berlieferung:}
	\pend
	\end{ledgroupsized}
	\begin{ledgroupsized}[r]{114mm}
	\footnotesize 
	\pstart \parindent -6mm
	\makebox[6mm][l]{\textit{L}}Konzept: LH XXXVII 5 Bl. 56. 1 Bl. 8\textsuperscript{o}. 1 S. auf 56~r\textsuperscript{o}, \unitfrac{1}{5} S. auf 56~v\textsuperscript{o}. Der Text befindet sich auf Bl. 56~r\textsuperscript{o} und wird durch eine Nebenrechnung erg\"{a}nzt, die Leibniz am unteren Rand von Bl. 56~v\textsuperscript{o} quer zur Schreibrichtung notiert. Die \"{u}brigen Neben\-rechnungen befinden sich am Rand und gr\"{o}{\ss}tenteils unterhalb des Textes. Das Papier des Texttr\"{a}gers wurde durch Erhaltungsma{\ss}nahmen stabilisiert. Auf den oberen \unitfrac{4}{5} der R\"{u}ckseite das St\"{u}ck N. ??2.\\Cc 2 Nr. 975 A\pend
	\end{ledgroupsized}
	\vspace*{5mm}
	\begin{ledgroup}
	\footnotesize
	\pstart
	\noindent\footnotesize{\textbf{Datierungsgr\"{u}nde}: Siehe oben die Einleitung zu N. ??Intro??.}\pend
	\end{ledgroup}
	
	\vspace*{8mm}
	\pstart 
	\normalsize
	\noindent[56~r\textsuperscript{o}] Deux pendules\protect\index{Sachverzeichnis}{pendule} \edtext{inegales}{\lemma{}\Bfootnote{inegales \textit{ erg.} \textit{ L}}} estant donn\'{e}es, et le nombre \edtext{des battements}{\lemma{nombre}\Bfootnote{ \textit{ (1) }\ du battement \textit{ (2) }\ des battements \textit{ L}}} de chacune dans un m\^{e}me temps, (: comme par exemple dans une heure :) estant connu; il faut diviser le plus grand nombre par le \edtext{moindre}{\lemma{le}\Bfootnote{ \textit{ (1) }\ nombre \textit{ (2) }\ moindre \textit{ L}}}; et prendre par apres le nombre quarr\'{e} du produit ou du quotient: \edtext{et autant de fois que le dit nombre}{\lemma{quotient:}\Bfootnote{ \textit{ (1) }\ et comme a le dit nombr \textit{ (2) }\ et [...] nombre \textit{ L}}}quarr\'{e} contient l'unit\'{e} autant de fois la longueur de la plus grande des deux pendules contiendra celle de la petite.\pend \pstart Par exemple si de deux pendules la plus grande fait 333 vibrations\protect\index{Sachverzeichnis}{vibration} dans \edtext{un}{\lemma{dans}\Bfootnote{\textit{ (1) }\ une \textit{ (2) }\ un \textit{ L}}} certain espace de temps, et la moindre en \edtext{m\^{e}me}{\lemma{}\Bfootnote{m\^{e}me \textit{ erg.} \textit{ L}}} temps 999, divisant 999 \edtext{vous aurez 3.}{\lemma{}\Afootnote{\textit{Links am Rand:} $\displaystyle \frac{999}{333}$}} dont le quarr\'{e} est 9 et par \edtext{consequent}{\lemma{aurez 3.}\Bfootnote{ \textit{ (1) }\ et par consequent \textit{ (2) }\ dont le [...] par consequent \textit{ L}}} la raison des longueurs sera comme d'un \`{a} 9.\pend 
\pstart De m\^{e}me, si \edtext{la moindre}{\lemma{si}\Bfootnote{ \textit{ (1) }\ l'une \textit{ (2) }\ la moindre \textit{ L}}} fait $\displaystyle1500$ battements, pendant que la plus grande fait $\displaystyle1000$; divisant $\displaystyle1500$ par $\displaystyle1000$, nous aurons $\displaystyle1+\frac{1}{2}$, ou reduisant tout \`{a} une fraction, nous aurons $\displaystyle\frac{3}{2}$, dont le quarr\'{e} est $\displaystyle\frac{9}{4}$, par consequent \edtext{la}{\lemma{consequent}\Bfootnote{ \textit{ (1) }\ une \textit{ (2) }\ la \textit{ L}}} moindre par exemple ayant quatre pouces la plus grande en aura 9.\pend

\vspace*{2mm}
\pstart
$\displaystyle\protect\begin{array}{lllllllll}
	\protect\vspace{1mm} \protect\frac{34}{21} \ \protect\mbox{\protect\large f} \ 1 +  \protect\frac{13}{21}\  & \protect\frac{35}{21}\ & \protect\frac{\protect\overset{\protect\scriptstyle 14}{\protect\cancel{3}\protect\cancel{4}}}{\protect\cancel{2}\protect\cancel{1}} \ \protect\mbox{\protect\large f} & 1 \protect\frac{14}{21}\ \lbrack \textit{sic!}\rbrack \ &
	&
\hspace{9pt} {999\atop 999} &
&\hspace{6pt} {333\atop \uline{333}} 
& 
&&&& 1\hspace{3pt}\protect\frac{2}{3}\hspace{1pt} &\
		&
		{999\atop 333} \ \protect\mbox{\protect\large f}\ \protect\frac{3}{1}
& \frac{9}{1}
%& \hspace{9pt} {\atop 999}
& \hspace{3pt}{\ 999\atop999} \\
	&&& \protect\overline{\protect\frac{5}{3} \hspace{1pt} \protect\frac{25}{9}} & \hspace{-7pt} \protect\mbox{\protect\large f} \ 2 \protect\frac{7}{9} 
	\ \protect\frac{\protect\raisebox{0ex}{\scriptsize A}^2}{\protect\raisebox{0ex}{\scriptsize B}^2}\protect
&&&\hspace{-3pt}{999\atop\overline{110889}}
\end{array}$\pend
\vspace*{2mm}
\pstart
\hspace{3pt}$1+\frac{13}{21}$ \hspace{3pt} $1+\frac{13}{21}$ \hspace{3pt} $\frac{34}{21}$ \hspace{3pt} $\frac{{13\atop13}}{9}$\pend

\vspace*{6mm}
\pstart
$\displaystyle\protect\begin{array}[t]{llllll}
1+\frac{13}{21}
&&1\hspace{9pt}\frac{13}{21}
&{13\atop\uline{13}}\\
&&\uline{1\hspace{9pt}\frac{13}{21}\hspace{2pt}}
&{39\atop}\\
\frac{26}{42}
&\frac{169}{441}
&\frac{13}{21}\hspace{3pt}\frac{169}{441}
&\frac{13}{\uline{169}}
&\hspace{3pt}{21\atop\uline{21}}\\
&&\uline{\frac{13}{21}\hspace{12pt}}
&&{{\hspace{3pt}21}\atop{\hspace{-3pt}{\uline{42}}}}\\
&&&&{441\atop}
\end{array}$\pend



\vspace*{6mm} \pstart
[56~v\textsuperscript{o}] \hspace*{4mm}\lbrack \textit{Rechnungen quer zur Schreibrichtung:}\rbrack \pend 
\vspace*{2mm}
\pstart
La moindre $1500$:\pend\pstart
La plus grande $1000$\pend
\vspace*{2mm}
\pstart
\begin{tabular}{c}
\cancel{1500}\\
\cancel{1000}\\
\end{tabular}
$\displaystyle \bigg \vert \frac{3}{2}$\pend
\vspace*{2mm}
\pstart
\hspace*{1mm}
$\displaystyle \frac{3}{2}$
\begin{tabular}{c}
\textemdash\\
\textemdash\\
\end{tabular}
$\displaystyle \frac{3}{2}$
\hspace*{5mm}
$\displaystyle \frac{9}{4}$\pend
\pstart
\vspace*{2mm}
\hspace*{4mm}
$\displaystyle 2\ \frac{1}{4}$
\pend


	 
	
