[87~r\textsuperscript{o}]
la bouche avec de l'eau de rose, le plantin ou d'ortie ou quelque autre eau astringente et vulneraire. Comme aussi estant mesl\'{e}e avec la decoction des clysteres astringents. Par exemple on fait une decoction avec les herbes appell\'{e}es tapsus \edtext{barbatus}{\lemma{tapsus}\Bfootnote{\textit{(1)} berbatus \textit{(2)} barbatus \textit{L}}} sive verbascum plantin, rubus, etc. les feuilles de chesne, les roses rouges et dans \includegraphics[width=0.014\textwidth]{images/uncia.pdf} viij de la decoction des dites herbes on m\^{e}le environ \includegraphics[width=0.014\textwidth]{images/uncia.pdf}iij ou iiij de l'essence foible \includegraphics[width=0.014\textwidth]{images/uncia.pdf}i des gros de\edtext{}{\lemma{}\Afootnote{\textit{Am Rand, ohne erkennbaren Zusammenhang mit dem Text:} 9752597 \Pfund\ sterling}\label{LH003,05_087r_sterling}} ros. seches ou des grap. de coins, ou de quelque autre syrop astringent, ou de tout cela on fait un lavement qui n'a pas son pareil dans toute la medecine pour la guerison de la vraye dysenterie, et de tous autres ulceres des boyaux. Mais il faut avoir soin de reiterer les m\^{e}mes remedes jusqu'\`{a} \edtext{la}{\lemma{la}\Bfootnote{\textit{ erg. L}}} parfaite guerison,
\edtext{adjoutant un}{\lemma{adjoutant}\Bfootnote{\textit{(1)} une \textit{(2)} un \textit{L}}} jaune d'oeuf aux dits lavements; lors que les douleurs dysenteriques sont grandes ou bien on mesle \includegraphics[width=0.014\textwidth]{images/uncia.pdf}iij ou iiij de cette essence foible avec une demie \Pfund\ \hspace{-1.8mm}\edtext{}{\lemma{}\Afootnote{\textit\textit{Am Rand, ohne erkennbaren Zusammenhang mit dem Text:} long soleil}} de lait de vache ferr\'{e}, que nous appellons lac chalybeatum vel ustulatum, ce qui se \edtext{fait en}{\lemma{fait}\Bfootnote{\textit{(1)} avec \textit{(2)} en \textit{L}}} \'{e}teignant un quarr\'{e} ou d'acier ou de petits cailloux rougis dans le dit lait; et puis on y mesle la dite essence, et on y dissout un jaune d'oeuf apres quoy on coule tout cela, et on le donne en clystere au malade. 
\pend%
\pstart%
On peut aussi tres \`{a} propos mesler \includegraphics[width=0.014\textwidth]{images/uncia.pdf}iiij de ce lait ferr\'{e} avec \includegraphics[width=0.014\textwidth]{images/uncia.pdf}iiij de la decoction des dites herbes astringentes et vulneraires et \includegraphics[width=0.014\textwidth]{images/uncia.pdf}iij ou iiij de l'essence foible un jaune d'oeuf et \includegraphics[width=0.014\textwidth]{images/uncia.pdf}i du
%\edtext{ }{\lemma{i}\Bfootnote{\textit{(1)}\ de \textit{(2)}\ du \textit{L}}}
syrop de coin pour un lavement excellent. 
\pend%
\pstart%
Je me suis un peu estendu sur la maniere de traiter cette maladie parce qu'elle est des plus cruelles qui puissent attaquer le corps humain, et parce qu'il se trouue peu de personnes qui soyent munies d'un remede capable de la guerir comme sont ceux que je viens de d\'{e}crire, estant animez et fortifiez par nostre precieuse essence sans laquelle ils ne seroient pas de grande energie.
\pend%
\pstart%
Pour le \textso{scorbut}. J'oubliois de vous dire une de ces plus grandes vertus et de vous apprendre un de ces plus utiles usages. C'est qu'elle guerit les ulceres de la bouche caus\'{e}s par le scorbut, mieux que tout autre remede, en se gargarisant et se lavant la bouche avec elle seule ou bien mesl\'{e}e avec la decoction des herbes vulneraires et astringentes ou bien mesl\'{e}e avec le lait ustulatus. Elle n'est pas moins bonne pour les contusions, les meurtrisseures, en adjoutant \`{a} \Pfund j de la foible \includegraphics[width=0.014\textwidth]{images/uncia.pdf}i d'esprit de vin ou vous tremp\'{e}s des compresses de linge et les appliqu\'{e}s sur la contusion ou meurtrisseure.
\pend%
\pstart%
Nota qu'il faut remarquer que pour les erisypeles bruslures inflammations, des \edtext{\mbox{[plaies]}}{\lemma{plies}\Bfootnote{\textit{L \"{a}ndert Hrsg.}}}, contusions, meurtrisseures, il faut tousjours se servir \edtext{de la}{\lemma{de}\Bfootnote{\textit{(1)} cette \textit{(2)} la \textit{L}}} foible et adjouter \includegraphics[width=0.014\textwidth]{images/uncia.pdf}ij d'esprit de vin \edtext{sur}{\lemma{vin}\Bfootnote{\textit{(1)} le \textit{(2)} sur \textit{ L}}}  \Pfund j au lieu que lors que c'est pour guerir les ulceres et les vieilles playes, il ne faut \includegraphics[width=0.014\textwidth]{images/uncia.pdf}i d'esprit de vin sur \Pfund j de la dite essence.
\pend%
\count\Bfootins=1500
\count\Cfootins=1500
\count\Afootins=1500
%%%% Hier endet das Stück.