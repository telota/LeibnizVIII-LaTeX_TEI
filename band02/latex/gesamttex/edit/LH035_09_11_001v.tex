% [1~v\textsuperscript{o}]
\pstart% PR: Normal einrücken, bitte.
\textso{Lemme.}
\pend
\pstart
\noindent% PR: Diesen Absatz bitte gar nicht einrücken.
\textso{Les accroissemens du temps en chaque endroit du lieu, sont en raison reciproque des vistesses}\protect\index{Sachverzeichnis}{vitesse}\textso{, que le mobile}\protect\index{Sachverzeichnis}{mobile}\textso{ y a.}
\pend
\pstart
\noindent% PR: Diesen Absatz bitte gar nicht einrücken.
Soit le lieu ou l'espace $\displaystyle EA$ divis\'{e} en parties \'{e}gales entre elles,
moindres qu'aucune ligne donn\'{e}e, $\displaystyle EB.$ $\displaystyle B(B).$ $\displaystyle (B)P$.
Je dis que les parties du temps
(:~qui seront aussi moindres qu'aucun temps donn\'{e}~:)
dans les quelles ces parties de l'espace sont parcourues,
seront entre elles en
\edtext{raison}{\lemma{}\Bfootnote{raison \textit{erg. L}}}
reciproque des vistesses avec les quelles le mobile parcourt les dites parties de l'espace:
parce que generalement les espaces estant \'{e}gaux,
comme le sont icy les parties $\displaystyle EB.$ $\displaystyle B(B).$ $\displaystyle (B)P,$
les temps sont en raison reciproque des vistesses.
Or ces parties du temps sont ce que j'appelle les accroissemens du temps en chaque endroit de l'espace.
\pend
\pstart% PR: Normal einrücken, bitte.
Theoreme III.
\pend
\pstart
\noindent% PR: Diesen Absatz bitte gar nicht einrücken.
\sloppy \textso{Les accroissemens du temps en chaque endroit du lieu qui retarde par tout \'{e}galement un mouuement, uniforme en soy même, pourront estre represent\'{e}s par les appliqu\'{e}es $\displaystyle EF,$ $\displaystyle BD,$ $\displaystyle (B)(D)$ etc. de l'Hyperbole $\displaystyle FD(D)Q$ men\'{e}es sur $\displaystyle EA,$ espace dans lequel tout le mouuement se doit faire, et qui est partie de l'Asymptote de l'Hyperbole, dont le centre $\displaystyle A$ est le même avec le point de repos.}
\pend
\count\Bfootins=1200
\pstart
\noindent% PR: Diesen Absatz bitte gar nicht einrücken.
Car par le\edtext{\textso{ th. 2.} ces accroissements}{\lemma{\textso{th. 2.}}\Bfootnote{\textit{(1)}\ les augm \textit{(2)}\ ces accroissements, \textit{L}}},
represent\'{e}z par les lignes, $\displaystyle EF$, $\displaystyle BD$, $\displaystyle (B)(D)$
\edtext{paralleles entre elles}{\lemma{}\Bfootnote{paralleles entre elles \textit{erg. L}}}[,]
sont en raison reciproque des espaces $\displaystyle AE.$ $\displaystyle AB.$ $\displaystyle A(B).$
par consequent \edtext{$\displaystyle EF$ est \`{a} $\displaystyle BD$}{\lemma{$\displaystyle EF$}\Bfootnote{\textit{(1)} ad $\displaystyle AB$ \textit{(2)} est \`{a}  \textit{(a)}\ $\displaystyle AB$ \textit{(b)}\ $\displaystyle BD$ \textit{L}}},
comme $\displaystyle AB$ \`{a} $\displaystyle AE$[,]
donc le rectangle
\edtext{$\displaystyle DBA$ est egal}{\lemma{}\Bfootnote{$\displaystyle DBA$\ \textbar\ est \textit{erg.}\ \textbar\ egal \textit{L}}}
au rectangle $\displaystyle FEA.$
et de même prenant le point $\displaystyle (B)$ quelconque au lieu du point $\displaystyle B.$
le rectangle $\displaystyle (D)(B)A$ est egal au dit rectangle fixe $\displaystyle FEA.$
et par consequent le lieu de tous les points $\displaystyle F.$ $\displaystyle D.$ $\displaystyle (D)$
sera la courbe de l'Hyperbole.
\pend
%\newpage
\pstart% PR: Normal einrücken, bitte.
Theoreme IV.
\pend
\pstart
\noindent% PR: Diesen Absatz bitte gar nicht einrücken.
\sloppy \textso{Le m\^{e}me estant pos\'{e}, les temps
}\edtext{\textso{m\^{e}mes employez \`{a} parcourir une certaine partie de l'espace comme $\displaystyle EB$ ou $\displaystyle E(B)$,}}
{\lemma{\textso{m\^{e}mes}}\Bfootnote{\textit{(1)}\ \textso{pris depuis le commencement du mouuement} \textit{(2)}\ \textso{employez} [...] \textso{l'espace}\ \textbar\ \textso{comme $\displaystyle EB$ ou $\displaystyle E(B)$} \textit{erg.}\ \textbar\  \textit{L}}}\textso{
seront representez par les portions hyperboliques, $\displaystyle FEBDF.$ $\displaystyle FE(B)(D)F$
comprises entre deux appliqu\'{e}es,
dont l'une $\displaystyle EF$ passe par $\displaystyle E$}
\edtext{\textso{point d'o\`{u} le mobile est parti,}}{\lemma{\textso{point}}\Bfootnote{\textit{(1)}\ \textso{de commencement du mouuement} \textit{(2)}\ \textso{d'o\`{u}} [...] \textso{parti,} \textit{L}}}\textso{
l'autre $\displaystyle BD$ ou $\displaystyle (B)(D)$ par $\displaystyle B$ ou $\displaystyle (B)$
point o\`{u} le mobile est arriv\'{e}.}
\pend
\newpage
\count\Afootins=1200
\count\Bfootins=1200
\count\Cfootins=1200
\pstart
\noindent% PR: Diesen Absatz bitte gar nicht einrücken.
Car les accroissemens du temps,
estant representez par les appliqu\'{e}es $\displaystyle EF.$ $\displaystyle BD.$ $\displaystyle (B)D$ etc.
\edtext{et une infinit\'{e} d'autres entre elles}{\lemma{}\Bfootnote{et une [...] entre \textit{(1)}\ deux \textit{(2)}\ elles \textit{erg. L}}}[,]
par le th. 3. les sommes des dits accroissemens, ou les temps employez
\edtext{depuis quelque point, comme $\displaystyle E$}{\lemma{depuis}\Bfootnote{\textit{(1)}\ le commencement du mouuement  \textit{(2)}\ un certain  \textit{(3)}\ quelque point, comme $\displaystyle E$  \textit{L}}},
seront represent\'{e}s par les sommes des dites ordonn\'{e}es prises
\edtext{depuis celle qui est prise pour la premiere comme}{\lemma{depuis}\Bfootnote{\textit{(1)}\ la premiere  \textit{(2)}\ celle [...] comme  \textit{L}}}
$\displaystyle EF$,
c'est \`{a} dire par les espaces compris entre la premiere, et celle qui est \`{a} present la derniere,
c'est \`{a} dire
\edtext{qui passe par le point $\displaystyle B$, ou $\displaystyle (B)$}{\lemma{}\Bfootnote{qui [...] ou $\displaystyle (B)$  \textit{erg. L}}}
o\`{u} le mobile est arriv\'{e},
s\c{c}avoir entre $\displaystyle EF$ et $\displaystyle BD$ ou $\displaystyle (B)(D)$,
c'est \`{a} dire par les espaces $\displaystyle FEBDF.$ $\displaystyle FE(B)(D)F.$
\pend
\pstart% PR: Normal einrücken, bitte.
Theoreme V.
\pend
\pstart
\noindent% PR: Diesen Absatz bitte gar nicht einrücken.
\textso{Si le mouuement d'un corps est uniforme en soy même,
mais retard\'{e} \'{e}galement par le lieu o\`{u} il passe, les espaces
}\edtext{\textso{$\displaystyle BA$ ou $\displaystyle (B)A$}}{\lemma{}\Bfootnote{\textso{$\displaystyle BA$ ou $\displaystyle (B)A$} \textit{erg. L}}}
\textso{qui restent \`{a} parcourir jusqu'au point de repos, $\displaystyle A$
}\edtext{\textso{depuis le point $\displaystyle B$ ou $\displaystyle (B)$ o\`{u} le mobile est arriv\'{e}}}{\lemma{}\Bfootnote{\textso{depuis} [...] \textso{arriv\'{e}} \textit{erg. L}}}\textso{,
estant comme les nombres,
les temps qui restent \`{a} employer jusqu'\`{a} un certain point $\displaystyle P$ pris en de\c{c}a du point de repos,
seront comme les Logarithmes des raisons de ces nombres $\displaystyle BA$ ou $\displaystyle (B)A$,
}\edtext{\textso{\`{a} $\displaystyle PA$ distance}}{\lemma{\textso{\`{a}}}\Bfootnote{\textit{(1)}\ \textso{la distan} \textit{(2)}\ \textso{la ligne} \textit{(3)}\ \textso{$\displaystyle PA$ distance} \textit{L}}}\textso{
de ce point $\displaystyle P$ du point de repos, pris pour l'unit\'{e}.}
\pend
\pstart
\noindent% PR: Diesen Absatz bitte gar nicht einrücken.
Car on s\c{c}ait que les droites $\displaystyle AP.$ $\displaystyle AB.$ $\displaystyle A(B).$ $\displaystyle AE$ estant en progression Geometrique
\edtext{con\-ti\-nuel\-le les portions Hyperboliques}{\lemma{continuelle}\Bfootnote{\textit{(1)}\ les espaces \textit{(2)}\ les portions Hyperboliques \textit{L}}}
$\displaystyle QP(B)(D)Q.$ $\displaystyle (D)(B)BD(D).$ $\displaystyle DBEFD$ seront \'{e}gales,
et par consequent
\edtext{non
\edtext{seulement}{\lemma{}\Afootnote{\textit{Am Rand:} NB\vspace{-6mm}}}
les portions hyperboliques $\displaystyle FEBDF.$ $\displaystyle FE(B)(D)F.$ $\displaystyle FEPQF$ ou\textso{ (par le th. 4.)} temps employez d\'{e}ja, mais aussi}{\lemma{}\Bfootnote{non seulement les  \textit{(1)}\ espaces \textit{(2)}\ portions [...] $\displaystyle FEPQF$\ \textit{(a)}\ mais aussi\ \textit{(b)}\ ou\ \textit{(aa)}\ temps employez d\'{e}ja, mais aussi (par le \textso{th. 4.})\ \textit{(bb)}\ (par [...] aussi\ \textit{erg. L}}}
\edtext{les portions}{\lemma{les}\Bfootnote{\textit{(1)}\ espaces \textit{(2)}\ portions \textit{L}}}
Hyperboliques $\displaystyle QP(B)(D)Q.$ $\displaystyle QPBDQ.$ $\displaystyle QPEFQ$ seront en progression Arithmetique,
d'o\`{u} il s'ensuit, comme
\edtext{d'autres}{\lemma{d'autres}\Cfootnote{\textsc{N. Mercator}, \cite{00141}\textit{Logarithmotechnia}, London 1668, prop. XIV-XV, S.~28f.
Leibniz hat in seinem Handexemplar der \textit{Logarithmotechnia} beide Theoreme kommentiert:
Siehe \cite{01189}\textit{LSB} VII,~4 N.~3\textsubscript{1}, S.~50f.}}
ont fait voir, que les
\edtext{dits dernieres portions Hyperboliques}{\lemma{dits}\Bfootnote{\textit{(1)}\ espaces \textit{(2)}\ dernieres [...] Hyperboliques \textit{L}}}
representent les Logarithmes des raisons des nombres $\displaystyle A(B).$ $\displaystyle AB.$ $\displaystyle AE$ \`{a} l'unit\'{e} $\displaystyle AP$.
\edtext{Or elles}{\lemma{Or}\Bfootnote{\textit{(1)}\ les  \textit{(a)}\ dites espaces \textit{(b)}\ dernieres portions Hyperboliques \textit{(c)}\ portions \textit{(2)}\ elles \textit{L}}}
representent aussi les
\edtext{temps \`{a} employer}{\lemma{temps}\Bfootnote{\textit{(1)}\ qui restent \textit{(2)}\ \`{a} employer \textit{L}}}
depuis les points, $\displaystyle E$ ou $\displaystyle B$ ou $\displaystyle (B)$ jusqu'au point $\displaystyle P$ (par le theor. 4.)[;] donc les dits temps seront aussi comme les logarithmes \edtext{susdits}{\lemma{logarithmes}\Bfootnote{\textit{(1)}\ de la maniere \textit{(2)}\ susdits. \textit{L}}}.
\pend
\pstart% PR: Normal einrücken, bitte.
Theoreme VI.
\pend
%\count\Afootins=1000
\pstart
\noindent% PR: Diesen Absatz bitte gar nicht einrücken.
\sloppy
\textso{Un point mobile estant port\'{e} par deux mouuements,
[dont]}\edtext{}{\Bfootnote{\textso{donc} \textit{\ L \"{a}ndert Hrsg.}}}\textso{
les lignes de direction font un angle constant entre elles;
l'un des ces deux mouuements estant et demeurant uniforme,
l'autre estant uniforme en soy même,
mais retard\'{e} \'{e}galement en chaque endroit du lieu
}\edtext{\textso{o\`{u} passe le mobile}}{\lemma{\textso{o\`{u}}}\Bfootnote{\textit{(1)}\ \textso{il passe} \textit{(2)}\ \textso{passe le mobile} \textit{L}}}\textso{
le dit point d\'{e}crira la ligne Logarithmique.}
\pend
\pstart
\noindent% PR: Diesen Absatz bitte gar nicht einrücken.
Conceuuons dans la\edtext{\textso{ seconde figure }}{\lemma{\textso{seconde figure}}\Cfootnote{Siehe [\textit{Fig. 2}].}}une
ligne droite immobile $\displaystyle AE$,
et qu'une regle inflexible $\displaystyle BF$ glisse
\edtext{d'un mouuement uniforme, et sans estre retard\'{e}}{\lemma{}\Bfootnote{d'un [...] retard\'{e} \textit{erg. L}}}
le long de cette droite $\displaystyle AE.$
gardant tousjours le même angle $\displaystyle FBE$ ou
\edtext{$\displaystyle (F)(B)E$ etc.
Et que cependant un autre mobile glisse ou roule sur la regle $\displaystyle BF$[,] de $\displaystyle B$ vers $\displaystyle F.$ d'un mouuement uniforme en soy même, mais retard\'{e} \'{e}galement en chaque endroit de la regle jusqu'au point de repos $\displaystyle F$[,] en sorte que pendant que la regle va de $\displaystyle A$ en $\displaystyle B$, de $\displaystyle B$ en $\displaystyle (B)$ etc. le mobile sur la regle aille de $\displaystyle B$ en $\displaystyle D$, de $\displaystyle D$ en $\displaystyle (D)$ etc.\\% PR: Hier bitte Absatz, normal eingerückt.
%\newpage
\hspace*{7,5mm}
Cela pos\'{e}, si les parties $\displaystyle AB$, $\displaystyle B(B),$ $\displaystyle (B)((B))$ etc. sont \'{e}gales}{\lemma{$\displaystyle (F)(B)E$ etc.}\Bfootnote{\textit{(1)}\ La droite $\displaystyle AE.$ estant divis\'{e}e en parties \'{e}gales $\displaystyle AB$, $\displaystyle B(B)$. $\displaystyle (B)((B))$ etc. \textit{(2)}\ Et que [...] $\displaystyle AB$, $\displaystyle B(B),$\ \textit{(a)}\ etc. sont \'{e}gales\ \textit{(b)}\ $\displaystyle (B)((B))$ etc. sont \'{e}gales \textit{L}}}
entre elles,
les espaces parcourus, $\displaystyle AB.$ $\displaystyle A(B).$ $\displaystyle A((B))$ etc.
seront en progression Arithmetique,
or les espaces parcourus par un mouuement
\edtext{uniforme (comme est celuy de la regle $\displaystyle BF$ le long de la ligne $\displaystyle AE$) sont comme les temps employez,}{\lemma{uniforme}\Bfootnote{\textit{(1)}\ sont comme les temps \textit{(2)}\ (comme [...] regle\ \textbar\ $\displaystyle BF$ \textit{erg.}\ \textbar\ le long [...] temps\ \textbar\ employez \textit{erg.}\ \textbar\ , \textit{L}}}
donc les temps employez seront aussi en progression arithmetique,
et par consequent par le th. 5.
(voyez sa demonstration \`{a} l'endroit marqu\'{e} de NB)
les espaces qui restent \`{a} parcourir dans la regle jusqu'au point de
\edtext{repos $\displaystyle F$,
s\c{c}avoir $\displaystyle DF.$ $\displaystyle (D)(F).$ $\displaystyle ((D))((F))$ etc. ordonn\'{e}es de la droite $\displaystyle F(F)((F))$ sur la courbe $\displaystyle D(D)((D))$ seront en progression geometrique, donc le lieu de [toutes leurs]}{\lemma{repos $\displaystyle F$,}\Bfootnote{\textit{(1)}\ seront en progression g \textit{(2)}\ s\c{c}avoir [...] etc.\ \textit{(a)}\ seront en progression geometrique, et par consequent les\ \textit{(b)}\ ordonn\'{e}es [...] geometrique,\ \textit{(aa)}\ et par consequent le lieu qui passe par leur\ \textit{(bb)}\ donc [...] toute leur \textit{L ändert Hrsg.}}}
\edtext{terminations, $\displaystyle D.$}{\lemma{}\Bfootnote{terminations,\ \textbar\ s\c{c}avoir \textit{gestr.}\ \textbar\ $\displaystyle D.$ \textit{L}}}
$\displaystyle (D).$
\edtext{$\displaystyle ((D))$ ou de tous les points, o\`{u} se trouve le mobile marchant sur la regle, et port\'{e} en même temps par la regle de la maniere susdite; sera la ligne Logarithmique}{\lemma{$\displaystyle ((D))$}\Bfootnote{\textit{(1)}\ sera la\ \textit{(a)}\ courbe\ \textit{(b)}\ Ligne Logarithmique \textit{(2)}\ ou\ \textit{(a)}\ port\'{e}\ \textit{(b)}\ de tous [...] port\'{e}\ \textbar\ en même temps \textit{ erg.}\ \textbar\ par la [...] Logarithmique. \textit{L}}}.% [2~r\textsuperscript{o}]
% \pend