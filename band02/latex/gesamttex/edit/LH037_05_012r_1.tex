\begin{ledgroupsized}[r]{120mm}
\footnotesize
\pstart
\noindent\textbf{\"{U}berlieferung:}
\pend
\end{ledgroupsized}
\begin{ledgroupsized}[r]{114mm}
\footnotesize
\pstart
\parindent -6mm
\makebox[6mm][l]{\textit{L}}Auszug: LH XXXVII 5 Bl. 12. 1 Bl. 4\textsuperscript{o}. 15 Z. am Anfang von Bl. 12~r\textsuperscript{o}; der \"{u}brige Text auf Vorder- und R\"{u}ckseite geh\"{o}rt zu den St\"{u}cken N. 31\textsubscript{2} und N. 31\textsubscript{3}. Blattr\"{a}nder unregelm\"{a}{\ss}ig, aber ohne Textverlust. Blatt durch Papiererhaltungsmaßnahmen gesichert. Teil eines Wasserzeichens. \\Cc 2, Nr. 944
\pend
\end{ledgroupsized}

\vspace*{8mm}
\pstart
\normalsize
\noindent
[12~r\textsuperscript{o}] Wallis. \textit{De motu} part. III. cap. 10. prop. 2. \textit{In motu retardato} [...] \textit{si posito aliquo celeritatis gradu quo feratur mobile, ut $C$. intelligatur vis impeditiva in se aequabilis, continuo accedere, quae propterea singulis momentis tantundem demat, fient celeritatis gradus continuo sequentes:} [...] \textit{$C - 1$. $C - 2$. $C - 3$. $C - 4$. etc. puta usque ad $C - C \, \sqcap \, 0$ ubi motus primo positus plane absumitur.}
\pend 
\pstart \textit{Adeoque si porro continuetur ablatio puta ad $C - C - 1$. $C - C - 2$. $C - C - 3$.} [...] \textit{etc: hoc est ad $0 - 1$. $0 - 2$. $0 - 3$.} [...] \textit{etc.} vel \textit{ad $- 1$. $- 2$. $- 3$.} [...] \textit{etc. sitque vis illa impeditiva, non impeditiva simpliciter, sed in contrarium motiva, habebitur motus in partes contrarias cum celeritatis gradibus, $1$. $2$. $3$.} [...] \textit{etc.} [...] \textit{si vero simpliciter impeditiva sit, ubi ad $C - C$ pervenitur, tollitur motus, sed quicunque deinceps succedat impedimenti gradus utut fortius impediat, non tamen in contrarias partes pellit. Supponitur }[\textit{utique}]\edtext{}{\Bfootnote{utrique\textit{\ L \"{a}ndert Hrsg. nach Vorlage}}}\textit{ vim motricem non habere. Prioris instantiam habemus in motu gravium sursum projectorum, seclusa consideratione impedientis medii, ubi post superatam a gravitate vim sursum projicientem, descendit grave. \textso{Posteriorem quadantenus refert motus projectorum (seclusa gravitatis consideratione), in quamcunque partem, ubi medii densitas vim projectricem obtundit, et sensim minuit, tandemque tollit, sed non in partes contrarias repellit.}} Haec \edtext{ille.}{\lemma{ille}\Cfootnote{\cite{00301}\cite{01008}J. \textsc{Wallis}, \textit{Mechanica}, London 1670-1671, S. 647f. (\textit{WO} I, S. 994). Unterstreichung von Leibniz. Auslassungs\-zeichen vom Hrsg.}}
\pend 
 
