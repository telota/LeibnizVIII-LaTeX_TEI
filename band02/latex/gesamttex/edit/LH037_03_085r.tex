\edtext{minuetur. [85~r\textsuperscript{o}] Cum ergo possit augeri calor, quin $\langle$tan$\rangle$tundem minuatur, manebit in statu priore.}{\lemma{minuetur}\Bfootnote{\textit{(1)}\ , ac proinde [85~r\textsuperscript{o}] $\langle$rar$\rangle$efactio quoque. \textit{(2)}\ . Cum [...] priore. \textit{L}}}
\pend%
\pstart%
Contra si ponamus calorem\protect\index{Sachverzeichnis}{calor} diminui, aere condensato, liquor ascendet ultra \textit{G} in \textit{T} in canna, ac proinde
\edtext{descendet in vase ex \textit{F}}{\lemma{descendet}\Bfootnote{\textit{(1)}\ ex vase \textit{(2)}\ in vase ex \textit{F} \textit{L}}}
in \textit{U}.
Ergo et corpus innatans \textit{H} ergo et baculus
\edtext{\textit{HI} depresso ergo laminae brachio \textit{LN} elevabitur oppositum seu operculum}{\lemma{\textit{HI}}\Bfootnote{\textit{(1)}\ attolletur ergo lamina \textit{(2)}\ depresso [...] oppositum\ \textbar\ seu operculum \textit{gestr.}\ \textbar\ seu operculum \textit{L}}}
\textit{NM} ergo plus aeris admittetur; augebitur ergo calor\protect\index{Sachverzeichnis}{calor}.
Ergo quantum diminuetur Calor, tantum eo ispo augebitur
\edtext{(donec scilicet ignis ita debilis devenerit, ut ne maxima quidem apertura data gradum caloris desideratum dare queat.)}{\lemma{}\Bfootnote{(donec [...] queat.) \textit{erg. L}}}
seu \edtext{quod eodem redit,}{\lemma{quod}\Bfootnote{\textit{(1)}\ idem est \textit{(2)}\ eodem redit, \textit{L}}}
manebit semper
\edtext{idem. Haec summatim Athanoris nostri Ratio est: sed in praxi circa proportiones partium inter se recte contemperandas multiplici consideratione opus est quam alias cum obstaculis remediisque fusius exponam.}{\lemma{idem.}\Bfootnote{\textit{(1)}\ Seu potius continua vibratione nunc amittet nunc recuperabit statum priorem. \textit{(2)}\ Haec  \textit{(a)}\ in summa \textit{(b)}\ summatim [...] Ratio est: \textit{(aa)}\ sed in quo exacte \textit{(bb)}\ sed in [...] contemperandas \textit{(aaa)}\ multa consi \textit{(bbb)}\ multiplici [...] cum \textit{(aaaa)}\ impedi \textit{(bbbb)}\ obstaculis [...] exponam. \textit{L}}}
\pend%
\pstart%
Praxis ergo Machinae haec est, ubi primum eum calorem\protect\index{Sachverzeichnis}{calor} nactus es, quem desideras, quemque conservari cupis, tum
\edtext{alimento igni praebito,}{\lemma{alimento}\Bfootnote{\textit{(1)}\ ignis tum ape \textit{(2)}\ igni praebito, \textit{L}}}
tum apertura constituta, baculum
\edtext{\textit{HI}}{\lemma{\textit{HI}}\Bfootnote{\textit{erg. L}}}
ei aperturae, quam tunc
\edtext{habes, seu laminae ei in statu positae alliga.}{\lemma{habes,}\Bfootnote{\textit{(1)}\ alliga. \textit{(2)}\ seu [...] alliga. \textit{L}}}
Quia enim baculus pluribus uncis sibi suppositis constare potest, et laminae brachium operculo \textit{NM} oppositum \textit{LN} annulos habet, potes quem velis uncum baculi in quem vis annulum laminae immittere.
Et quanto major apertura est tanto quoque inferior baculi uncus annulo
\edtext{laminae inseretur.}{\lemma{laminae}\Bfootnote{\textit{(1)}\ innectetur \textit{(2)}\ inseretur. \textit{L}}}
\pend%
\newpage
\pstart%
Notandum quoque est plurima hic contemperari posse. 
Nam quanto amplius est vas \textit{DF} tanto minus ascendit in eo liquor ex canna depressus, quippe se per totam vasis amplitudinem diffundens.
Ergo eo \edtext{casu debet}{\lemma{casu}\Bfootnote{\textit{(1)}\ quo \textit{(2)}\ debet \textit{L}}}
baculus \textit{HI}
\edtext{innecti annulo}{\lemma{innecti}\Bfootnote{\textit{(1)}\ parti \textit{(2)}\ annulo \textit{L}}}
\edtext{laminae centro}{\lemma{laminae}\Bfootnote{\textit{(1)}\ ad centrum \textit{(2)}\ centro \textit{L}}}
\textit{N} propiori, ita enim exigua elevatione vel depressione, plurimum operculi aperiet vel claudet.
\edtext{Contra si}{\lemma{Contra}\Bfootnote{\textit{(1)}\ quanto \textit{(2)}\ si \textit{L}}}
vas est angustius, baculus innectetur parti laminae magis a centro \textit{N} remotae seu
\edtext{propiori}{\lemma{propiori}\Bfootnote{\textit{erg. L}}}
vere, \textit{L}.
Crassities quoque vasis \textit{A} \edtext{et altitudo vasis \textit{DF}}{\lemma{}\Bfootnote{et altitudo vasis \textit{DF} \textit{erg. L}}}
et \edtext{longitudo crassities inclinatioque}{\lemma{longitudo}\Bfootnote{\textit{(1)}\ crassitiesque \textit{(2)}\ crassities inclinatioque \textit{L}}}
cannae \textit{BC}, et magnitudo Registri
\edtext{\textit{P}}{\lemma{\textit{P}}\Bfootnote{\textit{erg. L}}}
atque operculi
\edtext{\textit{M}}{\lemma{\textit{M}}\Bfootnote{\textit{erg. L}}}
distantiaque operculi \textit{M} a centro \textit{N} (quae
\edtext{omnia effectum variant}{\lemma{omnia}\Bfootnote{\textit{(1)}\ variant \textit{(2)}\ effectum variant \textit{L}}}%
) justa
\edtext{proportione (per ipsam experientiam determinanda,)}{\lemma{proportione (\phantom)\hspace{-1.2mm}}\Bfootnote{\textit{(1)}\ ipsa experientia determinanda \textit{(2)}\ per ipsam experientiam determinanda,\phantom(\hspace{-1.2mm})
 \textit{L}}}
caeteris accommodari debent.
Quanto enim major est inclinatio cannae \textit{BC}
\edtext{tanto minor est altitudo [Liquoris] in infimum%
}{\lemma{tanto}\Bfootnote{%
\textit{(1)}\ minore opus est altitudine mercurii, %
\textit{(a)}\ etiam infimi %
\textit{(b)}\ infimi; mercurius %
\textit{(2)}\ minor est altitudo %
\textbar\ Liquor \textit{ändert Hrsg.} \textbar\ %
in infimum \textit{L}}}
locum \textit{U} subsidentis.
\edtext{Nam liquor in vase \textit{DE} utcunque maxime subsidens nunquam}{%
\lemma{Nam}\Bfootnote{%
\textit{(1)}\ Mercurius numqu %
\textit{(2)}\ liquor [...] nunquam %
\textit{L}}} infra horizontem liquoris in canna utcunque maxime ascendentis descendere debet.
Nota quoque esse debet%
%%
\edtext{}{\lemma{}\Afootnote{\textit{Am Rand:}
NB. NB posset et tale artificium adhiberi ut % \textsuperscript{[a]} 
occlusio semel facta maneat, etsi redescendat liquor in vase, cum corpore et baculo et ita augeri possit, non vero minuatur nisi quando descendit infra gradum datum. Haec observatio magni momenti est,\textsuperscript{[a]} quae ne Drebelio quidem in mentem venit.\vspace{2mm}\\%
{\footnotesize%
\textsuperscript{[a]} est, \textit{(1)}\ ne a \textit{(2)}\ quae ne \textit{L}\vspace{-4mm}}}}
%%
\edtext{capacitas Ampullae}{\lemma{capacitas}\Bfootnote{\textit{(1)}\ vasis \textit{(2)}\ Ampullae \textit{L}}}
\textit{A} et vasis \textit{DE}
\edtext{in comparatione ad}{\lemma{in}\Bfootnote{\textit{(1)}\ ordine \textit{(2)}\ comparatione ad \textit{L}}}
capacitatem cannae;
ut quantitas aeris aut liquoris intrantis in cannam aut ex ea exeuntis in calculum venire possit.
Sed et ad summam exactitudinem opus est
ut aer in canna \textit{TG} aut \textit{TR}
eundem caloris\protect\index{Sachverzeichnis}{calor} gradum percipiat,
quem aer in Ampulla \textit{A.}
alioqui rarefactio aeris in canna non determinabit exacte calorem\protect\index{Sachverzeichnis}{calor} aeris in ampulla;
denique ea proportio danda est, ut quando exempli causa
\edtext{duplicatur vel triplicatur calor[,]\protect\index{Sachverzeichnis}{calor} ascensio}{\lemma{}\Bfootnote{duplicatur\ \textbar\ \textit{(1)}\ (triplicatur) \textit{(2)}\ vel triplicatur calor \textit{erg.}\ \textbar\ ascensio \textit{L}}}
quoque corporis \textit{H} seu baculi \textit{HI},
ac proinde angulus \textit{MNP} aperturae operculi,
atque ita sector aeris admissi
\edtext{angulo comprehensus,}{\lemma{angulo comprehensus}\Bfootnote{\textit{erg. L}}}
duplicetur aut triplicetur.%
\edtext{}{\lemma{}\Afootnote{\textit{Am Rand:}
Etiam ignis Lampadum mirarum mutationum capax, ab ipsa anni tempestate ab aeris statu, qui crassior lampades facit copiosius urere, ut in homine lampadem vitalem hyeme. Ut ergo hic quoque in ordinem redigatur res iterum regi potest tum aeris admissi, tum olei tum vero fortasse rectius, ellychnii inclinatione.\vspace{-8mm}}}
\pend%
\pstart%
Et poterunt subdivisiones fieri tanto accuratiores quanto foramen
\edtext{\textit{NP} et}{\lemma{\textit{NP}}\Bfootnote{\textit{(1)}\ aut \textit{(2)}\ et \textit{L}}}
operculum \textit{NM} sunt longiora.
\pend%
\count\Bfootins=1500
\count\Cfootins=1500
\count\Afootins=1500