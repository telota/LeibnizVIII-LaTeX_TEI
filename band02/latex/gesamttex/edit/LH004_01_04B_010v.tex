% [10~v\textsuperscript{o}]
%\vspace*{1.0em}%  PR: Diesen leeren Zeilenabstand bitte behalten !!!
\pstart%
% \edtext{}{\lemma{}\Afootnote{\textit{Am Rand:} Febr. 1648 partes.}}
\noindent% PR: Neuer Abschnitt.
Febr. 1648 partes.
Certum\edlabel{004_01_04b_009r_pU4a}
\edtext{}{{\xxref{004_01_04b_009r_pU4a}{004_01_04b_009r_pU4b}}\lemma{Certum [...] tendit}\Cfootnote{Für diese Passage aus Descartes' verschollenem Ms. besteht eine parallele Überlieferung in \cite{01144}\textsc{R.~Descartes}, \textit{Opuscula posthuma}, Amsterdam 1701, \glqq Primae co\-gi\-ta\-tio\-nes circa generationem animalium\grqq, S. 22f. Siehe \cite{00120}\textit{DO} XI, S.~537.20-538.10.}}%
est membra foetus inchoari ex solo semine antequam sanguis fluat per umbilicum alioquin omnes partes solidae fierent intortae cum cor magis
\edtext{vergat in}{\lemma{vergat}\Bfootnote{\textit{(1)}\ per \textit{(2)}\ in \textit{L}}}
sinistram partem quam in dextram.
\pend%
\pstart%
Arteriae ubique eo feruntur quo leges motus eas dirigunt non habita venarum ratione venae vero eo feruntur quo ipsis per arterias licet unde fit ut arteriae sint infra venas in cute quod minus a partibus internis impediebantur ab initio quam ab occursu externorum.
\pend%
\pstart%
Vena adiposa dextra est ab emulgente et sinistra a trunco cavae propter inclinationem hepatis versus sinistram.
\pend%
\pstart%
Ad ratiocinationem intelligendam quae exprimit in foetu ea quae a matre attentius cogitantur, supponendus est foetus in utero ita situs, ut caput habeat versus caput dorsum versus dorsum et latus dextrum versus dextrum matris et sanguinem a capite matris versus omnem uteri ambitum aequaliter dispergi ac deinde colligi in umbilico velut in centro, unde rursus eadem ratione ad omnes foetus partes tendit.%
\edlabel{004_01_04b_009r_pU4b}%
\pend%
\pstart%
Certum est cavitates oris et narium humoribus impleri initio quibus cutis distenditur, donec os et nares perforentur, vidi enim in pullis 5
\edtext{vel 6 dierum}{\lemma{vel 6}\Bfootnote{\textit{(1)}\ diebus \textit{(2)}\ dierum \textit{L}}}
locum rostri esse valde crassum et tumidum, et deinde in pullis 7 vel 8 dierum esse plene acutum rostri instar ore scilicet
\edtext{perforato elapsis}{\lemma{perforato}\Bfootnote{\textit{(1)}\ os \textit{(2)}\ elapsis \textit{L}}}
humoribus quibus cavitates illae implebantur.
\pend%
\pstart%
In vitulis recens natis clare patet oesophagum adhaerere sinistro lateri asperae arteriae versus spinam, et truncum descendentem aortae ire adhuc magis versus sinistrum et tamen non videri recedere a medio corporis, oesophagus autem juxta cor transit intra illum truncum aortae descendentem et venam cavam versus sinistrum latus, sicque cava manet versus pectus et latus dextrum. Hic apparet a dextro cordis ventriculo arteriam versus inferiora descendisse quae statim in duos ramos divisa est ex eo quod inter utrumque aer collectus sit qui asperam arteriam formare coepit inter duas arterias, quarum una versus spinam vergens abiit in pulmones et dicta fuit vena arteriosa alia versus pectus ascendens occurrit sanguini ex trunco aortae ascendentis versus inferiora
\edtext{[reflexo]}{\lemma{reflexi}\Bfootnote{\textit{L \"{a}ndert Hrsg.}}}
atque ideo versus inferiora reflexa est, et dicta truncus aortae descendentis;
quare vero haec aorta descendens versus sinistram partem asperae arteris potius quam versus dextram et versus spinam potius quam versus pectus, est quaerendum.
\pend%
\pstart%
Cor ascendens directe fuit in medio corporis versus
\edtext{spinam. Truncus cavae ab hepate ad caput ascendens}{\lemma{spinam}\Bfootnote{\textit{(1)}\ a  \textit{(a)}\ capite vero ad cor \textit{(b)}\ capite vero ad \textit{(2)}\ a corde ascend \textit{(3)}\ . Truncus [...] ascendens \textit{L}}}
inflexus fuit versus partem dextram et versus pectus, sicque conjunctus trunco aortae ascendentis ejus dextrum latus contingens; auricula dextra fere tota versus pectus, sinistra versus spinam vergebat, erat vero in parte anteriore inter duas auriculas intervallum venae arteriosae ex dextro ventriculo egredientis; in posteriore
\edtext{[nullum]}{\lemma{nullam}\Bfootnote{\textit{L \"{a}ndert Hrsg.}}}
nisi valvulae per quam sanguis ex cava in arteriam venosam fluebat.
\pend%
\pstart%
Vitulus in aqua suffocatus habebat utrumque cordis ventriculum concreto sanguine plenissimum, ut et venas non autem arterias; et extrahendo sanguinem ex dextro ventriculo, qui erat in sinistro, per valvulam arteriae venosae, sequebatur et crassities grumi sanguinis per illam venam egredientis, aequabat minimum meum digitum.
\pend%
\pstart%
Dexter ventriculus anteriorem partem omnem occupabat, sed magis in dextram vergebat, sinister vero ita occupabat partem posteriorem, ut plane in medio corporis situs videretur.
\pend%
\pstart%
Fibrae in superficie corporis recta descendere videbantur a basi ad mucronem venae vero sequi sanguinis descensum in cor; et arteriae ejus e corde egressum, atque ideo se invicem decussabant.
\pend%
\pstart%
Arteriae venosae duae valvulae erant omnium % \edtext{}{\lemma{omnium}\Cfootnote{Steht f\"{u}r valvularum.}}
cordis maxime vicinae spinae, eique parallelae apertaque illa quae erat spinae proxima, vidi alteram solam distinguere meatum aortae ab arteria venosa sanguinemque per hanc in cor labi
\edtext{premendo a}{\lemma{premendo}\Bfootnote{\textit{(1)}\ ex \textit{(2)}\ a \textit{L}}}
dextra parte tum ex pulmonibus tum praecipue ex cava per valvulam, atque inde transversim versus auriculae sinistrae extremitatem, ut postea tam ex dextra quam ex sinistra deorsum reflexus sinistrum hunc ventriculum egrediatur. Sanguis in dextrum latus incidebat a tribus
\edtext{partibus manifeste}{\lemma{partibus}\Bfootnote{\textit{(1)}\ maxime \textit{(2)}\ manifeste \textit{L}}}
distinctis, nempe sinistra, media et dextra, sinistra erat truncus cavae inferior, media erat truncus cavae superior; dextra erat extremitas auriculae ex qua reflectebatur: in eodem etiam ordine erat vena coronaria quae videbatur esse quartus meatus ex quo sanguis in dextrum latus fluebat, et omnium maxime a sinistra parte veniebat, sed aliis erat minor. Hicque apparuit sanguinem qui ex cava in sinistrum ventriculum fluit per valvulam non venire, nisi a
\edtext{[cavae]}{\lemma{cava}\Bfootnote{\textit{L \"{a}ndert Hrsg.}}}
parte inferiori, quae a superiori apparet esse distincta, ut et coronaria videtur ab utroque trunco distincta quanquam earum tria orificia in dextrum ventriculum simul incidant.
[11~r\textsuperscript{o}]
\pend%
%\count\Bfootins=1500
%\count\Cfootins=1500
%\count\Afootins=1500