[7~v\textsuperscript{o}]
\pend%
\count\Bfootins=1500
\count\Cfootins=1500
\count\Afootins=1500
\pstart%
Man kan aus der Music die Naturen und Temperamenten\protect\index{Sachverzeichnis}{Temperament} unterscheiden, einer hohret dieses, der andere ein anders Lied gern, und dahehr were guth flei{\ss}ige observationes mit den Tarantulen\protect\index{Sachverzeichnis}{Tarantel}, und denen so von ihnen gestochen anzustellen.
\pend%
\pstart%
Einen iedem Medico\protect\index{Sachverzeichnis}{medicus} soll verm\"{o}ge seiner pflicht aufgelegt werden alles notabels so er hohret und siehet umbst\"{a}ndiglich aufzuzeichnen, und sonderlich die ihm selbst begegnenden casus.
Es ist ja die opinion, da{\ss} Hippocrates,\protect\index{Namensregister}{\textso{Hippokrates} um 460-um 370 v.Chr.} das fundament seiner wi{\ss}enschafft auch im Tempel Aesculapii\protect\index{Namensregister}{\textso{Asklepios} (Aesculapius), Gott der Heilkunde}
\edtext{gelegt, der}{\lemma{gelegt,}\Bfootnote{\textit{(1)} alda \textit{(2)} der \textit{L}}}
in der insel Cos\protect\index{Ortsregister}{Kos} als seinem vaterlande heut zu Tage Longa\protect\index{Ortsregister}{Kos} genant, war. Die von ihren Kranckheiten\protect\index{Sachverzeichnis}{Krankheit} genesen waren wurden daselbst einregistrirt, und die Mittel dadurch sie gene"sen aufgeschrieben. Diese hat Hippocrates\protect\index{Namensregister}{\textso{Hippokrates} um 460-um 370 v.Chr.} etwas abgek\"{u}rzet und den nachkommen hinterla{\ss}en, da{\ss} also die wi{\ss}enschafft noch \"{u}brig, ob gleich der Tempel l\"{a}ngst verbrand, weil nur solche wenige particular observationes uns ein solches liecht gegeben, ja die medicinam\protect\index{Sachverzeichnis}{medicina} rationalem erhalten, warumb sind wir denn so blind gewesen, da{\ss} wir ein solches nicht universaliter mit mehreren flei{\ss} und ordnung angestellt, w\"{u}rden gewi{\ss} in 100 jahren mehr lernen, als von Hippocrate\protect\index{Namensregister}{\textso{Hippokrates} um 460-um 370 v.Chr.} an, bis auf den anfang dieses seculi geschehen. Ja nicht allein in 100 sondern in 10.
\pend%
\pstart%
Man soll alle patienten die in Nosocomiis\protect\index{Sachverzeichnis}{nosocomium} sterben ofnen la{\ss}en.
\edtext{Zum wenigsten an dem orth ihrer Kranckheit\protect\index{Sachverzeichnis}{Krankheit}. Was gro{\ss}en herren nicht beschwehrlich sollen sich auch privati nicht beschwehrlich d\"{u}ncken la{\ss}en.}{\lemma{Zum [...] la{\ss}en.}\Bfootnote{\textit{erg. L}}}
\pend%
\pstart%
Da{\ss} die milz\protect\index{Sachverzeichnis}{Milz} eine seuere oder scharffe materi gebe, ist ein exempel in einem etlichen jahrigen kind, so stets hustete, und doch nichts aus warff, als mans nach seinem todt geofnet war milz\protect\index{Sachverzeichnis}{Milz} zu klein, lung\protect\index{Sachverzeichnis}{Lunge} und leber\protect\index{Sachverzeichnis}{Leber} zu gro{\ss}. Ergo materia die in die milz\protect\index{Sachverzeichnis}{Milz} gehohrt ist hier\"{u}ber gangen.%
% Hier folgt Bl. 8r.