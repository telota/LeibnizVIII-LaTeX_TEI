\begin{ledgroupsized}[r]{120mm}%
\footnotesize%
\pstart%
\noindent%
\textbf{\"{U}berlieferung:}%
\pend%
\end{ledgroupsized}%
\begin{ledgroupsized}[r]{114mm}%
\footnotesize%
\pstart%
\parindent -6mm%
\makebox[6mm][l]{\textit{LiH}}%
Anstreichungen und Anmerkungen in
\cite{01148}\textsc{F. de le Boe (Sylvius)}, \textit{Idea praxeos medicae}, Frank\-furt am Main 1671:
\textsc{G\"{o}ttingen}, Nieders\"{a}chsische Staats- und Universit\"{a}tsbibliothek, 8 MED PRACT 96/37.
\pend%
\end{ledgroupsized}%
%
\vspace*{5mm}%
\begin{ledgroup}%
\footnotesize%
\pstart%
\noindent%
\footnotesize{%
\textbf{Datierungsgr\"{u}nde:}
Ein Exemplar der \textit{Idea praxeos medicae} wurde am 3. M\"{a}rz 1671 an Leibniz geliefert (siehe \textit{LSB} I,~1 N.~291, S.~436.25).
% Das Buch wird von ihm auch anderweitig erw\"{a}hnt (siehe LH~III~4, 7a Bl.~1r), datiert auf ??.
Das fr\"{u}he Anschaffungsdatum
%und die weitere Besch\"{a}ftigung rechtfertigen 
legt eine Datierung dieser Marginalien auf die Zeit vor Leibniz' Aufenthalt in Paris nahe.
Eine spätere Datierung ist jedoch nicht ausgeschlossen.%
}%
\pend%
\end{ledgroup}%
%
%
\vspace*{8mm}%
\count\Bfootins=1200
\count\Cfootins=1200
\count\Afootins=1000
\pstart%
\normalsize%
\noindent%
[p. 20]
37. \textit{Fames confestim a Bile pinguiore}\edtext{}{\lemma{}\Afootnote{\protect\rule[-2.8mm]{0mm}{0mm}\hspace{1.8mm}\textit{Unterstrichen}: \textit{Bile pinguiore}}} \textit{Diminuta curabitur, emendando} illam Bilem, aut eandem, si copia simul redundaverit, \textit{educendo} vel sursum, vel deorsum.
\pend%
\pstart%
38. \textit{Bili} huic \textit{emendandae}\edtext{}{\lemma{}\Afootnote{\textit{Unterstrichen}: \textit{Bili} huic \textit{emendandae}. \textit{Am Rand}: Cura bilis pinguioris vid. p. 90. \S~25.}} conducit prae caeteris omnibus \textit{Elixir proprietatis}, ad guttas v. vel vj. ex vino, aut convenienti mistura assumptum, et imprimis paulo ante assumendum cibum.
\pend%
\pstart%
[p. 59] [...] vel parva usurpato, irritatur mox ad sui contractionem\edtext{}{\lemma{}\Afootnote{\textit{Leibniz streicht die Silbe} con \textit{in} contractionem \textit{durch und schreibt dar\"{u}ber} at}}, contentorumque suorum expulsionem.
\pend%
\pstart%
[p. 69] 28. Ad \edtext{Flatus vero tam in Ventriculo, quam Intestinis haerentes}{\lemma{}\Afootnote{\textit{Unterstrichen}: Flatus \textit{und} tam in Ventriculo, quam Intestinis haerentes}}, molestosque compescendos, discutiendosque conducet \textit{Mistura} sequens exemplaris loco Tyronibus servitura.
\pend%
\pstart%
[p. 77] 22. \textit{Iners} sit \textit{Sal Bilis volatile}, ob assumpta diutius et copiosius \textit{Alimenta} multum \textit{viscida} vel simul \textit{pinguia}, quin et quandoque \textit{Spirituosa}, ipsumque adeo Vini spiritum\edtext{}{\lemma{}\Afootnote{\textit{Am Rand}: Curam vid. p. 20. \S\ 38. sqq.}}; cujus abusum non infrequenter excipit aeque Morbus regius, quam Ascites Hydrops.
\pend%
\count\Afootins=1200
\pstart%
[p. 88]
18. \textit{Bilis acrior} sequitur [...] 4. \textit{Vigilias nimias.} 5. \textit{Iram et Curas} frequentiores. 6. \textit{Alvum adstrictiorem.}\edtext{}{\lemma{}\Afootnote{\textit{Am Ende des Absatzes angef\"{u}gt}: Vid. p. 166. \S\ 14.}}
\pend 
\pstart 19. \textit{Bilis pinguior} debetur [...] \textit{Oleis stillatitiis}, ut et \textit{Vini Spiritui}, cum oleosis praesertim parato, \textit{Anisato} puta, etc.\edtext{}{\lemma{}\Afootnote{\textit{Am Ende des Absatzes angef\"{u}gt}: Vid. p. 166. \S\ 15. Cura vid. p. 90. \S\ 25. 20. \S\ 38.}}
\pend%
\pstart%
[p. 90] 25. [...] vel admiscendo eorum aliquid cum \textit{Potu ordinario}, ac praecipue \textit{Spiritum Salis vel Nitri dulcem}, etc.\edtext{}{\lemma{}\Afootnote{\protect\rule[-2.8mm]{0mm}{0mm}\textit{Am Ende des Absatzes angef\"{u}gt}: Vid. p. 20. \S\ 38. sqq.}}
\pend%
\pstart%
[p. 98] 9. [...] postquam \textit{perpetuus est desidendi, dejiciendique Conatus, cum Excretione pauca tum mucosa, tum purulenta.}\edtext{}{\lemma{}\Afootnote{\textit{Am Ende des Absatzes angef\"{u}gt (in gr\"{u}ner Tinte, nicht sicher von Leibniz)}: Vid. p.~102.}}
\pend%
\pstart%
[p. 107] 44.\edtext{}{\lemma{}\Afootnote{\textit{Unterstrichen}: 44.}} \textit{Alvi Fluxus Cruentus curabitur}, si ab \textit{humore acri} vasa rodente ortum habuerit [...]
\pend%
\pstart%
[p. 166] 14. [...] \textit{Vigiliis, corporis exercitio nimio et protracto, Iracundia continua}, imprimis cum \textit{Solicitudine} juncta.\edtext{}{\lemma{}\Afootnote{\textit{Am Ende des Absatzes angef\"{u}gt}: Respice ad p. 88. \S\ 18. p. 77. \S}}
\pend%
\pstart%
15. \edtext{\textit{Pinguior} redditur \textit{Bilis} praesertim ex Usu nimio \textit{Alimentorum pinguium}, cum \textit{pinguedine} saltem multa, Butyro, Oleo, etc \textit{paratorum}; quo referri possunt \textit{Olea stillatitia} saepius usurpata.}{\lemma{}\Afootnote{\textit{Am Rand}: Curam Bil. pingu. vid. p. 90. \S\ 25. p. 46. \S\ 29.}}\edtext{}{\lemma{}\Afootnote{\textit{Am Ende des Absatzes angef\"{u}gt}: Resp. ad p. 88. \S\ 19. p. 77. \S\ 22.}}
\pend 
\pstart [p. 230] 14. [...] \textit{Sanguinisque Effusio}, una cum \textit{Peripneumonia}; quin \edtext{post apertum apostema \textit{Phthisis}, et ut plurimum tandem \textit{Mors.}}{\lemma{}\Afootnote{\textit{Unterstrichen}: post apertum apostema \textit{Phthisis}, et ut plurimum tandem \textit{Mors.}}}
\pend 
\newpage
%\pstart  [p. 231] 19. \textit{Nimia Condensatio, Refrigeratioque Sanguinis}, per Pulmones delati \textit{curabitur}, mutando Aerem asperiorem, gelidumque \edtext{calidiore atque}{\lemma{}\Afootnote{\textit{Leibniz unterstreicht gestrichelt}: calidiore atque}} tranquillo et accendendo [p. 232] in [...] \pend 
\pstart  [p. 252] 14. Et \textit{Humida} quidem\edtext{}{\lemma{}\Afootnote{\textit{Unterstrichen}: 14. Et \textit{Humida} quidem}} \textit{Tussis Causa} mul[p. 253]tiplex
 observatur: Alias enim quaedam \textit{forinsecus advenientia}, vel \textit{ore assumpta}, perperamque in Asperam Arteriam delata Tussim mox excitant molestam: Alias \textit{Humores a}\edtext{}{\lemma{}\Afootnote{\textit{Am Rand}: Cura p.}} \textit{Capite delabentes} [...]
\pend%
\pstart%
[p. 267] CAP. XXIV. / \edtext{\textit{De Pulmonum Nutritione laesa.}}{\lemma{}\Afootnote{\textit{Unterstrichen}: \textit{De Pulmonum Nutritione laesa.}}} / I. HActenus cum aliis Motum Sanguinis Circularem agnoscentibus existimavi non tantum [...]
\pend%
\pstart%
[p. 282] 6, \textit{Primariae} in Carpo explorati \textit{Pulsus Differentiae} ad \textit{tria summa} reduci possunt \textit{Capita,} \edtext{\textit{Pulsus Robur, Magnitudinem ac Frequentiam. Celeritas}}{\lemma{}\Afootnote{\protect\rule[-2.9mm]{0mm}{0mm}\textit{Unterstrichen}: \textit{Pulsus Robur, Magnitudinem ac Frequentiam. Celeritas}}} enim Pulsui adscripta \edtext{Mente quidem concipi potest, non item Digitis tangi ac percipi: \textit{Durities}}{\lemma{}\Afootnote{\textit{Unterstrichen}: Mente quidem concipi potest, non item Digitis tangi ac percipi: \textit{Durities}}} autem non \edtext{nisi raro}{\lemma{}\Afootnote{\textit{Unterstrichen}: nisi raro}} in Pulsu reperitur, ac semper in Statu Praeternaturali\edtext{}{\lemma{}\Afootnote{\textit{Unterstrichen}: semper in Statu Praeternaturali}}; cum modo dictae in Naturali quoque ac Non-naturali observentur.
\pend%
\pstart%
[p. 703] 35. \edtext{Nimius}{\lemma{}\Afootnote{\textit{Doppelt unterstrichen}: Nimius}} menstruorum fluxus curabitur minuendo sanguinis serum abundans per hydragoga satis nota.
\pend%
\pstart%
[p. 704] 41. Citius\edtext{}{\lemma{}\Afootnote{\textit{Doppelt unterstrichen}: Citius}} recurrens fluxus menstruus ob sanguinis abundantiam, curabitur [...]
\pend%
\pstart%
[p. 705] 43. Tardius\edtext{}{\lemma{}\Afootnote{\textit{Doppelt unterstrichen}: Tardius}} confluentes menses ob sanguinis penuriam curant [...]%
\pend%
\count\Bfootins=1500
\count\Cfootins=1500
\count\Afootins=1500
%%%%% Hier endet das Stück.