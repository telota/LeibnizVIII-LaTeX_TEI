\pstart 
[161 r\textsuperscript{o}] \edtext{Des\edlabel{desespecesstart}}{{\xxref{desespecesstart}{desespecesend}}\lemma{Des}\Bfootnote{Von Des especes bis Kromhouten vgl. \textsc{N. Witsen}, \cite{00153}a.a.O., S.~178.}} especes du bois\protect\index{Sachverzeichnis}{bois}, remarques tir\'{e}es du livre de Mons. Witsen\protect\index{Namensregister}{\textso{Witsen,} Nicolaes 1641\textendash 1717} de \edtext{la Navigation}{\lemma{de}\Afootnote{ \textit{ (1) }\ l' a \textit{ (2) }\ l' Architecture Navale \textit{ (3) }\ la Navigation \textit{ L}}}. En ces pays c'est quercus\protect\index{Sachverzeichnis}{quercus}, chesne\protect\index{Sachverzeichnis}{chêne}\footnote{\textit{Anmerkung zwischen den Zeilen, beginnend unter dem Wort} chesne \textit{bis unter} vaisseaux: le chene \`{a} cause de sa fermet\'{e} dicht, ne boit pas d'eau.} \textit{Eick}\protect\index{Sachverzeichnis}{eik} dont on bâtit les vaisseaux\protect\index{Sachverzeichnis}{vaisseau}. Ces chenes\protect\index{Sachverzeichnis}{chêne} viennent \edtext{du}{\lemma{viennent}\Afootnote{ \textit{ (1) }\ de \textit{ (2) }\ du \textit{ L}}} Rhin\protect\index{Ortsregister}{Rhein (Rhin)} et de la Westphalie\protect\index{Ortsregister}{Westfalen (Westphalie)}. Westphalische \edlabel{desespecesend}\textit{Kromhouten}\protect\index{Sachverzeichnis}{kromhout}, \edtext{\textit{en}\edlabel{rhijnschestart}}{{\xxref{rhijnschestart}{rhijnscheend}}\lemma{en}\Bfootnote{Von en Rhijnsche bis binnewaerts verbezigen vgl. \textsc{N. Witsen}, \cite{00153}a.a.O., S.~179.}} \textit{Rhijnsche recht-houten\protect\index{Sachverzeichnis}{recht\textendash hout} prijst men zeer. Van Greenen-hout}\protect\index{Sachverzeichnis}{grenehout}, on bâtit aussi des vaisseaux\protect\index{Sachverzeichnis}{vaisseau} \edtext{plus [legers]\edtext{}{\Afootnote{legeres\textit{\ L \"{a}ndert Hrsg. } }}, mais}{\lemma{vaisseaux}\Afootnote{ \textit{ (1) }\ mai \textit{ (2) }\ plus [legers], mais \textit{ L}}} plus foibles, c'est pourqouy on s'en sert rarement pour les vaisseuax de guerre\protect\index{Sachverzeichnis}{vaisseau!de guerre}, qui doiuuent resister \`{a} une grande force. \textit{Tot boven en binnenwerck}\protect\index{Sachverzeichnis}{binnenwerk} (au dessus, et au dedans) qui souffrent peu, ce bois\protect\index{Sachverzeichnis}{bois} est fort propre de même que pour les vaisseaux de charge\protect\index{Sachverzeichnis}{vaisseau!de charge}, car ce \textit{greenen hout}\protect\index{Sachverzeichnis}{grenehout} est leger. Ce bois\protect\index{Sachverzeichnis}{bois} vient de Norwegae\protect\index{Ortsregister}{Norwegen@Norwegen (Norwegae, Norw\`{e}ge, Noorwegen)} \textit{en oosten} (de la mer Balthique\protect\index{Ortsregister}{Ostsee (Mer Balthique)}), \textit{gelijck oock het vueren\protect\index{Sachverzeichnis}{vurehout} 't geen vast van een aert met het greenen\protect\index{Sachverzeichnis}{grenehout} is dog eer} lighter \textit{en} brosser \textit{als} schwaerder. Il est difficile de les flechir (courber, buygen) en vallen spintig. Les masts\protect\index{Sachverzeichnis}{m\`{a}t} de Norw\`{e}ge\protect\index{Ortsregister}{Norwegen@Norwegen (Norwegae, Norw\`{e}ge, Noorwegen)} et de Moscovie\protect\index{Ortsregister}{Moskau (Moscovie)} sont les meilleurs et les plus en usage. \edtext{Koninksbergen\protect\index{Ortsregister}{Konigsberg@K\"{o}nigsberg (Koninksbergen)}}{\lemma{usage.}\Afootnote{ \textit{ (1) }\ Konings \textit{ (2) }\ Koninks \textit{ L}}} nous donne \textit{de beste plancken zoo green als }\textit{eicken}\protect\index{Sachverzeichnis}{eik}\textit{, doch de }\textit{vueren}\protect\index{Sachverzeichnis}{vuren}\textit{ Noorwegen\protect\index{Ortsregister}{Norwegen@Norwegen (Norwegae, Norw\`{e}ge, Noorwegen)}. In 't bowen en hout-Kloven dient goede acht geslagen te werden op den draet van het }\textit{hout}\protect\index{Sachverzeichnis}{hout}\textit{ hoe meerder men met den draet arbeit en hoe minder men zaegt, hy starker het werck zyn zall: geliik mede het droge }\textit{hout}\protect\index{Sachverzeichnis}{hout}\textit{ te kiezen is voor 't natte, want dit dicht, en dat het splijten onderworpen is}. \edtext{Le}{\lemma{\edtext{is}.}\Afootnote{ \textit{ (1) }\ By W \textit{ (2) }\ Le \textit{ L}}} bois\protect\index{Sachverzeichnis}{bois} coup\'{e} en hyver, lorsque la feuille est tomb\'{e}e, \textit{en den boom} \edtext{\textit{geslooten}}{\lemma{\textit{boom}}\Afootnote{ \textit{ (1) }\ gesloten \textit{ (2) }\ geslooten \textit{ L}}} is (cet l'arbre [ferm\'{e}]\edtext{}{\Afootnote{ferm\'{e}e\textit{\ L \"{a}ndert Hrsg.}}}) \edtext{est le plus fort}{\lemma{[ferm\'{e}])}\Afootnote{ \textit{ (1) }\ het sterckste \textit{ (2) }\ felt \textit{ (3) }\ velt \textit{ (4) }\ est le plus fort \textit{ L}}}. Avant que de couper les arbres\protect\index{Sachverzeichnis}{couper les arbres}, il est bon qu'on les perce 4 ou 5. jours auparavant, par croix, afin que le suc vivant \edtext{de vere}{\lemma{}\Afootnote{de vere \textit{ erg.} \textit{ L}}} sorte, et le bois\protect\index{Sachverzeichnis}{bois} se seche. 
\pend 
\pstart Les puys secs \edtext{produisent}{\lemma{}\Afootnote{produisent \textit{ erg.} \textit{ L}}} des arbres courts, serrez (engedrongen) et fermes. Un fond humide produit \edtext{\textit{hooge}, voose \textit{en} liivige \textit{stammen}}{\lemma{produit}\Afootnote{ \textit{ (1) }\ des arbres (stammen) \textit{ (2) }\ \textit{hooge}, voose \textit{en} liivige \textit{stammen} \textit{ L}}}, \textit{doch} bross. \textit{Nat }\textit{hout}\protect\index{Sachverzeichnis}{hout}\textit{ is de worm zeer onderworpen, los, en valt eerder tot stoffe t'zaem als het drooge. Waer de reden van schijnt te zijn, dat als het van zyne waterige deelen ontbloot is, veel lediges van binnen heft, waerom ten zeer genegen om 't zamen te vallen}. Arbres au contraire ne\'{e}s dans les pays chauds et secs sont fermes et ne contiennent que du bois\protect\index{Sachverzeichnis}{bois}. Le soleil en tire l'eau pendant même que les arbres sont encor en vie; ainsi les places virides sont remplies de bois\protect\index{Sachverzeichnis}{bois} même. 
\pend 
\pstart \textit{T' }\textit{hoogduytsche Berghout}\protect\index{Sachverzeichnis}{hoogduytsche Berghout} \textit{is vaster as het moeraszig breems hout}\protect\index{Sachverzeichnis}{breemshout}\textit{. Nog stercker is }\textit{Brasilien hout}\protect\index{Sachverzeichnis}{Brasilien\textendash hout}\textit{, }\textit{Campesche hout}\protect\index{Sachverzeichnis}{Campesche hout}\textit{, }\textit{Ebbenhout}\protect\index{Sachverzeichnis}{ebbenhout}\textit{, Sackrendaen-hout}\protect\index{Sachverzeichnis}{sackrendaen\textendash hout}, \edtext{\`{a} cause de ces Pays chauds et secs}{\lemma{\textit{Sackrendaen-hout},}\Afootnote{ \textit{ (1) }\ om dat het voort komt \textit{ (2) }\ \`{a} [...] secs \textit{ L}}}. Mais ces bois\protect\index{Sachverzeichnis}{bois} \`{a} cause de leur pesanteur ne sont pas propres au bastiment des vaisseaux\protect\index{Sachverzeichnis}{bâtiment des vaisseaux}, quoyque les portugais ayent fabriqu\'{e} des vaisseaux\protect\index{Sachverzeichnis}{vaisseau} du bois de \edtext{Bresil}{\lemma{de}\Afootnote{ \textit{ (1) }\ Besil \textit{ (2) }\ Bresil \textit{ L}}}, dont les planches ont est\'{e} fort mincement \edtext{taill\'{e}es}{\lemma{mincement}\Afootnote{ \textit{ (1) }\ taill\'{e}s \textit{ (2) }\ taill\'{e}es \textit{ L}}} (ges\"{a}get) pour obtenir la legeret\'{e}. Et comme ce bois\protect\index{Sachverzeichnis}{bois} estoit mal propre \`{a} estre courb\'{e}, le Kromhout\protect\index{Sachverzeichnis}{kromhout} dans le vaisseau\protect\index{Sachverzeichnis}{vaisseau} \edtext{estoit}{\lemma{vaisseau}\Afootnote{ \textit{ (1) }\ et \textit{ (2) }\ estoit \textit{ L}}} compos\'{e} de quantit\'{e} de petites pieces. Mais ces choses affoiblissoient fort les vaisseaux\protect\index{Sachverzeichnis}{vaisseau}; outre que les bois\protect\index{Sachverzeichnis}{bois} trop solides, zeer light scheuren en barsten (crevent et le fendent aisement). Les instruments dont les artisans se servoient pour tailler ces bois\protect\index{Sachverzeichnis}{bois}, sont bien plus \edtext{grands}{\lemma{plus}\Afootnote{ \textit{ (1) }\ forts massi \textit{ (2) }\ grands \textit{ L}}}, et aigus, que ceux des ces pays cy. Ce bois\protect\index{Sachverzeichnis}{bois} dont on fabrique un vaisseau\protect\index{Sachverzeichnis}{vaisseau} doit estre dej\`{a} sec, car s'il seche pendant \edtext{quand}{\lemma{pendant}\Afootnote{ \textit{ (1) }\ qu'on est en mer, \textit{ (2) }\ quand \textit{ L}}} \edtext{le}{\lemma{quand}\Afootnote{ \textit{ (1) }\ il e \textit{ (2) }\ le \textit{ L}}} vaisseau\protect\index{Sachverzeichnis}{vaisseau} est tout fait \edtext{il}{\lemma{fait}\Afootnote{ \textit{ (1) }\ ils \textit{ (2) }\ il \textit{ L}}} se jette et gaste tout. 
\pend 
\pstart Les Italiens cachent leur bois\protect\index{Sachverzeichnis}{bois} longtemps sous l'eau, avant que de l'employer au bastiment, pretendant que cela le rend taei en sterck.
\pend 
\pstart Le bois de Biscaye\protect\index{Sachverzeichnis}{bois!de Biscaye} passe celuy du Norden fermet\'{e} et generalement le bois meridional\protect\index{Sachverzeichnis}{bois!meridianal} este meilleur que le septentrional\protect\index{Sachverzeichnis}{bois!septentrional}. Arbres qui croissent dans les vall\'{e}es et endroits couverts sont plus propres aux vaisseaux\protect\index{Sachverzeichnis}{vaisseau}, que ceux viennent en des lieux exposez au vent, \textit{want deesrechter en mingequast vallen, ook beter gesloten} zyn.\ \textendash\ C'est une regle asseur\'{e}e parmy les charpentiers, que lors qu' ils ont du bon et mauvais bois\protect\index{Sachverzeichnis}{bois}, pour \edtext{un vaisseau}{\lemma{pour}\Afootnote{ \textit{ (1) }\ l'eau \textit{ (2) }\ un vaisseau \textit{ L}}}, \textit{datze dan overshants het goet met het }\edtext{\textit{quaed}}{\lemma{\textit{het}}\Afootnote{ \textit{ (1) }\ \textit{quaet} \textit{ (2) }\ \textit{quaed} \textit{ L}}}\textit{ schicken, en het flechtste in't gemein binnewaerts verbezigen\edlabel{rhijnscheend}}.
\pend 
\pstart \edtext{\textso{\textit{Eipen}}\protect\index{Sachverzeichnis}{eipen hout}\edlabel{eipenstart}}{{\xxref{eipenstart}{eipenend}}\lemma{\textso{\textit{Eipen}}}\Bfootnote{Von Eipen bis Les anciens vgl. \textsc{N. Witsen}, \cite{00153}a.a.O., S.~180.}} \textit{en \textso{pockhout}\protect\index{Sachverzeichnis}{pokhout} is bequaem om scheeps-blocks en schijven van te maeken}. Inlands \textso{\textit{Bouken}} werd \edtext{\textit{voor}}{\lemma{werd}\Afootnote{ \textit{ (1) }\ vor \textit{ (2) }\ \textit{voor} \textit{ L}}} \textit{het inlants eiken\protect\index{Sachverzeichnis}{eik} gepresen}. 
\pend 
\pstart \textso{\textit{Engelshout}}\protect\index{Sachverzeichnis}{Engelshout} \textit{splintert weinig daerom zeer dienstig tot den Scheeps-bouw\protect\index{Sachverzeichnis}{scheepsbouw}, is ook stercker als het eick\protect\index{Sachverzeichnis}{eik}, dat in andere oorten valt. Hierom bevint men het hout\protect\index{Sachverzeichnis}{hout} aen hunne schepen}\protect\index{Sachverzeichnis}{schip} \edtext{\textit{de}}{\lemma{\textit{schepen}}\Afootnote{ \textit{ (1) }\ \textit{di} \textit{ (2) }\ \textit{de} \textit{ L}}}\textit{ elders, mogen zyn gemaeckt. Na het uyterlike moet men ordele van de innerlijke Kracht des }\textit{houts}\protect\index{Sachverzeichnis}{hout}\textit{, drooge en Knobbelige basten geven vast }\textit{hout}\protect\index{Sachverzeichnis}{hout}\textit{, van geliiken ingekrompen en harde }\edtext{\textit{vrucht}}{\lemma{\textit{harde}}\Afootnote{ \textit{ (1) }\ \textit{vrug} \textit{ (2) }\ \textit{vrucht} \textit{ L}}}\textit{. }
\pend
\pstart \textit{ Vlezige }\textit{\textso{Appel}}\textit{ peer of eeniges ander }\edtext{\textit{vrucht}}{\lemma{\textit{ander}}\Afootnote{ \textit{ (1) }\ \textit{vrught} \textit{ (2) }\ \textit{vrucht} \textit{ L}}}\textit{, geeft bros }\textit{hout}\protect\index{Sachverzeichnis}{hout}\textit{. Het }\textit{viige boomhout}\protect\index{Sachverzeichnis}{vijgeboomhout}\textit{ is zeer weeck en on bequaem tot te scheep bow}\protect\index{Sachverzeichnis}{scheepsbouw}. \textit{Het }\textit{vijge-boom}\protect\index{Sachverzeichnis}{vijgeboom}\textit{ is wel licht, maer haestig verdorven. Boomen die langzaem wassen en traeg opschieten brengen vast ein goet }\textit{hout}\protect\index{Sachverzeichnis}{hout}\textit{ voort. Von de }\textit{Eiken}\protect\index{Sachverzeichnis}{eik}\textit{ will men dat ze driehondert jaer konnen staen. In het Haegsche voorhout staet een Boom} (+ quercus\protect\index{Sachverzeichnis}{quercus} +) \textit{geplant by de eigen }\edtext{\textit{handt}}{\lemma{\textit{eigen}}\Afootnote{ \textit{ (1) }\ \textit{hant} \textit{ (2) }\ \textit{handt} \textit{ L}}}\textit{ van Keizer }\textit{Karel}\protect\index{Namensregister}{\textso{Kaiser Karl V.} (Keizer Karel), Dt. Reich 1519\textendash 1556 \textdagger 1558} (+ \edtext{quinto puto +)}{\lemma{(+}\Afootnote{ \textit{ (1) }\ sexto +) \textit{ (2) }\ 5 \textit{ (3) }\ quinto puto +) \textit{ L}}} \textit{die nog in volle wasdom is. }
\pend
 \pstart \textit{ }\textit{\textso{Els}}\protect\index{Sachverzeichnis}{els}\textit{ en }\textit{\textso{Linde}}\protect\index{Sachverzeichnis}{linde}\textit{ zyn haestig hoog, }\edtext{\textit{dock}}{\lemma{\textit{hoog,}}\Afootnote{ \textit{ (1) }\ \textit{dog} \textit{ (2) }\ \textit{dock} \textit{ L}}}\textit{ brengen lichte en brosse stoffe voort: maer }\textit{Els}\protect\index{Sachverzeichnis}{els}\textit{ wanneer het lange jaren onder water of aerde }\edtext{\textit{heeft}}{\lemma{\textit{aerde}}\Afootnote{ \textit{ (1) }\ \textit{heft} \textit{ (2) }\ \textit{heeft} \textit{ L}}}\textit{ gelegen, zal het een taeiheit en sterckte bekomen. Van alzulcke }\textit{Els}\protect\index{Sachverzeichnis}{els}\textit{ en Bouk zegt }\textit{Claudianus}\protect\index{Namensregister}{\textso{Claudianus,} Claudius 370?\textendash 404/5?}\textit{ dat zy zoo hart als marmer worden. Tot porto, wierd eertijds veel van dit }\textit{hout}\protect\index{Sachverzeichnis}{hout}\textit{ onder de aerde gevonden, en men maekte de afsniidinge tegen den vyand daer van, in 't beleg van }\textit{Ostia}\protect\index{Ortsregister}{Ostia}\textit{, ten tijde van }\textit{Paulus IV.}\protect\index{Namensregister}{\textso{Papst Paul IV.} 1555\textendash 1559}\textit{ }
 \pend 
 \pstart \textit{In 't bouwen staet een timmerman wel te letten op te keure van het }\textit{hout}\protect\index{Sachverzeichnis}{hout}\textit{; tot the }\textit{kiel}\protect\index{Sachverzeichnis}{kiel}\textit{ en imhout en kiest hy het beste }\textit{hout}\protect\index{Sachverzeichnis}{hout}\textit{; tot beelt en rinckelwerck voldoet het flechte, en alle het }\textit{binnen-werck}\protect\index{Sachverzeichnis}{binnenwerk}\textit{ vermag licht en slecht getimmert te zyn, alzo sulks }\edtext{\textit{de}}{\lemma{\textit{sulks}}\Afootnote{ \textit{ (1) }\ \textit{the} \textit{ (2) }\ \textit{de} \textit{ L}}}\textit{ stevigkeit van het }\textit{schip}\protect\index{Sachverzeichnis}{schip}\textit{ niet en hindert: dog spint en vierig }\textit{hout}\protect\index{Sachverzeichnis}{hout}\textit{ hy, overal weert}. 
 \pend 
 \pstart \textit{Wel dient gelet te zyn op het }\textit{hout}\protect\index{Sachverzeichnis}{hout}\textit{ daer men de }\textit{scheeps nagels}\protect\index{Sachverzeichnis}{scheepsnagel}\textit{ van maeckt: want door brosheit van die menig }\textit{schip}\protect\index{Sachverzeichnis}{schip}\textit{ te gronde gaet. Het droog en }\edtext{\textit{jong}}{\lemma{en}\Afootnote{ \textit{ (1) }\ \textit{so} \textit{ (2) }\ \textit{jong} \textit{ L}}}\textit{ }\textit{hout}\protect\index{Sachverzeichnis}{hout}\textit{ is hier toe het alder bequaemste, dees werden ons veel toegebragt uit oosten, }\textit{Yrland}\protect\index{Ortsregister}{Irland (Yrland)}\textit{ en Elders; pennen gedraeit uit Boom-quasten, welke zeer hart zijn, zoude mijns bedunckens sterck en bequaem hier toe zijn. Wel moet mede gelet worden, dat de gaeten, daer men te pennen in Komt te staen, met en scherpe boor gedraeit worden, en gelijk ront zijn, effen en niet schrompelig wanneer de gaten ruig en oneffen zyn }\edtext{\textit{drinckt}}{\lemma{\textit{zyn}}\Afootnote{ \textit{ (1) }\ \textit{dring} \textit{ (2) }\ \textit{drinckt} \textit{ L}}}\textit{ het water daer by in, en de springers (een slag van kleine beesjes) door eeten het }\textit{hout}\protect\index{Sachverzeichnis}{hout}\textit{ dies te }\edtext{\textit{lichter}}{\lemma{\textit{te}}\Afootnote{ \textit{ (1) }\ \textit{lighter} \textit{ (2) }\ \textit{lichter} \textit{ L}}}\textit{. Hierom is dat men dese gaeten altans te }\textit{schepe}\protect\index{Sachverzeichnis}{schip}\textit{ eer siet verrotten aen de zijde daer de boor het }\textit{hout}\protect\index{Sachverzeichnis}{hout}\textit{ sniidt tegen de draet, als booven of onder daer ze met de draet heeft gekerft. }
 \pend 
 \pstart \textit{ Als de }\textit{houte nagels}\protect\index{Sachverzeichnis}{houte nagels}\textit{ in het }\textit{ship}\protect\index{Sachverzeichnis}{schip}\textit{ geslagen zyn, dan werden daer kleine gaetjens boven in geboort waer pennen in geslagen worden, vastigheits halven; 't geen deutelen wert genaemt. Hoe meerder het }\textit{hout}\protect\index{Sachverzeichnis}{hout}\textit{ daer de onderhuit des }\textit{scips}\protect\index{Sachverzeichnis}{schip}\textit{ van gemaekt word het water wederstaet hoe heilzamer sulks voor t'geheele }\textit{ships}\protect\index{Sachverzeichnis}{schip}\textit{ lighaem is; hier wiert }\edtext{\textit{vormaels}}{\lemma{\textit{wiert}}\Afootnote{ \textit{ (1) }\ \textit{vormals} \textit{ (2) }\ \textit{vormaels} \textit{ L}}}\textit{ in }\textit{Italien}\protect\index{Ortsregister}{Italien}\textit{ zeker slag} (genus) \textit{van willig toe gebruickt, daer het }\textit{zoute water}\protect\index{Sachverzeichnis}{zoutwater}\footnote{\textit{Zwischen den Zeilen, oberhalb} zoute water: salsa aqua.}\textit{ naer hun meining niet door en drough, ook }\textit{Tammepijn}\protect\index{Sachverzeichnis}{Tammepijn}\textit{ daer de worm, om zijne bitterheit voor vliet. Heden smeert men by ons de }\textit{schepen}\protect\index{Sachverzeichnis}{schip}\textit{ so verre als men gist dat zy in't water zincken met zeker pap van }\edtext{\textit{harz}}{\lemma{\textit{van}}\Afootnote{ \textit{ (1) }\ \textit{hars} \textit{ (2) }\ \textit{harz} \textit{ L}}}\textit{, smeer en harpuis t'zaem geklenst}. Les anciens\edlabel{eipenend} se servoient d'un 