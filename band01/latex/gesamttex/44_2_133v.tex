[133 v\textsuperscript{o}] \textso{Exper. Florent. in Vacuo p. 113 sqq.} Una Mignatta \edtext{(hirudo\protect\index{Sachverzeichnis}{hirudo})}{\lemma{}\Afootnote{(hirudo\protect\index{Sachverzeichnis}{hirudo}) \textit{ erg.} \textit{ L}}} ultra horam in vacuo viva  sanaque mansit, quasi in aere. Idem fecit Lumaca, (Limax\protect\index{Sachverzeichnis}{limax} vel Cochlea\protect\index{Sachverzeichnis}{cochlea}) in qua  nulla plane differentia observata est. Duo Grilli per quartam horae partem in  vacuo mansere vivacissimi moventes se sed non saltantes. \selectlanguage{italian}\textit{Al entrar dell'aria spiccaron salti. Una farfalla }\selectlanguage{latin}(papilio\protect\index{Sachverzeichnis}{papilio})\edtext{}{\lemma{(papilio)}\Bfootnote{\cite{00143}\textsc{L. Magalotti}, a.a.O., S.~CXIV.}} sive a manibus inserentium  pressus sit, sive ab aeris absentia, certe aegre movit alas. Et paulo post  extractus obiit. Moscone, muscae\protect\index{Sachverzeichnis}{musca} grandiores scilicet, et, \selectlanguage{italian}\textit{quali volando  fanno ronzio per l'aria col frullar dell'ale.}\edtext{}{\lemma{\textit{dell'ale.}}\Bfootnote{\cite{00143}\textsc{L. Magalotti}, a.a.O., S.~CXIV.}}\selectlanguage{latin} Una clausa satis  fortiter se movit, sed vacuo facto jacuit ut mortua, et aere immisso licet  mox obiit. Lucertola in vacuo facile interiit. Nota ipsi  vacuum fecere non ut Boylius\protect\index{Namensregister}{\textso{Boyle} (Boylius, Boyl., Boyl), Robert 1627\textendash 1691}, attractione, sed lapsu argenti vivi\protect\index{Sachverzeichnis}{argentum!vivum}.  Ideo eorum vacuum totum momento factum est. Aviculae subito defecere pisces\protect\index{Sachverzeichnis}{piscis} cum aqua inclusi supini ascendebant in summam, conabantur  se invertere, sed non poterant vesicae\protect\index{Sachverzeichnis}{vesica piscis} in ipsis repertae inflatae. Piscium vesicae\protect\index{Sachverzeichnis}{vesica piscis} habent meatus ad aerem recipiendum et emittendum pisces\protect\index{Sachverzeichnis}{piscis} aerem reddunt non per aures, sed os.\pend \pstart \textso{Exper. Florent. de Conglaciatione an aqua se dilatet inter conglaciandum.} Ita Galilaeus\protect\index{Namensregister}{\textso{Galilei} (Galilaeus, Galileus), Galileo 1564\textendash 1642} argumento minuti ponderis et auctae molis. Experim. 1.\edtext{}{\lemma{1.}\Bfootnote{\cite{00143}\textsc{L. Magalotti}, a.a.O., S.~CXXXI--CXXXIII.}} aqua in vase argenteo  congelare incipiens, vas rupit, non quasi contraheret sese, ob vacui fugam\protect\index{Sachverzeichnis}{fuga vacui}  alioquin operculum introrsum recessisset cum extrorsum convexum fuerit  redditum. Et \edtext{planum,}{\lemma{}\Afootnote{planum, \textit{ erg.} \textit{ L}}} mirum est tantam fuisse vim [congelationis]\edtext{}{\Afootnote{congelatio\textit{\ L \"{a}ndert Hrsg. } }} non nisi superiore  velut velo. \textso{P. 134.}\edtext{}{\lemma{\textso{134.}}\Bfootnote{\cite{00143}\textsc{L. Magalotti}, a.a.O., S.~CXXXIV.}} conglaciationes artificiales egregie fiunt,  glacie sale aspersa, in qua vasa ponuntur. Dicitur p. 137.\edtext{}{\lemma{137.}\Bfootnote{\cite{00143}\textsc{L. Magalotti}, a.a.O., S.~CXXXVII.}} colla phiolarum  clausarum minus firma, per conglaciationem projecta fuisse ad altitudinem duorum triumve brachiorum. Notabile \edtext{aureus Globus}{\lemma{Notabile}\Afootnote{ \textit{ (1) }\ pila\protect\index{Sachverzeichnis}{pila|textit} \textit{ (2) }\ aureus Globus \textit{ L}}}  aqua plenus, cum aqua conglaciari inciperet, extendit se servata rotunditate, ob auri scilicet ductilitatem exper. 6. p. 139.\edtext{}{\lemma{139.}\Bfootnote{\cite{00143}\textsc{L. Magalotti}, a.a.O., S.~CXXXIX.}} Jam ut addisceretur p. 143. maxima dilatatio quam aqua accipit per frigus\protect\index{Sachverzeichnis}{frigus}, duae factae  sunt experientiae, altera per mensuram, altera per pondus. Per mensuram  in canna vitrea aequabilis quantum possibile erat crassitiei et ex una parte  clausa, aqua ad medium impleta posita in nive minutissime trita et cum sale\protect\index{Sachverzeichnis}{sal} incorporata compertum est, altitudinem aquae glaciatae ad non glaciatam  esse ut 9. ad 8. circiter. Sed quia haec aequabilitas non satis perfecta sumta est canna  da pistola. Utile est aquam non sale\protect\index{Sachverzeichnis}{sal} tantum sed et aqua ardente aspergi,  quae ut omnes norunt, mirifice fortificat virtutem glaciei in conglaciando.  Alia inita est via per pondus, in canna vitrea. \edtext{Ponderata}{\lemma{vitrea.}\Afootnote{ \textit{ (1) }\ Mensurata \textit{ (2) }\ Ponderata \textit{ L}}} enim aqua  quae intraret in cannam ultra immissam, simplicem, quae intraret  ultra immissam conglaciatam, deprehensum est bilance $\displaystyle \frac{1}{48}\rule[-4mm]{0mm}{10mm}$ grani monstrante, \edtext{pondus aquae initio intrantis, ad pondus}{\lemma{pondus}\Afootnote{ \textit{ (1) }\ hujus ad pondus \textit{ (2) }\ aquae glaci \textit{ (3) }\ aquae initio intrantis, ad pondus \textit{ L}}} postea  intrantis esse ut $\displaystyle28\frac{1}{19}$  quod parum differt a proportione 8 ad 9.  seu $\displaystyle 28\frac{1}{8}\rule[-4mm]{0mm}{10mm}$.\edtext{}{\lemma{$\protect\displaystyle28\protect\frac{1}{8}$.}\Bfootnote{\cite{00143}\textsc{L. Magalotti}, a.a.O., S.~CXXXXIII--CXXXXVI.}} Quae p. 147\textendash49 dicuntur notabilia sunt, circa progressum  conglaciationum, adhibita est ampulla vitrea diametro octavae \edtext{unae}{\lemma{}\Afootnote{unae \textit{ erg.} \textit{ L}}} brachii partis  cum \edtext{[collo]}{\lemma{}\Afootnote{collo \textit{ erg.} \textit{ Hrsg. }\ }} oblongo circiter brachii et dimidii subtili, et accurate divisi in gradus. Intus posuimus  aquam naturalem, et ascendere fecimus ad sextam circiter partem colli. \edtext{Ampulla postea}{\lemma{colli.}\Afootnote{ \textit{ (1) }\ Aqua po \textit{ (2) }\  Ampulla postea \textit{ L}}} in glaciem salitam immissa, subito ad contactum nonnihil \edtext{sed subito}{\lemma{}\Afootnote{sed subito \textit{ erg.} \textit{ L}}} aqua intumuit seu assurrexit, id ut postea ostendemus, ob vasis contractionem. \edtext{(+ Ergo}{\lemma{contractionem.}\Afootnote{ \textit{ (1) }\ Inde \textit{ (2) }\ (+ Ergo \textit{ L}}} sic observari possunt contractiones per frigus\protect\index{Sachverzeichnis}{frigus}, et extensiones per calorem\protect\index{Sachverzeichnis}{calor} etiam in duris +) postea  cum motu salis ordinato aequabilique ac media velocitate descendit ad certum usque  gradum atque ibi aliquandiu quievit, postea paulatim resurgere incepit motu tardissimo et apparenter aequabili, unde sine ulla proportionali acceleratione\protect\index{Sachverzeichnis}{acceleratio} erupit  subito in saltum furiosissimum oculo non mensurabilem, sed haec summa velocitas subito  desiit in alium motum, satis quidem velocem sed incomparabaliter minus praecedente, ita assurgere prosequens pervenit ad summitatem colli, et inde effluxit. Inter conglaciandum  postquam aqua forte illud frigus\protect\index{Sachverzeichnis}{frigus} percipere coeperat Bullae aereae copiose surgebant.  Aqua semper aut tota fuit fluida, aut tota agglaciata. Agglaciationis tempus brevissimum. [132 r\textsuperscript{o}]