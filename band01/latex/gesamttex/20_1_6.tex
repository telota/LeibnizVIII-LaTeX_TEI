      
               
                \begin{ledgroupsized}[r]{120mm}
                \footnotesize 
                \pstart                
                \noindent\textbf{\"{U}berlieferung:}   
                \pend
                \end{ledgroupsized}
            
              
                            \begin{ledgroupsized}[r]{114mm}
                            \footnotesize 
                            \pstart \parindent -6mm
                            \makebox[6mm][l]{\textit{LiH}}Unterstreichung und Marginalie in \textsc{J. Gregory}, \cite{00146}\textit{Optica promota}, London 1663: Leibn. Marg. 94. \pend
                            \end{ledgroupsized}
                %\normalsize
                \vspace*{5mm}
                \begin{ledgroup}
                \footnotesize 
                \pstart
            \noindent\footnotesize{\textbf{Datierungsgr\"{u}nde}: Die Marginalie bezieht sich auf ein Problem, das in einem inhaltlichen Zusammenhang mit N. 35 und N. 36 steht.}
                \pend
                \end{ledgroup}
            
                \vspace*{8mm}
                \pstart 
                \normalsize
            \selectlanguage{latin}[p.~6] Satis patet ex Opticis elementis, multa Catoptricae, et Dioptricae esse communia; forsan igitur, et in reflectionum\protect\index{Sachverzeichnis}{reflexio}, et in refractionum\protect\index{Sachverzeichnis}{refractio} mensuris, aliquid commune haerebit: Totum autem reflectionum\protect\index{Sachverzeichnis}{reflexio} mysterium, in sectionibus conicis latere compertum est; (ut deinceps patebit) forte igitur et refractionum\protect\index{Sachverzeichnis}{refractio} mensura illic latebit. Secundo non sit regularis reflectio\protect\index{Sachverzeichnis}{reflexio}, nisi superficies reflectionis\protect\index{Sachverzeichnis}{reflexio} sit sectio conica\footnote{\textit{Links oben am Rand}: imo et in alterioribus}\textsuperscript{,}\footnote{\textit{Leibniz unterstreicht}: sit sectio conica}; fortassis ergo nec regularis refractio\protect\index{Sachverzeichnis}{refractio}, nisi refractionis\protect\index{Sachverzeichnis}{refractio} superficies, sit sectio etiam conica.\selectlanguage{latin}\pend 