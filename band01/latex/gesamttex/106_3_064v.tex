[64 v\textsuperscript{o}] manifesta consideranti ratio est. Esto Polus  Arcticus\protect\index{Sachverzeichnis}{polus!arcticus} \textit{a} Antarcticus\protect\index{Sachverzeichnis}{polus!antarcticus} \textit{b} @@@ G R A F I K @@@% \begin{wrapfigure}{l}{0.4\textwidth}                    
                %\includegraphics[width=0.4\textwidth]{../images/De+longitudinum+determinatione+scheda+prima/LH035%2C15%2C06_064v/files/100058.gif}
                        %\caption{Bildbeschreibung}
                        %\end{wrapfigure}
                        %@ @ @ Dies ist eine Abstandszeile - fuer den Fall, dass mehrere figures hintereinander kommen, ohne dass dazwischen laengerer Text steht. Dies kann zu einer Fahlermeldung fuehren. @ @ @ \\
                     Acus magnetica\protect\index{Sachverzeichnis}{acus!magnetica} \textit{cd} sub linea aequinoctiali\protect\index{Sachverzeichnis}{linea!aequinoctialis} \edtext{\textit{ef}}{\lemma{}\Afootnote{\textit{ef} \textit{ erg.} \textit{ L}}} posita  cujus extremitas \textit{c} arcticum\protect\index{Sachverzeichnis}{polus!arcticus}, at \textit{d} antarcticum  polum \protect\index{Sachverzeichnis}{polus!antarcticus}respiciat. Manifestum est, cum aequalis  sit conatus\protect\index{Sachverzeichnis}{conatus} \textit{d} versus \textit{b} et \textit{c} versus \textit{a} acum\protect\index{Sachverzeichnis}{acus!magnetica}  sub \edtext{aequatore\protect\index{Sachverzeichnis}{aequator} \textit{ef}}{\lemma{sub}\Afootnote{ \textit{ (1) }\ linea \textit{ (2) }\ aequatore \textit{ef} \textit{ L}}} in aequilibrio\protect\index{Sachverzeichnis}{aequilibrium} ac proinde horizonti parallelam manere \edtext{ac proinde sub linea navigantibus, qui scilicet variis maeandris ut \textit{eix} lineam crebro secant, acum, ut a Lusitanis observatum est, perpetuo titubare}{\lemma{}\Afootnote{BITTE UEBERPRUEFEN!!! ac [...] ut\textit{eix}lineam [...] titubare \textit{ erg.} \textit{ L}}}. At cis lineam inter \textit{e} et \textit{a}  vel \textit{f} et \textit{a} seu  sub  parallelo\protect\index{Sachverzeichnis}{circulus parallelus} \textit{gh}  praevalebit utique polus\protect\index{Sachverzeichnis}{polus} propinquior \textit{a} ac  proinde illuc magis declinabit \edtext{acus}{\lemma{declinabit}\Afootnote{ \textit{ (1) }\ navis\protect\index{Sachverzeichnis}{navis|textit} \textit{ (2) }\ acus \textit{ L}}}  contraria trans Lineam ratio est.  Idem \edtext{Terrellae  seu Magnetis in Globum tornati  experimento confirmari potest}{\lemma{Idem}\Afootnote{ \textit{ (1) }\ experimento Terrellae magneticae\protect\index{Sachverzeichnis}{magnetica!terrella|textit} confirmari potest \textit{ (2) }\ Terrellae  seu  \textit{(a)}\ globi \textit{(b)}\ Magnetis [...] potest \textit{ L}}}, \edtext{cui acus\protect\index{Sachverzeichnis}{acus!magnetica}imposita eodem plane  modo}{\lemma{potest,}\Afootnote{ \textit{ (1) }\ ubi acus eodem  plane modo \textit{ (2) }\ cui [...] modo \textit{ L}}} se disponit.\pend \pstart Et scripsit mihi  aliquando R. P. Kircherus\protect\index{Namensregister}{\textso{Kircher} (Kircherus), Athanasius SJ 1602\textendash 1680} \edtext{}{\lemma{}\Afootnote{Kircherus  \textbar\ cui caeteris certe \textit{ gestr.}\ \textbar\ novissimis \textit{ L}}}  novissimis Patrum societatis in omnes Mundi  plagas navigationibus plane extra dubium  positam esse acus inclinatoriae\protect\index{Sachverzeichnis}{acus!inclinatoria} veritatem.\edtext{}{\lemma{veritatem.}\Bfootnote{\textit{LSB} II, 1 N. 23.}}  Quo posito ut ad certam \edtext{universalemque ab omni observatione coelesti et aeris injuria independentem}{\lemma{}\Afootnote{BITTE UEBERPRUEFEN!!! universalemque ab omni   \textbar\ observatione coelesti et \textit{ erg.}\ \textbar\  aeris injuria independentem \textit{ erg.} \textit{ L}}} Elevationis Poli\protect\index{Sachverzeichnis}{elevatio!poli} \edtext{investigationem}{\lemma{}\Afootnote{investigationem \textit{ erg.} \textit{ L}}}  per magnetem\protect\index{Sachverzeichnis}{magnes} \edtext{perveniatur}{\lemma{?LEMMA?:magnetem}\Afootnote{ \textit{ (1) }\ universaliter sine ulla coeli  observatione etiam in magnetis\protect\index{Sachverzeichnis}{magnes|textit} observationem \textit{ (2) }\ perveniatur \textit{ L}}}, duplex iniri via potest, altera Geometrica  altera Mechanica, \edtext{ambaeque}{\lemma{Mechanica,}\Afootnote{ \textit{ (1) }\ utraque \textit{ (2) }\ ambaeque \textit{ L}}} inter se et  experimentis sunt conjungendae. Quod Mechanicam  attinet, tornandus est globus ex magnete\protect\index{Sachverzeichnis}{magnes}  quantus optimus maximusque haberi potest, observandumque \edtext{quo in parallelo posita acus quo angulo}{\lemma{observandumque}\Afootnote{ \textit{ (1) }\ quos \textit{ (2) }\ quibus angulis \textit{ (3) }\ quo in parallelo posita acus  \textit{(a)}\ quibus angulis \textit{(b)}\ quo angulo \textit{ L}}} inclinetur. Credibile est similem  fore inclinationem acus\protect\index{Sachverzeichnis}{acus!inclinatoria} in tellure. \edtext{Via  Geometrica est}{\lemma{tellure.}\Afootnote{ \textit{ (1) }\ Quod Viam  Geometricam attinet \textit{ (2) }\ Via  Geometrica est \textit{ L}}}, ut progressum inclinationis\protect\index{Sachverzeichnis}{inclinatio} crescentis decrescentisque observemus, ejusque \edtext{in regulas reductae}{\lemma{}\Afootnote{in regulas reductae \textit{ erg.} \textit{ L}}}  tabulam si fieri potest ad minuta usque computatam  condamus.\pend \pstart Experimentis autem sumtis repertum  est, ipsum inclinationis\protect\index{Sachverzeichnis}{inclinatio} incrementum non esse  uniforme, sed continue crescens. Hinc injecta  mihi \textso{suspicio} est, \textso{inclinationes}\protect\index{Sachverzeichnis}{inclinatio}\textso{  esse sinubus proportionales.}\pend \pstart Notavi enim \edtext{plerosque naturae}{\lemma{?LEMMA?:enim}\Afootnote{ \textit{ (1) }\ naturam non angulos sed \textit{ (2) }\ plerosque naturae \textit{ L}}}  effectus qui \edtext{angulis mutatis}{\lemma{qui}\Afootnote{ \textit{ (1) }\ pro angulorum ea ratione \textit{ (2) }\ angulis mutatis \textit{ L}}} variantur,  non angulis sed sinubus \edtext{tangentibus, secantibusve}{\lemma{}\Afootnote{tangentibus, secantibusve \textit{ erg.} \textit{ L}}}, esse proportionales ita ictuum\protect\index{Sachverzeichnis}{ictus}  obliquorum quantitas, et corporis in plano inclinato  descendentis gravitas\protect\index{Sachverzeichnis}{gravitas} est ad gravitatem\protect\index{Sachverzeichnis}{gravitas} aut vim  recta ferientis aut descendentis, \edtext{in}{\lemma{descendentis,}\Afootnote{ \textit{ (1) }\ ut \textit{ (2) }\ in \textit{ L}}}  reciproca ratione secantis anguli inclinationis\protect\index{Sachverzeichnis}{inclinatio}  ad radium.\pend \pstart Refractiones\protect\index{Sachverzeichnis}{refractio} quoque non ab angulis,  sed sinubus pendere, nunc apud plerosque confirmatur.  Idem de pendulorum\protect\index{Sachverzeichnis}{pendulum} vibrationibus in confesso est, et  Illustris vir, Robertus Moraeus\protect\index{Namensregister}{\textso{Moray} (Moraeus), Robert 1608\textendash 1673} suspicatus est, etiam 