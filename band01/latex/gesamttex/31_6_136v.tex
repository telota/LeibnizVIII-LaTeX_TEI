
\pstart Mais comme il est \`{a} croire, qu'une \edlabel{pourend2}\edtext{}{\lemma{\textso{l'aimant.}}\xxref{pourend}{pourend2}
\Afootnote{ \textit{ (1) }\ Du reste \textit{ (2) }\ Mais \textit{ L}}}mesme raison suspend la liqueur purg\'{e}e\protect\index{Sachverzeichnis}{liqueur!purg\'{e}e}, et joint les placques, il est aussi vraysemblable, \edtext{d\'{e}ja par avance,}{\lemma{}\Afootnote{d\'{e}ja par avance, \textit{ erg.} \textit{ L}}} que \edtext{les liqueurs n'ont point de gl\"{u}e, non plus que les placques\edlabel{placquesstart}}{\lemma{que}\Afootnote{ \textit{ (1) }\ la liqueur \textit{ (2) }\  les placques estant sans gl\"{u}e, \textit{ (3) }\ les liqueurs  \textit{(a)}\ en ayant non plus \textit{(b)}\ n'ont [...] placques \textit{ L}}}.\footnote{\textit{In der rechten Spalte}: \#}
\pend 
\pstart \edtext{}{\lemma{}\xxref{placquesstart}{placquesend}\Afootnote{placques.  \textbar\ (3) \textit{ gestr.}\ \textbar\ \textso{Mons.} \textit{ L}}} \textso{Mons. }\edlabel{placquesend}\textso{des Cartes}\protect\index{Namensregister}{\textso{Descartes} (Cartesius, des Cartes, Cartes.), Ren\'{e} 1596\textendash 1650} ayant \edtext{crû pouuoir}{\lemma{}\Afootnote{crû pouuoir \textit{ erg.} \textit{ L}}} rendre raison de l'union ou fermet\'{e} des corps par le seul repos et contiguit\'{e} des parties, a conclu que la fluidit\'{e} consiste dans leur mouuement.\edtext{}{\lemma{mouuement.}\Bfootnote{\textsc{R. Descartes, }\cite{00035}\textit{Principia philosophiae}, Amsterdam 1644, S.~62f. (\textit{DO} VIII, 1, S.~71). }} J'ay remarqu\'{e} pourtant dans un petit Essay, imprim\'{e}, il y a deux ans, qu'il n'y a rien de plus fluide qu'un corps sans mouuement et sans effort\protect\index{Sachverzeichnis}{effort} (sine motu et conatu\protect\index{Sachverzeichnis}{conatus}) s'il y en a: et j'ay demontr\'{e} que la solidit\'{e} premiere\edtext{}{\lemma{premiere}\Bfootnote{\textsc{G. W. Leibniz, }\textit{Hypothesis physica nova} (\cite{00256}\textit{LSB} VI, 2, N.~40, S.~248).}} et pour ainsi dire radicalle \edtext{dans les corps}{\lemma{}\Afootnote{dans les corps \textit{ erg.} \textit{ L}}} vient d'un mouuement\protect\index{Sachverzeichnis}{mouvement!uniforme} ou effort\protect\index{Sachverzeichnis}{effort} uniforme \edtext{dans leur interieur}{\lemma{}\Afootnote{dans  \textit{ (1) }\ leurs parties \textit{ (2) }\ leur interieur \textit{ erg.} \textit{ L}}}, et qu'au contraire les parties des corps fluides\protect\index{Sachverzeichnis}{corps!fluide} peuuent avoir un mouuement troubl\'{e}\edtext{}{\lemma{}\Afootnote{troubl\'{e}  \textbar\ comme Mons. Boyle\protect\index{Namensregister}{\textso{Boyle} (Boylius, Boyl., Boyl), Robert 1627\textendash 1691} a montr\'{e} par des experiences \textit{ erg. u. gestr.}\ \textbar\ . \textso{Le }\textit{ L}}}. \textso{Le mouuement uniforme}\protect\index{Sachverzeichnis}{mouvement!uniforme} est dans lequel tous les points du corps qui est en mouuement gardent tousjours la même distance \`{a} tous les autres points du même corps: et par consequent les lignes du mouuement sont paralleles, les vîtesses\protect\index{Sachverzeichnis}{vitesse} proportionnelles, et les periodes sont syndromes, ou achev\'{e}s en même temps, \edtext{comme nous voyons qu'il arrive quand}{\lemma{temps}\Afootnote{ \textit{ (1) }\ si \textit{ (2) }\ , comme [...] quand \textit{ L}}} un globe solide tourne \`{a} l'entour de son axe, et \edtext{même}{\lemma{et}\Afootnote{ \textit{ (1) }\ quand \textit{ (2) }\ même \textit{ L}}} quelqu' \edtext{autre}{\lemma{}\Afootnote{autre \textit{ erg.} \textit{ L}}} mouuement qu'on luy donne.
[137 r\textsuperscript{o}] De sorte que ce \edtext{mouuement que nous voyons estre la}{\lemma{ce}\Afootnote{ \textit{ (1) }\ que nous voyons une \textit{ (2) }\ mouuement [...] la \textit{ L}}} propriet\'{e}, et même de la definition des corps solides\protect\index{Sachverzeichnis}{corps!solide}, quand on les pousse, ou quand on leur donne un mouuement; est en effect la cause de la solidit\'{e}, quand ils l'ont d\'{e}ja dans leur interieur. \edtext{Car supposons qu'un corps}{\lemma{interieur.}\Afootnote{ \textit{ (1) }\ Car tout le c \textit{ (2) }\ Car supposons qu'un corps \textit{ L}}} aye un tel mouuement\protect\index{Sachverzeichnis}{mouvement!uniforme} ou effort uniforme, il est manifeste, qu'il demeurera solide pendant qu'il retient ce mouuement\protect\index{Sachverzeichnis}{mouvement!uniforme} ou effort. Or il le retiendra jusques \`{a} ce que ce mouuement\protect\index{Sachverzeichnis}{mouvement!uniforme} soit surmont\'{e} par une \edtext{plus grande}{\lemma{}\Afootnote{plus grande \textit{ erg.} \textit{ L}}} force exterieure (puisque c'est une propriet\'{e} generalle de tous les mouuements) \edtext{il est}{\lemma{mouuements)}\Afootnote{ \textit{ (1) }\ et même de continuer s'il n'y a point de force plus grande en \textit{ (2) }\ il est \textit{ L}}} donc \textso{solide} \edtext{si ce mouuement ou effort est assez fort}{\lemma{}\Afootnote{si  \textit{ (1) }\ la force \textit{ (2) }\ ce mouuement ou effort est assez fort \textit{ erg.} \textit{ L}}} puisque c'est la definition d'un corps solide\protect\index{Sachverzeichnis}{corps!solide} que ses parties ne peuuent estre separ\'{e}es que par une force \edtext{exterieure}{\lemma{}\Afootnote{exterieure \textit{ erg.} \textit{ L}}} considerable. Il est vray que les parties qui sont en repos gardent aussi tousjours la même distance entre elles, mais elles n'ont pas le pouuoir de se conserver dans un tel estat \edtext{contre un choc exterieur}{\lemma{}\Afootnote{contre un choc exterieur \textit{ erg.} \textit{ L}}}, parce qu'un simple repos n'a point de resistance contre le moindre mouuement\edtext{}{\lemma{mouuement}\Afootnote{ \textbar\ (car un Repos est en effect un mouuement ou effort moindre de tout autre effort assignable, ayant besoin \textit{ erg. u.}\  \textit{ gestr.}\ \textbar\ ; quoyque \textit{ L}}}; quoyque Mons. des Cartes\protect\index{Namensregister}{\textso{Descartes} (Cartesius, des Cartes, Cartes.), Ren\'{e} 1596\textendash 1650} aye asseur\'{e} le contraire\edtext{}{\lemma{contraire}\Bfootnote{\textsc{R. Descartes}, \cite{00035}a.a.O., S.~60 (\textit{DO} VIII, 1, S.~68). }} par un Paralogisme qu'on n'attendroit pas d'un si grand homme; \edtext{so\^{u}tenant même que le Repos ne cede qu'\`{a} un mouuement d'une force infinie.}{\lemma{}\Afootnote{so\^{u}tenant [...] infinie. \textit{ erg.} \textit{ L}}} Je crois pourtant que tout philosophe sans prejug\'{e} demeurera icy d'accord avec moy. Car on peut expliquer le \textso{Repos} \edtext{\textso{(absolu)}}{\lemma{}\Afootnote{\textso{(absolu)} \textit{ erg.} \textit{ L}}} par un Effort moindre que tout autre effort assignable, et qui a besoin justement d'une eternit\'{e} pour faire sortir un corps qui a un tel effort, tant soit peu, de sa place: il faut donc que le Repos cede \`{a} tout autre effort, de quelque foiblesse qu'il puisse estre. \edtext{Que si nous entendons un Repos \textso{respectif,} comme est celuy des parties entre elles et dans leur}{\lemma{estre.}\Afootnote{ \textit{ (1) }\ Que si nous expliquons le Repos des parties dans leur \textit{ (2) }\ Que [...] leur \textit{ L}}} tout, sans se mettre en peine des autres corps hors du corps donn\'{e}, nous pourrons dire qu'un mouuement uniforme est un Repos, et que le \edtext{seul}{\lemma{seul}\Afootnote{\textit{ erg.} \textit{ L}}} mouuement troubl\'{e} est un mouuement, \edtext{et que par consequent}{\lemma{mouuement,}\Afootnote{ \textit{ (1) }\ et de cette fa\c{c}on je me \textit{ (2) }\ de sorte que je  \textit{(a)}\ me pourrois servir aussi de cette maniere de parler de M \textit{(b)}\  dire aussi que \textit{ (3) }\ et que par consequent \textit{ L}}} la fermet\'{e} vient du repos et la fluidit\'{e} du mouuement; cette explication estant conforme entierement avec ce que \textso{Mons. }\textso{des Cartes}\protect\index{Namensregister}{\textso{Descartes} (Cartesius, des Cartes, Cartes.), Ren\'{e} 1596\textendash 1650} avoit dit ailleurs de la Nature du Mouuement et du Repos; mais point du tout applicable \`{a} ce qu'il dit sur cette matiere. Car si le Repos est un mouuement\protect\index{Sachverzeichnis}{mouvement!uniforme} \edtext{ou effort}{\lemma{ou}\Afootnote{effort \textit{ erg.} \textit{ L}}} uniforme il ne sera pas d'une Resistance infinie ny \`{a} l'\'{e}preuue de tout autre choc, \`{a} moins que cet \edtext{effort uniforme}{\lemma{cet}\Afootnote{ \textit{ (1) }\ autre effort \textit{ (2) }\ effort uniforme \textit{ L}}} ne soit infini, capable de produire un mouuement par un intervalle assignable dans un instant.
\pend