\pend \pstart [p.~120] [...] talis figura humana arte laborari nequit, licet alioquin mente concipi immo et demonstrari possit: in eo igitur artificis industria posita est, quod superficiem sphaericam quantum fieri potest lentibus\protect\index{Sachverzeichnis}{lens} inducat, et is artifices inter primos in hac arte tulisse censendus est, qui perfectiorem sphaeram tornarit, abstersis etiam ex laeuigato virto minimis et omnem sensum fugientibus salebris\footnote{\textit{Am Rand angestrichen}: et is artifices [...] salebris}, quae minimam [p.~121] etiam aliquam asperitatem, vel inaequalitatem concilient; pro quo non modo lentis\protect\index{Sachverzeichnis}{lens} proplasma probe laboratum requiritur, verum etiam, idque praesertim, vltima laeuigationis perfectio, quam prae caeteris, vir praestantissimus Eustachius Diuinius\protect\index{Namensregister}{\textso{Divini} (Divinius), Eustachio 1610\textendash 1685}, arte singulari, quam nullus hucusque, saltem quod sciam, assequutus est, vitris inducere solet; haec enim vltima, vt vocant, politura, vltimam etiam perfectionem vitris conciliat,\footnote{\textit{Am Rand angestrichen}: verum etiam [...] vitris conciliat} qua fiat vt minimae salebrae a superficie vitri sphaerica abstergantur, ac proinde omnes radij paralleli illapsi in eodem foco\protect\index{Sachverzeichnis}{focus} physice colligantur; [...].