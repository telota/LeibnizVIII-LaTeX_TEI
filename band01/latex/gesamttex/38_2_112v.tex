
\pstart Si duo Corpora Elastica non comprimenda sed tendenda sint,\edlabel{qed112v2}
calculus ita inibitur: Eadem corpora \textit{ab} et \textit{cd} \edtext{in statu non tenso nunc}{\lemma{\textit{cd}}\Afootnote{ \textit{ (1) }\ ex statu naturali\protect\index{Sachverzeichnis}{status naturalis|textit} in quo \textit{ (2) }\ in statu  \textit{(a)}\ libero \textit{(b)}\ non tenso nunc \textit{ L}}} constituta, tendenda sunt eadem vi, ut pondere appenso aequali \textit{l} vel \textit{ll}. Ponamus corpus \textit{cd} a pondere appenso \textit{ll} ita tendi posse, ut cum antea occupavit spatium \textit{cd} quod est ut 1. nunc occupet spatium \textit{cm} quod est ut $\rule[-4mm]{0mm}{10mm}
                    \displaystyle\frac{3}{2}$. Pars ergo \textit{cf} \edtext{corpori}{\lemma{\textit{cf}}\Afootnote{ \textit{ (1) }\ parti \textit{ (2) }\ corpori \textit{ L}}} \textit{ab} aequalis implebit itidem \edtext{spatium \textit{cn} quod sit}{\lemma{itidem}\Afootnote{ \textit{ (1) }\ partem quae sit \textit{ (2) }\ spatium \textit{cn} quod sit \textit{ L}}} similiter ad prius \textit{cf} ut 3 ad 2 \edtext{et ad spatium a toto implendum ut 1. ad 2.}{\lemma{}\Afootnote{et [...] 2. \textit{ erg.} \textit{ L}}} \edtext{Nec nisi}{\lemma{2.}\Afootnote{ \textit{ (1) }\ Ergo \textit{ (2) }\ Nec nisi \textit{ L}}} vi ponderis\protect\index{Sachverzeichnis}{vis!ponderis} \textit{ll} dimidia tendetur, cum reliqua \edtext{pars ponderis}{\lemma{}\Afootnote{pars ponderis \textit{ erg.} \textit{ L}}} reliquam corporis \textit{cd} \edtext{nempe \textit{fd} in spatium proportionale}{\lemma{\textit{cd}}\Afootnote{ \textit{ (1) }\ dimidium  \textit{(a)}\ in idem spatium \textit{(b)}\ spatium proportionale nempe \textit{ (2) }\ nempe \textit{fd} in spatium proportionale \textit{ L}}} (hoc loco aequale) \textit{nm} tendat. Ergo corpus \edtext{minus \textit{ab} aequale parti \textit{cf} itidem dimidio pondere \textit{l}}{\lemma{corpus}\Afootnote{ \textit{ (1) }\ \textit{ab} \textit{ (2) }\ minus [...] \textit{l} \textit{ L}}} tendetur in spatium \textit{ap} aequale spatio \textit{cn} \edtext{in dimidium}{\lemma{\textit{cn}}\Afootnote{ \textit{ (1) }\ a ponderis seu \textit{ (2) }\ in dimidium \textit{ L}}} spatii \textit{cm} a toto \textit{cd} seu corpore majore occupandi.\edlabel{occu112v1}\pend 
                    \pstart \edtext{\edlabel{occu112v2}Quia vero dimidio tantum pondere \textit{l}}{\lemma{occupandi.}\xxref{occu112v1}{occu112v2}\Afootnote{ \textit{ (1) }\ At dimidio tantum pondere \textit{ (2) }\ Quia vero dimidio tantum pondere   \textbar\ \textit{l} \textit{ erg.}\ \textbar\  \textit{ L}}} tenditur ad dimidium \edtext{spatium corporis}{\lemma{dimidium}\Afootnote{ \textit{ (1) }\ spatii \textit{ (2) }\ spatium corporis \textit{ L}}} majoris, ideo integro tendetur ad duplum dimidii, seu aequale pro dimidio, substituenda in genere ratio minoris ad majus, divisa per se ipsam, ut tertia pars tripli seu triplum tertiae partis, id est semper idem. \edtext{Ita demonstrata}{\lemma{idem.}\Afootnote{ \textit{ (1) }\ Hinc demonstratio \textit{ (2) }\ Ita demonstrata \textit{ L}}} est \edtext{propositio \protect\pgrk{paradozot'ath}}{\lemma{propositio}\Afootnote{ \textit{ (1) }\ omnium quas \textit{ (2) }\ 
                    \selectlanguage{polutonikogreek}paradozot'ath \selectlanguage{latin}
                   % \protect\pgrk{paradozot'ath} 
                    \textit{ L}}} duo corpora homogenea tendibilia sed non tensa majus minusve ab eadem vi tensum iri in spatium idem.  
                    \pend 