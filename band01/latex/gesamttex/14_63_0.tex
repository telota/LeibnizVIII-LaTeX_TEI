      
               
                \begin{ledgroupsized}[r]{120mm}
                \footnotesize 
                \pstart                
                \noindent\textbf{\"{U}berlieferung:}   
                \pend
                \end{ledgroupsized}
            
              
                            \begin{ledgroupsized}[r]{114mm}
                            \footnotesize 
                            \pstart \parindent -6mm
                            \makebox[6mm][l]{\textit{LiH}}Marginalien, An- und Unterstreichungen in \textsc{H. Fabri}, \cite{00147}\textit{Synopsis optica}, London 1667. Die Marginalien, An- und Unterstreichungen auf den Seiten 5, 8, 25, 62, 65, 133, 153, 154, 155, 158 sowie in der Approbatio wurden in Tinte ausgef\"{u}hrt, alle anderen mit Bleistift. Die Seiten 153, 154 und 158 enthalten zus\"{a}tzlich An- und Unterstreichungen mit Bleistift. Geringe Textverluste an der Marginalie S.~153 durch Beschnitt.\pend
                            \end{ledgroupsized}
                %\normalsize
                \vspace*{5mm}
                \begin{ledgroup}
                \footnotesize 
                \pstart
            \noindent\footnotesize{\textbf{Datierungsgr\"{u}nde}: Leibniz zitiert in \textit{LSB} VII, 1 N. 8 einen Passus aus der \cite{00147}\textit{Synopsis optica} des Honor\'{e} Fabri, den er in seinem Handexemplar unterstrichen hat. Ein weiterer Verweis auf Fabris \cite{00147}\textit{Optik} findet sich in einer Marginalie zu Barrows \cite{00144}\textit{Lectiones opticae}. F\"{u}r beide ist eine Entstehungszeit im Fr\"{u}hjahr 1673 wahrscheinlich. Wir datieren unser St\"{u}ck daher auf Anfang 1673.}
                \pend
                \end{ledgroup}
            
                \vspace*{8mm}
                \pstart 
                \normalsize
            \selectlanguage{latin}\centering APPROBATIO \pend \vspace{1.0ex} \pstart D. \textit{Ioseph Costalta}\protect\index{Namensregister}{\textso{Costalta,} Joseph belegt 1621}\footnote{\textit{Leibniz unterstreicht mit Tinte}: Costalta}\textit{ Abbas} Cong. Cassinensis\selectlanguage{latin}