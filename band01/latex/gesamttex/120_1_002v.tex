[2 v\textsuperscript{o}] Si vitrum sit plano convexum et $\langle$pl$\rangle$anities objecto obversa, radii unientur ad distantiam \edtext{diametri.}{\lemma{distantiam}\Afootnote{ \textit{ (1) }\ semidiametri \textit{ (2) }\ diametri. \textit{ L}}}\pend \pstart Si utrinque convexum ad distantiam diametri vitrum concavum debet abesse longius a convexo quando pupilla \edtext{humor crystall.}{\lemma{}\Afootnote{humor crystall. \textit{ erg.} \textit{ L}}} magis tumida et sphaerica est per abundantiam humidi ut in juvenibus, contra in senibus.\pend \pstart Proportione longitudinis apertura crescere debet\edtext{. Radii vitri objectivi}{\lemma{debet}\Afootnote{ \textit{ (1) }\ tam vitri concavi quam convexi \textit{ (2) }\ si ocularis\protect\index{Sachverzeichnis}{ocular|textit} quin \textit{ (3) }\ . Radii vitri objectivi \textit{ L}}} paralleli incidentes ab axe optico\protect\index{Sachverzeichnis}{axis!opticus} remotiores tanto uniuntur citius quanto magis distant ab axe optico\protect\index{Sachverzeichnis}{axis!opticus} seu magis obliqui sunt. Si vitrum sit Hyperbolicum, aut sphaera bene elaborata potest major esse apertura. Pauciores enim radii inutiles, seu qui non ad idem punctum uniantur. \pend \pstart Si objectum sit valde illum malum ut stellae clarae debet apertura esse minor ut abscindantur radii inutiles. \pend \pstart Con una occhiata plus videre possumus concavo brevi. Hinc major etiam apertura vitri ocularis, si longior Tubus. \pend \pstart Vitrum objectivum minus convexum \edtext{seu sphaerae majoris}{\lemma{}\Afootnote{seu sphaerae majoris \textit{ erg.} \textit{ L}}} requirit concavum quoque minus minuens seu sphaerae majoris. Quia si convexitas convexi major, radii refracti prius uniuntur; et ita faciunt angulum majorem. \pend \pstart Jam quanto angulus minoris radiorum major tanto debet esse major concavitas, ut \edtext{nimis maturam}{\lemma{ut}\Afootnote{ \textit{ (1) }\ nimiam \textit{ (2) }\ nimis maturam \textit{ L}}} magnae concavitatis unionem, compenset, nimia magnae concavitatis dilatatur. \pend \pstart Si vitrum concavum nimis concavum repraesentat objectum clarius sed minus. \pend \pstart Si convexum ejusdem sphaerae requiret posita eadem longitudine tubi \edtext{aut lentem magis convexa }{\lemma{}\Afootnote{aut lentem magis convexa \textit{ erg.} \textit{ L}}} concavum acutius, signum erit elaborationis melioris vitri, quod augeat magnitudinem non obscuret. \pend \pstart Eundem effectum faciunt concavum unius unciae diametro una parte laboratum et concavum biunciale utrinque laboratum. Sed in convexis illud feret Tubum duorum hoc unius palmi, si pro unius palmi in diametro. \pend \pstart Sint duo tubi unus duorum alius quatuor palmorum. Si illo objectum discernes ad distantiam miliaris, hoc discernes ad distantiam duorum miliarium. Sed contra primo integram secundo dimidiam domum deteges, sed hoc circiter non exacte.\pend \pstart Eodem Tubo crescit eadem quasi proportione distantia objecti visi, et magnitudo defecturae. Modo apertura vitri non fiat nimis angusta.\pend \pstart Si objectum est vicinum debet produci, si longinquum contrahi Tubus quia radii ex dato puncto in superficiem venientes si vicinum est sunt minus paralleli seu obliquiores, ideo radii non aeque uniuntur. \edtext{P. 178, 231, 244.}{\lemma{}\Afootnote{P. 178, 231, 244. \textit{ erg.} \textit{ L}}}\edtext{}{\lemma{244.}\Bfootnote{\textsc{F. Lana}, \cite{00069}a.a.O., S.~178, S.~231, S.~244.}}\edtext{ Nam radii a vicino incidunt majori angulo et ideo longius uniuntur.}{\lemma{}\Afootnote{244. Nam [...] uniuntur. \textit{ erg.} \textit{ L}}}\pend \pstart Objecta longiora melius deteguntur vitro magis concavo seu magis acuto. Quia longinqua minorem habent angulum incidentiae\protect\index{Sachverzeichnis}{angulus!incidentiae} et proinde et refractionis\protect\index{Sachverzeichnis}{angulus!refractionis} et ideo opus est concavo quod divertendo magis faciat angulum majorem.\pend \pstart Pro videndis objectis vicinis apertura vitri concavi debet esse minor quam pro longinquis eodem posito tubo, quia pro vicino producendus tubus. Ergo \edtext{ angulus radiorum seu coni radiosi apex minor. Ergo et minore opus apertura.}{\lemma{Ergo}\Afootnote{ \textit{ (1) }\ angulus radiorum seu punctum mechanicum minus, ergo non opus majore apertu \textit{ (2) }\ angulus [...] apertura. \textit{ L}}} \pend \pstart Vitra objectiva diametri majoris requirunt lentem\protect\index{Sachverzeichnis}{lens} diametri majoris.\pend \pstart Lentes\protect\index{Sachverzeichnis}{lens} majorum sphaerarum repraesentant objectum clarius sed minus. \pend \pstart Si vitrum objectivum 10 palmorum ferat lentem \protect\index{Sachverzeichnis}{lens} 6\textsuperscript{tae} partis palmi perfecte valde laboratum habendum est. Et hoc facit objectum sexagies majus quam apparet oculo nudo. \pend \pstart Magnitudo apparens oculo nudo ad apparentem armato est ut diameter objectivi\protect\index{Sachverzeichnis}{objectivum} ad diametrum lentis\protect\index{Sachverzeichnis}{lens}, scilicet si objectum non sit longius diametro aut semidiametro convexitatis vitri seu cum tubus habet effectum microscopii\protect\index{Sachverzeichnis}{microscopium}.\pend \pstart Et ideo non debet crescere diameter lentium\protect\index{Sachverzeichnis}{lens} et objectivorum\protect\index{Sachverzeichnis}{objectivum} eadem proportione, nam manente eadem proportione manet magnitudo.\pend \pstart Objectivum\protect\index{Sachverzeichnis}{objectivum} vitrum quod facit duplo majus non facit tamen duplo obscurius, si apertura ejus fiat tanto major et claritas compensetur modo hoc semper fieri posset. Fieri autem non potest. Nonnihil tamen augeri potest quia apertura pendet a quantitate superficiei[,] superficies autem non crescunt in ratione diametrorum sed in quadrata ratione diametrorum.\pend \pstart Si apertura non potest augeri lucis causa lens\protect\index{Sachverzeichnis}{lens} augenda est, etsi sic imminuatur.\pend \pstart NB. Optimum est, diversos \edtext{}{\lemma{}\Afootnote{diversos \textbar\ solos \textit{ gestr.}\ \textbar\ aspectus \textit{ L}}}aspectus inter se conferre. Quod hactenus non observatum. Ita multa detegentur non circa magnitudines tantum et distantias, sed et figuras objectorum.\pend \pstart Quanto objectum est remotius tanto minus augetur ejus magnitudo per eundem Tubum. \pend \pstart Si vitrum fenestrae inspiciam distantia quinque passuum mox 10, non apparebit duplo minus quam ante, sed paulo minus quam ante. Ergo magnitudo non tantum pendet ex angulo 
incidentiae\protect\index{Sachverzeichnis}{angulus!incidentiae}.\pend \pstart Tubus meus 7 palmorum cum lente $\displaystyle\frac{1}{6}$\rule[-4mm]{0mm}{10mm} palmi ac \edtext{\textso{proinde}}{\lemma{\textso{proinde:}}\Afootnote{\textit{doppelt unterstrichen}}} objectum faciens 42 vicibus majus vix facit lineam majorem vicibus 5. \pend \pstart Lana\protect\index{Namensregister}{\textso{Lana,} Francesco 1631\textendash 1687} p. 197.\edtext{}{\lemma{p. 197.}\Bfootnote{\textsc{F. Lana}, \cite{00069}a.a.O., S.~197. }} initio non si puo determinare \textit{l'ingrandimento se non si deter\-mina la distanza.} \pend \pstart Data distantia poterimus cognoscere quanto sit auctius objectum practice hoc modo collo circellum in foco\protect\index{Sachverzeichnis}{focus} lentis\protect\index{Sachverzeichnis}{lens} versus objectum duobus filis per trans\-versum ductis parallelis sibi lentium\protect\index{Sachverzeichnis}{lens} distantibus ut objectum praecise in medio appareat. Noteturque diligenter distantia filorum post \edtext{sublatis vitris inspiciatur}{\lemma{post}\Afootnote{ \textit{ (1) }\ fiant \textit{ (2) }\ sublatis vitris inspiciatur \textit{ L}}} objectum per eundem tubum\edtext{}{\lemma{}\Afootnote{tubum \textbar\ noteturque \textit{ gestr.}\ \textbar\ per \textit{ L}}} per alium circulum loco lentis\protect\index{Sachverzeichnis}{lens} prope oculum positum exiguo foramine perforatum, et moveatur circellus filorum huc illuc donec objectum praecise se compareat in medio duorum filorum. Sufficiunt lentibus\protect\index{Sachverzeichnis}{lens} ocularibus\protect\index{Sachverzeichnis}{ocular} 18 gradus convexitatis. Humor cristallinus interiore parte magis convexus, imitandum in oculo.\pend \clearpage \pstart In \edtext{Tubo}{\lemma{In}\Afootnote{ \textit{ (1) }\ vitro \textit{ (2) }\ Tubo \textit{ L}}} quatuor lentium\protect\index{Sachverzeichnis}{lens} quodammodo \edtext{\edlabel{nedhamistart}eodem fere modo ut in invento Ned\-hami\protect\index{Namensregister}{\textso{Needham,} Walter 1631\textendash 1691}\edlabel{nedhamiend}}{{\xxref{nedhamistart}{nedhamiend}}\lemma{}\Afootnote{eodem [...] invento Nedhami\protect\index{Namensregister}{\textso{Needham,} Walter 1631\textendash 1691} \textit{ erg.} \textit{ L}}}\edtext{}{\lemma{Nedhami}\Bfootnote{Dieses Instrument wurde auch kurz vor Leibniz' Besuch bei der Royal Society vorgestellt, vgl. \cite{00154}\textit{BH} III, S. 69.}} faciunt duae lentes\protect\index{Sachverzeichnis}{lens} oculo vicinae \edtext{effectum}{\lemma{vicinae}\Afootnote{ \textit{ (1) }\ aliquid \textit{ (2) }\ effectum \textit{ L}}} microscopii\protect\index{Sachverzeichnis}{microscopium} augendo \edtext{species}{\lemma{augendo}\Afootnote{ \textit{ (1) }\ quod \textit{ (2) }\ species \textit{ L}}} acceptas a tertia lente\protect\index{Sachverzeichnis}{lens}. \pend \pstart Maculae aliarum lentium\protect\index{Sachverzeichnis}{lens} vitrorum non apparent, nisi in superficie ultima. \pend \pstart Lana\protect\index{Namensregister}{\textso{Lana,} Francesco 1631\textendash 1687} p. 204.\edtext{}{\lemma{p. 204.}\Bfootnote{\textsc{F. Lana}, \cite{00069}a.a.O., S.~204. }} vitrum objectivum \edtext{}{\lemma{}\Afootnote{objectivum \textbar\ duarum lentium\protect\index{Sachverzeichnis}{lens} \textit{ gestr.}\ \textbar\ duplex \textit{ L}}}duplex alterum ab altero non multo remotum adhibita vel una lente\protect\index{Sachverzeichnis}{lens}, vel tribus ut alias. Ita abbreviato Tubo eadem magnitudo et claritas (+ $\langle$--$\rangle$gone aucto major +) differentia diametrorum duorum objectivorum 4\textsuperscript{ta} aut 5\textsuperscript{ta} pars ut unum 5 palmorum alterum 4 majus versus objectum quanto magis distant objectum majus, si minus clarius.\pend 