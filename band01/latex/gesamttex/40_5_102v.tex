[102 v\textsuperscript{o}] Mirabitur aliquis \edtext{merito}{\lemma{}\Afootnote{merito \textit{ erg.} \textit{ L}}}, cur ex tot phaenomenis aeris gravitati\protect\index{Sachverzeichnis}{gravitas!aeris} ascriptis solum Baroscopium\protect\index{Sachverzeichnis}{baroscopium!Torricellianum} ei vere tribuatur,  hujus rei \edtext{rationem}{\lemma{rei}\Afootnote{ \textit{ (1) }\ causam \textit{ (2) }\ rationem \textit{ L}}} ut scrutemur 
\edtext{\edlabel{prof102v1}profundius.}{\lemma{profundius}\xxref{prof102v1}{tubusend}\Afootnote{ \textit{ (1) }\ , experimentis quibusdam novis paranda est via \textit{ (2) }\ . Sumatur  Tubus Torricellianus vasi stagnanti subjecto eximatur,  constat eum non ideo effluere. Prolongetur Tubus  \textit{(a)}\ id est vel alius Tubus utrinque \textit{(b)}\ Tubo alio utrinque aperto  ad orificium prioris apertum exacte accommodato,  necesse est  \textbar\ non \textit{ gestr.}\ \textbar\  suspensum in Tubo manere Mercurium\protect\index{Sachverzeichnis}{mercurius|textit},  inter duos aeris Tubi superiorem et inferiorem,  \textit{(aa)}\ quod erit \textit{(bb)}\ qui non minus quam Baroscopium\protect\index{Sachverzeichnis}{baroscopium!Torricellianum|textit} commune  aeris   \textbar\ externi \textit{ erg.}\ \textbar\  gravitate\protect\index{Sachverzeichnis}{gravitas|textit} variante nunc ascendet nunc descendet.  Ergo si Tubus Torricellianus\protect\index{Sachverzeichnis}{Tubus!Torricellianus|textit} non sit totus \textit{ (3) }\   Constat in Baroscopio\protect\index{Sachverzeichnis}{baroscopium!Torricellianum|textit} ordinario Tubum  \textit{(a)}\ aqua \textit{(b)}\ Mercurio\protect\index{Sachverzeichnis}{mercurius|textit}  repleri; at vero tentemus quid futurum sit, si Tubus  non sit Mercurio\protect\index{Sachverzeichnis}{mercurius|textit} plenus? Ajo futurum esse, ut Mercurius\protect\index{Sachverzeichnis}{mercurius|textit}  solito altius suspensus maneat \textit{ (4) }\ Considerandum est Mercurium e Tubo Torricelliano inverso  in vas subjectum descendentem duo superare, primo  pressionem atmosphaerae contraponderantis, deinde  repugnantiam aeris ad difformitatem, \textit{(a)}\ est enim \textit{(b)}\ seu seu ad tensionem intra Tubum, et compressionem compensantem extra Tubum. Nam aer \textit{(aa)}\ in Tubo dilatata \textit{(bb)}\ vel qui \textit{(cc)}\ enim \textit{(dd)}\ in Tubi capacitate relictus dilatatus seu tensus est, quod  \textbar\ ita \textit{ gestr.}\ \textbar\ sentiri potest, si foramen in vesica qua  obligari potuit aperiatur, aer enim externus irrumpet.  Ergo aer extra tubum erit tantundem compressus; et  quanto altior Tubus est, tanto major erit aeris  interni dilatatio, quippe majus spatium occupantis,  et proinde tanto major aeris externi compressio,  quippe occupantis spatium tanto minus. At cur ergo Mercurius non magis suspenditur in Tubo  altiore seu cur semper intra 30  circiter pollices consistit ejus altitudo? Dicam  quia  \textit{(aaa)}\ omnis ille conatus funiculi\protect\index{Sachverzeichnis}{funiculus|textit} \textit{(bbb)}\ Tubus  fuit  plenus, ergo quanto est altior, tanto etiam plus Mercurii ei infuit, et ex eo effluxit. Mercurius  autem qui effluxit in vas subjectum hanc difformitatem tensionis seu aeris \textit{(aaaa)}\ inclusi \textit{(bbbb)}\ in Tubo relicti  dilatationem procuravit. Duplicata Tubi capacitate  \textbar\ et altitudine \textit{ erg.}\ \textbar\ duplicatus etiam fuit Mercurius, ac proinde idem  evenire necesse est in Tubis altioribus et brevioribus;  quia repugnantiam aeris ad difformitatem; Mercurii in vas subjectum ex Tubo dilapsi  pondus vicit. Hinc vis Funiculi seu dilatationis  inclusi in Tubo aeris nihil agit  in Mercurium in Baroscopio pendentem, quia  a pondere Mercurii delapsi dilatantis destructa  victaque est. \textit{(aaaaa)}\ Sed tantum Mercurii\protect\index{Sachverzeichnis}{mercurius|textit} quantum  totius \textit{(bbbbb)}\ Et si vero \textit{(ccccc)}\ At vero tota  aeris gravi \textit{(ddddd)}\ Quare nec aeris externi  \textit{(aaaaa-a)}\ extra  compr \textit{(bbbbb-b)}\ tantum compressi quantum relictus in Tubo  est dilatatus, Elaterium  \textbar\ defluxu reliqui Mercurii procuratum, et \textit{ erg.}\ \textbar\ tanto majus quanto  Tubus est altior, agit in Mercurium residuum suspensum,  quia Mercurius jam delapsus huic Elaterio aequiponderat. \textit{(aaaaa-aa)}\ Sola ergo gravitas\protect\index{Sachverzeichnis}{gravitas|textit} massae\protect\index{Sachverzeichnis}{massa|textit} atmosphaerae\protect\index{Sachverzeichnis}{atmosphaera|textit}  in aere \textit{(bbbbb-bb)}\ Ergo non compressio sed gravitas, (quae rei pressae dilatataeque  \textbar\ in se \textit{ erg.}\ \textbar\  eadem manet) massae\protect\index{Sachverzeichnis}{massa|textit} aereae extra Tubum, Mercurio\protect\index{Sachverzeichnis}{mercurius|textit} suspenso  contraponderat, hinc eadem semper ejus altitudo,  quaecunque sit altitudo Tuborum. Hoc ut clarius  intelligatur Experimentum fiat in aqua  \textit{(aaaaa-aaa)}\ sed Elastica,  qualis calida \textit{(bbbbb-bbb)}\  seu quae comprimi potest, qualis  est calida, aut quae aerem habet immixtum \textit{ (5) }\ Esto Tubus Torricellianus\protect\index{Sachverzeichnis}{Tubus!Torricellianus|textit}   \textbar\ \textit{AB} \textit{ erg.}\ \textbar\  cum vase subjecto   \textbar\ C. \textit{ erg.}\ \textbar\  positus in  \textit{(a)}\ aqua  \textit{(aa)}\ ponderis \textit{(bb)}\ tantae altitudinis, ut Mercurium\protect\index{Sachverzeichnis}{mercurius|textit} elevet  ad  \textit{(aaa)}\ tantam \textit{(bbb)}\ eandem altitudinem quae est Baroscopii\protect\index{Sachverzeichnis}{baroscopium|textit}:  \textit{(aaaa)}\ necesse est \textit{(bbbb)}\ \textit{BE} \textit{(cccc)}\ Reliquus Mercurius\protect\index{Sachverzeichnis}{mercurius|textit} \textit{(b)}\  vase clauso \textit{D} aere ordinario  pleno \textit{ (6) }\ . Esto Tubus Torricellianus   \textbar\ Mercurio plenus \textit{ erg.}\ \textbar\  \textit{AB}   \textbar\ cum \textit{ erg.}\ \textbar\  vase subjecto \textit{C}   \textit{(a)}\ positus eodem mercurio\protect\index{Sachverzeichnis}{mercurius|textit} ple  \textit{(b)}\ eundem [...] clauso  \textbar\ ordinario \textit{ erg.}\ \textbar\  pleno. \textit{ L}}}\footnote{\textit{In der rechten Spalte ungestrichen:} NB. modus admittendi aeris, auferendique Experimentum item Tubi altioris.\\
\textit{Darunter}: Evacuetur omnino ex Tubo superiore, [\textit{Satz bricht ab}]}