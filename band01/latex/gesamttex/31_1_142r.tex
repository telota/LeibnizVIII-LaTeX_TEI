[142 r\textsuperscript{o}] pendant que la bulle estoit au dessous de \textit{B} et devient plus legere \edtext{que l'oppos\'{e}e}{\lemma{}\Afootnote{que l'oppos\'{e}e \textit{ erg.} \textit{ L}}} quand la bulle a pass\'{e} \textit{B}. Tout cela a bien besoin \edtext{d'\'{e}claircissement, \edlabel{maispeustart}}{\lemma{d'\'{e}claircissement,}\Afootnote{ \textit{ (1) }\ \textso{Mais nous} \textit{ (2) }\ \textso{mais une} \textit{ L}}}\edtext{}{\lemma{\textso{mais}}\xxref{maispeustart}{maispeuend}\Afootnote{[...] \textso{peu;} \textit{Markierung am Rand}}}
\textso{mais une}\textso{ experience assez ais\'{e}e nous deliurera de cette peine. Car si la liqueur tombe \`{a} cause de l'in\'{e}galit\'{e} de ces deux colomnes; il faut qu'elle ne tombe pas, si la bulle monte au milieu de la phiole par la ligne point\'{e}e; et pour cet effect on pourra }\edtext{\textso{planter un petit bâton  ou fil de fer presqu'au milieu de la phiole, \`{a} fin que la bulle s'y appuyant demeure au milieu en montant. Si la liqueur tombe neantmoins, on peut estre asseur\'{e}, que l'in\'{e}galit\'{e} des colomnes n'y fait rien. On peut faire aussi que tantost,  le petit bâton arrive au haut du tuyau, tantost, qu'il}}{\lemma{\textso{planter}}\Afootnote{ \textit{ (1) }\ \textso{des perches ou fils de }\textso{fer}\protect\index{Sachverzeichnis}{fer|textit}  \textbar\ \textso{presqu'} \textit{ erg.}\ \textbar\ \textso{au milieu de la phiole, \`{a} fin que la bulle } \textit{(a)}\ \textso{attach\'{e}e \`{a} cet} \textit{(b)}~\textso{s'y appuyant demeure au milieu en montant. Si la liqueur tombe neantmoins, on peut estre asseur\'{e}, que l'in\'{e}galit\'{e} des colomnes n'y fait rien. On peut faire aussi que tantost, la perche arrive jusqu' \`{a} haut du tuyau, tantost, qu'elle} \textit{ (2) }\ \textso{un} [...] \textso{qu'il} \textit{ L}}}\textso{ n'y arrive pas, et on pourroit m\^{e}me afficher \`{a} son bout une espece de chapeau, pour voir si la bulle s'arreste- roit la dedans, et si la liqueur ne tomberoit pas, sans que la bulle arrive \`{a} la superficie interieure du verre. De m\^{e}me si l'on feroit entrer deux bulles de deux costez, l'in\'{e}galit\'{e} pretend\"{u}e des colomnes cesseroit. Il faut observer aussi dans le }\textso{Mercure purg\'{e}e}\protect\index{Sachverzeichnis}{mercure!purg\'{e}}\textso{, si la bulle s'augmente tousjours en montant \`{a} proportion; cela estant plus }\edtext{\textso{observable la dedans}}{\lemma{\textso{observable}}\Afootnote{ \textit{ (1) }\ \textso{dans le }\textso{mercure}\protect\index{Sachverzeichnis}{mercure|textit} \textit{ (2) }\ \textso{la dedans} \textit{ L}}}\textso{ (pourveu que la bulle }\edtext{\textso{soit visible, ou}}{\lemma{}\Afootnote{\textso{soit visible, ou} \textit{ erg.} \textit{ L}}}\textso{ se trouue entre le }\textso{Mercure}\protect\index{Sachverzeichnis}{mercure}\textso{ et le verre) que dans l'eau, \`{a} cause du grand chemin qu'elle y a \`{a} faire }\edtext{\textso{, car tout cela nous pourroit donner quelque lumiere dans une question de cette importance, puisque l'union des corps sensibles, et apparemment la raison de la solidit\'{e} et de l'attachement d'une liqueur gel\'{e}e, depend de l\`{a}. Peut estre ne seroit il pas aussi inutile d'observer, si}}{\lemma{\textso{faire}}\Afootnote{ \textit{ (1) }\ \textso{dans le }\textso{mercure}\protect\index{Sachverzeichnis}{mercure|textit}\textso{. Peut estre ne sera-t-il pas aussi inutile d'observer, si} \textit{ (2) }\ \textso{, car} [...] \textso{importance,}  \textbar\ \textso{puisque [...] l\`{a}} \textit{ erg.}\ \textbar\ \textso{. Peut} [...] \textso{si} \textit{ L}}}\textso{ le tuyau estant perc\'{e} d'un trou fort petit, ou m\^{e}me de plusieurs, ne se vuide pas, dans le }\edtext{\textso{Recipient, ou hors du Recipient}}{\lemma{\textso{le}}\Afootnote{ \textit{ (1) }\ \textso{vuide}\protect\index{Sachverzeichnis}{vide|textit}\textso{, ou hors du }\textso{vuide}\protect\index{Sachverzeichnis}{vide|textit} \textit{ (2) }\ \textso{Recipient, ou hors du Recipient} \textit{ L}}}\textso{, avec la }\textso{liqueur purg\'{e}e}\protect\index{Sachverzeichnis}{liqueur!purg\'{e}e}\textso{ ou ordinaire. Item si le }\textso{siphon}\protect\index{Sachverzeichnis}{siphon}\textso{ \`{a} jambes in\'{e}gales cesseroit \`{a} couler  la joincture }\edtext{\textso{des jambes}}{\lemma{}\Afootnote{\textso{des jambes} \textit{ erg.} \textit{ L}}}\textso{ estant perc\'{e}e tant soit peu;} \edlabel{maispeuend}pour juger mieux des pores du verre pour le passage de la matiere\protect\index{Sachverzeichnis}{mati\`{e}re!subtile} plus subtile que l'air.
\pend
 \pstart Il semble que le choc d\'{e}tache la \edtext{liqueur}{\lemma{la}\Afootnote{ \textit{ (1) }\ matiere \textit{ (2) }\  liqueur \textit{ L}}} suspend\"{u}e, parce qu'il y fait naistre une petite bulle. Mais si la liqueur a est\'{e} longtemps en repos dans une m\^{e}me endroit, ce n'est pas merveille qu'elle se d\'{e}tache avec plus de difficult\'{e} \textso{ou} \edtext{s'estant inger\'{e}e}{\lemma{s'estant}\Afootnote{ \textit{ (1) }\ insinu\'{e}e \textit{ (2) }\ inger\'{e}e \textit{ L}}}, dans les sinuositez in\'{e}galles de la superficie interieure de l'endroit o\`{u} elle se trouue, \textso{ou} n'estant \edtext{plus assez promte}{\lemma{n'estant}\Afootnote{ \textit{ (1) }\ pas si propre comme a $\langle$--$\rangle$ \textit{ (2) }\  plus   \textbar\ assez \textit{ erg.}\ \textbar\  promte \textit{ L}}} \`{a} produire une bulle. Mais si l'un ou l'autre en est la cause, les experiences projett\'{e}es cy dessus nous \edtext{le}{\lemma{}\Afootnote{le \textit{ erg.} \textit{ L}}} feront s\c{c}avoir.\edlabel{142rstart}
 [142 v\textsuperscript{o}] \edlabel{142rend}\edtext{\textso{On}}{\lemma{sçavoir.}\xxref{142rstart}{142rend}\Afootnote{ \textit{ (1) }\ \textso{Cependant on} \textit{ (2) }\ \textso{On} \textit{ L}}}\textso{ examinera }\edtext{}{\lemma{\textso{On} [...] \textso{\'{e}clat.}}\xxref{142rend}{142v1}\Afootnote{\textit{Markierung am Rand}}}\edtext{\textso{aussi, la cause de l'\'{e}clat qui se fait quand la liqueur se d\'{e}tache ou partage,}}{\lemma{\textso{aussi,}}\Afootnote{ \textit{ (1) }\ \textso{ si } \textit{(a)}\ \textso{le }\textso{choc}\protect\index{Sachverzeichnis}{choc|textit} \textit{(b)}\ \textso{ l'\'{e}clat qui se fait quand la liqueur se d\'{e}tache ou partage, peut venir,} \textit{ (2) }\ \textso{d'o\`{u} l'\'{e}clat puisse venir qui se fait quand la liqueur se d\'{e}tache ou partage,} \textit{ (3) }\ \textso{la} [...] \textso{partage,} \textit{ L}}}\textso{ et s'il y a de la difference en cela, entre la }\textso{liqueur purg\'{e}e}\protect\index{Sachverzeichnis}{liqueur!purg\'{e}e}\textso{, et l'ordinaire, et si }\textso{deux placques}\protect\index{Sachverzeichnis}{deux placques}\textso{ se d\'{e}tachent avec \'{e}clat.}\edlabel{142v1}
\pend