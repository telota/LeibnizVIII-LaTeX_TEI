[100~r\textsuperscript{o}]  tamen suo ipsius pondere apud nos compressum  esse, et \edtext{hinc petendam multorum}{\lemma{hinc}\Afootnote{ \textit{ (1) }\ esse \textit{ (2) }\   \textbar~petendam \textit{ erg.}\ \textbar\  multorum \textit{ L}}} naturae effectuum causam, Boylius\protect\index{Namensregister}{\textso{Boyle} (Boylius, Boyl., Boyl), Robert 1627\textendash 1691} egregie patefecit. \edtext{Sed ajo tamen pressionem esse causam immediatam, quia experimentum reperi, (infra 2.) quo auferri potest in Tubo dilatatio, etsi maneat in corporum summa relicta tantum compressione, remanenteque nihilominus Mercurii\protect\index{Sachverzeichnis}{mercurius} elevatione, et puto aliud experimentum excogitari posse, quo tentetur an adempta compressione, relictaque tantum dilatatione, maneat Mercurii elevatio.}
{\lemma{}\Afootnote{Sed [...] potest  \textbar\ in Tubo \textit{ erg.}\ \textbar\  dilatatio,  \textbar\ etsi maneat in corporum summa \textit{ erg.}\ \textbar\  relicta  \textbar\ tantum \textit{ erg.}\ \textbar\  compressione, remanenteque
%\edtext{}{\lemma{}\Afootnote{remanenteque
  \textbar\ proinde \textit{ gestr.}\ \textbar\ nihilominus %\textit{ L }\ }}
 % nihilominus Mercurii\protect\index{Sachverzeichnis}{mercurius} elevatione, et puto aliud experimentum excogitari posse, quo tentetur an adempta compressione, relictaque tantum dilatatione, maneat
% \edtext{Mercurii}{\lemma{maneat}\Afootnote{ 
[...] maneat \textit{ (1) }\ aeris \textit{ (2) }\ Mercurii %\textit{ L }\ %}}
 elevatio. \textit{ erg.} \textit{ L}}}
  Cum enim manifestum  sit Mercurium\protect\index{Sachverzeichnis}{mercurius} Barometri\protect\index{Sachverzeichnis}{barometrum} etiam in aere clauso suspensum manere,  non potest dici eum sustineri a \edtext{gravitate massae}{\lemma{a}\Afootnote{ \textit{ (1) }\ massa\protect\index{Sachverzeichnis}{massa|textit} \textit{ (2) }\  gravitate massae \textit{ L}}} aereae, necesse est ergo eum sustineri ab aeris Elaterio\protect\index{Sachverzeichnis}{elaterium} comprimi ultra negantis. \edtext{Contra ubi comprimi potest,}{\lemma{negantis.}\Afootnote{ \textit{ (1) }\ Idem ostendit vesicas flaccidas  intumescere \textit{ (2) }\ Contra ubi comprimi potest, \textit{ L}}} \edtext{quod fit cum}{\lemma{potest,}\Afootnote{ \textit{ (1) }\ id est ubi  so \textit{ (2) }\ quod fit cum \textit{ L}}} \edtext{exucta  magna parte}{\lemma{cum}\Afootnote{ \textit{ (1) }\ exhausta magna parte \textit{ (2) }\ exucta  magna parte \textit{ L}}} solito dilatatior est, Baroscopii\protect\index{Sachverzeichnis}{baroscopium} effectus  cessat. Observatum enim est a Boylio\protect\index{Namensregister}{\textso{Boyle} (Boylius, Boyl., Boyl), Robert 1627\textendash 1691} proportione evacuationis Mercurium\protect\index{Sachverzeichnis}{mercurius}, aquamque ipsam descendisse, nec nisi  digitum Mercurii\protect\index{Sachverzeichnis}{mercurius} aut pedem aquae, ultra superficiem  stagnantis in fundo liquoris \edtext{eminuisse,}{\lemma{eminuisse,}\Bfootnote{\textsc{R. Boyle}, \cite{00015}a.a.O., S.~68 (\textit{BW} I, S.~192).}} quod scilicet indicio  fuit aerem \edtext{sensibilem}{\lemma{}\Afootnote{sensibilem \textit{ erg.} \textit{ L}}} nondum satis fuisse exhaustam. Nam Hugenius \protect\index{Namensregister}{\textso{Huygens} (Hugenius, Vgenius, Hugens, Huguens), Christiaan 1629\textendash 1695}  postea eo rem produxit, ut Mercurius\protect\index{Sachverzeichnis}{mercurius} aut aqua ex Tubo  penitus descenderent in vas subjectum. Hoc Experimento  plane demonstratum est nullam aliam esse suspensi,  in Tubo \edtext{ad certam tantum altitudinem}{\lemma{}\Afootnote{ad certam tantum altitudinem \textit{ erg.} \textit{ L}}}  Mercurii\protect\index{Sachverzeichnis}{mercurius} causam, quam disjunctive, vel Gravitatem aeris\protect\index{Sachverzeichnis}{gravitas!aeris} \edlabel{liberistart}
 % \edtext{
 liberi, vel Elaterem clausi. Cumque  aer sit a massa incumbente ad certum usque terminum compressus, et massae ultra compressurae resistat, hinc ei quoque resistet quod massae aereae aequiponderat, scilicet Mercurio ex determinata altitudine suspenso. Ultima ergo  ratio Baroscopii  massae aereae gravitas unice censenda est, quia Elaterium ejus determinatum a determinata massae gravitate  pendet.%
\pend \pstart%
Ita explosa eorum sententia est qui  corpus a corpore avelli posse negabant, nisi aliud sensibile novum intrare posset. Mercurium  enim avelli a summitate tubi manifestum est in Experimento Torricelliano, et in antlia embolus aquam in minimam altitudinem evectam tandem deserit.%
\pend \pstart %
Quidam  cum Mercurium a summo Tubo avelli viderent,  spatium intus relictum materia vi lapsus  ponderisque tensa, quae Mercurium longius descendere  non pateretur seu tendi ultra non posset,  plenum esse dixere hujus sententiae Hyperaspistem doctum Franciscus Linus egit.  Sed si rem  exacte consideres  apparet eodem recidere sententiam  eorum, qui aeris  extra Tubum pressioni, et  qui aeris in Tubo tensioni  hoc phaenomenon attribuunt, quia quantum  aer intus dilatatus%\rule[-3.5cm]{0cm}{1cm} 
tantum extra compressus. Et posset proinde ascribi effectus conatui pressionis universalis ad\edlabel{liberiend} \edtext{uniformitatem.}{\lemma{aeris}\xxref{liberistart}{liberiend}\Afootnote{ \textit{ (1) }\ vel Elaterem\protect\index{Sachverzeichnis}{elater|textit}. \textit{ (2) }\ liberi, vel Elaterem clausi.  \textit{(a)}\ Est \textit{(b)}\ Cumque \textbar\ enim \textit{ gestr.}\ \textbar\ aer [...] pendet. Ita [...] in \textit{(aa)}\ Baroscopio\protect\index{Sachverzeichnis}{baroscopium|textit} \textit{(bb)}\ Experimento [...] altitudinem evectam tandem deserit.  Quidam  cum Mercurium a summo  \textit{(aaa)}\ vase \textit{(bbb)}\ Tubo [...] Sed \textit{(aaaa)}\ cum  ex hoc funiculo\protect\index{Sachverzeichnis}{funiculus|textit} seq \textit{(bbbb)}\ si [...] compressus. \textit{(aaaaa)}\ Quia tamen ex \textit{(bbbbb)}\ Et [...] uniformitatem. \textit{ L}}}\pend
 \pstart  At circa alios Vacui Fugae\protect\index{Sachverzeichnis}{fuga vacui} ascriptos vulgo effectus nonnihil \edtext{turbatum est}{\lemma{nonnihil}\Afootnote{ \textit{ (1) }\ titubatum est \textit{ (2) }\ turbatum est \textit{ L}}}, observavit enim Boylius\protect\index{Namensregister}{\textso{Boyle} (Boylius, Boyl., Boyl), Robert 1627\textendash 1691} duas laminas politas\protect\index{Sachverzeichnis}{laminae politae} in exhausto Recipiente nihilominus  cohaesisse,\edtext{}{\lemma{cohaesisse,}\Bfootnote{\textsc{R. Boyle}, \cite{00015}a.a.O., S.~156f. (\textit{BW} I, S.~239).}} et Hugenius\protect\index{Namensregister}{\textso{Huygens} (Hugenius, Vgenius, Hugens, Huguens), Christiaan 1629\textendash 1695} idem vidit tribus licet libris  laminae inferiori\edtext{}{\lemma{appensis.}\Bfootnote{\textsc{Chr. Huygens}, \cite{00062}\textit{Extrait d'une lettre}, \textit{JS} (1672), S.~139 (\textit{HO} VII, S.~205f.).}} \edtext{appensis. Clarissimus Perierius qui Experimentum illud  memorabile in monte Arverniae (Le puis de domme)  sumsit, et postea scripta quaedam Pascalii posthuma  de hoc argumento publicavit,}{\lemma{appensis.}\Afootnote{ \textit{ (1) }\ Responsum est a \textit{ (2) }\ Clarissimus Perierius \textit{(a)}\ Pascalianarum disserta \textit{(b)}\ qui [...] monte  \textbar\ illo \textit{ gestr.}\ \textbar\ Arverniae [...] publicavit, \textit{ L}}} vidit difficultatem ex hoc Boylii\protect\index{Namensregister}{\textso{Boyle} (Boylius, Boyl., Boyl), Robert 1627\textendash 1691} experimento natam, responditque aerem  a Boylio\protect\index{Namensregister}{\textso{Boyle} (Boylius, Boyl., Boyl), Robert 1627\textendash 1691} non fuisse satis exhaustum, \edtext{nam}{\lemma{exhaustum,}\Afootnote{ \textit{ (1) }\ ac proinde \textit{ (2) }\ nam \textit{ L}}}  et Mercurium\protect\index{Sachverzeichnis}{mercurius} aquamque nondum plane descendisse, et  ideo nec \edtext{Tabulas politas}{\lemma{nec}\Afootnote{ \textit{ (1) }\ laminas \textit{ (2) }\ Tabulas politas \textit{ L}}} fuisse divulsas, si quis tamen explorare  potuisset manu, sensurum fuisse \edtext{exiguam}{\lemma{fuisse}\Afootnote{ \textit{ (1) }\ summam \textit{ (2) }\ exiguam \textit{ L}}} divellendarum  Tabularum difficultatem. Addit \edtext{aerem, residuum  ab alio circumjacente non pressum Elaterio suo}{\lemma{Addit}\Afootnote{ \textit{ (1) }\ Elaterium\protect\index{Sachverzeichnis}{elaterium|textit} aeris  minus a \textit{ (2) }\ aerem, [...] suo \textit{ L}}} se  dilatantem Tabulas velut compressisse. Sed \edtext{si}{\lemma{Sed}\Afootnote{ \textit{ (1) }\ ex quo \textit{ (2) }\ si \textit{ L}}} experimentum Hugenii\protect\index{Namensregister}{\textso{Huygens} (Hugenius, Vgenius, Hugens, Huguens), Christiaan 1629\textendash 1695} tunc vidisset Perierius\protect\index{Namensregister}{\textso{P\'{e}rier} (Perrier, Perier, Perierius), Florin 1605\textendash 1702} \edtext{agnovisset}{\lemma{Perierius}\Afootnote{ \textit{ (1) }\ quaerenda ei fuisset alia respondendi ratio \textit{ (2) }\ agnovisset \textit{ L}}}\edtext{}{\lemma{Perierius}\Bfootnote{Vermutlich ist folgende Stelle bei Pascal gemeint: \textsc{B. Pascal},  \cite{00081}a.a.O., S. 65\textendash67 (\textit{PO} III, S. 209f.).}}  credo Aeris pressionem \edtext{non sufficere}{\lemma{pressionem}\Afootnote{ \textit{ (1) }\ sustineri \textit{ (2) }\ non potui \textit{ (3) }\ in eo casu  non posse \textit{ (4) }\ non sufficere \textit{ L}}} ad phaenomeni explicationem.  Nam cum \edtext{ejus dilatati pressio Elateriumve tunc}{\lemma{cum}\Afootnote{ \textit{ (1) }\ ea \textit{ (2) }\ ejus dilatati pressio Elateriumve tunc \textit{ L}}} ne digito quidem imo ne lineae\edtext{}{\lemma{}\Afootnote{lineae  \textbar\ quidem \textit{ gestr.}\ \textbar\ Mercurii, \textit{ L}}} Mercurii\protect\index{Sachverzeichnis}{mercurius}, ac nec pollici aquae sustinendae \edtext{sufficiat, constat}{\lemma{sufficiat,}\Afootnote{ \textit{ (1) }\ quomodo \textit{ (2) }\ constat \textit{ L}}} enim omnem penitus aquam et tanto  magis Mercurium\protect\index{Sachverzeichnis}{mercurius} \edtext{in Recipiente bene nunc evacuato}{\lemma{}\Afootnote{in Recipiente bene nunc evacuato \textit{ erg.} \textit{ L}}} ex Tubo Torricelliano\protect\index{Sachverzeichnis}{Tubus!Torricellianus} descendere  in vas subjectum, quomodo sustineret laminam tribus  libris appensis gravatam? Sustinetur tamen Lamina  cum pondere suo, necesse est ergo \edtext{ab aeris pressione differentem}{\lemma{ergo}\Afootnote{ \textit{ (1) }\ aliam \textit{ (2) }\ ab aeris pressione differentem \textit{ L}}}  sustentationis esse causam. Insigni plane documento  quam non sit  festinandam in regulis generalibus ex particularibus experimentis  condendis. Cum post tot plausibilia argumenta experimentaque ipsum subtilissimum Pascalium\protect\index{Namensregister}{\textso{Pascal} (Pascalius), Blaise 1623\textendash 1662} et quotquot praematurius in ejus  opinionem se dedere, lapsos tempus docuerit. Eodem  modo necesse est effectum Siphonis\protect\index{Sachverzeichnis}{sipho} iniquicruri non esse a sola aeris pressione, quando et ipse in \edtext{Recipiente sive}{\lemma{Recipiente}\Afootnote{sive \textit{ erg.} \textit{ L}}} Vacuo Magdeburgico\protect\index{Sachverzeichnis}{Recipiens!Magdeburgicum} succedit. 