[97 v\textsuperscript{o}] Nam etsi corporis duri politi superficies facilius radatur, quam mollis rugosi; non tamen facilius penetratur \edlabel{satisfecitend}crassities. \edtext{Quod enim}{\lemma{crassities.}\Afootnote{ \textit{ (1) }\ Nam \textit{ (2) }\ Quod enim \textit{ L}}} globus aliquis plumbeus tormento projectus saccum laneum non rumpit, ratio est, quia impetus ejus semel in materia molli perditus, nullo novo reparatur, quod secus est in \edtext{lumine}{\lemma{in}\Afootnote{ \textit{ (1) }\ flumine \textit{ (2) }\ lumine \textit{ L}}}, cujus fluxus est continuus. Ergo refractio\protect\index{Sachverzeichnis}{refractio} in medio densiore ad perpendicularem, potius in aliis corporibus quam in lumine evenire deberet. \edtext{Si lumen difficilius iter in aere quam aqua invenit}{\lemma{deberet.}\Afootnote{ \textit{ (1) }\ Nec sufficeret dicere in liquidis \textit{ (2) }\ Si [...] invenit \textit{ L}}}, non video cur non et alia subtilia corpora, motusve non debeant facilius per aquam quam aerem propagari; at constat tamen sonum per aquam longe obscurius quam per aerem \edtext{audiri}{\lemma{aerem}\Afootnote{ \textit{ (1) }\ meare \textit{ (2) }\ audiri \textit{ L}}}. Considerandum quoque est villositati duo inesse, \edtext{viscositatem}{\lemma{inesse,}\Afootnote{ \textit{ (1) }\ mollitiem \textit{ (2) }\ viscositatem \textit{ L}}} partium, et vim\protect\index{Sachverzeichnis}{vis!elastica} quandam Elasticam. Si \edtext{vis\-cositas, (seu mollities crassa) aeris}{\lemma{Si}\Afootnote{ \textit{ (1) }\ mollities eaque den \textit{ (2) }\ viscositas, (seu mollities crassa) aeris \textit{ L}}}  spectatur, ea utique \edtext{in}{\lemma{in}\Afootnote{ \textit{ (1) }\ aere \textit{ (2) }\ aqua \textit{ L}}} aqua major est; si vis Elastica\protect\index{Sachverzeichnis}{vis!elastica} \edtext{velleris aerei}{\lemma{Elastica}\Afootnote{ \textit{ (1) }\ hujus \textit{ (2) }\ velleris aerei \textit{ L}}}, jam major utique Elater\protect\index{Sachverzeichnis}{elater} in vitro est. \edtext{Oleum quoque est aere villosius, et tamen radius ex aere veniens in oleo refringitur ad perpendicularem; et generaliter observatur nulla sive villositatis, sive duritiei ratione habita refractionem ad perpendicularem esse semper majorem in corpore densiore, cum tamen non sit necesse omne corpus densum esse minus villosum nec sit credibile omne rarius esse villosius; alioqui spiritus vini foret oleo communi villosior quod nemo credet}{\lemma{est.}\Afootnote{ \textit{ (1) }\ Sed nec credo quamquam negari posse \textit{ (2) }\ Oleum [...] credet \textit{ L}}}. Tota ratiocinatio huc redit: omne rarius est villosius, omne villosius difficilius \hspace{1pt}permeatur\hspace{1pt} ergo\hspace{1pt} omne\hspace{1pt} rarius\hspace{1pt} difficilius\hspace{1pt} quam\hspace{1pt} densius\hspace{1pt} permeatur.\pend\pstart\noindent Utraque propositio \edtext{neganda est}{\lemma{propositio}\Afootnote{ \textit{ (1) }\ concedi nequit \textit{ (2) }\ neganda est \textit{ L}}}. Primum: omne rarius esse villosius, nam ex duobus corporibus villosis, necesse est densius esse villosius: deinde omne villosius difficilius permeari, concedi non potest; lumen\protect\index{Sachverzeichnis}{lumen} enim aut permeat poros, aut tantum pressione conatum\protect\index{Sachverzeichnis}{conatus} propagat. Si permeat poros; patet posse corpus aliquod \edtext{non villosum, sed durum esse minus}{\lemma{aliquod}\Afootnote{ \textit{ (1) }\ minus villosum, sed durum; reddi \textit{ (2) }\ non [...] minus \textit{ L}}} porosum seu difficilius permeabile \edtext{ut vitrum quam aerem}{\lemma{}\Afootnote{ut vitrum quam aerem \textit{ erg.} \textit{ L}}}; si pressione tantum propagatur lumen\protect\index{Sachverzeichnis}{lumen} necesse est omne corpus durum magis refringere quam molle, quia molle non est aeque capax pressionis cum cedat in omnem partem, ergo refractio\protect\index{Sachverzeichnis}{refractio} aquae refractioni\protect\index{Sachverzeichnis}{refractio} vitri tam prope accedere non posset, cum vitrum sit sine comparatione \edlabel{molliusstart}\edtext{mollius.}{{\xxref{molliusstart}{molliusend}}\lemma{mollius.}\Afootnote{ \textit{ (1) }\ Sed frustra in Hypothesi tam illab \textit{ (2) }\ Sed frustra in Hypothesi \textit{ (3) }\ Denique [...] seu \textit{(a)}\ minus refringens ad \textit{(b)}\ refringens [...] partibus  \textbar\ varie \textit{ erg.}\ \textbar\  consistit, quarum aliae cedunt aliae  \textit{(aa)}\  transmittunt \textit{(bb)}\ obsistunt, [...] conspiraturae.  \textbar\ Ad [...] resistentiam. \textit{ erg.}\ \textbar\  Sed frustra in Hypothesi \textit{ L}}} \pend \pstart  Denique si admitteretur utraque propositio et omne rarius seu refringens a perpendiculari esse villosius, et omne villosius difficilius permeari, non poterit tamen explicari refractio. Nam difficultas villosa permeandi non est homogenea per totum corpus, sed in partibus varie consistit, quarum aliae cedunt aliae obsistunt, ita ut futurae sint in villoso antequam a radio penetretur infinitae refractiones, diversae, nunquam in eandem regulam universalem in eodem corpore conspiraturae. Ad refractionem autem homogeneam necesse est eandem in prima et secunda et quavis alia superficie esse resistentiam. Sed frustra in\edlabel{molliusend} 
\edtext{Hypothesi tam inculta impeditaque refutanda tempus perditur}{\lemma{Hypothesi}\Afootnote{ \textit{ (1) }\ tam inculta tempus per$\langle$di$\rangle$tur difficultatibus plena \textit{ (2) }\ tam [...] perditur \textit{ L}}}, cum in promtu sit clarissima demonstratio mechanica causae verae. \pend 