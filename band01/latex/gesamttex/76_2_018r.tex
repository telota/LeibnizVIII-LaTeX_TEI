[18~r\textsuperscript{o}] go est) circumscriberentur, aut connecterentur. Ita differentia punctorum ejusdem objecti inter se, quod ad focum\protect\index{Sachverzeichnis}{focus} projiciendum attinet ingens, discriminisque ratio atque illud simul apparuit, cur apertura exigua reddita, tot radios velut inutiles excludere cogamur.\pend \pstart  Reperto mali fonte, remedium sponte patuit, \textso{inventumque est a me LENTIUM}\protect\index{Sachverzeichnis}{lens}\textso{, quas quia quantamcumque aperturam fer-}\linebreak \textso{unt, PANDOCHAS appellare soleo, genus novum}, a nullo, quod sciam, tactum; cujus species variae una omnium simplicissima, figurae sic satis parabilis, ex qua caeterae pro commoditate mutilatae: \textso{Lentium }\protect\index{Sachverzeichnis}{lens}autem nomine tam \textso{perspicilla }\protect\index{Sachverzeichnis}{perspicilla}quam \textso{specula }\protect\index{Sachverzeichnis}{speculum}sine discrimine comprehendo.\pend \pstart  Omnium autem commune est, nullo distantiae figuraeque objecti, aut fundi excipientis discrimine, ut omnia objecti puncta non minus distincte repraesententur, ac si unumquodque eorum in axe Optico\protect\index{Sachverzeichnis}{axis!opticus} esset. Quod hactenus in mentem venit nulli.\pend \pstart  Quantus sit hujus inventi fructus neminem Opticae intelligentem latet. Constat enim magnitudinem quidem apparentem posse vitris augeri in infinitum, sed ea aucta deminui lucem\protect\index{Sachverzeichnis}{lux}. Unde, ut nunc sunt \textso{lentes,}\protect\index{Sachverzeichnis}{lens} defectu Lucis\protect\index{Sachverzeichnis}{lux} in augenda magnitudine apparente parci esse debemus praesertim ubi objecta non sunt pro arbitrio nostro illustrabilia, (quanquam nunc quoque posita eadem objecti illustratione, plus radiorum apertura major colligat). Si vero aperturas maximas adhibere liceret, cum radii quoque futuri sint proportione plures, ac proinde lux\protect\index{Sachverzeichnis}{lux} major, poterunt radii quoque in majorem amplitudinem imaginis, salva luce\protect\index{Sachverzeichnis}{lux} et distinctione, refringi.\pend \pstart  Cum tamen neque Lentes Pandochae\protect\index{Sachverzeichnis}{lens!pandocha}, neque ullae aliae ex cognitis, et forte ex possibilibus quoque, omnes unius cujuscunque puncti radios in aliud punctum recolligant (Lentium\protect\index{Sachverzeichnis}{lens!pandocha} enim Pandocharum est, id tantum praestare punctis ob~