      
               
                \begin{ledgroupsized}[r]{120mm}
                \footnotesize 
                \pstart                
                \noindent\textbf{\"{U}berlieferung:}   
                \pend
                \end{ledgroupsized}
            
              
                            \begin{ledgroupsized}[r]{114mm}
                            \footnotesize 
                            \pstart \parindent -6mm
                            \makebox[6mm][l]{\textit{L}}Konzept: LH XXXVII 3 Bl. 99\textendash104. 3 Bog. 2\textsuperscript{o}. 12 S. Alle Seiten zweispaltig, links fortlaufender Text, rechts Korrekturen, Marginalien und z. T. umfangreiche Erg\"{a}nzungen. Bl.~101~r\textsuperscript{o} rechte Spalte oben eine Zeichnung. Weitere Zeichnungen auf Bl.~102~r\textsuperscript{o} und Bl. 103~r\textsuperscript{o} sowie Bl. 104~v\textsuperscript{o}. Am rechten oberen Rand von Bl. 99 v\textsuperscript{o} zwei kleinere Rechnungen.\\Cc 2, Nr. 486 D \pend
                            \end{ledgroupsized}
                %\normalsize
                \vspace*{5mm}
                \begin{ledgroup}
                \footnotesize 
                \pstart
            \noindent\footnotesize{\textbf{Datierungsgr\"{u}nde}: Auf Bl.~99~v\textsuperscript{o} erw\"{a}hnt Leibniz das Erscheinen von Guerickes \cite{00055}\textit{Experimenta nova} mit den Worten: ab autore novissime publicato. Bl.~99~r\textsuperscript{o} enth\"{a}lt einen Hinweis auf Leibniz' Erfindung einer gleichf\"{o}rmig gehenden Wasseruhr, die er, wie es heißt, sp\"{a}ter separat erkl\"{a}ren wolle. Der Text muss daher vor dem St\"{u}ck N. 39 \textit{Experimenta novissima pneumatica illustris Hugenii} entstanden sein, dessen Texttr\"{a}ger tlw. mit denen der Beschreibung der Wasseruhr \"{u}bereinstimmen. Da Guerickes \cite{00055}\textit{Experimenta nova} im Mai 1672 ausgeliefert wurden und das erw\"{a}hnte St\"{u}ck in der Zeit zwischen dem 25. Juli und dem 12. Dezember 1672 entstanden sein muss, gehen wir von dem gleichen Entstehungszeitraum aus.}
                \pend
                \end{ledgroup}
            
                \vspace*{8mm}
                \pstart 
                \normalsize
            [99 r\textsuperscript{o}]  Recepta fuit sententia in scholis effectus quosdam  extraordinarios Naturae, qui scilicet eveniunt, quoties  alioquin corpore uno ex suo loco exeunte aliud sensibile  intrare non posset, a  \textso{fuga Vacui}\protect\index{Sachverzeichnis}{fuga vacui} oriri; ut quod  duae Tabulae politae\protect\index{Sachverzeichnis}{tabulae!politae} sibi applicatae, \edtext{cohaerere videntur divellentique}{\lemma{applicatae,}\Afootnote{ \textit{ (1) }\ cohaerent, divellen \textit{ (2) }\  cohaerere videntur divellentique \textit{ L}}} resistunt, quoties  partes earum non separantur successive; \edtext{cum}{\lemma{successive;}\Afootnote{ \textit{ (1) }\ ita \textit{ (2) }\ cum \textit{ L}}}  enim aer aut aliud liquidum circumfusum non omnia  loca ab omnibus partibus deserta simul implere possit,  quia margini viciniora primum impleri necesse est;  at vero loca tam margini propinqua, quam  introrsum recedentia tali divulsione simul vacuentur;  sequeretur, inquiunt, ea vacua mansura, quod ne fiat,  divulsio impeditur.\pend \pstart  Quod item aqua ex vase  non effluit, aut embolus\protect\index{Sachverzeichnis}{embolus} ex tubo extrahi non potest,  cujus unum tantum foramen apertum est; aut follis  aperiri non potest cujus nullum. Aut quod foramine  debito aperto in tali casu corpus etiam alias  grave ascendit in tubum embolo\protect\index{Sachverzeichnis}{embolus} extracto (ut in antlia  experimur); aut in follem tabulis diductis \edtext{ad locum, ut ajunt, implendum}{\lemma{}\Afootnote{ad locum, ut ajunt, implendum \textit{ erg.} \textit{ L}}}; qui  sunt effectus suctionis, qualem etiam nos ore exercemus,  ut cum \edtext{aerem adducimus, cum}{\lemma{}\Afootnote{aerem adducimus, cum \textit{ erg.} \textit{ L}}} globum plumbeum \edtext{ex}{\lemma{plumbeum}\Afootnote{ \textit{ (1) }\ si \textit{ (2) }\ ex \textit{ L}}} canali sclopetario\protect\index{Sachverzeichnis}{sclopetum}  cum dentium periculo evocamus. Quo  pertinet phaenomenon quoque siphonis\protect\index{Sachverzeichnis}{sipho} bicruri
            \selectlanguage{polutonikogreek}ἑτερομήκους \selectlanguage{latin}%
            liquore pleni, qui altero crure breviore  in aquam vase quodam contentam intrans, altero  longiore \edtext{extra vas}{\lemma{longiore}\Afootnote{ \textit{ (1) }\ infra aquae superficiem \textit{ (2) }\ extra vas \textit{ L}}}  descendens aquam ex vase elicit, quod mihi, ut obiter dicam, occasionem  dedit inveniendae clepsydrae\protect\index{Sachverzeichnis}{clepsydra} cujusdam uniformiter  fluentis hactenus frustra tentatae, quam postea  separatim exponam.\footnote{\textit{In der rechten Spalte}: \textso{Experiment. 1}}\edtext{}{\lemma{exponam.}\Bfootnote{Zur Clepsydra vgl. \cite{00269}N. 63.}} \edtext{Ut de ventosis de ratione item aquam in aeolipilam\protect\index{Sachverzeichnis}{aeolipilae} aliosque angustos canales unius tantum aperturae immittendi, si calefiant, et postea aquae orificiis immissa refrigescant, non dicam.}{\lemma{}\Afootnote{Ut [...] dicam. \textit{ erg.} \textit{ L}}}\pend 
            \pstart  Haec ab omni  retro memoria, \edtext{ad horrorem quendam vacui referebantur}{\lemma{memoria,}\Afootnote{ \textit{ (1) }\ a horrore quodam vacui oriri nemo dubitabat \textit{ (2) }\ ad horrorem quendam vacui referebantur \textit{ L}}}.  Primus Galilaeus\protect\index{Namensregister}{\textso{Galilei} (Galilaeus, Galileus), Galileo 1564\textendash 1642} cum ab artificibus experimento \edtext{edoctis}{\lemma{experimento}\Afootnote{ \textit{ (1) }\ suo \textit{ (2) }\ edoctis \textit{ L}}}  didicisset, aquam in antliis\protect\index{Sachverzeichnis}{antlia} non posse elevari in infinitum,  ut veteres credebant, \edtext{nec}{\lemma{credebant,}\Afootnote{ \textit{ (1) }\ sed \textit{ (2) }\ nec \textit{ L}}} ultra 30 pedes multum  attolli \edtext{posse,}{\lemma{posse,}\Bfootnote{\textsc{G. Galilei, }\cite{00050}\textit{Discorsi}, Leiden 1638, S.~17 (\textit{GO} VIII, S.~64).}} nescio quid aliud causae subesse suspicatus est.  Nam si \edtext{Aqua}{\lemma{si}\Afootnote{ \textit{ (1) }\ haec \textit{ (2) }\ Aqua \textit{ L}}} ascenderet ob Vacui horrorem\protect\index{Sachverzeichnis}{horror vacui}, aut  potius Mundi\protect\index{Sachverzeichnis}{mundus} plenitudinem, utique ascenderet in infinitum; \edtext{Tuborum autem}{\lemma{infinitum;}\Afootnote{ \textit{ (1) }\ si \textit{ (2) }\ Tuborum autem \textit{ L}}} rupturae terminationem effectus ascribi non  posse, compertum enim erat, eosdem Tubos nihilo quam  ante factos ineptiores nec aquam jam attractam \edtext{si praecise cis terminos consisteres}{\lemma{}\Afootnote{si  \textit{ (1) }\ intra \textit{ (2) }\ praecise cis terminos consisteres \textit{ erg.} \textit{ L}}} fuisse  relapsam, quod fecisset utique si Tubo rimas agente intrasset  aer. \rule[-1cm]{0cm}{1cm}Et vero nec capi poterat quomodo diversae materiae  tubi \edtext{ad}{\lemma{}\Afootnote{ad \textit{ erg.} \textit{ L}}} rimas eodem tempore \edtext{agendas}{\lemma{tempore}\Afootnote{ \textit{ (1) }\ agerent, ad \textit{ (2) }\ agendas \textit{ L}}} effectumque  ubique eundem edendum conspirare possent.\pend \pstart  Primus Evangelista Torricellius\protect\index{Namensregister}{\textso{Torricelli} (Torricellius), Evangelista 1608\textendash 1647} Mathematicus Florentinus Galilaei\protect\index{Namensregister}{\textso{Galilei} (Galilaeus, Galileus), Galileo 1564\textendash 1642} discipulus Experimento illo celebri \edtext{in Mercurio\protect\index{Sachverzeichnis}{mercurius}, liquido tractabiliore,}{\lemma{}\Afootnote{in Mercurio\protect\index{Sachverzeichnis}{mercurius}, liquido tractabiliore, \textit{ erg.} \textit{ L}}} sumto\edtext{}{\lemma{sumto}\Bfootnote{\textsc{E. Torricelli, }\cite{00107}\textit{Brief an Ricci vom 11. Juni 1644}, Florenz 1663, S.~20f. (\textit{TO} III, S.~186\textendash188).}}  quod Valerianus M.\protect\index{Namensregister}{\textso{Magni} (Valerianus M.), Valeriano 1586\textendash 1661} sibi quoque vendicare voluerat,  et nunc \edtext{apud doctos}{\lemma{nunc}\Afootnote{ \textit{ (1) }\ a doctis \textit{ (2) }\ apud doctos \textit{ L}}} \edtext{a mensuranda aeris gravitate\protect\index{Sachverzeichnis}{gravitas!aeris}}{\lemma{}\Afootnote{a mensuranda aeris gravitate\protect\index{Sachverzeichnis}{gravitas!aeris} \textit{ erg.} \textit{ L}}} Barometer\protect\index{Sachverzeichnis}{barometrum} appellari solet de \edtext{massae aereae contrapondio  suspicionem fecit.}{\lemma{de}\Afootnote{ \textit{ (1) }\ gravitate aeris\protect\index{Sachverzeichnis}{gravitas!aeris|textit} suspicari coepit \textit{ (2) }\ massae aereae contrapondio  suspicionem fecit. \textit{ L}}} Quam\edtext{}{\lemma{}\Afootnote{Quam  \textbar\ a clarissimo Petito\protect\index{Namensregister}{\textso{Petit} (Petitus), Pierre 1598\textendash 1677} anno 1646 in Gallia\protect\index{Ortsregister}{Frankreich (Gallia, Francia)} \textit{ erg. u.}\  \textit{ gestr.}\ \textbar\ ingeniosissimus \textit{ L}}}  ingeniosissimus Pascalius\protect\index{Namensregister}{\textso{Pascal} (Pascalius), Blaise 1623\textendash 1662} avide arreptam \edtext{praeclara illa  observatione}{\lemma{arreptam}\Afootnote{ \textit{ (1) }\ celebri illo experimento \textit{ (2) }\ praeclara illa  observatione \textit{ L}}} in vertice montis cujusdam Arverniae\protect\index{Ortsregister}{Auvergne (Arvernia)} vulgo le Puy de domme\protect\index{Ortsregister}{Puy de Dome@Puy de D\^{o}me (Mons Averniae, Puy de domme)} per \edtext{clarum}{\lemma{}\Afootnote{clarum \textit{ erg.} \textit{ L}}} Perierium\protect\index{Namensregister}{\textso{P\'{e}rier} (Perrier, Perier, Perierius), Florin 1605\textendash 1702} facta 