\pend \pstart [p.~230] [...] quamquam in certo motu definiendo, nondum plene consentiunt; vnde Martis\protect\index{Sachverzeichnis}{Mars} reuolutio\footnote{\textit{Leibniz unterstreicht}: Martis reuolutio} circum axem, quam veteres coniectura tantum et analogia quadam assequuti sunt, certis et indubitatis obseruationibus firmata manet.\pend \pstart In Venere\protect\index{Sachverzeichnis}{Venus} ac Mercurio\protect\index{Sachverzeichnis}{Mercurius} ex obseruationibus huc vsque factis, nihil tale habemus, et quantum conjicio, nunquam habebimus; nempe Venus\protect\index{Sachverzeichnis}{Venus} ad instar cuiusdam lunae\protect\index{Sachverzeichnis}{luna} est, quae circa solem suos agit orbes.\footnote{\textit{Leibniz unterstreicht}: quadam assequuti [...] agit orbes} [...]. \pend \pstart Ad Saturnum\protect\index{Sachverzeichnis}{Saturnus} venio, in quem non ita pridem obseruatores nostri Diuiniana telescopia\protect\index{Sachverzeichnis}{telescopium} conuerterunt; illi autem erant D. Saluator Serra\protect\index{Namensregister}{\textso{Serra,} Salvatore SJ 17. Jh.}, D. Ioannes Lucius\protect\index{Namensregister}{\textso{Lucius,} Joannes SJ 1604\textendash 1679}\footnote{\textit{Leibniz unterstreicht}: D. Ioannes Lucius\protect\index{Namensregister}{\textso{Lucius,} Joannes SJ 1604\textendash 1679}}, ornatissimi homines, et nunquam satis laudandus Diuinius\protect\index{Namensregister}{\textso{Divini} (Divinius), Eustachio 1610\textendash 1685} noster; [...] vident enim, globum Saturni\protect\index{Sachverzeichnis}{Saturnus} modo attolli, modo deprimi intra annulum apparentem;\footnote{\textit{Leibniz unterstreicht}: vident enim [...] apparentem} ac proinde dum attollitur, annuli verticem superat, a quo superatur, cum deprimitur; sed inde illud longe maioris momenti deducerem, nimirum Saturnium annulum ablegandum videri;\footnote{\textit{Leibniz unterstreicht}: Saturnium annulum ablegandum videri} [...].