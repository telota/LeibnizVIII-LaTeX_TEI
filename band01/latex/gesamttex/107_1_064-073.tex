\pstart Bei den folgenden beiden St\"{u}cken handelt es sich um die Wiederaufnahme des Problems der L\"{a}ngengradbestimmung in Paris. Die \"{U}berlegungen sind als Konzept LH XXXV 15, 6 Bl. 64\textendash65 und als Abschrift von Schreiberhand LH XXXV 15, 6 Bl. 66\textendash 73 \"{u}berliefert. Bl. LH XXXV, 15, 6 Bl. 66 stimmt im Wesentlichen mit den ersten zwei Dritteln des Textbefundes von LH XXXV 15, 6 Bl. 64 r\textsuperscript{o} und weiteren f\"{u}nf Zeilen in der Mitte von LH XXXV, 15, 6 Bl. 65 v\textsuperscript{o} \"{u}berein. Die dazwischen liegenden und das Gros des Konzepts ausmachenden Textteile sind in der Abschrift nicht \"{u}berliefert, so dass von der Existenz einer dritten Version auszugehen ist, die als Vorlage f\"{u}r den Schreiber diente und im Nachlass bislang nicht aufgefunden wurde. Obwohl sie dasselbe Problem behandeln und zum Teil w\"{o}rtlich \"{u}bereinstimmen, weisen die beiden Texte signifikante inhaltliche Differenzen auf. Sie werden daher im Folgenden separat wiedergegeben. Die Datierung erfolgt aufgrund des Wasserzeichens, das sich bei einer Reihe von Texten zur Pneumatik findet, die zwischen August und Dezember 1672 entstanden sind.\pend