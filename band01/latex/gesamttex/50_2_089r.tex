[89~r\textsuperscript{o}] efficiat, nihilominus tamen, haec saltem in multis occasionibus ita parva simulque usui futura reddi possunt, ut pro puncto mechanico\protect\index{Sachverzeichnis}{punctum!mechanicum} habenda sint. Nam \rule[-9mm]{0mm}{9mm}\\
\renewcommand{\arraystretch}{2.3}
$\parbox{1.3cm}{\textit{FB} posita aequal.}$$\left\{
\begin{tabular}{c}
$\displaystyle\frac{7}{25}$\\$\displaystyle\frac{9}{41}$\\$\displaystyle\frac{31}{481}$\\$\displaystyle\frac{49}{1201}$\\$\displaystyle\frac{81}{3281}$
\end{tabular}
\right\}$ $\parbox{2.3cm}{erit praedic-\\ti axis lineola minor quam}$ $\left\{
\begin{tabular}{c}
$\displaystyle\frac{2}{11}$\\$\displaystyle\frac{1}{9}$\\$\displaystyle\frac{1}{109}$\\$\displaystyle\frac{1}{273}$\\$\displaystyle\frac{1}{745}$
\end{tabular}
\right\}$ $\parbox{2.3cm}{ac semidia-\\meter foci\\ minor quam}$ $\left\{
\begin{tabular}{c}
$\displaystyle\frac{1}{37}$\\$\displaystyle\frac{1}{79}$\\$\displaystyle\frac{1}{3151}$\\$\displaystyle\frac{1}{12435}$\\$\displaystyle\frac{1}{56125}$
\end{tabular}
\right\}$$\parbox{2.3cm}{semi-\\diametri \\ \textit{ND}.}$\\
\advanceline{4}Potest hoc eodem modo quo supra per Calculum \edtext{inveniri.}{\lemma{Calculum}\Afootnote{ \textit{ (1) }\ vitri in \textit{ (2) }\ inveniri. \textit{ L}}} \rule[0mm]{0mm}{10mm} Apparet deinceps \edtext{etiam ex hoc calculo remotissimum a vertice \textit{D} radium cadere aequali distantia a \textit{D} quam}{\lemma{etiam}\Afootnote{ \textit{ (1) }\ a \textit{D}, quam \textit{ (2) }\ ex hoc calculo remotissimum a vertice \textbar\ nam \textit{ gestr.}\ \textbar\ \textit{D} [...] quam \textit{ L}}}, \textit{K} ab \textit{N} hoc est, (posita \textit{ND} aequali,) ad distantiam $\displaystyle1\frac{6}{7}$\rule[-4mm]{0mm}{10mm}; ac quanto \textit{BF}, minor sumatur tanto etiam radios in minori axis \edtext{congregari etc. vide sequentia, vertendo dextrorsum bis, sub signo \protect\rotatebox{90}{$\circledast$}}{\lemma{axis}\Afootnote{ \textit{ (1) }\ lineola \textit{ (2) }\ congregari [...] signo \protect\rotatebox{90}{$\circledast$} \textit{ L}}} quam\footnote{\textit{An der Mittelfalz}: \#\hspace{-8.8pt}{$\Circle$} vide praecedentia ante vocem, quam ea sub signo \# vertendo retrorsum; nempe: \textso{majora esse quam ea} etc.} ea, quae hactenus in usu fuere, ac saepe longe minora sumi posse, ac debere; quo facto, sequitur, radios tum intra multo minorem longitudinem axis, ac focum\protect\index{Sachverzeichnis}{focus}, congregatum iri, nam quanto \textit{BF} minor est, tanto etiam radii in axem incidentes intra minorem longitudinem congregabuntur, focusque\protect\index{Sachverzeichnis}{focus} minor est.\pend 
