[83 v\textsuperscript{o}]  potuisset, si ea quae de machina illa, quam ad vitra secundum  determinatam figuram polienda, excogitaverat, ita accurate  in praxi potuissent observari, quam ab ipso ingeniose  fuere excogitata; sed cum nondum eo, quod sciam, pervenerit  artificum dexteritas, atque incerto sit, an nostris temporibus  eo perventura sit, nullo modo hic subsistendum, sed eo omni  conatu annitendum esse judicavi, ut id quod minus dextra  artificum manus exsequi non potest, aliis modis efficere  conemur. Cumque inquirerem, commodiorem non inveni,  quam ut ostendam quo modo per simplicissimas figuras,  et factu facillimas, id quod per magis compositas factum  est, fieri possit, ita ut nulla notabilis differentia inter  earum effectus reperiatur.\pend \pstart  Simpliciores autem quam hyperbola (qua cl. vir Ren. Descartes\protect\index{Namensregister}{\textso{Descartes} (Cartesius, des Cartes, Cartes.), Ren\'{e} 1596\textendash 1650} una cum linea recta ad figuranda vitra sua  utitur) nullae lineae sunt praeter rectam et circulum.  Sola autem recta nulla nobis ad hoc usui esse potest. Circulum  vero huic usui inservire posse comperi. Id quomodo fieri possit  nunc publice notum facere constitui, ut brevi perfectiora telescopia\protect\index{Sachverzeichnis}{telescopium}, et Microscopia\protect\index{Sachverzeichnis}{microscopium}, quam hactenus nacti, in naturae  notitia proficere possimus fructusque ejus quam primum degustare. 