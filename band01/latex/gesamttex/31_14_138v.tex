\pstart \textso{La Reponse }\edtext{\textso{\`{a} cette objection}}{\lemma{\textso{\`{a} cette objection}}\Afootnote{ \textit{ erg.} \textit{ L}}}\edtext{\textso{ que ceux qui soûtien- nent cette}}{\lemma{\textso{La} [...] \textso{cette}}\Afootnote{\textit{am Rand doppelt angestrichen}}} Hypothese nous ont donn\'{e}e, est telle: que les coups
[138 v\textsuperscript{o}] de cette nature, qui heurtent les superficies interieures des corps joincts,\edtext{}{\lemma{joincts,}\Afootnote{ \textbar\ comme \textit{gik} \textit{ gestr.}\ \textbar\ et \textit{ L}}} et qu'on pourroit appeller \textso{separatifs} sont tous compensez par des autres coups \textso{unitifs,} qui frappent en même temps en contre-sens la superficie exterieure du même corps, au même endroit, comme le coup \textit{gik} est compens\'{e} ou d\'{e}truit par le coup \textit{qp} et que par consequent tous deux, les separatifs aussi bien que les unitifs sont sans effect, estant separatifs d'un cost\'{e}, et unitifs de l'autre oppos\'{e}. Mais qu'il y a d'autres coups, unitifs de deux costez, contre les deux superficies exterieures \`{a} un même endroit, qui ne sont pas compensez, et qui sont capables tous seuls de conserver l'union.
\pend 
\pstart  Car il y a trois cas possibles; \textso{tantost} un pore d'un corps estant joinct contre un pore de l'autre corps, comme \textit{m} contre \textit{n} ou les coups sont indifferents, c'est \`{a} dire ny unitifs ny separatifs; \textso{tantost} un pore estant joinct contre une partie solide, comme \textit{i} contre \textit{k} ou l'un coup du cost\'{e} du pore est separatif, l'autre du cost\'{e} de la partie solide est unitif, d\'{e}truisant mutuellement l'effect l'un de l'autre; \textso{tantost} une partie solide estant joincte contre une partie solide, comme \textit{r} contre \textit{s} ou les coups de deux costez \textit{tr} et \textit{us} sont unitifs, et par consequent ny compensez ny d\'{e}truits par des autres separatifs. Voila \edtext{ce qu'on}{\lemma{Voila}\Afootnote{ \textit{ (1) }\ l'essence \textit{ (2) }\ ce qu'on \textit{ L}}} auroit pû dire de plus specieux pour le maintien de cette hypothese: On pourroit bien repartir, que le corps \edtext{ou partie solide est tousjours joincte \`{a} pore, et jamais \`{a} partie}{\lemma{corps}\Afootnote{ \textit{ (1) }\ ou pore est tousjours joinct \`{a} pore \textit{ (2) }\ ou [...] partie \textit{ L}}} solide, seroient sans union, item que deux corps bien joincts ou solide\protect\index{Sachverzeichnis}{corps!solide} est joint \`{a} solide, changeroient ais\'{e}ment de connexion, et perdroient tout \`{a} fait, ou deminueroient leur union, en glissant un peu l'un sur l'autre, parce qu'il pourroit arriver, que solide vinst \`{a} tomber \edtext{partout}{\lemma{tomber}\Afootnote{ \textit{ (1) }\ tousjours \textit{ (2) }\ partout \textit{ L}}} ou pour la pluspart contre pore. Mais on \edtext{auroit aussi raison de repliquer}{\lemma{on}\Afootnote{ \textit{ (1) }\ repliqueroit \`{a} cela avec justice \textit{ (2) }\ auroit aussi raison de repliquer \textit{ L}}} \`{a} cette repartie, que dans la pluspart des corps \edtext{et peut estre dans tous les corps sensibles\protect\index{Sachverzeichnis}{corps!sensible}}{\lemma{et}\Afootnote{ [...] sensibles\protect\index{Sachverzeichnis}{corps!sensible} \textit{ erg.} \textit{ L}}} il y a tant de solide joint contre \edtext{solide, et un tel m\'{e}lange des pores et parties solides qu'on gagneroit rien de sensible}{\lemma{solide,}\Afootnote{ \textit{ (1) }\ qu'on gagneroit rien, autant qu'il est sensible \textit{ (2) }\ et [...] sensible \textit{ L}}} quoyqu'on feroit glisser l'un sur l'autre, parce qu'un \edtext{solide quittant un solide}{\lemma{qu'un}\Afootnote{ \textit{ (1) }\ des solides trouvant un solide \textit{ (2) }\ solide quittant un solide \textit{ L}}} pour un pore; un autre solide en \'{e}change retrouvera pour un pore, un solide.
\pend
 \pstart \edtext{\textso{Il faut donc d'autres recherches pour s'asseurer}}{\lemma{\textso{Il} [...] \textso{s'asseurer}}\Afootnote{\textit{am Rand doppelt angestrichen}}} de ce qu'on doit juger de cette Hypothese, par le moyen des experiences plus particulieres. \edlabel{138va}
\edtext{\textso{Et}}{\lemma{\textso{Et}}\xxref{138va}{138ve}\Afootnote{[...] \textso{soubstraicte;} \textit{Markierung am Rand}}}
\textso{premierement, si l'espece de la liqueur, dans laquelle on suspend les }\edlabel{commestart}\textso{corps joincts purgez d'air,} \linebreak\textso{comme}\edlabel{commeend}\edtext{}{\lemma{}\xxref{commestart}{commeend}\Afootnote{\textso{corps joincts }  \textbar\ \textso{purgez d'air,} \textit{ erg.}\ \textbar\ \textso{ comme} \textit{ erg.} \textit{ L}}}\textso{ }\textso{deux placques}\protect\index{Sachverzeichnis}{deux placques}\textso{, ou le tuyau avec le }\textso{mercure pur- g\'{e}}\protect\index{Sachverzeichnis}{mercure!purg\'{e}}\textso{, ne change rien \`{a} la force de l'union, la pesanteur de cette }\edtext{\textso{liqueur}}{\lemma{\textso{cette}}\Afootnote{ \textit{ (1) }\ \textso{force} \textit{ (2) }\ \textso{liqueur} \textit{ L}}}\textso{ ambiente estant soubstraicte;}\edlabel{138ve}
%Fussnote nach oben verschoben \edtext{}{\lemma{Et [...] soubstraicte;}\xxref{138va}{138ve}\Afootnote{\textit{Markierung am Rand}}}
il y aura peu d'apparence, que l'union des corps joincts vienne du mouuement interieur de la liqueur ambiente; puisqu'il faut croire que la force de ce mouuement change avec l'espece de la liqueur, tant
[139 r\textsuperscript{o}] pour la quantit\'{e} des vagues, que pour la vitesse\protect\index{Sachverzeichnis}{vitesse} des coups; \edtext{estant constant}{\lemma{coups;}\Afootnote{ \textit{ (1) }\ car nous voyons \textit{ (2) }\ estant constant \textit{ L}}} qu'une liqueur est plus corrosive que l'autre, sans comparaison, et qu'il est vraysemblable que la corrosion des corps enfonc\'{e}s dans une liqueur vient du mouuement des parties de la liqueur.
\pend 