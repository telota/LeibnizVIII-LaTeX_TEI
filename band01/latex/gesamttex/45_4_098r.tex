\pstart 
[98 r\textsuperscript{o}] \textso{cap. 14.}\edtext{}{\lemma{\textso{14.}}\Bfootnote{\textsc{O. v. Guericke, }\cite{00055}a.a.O., S.~145f.}} Astra eminus invisibilia, cominus sunt visibilia ob restrictionem  sic enim lucida apparent. (+ Ut terra\protect\index{Sachverzeichnis}{terra} Frolichio\protect\index{Namensregister}{\textso{Fr\"{o}lich} (Frolichius, Fr\"{o}lichius), David 1595\textendash 1648} non apparebat ex monte Carpathio\protect\index{Ortsregister}{Karpaten (Mons Carpathius)}\edtext{}{\lemma{Carpathio}\Bfootnote{\textsc{D. Fr\"{o}lich, }\cite{00047}\textit{Bibliothecae sive Cynosura peregrinatium}, Ulm 1643, S.~268\textendash288.}}, apparebit tum in Luna\protect\index{Sachverzeichnis}{luna} +). Mars\protect\index{Sachverzeichnis}{Mars} cum  est in oppositione \astrosun $^{lis}$ et ideo terrae\protect\index{Sachverzeichnis}{terra} sexies vicinior quam conjunctioni proximus, vix apparet duplo major, cum debeat quintuplo. Nubes  quoque ultiores ob restrictionem luminis\protect\index{Sachverzeichnis}{lumen} eodem colore\protect\index{Sachverzeichnis}{color}\protect\index{Sachverzeichnis}{color|see{couleur}} nobis apparent, quo Luna\protect\index{Sachverzeichnis}{luna} de die  vesperi vero a sole\protect\index{Sachverzeichnis}{sol} illuminatae igneae  videntur sic quoque \edtext{aer}{\lemma{quoque}\Afootnote{ \textit{ (1) }\ aether\protect\index{Sachverzeichnis}{aether|textit} \textit{ (2) }\ aer \textit{ L}}} conglobatus  nobis Cometa\protect\index{Sachverzeichnis}{cometa} apparet valde autem  remotus ob constrictionem Luminis\protect\index{Sachverzeichnis}{lumen} a sole\protect\index{Sachverzeichnis}{sol}  accepti videtur stella nova\protect\index{Sachverzeichnis}{stella!nova}. \textit{Astra}\protect\index{Sachverzeichnis}{astrum}\textit{  oculis nostris non repraesentantur simplicia  et pura, sed radiis adventitiis partim brevibus,  partim longis ita ut corpus ipsorum multoties  auctius appareat, sicut quoque in accensis candelis  et faculis eminus positis videmus, quae noctu visae  e longinquo multis ejusmodi radiis circumdatae videntur e propinquo autem flammulam suam terminatam et exiguam ostendunt.  Causa breviorum radiorum est, quia }\textit{lux}\protect\index{Sachverzeichnis}{lux}\textit{ et }\textit{lumen}\protect\index{Sachverzeichnis}{lumen}\textit{  in laevitate tunicarum quae supra nostrorum }\textit{oculorum}\protect\index{Sachverzeichnis}{oculus}\textit{ }\textit{pupillas}\protect\index{Sachverzeichnis}{pupilla}\textit{, sunt reflectitur quae }\textit{reflexio}\protect\index{Sachverzeichnis}{reflexio}\textit{ cominus ob }\textit{sphaeram luminosam}\protect\index{Sachverzeichnis}{sphaera!luminosa}\textit{ percipi non potest. Sed eminus dum magis ibi umbrosum est, percipitur, nobisque videtur}\edtext{}{\lemma{\textit{videtur}}\Afootnote{ \textbar\ \textit{hos radios} \textit{streicht Hrsg.}\ \textbar\ \textit{esse} \textit{ L}}}\textit{ esse hos radios in }\textit{astris}\protect\index{Sachverzeichnis}{astrum}\textit{ vel  candelis, cum sint super pupillas}\protect\index{Sachverzeichnis}{pupilla}\textit{.} Sed longiores radii a palpebris existunt, quia  quanto angustiores tenentur genae tanto  longiores fiunt hi radii apertis genis omnino  non percipiuntur. Hinc Telescopium\protect\index{Sachverzeichnis}{telescopium} tales radios  astris detrahit, ut cani majori stellis omnibus  pulchriori, qui per Telescopium\protect\index{Sachverzeichnis}{telescopium} apparet  multoties, minor. Et Galil. \textit{System.}  \edtext{dial. 5,}{\lemma{dial. 5}\Bfootnote{Bei Guericke: dial. 3}} stellae fixae\protect\index{Sachverzeichnis}{stella!fixa} \edtext{primae magn. apparentem}{\lemma{}\Afootnote{primae magn. apparentem \textit{ erg.} \textit{ L}}} diametrum duorum  et aliquando judice Tychone\protect\index{Namensregister}{\textso{Brahe} (Tycho), Tycho 1546\textendash 1601} trium diametrorum revera esse vix 5 secundorum\edtext{}{\lemma{secundorum}\Bfootnote{\textsc{G. Galilei, }\cite{00277}\textit{Dialogo}, Florenz 1632, S.~394 (\textit{GO}, VII, S.~389).}}  fortasse. Quaedam ergo parva apparent et sine his radiis vel ob nimiam lucis\protect\index{Sachverzeichnis}{lux} non satis vivaciter radiantis  debilitatem, vel si propinqua, ob sphaerae luminosae\protect\index{Sachverzeichnis}{sphaera!luminosa} veram lucem\protect\index{Sachverzeichnis}{lux}  praedominantem. \textit{Venus}\protect\index{Sachverzeichnis}{Venus}\textit{ in conjunctione  sua }\edtext{\textit{vespertina}}{\lemma{}\Afootnote{\textit{vespertina} \textit{ erg.} \textit{ L}}}\textit{ }\textit{solem}\protect\index{Sachverzeichnis}{sol}\textit{ subiens multo major apparere deberet, quam in altera matutina et tamen ne duplicata quidem  videtur. Ratio quia tunc in falcem sinuatur, et propter cornua exiguum  ejus fit }\textit{lumen}\protect\index{Sachverzeichnis}{lumen}\textit{ et debile.} Jupiter\protect\index{Sachverzeichnis}{Jupiter} prae  caeteris magna atmosphaera\protect\index{Sachverzeichnis}{atmosphaera} circumdatus  hinc apparet major.\pend \pstart \textso{Gerick. lib. 4. cap. 15 }\edtext{}{\lemma{\textso{15}}\Bfootnote{\textsc{O. v. Guericke}, \cite{00055}a.a.O., S.~149.}} De Globo sulphureo\protect\index{Sachverzeichnis}{globus!sulphureus}  notat\edtext{}{\lemma{}\Afootnote{notat  \textbar\ ejus \textit{ gestr.}\ \textbar\ virtutem \textit{ L}}} virtutem Electricam\protect\index{Sachverzeichnis}{virtus electrica} non  videri in aere consistere, quia etiam longe  per filum lineum agat\linebreak (+ instar soni +).  Non reddit rationem phaenomenorum globi 
 sulphurei\protect\index{Sachverzeichnis}{globus!sulphureus}.\pend 
 \pstart \textso{lib. 5. cap. 1 }\edtext{}{\lemma{1}\Bfootnote{\textsc{O. v. Guericke}, \cite{00055}a.a.O., S.~152.}} Terra\protect\index{Sachverzeichnis}{terra} nunquam  ad dimidium unius miliaris Germanici  quadrantem perfossa est. Omnes  cryptae speluncaeque (cap. 3)\edtext{}{\lemma{(cap. 3)}\Bfootnote{\textsc{O. v. Guericke}, \cite{00055}a.a.O., S.~154f.}} quas nos  subterraneas vocamus non subterraneae sed superficiariae et quasi in cortice sunt. Videri tonitrua, fulmina,  ventos generari sub terra, et e montibus erumpere. Tales sunt Orkan  ajunt \edtext{Gurgitem}{\lemma{ajunt}\Afootnote{ \textit{ (1) }\ horis \textit{ (2) }\ Gurgitem \textit{ L}}} \edtext{Nautis Maelstrom, esse celeberrimum in Norwegia\protect\index{Ortsregister}{Norwegen@Norwegen (Norwegae, Norw\`{e}ge, Noorwegen)}}{\lemma{Gurgitem}\Afootnote{ \textit{ (1) }\ in Norwegia\protect\index{Ortsregister}{Norwegen@Norwegen (Norwegae, Norw\`{e}ge, Noorwegen)|textit} \textit{ (2) }\ Nautis [...] Norwegia \textit{ L}}}  13 miliarum in circuitu, medium  petra occupat, quam vocant  Moußke, is horis 6 absorbet omnia, et aliis 6 horis quae displicent revomit.\pend 
 \pstart \textso{cap. 3}.\edtext{}{\lemma{\textso{cap. 3.}}\Bfootnote{\textsc{O. v. Guericke}, \cite{00055}a.a.O., S.~155.}} Anno 1663  hoc ipso anno quo autor se haec scribere  ait Quedlinburgi\protect\index{Ortsregister}{Quedlinburg (Quedlinburgum)} in monte vulgo \textit{den }\textit{Zeunickenberg}\protect\index{Ortsregister}{Seveckenberg (Zeunickenberg)}\textit{, ubi materia  calcis effoditur, et quidem, in quadam  ejus rupe repertum est sceleton unicornis, in posteriori corporis parte  ut bruta solent reclinatum, capite  vero sursum elevato, ante frontem  gerens longe extensum cornu  crassitie cruris humani, atque ita  secundum proportionem longitudine  quinque fere ulnarum. }\textit{Animalis}\protect\index{Sachverzeichnis}{animal}\textit{ hujus  sceleton primum ex ignorantia  fuit contritum, et particulatim  extractum donec caput }\edtext{\textit{una}}{\Afootnote{ \textit{ (1) }\ \textit{cum} \textit{ (2) }\ \textit{caput} \textit{ (3) }\ \textit{cum cornu} \textit{ L}}}\textit{ cum cornu et aliquibus  costis, spina dorsi, atque ossibus Reverendissimae principi Abbatissae  ibidem degenti fuerit traditum.} Unde  tempus longo temporis tractu ut  viventia incrementum sumit speciemque  habet vegetabilis augmentum prae  se ferentis in pantheon jam descenditur in quod olim multis ascendebatur gradibus (+ ego contra puto illuvionibus terram\protect\index{Sachverzeichnis}{terra} complanari et valles attolli +).\pend \pstart \textso{Cap. 4.}\edtext{}{\lemma{\textso{4.}}\Bfootnote{\textsc{O. v. Guericke}, \cite{00055}a.a.O., S.~156f.}} Putat tellurem\protect\index{Sachverzeichnis}{tellus}  praeditam anima sensibili\protect\index{Sachverzeichnis}{anima sensibilis}. Telluris\protect\index{Sachverzeichnis}{tellus} ait  motum vertentem esse ab ipsius appetitu Poli\protect\index{Sachverzeichnis}{polus} nunc has nunc illas partes opponendi. Ejus vitam a solis\protect\index{Sachverzeichnis}{sol} flamma mutuo sumtam.  Etiam Aristarchus\protect\index{Namensregister}{\textso{Aristarch v. Samos,} um 230 v. Chr.} \cite{00211}\textit{De Mundi\protect\index{Sachverzeichnis}{mundus} systemate} subsit de cometis\protect\index{Sachverzeichnis}{cometa} terram\protect\index{Sachverzeichnis}{terra} videri \edtext{animatam.}{\lemma{animatam.}\Bfootnote{\textsc{Aristarch von Samos}, \textit{De mundi systemate}, Paris 1647, S.~50.}}\pend 