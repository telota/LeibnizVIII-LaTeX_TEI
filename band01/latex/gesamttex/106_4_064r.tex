      
               
                \begin{ledgroupsized}[r]{120mm}
                \footnotesize 
                \pstart                
                \noindent\textbf{\"{U}berlieferung:}   
                \pend
                \end{ledgroupsized}
            
              
                            \begin{ledgroupsized}[r]{114mm}
                            \footnotesize 
                            \pstart \parindent -6mm
                            \makebox[6mm][l]{\textit{L}}Konzept: LH XXXV 15, 6 Bl. 64\textendash65. 1 Bog., 4 S.
                2\textsuperscript{o}, zweispaltig. Linke Spalte fortlaufender Text, rechte Spalte
                umfangreiche Korrekturen und Erg\"{a}nzungen, die auf Bl. 64 r\textsuperscript{o} die
                rechte Spalte vollst\"{a}ndig ausf\"{u}llen. Bl. 64 v\textsuperscript{o} eine
                Zeichnung. Bl. 65 v\textsuperscript{o} vier Zeichnungen und drei Nebenrechnungen. Die
                Nebenrechnungen haben keine Beziehung zum Text.\\Cc 1, Nr. 193 J \pend
                            \end{ledgroupsized}
                \vspace*{8mm}
                \pstart 
                \normalsize
            [64 r\textsuperscript{o}] Ex quo \edtext{horologium funependulo                    animatum omnibus hactenus  cognitis accuratius, detectum est in}{\lemma{quo}\Afootnote{ \textit{ (1) }\ Illustris Hugenius\protect\index{Namensregister}{\textso{Huygens} (Hugenius, Vgenius, Hugens, Huguens), Christiaan 1629\textendash 1695|textit} horologium\protect\index{Sachverzeichnis}{horologium|textit} omnibus hactenus  cognitis accuratius, funependulo\protect\index{Sachverzeichnis}{funependulum|textit} animatum, detexit, in \textit{ (2) }\ horologium [...] in \textit{ L}}}  magnam omnes spem erecti sumus. Negotii Longitudinum\protect\index{Sachverzeichnis}{longitudo}  aliquando penitus conficiendi \edtext{quantum ab observatione coeli sperari potest}{\lemma{}\Afootnote{BITTE UEBERPRUEFEN!!! quantum [...] potest \textit{ erg.} \textit{ L}}}.\pend \pstart  Constat \edtext{locum navis  praecise cognosci  cognita Latitudine longitudineque                    loci, et longitudinem  cognosci cognita}{\lemma{Constat}\Afootnote{ \textit{ (1) }\ ad locum navis\protect\index{Sachverzeichnis}{navis|textit} cognitionem necessariam esse \textit{ (2) }\ ad locum navis\protect\index{Sachverzeichnis}{navis|textit} cognoscendum  necessariam esse                    cognitionem Latitudinis\protect\index{Sachverzeichnis}{latitudo|textit} longitudinisque\protect\index{Sachverzeichnis}{longitudo|textit} loci, et ad cognitionem longitudinis\protect\index{Sachverzeichnis}{longitudo|textit} \textit{ (3) }\ locum [...] cognita \textit{ L}}} hora praesenti tum  loci navis\protect\index{Sachverzeichnis}{navis} per observationem            coeli, tum loci discessus  per observationem \edtext{                    horologii                                    }{\lemma{observationem}\Afootnote{ \textit{ (1) }\                     penduli                    \protect\index{Sachverzeichnis}{pendulum|textit}                 \textit{ (2) }\                     horologii                                     \textit{ L}}} accurati inde  a loco discessus in navi\protect\index{Sachverzeichnis}{navis} allati.\pend \pstart \edtext{Seposito horologio seu hora presenti loci}{\lemma{?LEMMA?:allati.}\Afootnote{ \textit{ (1) }\ Horam  loci \textit{ (2) }\ Seposito horologio seu hora   \textbar\ presenti \textit{ erg.}\ \textbar\  loci \textit{ L}}} discessus,  tantum de hora navis\protect\index{Sachverzeichnis}{navis} seu coeli            observatione  hoc loco dicam. Ad Horam loci  computandam, sufficere            cognitionem \textso{Latitudinis}\protect\index{Sachverzeichnis}{latitudo} loci, et declinationis
                solaris\protect\index{Sachverzeichnis}{declinatio!solaris}, vulgo            constat. Hugenius\protect\index{Namensregister}{\textso{Huygens} (Hugenius, Vgenius, Hugens, Huguens), Christiaan 1629\textendash 1695} rationem proposuit \edtext{quae}{\lemma{proposuit}\Afootnote{ \textit{ (1) }\ commodiorem,  quae neque \textit{ (2) }\ quae \textit{ L}}} neutra indigeret \edtext{ observatis tantum                duabus}{\lemma{indigeret}\Afootnote{ \textit{ (1) }\ sed fieret factis  duabus observationibus \textit{ (2) }\  observatis   \textbar\ tantum \textit{ erg.}\ \textbar\                 duabus \textit{ L}}}  aequalibus solis\protect\index{Sachverzeichnis}{sol} aut etiam alterius stellae\protect\index{Sachverzeichnis}{stella} \edtext{satis alte super horizontem                emergentis }{\lemma{}\Afootnote{satis alte super horizontem                emergentis   \textbar\ (quanquam  hoc cognosci non possit,  \textit{ (1) }\ nisi qua \textit{ (2) }\ altene an non assurrectura sit, nisi altitudine loci circiter cognita) \textit{ gestr.}\ \textbar\   \textit{ erg.} \textit{ L}}}\edtext{altitudinibus}{\lemma{?LEMMA?:cognita)}\Afootnote{ \textit{ (1) }\ observationibus \textit{ (2) }\ altitudinibus \textit{ L}}} \edtext{aut unica sed tunc facta, cum sidus\protect\index{Sachverzeichnis}{sidus}                est praecise in meridiano\protect\index{Sachverzeichnis}{meridianus} loci.}{\lemma{}\Afootnote{BITTE UEBERPRUEFEN!!! aut [...] cumsidus\protect\index{Sachverzeichnis}{sidus}                est praecise in meridiano\protect\index{Sachverzeichnis}{meridianus} loci. \textit{ erg.} \textit{ L}}} \edtext{Optimum esse                        eligere  altitudines}{\lemma{?LEMMA?:loci.}\Afootnote{ \textit{ (1) }\ Ex  quibus   \textbar\ duabus \textit{ erg.}\ \textbar\  altitudinib \textit{ (2) }\ Optimum  \textit{(a)}\ est autem \textit{(b)}\ esse                        eligere  altitudines \textit{ L}}} minimas, id est ipsum praecise tempus solis\protect\index{Sachverzeichnis}{sol} surgentis aut cadentis                \edtext{observato momento quo dimidia solis\protect\index{Sachverzeichnis}{sol} pars extat supra horizontem, dimidia infra                horizontem deprimitur.}{\lemma{}\Afootnote{BITTE UEBERPRUEFEN!!! observato  \textit{ (1) }\ tempo \textit{ (2) }\ momento quo dimidia solis\protect\index{Sachverzeichnis}{sol}pars [...] deprimitur. \textit{ erg.} \textit{ L}}} \edtext{ Nave autem interea progrediente differentiam                    locorum cujusque observationis}{\lemma{?LEMMA?:deprimitur.}\Afootnote{ \textit{ (1) }\ Tempus autem inter  utramque observationem elapsum eorum \textit{ (2) }\ Nave [...] observationis \textit{ L}}}  illa ipsa ratione, qua vulgo nautae in mensuranda  per            conjecturas navis\protect\index{Sachverzeichnis}{navis} via utuntur, definiendam.\pend \pstart Caeterum quod \edtext{ad hunc  calculum}{\lemma{quod}\Afootnote{ \textit{ (1) }\ hac ratione observat \textit{ (2) }\ ad hunc  calculum \textit{ L}}}\edtext{Longitudinum cognitio Latitudinum}{\lemma{calculum}\Afootnote{ \textit{ (1) }\                     \textso{                        }\textso{Latitudinis}\textso{                        }\protect\index{Sachverzeichnis}{latitudo|textit}\textso{                    }                 \textit{ (2) }\ Longitudinum cognitio Latitudinum \textit{ L}}} necessaria  non est, id quidem mea sententia lucro caret, indaganda             enim nihilominus Latitudo\protect\index{Sachverzeichnis}{latitudo} est separatim, ut locus navis\protect\index{Sachverzeichnis}{navis} verus            inveniatur, nam uti \edtext{                    Latitudo                                    }{\lemma{uti}\Afootnote{ \textit{ (1) }\                     Longitudo                    \protect\index{Sachverzeichnis}{longitudo|textit}                 \textit{ (2) }\                     Latitudo                                     \textit{ L}}} sine Longitudine\protect\index{Sachverzeichnis}{longitudo}, ita \edtext{vicissim longitudo}{\lemma{ita}\Afootnote{ \textit{ (1) }\                     longitudine                    \protect\index{Sachverzeichnis}{longitudo|textit}                 \textit{ (2) }\ vicissim longitudo \textit{ L}}} sine latitudine\protect\index{Sachverzeichnis}{latitudo} ad regendam  navigationem non            sufficit.\pend \pstart Accedit, quod magna spes est \textso{Rationem }\textso{Latitudinum}\protect\index{Sachverzeichnis}{latitudo}\textso{  investigandarum universalem} ab            observatione coeli  et tempestatibus independentem quamprimum plene  detectum            iri, ubi modo \textso{                }\textso{Acus inclinatoria}\textso{                }\protect\index{Sachverzeichnis}{acus!inclinatoria}\textso{            } ad  regulam reducta fuerit, quod mihi jam in potestate nostra             esse videtur\footnote{             Verte  et vide sign. ♀. Sequuntur                enim verba: Quod  ergo   \textbar\ Illustris Hugenius\protect\index{Namensregister}{\textso{Huygens} (Hugenius, Vgenius, Hugens, Huguens), Christiaan 1629\textendash 1695} \textit{ gestr.}\ \textbar\ hoc  modo        } \edtext{ut postea fusius                dicam.}{\lemma{}\Afootnote{ut postea fusius                dicam. \textit{ erg.} \textit{ L}}} Constat enim acum \edtext{ puram}{\lemma{acum}\Afootnote{ \textit{ (1) }\ magneticam \textit{ (2) }\  puram \textit{ L}}}, ubi primum magneti\protect\index{Sachverzeichnis}{magnes} affricta est, \edtext{quasi pondusculo appenso in nostris}{\lemma{est,}\Afootnote{ \textit{ (1) }\ in nostri  quasi pondere \textit{ (2) }\ quasi pondusculo appenso in nostris \textit{ L}}} oris  nonnihil versus polum
                arcticum\protect\index{Sachverzeichnis}{polus!arcticus}            propendere; et quanto \edtext{}{\lemma{}\Afootnote{quanto  \textbar\ illuc \textit{ gestr.}\ \textbar\ propius \textit{ L}}}propius accedat polo\protect\index{Sachverzeichnis}{polus}, eo \edtext{magis inlinari}{\lemma{eo}\Afootnote{ \textit{ (1) }\ fieri \textit{ (2) }\ magis inlinari \textit{ L}}};  donec in Regionibus Arcticis \edtext{ad            situm}{\lemma{}\Afootnote{ad            situm \textit{ erg.} \textit{ L}}} perpendicularem propemodum accedat.\pend \pstart Quod nautae illuc euntes            malo suo  experti sunt. Unde illi contraria \edtext{seu ant\-arcticum polum\protect\index{Sachverzeichnis}{polus
                    antarcticus} respiciente}{\lemma{}\Afootnote{seu antarcticum polum\protect\index{Sachverzeichnis}{polus
                    antarcticus} respiciente \textit{ erg.} \textit{ L}}} acus\protect\index{Sachverzeichnis}{acus
            magnetica} parte magis  cis lineam gravata, aut trans lineam levata;             acum in aequilibrium\protect\index{Sachverzeichnis}{aequilibrium} redigere conantur. Hujus rei \edtext{Horologio jam accurato}{\lemma{rei}\Afootnote{ \textit{ (1) }\ Eo \textit{ (2) }\ Tali horologio\protect\index{Sachverzeichnis}{horologium|textit} \textit{ (3) }\ Horologio jam accurato \textit{ L}}} supposito variae \edtext{propositae}{\lemma{variae}\Afootnote{ \textit{ (1) }\ adhibitae \textit{ (2) }\ propositae \textit{ L}}}  sunt \edtext{loci navis}{\lemma{sunt}\Afootnote{ \textit{ (1) }\ coeli \textit{ (2) }\ loci navis \textit{ L}}} per observationes coelestes inveniendi  rationes,            alia alia commodior;  ex quibus una mihi\edtext{}{\lemma{}\Afootnote{mihi  \textbar\ novissime \textit{ gestr.}\ \textbar\ in \textit{ L}}} in mentem             venit, universalis admodum et simplex, et  satis, ut credo, accurata.                \textso{Simplex} quia  non nisi una observatione coelesti            transigitur, \edtext{ cum contra}{\lemma{transigitur,}\Afootnote{ \textit{ (1) }\ et \textit{ (2) }\  cum contra \textit{ L}}} ubi duabus pluribusque observationibus diverso  tempore factis opus            est, interea navi\protect\index{Sachverzeichnis}{navis} provecta, difficillima reddatur computatio.                \textso{Universalis,} quia nulli fere tempori, non            diei, non nocti, non certis siderum
                altitudinibus\protect\index{Sachverzeichnis}{altitudo
            sideris}  alligata est; sed \edtext{solo}{\lemma{sed}\Afootnote{ \textit{ (1) }\ non nisi soli simplici \textit{ (2) }\ solo \textit{ L}}} solis\protect\index{Sachverzeichnis}{sol} \edtext{ Lunaeve}{\lemma{?LEMMA?:solis}\Afootnote{ \textit{ (1) }\                     Lunaeque                    \protect\index{Sachverzeichnis}{luna|textit}                 \textit{ (2) }\  Lunaeve \textit{ L}}} aut stellae cujusdam fixae\protect\index{Sachverzeichnis}{stella!fixa} conspectu \edtext{contenta est}{\lemma{conspectu}\Afootnote{ \textit{ (1) }\ indiget \textit{ (2) }\ contenta est \textit{ L}}} \edtext{qui raro per tempus  notabile deesse            potest}{\lemma{}\Afootnote{BITTE UEBERPRUEFEN!!! qui raro per  \textit{ (1) }\ longum \textit{ (2) }\ tempus   \textbar\  notabile \textit{ erg.}\ \textbar\  deesse            potest \textit{ erg.} \textit{ L}}}. Cum contra solutiones quae ex \textso{                }\textso{Lunae}\textso{                }\protect\index{Sachverzeichnis}{luna}\textso{            } observatione pendent, dimidio fere mensis tempore            ante et post novilunium  conspectu scilicet Lunae\protect\index{Sachverzeichnis}{luna} negato, cessent,             et quae \edtext{                    \textso{                        Sole                                            }                }{\lemma{quae}\Afootnote{ \textit{ (1) }\ in Sole\protect\index{Sachverzeichnis}{sol|textit} \textit{ (2) }\                     \textso{                        Sole                                            }                 \textit{ L}}} indigent, noctu fieri nequeant;  et eae in quibus duabus aequalibus            ejusdem sideris altitudinibus\protect\index{Sachverzeichnis}{altitudo
                sideris} observatis opus est, hoc \edtext{praeter            caetera}{\lemma{}\Afootnote{praeter            caetera \textit{ erg.} \textit{ L}}} incommodum habeant, ut priore observatione facta \edtext{posterior}{\lemma{facta}\Afootnote{ \textit{ (1) }\ secunda \textit{ (2) }\ posterior \textit{ L}}} aeris \edtext{marisve}{\lemma{aeris}\Afootnote{ \textit{ (1) }\ navisque \textit{ (2) }\ marisve \textit{ L}}} injuria facile intercipiatur, ac proinde prior reddatur            inutilis. De  quibus aliisque in hoc negotio observandis  legi possunt, tum            quae ab Illustri Hugenio\protect\index{Namensregister}{\textso{Huygens} (Hugenius, Vgenius, Hugens, Huguens), Christiaan 1629\textendash 1695} \edtext{                Horologii penduli                \protect\index{Sachverzeichnis}{horologium!pendulum}            }{\lemma{}\Afootnote{                Horologii penduli                \protect\index{Sachverzeichnis}{horologium!pendulum}             \textit{ erg.} \textit{ L}}} inventore circa applicationem ejus ad Longitudines\protect\index{Sachverzeichnis}{longitudo} sunt scripta,                \edtext{}{\lemma{scripta,}\Bfootnote{\textsc{Chr. Huygens, }\cite{00212}\textit{Kort onderwijs, Den Haag 1665 (\textit{HO} V, S.~214\textendash230).}} tum quae            Transactionibus Anglicanis num. 47. sunt inserta.\edtext{}{\lemma{inserta.}\Bfootnote{\textsc{Chr. Huygens,                    }\cite{00064}\textit{Instructions concerning the use of pendulum\textendash watches},                    \textit{Philosophical Transactions}, Bd 4, Nr. 47, 1669, S.                937\textendash953. }} \textso{Accurata}  denique                \edtext{satis}{\lemma{}\Afootnote{satis \textit{ erg.} \textit{ L}}} est quam propono, ratio, tum quia             calculo exiguo aut facillimo \edtext{ ac ne nautas quidem turbaturo indiget}{\lemma{facillimo}\Afootnote{ \textit{ (1) }\ indiget, nec  \textit{ (2) }\ ac [...] indiget \textit{ L}}}  tum quia \edtext{sola}{\lemma{quia}\Afootnote{ \textit{ (1) }\ a refractionibus\protect\index{Sachverzeichnis}{refractio|textit} \textit{ (2) }\ sola \textit{ L}}} sideris\protect\index{Sachverzeichnis}{sidus} \edtext{cujusdam elevatione}{\lemma{?LEMMA?:sideris}\Afootnote{ \textit{ (1) }\ conspecti \textit{ (2) }\ cujusdam elevatione \textit{ L}}} ultra Horizontem loci navis\protect\index{Sachverzeichnis}{navis} observata, quae certe instrumentis bonis                \edtext{ad quartas usque minutorum partes et ultra divisis,}{\lemma{}\Afootnote{BITTE UEBERPRUEFEN!!! ad quartas usque minutorum partes   \textbar\ et ultra \textit{ erg.}\ \textbar\  divisis, \textit{ erg.} \textit{ L}}} satis \edtext{recte}{\lemma{satis}\Afootnote{ \textit{ (1) }\ bene \textit{ (2) }\ recte \textit{ L}}} sumi potest, perficitur: neque enim nisi angulo  indigemus, quem            radius e sidere\protect\index{Sachverzeichnis}{sidus} dato ductus facit ad loci  horizontem. \edtext{Nec}{\lemma{horizontem.}\Afootnote{ \textit{ (1) }\ Qui instrumento \textit{ (2) }\ Nec \textit{ L}}} \edtext{a}{\lemma{}\Afootnote{a \textit{ erg.} \textit{ L}}} refractionibus\protect\index{Sachverzeichnis}{refractio} metuere            nobis magnopere debemus praeterquam enim quod sideris\protect\index{Sachverzeichnis}{sidus} ultra             horizontem satis evecti refractio\protect\index{Sachverzeichnis}{refractio} minus turbat, et Tabula etiam computandarum                Refractionum\protect\index{Sachverzeichnis}{refractio} ex crepusculorum quantitate aliisque  indiciis ab            Astronomis \edtext{condita est}{\lemma{Astronomis}\Afootnote{ \textit{ (1) }\ computata est \textit{ (2) }\ condita est \textit{ L}}}; praeter inquam haec omnia en facilem occurrendi rationem. Si eodem tempore             duo pluraque sidera\protect\index{Sachverzeichnis}{sidus} (semper enim plures fixae\protect\index{Sachverzeichnis}{stella!fixa} simul            videntur) observentur, cum enim eorum refractionem\protect\index{Sachverzeichnis}{refractio} necesse            sit esse diversam, sese mutuo corrigent observationes, quae cum eodem            tempore fiant, uni observationi aequipollent. Utile autem est sidus\protect\index{Sachverzeichnis}{sidus} eligi,  prae            caeteris quod alte supra horizontem loci assurget. 