[87 v\textsuperscript{o}] poterit tantae magnitudinis telescopium\protect\index{Sachverzeichnis}{telescopium}, ut majus fortassis hactenus nullum factum fuerit, ac in posterum cum fructu fieri poterit, (nam ut postea dicetur, focus hujus altera 34 pedes ab exteriore vitri superficie distabit.) sitque diameter aperturae aequalis $\displaystyle11\frac{1}{4}$\rule[-4mm]{0mm}{10mm} digitorum mensurae, incident omnes radii in axem intra longitudinem lineolae quae minor erit quam $\displaystyle\frac{1}{994}$\rule[-4mm]{0mm}{10mm} pedum 12, hoc est $\displaystyle\frac{144}{994}$\rule[-4mm]{0mm}{10mm} digiti, quae longitudo respectu tanti Circuli fortasse non consideratu digna judicabitur, praesertim si inter alia etiam consideretur tum ejus foci diametrum fore $\displaystyle\frac{1}{241}$\rule[-4mm]{0mm}{10mm} digiti minorem.
\pend 
