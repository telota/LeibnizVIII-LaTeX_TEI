      
               
                \begin{ledgroupsized}[r]{120mm}
                \footnotesize 
                \pstart                
                \noindent\textbf{\"{U}berlieferung:}   
                \pend
                \end{ledgroupsized}
            
              
                            \begin{ledgroupsized}[r]{114mm}
                            \footnotesize 
                            \pstart \parindent -6mm
                            \makebox[6mm][l]{\textit{L}}Notiz: LH XXXV 12, 2 Bl. 155\textendash156. 1 Bog. 2\textsuperscript{o}. 1/3 S. auf Bl. 156 v\textsuperscript{o}. Der ver\-bleibende Teil dieser sowie die \"{u}brigen Seiten \textit{LSB} VII, 1 N. 6\raisebox{-0.5ex}{\tiny{6}}.\\Cc 2, Nr. 632 tlw. \pend
                            \end{ledgroupsized}
                %\normalsize
                \vspace*{5mm}
                \begin{ledgroup}
                \footnotesize 
                \pstart
            \noindent\footnotesize{\textbf{Datierungsgr\"{u}nde}: Die Exzerpte befinden sich im oberen Teil einer Seite, die den Abschluss eines umfassenderen, in \textit{LSB} VII, 1 N. 6\raisebox{-0.5ex}{\tiny{6}} mit dem Titel \textit{De figuris similibus metiendis} gedruckten St\"{u}cks bildet. Wir \"{u}bernehmen die dort angegebene Datierung.}
                \pend
                \end{ledgroup}
            
                \vspace*{8mm}
                \pstart 
                \normalsize
            [156 v\textsuperscript{o}] Varen. lib. 2. cap. 27. prop.\edlabel{13start} [13.]\edtext{}{\Afootnote{14.\textit{\ L \"{a}ndert Hrsg. } }}\pend \pstart \edtext{\textit{Sub Zona torrida dum sol versatur in arcu Eclipticae inter Tropicum  vicinum et loci parallelum intercepto illis diebus umbra}\edlabel{13end}}{\lemma{[13.]}\xxref{13start}{13end}\Afootnote{ \textit{ (1) }\ \textit{Umbras n} \textit{ (2) }\ \textit{Sub} [...] \textit{umbra} \textit{ L}}}\textit{ styli erecti bis regreditur et relictas lineas repetit, semel ante meridiem semel post meridiem  ipse quoque }\textit{sol}\protect\index{Sachverzeichnis}{sol}\textit{ hisce diebus cursum suum inflectere videbitur.}\edtext{}{\lemma{\textit{videbitur.}}\Bfootnote{\textsc{B. Varenius, }\cite{00109}\textit{Geographia generalis}, Cambridge 1672, S.~367. }} \textit{\textso{Coroll.}}\textit{ Non itaque  praeter naturam est umbram in }\textit{horologiis sciathericis}\protect\index{Sachverzeichnis}{horologium!sciathericum}\textit{ regredi, sed tum demum  miraculum est, si subito fiat, insigni spatio, item si lineas horarias repetat, nempe si stylus non sit perpendicularis sed }\textit{axi mundi}\protect\index{Sachverzeichnis}{axis!mundi}\textit{ parallelus: imo etsi sit  perpendicularis, non tum lineae ipsius umbrae indicant horas, sed lineae umbrarum axis mundi, cujus pars mente concipitur in }\textit{horologiis}\protect\index{Sachverzeichnis}{horologium}\textit{ si absit.} \edtext{}{\lemma{\textit{absit.}}\Bfootnote{\textsc{B. Varenius}, \cite{00109}a.a.O., S.~368. }}\pend \pstart  Varen. lib. 3. cap. 40. prop. 4. nautae \textit{majore industria confectum  iter mensurant per naviculam et filum, cujus una extremitas alligata  est naviculae, altera cum globo est in }\textit{navi}\protect\index{Sachverzeichnis}{navis}\textit{ ipsa. Et enim immota }\textit{navi}\protect\index{Sachverzeichnis}{navis}\textit{ conceditur naviculae navigatio donec 10 vel 12 fili orgyiis remota sit, et observatur tempus interea} \edtext{\textit{elapsum.}}{\lemma{\textit{elapsum.}}\Bfootnote{\textsc{B. Varenius,} \cite{00109}a.a.O., S.~510f. }}\pend \pstart Observatio Latitudinis\protect\index{Sachverzeichnis}{latitudo} vitiosa saepe, ob navis\protect\index{Sachverzeichnis}{navis} agitationem, et quia oculus non  recte applicatur instrumentis, et quia refractio\protect\index{Sachverzeichnis}{refractio} negligitur. \edtext{}{\lemma{negligitur.}\Bfootnote{\textsc{B. Varenius, }\cite{00109}a.a.O., S.~511. }}\pend 