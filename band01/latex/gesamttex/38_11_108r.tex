\pstart
    \textso{Experim. II.} Sumatur Elaterium\protect\index{Sachverzeichnis}{elater} quodcunque, (ut est lamina ferrea) quod se restituens circumagat rotam \textit{a} et rota \textit{a} circumagendo allevet \edtext{communicantem cordae \textit{ab} Embolum \textit{bc}}{\lemma{allevet}\Afootnote{ \textit{ (1) }\ Embolum\protect\index{Sachverzeichnis}{embolus|textit} \textit{bc} cui chorda \textit{ab} communicat in \textit{ (2) }\ communicantem cordae \textit{ab} Embolum \textit{bc} \textit{ L}}} in Tubo \textit{de}\edtext{}{\lemma{\textit{de}}\Afootnote{\textbar\ exacte \textit{ gestr.}\ \textbar\ adaptatum. \textit{ L}}} adaptatum.
                     Notetur locus in quem Embolus\protect\index{Sachverzeichnis}{embolus} totius Elaterii\protect\index{Sachverzeichnis}{elaterium} restitutione elevatur, qui ponatur esse \textit{c}. Embolus\protect\index{Sachverzeichnis}{embolus} ergo usque ad \textit{c} ascendens oneretur pondere columnae \textit{fg} sive ea sit sicca, sive ex liquore\protect\index{Sachverzeichnis}{liquor} in tubum infuso constet. Quae tantae sit gravitatis\protect\index{Sachverzeichnis}{gravitas}, ut Embolum\protect\index{Sachverzeichnis}{embolus} deprimendo praecise \edtext{descendat}{\lemma{praecise}\Afootnote{ \textit{ (1) }\ deprimat \textit{ (2) }\ descendat \textit{ L}}} infra \textit{f} seu ut 
[108 r\textsuperscript{o}] punctum ejus summum \edtext{sit in \textit{f} quod commodissime fiet si columna \textit{fg} sit liquida cui facile affunditur aliquid aut adimitur. His ita constitutis futurum arbitror ajo, ut quantumcunque augeatur altitudo columnae, servata tantum latitudine seu crassitie eadem nunquam descendere possit amplius quam praecise infra \textit{f} seu ut summum ponderis utcunque aucti punctum semper maneat \textit{g}.}{\lemma{summum}\Afootnote{ \textit{ (1) }\ quod fuit in \textit{f} praecise perveniat in \textit{c} \textit{ (2) }\ sit [...] fiet \textit{(a)}\ per liquorem\protect\index{Sachverzeichnis}{liquor|textit} \textit{(b)}\ si [...] arbitror  \textbar\ ajo \textit{ erg.}\ \textbar\ , ut [...] eadem \textit{(aa)}\ futurum esse ut \textit{(bb)}\ nunquam [...] infra \textit{(aaa)}\ \textit{c} \textit{(bbb)}\ \textit{f} [...] \textit{g}. \textit{ L}}} Cum enim Elaterii\protect\index{Sachverzeichnis}{elaterium} \edtext{pressioni obnitentis tota vis in aequilibrio\protect\index{Sachverzeichnis}{aequilibrium} consistat cum pondere columnae \textit{fg}}{\lemma{Elaterii}\Afootnote{ \textit{ (1) }\ vis aequiponderetur seu destruatur aequipondio columnae contra tensionem\protect\index{Sachverzeichnis}{tensio|textit} \textit{ (2) }\  tensioni\protect\index{Sachverzeichnis}{tensio|textit} obnitentis tota vis aequiponderet \textit{ (3) }\ pressioni [...] \textit{fg} \textit{ L}}} ex \edtext{hypothesi, destruent}{\lemma{hypothesi,}\Afootnote{ \textit{ (1) }\ semper cum eo ae \textit{ (2) }\ destruent \textit{ L}}} se mutuo, ac proinde invariabilia manebunt. Et quicquid ipsis addetur, aget sine impedimento\protect\index{Sachverzeichnis}{impedimentum} rem suam. Quare si augeatur columna quicquid est ultra \textit{fg} descendet infra \textit{fg} idque quousque per naturam Elaterium\protect\index{Sachverzeichnis}{elaterium} tensionis\protect\index{Sachverzeichnis}{tensio} capax est. Hinc patet Elaterii\protect\index{Sachverzeichnis}{elaterium} solidi vim posse altitudine \edtext{suspensi ponderis}{\lemma{}\Afootnote{suspensi ponderis \textit{ erg.} \textit{ L}}} mensurari ex crassitie ponderis data. At si Elaterium\protect\index{Sachverzeichnis}{elaterium} sit liquidum, ut aer Tubo \textit{fe} subjectus nihil refert quanta sit ponderis \textit{fg} vel tubi \textit{fe} crassities, quia liquor\protect\index{Sachverzeichnis}{liquor} in tubo arcto tantum potest quantum in crasso.\footnote{\textit{In der rechten Spalte}: Quemadmodum si Elaterii\protect\index{Sachverzeichnis}{elaterium} tendendi loco alius liquor\protect\index{Sachverzeichnis}{liquor} in Tubo \textit{fghi} esset elevandus. Is enim si a pondere \textit{fg} eousque elevatus est ut pondus\protect\index{Sachverzeichnis}{pondus} illud praecise consistat in loco \textit{fg} quicquid etiam ponderi \textit{fg} addetur, nunquam totam descendet infra \textit{f}.} 
Hoc experimentum \edtext{suspensionem liquidi}{\lemma{experimentum}\Afootnote{ \textit{ (1) }\ si experimento comprobatur \textit{ (2) }\ Mercurii\protect\index{Sachverzeichnis}{mercurius|textit} \textit{ (3) }\ suspensionem liquidi \textit{ L}}} in Tubo Torricelliano\protect\index{Sachverzeichnis}{Tubus!Torricellianus} egregie illustrat. Ostendit enim cur \edtext{eadem semper}{\lemma{cur}\Afootnote{ \textit{ (1) }\ nunquam \textit{ (2) }\ eadem semper \textit{ L}}} sit altitudo Mercurii\protect\index{Sachverzeichnis}{mercurius} suspensi, quantacunque fuerit quantitas delapsi. \edtext{Non tantum cum columna aeris liberi}{\lemma{delapsi.}\Afootnote{ \textit{ (1) }\ Quia sive pondus aeris\protect\index{Sachverzeichnis}{pondus!aeris|textit} liberi, sive Elaterium aeris\protect\index{Sachverzeichnis}{elaterium!aeris|textit} amplius \textit{ (2) }\ Nam  \textit{(a)}\ pondus aeris\protect\index{Sachverzeichnis}{pondus!aeris|textit} liberi prae \textit{(b)}\ massa\protect\index{Sachverzeichnis}{massa|textit} \textit{ (3) }\ Non [...] liberi \textit{ L}}} aequiponderat altitudini Mercurii\protect\index{Sachverzeichnis}{mercurius}, ubi se mutuo destruunt, et quicquid ultra addis a pondere aeris quippe jam destructo non impeditur. \edtext{Uti si \textit{fg} Mercurium in Baroscopio pendulum et \textit{hi} columnam aeris ei aequiponderantem esse ponas, sed et si aer in ampulla \textit{gel} clausus ei sit comprimendus. Nam earundem}{\lemma{impeditur.}\Afootnote{ \textit{ (1) }\ Eadem porro vis est ponderis \textit{ (2) }\ Uti [...] earundem \textit{ L}}}  virium est pondus aeris\protect\index{Sachverzeichnis}{pondus!aeris} liberi et Elaterium\protect\index{Sachverzeichnis}{elaterium} comprimi ultra renitentis clausi. Quia aer noster ut est in statu suo ordinario comprimi ultra ab incumbente \edtext{aeris massa}{\lemma{incumbente}\Afootnote{ \textit{ (1) }\ pondere \textit{ (2) }\ aeris massa \textit{ L}}} non potuit, ac proinde cum ea praecise in aequilibrio\protect\index{Sachverzeichnis}{aequilibrium} constitit.
\pend 
\pstart \textso{Experim. III.} Et ut appareat plus etiam Mercurii\protect\index{Sachverzeichnis}{mercurius} quam Tubi Torricelliani\protect\index{Sachverzeichnis}{Tubus!Torricellianus} altitudo capit delabi posse\edtext{, in}{\lemma{posse}\Afootnote{ \textit{ (1) }\ . Esto \textit{ (2) }\ , in \textit{ L}}} fig. 1. vas \textit{D} plenum Mercurio\protect\index{Sachverzeichnis}{mercurius} cogitetur. Aperiatur Epistomium\protect\index{Sachverzeichnis}{epistomium}, \textit{F} \edtext{quantumcunque Mercurii effluat}{\lemma{\textit{F}}\Afootnote{ \textit{ (1) }\ Mercurius\protect\index{Sachverzeichnis}{mercurius|textit}, qui effluet, \textit{ (2) }\ quantumcunque Mercurii effluat \textit{ L}}} semper tamen \edtext{\textit{BF} ultra Mercurii in subjecto vase horizontem extabit.}{\lemma{tamen}\Afootnote{ \textit{ (1) }\ pendulus manebit \textit{BF} \textit{ (2) }\ suspensus \textit{ (3) }\ \textit{BF} ultra   \textbar\ Mercurii in \textit{ erg.}\ \textbar\  subjecto vase horizontem extabit. \textit{ L}}} Apertum maneat Epistomium\protect\index{Sachverzeichnis}{epistomium}\edtext{. Ajo Mercurium omnem effluxurum}{\lemma{Epistomium}\Afootnote{ \textit{ (1) }\ donec Mercurius\protect\index{Sachverzeichnis}{mercurius|textit} pene omnis ex vase \textit{D} effluat residua \textit{ (2) }\ . Ajo Mercurium omnem effluxurum \textit{ L}}}, nec in vase \textit{D} suspensum iri, nisi Epistomio\protect\index{Sachverzeichnis}{epistomium} clauso fluxus sistatur. \edtext{Etsi ergo}{\lemma{sistatur.}\Afootnote{ \textit{ (1) }\ Quo facto \textit{ (2) }\ Etsi ergo \textit{ L}}} aer in vase \textit{D} tendatur non tamen Mercurius\protect\index{Sachverzeichnis}{mercurius} in eo suspensus manebit ad altitudinem ullam. Quemadmodum contra maneret suspensus ad altitudinem consuetam si ex vase \textit{D} recta non per Tubum \textit{AB} effluxisset. Idem est si \textit{D} sit vesica Mercurio\protect\index{Sachverzeichnis}{mercurius} plena quae comprimi possit, flaccidaque reddi. Ita enim nullus in ea erit aer tensus. Nec proinde aeris tensio\protect\index{Sachverzeichnis}{tensio} Tubique capacitas ad rem pertinet. 
[108 v\textsuperscript{o}] \textso{Experim. IV.} Quantacunque sit Tubi longitudo \edtext{aut Mercurii\protect\index{Sachverzeichnis}{mercurius} altitudo}{\lemma{}\Afootnote{aut Mercurii\protect\index{Sachverzeichnis}{mercurius} altitudo \textit{ erg.} \textit{ L}}} \edtext{ut spatii  regula generalis haec statui potest, ut Mercurium eousque descendere,}{\lemma{altitudo}\Afootnote{ \textit{ (1) }\ ex illa semper ejus altitud \textit{ (2) }\ tantum semper descendet Mercurius\protect\index{Sachverzeichnis}{mercurius|textit}, ut in spatio quod primo descensus momento occupat \textit{ (3) }\ ut spatii  \textit{(a)}\ tantum sui \textit{(b)}\ non minus sui relinquat, quam spatii ante descensum occupati \textit{(c)}\  regula generalis  \textit{(aa)}\ descensus Mercurii\protect\index{Sachverzeichnis}{mercurius|textit} \textit{(bb)}\ haec [...] descendere, \textit{ L}}} donec ejus pollices 27 (30 etc.) summi perveniant in locum pollicum totidem infimorum, seu ut descendat per altitudinem tantam, quanta est ipsiusmet, demtis pollicibus dictis.\edlabel{dict108v1} 
\pend