  \vspace{8mm}
  \footnotesize        			
  \selectlanguage{latin}
\pstart In den folgenden zwei St\"{u}cken behandelt Leibniz das Problem der Bestimmung der Tiefe eines Wassergrabens mit Hilfe eines Stabes, ohne dass man diesen daf\"{u}r aus dem Wasser ziehen muss.\\ Beide Texte wurden von Leibniz mit dem Datum Mai 1675 versehen. In N. 69\raisebox{-0.5ex}{\notsotiny{2}} entwickelt Leibniz das Problem bis zur Formulierung einer Regel. Daraus ergibt sich die Einordnung dieses Textes als die zweite Version der Probleml\"{o}sung.\pend
  \vspace{10mm}
  \normalsize