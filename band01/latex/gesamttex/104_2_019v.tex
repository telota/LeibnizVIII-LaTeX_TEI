[19 v\textsuperscript{o}] secundus casus est, si acus\protect\index{Sachverzeichnis}{acus!magnetica} declinat in contrariam partem a navi\protect\index{Sachverzeichnis}{navis}.  Pone navem\protect\index{Sachverzeichnis}{navis} ut ante declinare ex \textit{d} in \textit{c} acum\protect\index{Sachverzeichnis}{acus!magnetica} ex \textit{d} in \textit{e} manifestum est in idem latus esse mutationem, sive acus\protect\index{Sachverzeichnis}{acus!magnetica}, sive navis\protect\index{Sachverzeichnis}{navis} declinet. Semper enim circulus ibit versus \textit{c} acus\protect\index{Sachverzeichnis}{acus!magnetica} versus \textit{e}. Sed quod discrimen sensibile in hoc motu. Si \edtext{construatur pyxis simul et verticalis  et perpendicularis, id est dupliciter suspensa, poterit inveniri magnetis}{\lemma{construatur}\Afootnote{ \textit{ (1) }\ acus\protect\index{Sachverzeichnis}{acus!magnetica|textit} simul et magnetica \textit{ (2) }\ pyxis [...] magnetis \textit{ L}}} declinatio\protect\index{Sachverzeichnis}{declinatio} sine omni observatione \edlabel{coelstart}coeli.\pend \pstart \edtext{Ponatur enim quoties acus exacte polum respicit.\edlabel{coelend}}{\lemma{coeli.}\xxref{coelstart}{coelend}\Afootnote{ \textit{ (1) }\ Nam si acus\protect\index{Sachverzeichnis}{acus!magnetica|textit}  inclinata exacte respicit polum\protect\index{Sachverzeichnis}{polus|textit}, necesse est eam \textit{ (2) }\ Ponatur enim  \textit{(a)}\ nunc ita locata \textit{(b)}\ quoties acus exacte polum respicit. \textit{ L}}} \pend 