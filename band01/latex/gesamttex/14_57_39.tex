\pend \pstart [p.~39] X. Melius forte, si assumam punctum V\footnote{\textit{Gedruckte Marginalie}: Figur. 29.}, vt enim FV, ad FG, ita VA ad AC; nec dicas, punctum V haberi non posse, nempe facile illud assequar; cognita enim recta FD, quam habeo, et sublata DE, quam etiam habeo, ex FG cognita, vt residuum ad AE, ita FD ad DV, vnde tota FV habetur; nec dicas, idem restare incommodum, quod scilicet non habeatur punctum F, vt pote insensibile, esto enim non habeatur, rectae tamen 