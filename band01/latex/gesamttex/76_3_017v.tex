[17~v\textsuperscript{o}] Cum saepe mecum cogitarem, quantum a perfectione Optices in res humanas redundare utilitatis necesse sit; pandente nobis natura arcanos sinus, faciemque mundi centuplicante, atque insensibiles illas machinas detegente, quibus pleraeque etiam in corporibus nostris in peius meliusque mutationes peraguntur: officii mei esse putavi, nonnihil temporis, quod mihi plurimum distracto exiguum superest, impendere scientiae tanti fructus; sed more scilicet atque instituto meo, quo assuevi eam operam studiis insumtam pro perdita habere, qua didici tantum, sed quod adjicerem non inveni.\pend \pstart  Occasio rem penitius scrutandi haec fuit: Diu est, ut amici norunt, quod mihi in mentem venit \textso{ratio,} quaedam \textso{Optica metiendi ex una statione distantias magnitudinesque veras objectorum,} ita comparata, ut spes sit ad coelestia usque, ultra parallaxes\protect\index{Sachverzeichnis}{parallaxis}, extendi posse quando et fundamentum illud cui innititur, eousque vim notabiliter exerit.\pend \pstart  Hanc cum nuper poliendam resumerem fuit in optices interiora inquirendum paulo diligentius, atque inprimis cogitandum de figuris quibusdam, quas ego novis nominibus (quando et res aliis intacta est) \textso{Isoptricas}\protect\index{Sachverzeichnis}{isoptrica}\textso{, }\textso{Dioptricas,}\protect\index{Sachverzeichnis}{dioptrica} et \textso{Paroptricas}\protect\index{Sachverzeichnis}{paroptrica} appellavi: quibus objecta aequiapparentia, superficies ordinate refringentes aut reflectentes, denique imago ejusdem objecti, focique\protect\index{Sachverzeichnis}{focus} ejusdem puncti, (nullius enim puncti focus\protect\index{Sachverzeichnis}{focus} unus, distantia quavis, nullius objecti unica ima~