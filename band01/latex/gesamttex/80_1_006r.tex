   
        
        \begin{ledgroupsized}[r]{120mm}
        \footnotesize 
        \pstart        
        \noindent\textbf{\"{U}berlieferung:}  
        \pend
        \end{ledgroupsized}
      
       
              \begin{ledgroupsized}[r]{114mm}
              \footnotesize 
              \pstart \parindent -6mm
              \makebox[6mm][l]{\textit{L}}Notiz: LH XXXVII 2 Bl. 6. 1 Bl. dreieckig, L\"{a}nge der Katheten 9 und 16 cm. 19 Zeilen, R\"{u}ckseite leer.\\Kein Eintrag in KK 1 oder Cc 2. \pend
              \end{ledgroupsized}
        %\normalsize
        \vspace*{5mm}
        \begin{ledgroup}
        \footnotesize 
        \pstart
      \noindent\footnotesize{\textbf{Datierungsgr\"{u}nde}: Wir ordnen dieses St\"{u}ck in Leibniz' fr\"{u}he Auseinandersetzung mit dem Cartesischen Brechungsgesetz ein. Im Unterschied zu N. 21 wird nun die actio instantanea zur Begr\"{u}ndung herangezogen.}
        \pend
        \end{ledgroup}
      
        \vspace*{8mm}
        \pstart 
        \normalsize
       \centering [6 r\textsuperscript{o}] Ratio aequalitatis angulorum reflexionis\protect\index{Sachverzeichnis}{angulus!reflexionis}\protect\index{Sachverzeichnis}{angulus!reflexionis|see{angle de r\'{e}flexion}} et incidentiae\protect\index{Sachverzeichnis}{angulus!incidentiae}\pend \vspace{1.0ex} \pstart quam Cartesius\protect\index{Namensregister}{\textso{Descartes} (Cartesius, des Cartes, Cartes.), Ren\'{e} 1596\textendash 1650}\edtext{}{\lemma{quam}\Bfootnote{\textsc{R. Descartes, }\cite{00038}\textit{La dioptrique}, Leiden 1637, S.~21 (\textit{DO} VI, S.~103.)}} attulit videtur jam allata fuisse a Proclo\protect\index{Namensregister}{\textso{Proclus,} 410\textendash 485} lib. 1. com. in Euclid. cap. 4.\edtext{}{\lemma{cap. 4.}\Bfootnote{\textsc{Heliodor v. Larissa}, \cite{00000}\textit{Opticorum libri duo}, Paris 1657, S.~118f. }} Ptolemaei\protect\index{Namensregister}{\textso{Ptolemaeus,} Claudius v. Alexandria 85?\textendash 165?} ratio de minima in speculis concavis\protect\index{Sachverzeichnis}{speculum!concavum}\edtext{}{\lemma{in}\Bfootnote{\textsc{Heliodor v. Larissa, } \cite{00000}a.a.O., S.~113. }} \edtext{non succedit, si subintelligatur Tangens}{\lemma{succedit,}\Afootnote{ \textit{ (1) }\ applicari potest ad tangen \textit{ (2) }\ si subintelligatur Tangens \textit{ L}}}, non tamen res succedere videtur, nec forte remedium meum valet, de actione instantanea quod Fermatianae opinioni adhibui. \pend 