\pend \pstart [p.~192] Denique cum nullus vnquam cometes\protect\index{Sachverzeichnis}{cometa} maiorem arcum\protect\index{Sachverzeichnis}{arcus} semicirculo circuli sui maximi decurrerit,\footnote{\textit{Leibniz unterstreicht}: cum nullus [...] decurrerit} ad quem tamen is proxime accessit, qui an. 1472. visus est, et quem Pontanus\protect\index{Namensregister}{\textso{Pontano,} Giovanni Giovano 1426\textendash 1503}  descripsit; [...] cum igitur nullus ex coelestibus globis infra stellas\protect\index{Sachverzeichnis}{stella} in circulo moueatur, qui suum integrum  orbem non conficiat, et cometa\protect\index{Sachverzeichnis}{cometa} nunquam integrum semicirculum decurrat, huius rei germana ratio est, si dicamus, cometam\protect\index{Sachverzeichnis}{cometa} moueri in linea recta.