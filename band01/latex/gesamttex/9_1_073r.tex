[73 r\textsuperscript{o}] \textso{Astronom.}\\
Mechan.\\
\textso{Pneumat.}\\
\textso{Meteorolog.}\\
Hydrostat.\\
\textso{Naut.}\\
Magnet.\\
\textso{Physica}\\
Botanica\\
\textso{Anatom.}\\
Medica\\
\textso{Miscellan.}
\pend
\pstart \edtext{Schreiben\edlabel{schreibenstart}}{{\xxref{schreibenstart}{schreibenend}}\lemma{Schreiben}\Bfootnote{Dieser Absatz von \textit{Schreiben} bis \textit{Carniol auszusehen.} eingef\"{u}gt aus der Fortf\"{u}hrung der Miscellanea-Rubrik auf Bl. 73 r\textsuperscript{o}.}} das man die hande nicht besudelt, U) mit waßer, wird schwarz aufen papyr. Schreiben das die buchstaben scheinen l\"{a}ngst geschrieben zu seyn, man neze die wort mit \protect\includegraphics[width=0.025\textwidth]{images/oleum.pdf} Terp.\protect\index{Sachverzeichnis}{Terpentin} und wenich oder viel clar waßer U). Leder\protect\index{Sachverzeichnis}{Leder} bereiten ohne eichen\protect\index{Sachverzeichnis}{Eiche} oder andre Rinde\protect\index{Sachverzeichnis}{Rinde} U). Tortoisen\protect\index{Sachverzeichnis}{tortoise} schahlen moulden, wenn man sie durch ein menstruum weich gemacht. Ein secr. eines   amber\edtext{}{\lemma{amber}\Bfootnote{Diese Notiz m\"{o}glicherweise in Zusammenhang mit einem St\"{u}ck Bernstein aus D\"{a}nemark an Oldenburg, vgl. \cite{00154}\textit{BH} III, S.~75, oder \textsc{R. Boyle}, \cite{00155}\textit{Usefulness}, Teil II, S. 22 (\textit{BW} 3, S.~308).}} im Hag hat holz gemouldet; weis nicht ob per aggluti[n]ationem pulveris. Dessen, eine approximation ich gemacht, mit ichthyocolla\protect\index{Sachverzeichnis}{ichthyocolla} bey Hereo\protect\index{Namensregister}{\textso{Hereo}} der alte Zeugwart, wo mir recht aus Terpentin\protect\index{Sachverzeichnis}{Terpentin} figuren gegossen so wie Carniol\protect\index{Sachverzeichnis}{Carniol} auszusehen\edlabel{schreibenend}.\pend