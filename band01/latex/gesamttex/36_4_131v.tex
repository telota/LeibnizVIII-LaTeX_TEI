[131 v\textsuperscript{o}] \selectlanguage{french}\textso{Conseq. 6.} Il semble qu'on peut tirer de ces phenomenes ensemble une observation generalle, s\c{c}avoir, \textso{que la nature tache d'empecher la discontinuation des corps sensibles.}\protect\index{Sachverzeichnis}{corps!sensible} Car même dans le vuide\protect\index{Sachverzeichnis}{vide} \edtext{ou plustost Recipient \'{e}puis\'{e},}{\lemma{}\Afootnote{ou plustost Recipient \'{e}puis\'{e}, \textit{ erg.} \textit{ L}}} deux corps solides\protect\index{Sachverzeichnis}{corps!solide} ne se separent pas ais\'{e}ment comme on voit par le \textso{phenom. 9.} des placques, ny\edtext{}{\lemma{}\Afootnote{ny  \textbar\ même \textit{ gestr.}\ \textbar\ deux \textit{ L}}} deux liquides, par le \textso{phen. 10.} du siphon\protect\index{Sachverzeichnis}{siphon} \`{a} jambes in\'{e}galles, ny un solide d'un liquide, par \textso{les phenomenes 5. et 7.} de la liqueur purg\'{e}e\protect\index{Sachverzeichnis}{liqueur!purg\'{e}e}. Mais aussitost qu'un corps\protect\index{Sachverzeichnis}{corps}\edtext{}{\lemma{}\Afootnote{corps  \textbar\ sensible \textit{ gestr.}\ \textbar\ qui \textit{ L}}} qui se peut \'{e}tendre ou prendre un plus grand volume \edtext{comme l'air}{\lemma{}\Afootnote{comme l'air \textit{ erg.} \textit{ L}}} est interpos\'{e}, \edtext{les corps joints}{\lemma{interpos\'{e},}\Afootnote{ \textit{ (1) }\ alors l'union cesse \textit{ (2) }\ les corps  \textbar\ attachez \textit{ erg. u.}\  \textit{ gestr.}\ \textbar\  joints \textit{ L}}} reprennent la libert\'{e} de se separer.\selectlanguage{latin}\pend 