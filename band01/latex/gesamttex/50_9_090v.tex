[90 v\textsuperscript{o}]  si jam hoc planum in praxi consideretur ut focus\protect\index{Sachverzeichnis}{focus}, manifesta est  ratio quam hi circuli ad se mutuo habent, nempe cum \textit{ND}  est 1, quod tum \textit{KH} minor esse debeat quam $\displaystyle2\frac{6}{7}$% \begin{wrapfigure}{l}{0.4\textwidth}                    
                %\includegraphics[width=0.4\textwidth]{../images/Zu+Johann+Hudde%2C+Specilla+circularia/LH037%2C02_090v/files/100056.png}
                        %\caption{Bildbeschreibung}
                        %\end{wrapfigure}
                        %@ @ @ Dies ist eine Abstandszeile - fuer den Fall, dass mehrere figures hintereinander kommen, ohne dass dazwischen laengerer Text steht. Dies kann zu einer Fahlermeldung fuehren. @ @ @ \\
                    .\rule[-4mm]{0mm}{10mm}\pend \pstart  Debet itaque figura vitri concipi eadem quam describeret \textit{HQBD}\footnote{\textit{Am Rand}: fig. 2.} rotata circa axem \textit{DH}. Et notandum est, non  necesse esse, ut postquam una superficies vitri polita est, ex:  gr: convexa \textit{RDB}, ad alteram poliendam, centrum \textit{K} maneat  in axe \textit{DNK}, prout accurate attendendum esset, si \textit{ADB}  esset Ellipsis, aut hyperbola, ut planum hanc secans esset  ad angulos rectos ad axem: sed tantum videndum est, ut  maxima vitri crassities, mensurata secundum perpendicularem  in convexam et concavam superficiem incidentem, aequalis  sit \textit{DH}.
                    \pend 
                    \pstart  Magnitudo porro segmentorum omnium circulorum,  quatenus parallelos radios in unum punctum mechanicum\protect\index{Sachverzeichnis}{punctum!mechanicum}  congregant, facile aut per suprapositum calculum, aut  per ipsam experientiam inveniri potest.
                    \pend 
                    