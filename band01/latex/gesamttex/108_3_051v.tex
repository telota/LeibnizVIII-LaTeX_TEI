[51 v\textsuperscript{o}] unius suber majoris diametri quam alterum, et ita se juvabunt, et ex loco cui insistit stylus suberis superioris poterit suspendi filum terrellae inferioris. Ne autem omnia concutiantur sint omnia disposita, ut quantumcunque jactata nave\protect\index{Sachverzeichnis}{navis} disponant se perpendiculariter ad horizontem sed ne noceant sibi invicem si similes obvertantur \edtext{poli\protect\index{Sachverzeichnis}{polus}}{\lemma{}\Afootnote{poli\protect\index{Sachverzeichnis}{polus} \textit{ erg.} \textit{ L}}} cavendum est. Ergo unus erit positus in polo\protect\index{Sachverzeichnis}{polus} Austrino, \edtext{proximus}{\lemma{Austrino,}\Afootnote{ \textit{ (1) }\ alius \textit{ (2) }\ proximus \textit{ L}}} in Boreali. Sed ita unum punctum magnetis\protect\index{Sachverzeichnis}{magnes} erit orientale, quod in alio est occidentale observante Schotto\protect\index{Namensregister}{\textso{Schott} (Schottus), Caspar SJ 1608\textendash 1666}, sed quid tum? nobis hoc loco solius lineae meridianae, non orientis et occidentis cura est deinde potest esse tanta polorum\protect\index{Sachverzeichnis}{polus} distantia, ut nihil intersit, quomodocunque locentur. Caeterum notabile est quod P. Kircher\protect\index{Namensregister}{\textso{Kircher} (Kircherus), Athanasius SJ 1602\textendash 1680} observavit, si Magnes\protect\index{Sachverzeichnis}{magnes} ponatur ita axe orthogonali et circumagatur circa axem versorium intra sphaeram licet ejus positum, nihil conmotum iri \cite{00067}\textit{Art. Magn.} lib. 1. p. 2. prop. 13. fin. experimento 2. consectar. 2.\edtext{}{\lemma{consectar. 2.}\Bfootnote{\textsc{A. Kircher, }\cite{00067}\textit{Magnes}, Rom 1654, S.~69. }} Contra nostram rationem \edtext{procurandae quietis in insistente rotato}{\lemma{rationem}\Afootnote{ \textit{ (1) }\ meditatam de quiete insistentis \textit{ (2) }\ procurandae  \textbar\ de \textit{streicht \hspace{2mm}Hrsg.}\ \textbar\ quietis in insistente \textit{(a)}\ moto \textit{(b)}\ rotato \textit{ L}}} licet sustentate, hoc unum maxime obstat, quod aucto pondere insistentis, ut rotanti magis obsistat, perit tanto magis libertas, tanto magis enim infigit se ei et insensibiles velut lacunas imprimit. Non tamen despero plane, quin res procurari possit. Sed si Grandamici\protect\index{Namensregister}{\textso{Grandami} (Grandamicus), Jacques SJ 1588\textendash 1672} inventum exactum est, eo carere possumus, et demto eo, etsi difficiliore, tamen non incertiore \edtext{aut minus universali}{\lemma{}\Afootnote{aut minus universali \textit{ erg.} \textit{ L}}} ratione omnia peragere possumus. In eo est circa filum difficultas quod avertente se Magnete\protect\index{Sachverzeichnis}{magnes} implicatur et contorquetur, \edtext{v.g.}{\lemma{contorquetur,}\Afootnote{ \textit{ (1) }\ cum tamen nunquam fieri possit ut nimis in latus \textit{ (2) }\ v.g. \textit{ L}}} \edtext{nave se torquente in circulum}{\lemma{v.g.}\Afootnote{ \textit{ (1) }\ magnete\protect\index{Sachverzeichnis}{magnes|textit} se torquente in circulum \textit{ (2) }\ nave se torquente in circulum \textit{ L}}}. Et deinde remisso in contrarium filo, erit laxum et non amplius sustinens, forte commodius adhibebitur filum ferreum, \edtext{vel adhaereat magneti\protect\index{Sachverzeichnis}{magnes}, aut alii filo quo casu etsi obsistit detractioni non tamen forte obsistet gyrationi}{\lemma{}\Afootnote{vel adhaereat magneti\protect\index{Sachverzeichnis}{magnes}, [...] gyrationi \textit{ erg.} \textit{ L}}}, quod superius se contorquere possit.\footnote{\textit{In der rechten Spalte}: NB. Videndum haec esset optima libratio rerum. Item fortasse effici potest arte aliqua ut quod ascendit attractum ab magnete\protect\index{Sachverzeichnis}{magnes} ubi satis ascendit non ultra ascendere possit. Finge continere aciculas, quas magnes\protect\index{Sachverzeichnis}{magnes} datus hinc repellit, illinc attrahit, et inter ascendendum machina quadam converti ut jam oppositam partem obvertant, et ita difficilius attrahantur. Item fiat machina quaedam in aere perpetuo manens vi Elastica\protect\index{Sachverzeichnis}{vis!elastica} circularis sese in locum priorem motu perpetuo Navem\protect\index{Sachverzeichnis}{navis} tamen sequens quia superior pars adhaeret magneti\protect\index{Sachverzeichnis}{magnes} divulsioni quidem non tamen gyrationi resistenti. Et haec ait forte perfectissima ad retinendum semper situm priorem, seu ad habendum indicem qui semper praecise monstret locum portus soluti. Et ita careri potest Loxodromiis\protect\index{Sachverzeichnis}{loxodromia}.} Sed quomodo cum debeat alicui inniti. Innitatur igitur, sed non nisi fere in puncto. Aut potius sic ut innitatur \selectlanguage{ngerman}wie in einer schrauben.\selectlanguage{latin} Sed ac hoc amplius videndum. Res enim dubia est, et ideo omnibus modis quaerenda tuta quaedam et stabilis magnetis\protect\index{Sachverzeichnis}{magnes} libratio. \pend \pstart Versorium non debet longioris esse radii, quam est Sphaera activitatis \edtext{magnetis.} {\lemma{magnetis.}\Afootnote{ \textit{ (1) }\ Nota \textit{ (2) }\ Kircher \textit{ L}}} Kircher Lib. II. part. 1. progymn. 3. de versoriis pragm. 1.\edtext{}{\lemma{pragm. 1.}\Bfootnote{\textsc{A. Kircher}, \cite{00067}a.a.O., S.~132f. }} ubi et observat majoris magnetis\protect\index{Sachverzeichnis}{magnes} majorem esse sphaeram activitatis quam minoris fortioris, sed ut majorem, ita debiliorem, rem quidem distantiorem, sed non graviorem attrahit. Fortasse posset Terrella librari in hydrargyro, quippe quod stabilius aqua, et tamen gyrationem magnetis\protect\index{Sachverzeichnis}{magnes} in subere non impediens, sed dubito, quia nimis crassum est. Fortasse tamen libratio fieri potest hoc solo modo ut \edtext{pileolus}{\lemma{ut}\Afootnote{ \textit{ (1) }\ vi \textit{ (2) }\ pileolus \textit{ L}}} supra centrum suberis conicus \edtext{chalybius}{\lemma{conicus}\Afootnote{ \textit{ (1) }\ aeneus \textit{ (2) }\ chalybius \textit{ L}}} supra rursus \edtext{in magnetem}{\lemma{rursus}\Afootnote{ \textit{ (1) }\ laxatus in patellam et \textit{ (2) }\ in magnetem \textit{ L}}}