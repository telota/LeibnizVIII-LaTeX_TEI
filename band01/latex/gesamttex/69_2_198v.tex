[198 v\textsuperscript{o}] aquae attrahere, aut primae regionis diducere potest. Aperiat sibi regionem secundam. Sed ne primae diductae etiam sustinendae onere gravetur, eam pessulo ita inserat, ut cogatur diducta manere, ipsum vero pondus \textit{m} eadem opera qua primae \edtext{regionis tabulam inferiorem pessulo inserit}{\lemma{regionis}\Afootnote{ \textit{ (1) }\ pessulo inseri \textit{ (2) }\ tabulam \textit{(a)}\ imam \textit{(b)}\ inferiorem pessulo inserit \textit{ L}\ %hier eingefuegt aus 198r%
\hspace{10mm} 23 \hspace{3mm} trahunt, \textit{ (1) }\ quod \textit{ (2) }\ nec \textit{ L }\ \hspace{10mm} 24 \hspace{3mm} sunt \textbar\ sub \textit{ gestr.}\ \textbar\ divisiones. \textit{ L}}}, ne possit redire ad superiorem, secundae regionis inferiorem a pessulo liberet, ut possit diducere a superiore. Ita diducet etiam secundam regionem quantum potest, et in eam attrahit aquam; et sic multiplicari possunt regiones. Et potest elevari a pondere \textit{m} decuplum ponderis sui per partes. Quo facto liberentur pessula, ita ut omnium siphonum\protect\index{Sachverzeichnis}{siphon} aquas simul sustinere cogatur. Hoc facto statim superabitur a pondere decuplo, et sursum attolletur omnium nisu in eum statum, ut non quidem omnino omnes claudantur folles praeter primum, sed ut ea omnibus aliquid, quantum eorum diducere et aquae in siphones\protect\index{Sachverzeichnis}{siphon} attollere potuisset pondus simul. Et etsi non restituatur sic in statum priorem quem initio descripsimus, incidet tamen in eum statum quem potuissemus initio describere, si posuissemus folles initio nullo pessulo clausos, et ab eo tempore pessulos incepissemus, quo ex omnibus simul nihil potuisset amplius diducere. Et hoc statu semper idem aget pondus, semperque in eandem altitudinem restituetur. Et suberit elegans contemplatio, quomodo folles et siphones\protect\index{Sachverzeichnis}{siphon} pro altitudine capacitateque inter se proportiones partiantur, et si omnium repetitiones sint aeque diuturnae ut in pendulis\protect\index{Sachverzeichnis}{pendulum} et chordis. Nec capere ego possum quid huic machinamento objici queat. Porro aquae loco potest praestari mercurio\protect\index{Sachverzeichnis}{mercurius}, et mercurii\protect\index{Sachverzeichnis}{mercurius} loco subtilissima arena seu pulvere. Ut proinde liquidis proprie dictis non omnino opus sit. Et certe Mercurius\protect\index{Sachverzeichnis}{mercurius} nihil \edtext{nisi subactissima}{\lemma{nisi}\Afootnote{ \textit{ (1) }\ dilutissima subactissimaque \textit{ (2) }\ subactissima \textit{ L}}} arena est, quod tactus ipse facile deprehendit. \pend \pstart Nunc restat ut tentemus an idem communibus potentiis mechanicis ponderibusque adhibitis praestari possit. Nam alioquin notandum est etiam follem esse addendum potentiis mechanicis communibus etc. quia efficit, ut pondus moveatur tarde onus celeriter, et onus sit multipliciter divisum, pondus simplex quod in communibus potentiis est inversum. Sed eodem res redit pondere onera, an onere pondera movere velis. Sed nos si fieri potest ad communia pondera vectemque\protect\index{Sachverzeichnis}{vectis} et trochleam\protect\index{Sachverzeichnis}{trochlea} \edtext{imo et vim Elasticam rem transferemus}{\lemma{trochleam}\Afootnote{ \textit{ (1) }\ rem transferemus \textit{ (2) }\ imo [...] transferemus \textit{ L}}}. \pend