      
               
                \begin{ledgroupsized}[r]{120mm}
                \footnotesize 
                \pstart                
                \noindent\textbf{\"{U}berlieferung:}   
                \pend
                \end{ledgroupsized}
            
              
                            \begin{ledgroupsized}[r]{114mm}
                            \footnotesize 
                            \pstart \parindent -6mm
                            \makebox[6mm][l]{\textit{L}}Notiz: LH XXXVII 2 Bl. 8. 1 Bl. 6 x 12 cm. 1 S., 11 Zeilen. Rechter und unterer Seitenrand mit Verlust der rechten unteren Ecke beschnitten. R\"{u}ckseite leer.\\Kein Eintrag in KK 1 oder Cc 2. \pend
                            \end{ledgroupsized}
                %\normalsize
                \vspace*{5mm}
                \begin{ledgroup}
                \footnotesize 
                \pstart
            \noindent\footnotesize{\textbf{Datierungsgr\"{u}nde}: Leibniz hat Johann Hudde\protect\index{Namensregister}{\textso{Hudde} (Huddenius), Jan 1628\textendash 1704} im November 1676 auf dem Weg von Paris\protect\index{Ortsregister}{Paris (Parisii)} nach Hannover\protect\index{Ortsregister}{Hannover} besucht. Es ist anzunehmen, dass er bei dieser Gelegenheit die Dissertation, von der in unserem St\"{u}ck die Rede ist, gesehen hat. F\"{u}r die Datierung gehen wir davon aus, dass sich Leibniz kurz danach eine Notiz \"{u}ber den Inhalt der Dissertation angefertigt hat.}
                \pend
                \end{ledgroup}
            
                \vspace*{8mm}
                \pstart 
                \normalsize
            [8 r\textsuperscript{o}] Hugenius\protect\index{Namensregister}{\textso{Huygens} (Hugenius, Vgenius, Hugens, Huguens), Christiaan 1629\textendash 1695} invenit modum quo \textit{radii }\textit{lucis}\protect\index{Sachverzeichnis}{lux}\textit{ ad punctum aliquod tendentes ope superficiei sphaericae ad datum aliud punctum omnes accurate cogi possint,} Schoten.\protect\index{Namensregister}{\textso{Schooten} (Schoten), Frans van 1615\textendash 1660} ad \textit{Geom.} Cartes.\protect\index{Namensregister}{\textso{Descartes} (Cartesius, des Cartes, Cartes.), Ren\'{e} 1596\textendash 1650} lit. OO. pag. 270. \edtext{}{\lemma{pag. 270.}\Bfootnote{\textsc{R. Descartes, }\cite{00036}\textit{Geometria}, Teil 1, Frankfurt 1659, S.~270. }}\pend \pstart  Vidi et Manuscriptam diss. Huddenii\protect\index{Namensregister}{\textso{Hudde} (Huddenius), Jan 1628\textendash 1704} lingua Belgica conscriptam, ubi modum ostendit \edtext{ope superficiei sphaericae}{\lemma{}\Afootnote{ope superficiei sphaericae \textit{ erg.} \textit{ L}}} quo omnes radii ad punctum aliquod tendentes refringi possunt ad datum aliud punctum, vel modum quo ex dato puncto venientes ita refringi possunt ac si ex alio dato puncto venirent. Fatetur tamen non esse adeo magni usus in praxi. \pend 