\pend \pstart [p.~122] XI. Alius modus est, isque, ni fallor, nouus, quo scilicet conuexum maioris sphaerae compensatur; sit enim vitrum obiectiuum cuius antica facies versus obiectum, sit conuexa,\footnote{\textit{Gedruckte Marginalie}: Fig. 100.} altera versus oculum\protect\index{Sachverzeichnis}{oculus}, sit caua, v.g. AC\footnote{\textit{Am Rand angestrichen:} Alius modus [...] v.g. AC} [p.~123] iam constat ex dictis 1. si AD sit recta, et G centrum circuli ABC, distantia foci\protect\index{Sachverzeichnis}{focus} a puncto B erit dupla GB, sit in S. 2. si accipiatur DK tripla GB, sitque arcus ADC descriptus ex centro K, cum radius refractus\protect\index{Sachverzeichnis}{radius!refractus} OK cadat perpendiculariter in arcum IDC, nullo modo refringetur, vnde focus erit in K.\selectlanguage{latin}