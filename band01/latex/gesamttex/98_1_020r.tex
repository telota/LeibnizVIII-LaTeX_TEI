   
        
        \begin{ledgroupsized}[r]{120mm}
        \footnotesize 
        \pstart        
        \noindent\textbf{\"{U}berlieferung:}  
        \pend
        \end{ledgroupsized}
      
       
              \begin{ledgroupsized}[r]{114mm}
              \footnotesize 
              \pstart \parindent -6mm
              \makebox[6mm][l]{\textit{L}}Konzept: LH XXXVIII Bl. 19\textendash20. 1 Bog. 4\textsuperscript{o}. 1 3/4 S. Textfolge: Bl. 20 r\textsuperscript{o}, Bl. 19 v\textsuperscript{o}, unteres Drittel. Am oberen Rand von Bl. 19 v\textsuperscript{o} die Nebenrechnungen. Darunter 4 Zeilen zu N. 13\raisebox{-0.5ex}{\tiny{3}}. geh\"{o}rig. Danach die Zeichnung. Die folgenden 5 Zeilen ebenfalls zu N. 13\raisebox{-0.5ex}{\tiny{3}} geh\"{o}rig. Die verbleibenden 2 S. enthalten weitere Teile von N. 13\raisebox{-0.5ex}{\tiny{3}} auf Bl. 19 r\textsuperscript{o} und Bl. 20 v\textsuperscript{o}. Auf Bl. 20 v\textsuperscript{o} auch N. 13\raisebox{-0.5ex}{\tiny{2}}. \pend
              \end{ledgroupsized}
       
              \begin{ledgroupsized}[r]{114mm}
              \footnotesize 
              \pstart \parindent -6mm
              \makebox[6mm][l]{\textit{E}}\cite{00243}\textsc{Gerland} 1906, S.~201.\\Cc 2, Nr. 477 tlw. \pend
              \end{ledgroupsized}
        \vspace*{8mm}
        \newpage
        \pstart \centering
        \normalsize
      [20 r\textsuperscript{o}] Inventio Meridianorum\protect\index{Sachverzeichnis}{meridianus} supposita veritate inclinationum\protect\index{Sachverzeichnis}{inclinatio} \edlabel{magnetostart}magneticarum.%\end{center}
      \pend \vspace{1.0ex} \pstart \edtext{Cum mutetur inclinatio acus\protect\index{Sachverzeichnis}{acus!magnetica} mutata elevatione poli\edlabel{magnetoend}}{{\xxref{magnetostart}{magnetoend}}\lemma{magneticarum.}\Afootnote{ \textit{ (1) }\ Suppono (1) \textit{(a)}\ constare elevationem Poli \protect\index{Sachverzeichnis}{elevatio!poli|textit} seu Parallelum\protect\index{Sachverzeichnis}{circulus parallelus|textit} ex inclinatione\protect\index{Sachverzeichnis}{inclinatio|textit} magnetica \textit{(b)}\ mutata elevatione Poli\protect\index{Sachverzeichnis}{elevatio!poli|textit} seu Parallelo\protect\index{Sachverzeichnis}{circulus parallelus|textit} mutari constanter inclinationem\protect\index{Sachverzeichnis}{inclinatio|textit} magneticam (2) constare de celeritate\protect\index{Sachverzeichnis}{celeritas|textit} navis\protect\index{Sachverzeichnis}{navis|textit}, seu quanto tempore quantum spatii absolvat, quod adhibitis rotis\protect\index{Sachverzeichnis}{rota|textit} numericis facile sciri continue potest. (3) Suppono constare nobis \textit{ (2) }\ Si verum est \textit{(a)}\ inclinationes\protect\index{Sachverzeichnis}{inclinatio|textit} \textit{(b)}\ inclinari acum\protect\index{Sachverzeichnis}{acus!magnetica|textit} tanto \textit{(c)}\ mutata inclinatione\protect\index{Sachverzeichnis}{inclinatio|textit}. \textit{ (3) }\ Cum [...] poli \textit{ L}}}, ex Hypothesi, \edtext{sequitur}{\lemma{Hypothesi,}\Afootnote{ \textit{ (1) }\ et elevatio Poli\protect\index{Sachverzeichnis}{elevatio!poli|textit}  \textbar\ seu latitudo\protect\index{Sachverzeichnis}{latitudo|textit} \textit{ erg.}\ \textbar\ mutetur magis minusve \textit{ (2) }\ quanto meridianus\protect\index{Sachverzeichnis}{meridianus|textit} seu longi \textit{ (3) }\ sequitur \textit{ L}}} construi posse pyxidem\protect\index{Sachverzeichnis}{pyxis!nautica} Horizonti \edtext{perpendicularem}{\lemma{Horizonti}\Afootnote{ \textit{ (1) }\ parallelam \textit{ (2) }\ perpendicularem \textit{ L}}}, quae monstret exacte, quando vel unico miliari magis quam ante a polo\protect\index{Sachverzeichnis}{polus} recessimus. \edtext{Etsi enim inaequali proportione crescant decrescantve inclinationes et elevationes, constat}{\lemma{recessimus.}\Afootnote{ \textit{ (1) }\ Constat enim versus Polum\protect\index{Sachverzeichnis}{polus|textit} \textit{ (2) }\ Etsi [...] constat \textit{ L}}} tamen in Regionibus circumpolaribus 5. circiter gradus elevationis, mutare duos inclinationis\protect\index{Sachverzeichnis}{inclinatio}, in regionibus aequatori\protect\index{Sachverzeichnis}{aequator} vicinis contra unum gradum \edtext{elevationis}{\lemma{gradum}\Afootnote{ \textit{ (1) }\ inclinationis\protect\index{Sachverzeichnis}{inclinatio|textit} \textit{ (2) }\ elevationis \textit{ L}}} mutare 5. inclinationis\protect\index{Sachverzeichnis}{inclinatio}, versus aequatorem\protect\index{Sachverzeichnis}{aequator}, et in mediis \edtext{magis}{\lemma{mediis}\Afootnote{ \textit{ (1) }\ circiter \textit{ (2) }\ magis \textit{ L}}} pari passu ambulare. Nec fere unquam major differentiae proportio est, quam ut 1. ad 5. Porro quando inclinationis\protect\index{Sachverzeichnis}{inclinatio} mutatio celerior, tanto est sensibilior utique elevationis. Sed \edtext{fingamus}{\lemma{Sed}\Afootnote{ \textit{ (1) }\ ponamus \textit{ (2) }\ suppono \textit{ (3) }\ fingamus \textit{ L}}} semper inclinationem\protect\index{Sachverzeichnis}{inclinatio} esse quinquies tardiorem elevatione, \edtext{cumque}{\lemma{elevatione,}\Afootnote{ \textit{ (1) }\ eo casu \textit{ (2) }\ cumque \textit{ L}}} unum miliare Italicum \edtext{sit minutum unum gradus sequitur data quinta parte minuti primi}{\lemma{Italicum}\Afootnote{ \textit{ (1) }\ sit sexta pars minuti primi \textit{ (2) }\  \textbar\ circiter \textit{ gestr.}\ \textbar\ sit [...] primi \textit{ L}}} deprehendi inclinatio\protect\index{Sachverzeichnis}{inclinatio}nis mutationem, etiam quando est tardissima. \edtext{Pone}{\lemma{tardissima.}\Afootnote{ \textit{ (1) }\ Necesse est ergo \textit{ (2) }\ Pone \textit{ L}}} pyxidem \protect\index{Sachverzeichnis}{pyxis} divisam minimum in\footnote{\textit{Nebenrechnungen auf Bl. 19 v\textsuperscript{o} zur gestrichenen Textvariante:}\\
      \vspace{2.0ex}
      $\protect\begin{array}{l} \hspace{5.5pt}3\\\cancel{2}\cancel{1}6\\\hspace{5.5pt}\cancel{3}\cancel{6}\protect\end{array}$\protect\rule[-6mm]{0.1mm}{14mm}$\protect\begin{array}{l}\\00\hspace{5.5pt}f\hspace{5.5pt}6\\00\protect\end{array}$
      \hspace{7mm} $\protect\begin{array}{l} 600\\6\\\overline{3600}\\\hspace{2mm}6\\\overline{~~00}
      %\protect\rule{25pt}{0.5pt}
      %\\\hspace{4mm}00
      \protect\end{array}$ [\textit{Rechnung bricht ab}]\hspace{7mm} $\protect\begin{array}{l} 5400\\\hspace{5.5pt}4\\\overline{21600}\protect\end{array}$\hspace{7mm} $\protect\begin{array}{l} 3600\\\hspace{5.5pt}3\\\overline{10800}\protect\end{array}$} \edtext{secunda minuta}{\lemma{in}\Afootnote{ \textit{ (1) }\ 10800 partes \textit{ (2) }\ secunda minuta \textit{ L}}}. \edtext{Ajo}{\lemma{minuta.}\Afootnote{ \textit{ (1) }\ Hoc posito dico in pyxide\protect\index{Sachverzeichnis}{pyxis|textit} notari posse singulas  \textbar\ miliaris itineribus \textit{ erg.}\ \textbar\ mutationes horarias. Idque vel \textit{(a)}\ acu \textit{(b)}\ stylo brevi, sed umbram longissimam projiciente, vel stylo longissimo, sed a multis acubus conjunctis moto. Hoc jam exacte determinato, si per \textit{(aa)}\ unam tantum horam \textit{(bb)}\ unius tantum miliaris iter tantum horam nauta certus sit se nimia linea recta navigasse, aut sin minus, sciat quantum ab ea deflexerit circiter, exploratum erit, \textit{(aaa)}\ in quo circulo \textit{(bbb)}\  \textbar\ an manserit \textit{streicht Hrsg.}\ \textbar\ in eodem parallelo \protect\index{Sachverzeichnis}{circulus parallelus|textit} mutato semper meridiano\protect\index{Sachverzeichnis}{meridianus|textit}, \textit{ (2) }\ Ajo \textit{ L}}} si \edtext{constet nautae per aliquod tempus saltem itineris miliaris, navem}{\lemma{si}\Afootnote{ \textit{ (1) }\ acus in \textit{ (2) }\ navis\protect\index{Sachverzeichnis}{navis|textit} \textit{ (3) }\ constet [...] navem \textit{ L}}} aut recta linea cucurrisse, aut quantus exacte flexus fuerit, quod scire facile potest ope \edtext{tum}{\lemma{}\Afootnote{tum \textit{ erg.} \textit{ L}}} magnetis\protect\index{Sachverzeichnis}{magnes} rotae\protect\index{Sachverzeichnis}{rota} alterius, ajo inquam hoc posito ei perfecte constare in quo sit meridiano\protect\index{Sachverzeichnis}{meridianus}. Quod ita demonstro: si \edtext{nulla est mutatio inclinationis, tota mutatio fuit meridianorum, transit ergo navis in parallelo}{\lemma{si}\Afootnote{ \textit{ (1) }\ navis\protect\index{Sachverzeichnis}{navis|textit} in eodem manet parallelo\protect\index{Sachverzeichnis}{circulus parallelus|textit}, \textit{ (2) }\ nulla [...] parallelo \textit{ L}}} dato, de meridiano\protect\index{Sachverzeichnis}{meridianus} in meridianum\protect\index{Sachverzeichnis}{meridianus}, et cognita celeritate\protect\index{Sachverzeichnis}{celeritas} cursus cognita est mutatio meridianorum\protect\index{Sachverzeichnis}{meridianus}. Si navis\protect\index{Sachverzeichnis}{navis} movetur de parallelo\protect\index{Sachverzeichnis}{circulus parallelus} in parallelum\protect\index{Sachverzeichnis}{circulus parallelus},  inclinatio\protect\index{Sachverzeichnis}{inclinatio} acus\protect\index{Sachverzeichnis}{acus!magnetica} crescit summo modo. Si navis\protect\index{Sachverzeichnis}{navis} transit simul mutat meridianum\protect\index{Sachverzeichnis}{meridianus} et parallelum\protect\index{Sachverzeichnis}{circulus parallelus} cum \edtext{tanto major sit mutatio meridianorum, quanto minor parallelorum, sequitur constare continuo ex mutatione parallelorum per inclinationem,}{\lemma{cum}\Afootnote{ \textit{ (1) }\ constet ex incl \textit{ (2) }\ tanto [...] inclinationem, \textit{ L}}} residuam esse mutationem meridianorum\protect\index{Sachverzeichnis}{meridianus}, seu quae sit obliquitas lineae motus sive quis angulus ad meridianos\protect\index{Sachverzeichnis}{meridianus} 