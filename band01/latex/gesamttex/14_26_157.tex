\pend \pstart [p.~157] V. Lentes\protect\index{Sachverzeichnis}{lens} huiusmodi, ellipticae quidem, conuersa ad obiectum superficie elliptica, hyperbolicae vero, conuersa ad idem obiectum superficie plana in telescopio\protect\index{Sachverzeichnis}{telescopium}, vitri obiectiui\protect\index{Sachverzeichnis}{vitrum!objectivum} loco statui possent, idque esset ex iis commodi, quod assumpta modica portione, radij refracti\protect\index{Sachverzeichnis}{radius!refractus} in eodem puncto colligerentur, meliore euentu, quam in lente sphaerica\protect\index{Sachverzeichnis}{lens!sphaerica}; non tamen suppleri posset tubi longitudo;\footnote{\textit{Leibniz unterstreicht:} idque esset [...] tubi longitudo} nempe radij in maiorem portionem incidentes in dictum focum\protect\index{Sachverzeichnis}{focus} non irent, vt dictum est supra; igitur ad augendam obiecti molem ellipsis vel hyperbole longioris diametri adhibenda esset, vt fit in lente sphaerica\protect\index{Sachverzeichnis}{lens!sphaerica}; sed neque hoc iuuaret cum reuera huiusmodi figurae arte humana in vitrum induci non possint; cum enim smiri et puluere formentur,\footnote{\textit{Leibniz unterstreicht}: igitur ad [...] puluere formentur} quis amabo obtineat, vt omnia granula, quae omnem numerum superant, in circulos parallelos eant, in quorum plana axis a vertice lentis\protect\index{Sachverzeichnis}{lens} perpendiculariter cadat; haec igitur inter adinata reponenda sunt: hic etiam demonstrandum esset, praedicatum figuram esse hyperbolem, sed cum alij hoc iam demonstrarint, supersedeo; praesertim cum ex sola constructione, res praesentis instituti satis constet.\pend \pstart VI. Hic etiam obiter significandum videtur, tria inuenta a nonnullis excogitata, ad supplendam tuborum longitudinem pro votis non succedere; \footnote{\textit{Leibniz unterstreicht}: tria [...] non succedere} primum est, vt statuto vitro obiectiuo\protect\index{Sachverzeichnis}{vitrum!objectivum} in eo situ, quem radij ab obiectiuo\protect\index{Sachverzeichnis}{objectivum} incidentes postulant, lens ocularis\protect\index{Sachverzeichnis}{lens!ocularis}, citra vllum tubum,\footnote{\textit{Leibniz unterstreicht}: lens ocularis [...] vllum tubum} oculo\protect\index{Sachverzeichnis}{oculus} in debita distantia admoueatur; [...].