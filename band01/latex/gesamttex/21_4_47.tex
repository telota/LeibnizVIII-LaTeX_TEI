\pstart \selectlanguage{italian} [p.~47] [...] \textit{e che entrando nei diafani pi\`{u} rari, da quella si discostano, facendosi maggiore, \`{o} minor'angolo di refrattione\protect\index{Sachverzeichnis}{angolo!di rifrazione}, quanto \`{e} maggiore \`{o} minore l'angolo dell'incidenza\protect\index{Sachverzeichnis}{angolo!di incidenza}, ma con che regola si vadano diminuendo gli angoli della Refrattione\protect\index{Sachverzeichnis}{angolo!di rifrazione}\protect\index{Sachverzeichnis}{angolo!di rifrazione|see{angulus refractionis}} in vn diafano, ouero accrescendo in relatione de gli angoli dell'incidenza\protect\index{Sachverzeichnis}{angolo!di incidenza}\protect\index{Sachverzeichnis}{angolo!di incidenza|see{angle d'incidence}}\protect\index{Sachverzeichnis}{angolo!di incidenza|see{angulus incidentiae}}, ci\`{o} sin'hora non si\`{e} con modo sicuro, e dimostratiuamente, per quanto io sappia\footnote{\textit{Leibniz unterstreicht:} per quanto io sappia}, potuto prouare; tengono alcuni, che la Parabola cristallina vnisca le parallele in vn punto: Il Keplero\protect\index{Namensregister}{\textso{Kepler} (Keplerus), Johannes 1571\textendash 1630} nell'Astronomia \edtext{Ottica}{\lemma{Keplero}\Bfootnote{\textsc{J. Kepler}, \cite{00066}\textit{Astronomiae pars optica}, Frankfurt 1604, S.~95f. (\textit{KGW} II, S.~92f.).}} stima, che sia vn'Iperbola\footnote{\textit{Leibniz unterstreicht:} sia vn'Iperbola}, come la Mecanica gli dimostra, se ben dice vederla vn poco pi\`{u} acuta della Iperbola nella cima, com'egli accenna al Cap. 4. trattando della misura delle Refrattioni}\protect\index{Sachverzeichnis}{rifrazione},\footnote{\textit{Unten am Rand:} Questo si h\`{a} adesso; e l'inventore della vera regola delle refrattioni \`{e} stato il Snellio\protect\index{Namensregister}{\textso{Snell van Royen} (Snellius), Willebrord 1580\textendash 1625}, seguito del Cartesio\protect\index{Namensregister}{\textso{Descartes} (Cartesius, des Cartes, Cartes.), Ren\'{e} 1596\textendash 1650}, Fermatio\protect\index{Namensregister}{\textso{Fermat} (Fermatius), Pierre de 1601\textendash 1665} e Vgenio.\protect\index{Namensregister}{\textso{Huygens} (Hugenius, Vgenius, Hugens, Huguens), Christiaan 1629\textendash 1695}} [...].\pend