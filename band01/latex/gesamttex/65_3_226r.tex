      
               
                \begin{ledgroupsized}[r]{120mm}
                \footnotesize 
                \pstart                
                \noindent\textbf{\"{U}berlieferung:}   
                \pend
                \end{ledgroupsized}
            
              
                            \begin{ledgroupsized}[r]{114mm}
                            \footnotesize 
                            \pstart \parindent -6mm
                            \makebox[6mm][l]{\textit{L}}Konzept: LH XXXVIII Bl. 226\textendash 227. 3/4 Bog. 4\textsuperscript{o}. 3 S. Bl. 227 v\textsuperscript{o} leer, oberer Rand beschnitten.\pend
                            \end{ledgroupsized}
                %\normalsize
                \vspace*{5mm}
                \begin{ledgroup}
                \footnotesize 
                \pstart
            \noindent\footnotesize{\textbf{Datierungsgr\"{u}nde}: Leibniz bezieht sich bei seinen \"{U}berlegungen zur Synchronisation von Sonnen- und Pendeluhr auf die von Huygens im \cite{00123}\textit{Horologium oscillatorium}, S.~14 mitgeteilte Tabelle. Daher kann die Abfassung des vorliegenden Textes in der Zeit der Lekt\"{u}re dieses Buches angenommen werden. Die Datumsangabe \textit{heut nehmlich den 26 julii} erlaubt dann eine genaue Datierung auf den 26. Juli 1673.\\Kein Eintrag in KK 1 oder Cc 2.}
                \pend
                \end{ledgroup}
            
                \vspace*{8mm}
                \pstart 
                \normalsize
            [226 r\textsuperscript{o}] \selectlanguage{ngerman}Gesezt man habe die penduluhr\protect\index{Sachverzeichnis}{Pendeluhr} den \edtext{1 julii}{\lemma{den}\Afootnote{ \textit{ (1) }\ 26 julii \textit{ (2) }\ 1 julii \textit{ L}}} nach der Sonnen Uhren\protect\index{Sachverzeichnis}{Sonnenuhr} gestellet, und wolle heut nehmlich den 26 julii, bey der pendul-uhr\protect\index{Sachverzeichnis}{Pendeluhr} wißen, wie viel es auff der Sonnen-Uhr\protect\index{Sachverzeichnis}{Sonnenuhr} seyn w\"{u}rde,\edtext{  wenn}{\lemma{w\"{u}rde,}\Afootnote{ \textit{ (1) }\ ob schohn die Son \textit{ (2) }\  wenn \textit{ L}}} \edtext{das tr\"{u}be Wetter}{\lemma{wenn}\Afootnote{ \textit{ (1) }\ die Sonne \textit{ (2) }\ das tr\"{u}be Wetter \textit{ L}}} \edtext{den Sonnenschein nicht verhinderte}{\lemma{Wetter}\Afootnote{ \textit{ (1) }\ solche zu sehen nicht ver \textit{ (2) }\ den Sonnenschein nicht verhinderte \textit{ L}}}; so kan man vermittelst der bey dem Hugenio\protect\index{Namensregister}{\textso{Huygens} (Hugenius, Vgenius, Hugens, Huguens), Christiaan 1629\textendash 1695} tr. de \textit{Horolog. Oscill.} pag. 14 befindtlichen  Tabell\edtext{}{\lemma{Tabell}\Bfootnote{\textsc{Chr. Huygens, }\cite{0123}\textit{Horologium oscillatorium}, Paris 1973, S.~14 (\textit{HO} XVIII, S. 112f., Tabelle auf S. [15]).}} also verfahren.\pend \pstart  Man nimt die Zahl \edtext{so in der Tabell}{\lemma{Zahl}\Afootnote{ \textit{ (1) }\ so \textit{ (2) }\ so in der Tabell \textit{ L}}} bey dem 1 julii  stehet, nehmlich 12 minut. 19 secund. item die Zahl bey dem 26 jul. nehmlich 9 minut. 46 secund. Und ziehet eine Zahl von der anderen, bleiben 2 minut. 33 secund.  Umb so viel gehet die pendul-uhr\protect\index{Sachverzeichnis}{Pendeluhr} geschwinder  als die \edtext{Sonne,}{\lemma{Sonne,}\Afootnote{ \textit{ (1) }\ weilen die erste zahl großer; wenn aber die andere \textit{ (a) } Zahl großer \textit{ (b) } Zahl gr\"{o}ßer gewesen w\"{a}re; so w\"{a}re \textit{ (2) }\ und \textit{ L}}} und muß man solche Differenz von der auff der Pendul-uhr\protect\index{Sachverzeichnis}{Pendeluhr} \edtext{bemerckten}{\lemma{Pendul-uhr}\Afootnote{ \textit{ (1) }\ selbe \textit{ (2) }\ bemerckten \textit{ L}}} Zeit abziehen  wenn man die Zeit haben will, \edtext{welche}{\lemma{will,}\Afootnote{ \textit{ (1) }\ wenn man \textit{ (2) }\   welche \textit{ L}}} die Sonnen Uhr\protect\index{Sachverzeichnis}{Sonnenuhr} bemercken w\"{u}rde; Weilen nehmlichen die erste Zahl \edtext{der tafel}{\lemma{}\Afootnote{der tafel \textit{ erg.} \textit{ L}}} gr\"{o}ßer