\pend \pstart [p.~182] [...] hinc stellae\protect\index{Sachverzeichnis}{stella} per mediam caudam\protect\index{Sachverzeichnis}{cauda cometae} visae sunt saepius,\footnote{\textit{Leibniz unterstreicht}: stellae [...] saepius} tum a me, tum a multis aliis, tum Romae\protect\index{Ortsregister}{Rom (Roma)}, tum in Gallia\protect\index{Ortsregister}{Frankreich (Gallia, Francia)}, tum in Germania\protect\index{Ortsregister}{Deutschland (Germania, Duitsland)}. [...] cum in tot locis eadem ab \edtext{[eadem]}{\lemma{cadem}\Afootnote{\textit{\"{a}ndert Hrsg.}}} stella\protect\index{Sachverzeichnis}{stella} distantia obseruata fuerit; et licet pro duobus tantum locis, hoc argumentum paralaxeos, in quolibet situ, non concludat, vt iam alij ostendere conati\protect\index{Sachverzeichnis}{conatus} sunt; si tamen ex 4. 5. imo et pluribus locis, illa eadem distantia obseruata sit,\footnote{\textit{Am Rand angestrichen}: cum in tot [...] obseruata sit} haud dubie concludit argumentum, sed de altitudine cometae\protect\index{Sachverzeichnis}{cometa} adhuc \textit{infra}. 