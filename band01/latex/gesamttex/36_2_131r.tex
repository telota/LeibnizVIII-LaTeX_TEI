 \edtext{convainquantes, dont la principalle est, que ce que nous venons [131 r\textsuperscript{o}] de raisonner de deux corps solides ou placques \textit{AB} et \textit{CD}  ne se laisse pas appliquer}{\lemma{convainquantes,}\Afootnote{ \textit{ (1) }\ s\c{c}avoir \textit{ (2) }\ que nous nous sommes servis, de deux corps solides\protect\index{Sachverzeichnis}{corps!solide|textit} comme deux placques\protect\index{Sachverzeichnis}{deux placques|textit}, \textit{AB} et \textit{CD} mais que l'applic \textit{ (3) }\ dont [...]  appliquer \textit{ L}}} aux liqueurs\protect\index{Sachverzeichnis}{liqueur} suspendues; dont nous avons mêmes des experiences manifestes. Car deux placques estant jointes, quoyqu'\edtext{il y aye}{\lemma{quoyqu'}\Afootnote{ \textit{ (1) }\ elles ayent \textit{ (2) }\ il y aye \textit{ L}}} des trous dans une ou dans toutes les deux, ou quoyqu'on les perce, ne laissent pas de demeurer jointes; mais \edtext{si}{\lemma{mais}\Afootnote{ \textit{ (1) }\ si \textit{ (2) }\ s'il y a \textit{ (3) }\ si \textit{ L}}} l'un de deux corps\protect\index{Sachverzeichnis}{corps!solide} joints est solide, comme un tuyau de verre, l'autre liquide, comme la liqueur purg\'{e}e\protect\index{Sachverzeichnis}{liqueur!purg\'{e}e} suspend\"{u}e dans le tuyau, et si l'on perce le haut du tuyau tant soit peu, la liqueur\protect\index{Sachverzeichnis}{liqueur} tombe subitement, la raison est, parce que la liqueur\protect\index{Sachverzeichnis}{liqueur} se plie et donne passage \`{a} l'air, ou \`{a} la matiere pressante, partout, ce que les placques roides ne peuuent pas. Sans cela la petite bulle d'air n\'{e}e dans la liqueur purg\'{e}e\protect\index{Sachverzeichnis}{liqueur!purg\'{e}e} suspendue, ne pourroit pas s'insinuer entre \edtext{la superficie interieure du verre}{\lemma{entre}\Afootnote{ \textit{ (1) }\ le \textit{ (2) }\ le verre est \textit{ (3) }\ la superficie interieure du verre \textit{ L}}} et la liqueur\protect\index{Sachverzeichnis}{liqueur}, malgr\'{e} la pression unitive si la facilit\'{e} de la liqueur\protect\index{Sachverzeichnis}{liqueur} \`{a} se plier ne donneroit pas passage \edtext{partout}{\lemma{}\Afootnote{partout \textit{ erg.} \textit{ L}}} apres la moindre ouuerture:\edtext{}{\lemma{}\Afootnote{ouuerture:  \textbar\ Et asseurement \textit{ gestr.}\ \textbar\ toute \textit{ L}}} toute la masse de l'Atmosphere\protect\index{Sachverzeichnis}{atmosph\`{e}re} n'estant pas capable de s'insinuer entre deux placques roides. Et il y a un dilemme manifeste, et ce me semble, inevitable: Puisqu'il faut, ou que pour faire tomber \edtext{le corps attach\'{e}}{\lemma{tomber}\Afootnote{ \textit{ (1) }\ la liqueur\protect\index{Sachverzeichnis}{liqueur|textit} \textit{ (2) }\ le corps attach\'{e} \textit{ L}}}, la pression interieure ou separatifve soit au moins \'{e}gale, \edtext{en nombre des coups}{\lemma{\'{e}gale,}\Afootnote{ \textit{ (1) }\ (en faisant le denombrement des coups) \textit{ (2) }\ en nombre des coups \textit{ L}}} \`{a} l'unitifve; ou que plustost, au moins \`{a} l'\'{e}gard de la liqueur\protect\index{Sachverzeichnis}{liqueur} suspend\"{u}e, il suffit que la matiere pressante trouue entr\'{e}e \edtext{partout, et presse de deux costez}{\lemma{entr\'{e}e}\Afootnote{ \textit{ (1) }\ de tous coste \textit{ (2) }\ partout, [...] costez \textit{ L}}}. Si l'on soûtient le premier generalement, il ne suffira pas, pour faire tomber la liqueur\protect\index{Sachverzeichnis}{liqueur}, qu'on perce le haut du tuyau, ou qu'une petite bulle \edtext{d'air}{\lemma{}\Afootnote{d'air \textit{ erg.} \textit{ L}}} s'insinue, dans laquelle \edtext{asseurement}{\lemma{}\Afootnote{asseurement \textit{ erg.} \textit{ L}}} les coups ou vagues \edtext{de la matiere\protect\index{Sachverzeichnis}{mati\`{e}re!subtile} plus subtile que l'air}{\lemma{}\Afootnote{de la matiere\protect\index{Sachverzeichnis}{mati\`{e}re!subtile}   \textbar\ plus \textit{ erg.}\ \textbar\  subtile que l'air \textit{ erg.} \textit{ L}}} ne peuuent pas estre \'{e}galles \edtext{en nombre}{\lemma{}\Afootnote{en nombre \textit{ erg.} \textit{ L}}} \`{a} tous les autres \edtext{de la même matiere dans tout le Recipient}{\lemma{}\Afootnote{de [...] Recipient \textit{ erg.} \textit{ L}}} qui poussent la liqueur\protect\index{Sachverzeichnis}{liqueur} vers le verre; Que si l'on avoue la derniere partie du dilemme, \edtext{sans exiger la compensation des coups}{\lemma{dilemme,}\Afootnote{\textit{ (1) }\ et qu'on \textit{ (2) }\ sans exiger  \textit{(a)}\ l'\'{e}galit\'{e} \textit{(b)}\ la compensation des coups \textit{ erg.} \textit{ L}}} il ne faut pas \edtext{conceder qu'il y a des}{\lemma{pas}\Afootnote{ \textit{ (1) }\ avouer des \textit{ (2) }\ conceder qu'il y a des \textit{ L}}} pores dans le verre, par lesquels la matiere pressante puisse passer, car alors \edtext{on seroit oblig\'{e}}{\lemma{alors}\Afootnote{ \textit{ (1) }\ on se \textit{ (2) }\ pour expliquer \textit{ (3) }\ on seroit oblig\'{e} \textit{ L}}} pour sauuer la suspension de la liqueur\protect\index{Sachverzeichnis}{liqueur}, de recourir \`{a} la necessit\'{e} de l'\'{e}galit\'{e} des coups, \`{a} laquelle on a déja renonc\'{e}. Enfin si l'on se veut servir de l'exemple de la pression de l'atmosphere\protect\index{Sachverzeichnis}{atmosph\`{e}re}, pour expliquer comment une petite bulle d'air fait tomber la liqueur\protect\index{Sachverzeichnis}{liqueur}, la pression \edtext{commen\c{c}ant par l\`{a} estre de deux costez}{\lemma{pression}\Afootnote{ \textit{ (1) }\ estant de deux costez \textit{ (2) }\ devenant par la \textit{ (3) }\ commen\c{c}ant [...] costez \textit{ L}}}: il ne faut pas desavouer le même exemple, en concedant des pores pour le passage de la matiere pressante sans qu'ils fassent tomber la liqueur\protect\index{Sachverzeichnis}{liqueur} sus\-pend\"{u}e. J'obmets de montrer icy, que ce qu'on a dit selon la même hypothese, pourquoy \edtext{la bulle ne fait tomber la liqueur}{\lemma{pourquoy}\Afootnote{ \textit{ (1) }\ la liqueur\protect\index{Sachverzeichnis}{liqueur|textit} fait to \textit{ (2) }\ la bulle   \textbar\ ne \textit{ erg.}\ \textbar\  fait tomber la liqueur \textit{ L}}} qu'apres avoir pass\'{e} les bornes de la suspension de la même liqueur\protect\index{Sachverzeichnis}{liqueur} quand elle sera tomb\'{e}e ne satisfait pas l'esprit.\pend \pstart 