
\pstart  
Mais il est question \`{a} present, si ce mouuement des Liqueurs peut estre la cause de l'union des corps purgez\protect\index{Sachverzeichnis}{corps!purg\'{e}} d'air. Et je crois qu'il est manifeste, quelque raison \edtext{exterieure \`{a} l'egard des corps joints}{\lemma{exterieure}\Afootnote{ \`{a} l'egard des corps joints \textit{ erg.} \textit{ L}}} qu'on veuille apporter de ces phenomenes, qu'il faut presupposer \edtext{dans ces corps}{\lemma{}\Afootnote{dans ces corps \textit{ erg.} \textit{ L}}} une solidit\'{e} primitive, independente de cette raison. Car \edtext{posons}{\lemma{Car}\Afootnote{ \textit{ (1) }\ supposons \textit{ (2) }\ posons \textit{ L}}} que les placques, ou tout autre corps, soyent pressez l'un vers l'autre\edtext{}{\lemma{}\Afootnote{l'autre  \textbar\ par un mouuement \textit{ gestr.}\ \textbar\ , il \textit{ L}}}, il faut asseur\'{e}ment presupposer les placques m\^{e}mes, ou quelque corps que ce puisse estre, \edtext{(quoyque}{\lemma{(quoyque}\Afootnote{  \textbar\ il soit \textit{ gestr.}\ \textbar\ liquide, \textit{ L}}} liquide, comme l'eau ou le mercure\protect\index{Sachverzeichnis}{mercure}) d\'{e}ja form\'{e} et muni d'une certaine \edtext{grossieret\'{e} et}{\lemma{}\Afootnote{grossieret\'{e} et \textit{ erg.} \textit{ L}}} connexion de ses parties, pour ne tomber pas en poudre. Puisque toutes ces Raisons\edtext{}{\lemma{Raisons}\Afootnote{ \textbar~et pressions \textit{ gestr.}\ \textbar\ sont \textit{ L}}} sont fond\'{e}es sur ce que les corps joints ne se pouuant d\'{e}tacher aisement, \edtext{qu'avec toutes leurs parties ensemble}{\lemma{aisement,}\Afootnote{ \textit{ (1) }\ qu'ensemble \textit{ (2) }\ qu'avec toutes leurs parties ensemble \textit{ L}}} en sont empech\'{e}s par des \edtext{matieres liquides}{\lemma{des}\Afootnote{ \textit{ (1) }\ corps solides\protect\index{Sachverzeichnis}{corps!solide|textit} \textit{ (2) }\  matieres liquides \textit{ L}}} qui ne se peuuent aussi aisement \edtext{qu'il est besoin}{\lemma{}\Afootnote{qu'il est besoin \textit{ erg.} \textit{ L}}} insinuer \`{a} cause de la figure et connexion des corps \edtext{joints}{\lemma{corps}\Afootnote{ \textit{ (1) }\ attachez \textit{ (2) }\ joints \textit{ L}}} difficile \`{a} estre chang\'{e}e et romp\"{u}e. 
\pend 
\pstart C'est pourquoy Galilaei\protect\index{Namensregister}{\textso{Galilei} (Galilaeus, Galileus), Galileo 1564\textendash 1642} ayant expliqu\'{e} \edtext{la fermet\'{e}}{\lemma{expliqu\'{e}}\Afootnote{ \textit{ (1) }\ toutes \textit{ (2) }\ la fermet\'{e} \textit{ L}}} des corps sensibles\protect\index{Sachverzeichnis}{corps!sensible} par le phenomene des placques \edtext{et}{\lemma{}\Afootnote{et \textit{ erg.} \textit{ L}}} supposant les corps\protect\index{Sachverzeichnis}{corps!solide} solides composez de quantit\'{e} de telles petites placques\edtext{}{\lemma{placques}\Bfootnote{\textsc{G. Galilei, }\cite{00050}\textit{Discorsi}, Leiden 1638, S.~11f. (\textit{GO} VIII, S.~59).}};  seroit oblig\'{e} de passer plus outre, et de rendre raison \edtext{de la fermet\'{e} de}{\lemma{de}\Afootnote{ \textit{ (1) }\ l'union \textit{ (2) }\ la  \textit{(a)}\ solidit\'{e} d \textit{(b)}\ fermet\'{e} de \textit{ L}}} ces petites placques m\^{e}mes, pour parvenir \`{a} la derniere resolution de l'essence de la solidit\'{e} \edtext{primitive}{\lemma{}\Afootnote{primitive \textit{ erg.} \textit{ L}}}. Laquelle ne peut estre autre que celle du mouuement\protect\index{Sachverzeichnis}{mouvement!uniforme} ou effort uniforme.
\pend 
\pstart \textso{\`{A} present laissons \`{a} part la solidit\'{e} primitive}, \edtext{(quoyqu'il a est\'{e} necessaire de remarquer combien elle est differente de la sensible)}{\lemma{(quoyqu'il}\Afootnote{ [...] sensible) \textit{ erg.} \textit{ L}}} \edtext{et reprenant nos phenomenes,  cherchons d'ou puisse}{\lemma{et}\Afootnote{ \textit{ (1) }\ cherchons pour \textit{ (2) }\ reprenant [...] puisse \textit{ L}}} venir l'union de deux placques\protect\index{Sachverzeichnis}{deux placques}, ou autres corps purg\'{e}s\protect\index{Sachverzeichnis}{corps!purg\'{e}} d'air, \edtext{et puisqu'il n'y a}{\lemma{et}\Afootnote{ \textit{ (1) }\ apres avoir \textit{ (2) }\  puisqu'il n'y a \textit{ L}}} point de gl\"{u}e, au moins entre les placques; voyons si ce qu'on a dit ingenieusement pour cet effect \edtext{en se servant}{\lemma{}\Afootnote{en se servant \textit{ erg.} \textit{ L}}} du mouuement des fluides en tous sens, dans une lettre imprim\'{e}e\edtext{}{\lemma{imprim\'{e}e}\Bfootnote{%\textsc{Chr. Huygens, }\cite{00062}\textit{Extrait d'une lettre}, Journal des S\c{c}avans, 25. Juli 1672, S.~133\textendash140. (HO VII, S.~201\textendash206)
\textsc{P.-D. Huet,} \cite{00163}\textit{Lettre à M. Chouet}, Paris 1673. Vgl. auch \cite{00267}N. 48.}} il n'y a gueres longtemps, peut contenter l'esprit. 
\pend 
