[52 r\textsuperscript{o}] fixum \edtext{firmatumque}{\lemma{firmatumque}\Afootnote{ \textit{ (1) }\ sustineat \textit{ (2) }\ sustineret \textit{ L}}} sustineret, forte hac sola ratione effici potest ut gyretur. Forte tandem et libratio in aqua sufficit, si omnia sint libera et horizonti parallela, ita ut mota quomodocunque navi\protect\index{Sachverzeichnis}{navis} aequaliter perpetuo moveatur aqua, id est non moveatur; quo\hspace{1pt} pertinere\hspace{1pt} potest\hspace{1pt} et\hspace{1pt} industria\hspace{1pt} illa\hspace{1pt} superior\hspace{1pt} de\hspace{1pt} \edtext{rigida pendulatione}{\lemma{de}\Afootnote{ \textit{ (1) }\ re pendu \textit{ (2) }\ rigida pendulatione \textit{ L}}}.\hspace{1pt} Quid ve-\pend\pstart\noindent ro si vas ita sit plenum aqua ut non possit esse plenius? \edtext{Nec tamen et possit quicquam effluere, tunc}{\lemma{plenius?}\Afootnote{ \textit{ (1) }\ Nulla tunc \textit{ (2) }\ Nec tamen et possit \textit{(a)}\ emergere: Tunc \textit{(b)}\  quicquam effluere, tunc \textit{ L}}} non fluctuabit. Ergo ita in eo libretur subere magnes\protect\index{Sachverzeichnis}{magnes}, ut paulum intra aquam descendat, superiori ex polo\protect\index{Sachverzeichnis}{polus} exeat obelus tenuissimus sursum, papyrum circumvolans, is obelus transeat per foramen vitreum ita exacte, ut nullus aer transpirare aut exhalare aqua possit, \edtext{aut}{\lemma{possit,}\Afootnote{ \textit{ (1) }\ et \textit{ (2) }\ aut \textit{ L}}} ne metuendi causa sit, ne forte aqua exhalet, aut non sit vas plenum, Vas hoc aqua plenum in alio vase aqua pleno submergatur, per quam aeneus obelus transeat. Aeneus inquam obelus, quia ferreus rubigine exeditur, at magneti in aqua librato hoc nihil nocebit. Si vero vas librationis aqua sit exacte plenum, quomodocumque moveatur, nisi exacte circa proprium axem aqua in eo non fluctuabit, neque enim potest pars ejus aliquo cedere. Imo etsi moveatur circa proprium axem aqua tamen cum ea similiter quasi conglaciata esset, movebitur, sed an tam exacte effici plenitudo possit, experimento tentandum est. \edtext{Quis}{\lemma{est.}\Afootnote{ \textit{ (1) }\ Ut \textit{ (2) }\ Quis \textit{ L}}} et tot modis librandi melior praxis dabit, interim aliquis, volente deo deesse non poterit, ne tanti inventi fructus nobis pereat!\pend \pstart Ferrum candens juxta situm meridiani\protect\index{Sachverzeichnis}{meridianus} cusum se versus polos\protect\index{Sachverzeichnis}{polus} supponit\edtext{ inquit}{\lemma{supponit}\Afootnote{ \textit{ (1) }\ , si hoc verum esset, sublata esset de \textit{ (2) }\  inquit \textit{ L}}} Kircherus, lib. 1. part. 2. paradox. analysi 8.\edtext{}{\lemma{analysi 8.}\Bfootnote{\textsc{A. Kircher, } \cite{00067}a.a.O., S.~104f. }} Quod si verum esset exacte labore delibranda terrella sublevaremur. Idem paulo ante analysi 2.\edtext{}{\lemma{analysi 2.}\Bfootnote{\textsc{A. Kircher, } \cite{00067}a.a.O., S.~103.}} Constat ferramenta illa longiora quibus fenestrae saepiuntur, in longum aut etiam juxta lineam meridianam extensa magneticam qualitatem etiam sine attactu magnetis\protect\index{Sachverzeichnis}{magnes} \edtext{contrahere}{\lemma{magnetis}\Afootnote{ \textit{ (1) }\ recipere \textit{ (2) }\ contrahere \textit{ L}}},  ita ut exemta talia ferramenta, et magnetice librata perfecte se ad polos\protect\index{Sachverzeichnis}{polus} disponant, et quidem pars illa clathrorum ferreorum quae terrae obvertitur, semper et infallibiliter septentrionem petet, opposita austrum. Quod non solum de ferramentis dictis quamcunque plagam coeli respexerint, verificatur, sed et de omnibus instrumentis ferreis quibus ignem tractamus, verum esse, non mihi tantum, sed et Cabaeo\protect\index{Namensregister}{\textso{Cabeo,} Niccol\`{o} SJ 1586\textendash 1650} aliisque innotuit. Haec enim librata semper inferiore parte sua in Boream, manubrio vero sive superiore parte 