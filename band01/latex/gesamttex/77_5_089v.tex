[89 v\textsuperscript{o}]  minimi generis facile de omnibus aliis intermediis ostendi  potest. Ita ut satis demonstratum putem, partem aliquam  circuli radios axi parallelos ex aere in eum incidentes  refractione ita posse congregari, ut focus\protect\index{Sachverzeichnis}{focus}, ac lineola illa in  in axe in quam incidunt, pro puncto mechanico\protect\index{Sachverzeichnis}{punctum!mechanicum} habenda sint,  hancque circuli partem satis magnam esse ut Conspicilla\protect\index{Sachverzeichnis}{conspicillum}  tam senibus quam juvenibus inservientia, telescopia\protect\index{Sachverzeichnis}{telescopium} ac microscopia\protect\index{Sachverzeichnis}{microscopium} ex ea formari possint.\pend \pstart  Hactenus itaque ostensum est praedictos parallelos radios ex  aere in vitri superficiem incidentes ac per eam transeuntes  ita refringi debere, ut dein omnes ad unum punctum mechanicum\protect\index{Sachverzeichnis}{punctum!mechanicum}  tendant, aut etiam si vitrum sufficientem haberet crassitudinem,  in eo congregentur: Sed cum tantae crassitudinis vitrum, aut  vix haberi, aut nobis usui esse non possit, ex praedicto puncto  tanquam centro circulus erit ducendus, qui priorem Circulum  secet, ut videre est in \textso{secunda figura}\footnote{fig. 2.}, ubi ex \textit{K} tanquam  ex foco\protect\index{Sachverzeichnis}{focus}, ductus est Circulus \textit{KHQR}. Diameter autem  haec \textit{KH} pro libitu aut major, aut minor sumi potest, prout  vitrum aut crassius, aut subtilius desideratur, hoc solummodo  adhibita cautione major sumatur quam est \textit{DK}.\pend \pstart 