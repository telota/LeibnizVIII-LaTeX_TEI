[72 r\textsuperscript{o}] mensium similem esse rationem quia luna\protect\index{Sachverzeichnis}{luna} ad  easdem semper fixas\protect\index{Sachverzeichnis}{stella!fixa} eodem modo redit.\pend \pstart Ad hoc institutum tabulae peculiares usui nautico accommodatae fieri possunt quae monstrent, qua mensis lunaris, non tantum die horaque lunari, sed et minuto minimum primo, quis sit locus lunae\protect\index{Sachverzeichnis}{luna} in sphaera fixarum\protect\index{Sachverzeichnis}{sphaera!fixarum}. Ad eandem rem machina fieri, vel potius ad sphaeram nostram artificialem, eo quo supra dixi modo dispositam, accommodari potest, in qua Luna\protect\index{Sachverzeichnis}{luna} artificialis Zodiacum\protect\index{Sachverzeichnis}{zodiacus} sphaerae suo modo  percurrens, sphaera ipsa interim suo motu proprio revoluta, monstrabit, et (modo sphaera ipsa satis ampla sit), ad minuta prima usque \edtext{definiet}{\lemma{}\Afootnote{definiet \textit{ erg.} \textit{ L}}}, si volumus, quo minuto primo, quo in loco coeli versetur Luna\protect\index{Sachverzeichnis}{luna}. Nec opus est machinam  istam in Navi\protect\index{Sachverzeichnis}{navis} esse, sufficeret eum, qui primus tabulam ejusmodi locorum Lunae\protect\index{Sachverzeichnis}{luna}, sine calculi molestia de novo condere, aut tabulas ab aliis conditas examinare vellet, talem machinam ante oculos habere, et quidquid in ea de minuto in minutum observat in tabulam referre suspenso cum volet et postea rursus cum volet liberato motu machinae, prout ei otium ad eam rem \edtext{observandam}{\lemma{}\Afootnote{observandam \textit{ erg.} \textit{ L}}} erit, aut non erit. Idque exactissime poterit haberi, si machina in loco stabili posita, pendulo\protect\index{Sachverzeichnis}{pendulum} animetur. Imo nec homine opus erit machinam continuo observante: fieri enim poterit, ut ipsamet Luna\protect\index{Sachverzeichnis}{luna} artificialis punctis quibusdam coloratis quolibet minuto \edtext{impressis}{\lemma{}\Afootnote{impressis \textit{ erg.} \textit{ L}}}, locum suum in sphaera artificiali monstret.\pend \pstart Qua \textso{tabulas} astronomicas condendi aut  examinandi ratione sane nova et satis ad usum Geographicum accurata, nihil facile  elegantius jucundiusque fingi potest. \pend \pstart Ex his intelligi potest non in \textso{Luna}\protect\index{Sachverzeichnis}{luna} tantum sed et in omnibus Planetis aliis idem esse