\pend \pstart [p.~231] Alterum est, quod tamen nostri obseruatores pro certo non venditant, maculas scilicet in Saturno\protect\index{Sachverzeichnis}{Saturnus} videri,\footnote{\textit{Leibniz unterstreicht:} pro certo [...] Saturno videri} sed donec longius telescopium\protect\index{Sachverzeichnis}{telescopium} adhibeatur, et aestiui halitus, qui non parum obsunt, abigantur, res incerta manebit.\pend \pstart Iam ad alia descendo. Diuinius\protect\index{Namensregister}{\textso{Divini} (Divinius), Eustachio 1610\textendash 1685}\footnote{\textit{Leibniz unterstreicht:} Diuinius} noster pro singulari, qua pollet industria, excogitauit facilem modum, quo citra solitam telescopij\protect\index{Sachverzeichnis}{telescopium} probationem, tornatae lentis\protect\index{Sachverzeichnis}{lens} vitium facile deprehendi possit: rationem item ita disponendi quatuor lentes\protect\index{Sachverzeichnis}{lens}, vt, siue longior, siue breuior tubus adhibeatur, aeque distinctum obiectum appareat;\footnote{\textit{Leibniz unterstreicht:} rationem [...] appareat} vtrumque suo tempore publici iuris faciet, et si ita vir antiqua consuetudine mihi deuinctus iusserit, ego vtrumque pariter demonstrabo. Idem quoque Diuinius\protect\index{Namensregister}{\textso{Divini} (Divinius), Eustachio 1610\textendash 1685}, fracto, casu, pyrite durissimo, concham probe tornatam et striatam in eo reperit. Ludit etiam in saxis natura. Quaedam etiam experimenta physica noua incogitanti mihi occurrerunt, de quibus ex professo alias.\footnote{\textit{Am Rand angestrichen:} Iam ad [...] professo alias.}