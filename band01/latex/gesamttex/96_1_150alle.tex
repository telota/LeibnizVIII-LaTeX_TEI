Bei den folgenden St\"{u}cken handelt es sich um eine Systematisierung der Leibniz'schen Untersuchungen zu Problemen der Vakuumphysik. In ihrem Zentrum stehen die Experimente mit Adh\"{a}sionsplatten sowie Huygens' Beobachtung von Anomalien beim Experimentieren mit einer Torricelli'schen R\"{o}hre im Vakuumrezipienten. Leibniz sucht nach einer gemeinsamen Ursache f\"{u}r diese Ph\"{a}nomene und zwar auf dem Hintergrund aller ihm bekannten Vakuumph\"{a}nomene. Die Ph\"{a}nomene selbst werden hinsichtlich ihrer Konsequenzen untersucht, mit m\"{o}glichen Entgegnungen konfrontiert und durch neue Experimente detaillierter erschlossen. Den theoretischen Ausgangspunkt der \"{U}berlegungen bilden die fr\"{u}hen Ansichten der \textit{Hypothesis physica nova} (\cite{00256}\textit{LSB} VI, 2 N. 40) sowie der \cite{00257}\textit{Propositiones quaedam physicae} (\textit{LSB} VI, 3 N. 2). Am Ende seiner Überlegungen postuliert Leibniz \glqq une pression differente de celle de l'atmosphere``. Letztere verweist auf N. 51, wo Leibniz sich diesen Druck durch ein \glqq mouuement en tous sens`` erzeugt denkt. Der Terminus \glqq mouuement en tous sens`` ist vermutlich bei Huet\protect\index{Namensregister}{\textso{Huet,} Pierre Daniel 1630\textendash 1721} entlehnt, der ihn in einem Brief an Chouet\protect\index{Namensregister}{\textso{Chouet,} Jean-Robert 1642\textendash 1731} verwendet. Auf Leibniz' Exemplar des Briefes vom M\"{a}rz 1673 beruht \cite{00267}N. 48. Zusammengeh\"{o}rigkeit und Abfolge der St\"{u}cke lassen sich aus der mehrfachen Verwendung von Zeichnungen auf Bl. 134 r\textsuperscript{o} und Bl. 135 r\textsuperscript{o} erschlie"sen, die zu diesem Zweck neu durchnummeriert wurden. Die vorliegenden St\"{u}cke d\"{u}rften zeitnah zum Datum des Huet-Briefes entstanden sein. Wir gehen vom Fr\"{u}hjahr 1673 als dem wahrscheinlichsten Entstehungszeitraum aus. Die Datierung wird durch die Wasserzeichen der von Leibniz verwendeten Papiere gest\"{u}tzt.