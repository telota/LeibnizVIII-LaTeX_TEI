      
               
                \begin{ledgroupsized}[r]{120mm}
                \footnotesize 
                \pstart                
                \noindent\textbf{\"{U}berlieferung:}   
                \pend
                \end{ledgroupsized}
            
              
                            \begin{ledgroupsized}[r]{114mm}
                            \footnotesize 
                            \pstart \parindent -6mm
                            \makebox[6mm][l]{\textit{L}}Konzept: LH XXXVII 3 Bl. 136\textendash143. 4 Bog. 2\textsuperscript{o}. 16 S. zweispaltig. Etwa in der Mitte der rechten Spalte von Bl. 142 v\textsuperscript{o} eine Zeichnung. Zwei umfassendere Texteinsch\"{u}be zu Bl. 142 r\textsuperscript{o} in der rechten Spalte von Bl. 143 v\textsuperscript{o}. Bl. 143 v\textsuperscript{o} enth\"{a}lt den Hinweis auf eine Zeichnung, die sich auf Bl. 134 r\textsuperscript{o} befindet. Diese wird hier zusammen mit einer weiteren Zeichnung desselben Blattes reproduziert.\\Cc 2, Nr. 491 C \pend
                            \end{ledgroupsized}
                %\normalsize
                \vspace*{5mm}
                \begin{ledgroup}
                \footnotesize 
                \pstart
            \noindent\footnotesize{\textbf{Datierungsgr\"{u}nde}: Wie aus den Platzhaltern fig. 2, fig. 3 und fig. 4 f\"{u}r die Zeichnungen aus N.~49\raisebox{-0.5ex}{\notsotiny 2} hervorgeht, ist auch dieses St\"{u}ck dem Komplex der Systematisierung der Vakuumph\"{a}nomene im Fr\"{u}jahr 1673 zuzuordnen. Inhaltlich kommen die \"{U}berlegungen mit der Formulierung einer neuen Hypothese zu einem gewissen Abschluss. Die Datierung wird durch das Wasserzeichen gest\"{u}tzt.}\selectlanguage{french}
                \pend
                \end{ledgroup}
            
                \vspace*{8mm}
                \pstart 
                \normalsize
           \centering [136 r\textsuperscript{o}] \edlabel{136rstart} \textso{Recherche de la Raison de ces phenomenes,\\  avec des Experiences projett\'{e}es pour\\s'en \'{e}claircir d'avantage; et une\\Hypothese Nouuelle.}\edlabel{Nouuellestart}
            \pend
            \rule[-0.5cm]{0cm}{0.5cm}
            \pstart
           \edtext{}{\lemma{\textso{Recherche} [...] \textso{Nouuelle.}}\xxref{136rstart}{Nouuellestart}\Afootnote{\textit{
            doppelt \hspace{5mm}unterstrichen}}}
              \edtext{\edlabel{Nouuelleend}}{\lemma{}\xxref{Nouuellestart}{Nouuelleend}\Afootnote{\textso{Nouuelle.} \textbar\ (1) \textit{ gestr.}\ \textbar\ Touchant \textit{ L}}} Touchant l'union de \textso{deux placques }\protect\index{Sachverzeichnis}{deux placques}bien polies il est important \edlabel{chercherstart}\textso{de chercher }\edtext{}{\lemma{\textso{de chercher }[...]\textso{ vuide.}}\xxref{chercherstart}{chercherend}\Afootnote{\textit{Markierung am Rand}}}\edtext{\textso{les moindres forces}}{\lemma{\textso{chercher}}\Afootnote{ \textit{ (1) }\ \textso{la moindre force capable} \textit{ (2) }\ \textso{les moindres forces} \textit{ L}}}\textso{ dont elle peut estre surmont\'{e}e; dans le vuide}\protect\index{Sachverzeichnis}{vide}\textso{ aussi bien que hors du vuide.}\edlabel{chercherend}\protect\index{Sachverzeichnis}{vide} Car quoyque Mons. Guericke\protect\index{Namensregister}{\textso{Guericke} (Gerickius, Gerick.), Otto v. 1602\textendash 1686} pretend de prouuer par un Calcul fond\'{e} sur des experiences qu'il a faites, que la force qui separe les placques dans l'air libre \edtext{est un peu plus qu'}{\lemma{libre}\Afootnote{ \textit{ (1) }\ est \textit{ (2) }\ est un peu plus qu' \textit{ L}}}\'{e}gale \`{a} la pesanteur\protect\index{Sachverzeichnis}{pesanteur!de l'air} de la colomne d'air:\edtext{}{\lemma{d'air:}\Bfootnote{\textsc{O. v. Guericke, }\cite{00055}\textit{Experimenta nova}, Amsterdam 1672, S.~101\textendash103. }} les dernieres experiences pourtant semblent inferer \edtext{que cette connexion n'est pas necessaire.}{\lemma{inferer}\Afootnote{ \textit{ (1) }\ le contraire. \textit{ (2) }\ que [...] necessaire. \textit{ L}}} \edtext{Puisque}{\lemma{necessaire.}\Afootnote{ \textit{ (1) }\ Car \textit{ (2) }\ Puisque \textit{ L}}} le Mercure purg\'{e}\protect\index{Sachverzeichnis}{mercure!purg\'{e}} d'air, demeurant suspendu\edtext{}{\lemma{}\Afootnote{suspendu  \textbar\ dans un Tuyau \textit{ gestr.}\ \textbar\ d'une \textit{ L}}} d'une hauteur plus que double de celle que la pression de l'atmosphere\protect\index{Sachverzeichnis}{atmosph\`{e}re} peut so\^{u}tenir; il est \`{a} croire \edtext{par plus forte raison,}{\lemma{}\Afootnote{par plus forte raison, \textit{ erg.} \textit{ L}}} que la force necessaire \`{a} separer les deux placques\protect\index{Sachverzeichnis}{deux placques} doivue aussi surpasser de beaucoup celle de l'atmosphere\protect\index{Sachverzeichnis}{atmosph\`{e}re}: \edtext{Par ce}{\lemma{l'atmosphere:}\Afootnote{ \textit{ (1) }\ La connexion \textit{ (2) }\ parce que deux placques\protect\index{Sachverzeichnis}{deux placques|textit} \textit{ (3) }\ Par ce \textit{ L}}} qu'au lieu qu'on purge le Mercure\protect\index{Sachverzeichnis}{mercure!purg\'{e}}, d'air, les placques en sont purg\'{e}es naturellement, estant corps solides\protect\index{Sachverzeichnis}{corps!solide}, et \edtext{aussi}{\lemma{aussi}\Afootnote{\textit{ erg.} \textit{ L}}} par consequent moins capables de se plier, et de donner passage \`{a} l'air. \`{A} moins qu'on ne dise, que le Mercure\protect\index{Sachverzeichnis}{mercure} se trouue attach\'{e} en plus d'endroits, et non seulement en haut, mais aussi aux costez, \`{a} la superficie interieure du \edtext{tuyau. Mais le m\^{e}me se pourroit dire, aussi du Mercure non purg\'{e}.}{\lemma{tuyau.}\Afootnote{ \textit{ (1) }\ , au lieu \textit{ (2) }\ . Mais [...] purg\'{e}. \textit{ L}}} Quoyqu' il en soit, s'il est veritable que les placques ne se separent pas \edtext{aisement}{\lemma{}\Afootnote{aisement \textit{ erg.} \textit{ L}}} dans le vuide\protect\index{Sachverzeichnis}{vide} m\^{e}me; \edtext{la pression}{\lemma{la}\Afootnote{ \textit{ (1) }\ force \textit{ (2) }\ pression \textit{ L}}} de l'atmosphere\protect\index{Sachverzeichnis}{atmosph\`{e}re} ne reglera pas\edtext{, autant que nous en pouuons juger,}{\lemma{, autant}\Afootnote{que nous   \textbar\ en \textit{ erg.}\ \textbar\  pouuons juger, \textit{ erg.} \textit{ L}}} la force qui les joint, ny celle qui les separe. \edlabel{dureste1}%
\edtext{}{\lemma{\textso{Du reste}}\xxref{dureste1}{dureste2}\Afootnote{[...] \textso{\'{e}clat.} \textit{Markierung am Rand}}}%
 \textso{Du reste il est bon de se servir des placques d'une largeur considerable, }\edtext{\textso{d'attacher}}{\lemma{\textso{considerable,}}\Afootnote{ \textit{ (1) }\ \textso{de toutes} \textit{ (2) }\ \textso{d'attacher} \textit{ L}}}\textso{ le }\textso{poids}\protect\index{Sachverzeichnis}{poids}\textso{ tantost au milieu, tantost aux costez de la placque inferieure, d'employer matieres differentes, }\edtext{\textso{de mo\^{u}iller }\textso{quelques fois}\textso{ leurs superficies interieures de }\textso{liqueurs}\protect\index{Sachverzeichnis}{liqueur!purg\'{e}e}\textso{ }\textso{purg\'{e}es ou non purg\'{e}es}\textso{,}}{\lemma{\textso{de mo\^{u}iller}}\Afootnote{ \textbar\ \textso{quelques fois} \textit{ erg.}\ \textbar\ \textso{ leurs superficies interieures de }\textso{liqueurs} \textbar\ \textso{purg\'{e}es ou non purg\'{e}es} \textit{ erg.}\ \textbar\ \textso{,} \textit{ erg.} \textit{ L}}}\textso{ d'essayer non seulement }\edtext{\textso{quelle est}}{\lemma{\textso{quelle est}}\Afootnote{\textit{ erg.} \textit{ L}}}\textso{ la force de l'union dans le }\textso{vuide}\protect\index{Sachverzeichnis}{vide}\textso{, ou dans l'air, mais m\^{e}me dans des liqueurs }\edtext{\textso{de toutes sortes}}{\lemma{\textso{liqueurs}}\Afootnote{ \textit{ (1) }\ \textso{differentes} \textit{ (2) }\ \textso{de toutes sortes} \textit{ L}}}\textso{, pour juger par la difference des }\textso{poids}\protect\index{Sachverzeichnis}{poids}\textso{ necessaires \`{a} la separation (la pesanteur de la liqueur estant soubstraicte) si cette union vient d'un mouuement invisible des liqueurs puisque il est \`{a} croire que ce mouuement s'il y en a n'est pas tousjours d'une m\^{e}me force dans des liqueurs differentes. Enfin on observera si l'union\linebreak estant surmont\'{e}e les placques se separent avec quelque son ou \'{e}clat. }\edlabel{dureste2}\edtext{}{\lemma{\textso{\'{e}clat.}}\Afootnote{ \textbar\ (2) \textit{ gestr.}\ \textbar\ Pour \textit{ L}}}%Fussnote nach oben verschoben\edtext{}{\lemma{\textso{Du reste}}\xxref{dureste1}{dureste2}\Afootnote{[...] \textso{\'{e}clat.} \textit{Markierung am Rand}}}
 [136 v\textsuperscript{o}] %\edtext{}{\lemma{}\Afootnote{  \textbar\ (2) \textit{ gestr.}\ \textbar\ Pour \textit{ L}}} 
Pour chercher la Raison des phenomenes de l'union des corps purgez\protect\index{Sachverzeichnis}{corps!purg\'{e}} d'air; il faut surtout examiner \edtext{si l'on les pourroit sauuer en supposant}{\lemma{examiner}\Afootnote{ \textit{ (1) }\ s'il est possible de supposer \textit{ (2) }\ si l'on  \textit{(a)}\ pourroit supposer commode \textit{(b)}\ les pourroit sauuer en supposant \textit{ L}}} une certaine \textso{gl\"{u}e} \edtext{ou \textso{colle}}{\lemma{ou \textso{colle}}\Afootnote{ \textit{ erg.} \textit{ L}}} ou autre raison de l'union qui se trouue dans les corps joints \edtext{même.}{\lemma{même}\Afootnote{ \textbar\ , comme est celle de l'aimant et fer\protect\index{Sachverzeichnis}{fer} \textit{ gestr.}\ \textbar\ . Et \textit{ L}}} Et il semble, que s'il y avoit une \textso{gl\"{u}e} les corps joints ne glisseroient pas l'un sur l'autre, comme on dit que les placques font, même dans le vuide\protect\index{Sachverzeichnis}{vide}. On pourroit \edtext{repartir}{\lemma{pourroit}\Afootnote{ \textit{ (1) }\ dire \textit{ (2) }\ repartir \textit{ L}}} que la liqueur purg\'{e}e\protect\index{Sachverzeichnis}{liqueur!purg\'{e}e} \edtext{au moins (puisque il faut renoncer aux placques)}{\lemma{}\Afootnote{au [...] placques) \textit{ erg.} \textit{ L}}} soit attach\'{e}e par une espece de gl\"{u}e non seulement au haut, mais aussi aux costez du tuyau.\edlabel{pourstart} %
\edtext{}{\lemma{\textso{Pour}}\xxref{pourstart}{pourend}\Afootnote{[...] \textso{l'aimant.} \textit{Markierung am Rand}}}%
\textso{Pour s'en \'{e}claircir }\edtext{\textso{donc}}{\lemma{}\Afootnote{\textso{donc} \textit{ erg.} \textit{ L}}}\textso{ on pourra \'{e}prouuer si la }\textso{liqueur }\protect\index{Sachverzeichnis}{liqueur!purg\'{e}e}\edtext{\textso{purg\'{e}e suiura le piston dans}}{\lemma{\textso{purg\'{e}e}}\Afootnote{ \textit{ (1) }\ \textso{s'eleve avec} \textit{ (2) }\ \textso{suiura le piston dans} \textit{ L}}}\textso{ une }\textso{pompe}\protect\index{Sachverzeichnis}{pompe}\textso{ outre la hauteur de la port\'{e}e ordinaire des }\textso{pompes}\protect\index{Sachverzeichnis}{pompe}\textso{: dans le }\textso{vuide}\protect\index{Sachverzeichnis}{vide}\textso{, et hors du }\textso{vuide}\protect\index{Sachverzeichnis}{vide}\textso{; }\edtext{\textso{car alors il n'y a point de gl\"{u}e}}{\lemma{}\Afootnote{\textso{car} [...] \textso{gl\"{u}e} \textit{ erg.} \textit{ L}}}\textso{. Item si un }\textso{siphon}\protect\index{Sachverzeichnis}{siphon}\textso{ \`{a} jambes inegalles plein}\edtext{\textso{ de la liqueur}}{\lemma{\textso{plein}}\Afootnote{ \textit{ (1) }\ \textso{d'eau} \textit{ (2) }\ \textso{ de la liqueur} \textit{ L}}}\textso{ purg\'{e}e ayant est\'{e} longtemps sans jouer, coule aisement }\edtext{\textso{, aussitost qu'il nous plaist de le faire agir.}}{\lemma{\textso{, aussitost}}\Afootnote{ [...] \textso{agir.} \textit{ erg.} \textit{ L}}}\textso{ Ces deux experiences estant faites, quand la }\textso{liqueur purg\'{e}e}\protect\index{Sachverzeichnis}{liqueur!purg\'{e}e}\textso{ a est\'{e} longtemps dans un même endroit du tuyau, feront aussi juger; si le repos augmente l'union, et fortifie la colle }\edtext{\textso{(qu'on pretend joindre}}{\lemma{\textso{colle}}\Afootnote{ \textit{ (1) }\ \textso{qui joint} \textit{ (2) }\ \textso{(qu'on pretend joindre} \textit{ L}}}\textso{ la liqueur aux costez du tuyau,) par une espece d'incorporation inveter\'{e}e. Et pour le même propos, il faudroit essayer de combien une placque de }\textso{fer}\protect\index{Sachverzeichnis}{fer}\textso{ soûten\"{u}e par un aimant glisse }\edtext{\textso{plus}}{\lemma{}\Afootnote{\textso{plus} \textit{ erg.} \textit{ L}}}\textso{ aisement \`{a} travers }\edtext{\textso{pendant qu'elle demeure attach\'{e}e \`{a} l'aimant}}{\lemma{}\Afootnote{\textso{pendant qu'elle demeure attach\'{e}e \`{a} l'aimant} \textit{ erg.} \textit{ L}}}\textso{, quoyque la separation directe soit difficile }\edtext{\textso{et si}\textso{ l'attachement des corps purgez}\protect\index{Sachverzeichnis}{corps!purg\'{e}}\textso{ a quelque rapport \`{a} celuy de }\edlabel{pourend}\textso{l'aimant}.}{\lemma{}\Afootnote{ \textit{ (1) }\ \textso{car il y a de l'apparence que} \textit{ (2) }\ \textso{et si} [...] \textso{l'aimant} \textit{ erg.} \textit{ L}}}%Fussnote nach oben verschoben \edtext{}{\lemma{\textso{Pour}}\xxref{pourstart}{pourend}
%\Afootnote{[...] \textso{l'aimant.} \textit{Markierung am Rand}}}
\pend 