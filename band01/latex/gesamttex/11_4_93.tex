\pend \newpage \pstart [p.~93] [...] De l\`{a} on peut connoistre que si la porte a 3. pieds de large, comme ont celles\textendash cy; elle aura aussi 3. pieds pour son diametre\footnote{\textit{Leibniz korrigiert} diametre \textit{in} demydiametre} AC, [...].\pend \pstart  Toutes les Ouuertures se sont par les mesmes reigles, comme l'on void encore les Portes K, et L. La Porte K, monstre son dehors, et la porte L, monstre son dedans\footnote{\textit{Leibniz korrigiert} dehors, et la porte L, monstre son dedans \textit{in} dedans, et la porte L, monstre son dehors}; [...].