[160 v\textsuperscript{o}] Witsen\protect\index{Namensregister}{\textso{Witsen,} Nicolaes 1641\textendash 1717} pag. 269. \edtext{Eine\edlabel{auffgeroltestart}}{{\xxref{auffgeroltestart}{auffgerolteend}}\lemma{Eine}\Bfootnote{Von Eine auffgerolte bis wehen zu laßen vgl. \textsc{N. Witsen}, \cite{00153}a.a.O., S.~269.}} auffgerolte vlag\protect\index{Sachverzeichnis}{vlag}, hinden aus, bedeut daß iemand der nicht am bord ist, geruffen wird, oder sonst eine sach von n\"{o}then hat. \textit{De }\textit{vlag}\protect\index{Sachverzeichnis}{vlag}\textit{ ter halver steng beteekent een }\edtext{\textit{doode}}{\lemma{\textit{een}}\Afootnote{ \textit{ (1) }\ \textit{dode} \textit{ (2) }\ \textit{doode} \textit{ L}}}\textit{ te }\textit{scheep}\protect\index{Sachverzeichnis}{schip}. Wann das schiffvolck rebel ist, so in langen vaerten bis weilen geschiht, sind sie gewohnt \textit{alle vlaggen}\protect\index{Sachverzeichnis}{vlag} zu streichen, und das ge\"{u}ßje oder \textit{vlaggetje}\protect\index{Sachverzeichnis}{vlaggetje} von de blinde steng\protect\index{Sachverzeichnis}{steng} allein wehen zu laßen\edlabel{auffgerolteend}. \edtext{Aus\edlabel{ausaltenstart}}{{\xxref{ausaltenstart}{ausaltenend}}\lemma{Aus}\Bfootnote{Von Aus alten bis mehr gestrichen vgl. \textsc{N. Witsen}, \cite{00153}a.a.O., S.~270.}} alten Hehrkomen streichen die staat schiffen\protect\index{Sachverzeichnis}{Staatsschiff} vor Englischen orlogs schiffen\protect\index{Sachverzeichnis}{oorlogsschip}, \textit{haren top-vlag}\protect\index{Sachverzeichnis}{top\textendash vlag} en \edtext{overseegel}{\lemma{en}\Afootnote{ \textit{ (1) }\ overseil \textit{ (2) }\ overseegel \textit{ L}}}, \textit{en dat ter halver steng}\protect\index{Sachverzeichnis}{steng}. Schiffen\protect\index{Sachverzeichnis}{Schiff} der republicken begr\"{u}ßen erst k\"{o}nigliche. Und solches mit schießen, it achter umb zu lauffen, boot\protect\index{Sachverzeichnis}{Boot} außsezen, etc. doch ohne Zwang. Will man achter umblauffen, so schießet man alsdann, erst wann, man herumb ist. Jeder Konigs viceadmiral\protect\index{Sachverzeichnis}{Viceadmiral} gr\"{u}ßet die Staten amiral\protect\index{Sachverzeichnis}{Admiral} mit gleichen schießen. Wenn man in einen vremden hafen\protect\index{Sachverzeichnis}{Hafen} komt, gr\"{u}ßet man auch, und sie antworten, doch ohne verbindung. Das Seegel\protect\index{Sachverzeichnis}{Segel} streichen, ist minder als das flagge streichen, denn \edtext{jenes}{\lemma{denn}\Afootnote{ \textit{ (1) }\ solch \textit{ (2) }\ jenes \textit{ L}}} die Konige ehe zu geben. Kauffardey schiffe\protect\index{Sachverzeichnis}{koopvaardijschip}, so fremds orlogs schiffe\protect\index{Sachverzeichnis}{oorlogsschip}, \edtext{begegen}{\lemma{schiffe,}\Afootnote{ \textit{ (1) }\ ent \textit{ (2) }\ begegen \textit{ L}}} gr\"{u}ßen sie in see. Wenn man die flagge\protect\index{Sachverzeichnis}{Flagge} streicht laßet man das segel\protect\index{Sachverzeichnis}{Segel} stehen, beyde streichen ist uberfl\"{u}ßig. Dieweil aber die wenigsten schiffe\protect\index{Sachverzeichnis}{Schiff} flaggen\protect\index{Sachverzeichnis}{Flagge} f\"{u}hren, werden die segel\protect\index{Sachverzeichnis}{Segel} mehr gestrichen\edlabel{ausaltenend}. \edtext{Vor\edlabel{vordiessenstart}}{{\xxref{vordiessenstart}{vordiessenend}}\lemma{Vor}\Bfootnote{Von Vor dießen bis zur see gehalten vgl. \textsc{N. Witsen}, \cite{00153}a.a.O., S.~271. Die Notizen entsprechen nicht der Abfolge der Darstellung auf der Seite.}} dießen pflegten auch gemeine Staaten orlogs schiffe\protect\index{Sachverzeichnis}{oorlogsschip} flaggen\protect\index{Sachverzeichnis}{Flagge} zu wehen, aniezo aber umb streit zu vermeiden unterl\"{a}st mans. Venetianische schiffe\protect\index{Sachverzeichnis}{Schiff!venezianisch} werden von staten schif\protect\index{Sachverzeichnis}{Staatsschiff} erst begr\"{u}st, si pares, weil solches die \"{a}lteste republick. Genuesische\protect\index{Sachverzeichnis}{Schiff!genuesisch} aber hingegen \edtext{sollen}{\lemma{hingegen}\Afootnote{ \textit{ (1) }\ m\"{u}ßen \textit{ (2) }\ sollen \textit{ L}}} erst gr\"{u}ßen. In fremden haven\protect\index{Sachverzeichnis}{Hafen} ist $\langle$komen$\rangle$\edtext{}{\lemma{komen}\Afootnote{Lesung unsicher.}} keinem fremden ehre zu beweisen schuldig. Wenn orlog schiffe\protect\index{Sachverzeichnis}{oorlogsschip} scheiden schiest der geringere erst. Republicken erwarten erst, das grußen der schiffe\protect\index{Sachverzeichnis}{Schiff} der Heren so minder als Konige.\pend \pstart In aus gehen aus fremden hafen\protect\index{Sachverzeichnis}{Hafen} ist man nicht schuldig ehre zu beweisen denn kastell von orlogs schiff\protect\index{Sachverzeichnis}{oorlogsschip} man pflegts doch offt zu thun. \pend \pstart Bey den franzosen wird dieses gehalten daß keine amirals\protect\index{Sachverzeichnis}{Admiral} vlagge\protect\index{Sachverzeichnis}{vlag} auffgesteckt wird, es were denn eine flotte\protect\index{Sachverzeichnis}{Flotte} von 20 orlogs schiffen\protect\index{Sachverzeichnis}{oorlogsschip} umb solche vlagg zu vertheidigen. Und zweilf wehrbar orlogs schiffe\protect\index{Sachverzeichnis}{oorlogsschip} mußen seyn, daß eines franzen viceamirals\protect\index{Sachverzeichnis}{Viceadmiral} oder schulz bey nacht \textit{vlagge}\protect\index{Sachverzeichnis}{vlag} auffgesteckt werde. Von welche schiffen\protect\index{Sachverzeichnis}{Schiff} den geringsten f\"{u}hren mus 36 st\"{u}ck. \pend \pstart Vor einen nicht \"{u}ber lauffen, ob man schohn kan, wird vor eine hofligkeit zur see gehalten\edlabel{vordiessenend}. \edtext{Die Hinterflagge wird nimmer gestrichen als wenn ein schiff\protect\index{Sachverzeichnis}{Schiff} \"{u}bermannt. pag 272. etc}{\lemma{Hinterflagge}\Bfootnote{Diese Bemerkung zum Streichen der Hinterflagge bei Witsen \cite{00153}a.a.O., S.~270.}}\pend \pstart \edtext{Beschreibung von einem Hafen\protect\index{Sachverzeichnis}{Hafen} von Texel\protect\index{Ortsregister}{Texel} Witsen\protect\index{Namensregister}{\textso{Witsen,} Nicolaes 1641\textendash 1717} pag 481.}{\lemma{Beschreibung}\Bfootnote{F\"{u}r diese Beschreibung vgl. \textsc{N. Witsen}, \cite{00153}a.a.O., S.~477\textendash 481.}}\pend \pstart \edtext{De Navi}{\lemma{De Navi\protect\index{Sachverzeichnis}{navis}}\Bfootnote{Der Bericht \"{u}ber die Hebung des Schiffes aus dem See bei Nemi nicht bei Witsen.}} \edtext{in Nemorensi}{\lemma{Navi}\Afootnote{ \textit{ (1) }\ Nemorensis \textit{ (2) }\ in Nemorensi \textit{ L}}} lacu reperta relatio, quam Nic. Stenonis\protect\index{Namensregister}{\textso{Stensen} (Stenonis), Nils 1638\textendash 1686} misit ex Mso Francesci Gualdi\protect\index{Namensregister}{\textso{Gualdi,} Francesco 1576\textendash 1657} antiquitatum curiosi Card. Prosper Colonna dominus\protect\index{Namensregister}{\textso{Colonna,} Prospero 1452\textendash 1523} lacus, cum ab incolis id didicißet in eo latere naves\protect\index{Sachverzeichnis}{navis} duas veteres summersas vocavit \edtext{Leon. Bapt.}{\lemma{vocavit}\Afootnote{ \textit{ (1) }\ Leand. \textit{ (2) }\ Leon. Bapt. \textit{ L}}} Albertum\protect\index{Namensregister}{\textso{Alberti,} Leon Baptista 1404\textendash 1472}, qui extrahi curavit\edtext{}{\lemma{curavit}\Bfootnote{Card. P. Colonna beauftragte L. B. Alberti im Jahre 1446, zwei untergegangene, antike Schiffe aus dem Nemisee zu heben.}}. Es brach im ausziehen, und ward nur ein theil herausgezogen aus holz Laryx\protect\index{Sachverzeichnis}{Larchenholz@L\"{a}rchenholz} bedeckt mit eine krust. Beschlagen mit blatten von bley\protect\index{Sachverzeichnis}{Blei} und metallen Nagelen\protect\index{Sachverzeichnis}{Metallnagel}. Inwendig das holz\protect\index{Sachverzeichnis}{Holz} mit krijt\protect\index{Sachverzeichnis}{krijt} bestrichen eins fingers dick, op ander twee, auff drees plaister, \edtext{gegoßen eisen und uber den eisen, weil es noch heiß, wiederumb}{\lemma{plaister,}\Afootnote{ \textit{ (1) }\ wiederumb \textit{ (2) }\ gegoßen [...] wiederumb \textit{ L}}} bleyster von krijt\protect\index{Sachverzeichnis}{krijt}, daß es also eine krust machte. Man fand einige \edtext{loude buysten (bleyerne)}{\lemma{einige}\Afootnote{ \textit{ (1) }\ kupf \textit{ (2) }\ bleyerne \textit{ (3) }\ loude buysten (bleyerne) \textit{ L}}} mit buchstaben daraus man schloß daß es zu zeiten kayßers Caligulae\protect\index{Namensregister}{\textso{Caligula,} Kaiser in Rom 12\textendash 41}. 3 nageln\protect\index{Sachverzeichnis}{Nagel}, und ein stuck von holz larix\protect\index{Sachverzeichnis}{Larchenholz@L\"{a}rchenholz} so unter waßer nicht verrost, neben einen plaister von metall langer den ein palm und breiter als ein halber palm darinn einige pferde auffs beste abgebild waren, so eine jagt bedeuten solten aber ziemlich durch die zeit verschlißen hat der Herr Marcgraf Frangipani\protect\index{Namensregister}{\textso{Istrien: Franz Christoph}, Markgraf von Istrien ?\textendash 1671} herr des orths gegeben an den \edtext{Cabinet}{\lemma{den}\Afootnote{ \textit{ (1) }\ Cusien \textit{ (2) }\ Cabinet \textit{ L}}} von den Ritter Gualdi\protect\index{Namensregister}{\textso{Gualdi,} Francesco 1576\textendash 1657},\rule[-10mm]{0mm}{0mm} en daerna aen Seyne Maist Lodewijck de XIV konig von franckreich\protect\index{Namensregister}{\textso{Frankreich: Ludwig XIV.}, K\"{o}nig von Frankreich, 1643\textendash 1715}; welches mit andern ungemeinen \edtext{von taten}{\lemma{ungemeinen}\Afootnote{ \textit{ (1) }\ N \textit{ (2) }\ von taten \textit{ L}}} berichtet wird in dem Convent von Franciscus von Paola\protect\index{Namensregister}{\textso{Francesco di Paola} (Franciscus von Paola), 1416\textendash 1507} auff den Berg Pincius\protect\index{Ortsregister}{Pincio} zu Rom\protect\index{Ortsregister}{Rom (Roma)}.\pend \pstart \edtext{Die\edlabel{opnehmenstart}}{{\xxref{opnehmenstart}{opnehmenend}}\lemma{Die}\Bfootnote{Von Die Schiff bis seiten opnehmen vgl. \textsc{N. Witsen}, \cite{00153}a.a.O., S.~266.}} schiff\protect\index{Sachverzeichnis}{Schiff} segeln am besten die am schwersten in der mitt geballast (pag 266).\pend \pstart Unser kriegs schiffe\protect\index{Sachverzeichnis}{Kriegsschiff} oben gebogen, dann daraus eine großere menge personen fechten kan, und ihnen mehr am fechten als segeln\protect\index{Sachverzeichnis}{Segel} gelegen.\pend \pstart Kriegs schiff\protect\index{Sachverzeichnis}{Kriegsschiff} so oben \textit{naeuw} (+ puto eng +), dieweil sie nicht leicht geentert werden konnen, dann ihre \edtext{\textit{puilende}}{\lemma{ihre}\Afootnote{ \textit{ (1) }\ puti \textit{ (2) }\ puilende \textit{ L}}} \textit{buiken} machen das \textit{boort} und zu springlich aber hingegen \edtext{haben sie dieß das}{\lemma{}\Afootnote{haben sie dieß das \textit{ erg.} \textit{ L}}}, alße \edtext{\textit{op}}{\lemma{alße}\Afootnote{ \textit{ (1) }\ ob \textit{ (2) }\ ov \textit{ (3) }\ \textit{op} \textit{ L}}} \textit{'t zy legen meerder hellen} on lijchter water von der seiten opnehmen\edlabel{opnehmenend}.\pend \pstart \edtext{Ich\edlabel{icherinnerestart}}{{\xxref{icherinnerestart}{icherinnereend}}\lemma{Ich}\Bfootnote{Von Ich erinnere bis der golfen vgl. \textsc{N. Witsen}, \cite{00153}a.a.O., S.~274.}} erinnere mich sagt Witsen\protect\index{Namensregister}{\textso{Witsen,} Nicolaes 1641\textendash 1717} pag 274, daß ich in einen schiff\protect\index{Sachverzeichnis}{Schiff} der windhund genant nach Riga\protect\index{Ortsregister}{Riga} fuhr. Dieß schiff\protect\index{Sachverzeichnis}{Schiff} seegelte troz einen des ganzen landes aber sobald wier, mit unseren reis Zeug das schiff\protect\index{Sachverzeichnis}{Schiff} beladen giengs auch andere schiff\protect\index{Sachverzeichnis}{Schiff}, die doch ihre volle ladung hatten, beßer als unseres.\pend \pstart Viele schiffe\protect\index{Sachverzeichnis}{Schiff} in Barbarien\protect\index{Ortsregister}{Deutschland (Germania, Duitsland)} gebaut, die auff den staab gehen, sind vorn und hinten so \textit{geschoort}, daß sie anstatt von loßche \textit{in houten}, gaffel st\"{u}cken \edtext{op}{\lemma{st\"{u}cken}\Afootnote{ \textit{ (1) }\ auff \textit{ (2) }\ op \textit{ L}}} \edtext{}{\lemma{}\Afootnote{ \textbar\ op \textit{ gestr.}\ \textbar\  \textit{ erg.} \textit{ L}}} die \textit{kiel}\protect\index{Sachverzeichnis}{kiel}\textit{ hebben staen, die onder mit leemiger aerde gevult seyn: vooren zwaer }\textit{seil}\protect\index{Sachverzeichnis}{zeil}\textit{, en hun boegen seyn }\edtext{\textit{rondachtig}}{\lemma{\textit{seyn}}\Afootnote{ \textit{ (1) }\ \textit{rontachtig} \textit{ (2) }\ \textit{rondachtig} \textit{ L}}}\textit{. Soo dat de schot des waters na aen het hart van 't }\textit{schip}\protect\index{Sachverzeichnis}{schip}\textit{ erst gestuit word, als het }\textit{schip}\protect\index{Sachverzeichnis}{schip}\textit{ de }\edtext{\textit{zeebaer}}{\lemma{\textit{de}}\Afootnote{ \textit{ (1) }\ \textit{zeebar} \textit{ (2) }\ \textit{zeebaer} \textit{ L}}}\textit{ schier over ist} \edtext{die}{\lemma{\textit{ist}}\Afootnote{ \textit{ (1) }\ also \textit{ (2) }\ die \textit{ L}}} segeln sehr geschwind, welches unse kauffleite mit schaden erfahren. Ihre \edtext{masten}{\lemma{Ihre}\Afootnote{ \textit{ (1) }\ masen \textit{ (2) }\ masten \textit{ L}}} sind \textit{gaef}, und von guth holz\protect\index{Sachverzeichnis}{Holz}, beßer als das nordsche.\pend \pstart By still waßer \textit{is het even eens}, von was gestalt das shiff\protect\index{Sachverzeichnis}{Schiff} vorn ist plat, rond oder spiz, dann alsdann wenig waßer to \textit{verdouwen} falt, \textit{ronde boegen} anders sinst brecken am besten den anfall der golfen\edlabel{icherinnereend}.\pend \clearpage \pstart \edtext{Wann\edlabel{eingeschehenstart}}{{\xxref{eingeschehenstart}{eingeschehenend}}\lemma{Wann}\Bfootnote{Von Wann ein geschehen bis holz eingeschlagen vgl. \textsc{N. Witsen}, \cite{00153}a.a.O., S.~276.}} ein geschehen Loch unter waßer von innen nicht kan \edtext{gleich}{\lemma{}\Afootnote{gleich \textit{ erg.} \textit{ L}}} gestopt werden, als wenn etwa die last im wege stehet, so thut man (+ unterdeßen +) dießes: man l\"{a}ßet einen man außer bort mit ein \textit{prop} in der hand auff ein planckjen gesezt daer \textit{en dreg aen vast ist, die hem onde waßer haelt. En alduß stopt af deckt hy die ophening. Man }\edtext{\textit{geeft}}{\lemma{\textit{Man}}\Afootnote{ \textit{ (1) }\ \textit{geheft} \textit{ (2) }\ \textit{geeft} \textit{ L}}}\textit{ hem en }\textit{geolid lap}\protect\index{Sachverzeichnis}{geolid lap}\textit{ (+ oleo imbutum +) in de Mond, omt water uyt lichaem te weren}. Dieß muß in der Eil geschehen und der man geschwind wieder ubers waßer gezogen werden (+ videndum an non fieri poßit, ut aqua intrans secum ferat aliquid \edtext{injectum quod}{\lemma{aliquid}\Afootnote{ \textit{ (1) }\ quod \textit{ (2) }\ injectum quod \textit{ L}}} inflatum locum claudat \lbrack +) \rbrack. Ein kabel von 100 faden hat mehr als 3 mahl sovil krafft vonnothen umb gespannt zu werden, als eine von hunderten.\pend \pstart Es ist nicht das geringste miraculum der schiffart, daß ein ancker\protect\index{Sachverzeichnis}{Anker} mit seinem kabel das schiff\protect\index{Sachverzeichnis}{Schiff} fest h\"{a}lt, welches wohl 3 oder 400 mahl mehr wiegt als der ancker\protect\index{Sachverzeichnis}{Anker} mit seinem kabel. Es scheint man habe dieses von den \edtext{\textit{kreeften}}{\lemma{den}\Afootnote{ \textit{ (1) }\ kreften \textit{ (2) }\ \textit{kreeften} \textit{ L}}} (+ an kreebsen ? +) gelernt welche in sturms Zeit sich mit ihren potten am grund anckern, umb nicht gegen klippen geschlagen zu werden. \edtext{Die winkel der}{\lemma{werden.}\Afootnote{ \textit{ (1) }\ Das gewicht de \textit{ (2) }\ Die winkel der \textit{ L}}} armen mußen scharff ecken gnug fallen auff die roede, sonst (+ weren sie wie perpendicular auff den grund +) wurden nicht eingehen, wie ein nagel\protect\index{Sachverzeichnis}{Nagel} ins holz\protect\index{Sachverzeichnis}{Holz} eingeschlagen\edlabel{eingeschehenend}. \edtext{Ihr\edlabel{ihrgewichtstart}}{{\xxref{ihrgewichtstart}{ihrgewichtend}}\lemma{Ihr}\Bfootnote{Von Ihr gewicht bis guther wind vgl. \textsc{N. Witsen}, \cite{00153}a.a.O., S.~277.}} gewicht thut viel \edtext{dazu}{\lemma{viel}\Afootnote{ \textit{ (1) }\ zu ein \textit{ (2) }\ dazu \textit{ L}}} daß sie in den grund gehen. Und ie mehr der winkel der armen \edtext{zu}{\lemma{armen}\Afootnote{ \textit{ (1) }\ Zu \textit{ (2) }\ zu \textit{ L}}} der roede scharff ist, ie neher \edtext{und}{\lemma{}\Afootnote{und \textit{ erg.} \textit{ L}}} das gewicht vom ganzen ancker\protect\index{Sachverzeichnis}{Anker} auff der Klau\protect\index{Sachverzeichnis}{Klaue} ruhet, und also dieffer ingehen soll. Hierbey kan man mercken, wann das ancker\protect\index{Sachverzeichnis}{Anker} den grund erst r\"{u}hret, und platt darauff liegt daß wann der strick beginnt zu ziehen und die ruthe \edtext{auffgehoben}{\lemma{ruthe}\Afootnote{ \textit{ (1) }\ ? das \textit{ (2) }\ ob \textit{ (3) }\ auffgehoben \textit{ L}}} wird lastende die zwey armen l\"{a}ngs den grund \textit{glijden} (+ glitschen +), daß sie also alzeit vort ritschen solten, wenn sie nicht die drey \edtext{winckeligten Klauen}{\lemma{drey}\Afootnote{ \textit{ (1) }\ eckigten klauen des \textit{ (2) }\ winckeligten Klauen \textit{ L}}} des grundes ungleichheit befindende, \textit{de eene wat mehr, de andere was min hechtede\ \textendash\ tot dat de roe }\edtext{\textit{op}}{\lemma{\textit{roe}}\Afootnote{ \textit{ (1) }\ \textit{or} \textit{ (2) }\ \textit{op} \textit{ L}}}\textit{ de armen die 't meest in de grond vast ist, begint de }\edtext{\textit{draeien}}{\lemma{\textit{de}}\Afootnote{ \textit{ (1) }\ \textit{draien} \textit{ (2) }\ \textit{draeien} \textit{ L}}}\textit{ en sich te verheffen, 't welck danen na mate dat het }\textit{ancker}\protect\index{Sachverzeichnis}{Anker}\textit{ voortgetrocken }\edtext{\textit{word}}{\lemma{\textit{voortgetrocken}}\Afootnote{ \textit{ (1) }\ \textit{wird} \textit{ (2) }\ \textit{word} \textit{ L}}}\textit{ doer de scherphoeckigkeit, die de arm mit de roede maeckt, de }\textit{klaue}\protect\index{Sachverzeichnis}{klauw}\textit{ doet in de grond sincken}. Ein ancker\protect\index{Sachverzeichnis}{Anker} \edtext{mit}{\lemma{ancker}\Afootnote{ \textit{ (1) }\ soll \textit{ (2) }\ mit \textit{ L}}} einer langen ruthe wird fester halten als mit einer kurzen dieweil eine lange ruthe vom schiff\protect\index{Sachverzeichnis}{Schiff} bewogen, mehr bewegung an den sand bringt, als eine kurze, und also halt der ancker\protect\index{Sachverzeichnis}{Anker} fester.\pend \pstart Die anckerßtock\protect\index{Sachverzeichnis}{Ankerstock} die mit yserne \textit{spyen} \edtext{\textit{aen 't}}{\lemma{\textit{spyen}}\Afootnote{ \textit{ (1) }\ aent \textit{ (2) }\ \textit{aen 't} \textit{ L}}} ancker \textit{by de ring, vast gemaeckt} word, \textit{behoorde} solangen nach vieler meinungen, als die ruthe zu seyn, und etwa \edtext{ein}{\lemma{etwa}\Afootnote{ \textit{ (1) }\ die \textit{ (2) }\ ein \textit{ L}}} funftheil des gewichts des anckers\protect\index{Sachverzeichnis}{Anker} zu wiegen. Dieses holz\protect\index{Sachverzeichnis}{Holz} helfft mit, daß der ancker\protect\index{Sachverzeichnis}{Anker} auff einer der armen ruhte \textit{om dat belet} (quia vetat) das der ancker\protect\index{Sachverzeichnis}{Anker} nicht ganz platt nieder liegt. Und also werden die Klauen\protect\index{Sachverzeichnis}{Klaue} gezwungen ins land zu graben, sowohl auch dieweil ein Holz\protect\index{Sachverzeichnis}{Holz} im waßer \edtext{\textit{noit}}{\lemma{waßer}\Afootnote{ \textit{ (1) }\ niit \textit{ (2) }\ \textit{noit} \textit{ L}}} \textit{over eint maer waterpaß tracht to drijven}. Den anckerstock\protect\index{Sachverzeichnis}{Ankerstock} dan waterpaß sijnde, \textit{dwingt de eene} klauw\protect\index{Sachverzeichnis}{klauw} \textit{op 't} sand \textit{te staen, en by gevolg in de} grond zu bohren. Wenn das \edtext{meer}{\lemma{das}\Afootnote{ \textit{ (1) }\ schiff \textit{ (2) }\ meer \textit{ L}}} still und ohne wind und ohne strohm, so kan man nicht \edtext{anckern}{\lemma{nicht}\Afootnote{ \textit{ (1) }\ ancker werffen \textit{ (2) }\ anckern \textit{ L}}}. Ein ancker\protect\index{Sachverzeichnis}{Anker} zu einem großen schiff\protect\index{Sachverzeichnis}{Schiff} nach proportion kleiner als zu einem kleinen. Denn ein groß schiff\protect\index{Sachverzeichnis}{Schiff} ohne das das waßer viel wiederstehet, und was einen kleinen tempeste ist, ist einen großen guther wind\edlabel{ihrgewichtend}.\pend \pstart \edtext{Man\edlabel{manhohretstart}}{{\xxref{manhohretstart}{manhohretend}}\lemma{Man}\Bfootnote{Von Man hohret bis zum Ende des Textes vgl. \textsc{N. Witsen}, \cite{00153}a.a.O., S.~278.}} hohret selten daß Kabel von 18 bis 20 daum in der runde reißen, maßen ein dunnen tow ye ein man oder 2 krafft wiederstehet, was sollen 300 solche douvetje nicht thun k\"{o}nnen. Item ein langer kabel ist schwehrer zu spannen, also auch schwehrer zu reißen. 4 man sollen kaum ein tow spannen k\"{o}nnen, so ein mann tragen kan. Man wird lieber 2 kabels an einen ancker\protect\index{Sachverzeichnis}{Anker} thun, als 2 ancker\protect\index{Sachverzeichnis}{Anker} auswerffen, denn das schiff\protect\index{Sachverzeichnis}{Schiff} nur einen ziehet, und die\edlabel{manhohretend} [\textit{Satz bricht ab}]\pend 