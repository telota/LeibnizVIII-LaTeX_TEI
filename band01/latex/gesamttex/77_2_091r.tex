[91 r\textsuperscript{o}]  necessarium duxi adjungere hic quo modo unico solummodo vitro\footnote{@@@ G R A F I K @@@ vide sequentia: \textso{ac duorum pluriumne compositione} etc: retro signo @@@ G R A F I K @@@, vertendo ter}  congregari minoremque focum efficere debere; si itaque \textit{BF},  adeo parva sumatur ut illa pro puncto mechanico\protect\index{Sachverzeichnis}{punctum!mechanicum} habenda  sit, ponaturque; vitrum eam habere figuram, quam \textit{FDB}  circa axem \textit{DF} rotata, describit, manifestum est hujus  modi vitrum in tantum considerari posse, ac si figuram  hyperbolicam plano sectam haberet, et ope ejus componi  posse omnis generis specilla\protect\index{Sachverzeichnis}{specillum} eo modo quo id a Do\textsuperscript{no} des Cartes>\protect\index{Namensregister}{\textso{Descartes} (Cartesius, des Cartes, Cartes.), Ren\'{e} 1596\textendash 1650} ope hyperbolicorum vitrorum factum est. Denique  notandum etiam est nullas figuras politu esse faciliores,  quam hae ipsae sunt, cum constent circulari figura et plana,  quae nullam ad invicem habent relationem, \edtext{}{\lemma{}\Afootnote{relationem,  \textbar\ nec ut \textit{ gestr.}\ \textbar\ nam \textit{ L}}}nam  nec ut planum sit ad angulos rectos ad axem \textit{DN}, nec  ad vitri crassitiem attendere, necesse est.\pend \pstart  Facile praeterea ex his explicare possem figuras  ac compositionis modum vitrorum, tam telescopiorum\protect\index{Sachverzeichnis}{telescopium}, quam  etiam microscopiorum\protect\index{Sachverzeichnis}{microscopium} quae hactenus observavi effectum  aliquem notabilem habuisse; ac etiam quo modo ex utraque  parte convexa haec vitra aut inter sese, aut cum aliis 