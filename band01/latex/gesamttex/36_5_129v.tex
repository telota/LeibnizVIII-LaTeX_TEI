[129 v\textsuperscript{o}] se dilate, comme nous s\c{c}avons, qu'elle fait. Et il ne suffit pas de dire que cette matiere subtile\protect\index{Sachverzeichnis}{mati\`{e}re!subtile}, trouuant de la place dans la bulle frappe ainsi la liqueur\protect\index{Sachverzeichnis}{liqueur} suspend\"{u}e, de deux costez, et la repousse, autant qu'il l'a pouss\'{e} vers le verre.\pend \pstart \footnotesize Car sans insister sur ce que même cette pression \edtext{ne suffrira pas, que la bulle}{\lemma{pression}\Afootnote{ \textit{ (1) }\ empechera la generation de la Bulle, et sur tout qu'elle \textit{ (2) }\ ne [...] bulle \textit{ L}}} se place entre la liqueur\protect\index{Sachverzeichnis}{liqueur} suspend\"{u}e et la surface interieure du verre; il faut considerer, que le peu de coups du mouuement en tous sens de la matiere subtile\protect\index{Sachverzeichnis}{mati\`{e}re!subtile} insinu\'{e}e dans une petite bulle ne peut pas \'{e}galer ny d\'{e}truire tous les autres que la liqueur\protect\index{Sachverzeichnis}{liqueur!purg\'{e}e} purg\'{e}e recoit \edtext{du cost\'{e}}{\lemma{recoit}\Afootnote{ \textit{ (1) }\ de tous \textit{ (2) }\ du cost\'{e} \textit{ L}}} de l'ouuerture du tuyau par embas, par lesquels elle est pouss\'{e} de bas en haut, vers la superficie interieure du verre.\pend \pstart \footnotesize On repondra peut estre, qu'il suffit \`{a} la matiere subtile\protect\index{Sachverzeichnis}{mati\`{e}re!subtile} \edtext{pour faire tomber la liqueur\protect\index{Sachverzeichnis}{liqueur} sus\-pend\"{u}e}{\lemma{}\Afootnote{pour faire tomber la liqueur\protect\index{Sachverzeichnis}{liqueur} suspend\"{u}e \textit{ erg.} \textit{ L}}}, d'avoir trouu\'{e} un petit passage, pour la presser de deux costez, comme nous voyons que le Mercure\protect\index{Sachverzeichnis}{mercure} suspendu dans le tuyau de Torricelli\protect\index{Sachverzeichnis}{tuyau de Torricelli} tombe, si l'on perce le haut du tuyau avec une \'{e}pingle, parce que l'air le presse de deux costez. Mais il y a bien de repliques \`{a} faire \`{a} cette response. Car premierement je ne puis pas voir comment une bulle d'air donne plus de passage \`{a} cette matiere subtile\protect\index{Sachverzeichnis}{mati\`{e}re!subtile}, qu'elle n'avoit pas, et pourquoy la bulle d'air en soit plus remplie que la liqueur\protect\index{Sachverzeichnis}{liqueur} \edtext{suspend\"{u}e}{\lemma{}\Afootnote{suspend\"{u}e \textit{ erg.} \textit{ L}}} même. \`{A} moins qu'on ne dise, que cette matiere subtile\protect\index{Sachverzeichnis}{mati\`{e}re!subtile} est l'air même ou dans l'air seulement; \edtext{la matiere subtile\protect\index{Sachverzeichnis}{mati\`{e}re!subtile} ou plustost la force de l'air de la bulle \'{e}galent celle du reste de l'air dans le Recipient qui agit sur la liqueur\protect\index{Sachverzeichnis}{liqueur} suspend\"{u}e.}{\lemma{}\Afootnote{la matiere [...] suspend\"{u}e. \textit{ erg.} \textit{ L}}} Ce qui est mon sentiment en effect; sans employer autre matiere subtile\protect\index{Sachverzeichnis}{mati\`{e}re!subtile} que l'air même, ny autre mouuement ou plustost effort en tous sens, que celuy du ressort de l'air \edtext{comme je l'expliqueray plus amplement par apres.}{\lemma{}\Afootnote{comme [...] apres. \textit{ erg.} \textit{ L}}} Mais il me semble que ceux, qui se servent icy du mouuement d'une matiere subtile\protect\index{Sachverzeichnis}{mati\`{e}re!subtile} en tous sens s'y prennent d'un tout autre biais. Car voyant qu'on tire de l'air du Recipient, ils supposent que necessairement quelqu'autre matiere entre par les pores mêmes du verre, pour remplir la place de l'air tir\'{e}, \`{a} moins qu'on ne veuille avouer qu'il se fait un vuide dans le Recipient. Et ils se fondent sur cette maxime approuu\'{e}e de des \edtext{Cartes\protect\index{Namensregister}{\textso{Descartes} (Cartesius, des Cartes, Cartes.), Ren\'{e} 1596\textendash 1650}}{\lemma{Cartes}\Bfootnote{\textsc{R. Descartes}, \cite{00035}\textit{Principia philosophiae}, Amsterdam 1644, S. 36 (\textit{DO} VIII, 1, S.~43).}}; aussi bien que de Gassendi\protect\index{Namensregister}{\textso{Gassendi} (Gassendus), Pierre 1592\textendash 1655}, s\c{c}avoir qu'une quantit\'{e} determin\'{e}e de la matiere ne peut changer de volume, ny occuper une place plus ou moins grande que par interposition d'une autre \edtext{matiere.}{\lemma{matiere.}\Bfootnote{\textsc{P. Gassendi}, \cite{00244}\textit{Physica}, in: \textit{Opera omnia}, Bd. 1, Lyon 1656, S.~193f. (\textit{GOO} I, S.~193f.).}} Mais cette maxime,\edtext{}{\lemma{}\Afootnote{maxime,  \textbar\ contraire non seulement \`{a} Aristote\protect\index{Namensregister}{\textso{Aristoteles,} 384\textendash 322 v. Chr.}, mais aussi \`{a} Galilaei\protect\index{Namensregister}{\textso{Galilei} (Galilaeus, Galileus), Galileo 1564\textendash 1642}, \textit{ gestr.}\ \textbar\ a \textit{ L}}} a bien besoin de demonstration. Car je croy de pouuoir montrer, que le mouuement, ou l'effort au mouuement peut faire un même corps\protect\index{Sachverzeichnis}{corps} occuper ou plus ou moins d'espace dans un même moment donn\'{e}, quoyque je s\c{c}ache que cela paroîtra aussi paradoxe aux philosophes de nostre temps, que le contraire \`{a} ceux du pass\'{e}. \normalsize Ma demonstration depend de la connoissance 