[161 v\textsuperscript{o}] \edtext{certain\edlabel{certainstart}}{{\xxref{certainstart}{certainend}}\lemma{certain}\Bfootnote{Von certain viscus bis poetis pinus vgl. \textsc{N. Witsen}, \cite{00153}a.a.O., S.~181.}} viscus\protect\index{Sachverzeichnis}{viscum} qui resistoit au feu. Gellius\protect\index{Namensregister}{\textso{Gellius,} Aulus ca. 130\textendash 180} dit \edtext{[qu'on trouva]\edtext{}{\Afootnote{que trouva\textit{\ L \"{a}ndert Hrsg. } }}}{\lemma{dit}\Afootnote{ \textit{ (1) }\ que trouver \textit{ (2) }\ [qu'on trouva] \textit{ L}}} dans le port d'Athenes\protect\index{Ortsregister}{Athen (Athenes)} un vaisseau\protect\index{Sachverzeichnis}{vaisseau} pour Archelaus\protect\index{Namensregister}{\textso{Archelaus,} Feldherr im Dienste Mithridates VI.} General de Mithridate\protect\index{Namensregister}{\textso{Mithridates VI.}, K\"{o}nig von Pontos 132\textendash 63 v. Chr.}, charg\'{e} \edtext{de cette espece d'alun (aluin)}{\lemma{charg\'{e}}\Afootnote{ \textit{ (1) }\ d'alun \textit{ (2) }\ d'une certaine \textit{ (3) }\ de cette espece d'alun (aluin) \textit{ L}}} qui deuuoit servir, \`{a} garantir les vaisseaux\protect\index{Sachverzeichnis}{vaisseau} du feu. 
\pend 
\pstart \textit{Het} Eik \textit{hout\protect\index{Sachverzeichnis}{eikehout} spant de Kroon tot te sheep bow\protect\index{Sachverzeichnis}{scheepsbouw} boven alle} \edtext{\textit{Bomen}}{\lemma{\textit{alle}}\Afootnote{ \textit{ (1) }\ Boomen \textit{ (2) }\ \textit{Bomen} \textit{ L}}}, \textit{wan 't is taei, buight wel, is sterk en niet te} \edtext{\textit{zwaer}}{\lemma{\textit{te}}\Afootnote{ \textit{ (1) }\ swaer \textit{ (2) }\ \textit{zwaer} \textit{ L}}}. Oulincks \textit{men hier touwen sloeg uit de bast van Eiken boom\protect\index{Sachverzeichnis}{eik}, 't gen heden niet geschiet, om dat ons hennip\protect\index{Sachverzeichnis}{hennep} in overvloet uit verre landen toe wert} gebracht. \textit{Hoe minder quasten in 't hout\protect\index{Sachverzeichnis}{hout} zyn en hoe langer van draet het is, hoe nutter tot den bow dient. Het hout\protect\index{Sachverzeichnis}{hout} daer het} \edtext{\textit{meeste hars}}{\lemma{\textit{meeste}}\Afootnote{ \textit{ (1) }\ haers \textit{ (2) }\ \textit{hars} \textit{ L}}}, \textit{gom, of tarpentijn\protect\index{Sachverzeichnis}{terpentijn} in is alderbest het water weert. Hier in overtreft het green\protect\index{Sachverzeichnis}{grenehout} en vuuren hout}\protect\index{Sachverzeichnis}{vurehout}\edtext{}{\lemma{}\Afootnote{\textit{hout} \textbar\ het \textit{ gestr.}\ \textbar\ \textit{den} \textit{ L}}} \textit{den Eik\protect\index{Sachverzeichnis}{eik}. Het bruin zijn van het hout\protect\index{Sachverzeichnis}{hout} is een teeken dat het nat en vochtig}\footnote{\textit{\"{U}ber} vochtig: humidum.} \textit{is, waerom men de geele verw kiest. Het pit} (+ an medulla ? +) \textit{en binnenst van de boom geeft het beste hout\protect\index{Sachverzeichnis}{hout}, hierom men alzulcke bomen kiest, 't welcke het breetst zijn van pit}. Vitruvius\protect\index{Namensregister}{\textso{Vitruvius Pollio,} Marcus ca. 70\textendash 10 v. Chr.} parle de Larix\protect\index{Sachverzeichnis}{larix} sur la mer hadriatique\protect\index{Ortsregister}{Adria (mer hadriatique)} qui resistoit au feu. Theophrastus\protect\index{Namensregister}{\textso{Theophrastus}, ca. 371\textendash 287 v. Chr.} dit que les portes du temple de Diane\protect\index{Namensregister}{\textso{Diana} (Diane), r\"{o}m. G\"{o}ttin} estoit du Bois de Cypresse\protect\index{Sachverzeichnis}{bois!de cypr\`{e}s} et avoit dur\'{e} par [l'age]\edtext{}{\Afootnote{l'ange\textit{\ L \"{a}ndert Hrsg.}}} de 4 hommes adjoutez Pline\protect\index{Namensregister}{\textso{Plinius Secundus Maior,} Gaius 23?\textendash 79}. \textit{Iersch hout\protect\index{Sachverzeichnis}{Iersch hout} is hart, wederstaet worm en alle ongedierten}\protect\index{Sachverzeichnis}{ongedierte}, \textit{taei} (+ puto z\"{a}he +) \textit{als Leder, waerom tot ten scheepsbow\protect\index{Sachverzeichnis}{scheepsbouw} zeer bequaem. Aen folders} (+ puto soliv\'{e}s, planches +) \textit{die van dit hout\protect\index{Sachverzeichnis}{hout} gemaekt zyn vint men nimmer spin of eenig vergiftig ongedierte}\protect\index{Sachverzeichnis}{ongedierte}. \textit{De} \textso{\textit{echte Pijnboom}}\protect\index{Sachverzeichnis}{pijnboom} \textit{is van} te oude to te \textit{scheep bouw\protect\index{Sachverzeichnis}{scheepsbouw} zeer gepresen}. Navis\protect\index{Sachverzeichnis}{navis} \edtext{poetis}{\lemma{Navis}\Afootnote{ \textit{ (1) }\ ipsis \textit{ (2) }\ poetis \textit{ L}}} pinus\edlabel{certainend}. \pend 
\pstart Olmbaum-\textit{hout}\protect\index{Sachverzeichnis}{olm hout}\edtext{\edlabel{olmbaumstart}}{{\xxref{olmbaumstart}{olmbaumend}}\lemma{Olmbaum}\Bfootnote{Von Olmbaum-hout bis gebruyckt vgl. \textsc{N. Witsen}, \cite{00153}a.a.O., S.~182.}} \textit{lijdt qualijk spijkers. Het \textso{Lindenhout}\protect\index{Sachverzeichnis}{lindehout} wort geprezen, om dat het scheuren en barsten weinig onderworpen is}.
\pend 
\pstart \textit{De} \textso{\textit{Bastaerd-Pijn}}\protect\index{Sachverzeichnis}{Bastaerd-Pijn} i\textit{s bros, en de vergangelijkheit zeer onderworpen, ten ware} tat bepeckt \textit{wort wanneer zy lange duert. Dennen-hout}\protect\index{Sachverzeichnis}{dennehout} \textit{koos men tot rees en om sprieten van te maeken; 't is buygsam} (+ flexile +) \textit{maer verrot in Zout}\footnote{\textit{\"{U}ber} Zout: salß.} \textit{water}\protect\index{Sachverzeichnis}{zoutwater} light \textit{waerom dat beter op zoete}\protect\index{Sachverzeichnis}{zoete water}\footnote{\textit{\"{U}ber} zoete: suße.} \textit{vlieten} gebruyckt \textit{mag werden als in zee. Het wiert te scheep}\protect\index{Sachverzeichnis}{schip} \textit{boven} \edtext{wasser}{\lemma{\textit{boven}}\Afootnote{ \textit{ (1) }\ water \textit{ (2) }\ wasser \textit{ L}}} gebruickt, \textit{daer het om} zyn lichtigkeit \textit{bequaem} to \textit{was. De bomen werden by de zommige onterscheiden in de mannelijke en vrowelijke kunne, en by hen werd de manlijke voor de sterckste en vastste gehouden}. \pend 
\pstart \textso{\textit{Waeterboomen}}\protect\index{Sachverzeichnis}{Waeterboomen} \textit{hielden de ouden taei} (+ z\"{a}he +) \textit{hout}\protect\index{Sachverzeichnis}{hout} \textit{te} zyn \textit{en} darom \textit{tot schilden et} diergeliike \textit{zeer bequaem}. 
\pend 
\pstart \textit{De} \textso{\textit{Zwarte doorn}}\protect\index{Sachverzeichnis}{zwarte doorn} \textit{wiert tot inhouten om zyn onbederfzaemheit gebruyckt}\edlabel{olmbaumend}. 
\pend 
\pstart \edtext{\textit{Het}\edlabel{hetschipstart}}{{\xxref{hetschipstart}{hetschipend}}\lemma{\textit{Het}}\Bfootnote{Von Het schip bis 't is mit vgl. \textsc{N. Witsen}, \cite{00153}a.a.O., S.~183.}} \textit{schip}\protect\index{Sachverzeichnis}{schip} \textit{'t geen onlanks ten gronde uyt is gehaelt en het Nemorenser meir, 't welk Trajanus}\protect\index{Namensregister}{\textso{Trajanus,} Marcus Ulpius, Kaiser in Rom 98\textendash 117} \textit{ash bewaerde, bewijst klaerlijck de langdoursamkeit van de pijn}\protect\index{Sachverzeichnis}{pijn} \textit{en Cypressen boom}\protect\index{Sachverzeichnis}{cipres}, \textit{waer van det vaertuig was gebowt. Diet dient voor en algemeene wet angenomen te zyn, dat men} to \edtext{te}{\lemma{to}\Afootnote{ \textit{ (1) }\ se \textit{ (2) }\ te \textit{ L}}} \textit{schipsbow}\protect\index{Sachverzeichnis}{scheepsbouw} \textit{geen hout}\protect\index{Sachverzeichnis}{hout} \textit{kiest 't geen het aldergroodste is, want groote} \edtext{\textit{bomen}}{\lemma{\textit{groote}}\Afootnote{ \textit{ (1) }\ boomen \textit{ (2) }\ \textit{bomen} \textit{ L}}}, \textit{zyn oude bomen en oude bomen zijn als oude menschen} \edtext{\textit{krachteloos en}}{\lemma{\textit{menschen}}\Afootnote{ \textit{ (1) }\ krachtegoos \textit{ (2) }\ \textit{krachteloos en} \textit{ L}}} \textit{bros}. 
\pend 
\pstart \textit{Kircher}\protect\index{Namensregister}{\textso{Kircher} (Kircherus), Athanasius SJ 1602\textendash 1680} \textit{getuigt een boom-schoors gezien te hebben, daer een geheele Kudde}-\edtext{shappen}{\lemma{\textit{Kudde}-}\Afootnote{ \textit{ (1) }\ shapen \textit{ (2) }\ shappen \textit{ L}}} \textit{in verschool}.
\pend 
\pstart \textso{\textit{Pock}}\protect\index{Sachverzeichnis}{pokhout}\textso{ \textit{en Notebomen hout}}\protect\index{Sachverzeichnis}{notehout} \textit{dienstig} \edtext{\textit{tot}}{\lemma{\textit{dienstig}}\Afootnote{ \textit{ (1) }\ te \textit{ (2) }\ \textit{tot} \textit{ L}}} schiiven \textit{is, wan 't is hart en sterck. Linden}\protect\index{Sachverzeichnis}{linde} \textit{en Abelen}\protect\index{Sachverzeichnis}{abeel} \textit{tot pompen en andere} buisen zyn \textit{bequaem, om dat van binnen week en van buyten hart is. Mispel hout}\protect\index{Sachverzeichnis}{mispelhout} \textit{is} goed \textit{als} \edtext{\textit{'t}}{\lemma{}\Afootnote{'t \textit{ erg.} \textit{ L}}} \textit{in droog staet, dog} \edtext{\textit{wanneer}}{\lemma{\textit{dog}}\Afootnote{ \textit{ (1) }\ waneer \textit{ (2) }\ \textit{wanneer} \textit{ L}}} \textit{by water komt, staet het deur en drinckt water in. Els}\protect\index{Sachverzeichnis}{els} \textit{aen het water gewossen}, word \textit{van Vitruvius}\protect\index{Namensregister}{\textso{Vitruvius Pollio,} Marcus ca. 70\textendash 10 v. Chr.} \textit{zeer gepreesen en bequaem geoordeelt om in 't water te staen. De Zirnenboom}\protect\index{Sachverzeichnis}{Zirnenboom} \textit{verrot haestig, en daerom niet goet }\edtext{\textit{tot de}}{\lemma{\textit{tot}}\Afootnote{ \textit{ (1) }\ the \textit{ (2) }\ \textit{de} \textit{ L}}} \textit{scheep bow}\protect\index{Sachverzeichnis}{scheepsbouw}.
\pend 
\pstart \textit{Van de mastbomen}\protect\index{Sachverzeichnis}{mastboom} \textit{wierden by de ouden} te \textit{jocken tot de} ossen \edtext{gemaekt}{\lemma{ossen}\Afootnote{ \textit{ (1) }\ gemackt \textit{ (2) }\ gemaekt \textit{ L}}} \textit{om haere} stevigkeit. \textit{Hout}\protect\index{Sachverzeichnis}{hout} tat \textit{tusschen water en wint heft gelegen is de verrotting zeer onderworpen. De deugt} (dugend) \textit{van Masten bestaet en} \edtext{\textit{haer}}{\lemma{\textit{en}}\Afootnote{ \textit{ (1) }\ haere \textit{ (2) }\ \textit{haer} \textit{ L}}} \textit{dickte}, ronde \textit{en} langde, ook \textit{dat} sonder \textit{quasten en langdradig} \edtext{\textit{zijn}}{\lemma{\textit{langdradig}}\Afootnote{ \textit{ (1) }\ sijn \textit{ (2) }\ \textit{zijn} \textit{ L}}}. \textit{Nimmer moet een boom gehackt werden als hy} vrught \edtext{\textit{draegt}}{\lemma{vrught}\Afootnote{ \textit{ (1) }\ draght \textit{ (2) }\ \textit{draegt} \textit{ L}}}. \textit{'t is mit}\edlabel{hetschipend} \edtext{boomen\edlabel{boomenstart}}{{\xxref{boomenstart}{boomenend}}\lemma{boomen}\Bfootnote{Von boomen bis zal schieten vgl. \textsc{N. Witsen}, \cite{00153}a.a.O., S.~184.}} gilijck \textit{mit} vrowen, \textit{zwack als ze dragen}. Nemt \textit{acht} of \textit{the Mane}\protect\index{Sachverzeichnis}{maan} (Mond\protect\index{Sachverzeichnis}{Mond}) \textit{velt de} \edtext{\textit{bomen}}{\lemma{\textit{de}}\Afootnote{ \textit{ (1) }\ boomen \textit{ (2) }\ \textit{bomen} \textit{ L}}} \textit{in haer} afzyn (secate, velt f\"{a}llet) \textit{want zy gelooft-men vermeerdert de} vogtheit \textit{in de} selve. \textit{En laet mede het hout}\protect\index{Sachverzeichnis}{hout} \textit{niet al te zeer drogen op dat het niet berste en vermorzele}.
\pend 
