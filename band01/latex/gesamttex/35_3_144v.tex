\pstart[144 v\textsuperscript{o}] Mais l'effort de toute l'atmosphere\protect\index{Sachverzeichnis}{atmosph\`{e}re} contient non seulement ce poids\protect\index{Sachverzeichnis}{poids}, mais aussi tout son ressort. Et il est bon de remarquer icy, que la pression du poids\protect\index{Sachverzeichnis}{poids} de l'atmosphere\protect\index{Sachverzeichnis}{atmosph\`{e}re} n'est que de la colomne superincumbante. Mais une bulle d'air \edtext{libre}{\lemma{}\Afootnote{libre \textit{ erg.} \textit{ L}}} chez nous \edtext{souffre le ressort de l'ocean de tout l'Air libre}{\lemma{nous}\Afootnote{ \textit{ (1) }\ ne souffre le poids\protect\index{Sachverzeichnis}{poids|textit} de tout l'air libre to \textit{ (2) }\ souffre [...] libre \textit{ L}}} du monde. Car supposons qu'une Bulle d'air chez les Antipodes\protect\index{Sachverzeichnis}{antipodes} vienne \`{a} estre annihil\'{e}e, il est manifeste que tout l'air libre du monde en souffriroit quelque changement, et que la bulle donn\'{e}e chez nous seroit aussi de la partie, et se dilateroit \`{a} proportion, profitant de l'heritage de fe\"{u}e sa soeur. Or supposons que la Bulle annihil\'{e}e soit recre\'{e}e, il est manifeste que la nostre y perdroit autant qu'elle avoit gagn\'{e}, se trouuant oblig\'{e}e de rentrer dans sa coquille. Donc, si tout demeure dans le même estat, sans annihilation ny creation, il est ais\'{e} \`{a} juger, que la Bulle donn\'{e}e, chez nous, sera contrainte\edtext{}{\lemma{}\Afootnote{contrainte  \textbar\ et press\'{e}e \textit{ gestr.}\ \textbar\ en \textit{ L}}} en quelque facon par l'assign\'{e}e chez les antipodes, et par consequent chaque bulle assignable \edtext{en son particulier}{\lemma{}\Afootnote{en son particulier \textit{ erg.} \textit{ L}}} par toutes les autres ensemble. \edtext{Maintenant, pour}{\lemma{ensemble.}\Afootnote{ \textit{ (1) }\ Pour \textit{ (2) }\ Maintenant, pour \textit{ L}}} comprendre plus clairement une verit\'{e} assez importante, feignons que la bulle donn\'{e}e soit priv\'{e}e de son Ressort par la tout-puissance de Dieu retenant neantmoins la compressibilit\'{e}, mais sans reaction ny resistence. Cela fait, elle sera reduite subitement \`{a} un rien, \edtext{pour ainsi dire, ou plustost,}{\lemma{rien,}\Afootnote{ \textit{ (1) }\ ou \textit{ (2) }\ pour ainsi dire, ou plustost, \textit{ L}}} dans un point par \edtext{le ressort de toute}{\lemma{par}\Afootnote{ \textit{ (1) }\ l'effort de tout \textit{ (2) }\ le ressort de toute \textit{ L}}} la Masse ambiente, laquelle tâchant tousjours de se dilater, et trouuant une place sans resistence, s'en saisira avec \edtext{la derniere vîtesse}{\lemma{avec}\Afootnote{ \textit{ (1) }\ une vîtesse\protect\index{Sachverzeichnis}{vitesse|textit} incroyable \textit{ (2) }\ la derniere vîtesse \textit{ L}}}, partageant le gain entre ses parties, avec une justice inimitable. Or feignons \`{a} present que Dieu rende \`{a} nostre Bulle ses premieres forces, alors tout reviendra au premier estat, et nostre Bulle reprennant sa place, en chassera tout ce qu'elle y trouue, malgr\'{e} l'effort de toute la Masse, laquelle en souffrira quelque contrainte en toutes ses parties. C'est donc une v\'{e}rit\'{e} incontestable, que la bulle donn\'{e}e soûtient l'effort du Ressort de toute l'atmosphere\protect\index{Sachverzeichnis}{atmosph\`{e}re}, puisque toute l'atmosphere\protect\index{Sachverzeichnis}{atmosph\`{e}re} profiteroit incontinent de son absence, ou foiblesse, estant \`{a} present empech\'{e}e par le Ressort de la bulle. J'ay dit que cela se doit entendre de la bulle qui est dans l'air libre. C'est \`{a} dire qui communique avec l'ocean general de l'air, comme le pont Euxin\protect\index{Ortsregister}{Schwarzes Meer (pont Euxin)} avec l'ocean d'eau,  et qui ne soit pas renferm\'{e}e comme la Mer Caspie\protect\index{Ortsregister}{Kaspisches Meer (Mer Caspie)}.  Mais si la bulle donn\'{e}e \edtext{ou quelqu'autre petite portion d'air}{\lemma{}\Afootnote{ou quelqu'autre petite  \textit{ (1) }\ masse \textit{ (2) }\ portion d'air \textit{ erg.} \textit{ L}}} est renferm\'{e}e dans