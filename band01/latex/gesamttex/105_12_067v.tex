[67 v\textsuperscript{o}] \textso{puncti }\textso{Telluris}\protect\index{Sachverzeichnis}{tellus}\textso{ }\textso{latitudinis}\protect\index{Sachverzeichnis}{latitudo}\textso{ datae, dabitur ergo puncti }\textso{longitudo}\protect\index{Sachverzeichnis}{longitudo}\textso{; ac proinde, data jam ante }\textso{latitudine}\protect\index{Sachverzeichnis}{latitudo}\textso{, dabitur ipse puncti locus praecise}.  Hoc fieri potest tum per instrumentum \edtext{\textso{Mechanice}}{\lemma{instrumentum}\Afootnote{ \textit{ (1) }\ Mechanice \textit{ (2) }\ \textso{Mechanice} \textit{ L}}}, tum per calculum \textso{Geometrice.}\pend \pstart \edtext{\textso{Instrumentum}}{\lemma{?LEMMA?:Geometrice.}\Afootnote{ \textit{ (1) }\ Instrumentum \textit{ (2) }\ \textso{Instrumentum} \textit{ L}}}  nullum aptius reperio ipsa \edtext{\textso{sphaera artificiali, novo quodam more}}{\lemma{ipsa}\Afootnote{ \textit{ (1) }\ sphaera artificiali, novo quodam more \textit{ (2) }\ \textso{sphaera artificiali, novo quodam more} \textit{ L}}}, nunc explicando recte \edtext{\textso{instructa}}{\lemma{recte}\Afootnote{ \textit{ (1) }\ instructa  \textit{ (2) }\ \textso{instructa} \textit{ L}}},  satisque ampla ad accuratas circulorum subdivisiones habendas. In hac sphaera elevetur Polus\protect\index{Sachverzeichnis}{polus} supra horizontem immobilem artificialem quemadmodum postulat latitudo\protect\index{Sachverzeichnis}{latitudo} cognita puncti in quo est navis\protect\index{Sachverzeichnis}{navis}; quo facto globus ita circumagatur ut sidus\protect\index{Sachverzeichnis}{sidus} observatum tot gradibus minutisve distet a meridiano\protect\index{Sachverzeichnis}{meridianus}, horizonteque artificiali loci navis\protect\index{Sachverzeichnis}{navis}, quot monstrat  @ @ @ Kein indexComment fuer index gefunden @ @ @ sideris ejusdem elevatio super horizontem loci navis\protect\index{Sachverzeichnis}{navis} naturalem paulo ante observata\edtext{. Et quia distantia puncti cujusdam a linea quadam semper tanta est, quanta est perpendicularis a puncto ad lineam ducta. Ideo circumactio debita fiet}{\lemma{observata}\Afootnote{ \textit{ (1) }\ , quod fiet \textit{ (2) }\ , et quidem versus eandem \textso{plagam} \textit{ (3) }\ . [...] fiet \textit{ L}}} applicato ad sphaeram, ex horizonte, arcu quodam 90. grad. circuli ipsi sphaerae artificiali congruentis, qua et vulgo in sphaeris artificialibus utimur (supplendorum circulorum magnorum alioquin in sphaera ubique ducendorum, sumendarumque inter duo sphaerae puncta distantiarum causa) eoque arcu ita applicato \edtext{\textso{ut ad horizontem angulum faciat rectum}}{\lemma{applicato}\Afootnote{ \textit{ (1) }\ ut ad horizontem angulum faciat rectum \textit{ (2) }\ \textso{ut [...] rectum} \textit{ L}}}, seu ut portionem  \textbar\ seu \textso{Quadrantem} \textit{ erg.}\ \textbar\   repraesentet cujusdam verticalis. Quo facto sphaera \edtext{circa axem suum}{\lemma{sphaera}\Afootnote{ \textit{ (1) }\ super polum\protect\index{Sachverzeichnis}{polus|textit} \textit{ (2) }\ circa axem suum \textit{ L}}} eousque circumagatur, et arcus iste super horizonte eousque retento situ orthogonali moveatur, seu varie disponatur, donec sidus\protect\index{Sachverzeichnis}{sidus} observatum praecise incidat in eum numerum graduum minutorumque in hoc arcu inde ab horizonte computatorum, quem observata sideris\protect\index{Sachverzeichnis}{sidus} elevatio super horizontem loci dedit. Quae circumactio seu sideris arcusque\protect\index{Sachverzeichnis}{arcus} dicti applicatio nullo \edtext{negotio ad primos statim obtutus}{\lemma{nullo}\Afootnote{ \textit{ (1) }\ statim negotio ad primos obtutus \textit{ (2) }\ negotio ad primos statim obtutus \textit{ L}}} fiet: dummodo opera detur, ut situs arcus dicti ad horizontem sit orthogonalis, quod fieri secure poterit hac industria si crenae  \textbar\ cuidam \textit{ erg.}\ \textbar\  ipsi horizontis margini impressae quadrare possit, ut in ea constanter; dum nobis rursus eximere lubeat, circumagatur.\pend \pstart Habemus ergo primum mobile artificiale ita constitutum in sphaera artificiali in respectu ad horizontem ejus, meridianumque\protect\index{Sachverzeichnis}{meridianus}, qui meridianum\protect\index{Sachverzeichnis}{meridianus} loci repraesentat, uti primum mobile verum constitutum est in respectu ad horizontem, meridianumque\protect\index{Sachverzeichnis}{meridianus} loci navis\protect\index{Sachverzeichnis}{navis}. Sed quia de loco navis\protect\index{Sachverzeichnis}{navis} nihil nisi latitudo\protect\index{Sachverzeichnis}{latitudo} nobis explorata est, seu distantia a polo\protect\index{Sachverzeichnis}{polus} immobili, non vero distantia a meridiano\protect\index{Sachverzeichnis}{meridianus} primo in terra immobili supposito. Ideo coelum quidem sed nondum terram in globo artificiali\protect\index{Sachverzeichnis}{globus!artificialis} recte, et ad loci navis\protect\index{Sachverzeichnis}{navis} normam, disposita, habemus.\pend \pstart Ut ergo inveniatur etiam longitudo\protect\index{Sachverzeichnis}{longitudo}, adhibendum est quiddam repraesentans terram, sed quia globus artificialis\protect\index{Sachverzeichnis}{globus!artificialis} non potest commode fieri perspicuus, nec proinde terra in ejus centro locari, sufficiet unum terrae circulum, nimirum meridianum\protect\index{Sachverzeichnis}{meridianus} primum adhiberi, utcunque eum assumere lubuerit, dummodo constet per quam terrae regionem transeat, etsi utile foret unum eundemque constanter ab 