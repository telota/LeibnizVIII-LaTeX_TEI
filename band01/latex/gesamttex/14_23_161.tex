\pend \pstart [p.~161] III. Hac arte vti solent, ad colligendas eiusdem imaginis partes in tabella quapiam dispersas; si nempe\footnote{\textit{Leibniz unterstreicht}: Hac arte [...] si nempe} in orbe in totidem facies diuiso, quampiam imaginem descripseris notatisque diligenter faciebus polyoptri, in  planum oppositum proiectis, adhibitis etiam ad maiorem distinctionem, singularum numeris, in singulas facies plani proiectionis easdem imaginis partes traduxeris, quae in analogis faciebus tabellae, id est, eodem numero notatis, depictae fuerant, statuto in loco lucernae  oculo, omnes illas imaginis partes colliges, et imaginem  aeque videbis, atque si tabellam ipsam, in qua depicta  est, aspiceres. IV. Haec praxis, quae aliquid admirationis ante conciliabat, iam trita, et vulgaris est\footnote{\textit{Leibniz unterstreicht}: Haec praxis [...] vulgaris est}: habentur autem facierum proiectarum sedes, vel adhibita lucerna, vt dixi, vel statuto loco lucernae oculo\protect\index{Sachverzeichnis}{oculus}, virgula enim, cuius  extremitas diligenter obseruabitur, in plano opposito dictae facies designabuntur; [...] idem fiet opera fili et chartae interpositae; sed haec sunt facillima.\footnote{\textit{Leibniz unterstreicht}: idem fiet [...] facillima.}