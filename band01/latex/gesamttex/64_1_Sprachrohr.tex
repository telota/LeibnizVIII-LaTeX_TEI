      
               
                \begin{ledgroupsized}[r]{120mm}
                \footnotesize 
                \pstart                
                \noindent\textbf{\"{U}berlieferung:}   
                \pend
                \end{ledgroupsized}
            
              
                            \begin{ledgroupsized}[r]{114mm}
                            \footnotesize 
                            \pstart \parindent -6mm
                            \makebox[6mm][l]{\textit{E}}\cite{00243}\textsc{Gerland} 1906, S.~10. Eine Handschrift zu dieser Aufzeichnung ist in den Katalogen nicht verzeichnet. Es ist davon auszugehen, dass diese nach der Benutzung durch \cite{00243}Gerland verloren gegangen ist, so dass wir das St\"{u}ck nach dessen Edition drucken. \pend
                            \end{ledgroupsized}
                %\normalsize
                \vspace*{5mm}
                \begin{ledgroup}
                \footnotesize 
                \pstart
            \noindent\footnotesize{\textbf{Datierungsgr\"{u}nde}: Von Leibniz datiert.}
                \pend
                \end{ledgroup}
            
                \vspace*{8mm}
                \pstart 
                \normalsize
            [S. 10] \selectlanguage{german}1) \"{U}ber Moreland's\protect\index{Namensregister}{\textso{Morland,} Samuel 1625\textendash 1695} Sprachrohr\protect\index{Sachverzeichnis}{Sprachrohr}. Ein krummes leistet dieselben Dienste, wie ein gerades. Ce seroit une chose curieuse, si on le pouuoit cacher sous la perruque.\pend \pstart  2) Es ist anjezo ein Mann in England\protect\index{Ortsregister}{England (Anglia)}, der in ein gl\"{a}sern Instrument eigener Applikation redet und zwar leise ziemlich, wie man auch in einer Trompete nicht so stark bl\"{a}set, daß man durch den ganzen Parck oder Garten h\"{o}hret und zwar deutlich. Imgleichen wenn ers vors ohre h\"{a}lt, so h\"{o}rt er alles hahrkleine si hoc verum, potest magis augmentari. Hoc mihi dixit Dr. v. Helmont\protect\index{Namensregister}{\textso{Helmont,} Franciscus Mercurius van 1614\textendash 1699} anno 1671.\selectlanguage{latin}\pend 