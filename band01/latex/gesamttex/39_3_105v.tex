\pstart  Haec vero suspicio mea, argumento  postea ex ipsa rei natura ducto valde  confirmata est; quod hoc loco exponere,
[105~v\textsuperscript{o}]  et ad aliquam mali emendationem viam  mihi aperire operae fortasse pretium fuerit.
\pend 
\pstart  Constat in omni liquore, duplicem  considerari posse Gravitatem\protect\index{Sachverzeichnis}{gravitas!individualis}, individualem,  ut sic dicam et specificam; cum enim gravitas\protect\index{Sachverzeichnis}{gravitas} sit quantitas \edtext{conatus innati corporum,}{\lemma{quantitas}\Afootnote{ \textit{ (1) }\ absoluta, a machinis conatus\protect\index{Sachverzeichnis}{conatus|textit} \textit{ (2) }\ conatus innati corporum, \textit{ L}}}  ad centrum terrae, et vero omnis quantitas  aestimetur comparatione, et corporum comparari  possint vel individua vel species, ideo duplex gravitas individualis\protect\index{Sachverzeichnis}{gravitas!individualis} et specifica\protect\index{Sachverzeichnis}{gravitas!specifica} considerata est, \edtext{individualis aestimatur simpliciter dati corporis pondere}{\lemma{est,}\Afootnote{ \textit{ (1) }\ illa scilicet simpliciter pondus \textit{ (2) }\ individualis aestimatur simpliciter   \textbar\ dati \textit{ erg.}\ \textbar\  corporis pondere \textit{ L}}} absolute considerato;  at specifica pondere corporis in data mole  seu spatio; quo scilicet diversae species  discerni possunt. Nam si quis dicat simpliciter  corpus aliquod sibi fuisse ponderis tot librarum, nemo  speciem ejus definient, at si pyxidem sibi  fuisse dicat plenam, non  ponderis tantum sed et magnitudinis datae,  potest \edtext{aliquando ad speciem}{\lemma{potest}\Afootnote{ \textit{ (1) }\ non raro species \textit{ (2) }\ aliquando ad speciem \textit{ L}}} corporis ratiocinando perveniri.
\pend
 \pstart  Hanc distinctionem ab omnibus cognitam, ad  aerem non satis applicatam miror, ne ab iis  quidem qui in hoc negotio \edtext{maxime  laboravere.}{\lemma{negotio}\Afootnote{ \textit{ (1) }\ enuntiando \textit{ (2) }\ maxime  laboravere. \textit{ L}}} \edtext{Baroscopium}{\lemma{laboravere.}\Afootnote{ \textit{ (1) }\ Barometrum\protect\index{Sachverzeichnis}{barometrum|textit} \textit{ (2) }\ Baroscopium \textit{ L}}} enim  Torricellianum vere ac proprie non  nisi individualem\protect\index{Sachverzeichnis}{gravitas!individualis} aeris \edtext{gravitatem\protect\index{Sachverzeichnis}{gravitas!aeris}, definire}{\lemma{}\Afootnote{gravitatem,  \textbar\ id est quantum ponderaret totus aeris cylinder Mercurio\protect\index{Sachverzeichnis}{mercurius} obnitens, \textit{ gestr.}\ \textbar\ definire \textit{ L}}} potuit, specificam non nisi per accidens,  quatenus aliquando complicatas habet cum individuali rationes. Unde factum arbitror, ut Baroscopium\protect\index{Sachverzeichnis}{baroscopium!Torricellianum} \edtext{nonnunquam praeclare cum aere consenserit, nonnunquam}{\lemma{Baroscopium}\Afootnote{ \textit{ (1) }\ aliquando praeclare cum aere consenserit, aliquando \textit{ (2) }\ nonnunquam [...] nonnunquam \textit{ L}}} plane fefellerit fidem, \edtext{quoniam}{\lemma{fidem,}\Afootnote{ \textit{ (1) }\ quando \textit{ (2) }\ quoniam \textit{ L}}} tempestates proprie ac per se  specificae non vero nisi per consequentias  varie obliquatas individuali aeris gravitati\protect\index{Sachverzeichnis}{gravitas!aeris}  connectuntur.
 \pend
  \pstart  Utrumque ostendendum est  (1) Baroscopium Torricellianum\protect\index{Sachverzeichnis}{baroscopium!Torricellianum}, et quaecunque sunt  ejus transformationes individualem  proprie ac per se, specificam non nisi  per accidens contingenterque monstrare (2)  Tempestates \edtext{cum gravitate aeris specifica proprie ac per se, eam}{\lemma{Tempestates}\Afootnote{ \textit{ (1) }\ aeris cum \textit{ (2) }\ cum [...] eam \textit{ L}}}  individuali non nisi per accidens contingenterque  connecti.
  \pend 
  \pstart  I\textsuperscript{mum} ergo Baroscopium Torricellianum\protect\index{Sachverzeichnis}{baroscopium!Torricellianum}  non nisi individualem\protect\index{Sachverzeichnis}{gravitas!individualis} aeris gravitatem\protect\index{Sachverzeichnis}{gravitas!aeris} \edtext{proprie}{\lemma{proprie}\Afootnote{ \textit{ erg.} \textit{ L}}} ostendere,  demonstratu facile est. \edtext{Ostendit enim totius Cylindri}{\lemma{est.}\Afootnote{ \textit{ (1) }\ Monstrat enim  cylin \textit{ (2) }\ Ostendit enim totius Cylindri \textit{ L}}} aerei Mercurium\protect\index{Sachverzeichnis}{mercurius} sustinentis pondus; at vero hinc specifica\protect\index{Sachverzeichnis}{gravitas!specifica} aeris gravitas\protect\index{Sachverzeichnis}{gravitas!aeris} non definitur, quia ea ponderatio  non determinat pondus data mole seu spatio \edtext{a materia impleto;}{\lemma{spatio}\Afootnote{ \textit{ (1) }\ quod implet \textit{ (2) }\ a materia impleto; \textit{ L}}} sed pondus simpliciter massae\protect\index{Sachverzeichnis}{massa} cujusdam \edtext{\textso{incertae altitudinis}}{\lemma{}\Afootnote{\textso{incertae altitudinis} \textit{ erg.} \textit{ L}}}  dato cuidam loco imminentis. Nam si ponatur  aer duplo quam est rarior, seu specifice levior,  et tamen duplo quoque quam est, altior; \edtext{idem nihilominus manebit Cylindri Aerei pondus, idemque qui nunc, apparebit Tubi Torricelliani status.}{\lemma{idem}\Afootnote{ \textit{ (1) }\ qui nunc apparebit cylindri Mercurialis status \textit{ (2) }\ nihilominus [...] status. \textit{ L}}} 
  \edtext{Unde [106~r\textsuperscript{o}] frustra fuere docti quidam Viri qui certas de  Aereae massae altitudine demonstrationes}
{\lemma{Unde}\Afootnote{ \textit{ (1) }\ apparet doctorum quorundam Virorum demonstrationes de Atmosphaerae\protect\index{Sachverzeichnis}{atmosphaera|textit} aereae altitudine ex Baroscopio\protect\index{Sachverzeichnis}{baroscopium!Torricellianum|textit} \textit{ (2) }\ frustra [...] de \textit{(a)}\ Aeris \textit{(b)}\  Aereae massae altitudine demonstrationes \textit{ L}}} condere  conati sunt, supposita scilicet aeris  homogeneitate, qua tamen nihil incertius, imo  nihil a verisimilitudine alienius, cum aerem in  celsiore loco dilatare sese in quantam per minorem  incumbentis pressionem, potest \edtext{raritatem et fortasse in altitudinem quandam indicibilem expandi, institutis in aere nostro experimentis  judicari possit.}{\lemma{raritatem}\Afootnote{ \textit{ (1) }\ non sit  ex expe \textit{ (2) }\ et [...] possit. \textit{ L}}}
\pend 