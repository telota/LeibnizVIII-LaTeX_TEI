[7 v\textsuperscript{o}] \ \ \ \ \ sunt amplitudines. Illud tamen considerandum est quoque partem liquoris\protect\index{Sachverzeichnis}{liquor} aliquam \ non inter angustias ferri sed inde ob transitus difficultatem \ vel repercuti, vel diverti, moveri linea \textit{MBN}. Hinc \edtext{\ jam videtur sequi}{\lemma{Hinc}\Afootnote{ \textit{ (1) }\ sequi \textit{ (2) }\ \ jam videtur sequi \textit{ L}}} vim restituendi\protect\index{Sachverzeichnis}{vis!restituens} omnia ad uniformitatem non \ esse in ratione spatiorum \textit{I}, modo majorum modo minorum; \ sed in ratione differentiarum inter motus; quanto citius motus in \textit{I} \ celerior est quam in \textit{L}, eo celerius etiam pellit, quam is qui ei \ obsistit. Quod si majoris facilitatis causa, et ut \edtext{eo calculo subtilissimo \ Geometrico parci possit}{\lemma{ut}\Afootnote{ \textit{ (1) }\ sit calculus subtilissimus \ Geometricus \textit{ (2) }\ eo [...] possit \textit{ L}}}, qui ex circuliaris motus natura oritur \ et in superiore figura imaginemur nobis motum fluminis \ inter duos obices \textit{A}, \textit{B}, celerius moti. @@@ G R A F I K @@@% \begin{wrapfigure}{l}{0.4\textwidth}                    
                %\includegraphics[width=0.4\textwidth]{../images/Calculus+Elasticus/LH035%2C05%2C2_007v/files/100108.png}
                        %\caption{Bildbeschreibung}
                        %\end{wrapfigure}
                        %@ @ @ Dies ist eine Abstandszeile - fuer den Fall, dass mehrere figures hintereinander kommen, ohne dass dazwischen laengerer Text steht. Dies kann zu einer Fahlermeldung fuehren. @ @ @ \\
                     Manifestum est quod major \ erit celeritas\protect\index{Sachverzeichnis}{celeritas} inter \textit{A} et \textit{B}, quam inter \textit{A} et \textit{C}, et inter \textit{B} et \textit{D} \ quia eadem materia quae in obicis \textit{A} latitudinem impingit, aequaliter distri\ buetur in \edtext{intervalla}{\lemma{}\Afootnote{intervalla \textit{ erg.} \textit{ L}}} \textit{AC} et \textit{AD} et pars quae in \textit{B}, distribuetur aequaliter in \ intervalla \textit{BD} et \textit{BC}, jam quod tendit versus \textit{BC}, obsistit ei quod tendit versus \ \textit{AC}, necesse est igitur exitum in medio quaerere inter \textit{AB}. Hinc ergo \ impetus\protect\index{Sachverzeichnis}{impetus} fluminis conabitur utrumque rejicere in litus; et si alterum fixum sit, \ alterum tamen in ripam rejicietur, quae ratio est cur in fluminibus aut generaliter \ aquis fluctuantia paulatim rejiciantur in ripas; nam praesertim quod \ aliqua semper adsunt fixa; etiam illud considerandum est, ipsa fluctuantia \ aliquod habere fixitatis a pondere suo.\pend \pstart  In superiore figura circularis motus \ considerandus est, quod pila\protect\index{Sachverzeichnis}{pila} in medio resistit nonnihil motui \textit{I} etiamsi gravis \ non sit, vel ideo quod inter movendum findit. Sed etsi non finderet (per impossibile) \ motus tamen ejus non fieret in instanti sed in tempore determinato. Moveretur \ enim celeritate\protect\index{Sachverzeichnis}{celeritas} quae ipsi imprimitur, nunc paulo minore.\pend \pstart \ Videndum autem an de Elaterio\protect\index{Sachverzeichnis}{elaterium} ratiocinari liceat, etiam sine ulla hypothesi. \ @@@ G R A F I K @@@ Nimirum ponamus vim oriri a \edtext{magnitudine}{\lemma{a}\Afootnote{ \textit{ (1) }\ quantitate \textit{ (2) }\ parvitate \textit{ (3) }\ magnitudine \textit{ L}}} materiae in spatio parvo, vel \ a \edtext{parvitate spatii}{\lemma{a}\Afootnote{ \textit{ (1) }\ magnitudine \textit{ (2) }\ parvitate spatii \textit{ L}}} pro materia \edtext{magna}{\lemma{materia}\Afootnote{ \textit{ (1) }\ parva \textit{ (2) }\ magna \textit{ L}}}. Erit ergo \ vis in reciproca spatiorum ratione. Ergo vis in \textit{GH}, erit ad vim in \ \textit{(G)(H)} ut est \textit{(G)C} ad \textit{GC}. Ergo @@@ G R A F I K @@@% \begin{wrapfigure}{l}{0.4\textwidth}                    
                %\includegraphics[width=0.4\textwidth]{../images/Calculus+Elasticus/LH035%2C05%2C2_007v/files/100258.png}
                        %\caption{Bildbeschreibung}
                        %\end{wrapfigure}
                        %@ @ @ Dies ist eine Abstandszeile - fuer den Fall, dass mehrere figures hintereinander kommen, ohne dass dazwischen laengerer Text steht. Dies kann zu einer Fahlermeldung fuehren. @ @ @ \\
                     et @@@ G R A F I K @@@% \begin{wrapfigure}{l}{0.4\textwidth}                    
                %\includegraphics[width=0.4\textwidth]{../images/Calculus+Elasticus/LH035%2C05%2C2_007v/files/100260.png}
                        %\caption{Bildbeschreibung}
                        %\end{wrapfigure}
                        %@ @ @ Dies ist eine Abstandszeile - fuer den Fall, dass mehrere figures hintereinander kommen, ohne dass dazwischen laengerer Text steht. Dies kann zu einer Fahlermeldung fuehren. @ @ @ \\
                    , et \textit{(G)C} vel \textit{((G))C} appellando \textit{y}, \ \edtext{\textit{GC} appellando \textit{g}}{\lemma{\}\Afootnote{ \textit{ (1) }\ fiet \textit{ (2) }\ et vim  \textit{(a)}\ \textit{GC} \textit{(b)}\ \textit{HG} appellando \textit{ (3) }\ \textit{GC} appellando \textit{g} \textit{ L}}}, \edtext{et vim \textit{GH} appellando \textit{v}}{\lemma{g,}\Afootnote{ \textit{ (1) }\ fiet, \textit{ (2) }\ et vim \textit{GH} appellando \textit{v} \textit{ L}}}, \edtext{et}{\lemma{v,}\Afootnote{ \textit{ (1) }\ fiet \textit{ (2) }\ et \textit{ L}}} \ vis generaliter erit: @@@ G R A F I K @@@% \begin{wrapfigure}{l}{0.4\textwidth}                    
                %\includegraphics[width=0.4\textwidth]{../images/Calculus+Elasticus/LH035%2C05%2C2_007v/files/100308.png}
                        %\caption{Bildbeschreibung}
                        %\end{wrapfigure}
                        %@ @ @ Dies ist eine Abstandszeile - fuer den Fall, dass mehrere figures hintereinander kommen, ohne dass dazwischen laengerer Text steht. Dies kann zu einer Fahlermeldung fuehren. @ @ @ \\
                    . Vires ergo Elaterii aeris\protect\index{Sachverzeichnis}{elaterium!aeris} in quolibet \ puncto non ut alibi credideram per Hyperbolam cubicam, sed per Hyperbolam communem \ optime designantur. Jam de tempore restitutionis videndum est. Ponamus scilicet restitutionem \ fieri ex \textit{((G))((H))} et embolum\protect\index{Sachverzeichnis}{embolus} rejectum ex \textit{((G))((H))} pervenire in \textit{(G)(H)} tempore aliquo quod appellabimus \ \textit{t} quod infinite parvum ponemus, etiam spatio\protect\index{Sachverzeichnis}{spatium!infinite parvum} posito infinite parvo. \edtext{\ Porro embolus vi prima @@@ G R A F I K @@@% \begin{wrapfigure}{l}{0.4\textwidth}                    
                %\includegraphics[width=0.4\textwidth]{../images/Calculus+Elasticus/LH035%2C05%2C2_007v/files/100361.png}
                        %\caption{Bildbeschreibung}
                        %\end{wrapfigure}
                        %@ @ @ Dies ist eine Abstandszeile - fuer den Fall, dass mehrere figures hintereinander kommen, ohne dass dazwischen laengerer Text steht. Dies kann zu einer Fahlermeldung fuehren. @ @ @ \\
                     pervenit in \textit{(G)H}, at ex \textit{(G)H}}{\lemma{parvo.}\Afootnote{ \textit{ (1) }\ Vis autem prima erit \ @@@ G R A F I K @@@, et vis secunda erit tum prima acquisita, tum praeterea secunda quae est: @@@ G R A F I K @@@. \textit{ (2) }\ \ Porro embolus vi prima @@@ G R A F I K @@@pervenit [...] \textit{(G)H} \textit{ L}}}, tendit altius vi tum quam habet \ ab initio \edtext{nempe @@@ G R A F I K @@@% \begin{wrapfigure}{l}{0.4\textwidth}                    
                %\includegraphics[width=0.4\textwidth]{../images/Calculus+Elasticus/LH035%2C05%2C2_007v/files/100375.png}
                        %\caption{Bildbeschreibung}
                        %\end{wrapfigure}
                        %@ @ @ Dies ist eine Abstandszeile - fuer den Fall, dass mehrere figures hintereinander kommen, ohne dass dazwischen laengerer Text steht. Dies kann zu einer Fahlermeldung fuehren. @ @ @ \\
                     tum quam}{\lemma{initio}\Afootnote{ \textit{ (1) }\ tum quam \textit{ (2) }\ nempe @@@ G R A F I K @@@ tum quam \textit{ L}}} [nactus]\edtext{}{\Afootnote{nactum\textit{\ L \"{a}ndert Hrsg. } }} est \edtext{in itinere}{\lemma{est}\Afootnote{ \textit{ (1) }\ ab initio \textit{ (2) }\ in itinere \textit{ L}}} nempe \ @@@ G R A F I K @@@% \begin{wrapfigure}{l}{0.4\textwidth}                    
                %\includegraphics[width=0.4\textwidth]{../images/Calculus+Elasticus/LH035%2C05%2C2_007v/files/100390.png}
                        %\caption{Bildbeschreibung}
                        %\end{wrapfigure}
                        %@ @ @ Dies ist eine Abstandszeile - fuer den Fall, dass mehrere figures hintereinander kommen, ohne dass dazwischen laengerer Text steht. Dies kann zu einer Fahlermeldung fuehren. @ @ @ \\
                    . Spatium ergo \textit{(G)E} quod in secundo tempore percurret, erit ad spatium @@@ G R A F I K @@@% \begin{wrapfigure}{l}{0.4\textwidth}                    
                %\includegraphics[width=0.4\textwidth]{../images/Calculus+Elasticus/LH035%2C05%2C2_007v/files/100395.png}
                        %\caption{Bildbeschreibung}
                        %\end{wrapfigure}
                        %@ @ @ Dies ist eine Abstandszeile - fuer den Fall, dass mehrere figures hintereinander kommen, ohne dass dazwischen laengerer Text steht. Dies kann zu einer Fahlermeldung fuehren. @ @ @ \\
                     \ quod percurrit in primo, \edtext{ut @@@ G R A F I K @@@% \begin{wrapfigure}{l}{0.4\textwidth}                    
                %\includegraphics[width=0.4\textwidth]{../images/Calculus+Elasticus/LH035%2C05%2C2_007v/files/100405.png}
                        %\caption{Bildbeschreibung}
                        %\end{wrapfigure}
                        %@ @ @ Dies ist eine Abstandszeile - fuer den Fall, dass mehrere figures hintereinander kommen, ohne dass dazwischen laengerer Text steht. Dies kann zu einer Fahlermeldung fuehren. @ @ @ \\
                    }{\lemma{primo,}\Afootnote{ \textit{ (1) }\ ut @@@ G R A F I K @@@ ad \textit{ (2) }\ ut @@@ G R A F I K @@@ \textit{ L}}}\edtext{est ad @@@ G R A F I K @@@% \begin{wrapfigure}{l}{0.4\textwidth}                    
                %\includegraphics[width=0.4\textwidth]{../images/Calculus+Elasticus/LH035%2C05%2C2_007v/files/100416.png}
                        %\caption{Bildbeschreibung}
                        %\end{wrapfigure}
                        %@ @ @ Dies ist eine Abstandszeile - fuer den Fall, dass mehrere figures hintereinander kommen, ohne dass dazwischen laengerer Text steht. Dies kann zu einer Fahlermeldung fuehren. @ @ @ \\
                    }{\lemma{ut}\Afootnote{ \textit{ (1) }\ . Et porro \ vis quam percurrit in  \textit{(a)}\ secundo \textit{(b)}\ tertio temporis momento \textit{ (2) }\ est ad @@@ G R A F I K @@@ \textit{ L}}} \edtext{\ seu spatium secundum \textit{(G)E} est}{\lemma{ad}\Afootnote{ \textit{ (1) }\ seu est spat. sec. \textit{GE} = \textit{ (2) }\ \ seu spatium secundum \textit{(G)E} est \textit{ L}}} @@@ G R A F I K @@@% \begin{wrapfigure}{l}{0.4\textwidth}                    
                %\includegraphics[width=0.4\textwidth]{../images/Calculus+Elasticus/LH035%2C05%2C2_007v/files/100430.png}
                        %\caption{Bildbeschreibung}
                        %\end{wrapfigure}
                        %@ @ @ Dies ist eine Abstandszeile - fuer den Fall, dass mehrere figures hintereinander kommen, ohne dass dazwischen laengerer Text steht. Dies kann zu einer Fahlermeldung fuehren. @ @ @ \\
                     sive [@@@ G R A F I K @@@% \begin{wrapfigure}{l}{0.4\textwidth}                    
                %\includegraphics[width=0.4\textwidth]{../images/Calculus+Elasticus/LH035%2C05%2C2_007v/files/100435.png}
                        %\caption{Bildbeschreibung}
                        %\end{wrapfigure}
                        %@ @ @ Dies ist eine Abstandszeile - fuer den Fall, dass mehrere figures hintereinander kommen, ohne dass dazwischen laengerer Text steht. Dies kann zu einer Fahlermeldung fuehren. @ @ @ \\
                    ]\edtext{}{\Afootnote{@@@ G R A F I K @@@% \begin{wrapfigure}{l}{0.4\textwidth}                    
                %\includegraphics[width=0.4\textwidth]{../images/Calculus+Elasticus/LH035%2C05%2C2_007v/files/100433.png}
                        %\caption{Bildbeschreibung}
                        %\end{wrapfigure}
                        %@ @ @ Dies ist eine Abstandszeile - fuer den Fall, dass mehrere figures hintereinander kommen, ohne dass dazwischen laengerer Text steht. Dies kann zu einer Fahlermeldung fuehren. @ @ @ \\
                    \textit{\ L \"{a}ndert Hrsg. } }} sive: [@@@ G R A F I K @@@% \begin{wrapfigure}{l}{0.4\textwidth}                    
                %\includegraphics[width=0.4\textwidth]{../images/Calculus+Elasticus/LH035%2C05%2C2_007v/files/100440.png}
                        %\caption{Bildbeschreibung}
                        %\end{wrapfigure}
                        %@ @ @ Dies ist eine Abstandszeile - fuer den Fall, dass mehrere figures hintereinander kommen, ohne dass dazwischen laengerer Text steht. Dies kann zu einer Fahlermeldung fuehren. @ @ @ \\
                    ]\edtext{}{\Afootnote{@@@ G R A F I K @@@% \begin{wrapfigure}{l}{0.4\textwidth}                    
                %\includegraphics[width=0.4\textwidth]{../images/Calculus+Elasticus/LH035%2C05%2C2_007v/files/100438.png}
                        %\caption{Bildbeschreibung}
                        %\end{wrapfigure}
                        %@ @ @ Dies ist eine Abstandszeile - fuer den Fall, dass mehrere figures hintereinander kommen, ohne dass dazwischen laengerer Text steht. Dies kann zu einer Fahlermeldung fuehren. @ @ @ \\
                    \textit{\ L \"{a}ndert Hrsg. } }}. \edtext{In puncto autem \textit{E}}{\lemma{?LEMMA?:.}\Afootnote{ \textit{ (1) }\ Spatium autem quod tertio tempore percurritur, nempe \textit{EF}, \textit{ (2) }\ In puncto autem \textit{E} \textit{ L}}} vim rursus \ acquirit novam, nempe: quae sit ad vim primam, @@@ G R A F I K @@@% \begin{wrapfigure}{l}{0.4\textwidth}                    
                %\includegraphics[width=0.4\textwidth]{../images/Calculus+Elasticus/LH035%2C05%2C2_007v/files/100454.png}
                        %\caption{Bildbeschreibung}
                        %\end{wrapfigure}
                        %@ @ @ Dies ist eine Abstandszeile - fuer den Fall, dass mehrere figures hintereinander kommen, ohne dass dazwischen laengerer Text steht. Dies kann zu einer Fahlermeldung fuehren. @ @ @ \\
                    . ut est spatium \textit{((G))C} ad spatium \ \textit{EC}, seu ad spatium, [@@@ G R A F I K @@@% \begin{wrapfigure}{l}{0.4\textwidth}                    
                %\includegraphics[width=0.4\textwidth]{../images/Calculus+Elasticus/LH035%2C05%2C2_007v/files/100467.png}
                        %\caption{Bildbeschreibung}
                        %\end{wrapfigure}
                        %@ @ @ Dies ist eine Abstandszeile - fuer den Fall, dass mehrere figures hintereinander kommen, ohne dass dazwischen laengerer Text steht. Dies kann zu einer Fahlermeldung fuehren. @ @ @ \\
                    ]\edtext{}{\Afootnote{@@@ G R A F I K @@@% \begin{wrapfigure}{l}{0.4\textwidth}                    
                %\includegraphics[width=0.4\textwidth]{../images/Calculus+Elasticus/LH035%2C05%2C2_007v/files/100465.png}
                        %\caption{Bildbeschreibung}
                        %\end{wrapfigure}
                        %@ @ @ Dies ist eine Abstandszeile - fuer den Fall, dass mehrere figures hintereinander kommen, ohne dass dazwischen laengerer Text steht. Dies kann zu einer Fahlermeldung fuehren. @ @ @ \\
                    \textit{\ L \"{a}ndert Hrsg. } }}. Itaque spatium \textit{EF}, \ quod tertio tempore percurritur erit ad spatium @@@ G R A F I K @@@% \begin{wrapfigure}{l}{0.4\textwidth}                    
                %\includegraphics[width=0.4\textwidth]{../images/Calculus+Elasticus/LH035%2C05%2C2_007v/files/100474.png}
                        %\caption{Bildbeschreibung}
                        %\end{wrapfigure}
                        %@ @ @ Dies ist eine Abstandszeile - fuer den Fall, dass mehrere figures hintereinander kommen, ohne dass dazwischen laengerer Text steht. Dies kann zu einer Fahlermeldung fuehren. @ @ @ \\
                     primo tempore percursum, ut est \ \edtext{@@@ G R A F I K @@@% \begin{wrapfigure}{l}{0.4\textwidth}                    
                %\includegraphics[width=0.4\textwidth]{../images/Calculus+Elasticus/LH035%2C05%2C2_007v/files/100484.png}
                        %\caption{Bildbeschreibung}
                        %\end{wrapfigure}
                        %@ @ @ Dies ist eine Abstandszeile - fuer den Fall, dass mehrere figures hintereinander kommen, ohne dass dazwischen laengerer Text steht. Dies kann zu einer Fahlermeldung fuehren. @ @ @ \\
                     ad @@@ G R A F I K @@@% \begin{wrapfigure}{l}{0.4\textwidth}                    
                %\includegraphics[width=0.4\textwidth]{../images/Calculus+Elasticus/LH035%2C05%2C2_007v/files/100486.png}
                        %\caption{Bildbeschreibung}
                        %\end{wrapfigure}
                        %@ @ @ Dies ist eine Abstandszeile - fuer den Fall, dass mehrere figures hintereinander kommen, ohne dass dazwischen laengerer Text steht. Dies kann zu einer Fahlermeldung fuehren. @ @ @ \\
                    }{\lemma{\}\Afootnote{ \textit{ (1) }\ @@@ G R A F I K @@@ ad \textit{v} \textit{ (2) }\ @@@ G R A F I K @@@ ad @@@ G R A F I K @@@ \textit{ L}}}. Itaque \ spatium primo tempore percursum est γ. secundo \textit{(G)E} est @@@ G R A F I K @@@% \begin{wrapfigure}{l}{0.4\textwidth}                    
                %\includegraphics[width=0.4\textwidth]{../images/Calculus+Elasticus/LH035%2C05%2C2_007v/files/100493.png}
                        %\caption{Bildbeschreibung}
                        %\end{wrapfigure}
                        %@ @ @ Dies ist eine Abstandszeile - fuer den Fall, dass mehrere figures hintereinander kommen, ohne dass dazwischen laengerer Text steht. Dies kann zu einer Fahlermeldung fuehren. @ @ @ \\
                     tertio tempore \ percursum seu \textit{EF} est: [@@@ G R A F I K @@@% \begin{wrapfigure}{l}{0.4\textwidth}                    
                %\includegraphics[width=0.4\textwidth]{../images/Calculus+Elasticus/LH035%2C05%2C2_007v/files/100503.png}
                        %\caption{Bildbeschreibung}
                        %\end{wrapfigure}
                        %@ @ @ Dies ist eine Abstandszeile - fuer den Fall, dass mehrere figures hintereinander kommen, ohne dass dazwischen laengerer Text steht. Dies kann zu einer Fahlermeldung fuehren. @ @ @ \\
                    ]\edtext{}{\Afootnote{@@@ G R A F I K @@@% \begin{wrapfigure}{l}{0.4\textwidth}                    
                %\includegraphics[width=0.4\textwidth]{../images/Calculus+Elasticus/LH035%2C05%2C2_007v/files/100501.png}
                        %\caption{Bildbeschreibung}
                        %\end{wrapfigure}
                        %@ @ @ Dies ist eine Abstandszeile - fuer den Fall, dass mehrere figures hintereinander kommen, ohne dass dazwischen laengerer Text steht. Dies kann zu einer Fahlermeldung fuehren. @ @ @ \\
                    \textit{\ L \"{a}ndert Hrsg. } }}\ 