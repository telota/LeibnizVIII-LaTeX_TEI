      
               
                \begin{ledgroupsized}[r]{120mm}
                \footnotesize 
                \pstart                
                \noindent\textbf{\"{U}berlieferung:}   
                \pend
                \end{ledgroupsized}
            
              
                            \begin{ledgroupsized}[r]{114mm}
                            \footnotesize 
                            \pstart \parindent -6mm
                            \makebox[6mm][l]{\textit{LiH}}Marginalien, An- und Unterstreichungen in \textsc{J. Aleaume}, \cite{00003}\textit{La perspective speculative et pratique}, Paris 1643. Auf den Seiten 80, 155 und 156 Unterstreichungen mit Bleistift, die nicht eindeutig Leibniz zugewiesen werden k\"{o}nnen und daher keine Ber\"{u}cksichtigung finden. \pend
                            \end{ledgroupsized}
                %\normalsize
                \vspace*{5mm}
                \begin{ledgroup}
                \footnotesize 
                \pstart
            \noindent\footnotesize{\textbf{Datierungsgr\"{u}nde}: Wie in N. 27 beziehen wir uns auch in diesem St\"{u}ck zur Datierung auf die Gespr\"{a}chsnotiz N. 28, die einen Hinweis auf die Quellen der \cite{00034}\textit{Optik} von Desargues\protect\index{Namensregister}{\textso{Desargues,} Girard 1591\textendash 1661} enth\"{a}lt. Es ist anzunehmen, dass Leibniz' Desargues-Lekt\"{u}re den Anlass f\"{u}r Mariottes\protect\index{Namensregister}{\textso{Mariotte,} Edme, Seigneur de Chazeuil ca. 1620\textendash 1684} Bemerkung zu Desargues\protect\index{Namensregister}{\textso{Desargues,} Girard 1591\textendash 1661} gab, so dass wir die Datierungen von N. 27 und N. 28 \"{u}bernehmen.}
                \pend
                \end{ledgroup}
            
                \vspace*{8mm}
                \pstart 
                \normalsize
            [p.~4] La Figure apparente dans le Tableau, se nomme FIGVRE PERSPECTIVE; et les figures qui servent de base\footnote{\textit{Leibniz unterstreicht:} qui servent de base} aux Cones, ou Pyramides visuelles\protect\index{Sachverzeichnis}{pyramide visuelle}, se nomment PLANS GEOMETRAVX, ou Plans Primitifs, ou Figures Geometrales et Primitives.