      
               
                \begin{ledgroupsized}[r]{120mm}
                \footnotesize 
                \pstart                
                \noindent\textbf{\"{U}berlieferung:}   
                \pend
                \end{ledgroupsized}
            
              
                            \begin{ledgroupsized}[r]{114mm}
                            \footnotesize 
                            \pstart \parindent -6mm
                            \makebox[6mm][l]{\textit{L}}Notiz: LH XXXVII 2 Bl. 7. 1 Bl. 13 x 5 cm. 9 Zeilen, R\"{u}ckseite leer. Der linke sowie der obere und untere Seitenrand beschnitten.\\Kein Eintrag in KK 1 oder Cc 2. \pend
                            \end{ledgroupsized}
                %\normalsize
                \vspace*{5mm}
                \begin{ledgroup}
                \footnotesize 
                \pstart
            \noindent\footnotesize{\textbf{Datierungsgr\"{u}nde}: Die vorliegende Notiz bezieht sich auf den Titel \cite{00007}\textit{Opticorum libri duo} des Heliodor von Larissa. Ihr Inhalt kehrt in dem St\"{u}ck \textit{Ratio aequalitatis angulorum reflexionis et incidentiae}, N. 24, wieder, dessen Datierung wir \"{u}bernehmen.}
                \pend
                \end{ledgroup}
            
                \vspace*{8mm}
                \pstart 
                \normalsize
            \centering [7 r\textsuperscript{o}] Optica. \pend \vspace{1.0ex} \pstart Erasmi Bartholini\protect\index{Namensregister}{\textso{Bartholin} (Bartholinus), Erasmus 1625\textendash 1698} Heliodorus Larissaeus\protect\index{Namensregister}{\textso{Heliodor v. Larissa} } Paris. 1657.\edtext{}{\lemma{Paris. 1657.}\Bfootnote{\textsc{Heliodor v. Larissa}, \cite{00000}\textit{Opticorum libri duo}, Paris 1657, S.~118f. }} Vide ibi rationem allatam ab Heliodoro\protect\index{Namensregister}{\textso{Heliodor v. Larissa} } cur Anguli incidentiae\protect\index{Sachverzeichnis}{angulus!incidentiae} et reflexionis\protect\index{Sachverzeichnis}{angulus!reflexionis} aequales.\pend 