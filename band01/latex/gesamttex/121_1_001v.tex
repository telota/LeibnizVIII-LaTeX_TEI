      
               
                \begin{ledgroupsized}[r]{120mm}
                \footnotesize 
                \pstart                
                \noindent\textbf{\"{U}berlieferung:}   
                \pend
                \end{ledgroupsized}
            
              
                            \begin{ledgroupsized}[r]{114mm}
                            \footnotesize 
                            \pstart \parindent -6mm
                            \makebox[6mm][l]{\textit{L}}Exzerpt: LH XXXVII 2 Bl. 1\textendash 2. 1 Bog. 2\textsuperscript{o}. 1 S. zweispaltig auf Bl. 1 v\textsuperscript{o}. Linke Spalte fortlaufender Text, rechte Spalte oben eine Marginalie. Die verbleibenden Seiten des Bogens N. 16 und N. 18.\\
                            KK 1, Nr. 973 B\pend
                            \end{ledgroupsized}
              
                            \begin{ledgroupsized}[r]{114mm}
                            \footnotesize 
                            \pstart \parindent -6mm
                            \makebox[6mm][l]{\textit{}} \pend
                            \end{ledgroupsized}
                %\normalsize
                \vspace*{5mm}
                \begin{ledgroup}
                \footnotesize 
                \pstart
            \noindent\footnotesize{\textbf{Datierungsgr\"{u}nde}: Die Datierung erfolgt aufgrund des Wasserzeichens. Vgl. dazu N. 18.}
                \pend
                \end{ledgroup}
            
                \vspace*{8mm}
                \pstart 
                \normalsize
            [1 v\textsuperscript{o}] Cum Hyperbola\footnote{\textit{In der rechten Spalte}: Potest fieri quadam quasi \textso{Hyperbola} Mechanica, dispositionibus sectionum ex sphaeris. Ad Hyperbolam etc. % @@@Afootnote@@@ \textbar\ etc. \textit{ erg.}\ \textbar\  
            exacte elaborandam motus uno velut ictu ac momento, ope pulveris pyrii exercendus.} et Ellipsis colligant omnes radios ex puncto in axe optico\protect\index{Sachverzeichnis}{axis!opticus}, et vicinissimis tanto scil. pluribus, licet tam minus accurate, quanto ipsa Hyperbola aut Ellipsis obtusior. Hinc fieri potest figura optica quasi perfecta. Constans ex meris vel Ellipsibus Hyperbolisque sibi appositis quasi mechanica quadam construendi ratione, ut huic hoc illi aliud objecti punctum sit in axe optico\protect\index{Sachverzeichnis}{axis!opticus}, ita totum simul perfecte, quantum possibile est detegetur: inprimis si illae variae projectiones inter se uniantur, ut si in unum speculum concavum\protect\index{Sachverzeichnis}{speculum!concavum} incidant, ubi unientur ob auctam magnitudinem. Aut si in convexum\protect\index{Sachverzeichnis}{speculum!convexum} ubi unientur \edtext{ob arctitatem}{\lemma{ob}\Afootnote{ \textit{ (1) }\ arctam \textit{ (2) }\ arctitatem \textit{ L}\hspace{10mm}22 \hspace{3mm} etc. \textit{ erg.}\ \textit{ L }}} spatii poterunt autem inde projici in amplificans speculum Tubumque\protect\index{Sachverzeichnis}{tubus!opticus}. Amplificatio in parte erit. Haec tantum pro iis quae lucida non sunt, aut non illustrabilia. Illustrabilia satis radiis datis possumus videre solis sphaericis lentibus\protect\index{Sachverzeichnis}{lens!sphaerica}. Nota si objectum in centro pluries Hyperbolae circa collocatum ipsum respicientes quodammodo videbuntur\rule[-10mm]{0mm}{0mm} in unam figuram conjunctae et praestabunt lentem\protect\index{Sachverzeichnis}{lens} perfectam mechanicam. Videbitur \edtext{sic totum}{\lemma{Videbitur}\Afootnote{ \textit{ (1) }\ quidem \textit{ (2) }\ sic totum \textit{ L }}} objectum saepe simul, sed ita omnes ejus partes distinctissime.\pend \pstart  Magnitudo augeri potest in infinitum tum aucto vitro sphaerae majoris, et magis remoto a sphaera exigua. Quod tamen plurimum nocet luci, tamen quod aptius adhibito speculo concavissimo\protect\index{Sachverzeichnis}{speculum!concavum} aut etiam parabolico \protect\index{Sachverzeichnis}{speculum!parabolicum} dicendum est in infinitum auget, manente eadem via nec proinde aucta longitudine Tubi quae parit obscuritatem. \pend \pstart  An utile objectum in speculum\protect\index{Sachverzeichnis}{speculum} allapsum microscopio \protect\index{Sachverzeichnis}{microscopium} intueri. Ita arbitror est enim quasi pictura. Sed speculum\protect\index{Sachverzeichnis}{speculum} obscurandum quantum licet. Cogitandum de rationibus obscurandi in summo gradu.\pend \pstart  Hyperbolis et Ellipsibus in Tubis, parabolis in speculis\protect\index{Sachverzeichnis}{speculum} potest augeri Lux\protect\index{Sachverzeichnis}{lux} dati puncti seu numerus radiorum collectorum in infinitum, pluribus conjunctis defectura. \pend \pstart  Apertura in Hyperbolis \edtext{et parabolis}{\lemma{}\Afootnote{et parabolis \textit{ erg.} \textit{ L}}} in arbitrio est. Magnitudo deinde vel speculis concavis\protect\index{Sachverzeichnis}{speculum!concavum}, vel Tubis,  seu proportione lentis\protect\index{Sachverzeichnis}{lens} ad objectivum\protect\index{Sachverzeichnis}{objectivum}. Ita puto rem opticam posse augeri in infinitum. Ita videor reperire, quod summum potest. Nisi quis figuram simplicem reperiat quae praestet vicem Ellipsium vel Hyperbolarum conjunctarum. \pend \pstart  Nota. Quoniam \edtext{non anguli sed sinus refractionum}{\lemma{Quoniam}\Afootnote{ \textit{ (1) }\ Refractio\protect\index{Sachverzeichnis}{refractio|textit} \textit{ (2) }\ non anguli sed sinus refractionum \textit{ L}}} sunt proportionales. Ideo licet refractionibus multiplicatis lucrari aliquid facereque per exemplum. NB. Solis vitris concavis ac proinde pandochis \edtext{seu ordinatis}{\lemma{}\Afootnote{seu ordinatis \textit{ erg.} \textit{ L}}} radios convergentes. \pend \pstart  Adde modum \edtext{Hookii}{\lemma{modum}\Afootnote{ \textit{ (1) }\ Lanae\protect\index{Namensregister}{\textso{Lana,} Francesco 1631\textendash 1687|textit}  \textit{ (2) }\ Hookii \textit{ L}}} eundem Tubum\protect\index{Sachverzeichnis}{tubus!opticus} faciendi iisdem vitris longiorem.\edtext{}{\lemma{longiorem.}\Bfootnote{\textsc{R. Hooke, }\cite{00061}\textit{Micrographia}, London 1665, Vorwort, e\textendash  f. }} \pend \pstart  Consule propositiones Auzuti\protect\index{Namensregister}{\textso{Auzout} (Auzutus), Adrien 1622\textendash 1603} ad Hookium\protect\index{Namensregister}{\textso{Hooke} (Hookius, Hook), Robert 1635\textendash 1703} ubi petit ejus artem faciendi exiguae sphaerae vitro Tubum\protect\index{Sachverzeichnis}{tubus!opticus} magnum.\edtext{}{\lemma{magnum.}\Bfootnote{\textsc{Anonym}, \cite{00283}\textit{Considerations}, \textit{PT} 1 (1665), S. 60f.}} \pend 