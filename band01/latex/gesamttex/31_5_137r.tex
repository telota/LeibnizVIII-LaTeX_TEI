
 \pstart \textso{On voit par l\`{a} que ce que}
\edtext{\edlabel{onvoitstart}\textso{j'avois avanc\'{e} de la cause d'une solidit\'{e} primitive ne choque pas ce sentiment des meilleurs philosophes},}{\lemma{\textso{que}}\Afootnote{ \textit{ (1) }\hspace{10pt} \textso{Mons. des Cartes avoit bien dit, que les liqueurs ont un mouuement dans} \textit{ (2) }\ \textso{j'avois} [...] \textso{philosophes} \textit{ L}}}\edtext{\edlabel{onvoitend}}{\lemma{\textso{On voit}}\xxref{onvoitstart}{onvoitend}
\Afootnote{[...] \textso{philosophes}, \textit{am Rand doppelt angestrichen}}}
 que \textso{Mons. }\textso{Boyle}\protect\index{Namensregister}{\textso{Boyle} (Boylius, Boyl., Boyl), Robert 1627\textendash 1691} a prouu\'{e} par des experiences, que les liqueurs que nous voyons, ont dans leur interieur un mouuement troubl\'{e} et quasi en tous sens. Dont la raison est manifeste, parce que le moindre mouuement exit\'{e} dans une liqueur se propage incontinent par toute l'\'{e}tend\"{u}e de la liqueur et concourrant avec une grande quantit\'{e} d'autres impressions
 [137 v\textsuperscript{o}] produit une variet\'{e} de mouuements inconceuable, en tous sens, et par toute sorte de lignes. Car quoyque cette multitude en effect soit finie, et qu'il n'est pas possible, que le mouuement soit veritablement \edtext{dans la rigeur Geometrique}{\lemma{}\Afootnote{dans la rigeur Geometrique \textit{ erg.} \textit{ L}}} en tous sens, parce qu'une ligne de mouuement croisant l'autre tout le mouuement cesseroit en un moment; neantmoins comme la liqueur est divis\'{e}e en tant de petites parties, il est ais\'{e} \`{a} concevoir, comment \edtext{presque tout}{\lemma{comment}\Afootnote{ \textit{ (1) }\ un \textit{ (2) }\ presque tout \textit{ L}}} point sensible assign\'{e} puisse estre frapp\'{e} dans toute autre ligne sensible et de tout autre point sensible de la m\^{e}me liqueur. Parce que tous ces poincts, et toutes ces lignes sensibles sont comme des mondes\protect\index{Sachverzeichnis}{monde} \`{a} l'egard de la subtilit\'{e} dont la nature est capable par une subdivision continuelle.
\pend 