[69 r\textsuperscript{o}] \textso{cujuscunque,} tempore aliquo cognito; seu ut sciat fixam\protect\index{Sachverzeichnis}{stella!fixa} aliquam, aliquo tempore cognito a meridiano \protect\index{Sachverzeichnis}{meridianus} primo tantum abesse\edtext{}{\lemma{}\Afootnote{abesse  \textbar\ debere \textit{ gestr.}\ \textbar\ . \textit{ L}}}. Verbi gratia hanc, illamve fixam\protect\index{Sachverzeichnis}{stella!fixa} pro arbitrio assumptam, mille ab hinc diebus 13. horis, 50. minutis, et 45. secundis horariis, et die non a solis sed primi mobilis revolutione aestimato, a meridiano\protect\index{Sachverzeichnis}{meridianus} primo terrae abfuisse: 80. gradibus, 50. minutis, et tribus minuti graduum primi, quartis, in aequatore \protect\index{Sachverzeichnis}{aequator} aut ejus parallello\protect\index{Sachverzeichnis}{circulus parallelus} numeratis, quod semel in universum eum ab Astronomo aliquo dedicisse aut calculari sibi petiisse sufficit.\pend \pstart Ita enim sciet eam nunc abesse ab eodem Meridiano \protect\index{Sachverzeichnis}{meridianus} 25. gr. et tribus quartis minuti graduum primi, et quantum quocunque tempore sequenti per horologium\protect\index{Sachverzeichnis}{horologium} exactum sibi cognito, sit abfutura, quando postea opus erit calculabit. Quo calculo nihil est facilius, cum tantum horas ad gradus et minuta secundaque horarum ad minuta secundaque graduum reducat.\pend \pstart Hoc pacto si unius fixae\protect\index{Sachverzeichnis}{stella!fixa} longitudinem\protect\index{Sachverzeichnis}{longitudo} tempore dato sciet, sciet omnium, imo videbit in globo, si eam tantum cujus longitudinem\protect\index{Sachverzeichnis}{longitudo} cognovit, meridiano\protect\index{Sachverzeichnis}{meridianus} primo debite admoveat. Ita enim caeteras omnes et eadem opera ei debite admotas esse globus monstrabit.\pend \pstart Si quis hujus quoque calculi compendium facere vellet, posset ipsam sphaeram artificialem astris depictis notatam horologio\protect\index{Sachverzeichnis}{horologium} accurato accommodare, ut sua revolutione horas ostendat, Meridiano\protect\index{Sachverzeichnis}{meridianus} primo artificiali immobili manente, contra Horizonte et Meridiano\protect\index{Sachverzeichnis}{meridianus} loci, qui vulgo in  sphaeris artificialibus immobiles repraesentantur, totamque machinam velut sustentant,  mobilibus redditis.\pend \pstart Ut ita \edtext{eo casu}{\lemma{}\Afootnote{eo casu \textit{ erg.} \textit{ L}}} ad poli elevationem\protect\index{Sachverzeichnis}{elevatio!poli}, non polum\protect\index{Sachverzeichnis}{polus} cum sphaera elevari sed horizontem cui meridianus\protect\index{Sachverzeichnis}{meridianus} affixus est, deprimi necesse sit, et contra. Ita enim modo 