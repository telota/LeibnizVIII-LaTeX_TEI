[94~r\textsuperscript{o}] solamque causam antliae deficientis fore aeris ex Mercurio antea non purgato eliciti, inter Embolum et Mercurium interpositionem, quae faciet, ut Embolus postea aerem potius distendat quam Mercurium trahat; at in Barometro haec causa est tantum cur Mercurius non maneat in summo tubi (quod aere purgatus facit) non vero cur semel a summo tubi delapsus denuo suspensus maneat. Quod ob aeris contraprementis aequilibrium\protect\index{Sachverzeichnis}{aequilibrium} contingere\edlabel{confi93vb}  Recipiens exhaustus demonstrat, ubi Mercurius\protect\index{Sachverzeichnis}{mercurius} ob hoc   aequilibrium\protect\index{Sachverzeichnis}{aequilibrium} cessans subsidit. Quo experimento   mihi \edtext{funiculus\protect\index{Sachverzeichnis}{funiculus} maxime refutatus}{\lemma{funiculus}\Afootnote{ \textit{ (1) }\ satis valide refu \textit{ (2) }\ maxime refutatus \textit{ L}}} videtur;   nam alioquin dici poterat, Mercurium\protect\index{Sachverzeichnis}{mercurius} ultra subsidere non   posse, quia aerem intra summitatem \edtext{tubi}{\lemma{tubi}\Afootnote{\textit{ erg.} \textit{ L}}} et superficiem suam   comprehensum ultra rarefacere non possit, quod descendendo ulterius   faceret. Ad majorem tamen rei tanti momenti certitudinem,   experimentum sequens capi potest: Tubo Torricelliano\protect\index{Sachverzeichnis}{Tubus!Torricellianus}   aliud quoddam vas aut ampulla Epistomio\protect\index{Sachverzeichnis}{epistomium} communi   conjungatur, in hoc vase esto inclusus aer \edtext{ordinarius aut si placet nonnihil   compressus}{\lemma{aer}\Afootnote{ \textit{ (1) }\ ordinarius, aut   etiam si placet justo amplius  \textit{(a)}\ compressus \textit{(b)}\ nonnihil compre \textit{ (2) }\ ordinarius [...] compressus \textit{ L}}} \edtext{cum aere externo nullam habens communicationem}{\lemma{cum}\Afootnote{ [...] communicationem \textit{ erg.} \textit{ L}}}\edtext{}{\lemma{communicationem}\Afootnote{\textbar\ ultra mensuram \textit{ gestr.}\ \textbar\ . Ubi \textit{ L}}}. Ubi \edtext{ergo}{\lemma{}\Afootnote{ergo \textit{ erg.} \textit{ L}}} Mercurius\protect\index{Sachverzeichnis}{mercurius} ad altitudinem   consuetam subsederit, Epistomium\protect\index{Sachverzeichnis}{epistomium} aperiatur, manifestum est   nonnihil aeris ex Epistomio\protect\index{Sachverzeichnis}{epistomium} in tubum intraturum esse,   atque ita aerem in tubo a Mercurio\protect\index{Sachverzeichnis}{mercurius} dilatatum in \edtext{gradum}{\lemma{in}\Afootnote{ \textit{ (1) }\ modo \textit{ (2) }\ gradum \textit{ L}}} aeris naturalis rediturum, aut ei certe appropinquaturum, ac proinde facilius jam distendi posse, quam cum   jam valde tensus erat, \edtext{ac proinde permissurum   esse, ut Mercurius}{\lemma{erat,}\Afootnote{ \textit{ (1) }\ Mercurium\protect\index{Sachverzeichnis}{mercurius|textit} \textit{ (2) }\ ac [...] Mercurius \textit{ L}}} ad fundum usque descendat. Hoc experimentum   decretorium est, nam si descendet Mercurius\protect\index{Sachverzeichnis}{mercurius} necesse est eum   ab aeris distracti Elaterio\protect\index{Sachverzeichnis}{elaterium} sive funiculo\protect\index{Sachverzeichnis}{funiculus} fuisse antea   retentum; sin minus necesse est funiculi\protect\index{Sachverzeichnis}{funiculus} illius Elaterium\protect\index{Sachverzeichnis}{elaterium}   nihil egisse. Idem experimentum sumi potest facilius, si \edtext{vas ipsum stagnans}{\lemma{si}\Afootnote{ \textit{ (1) }\ totum vas stagnans \textit{ (2) }\ vas ipsum stagnans \textit{ L}}}, cum Tubo Torricelliano\protect\index{Sachverzeichnis}{Tubus!Torricellianus}   in \edtext{aere clauso}{\lemma{in}\Afootnote{ \textit{ (1) }\ loco clauso \textit{ (2) }\ aere clauso \textit{ L}}} collocetur, et per apertum Epistomium\protect\index{Sachverzeichnis}{epistomium}   ex aere libero nonnihil in Tubum Torricellianum\protect\index{Sachverzeichnis}{Tubus!Torricellianus} immittatur.\pend \pstart   Pro certo habeo, si recte administretur Experimentum, fore \edtext{ut  Mercurius}{\lemma{ut}\Afootnote{ \textit{ (1) }\ Tubus \textit{ (2) }\  Mercurius \textit{ L}}} propterea ultra solitum non subsidat.   Quo posito pro infallibili habendam erit aeris pressionem\protect\index{Sachverzeichnis}{pressio!aeris} esse Experimenti Torricelliani\protect\index{Sachverzeichnis}{experimentum!Torricellianum} \edtext{\edlabel{causam2start}causam.}{\lemma{causam.}\xxref{causam2start}{causam2end}\Afootnote{ \textit{ (1) }\ At vero Siphonis\protect\index{Sachverzeichnis}{sipho|textit}  \textit{(a)}\ aequicruri, et la \textit{(b)}\ bicruri et antliae\protect\index{Sachverzeichnis}{antlia|textit}   communis (modo quod dixi Experimentum succedat) et laminarum\protect\index{Sachverzeichnis}{laminae politae|textit} duarum \textit{ (2) }\ Antliae vero causam  \textit{(a)}\ esse aliam \textit{(b)}\ longe [...] disci. \textit{ L}}} \pend 
\pstart Antliae vero causam longe differre etiam alio experimento poterit disci.\edlabel{causam2end} \edtext{Sugatur}{\lemma{disci.}\Afootnote{ \textit{ (1) }\ Cum \textit{ (2) }\ Rationis \textit{ (3) }\ Sumatur \textit{ (4) }\   Sugatur \textit{ L}}} per antliam\protect\index{Sachverzeichnis}{antlia} Spiritus vini\protect\index{Sachverzeichnis}{spiritus!vini}, loco Mercurii\protect\index{Sachverzeichnis}{mercurius} vel aquae,   ajo Spiritum \edtext{vini\protect\index{Sachverzeichnis}{spiritus!vini} non aeque longe   elevatum iri, quam pro levitatis suae ratione,}{\lemma{vini}\Afootnote{ \textit{ (1) }\ facilius quam proportio \textit{ (2) }\ non [...] ratione, \textit{ L}}} \edtext{Mercurii\protect\index{Sachverzeichnis}{mercurius} comparatione}{\lemma{}\Afootnote{Mercurii\protect\index{Sachverzeichnis}{mercurius} comparatione \textit{ erg.} \textit{ L}}} deberet,   quia citius aerem intra se Embolumque\protect\index{Sachverzeichnis}{embolus} generabit,   ac proinde facile ab Embolo\protect\index{Sachverzeichnis}{embolus} deseretur.\pend \pstart   His positis assero eandem esse causam videri Antliae\protect\index{Sachverzeichnis}{antlia} suctoriae   communis \edtext{imo suctionis in genere}{\lemma{imo}\Afootnote{suctionis in genere \textit{ erg.} \textit{ L}}} \edtext{Siphonis\protect\index{Sachverzeichnis}{sipho} bicruri, duarum Tabularum politarum}{\lemma{Siphonis}\Afootnote{ \textit{ (1) }\ aequicruri, et duarum laminarum\protect\index{Sachverzeichnis}{laminae politae|textit} \textit{ (2) }\ bicruri, duarum Tabularum politarum \textit{ L}}}   \edtext{cohaerentium, ac denique Experimenti novi quo deprehensum   est, aquam aut Mercurium,}{\lemma{cohaerentium,}\Afootnote{ \textit{ (1) }\ eamque ab aeris pressione\protect\index{Sachverzeichnis}{pressio!aeris|textit} plane differentem \textit{ (2) }\   ac denique Experimenti novi  \textbar\ in aere \textit{ gestr.}~\textbar\  quo deprehensum   est,  \textit{(a)}\ aerem \textit{(b)}\ aquam aut Mercurium, \textit{ L}}} modo ab aere purgentur,   non descendere, ne in Tubo quidem Torricelliano\protect\index{Sachverzeichnis}{Tubus!Torricellianus}, et horum   omnium rationem parum differre, ab illa veterum fuga vacui\protect\index{Sachverzeichnis}{fuga vacui} toties decantata. Fuga vacui\protect\index{Sachverzeichnis}{fuga vacui}, inquam, scilicet \edtext{notabilis}{\lemma{scilicet}\Afootnote{ \textit{ (1) }\ sensibili \textit{ (2) }\ notabilis \textit{ L}}} seu acervati, an enim nullum omnino in Mundo sit vacuum non est hujus loci determinare.