[68 v\textsuperscript{o}] $\lbrack$a$\rbrack$ut Nauticum summe utilem, imo necessariam hactenus praetermissam\edtext{}{\lemma{}\Afootnote{praetermissam  \textbar\ esse \textit{ erg. u.}\  \textit{ gestr.}\ \textbar\ ; \textit{ L}}}; sed scilicet coelo potius quam terrae astronomia accommodabatur. Ephemeridibus\protect\index{Sachverzeichnis}{ephemeris} quocumque operosis si stellis fixis\protect\index{Sachverzeichnis}{stella!fixa} utamur, opus ad rem praesentem non habemus. Quia enim omnes fixarum\protect\index{Sachverzeichnis}{stella!fixa} revolutiones sunt aequales, hinc si semel noverimus quantum fixa\protect\index{Sachverzeichnis}{stella!fixa} vel unica, momento aliquo cognito a meridiano\protect\index{Sachverzeichnis}{meridianus} primo abfuerit sciemus quantum omnes quolibet momento sequente cognito, ab eo sunt abfuturae. Si scilicet tot revolutiones, quot 24. horas inter tempus, quo distantia stellae\protect\index{Sachverzeichnis}{stella}, a meridiano \protect\index{Sachverzeichnis}{meridianus} primo cognita est, et aliud quo quaeritur,  interjectas, numeremus, et pro residuis horis aut minutis eum calculum instituamus, ut unum minutum secundum horae; quindecim minutis secundis, vel quartae parti minuti primi aequatoris\protect\index{Sachverzeichnis}{aequator} aut circuli aequatori\protect\index{Sachverzeichnis}{aequator} paralleli, respondeat.\pend \pstart Et quia Aequator\protect\index{Sachverzeichnis}{aequator} Parallelorum\protect\index{Sachverzeichnis}{circulus parallelus} omnium est maximus et in circulo majore accuratiores haberi possunt subdivisiones, ideo utile erit calculum institui in fixa sub aequatore\protect\index{Sachverzeichnis}{aequator} manente, eique meridianum\protect\index{Sachverzeichnis}{meridianus}  primum artificialem, accomodari.\pend \pstart Nauta ergo mari se commissurus adhunc quidem usum ultra dudum necessaria, ut horologium\protect\index{Sachverzeichnis}{horologium} exactum, instrumentumque elevationibus poli\protect\index{Sachverzeichnis}{elevatio!poli} stellarumve\protect\index{Sachverzeichnis}{stella} aliarum capiendis accommodatum, notitiamque fixarum\protect\index{Sachverzeichnis}{stella!fixa} notabiliorum; neque instrumento, neque scientia alia opus habet, quam \textso{Sphaera} \textso{artificiali} novo hoc modo adornata, \edtext{notitiaque usus ejus, et longitudinis\protect\index{Sachverzeichnis}{longitudo} cujusdam fixae\protect\index{Sachverzeichnis}{stella!fixa}. Sphaera Artificialis, quae poterit esse}{\lemma{}\Afootnote{BITTE UEBERPRUEFEN!!! notitiaque usus ejus, et longitudinis\protect\index{Sachverzeichnis}{longitudo} cujusdam fixae\protect\index{Sachverzeichnis}{stella!fixa}. Sphaera Artificialis, quae poterit esse \textit{ erg.} \textit{ L}}} ampla satis, ut aequator\protect\index{Sachverzeichnis}{aequator} ejus accuratarum divisionum \edtext{}{\lemma{}\Afootnote{divisionum  \textbar\ saltem \textit{ gestr.}\ \textbar\ ad \textit{ L}}}ad quartam usque minuti graduum primi partem, \edtext{si velimus}{\lemma{}\Afootnote{si velimus \textit{ erg.} \textit{ L}}} capax sit. Si quidem velimus inquam, eousque subtilitatis progredi in circulo aliquo coelesti, quousque progressi sumus in Horologio\protect\index{Sachverzeichnis}{horologium}, cum horologium\protect\index{Sachverzeichnis}{horologium} habeamus, quod secunda minuta satis fide monstret et minuta secunda horaria, tribus quartis minuti gradus, respondeant tanta subtilitas ut jam supra dixi necessaria non est; et si esset, possent commode nihilominus rationes haberi, quibus plurimum divisionum lucremur. Praeter hanc sphaeram ergo opus est nautae, tum et \textso{cognitio}\textso{ne usus} ejus facillimi hactenus explicati, tum \edtext{\textso{notitia longitudinis fixae}}{\lemma{tum}\Afootnote{ \textit{ (1) }\ notitia longitudinis\protect\index{Sachverzeichnis}{longitudo|textit} fixae\protect\index{Sachverzeichnis}{stella!fixa|textit} \textit{ (2) }\ \textso{notitia longitudinis fixae} \textit{ L}}} (a polo remotioris) 