[70 v\textsuperscript{o}] reperiatur difficile est, cum etiam in nocte serena aut coelo sereno, multa sint momenta aut puncta non serena, et inter ea facile hoc quoque. Contra in multis noctibus non serenis, aliquot saltem momenta temporis aut loca coeli serena sunt, quod in nostra methodo sufficit.\pend
\pstart Solutio haec est: in sphaera artificiali qualem supra descripsi, polus\protect\index{Sachverzeichnis}{polus} ad datam (\edtext{in horizonte}{\lemma{}\Afootnote{in horizonte \textit{ erg.} \textit{ L}}} loci navis\protect\index{Sachverzeichnis}{navis}) elevationem erigatur; stellae\protect\index{Sachverzeichnis}{stella} quoque observatae ea supra horizontem loci, quae observatione comperta est, elevatio in eadem sphaera detur. Observetur jam \textso{lunae }\protect\index{Sachverzeichnis}{luna}non altitudo tantum seu \textso{elevatio super horizontem,} in instrumento ad horizontem perpendiculari: Sed et eadem opera \textso{differentia }seu declinatio\protect\index{Sachverzeichnis}{declinatio} \textso{plagae }\textso{Lunae}\protect\index{Sachverzeichnis}{luna} \textso{a plaga }\textso{stellae }\protect\index{Sachverzeichnis}{stella}dictae; in horizonte vel instrumento graduum, ut circulo, aut semicirculo, aut quadrante etc. ad horizontem parallelo, seu ut appareat quoad ejus fieri praecise potest quem planum Lunae\protect\index{Sachverzeichnis}{luna} per centrum, Horizontis ad angulos \edtext{rectos}{\lemma{}\Afootnote{rectos \textit{ erg.} \textit{ L}}} transiens, ad planum stellae\protect\index{Sachverzeichnis}{stella} eodem modo transiens angulum faciat.\pend \pstart Tametsi enim id planum stellae\protect\index{Sachverzeichnis}{stella} sit verticale, seu transeat per centrum telluris\protect\index{Sachverzeichnis}{tellus}, at planum lunae\protect\index{Sachverzeichnis}{luna} non sit verticale, cum productum non transeat per centrum terrae, nisi Luna\protect\index{Sachverzeichnis}{luna} sit loco verticalis; quia stella\protect\index{Sachverzeichnis}{stella} parallaxi\protect\index{Sachverzeichnis}{parallaxis} caret, luna\protect\index{Sachverzeichnis}{luna} vero habet parallaxin\protect\index{Sachverzeichnis}{parallaxis}; nihilominus tamen postea per parallaxeos\protect\index{Sachverzeichnis}{parallaxis} supputationem, errorem corrigendum esse dicemus.\pend \pstart Quo facto Luna\protect\index{Sachverzeichnis}{luna} in sphaera artificiali eo in loco notetur, ubi tum, supra horizontem elevationem, tum a stella declinationem\protect\index{Sachverzeichnis}{declinatio} habet datam, qui locus propterea
eo ipso fit determinatus, ac proinde \textso{habetur lunae}\protect\index{Sachverzeichnis}{luna}\textso{ latitudo.}\protect\index{Sachverzeichnis}{latitudo}\pend \pstart
Sed quia Parallaxis\protect\index{Sachverzeichnis}{parallaxis} lunae\protect\index{Sachverzeichnis}{luna} turbat, et facit ut locus in sphaera primi mobilis ei non debeat assignari, qui observatione deprehenditur (nisi \edtext{luna}{\lemma{}\Afootnote{luna \textit{ erg.} \textit{ L}}} sit \edtext{loco}{\lemma{}\Afootnote{loco \textit{ erg.} \textit{ L}}} verticalis, ubi parallaxi\protect\index{Sachverzeichnis}{parallaxis} caret) quia ex alio loco spectanti alia loci determinatio 