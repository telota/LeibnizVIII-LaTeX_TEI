      
               
                \begin{ledgroupsized}[r]{120mm}
                \footnotesize 
                \pstart                
                \noindent\textbf{\"{U}berlieferung:}   
                \pend
                \end{ledgroupsized}
            
              
                            \begin{ledgroupsized}[r]{114mm}
                            \footnotesize 
                            \pstart \parindent -6mm
                            \makebox[6mm][l]{\textit{L}}Konzept: LH XXXVII 3 Bl. 132\textendash 135. 2 Bog. 2\textsuperscript{o}. 6 S. zweispaltig auf Bl. 132\textendash 134. Die zwei verbleibenden Seiten N. 49\raisebox{-0.5ex}{\notsotiny 2}. Bl. 132 r\textsuperscript{o} in der Mitte der rechten Spalte eine Zeichnung, die mit fig. 13 \cite{00258}\textit{LSB} VI, 3, S.~43 (\cite{00257}\textit{Propositiones quaedam physicae}) \"{u}bereinstimmt. Die Zeichnung des Rezipienten in der Mitte der rechten Spalte auf Bl. 134~r\textsuperscript{o} hat Leibniz aus \cite{00267}N. 48 \"{u}bernommen. Wie aus der Bezeichnung fig. 4 hervorgeht, war sie urspr\"{u}nglich Teil des St\"{u}cks N. 49\raisebox{-0.5ex}{\notsotiny 2} und wurde sp\"{a}ter als fig. 1 in den vorliegenden Text \"{u}bernommen. In der Nachbarschaft von fig. 1 zwei weitere Zeichnungen. Eine dieser Zeichnungen l\"{a}ßt sich nicht zuordnen. Die zweite wird in N. 51 reproduziert.\\Cc 2, Nr. 491 G tlw. \pend
                            \end{ledgroupsized}
                \vspace*{8mm}
                              \pstart 
                \normalsize
            [132 r\textsuperscript{o}] \selectlanguage{french}  Dans un petit trait\'{e}\edtext{}{\lemma{trait\'{e}}\Bfootnote{\textsc{G. W. Leibniz, }\textit{Propositiones quaedam physicae} (\cite{00257}\textit{LSB} VI, 3 N. 2).}} \edtext{pas encor publi\'{e}}{\lemma{trait\'{e}}\Afootnote{ \textit{ (1) }\ o\`{u} je \textit{ (2) }\ pas encor publi\'{e} \textit{ L}}}, ny même assez poli \edtext{pour l'exposer \`{a} l'hazard de}{\lemma{pour}\Afootnote{ \textit{ (1) }\ est \textit{ (2) }\ meriter la \textit{ (3) }\ essayer \textit{ (4) }\ subir \textit{ (5) }\  estre \`{a} l'\'{e}preuve \textit{ (6) }\ hazarder \textit{ (7) }\ l'exposer \`{a} l'hazard de \textit{ L}}} la censure publique je me suis propos\'{e} de d\'{e}montrer avec un peu plus d'exactitude la pluspart des propositions, que j'avois publiez autresfois en forme de discours; et entre autres cellecy: \textso{Le mouuement} general (dont j'ay parl\'{e} auparavant) \textso{tache de ramasser la matiere heterogene }\edtext{\textso{ou troublante}}{\lemma{\textso{ou }}\Afootnote{\textso{troublante} \textit{ erg.} \textit{ L}}}\textso{, pour en estre moins troubl\'{e}} (en cas que les autres remedes \edtext{dont je viens de parler}{\lemma{}\Afootnote{dont  \textit{ (1) }\ j'ay mention \textit{ (2) }\ je viens de parler \textit{ erg.} \textit{ L}}} \edtext{ny puissent}{\lemma{parler}\Afootnote{ \textit{ (1) }\ ne peuuent \textit{ (2) }\ ny puissent \textit{ L}}} pas reussir) \edtext{dont la demonstration comme je l'avois conce\"{u}e d'abord}{\lemma{dont}\Afootnote{ \textit{ (1) }\ voicy \textit{ (2) }\ la [...] conce\"{u}e  \textbar\ d'abord \textit{ erg.}\ \textbar\  \textit{ L}}} rudement\edtext{}{\lemma{rudement}\Bfootnote{\cite{00258}\textit{LSB} VI, 3, S.~42f.}}, est telle: \edtext{\textso{Soit dans la figure cy joincte} tout l'espace \textit{hegf} plein d'une Matiere dont le mouuement est uniforme \`{a} l'entour du centre \textit{d} dans le cercle \textit{eif} (qui represente l'equateur) et dans ses paralleles et meridiens soyent de plus trois corps \textit{a}, \textit{b}, \textit{c}}{\lemma{telle:}\Afootnote{ \textit{ (1) }\ \textso{Soyent dans la figure cy joincte} trois corps \textit{a}, \textit{b}, \textit{c} \textit{ (2) }\ \textso{Soit} [...] mouuement  \textbar\ est \textit{ erg.}\ \textbar\ uniforme [...] paralleles  \textbar\ et meridiens \textit{ erg.}\ \textbar\ soyent [...] \textit{c} \textit{ L}}} 
                    \edtext{heterogenes, ou d'une matiere grossiere qui trouble ce mouuement}{\lemma{}\Afootnote{heterogenes, [...] mouuement \textit{ erg.} \textit{ L}}} \'{e}galement eloign\'{e}s du centre \textit{d} dans le cercle \textit{b-bb-a-cc-c} lequel ne soit pas parallele au cercle du mouuement, \textit{eif} (afin qu'on ne pense pas, que le mouuement
                        %@ @ @ Dies ist eine Abstandszeile - fuer den Fall, dass mehrere figures hintereinander kommen, ohne dass dazwischen laengerer Text steht. Dies kann zu einer Fahlermeldung fuehren. @ @ @ \\
                        \edtext{general}{\lemma{}\Afootnote{general \textit{ erg.} \textit{ L}}} \edtext{même doive}{\lemma{même}\Afootnote{ \textit{ (1) }\ doivue de la ma \textit{ (2) }\  doive \textit{ L}}} mener l'un de ces corps \`{a} l'autre) et de sorte, que ces corps ne puissent pas aller plus vers le centre, ny vers le pole, ny même estre dissipez, pour troubler moins. \edtext{De ces}{\lemma{}\Afootnote{De ces \textit{ erg.} \textit{ L}}} je dis qu'alors le mouuement general les ramassera ensemble, \edtext{et les transportera}{\lemma{}\Afootnote{et les transportera \textit{ erg.} \textit{ L}}} de la situation \textit{b}, \textit{a}, \textit{c} dans la situation \textit{bb}, \textit{a}, \textit{cc} parce qu'alors le trouble sera moindre. Car supposons qu'il y ait un \edtext{corps}{\lemma{corps}\Afootnote{\textit{ erg.} \textit{ L}}} seulement \textit{a} alors il est manifeste, qu'il y aura quelque trouble, ou difformit\'{e} du mouuement general dans l'espace \textit{hegf} car le mouuement sera plus viste entre \textit{a} et \textit{e}, qu'entre \textit{l} et \edtext{\textit{f} l'espace \textit{ae}   estant plus \'{e}troit \`{a} cause du}{\lemma{\textit{f}}\Afootnote{ \textit{ (1) }\ \`{a} cause que \textit{ (2) }\ l'espace [...] du \textit{ L}}} corps \textit{a} et de plus le mouuement \edtext{alors}{\lemma{}\Afootnote{alors \textit{ erg.} \textit{ L}}} sera plus tard en \textit{ae} qu'en \textit{bh} \edtext{par la même raison, pendant que  nous supposons, qu'il y a seulement}{\lemma{\textit{bh}}\Afootnote{ \textit{ (1) }\ parce qu'il n'y a point de \textit{ (2) }\  par la même raison,  \textit{(a)}\ car nous \textit{(b)}\ pendant [...] seulement \textit{ L}}} un corps en \textit{a} et rien en \textit{b}. Le même arrivera s'il y a seulement un corps en \textit{b} ou en \textit{c} et pas ailleurs. Mais \`{a} present, comme il y a un corps en \textit{a} un autre en \textit{b} et un troisieme en \textit{c} il arrivera necessairement un trouble des troublements mêmes. Car selon le trouble qui provient du corps \textit{a} la plus grande \edtext{tardet\'{e}}{\lemma{grande}\Afootnote{ \textit{ (1) }\ vitesse\protect\index{Sachverzeichnis}{vitesse|textit} \textit{ (2) }\ tardet\'{e} \textit{ L}}} doit estre dans son opposite \textit{l}\hspace{-0.2em}: selon \textit{b} dans son opposite \textit{c}\hspace{-0.2em}: selon \textit{c} dans son opposite \textit{b}. \edtext{Et il y aura des contradictions}{\lemma{\textit{b}.}\Afootnote{ \textit{ (1) }\ Voila donc une co \textit{ (2) }\ Et  \textbar\ ainsi \textit{ gestr.}\ \textbar\  il y aura des contradictions \textit{ L}}}