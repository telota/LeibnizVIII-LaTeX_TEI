\pstart[46~v\textsuperscript{o}] Hac ratione jam progrediente nave\protect\index{Sachverzeichnis}{navis} non potest non aer ingredi et egredi, isque tanto fortius, quanto celerior navis\protect\index{Sachverzeichnis}{navis} motus. Is ergo aeris ingressus mediam inter foramina rotam aliquam subtilem circumaget. \edtext{Quae}{\lemma{circumaget.}\Afootnote{ \textit{ (1) }\ Quae \textit{ (2) }\ Cum \textit{ (3) }\ Quae \textit{ L}}} nave\protect\index{Sachverzeichnis}{navis} stante stabit, movente movebitur, et tanto quidem fortius, quanto fortior motus. Ex hoc jam facilis est applicatio ut intus certo modo in aliquo objecto puncta singulis minutis, tanto magis distantia, quanto motus celerior, designentur. Hoc ita puto fieri posse. Id quod supra movetur vento sit globus ex materia levi, \edtext{sed dura et durabili v.g. subtilitas}{\lemma{}\Afootnote{sed [...] subtilitas \textit{ erg.} \textit{ L}}} circulis quasi parallelis aequatoris\protect\index{Sachverzeichnis}{aequator} circumdatus \edtext{polo\protect\index{Sachverzeichnis}{polus} foraminibus obverso}{\lemma{}\Afootnote{polo\protect\index{Sachverzeichnis}{polus} foraminibus obverso \textit{ erg.} \textit{ L}}} ut \edtext{omnes partes versus}{\lemma{ut}\Afootnote{ \textit{ (1) }\ in \textit{ (2) }\ omnes partes versus \textit{ L}}} ab aere apprehendi possit. Huic globo rotulae ita applicentur, ut tandem inferius per eam in partem navi\protect\index{Sachverzeichnis}{navis} oppositam moveatur tabula aliqua \edtext{solida seu rigida}{\lemma{aliqua}\Afootnote{ \textit{ (1) }\ asseribus vel forma \textit{ (2) }\ solida seu rigida \textit{ L}}} tardiore, tamen quam navis\protect\index{Sachverzeichnis}{navis} motu (alioqui  nimis multae semper tabulae essent adhibendae exacte quidem res fieret sed non sine confusione). Quod ita fiet, si tantum v.g. quater qualibet hora contingat machinae tabulam et promoveat aliquantulum. Quod facile aliqua applicatione fieri potest, ita non quidem tabula machina tamen perpetuo movebitur. \edtext{Hoc}{\lemma{movebitur.}\Afootnote{ \textit{ (1) }\ Sit \textit{ (2) }\ Hoc \textit{ L}}} praeter alios modos ita fieri potest. Si \edtext{Tabula peculiari}{\lemma{Si}\Afootnote{ \textit{ (1) }\ infima rota\protect\index{Sachverzeichnis}{rota|textit} \textit{ (2) }\ Tabula peculiari \textit{ L}}} aliqua machina circumacta tantum 4 qualibet hora sese promovendam offerat. Porro eadem machina quae tabulam circumagit fiat, ut qualibet hora instrumento aliquo \edtext{constanti et cum tabula non moto punctum in tabula designetur}{\lemma{aliquo}\Afootnote{ \textit{ (1) }\ punctum in tabula \textit{ (2) }\ constanti [...] designetur \textit{ L}}}. Ea ratione hora qualibet quantum navis\protect\index{Sachverzeichnis}{navis} confecerit, proportionaliter haberi potest, distantia enim punctorum dabit navis\protect\index{Sachverzeichnis}{navis} celeritatem\protect\index{Sachverzeichnis}{celeritas}. Si velis exactius habere, ut quolibet minuto puncta signentur. Eadem arte facile efficies, idque utile erit, si instrumentum istud non in mari sed aliis mensurationibus adhibeatur. Sed in mari opus esset nimis multis tabulis, et res nimium confunderetur, puto horas sufficere. Posset haec tabula esse \edtext{globus, sed puto tamen omnia in plano accuratius haberi}{\lemma{globus,}\Afootnote{ \textit{ (1) }\ et intus machina in globo quae \textit{ (2) }\ sed [...] haberi \textit{ L}}}. Ea ratione autem pro una septimana una tabula sufficiet in qua per consequens erunt puncta $\begin{array}{r} 24\\7\\\overline{168}\end{array}$. \pend\pstart\noindent%\hfill 
Machina quoque ita dirigi potest ut una tabula completa decidat, et  
nova succedat. Et decidentes tabulae ordine sibi invicem superimponantur. \pend \pstart  Jam unum restat, qua arte et flexus navis\protect\index{Sachverzeichnis}{navis} designari possit. Hic magnes\protect\index{Sachverzeichnis}{magnes}, divinum munus, subsidio venit. Globus \edtext{superior}{\lemma{Globus}\Afootnote{ \textit{ (1) }\ ita \textit{ (2) }\ superior \textit{ L}}}, quem aer movet liber praeterea  sic sit, ut nave\protect\index{Sachverzeichnis}{navis} versa v.g. versus meridiem aut quomodocunque ipse non simul vertatur, sed retineat priorem situm. Ut liber sit facile effici potest. Ut situm priorem retineat, non nisi ope magnetis\protect\index{Sachverzeichnis}{magnes}. Esto igitur globo applicato autem 