      
               
                \begin{ledgroupsized}[r]{120mm}
                \footnotesize 
                \pstart                
                \noindent\textbf{\"{U}berlieferung:}   
                \pend
                \end{ledgroupsized}
            
              
                            \begin{ledgroupsized}[r]{114mm}
                            \footnotesize 
                            \pstart \parindent -6mm
                            \makebox[6mm][l]{\textit{L}}Konzept: LH XXXV 15, 6 Bl. 46, 74. 1 Bog. 2\textsuperscript{o}. 3 1/4 S. zweispaltig. In der linken Spalte von Bl. 46 v\textsuperscript{o} ein Text, der als Nachtrag zur Reihe IV gedruckt wird.  Dieser Text wird in den rechten Spalten von Bl. 74 r\textsuperscript{o} und 74 v\textsuperscript{o} fortgesetzt. Das vorliegende St\"{u}ck endet mit der 8. Zeile von Bl. 74 v\textsuperscript{o}, linke Spalte. Am oberen und unteren Ende der Mittelfalz sowie an den unteren R\"{a}ndern der Seiten Papierabbr\"{u}che, die zu geringf\"{u}gigen Textverlusten f\"{u}hren.\\KK 1, Nr. 193 A \pend
                            \end{ledgroupsized}
                %\normalsize
                \vspace*{5mm}
                \begin{ledgroup}
                \footnotesize 
                \pstart
            \noindent\footnotesize{\textbf{Datierungsgr\"{u}nde}: Von Leibniz auf Blatt LH XXXV 15, 6 Bl. 46 r\textsuperscript{o} datiert.}
                \pend
                \end{ledgroup}
            
                %\vspace*{8mm}
                \newpage
                \pstart 
                \normalsize
            \centering [46 r\textsuperscript{o}] \edtext{Intra finem anni 1668 et initium 1669. Longit. 1}{\lemma{}\Afootnote{Intra finem anni 1668 et initium 1669. Longit. 1 \textit{ erg.} \textit{ L}}}\pend \vspace{1.0ex} \pstart \textso{De longitudinibus}\protect\index{Sachverzeichnis}{longitudo}\textso{ inveniendis} diu multumque laboratum est, tandemque anno 1665. Hugenii\protect\index{Namensregister}{\textso{Huygens} (Hugenius, Vgenius, Hugens, Huguens), Christiaan 1629\textendash 1695} inventum per globulos pendulos celebrari coepit.\edtext{}{\lemma{coepit.}\Bfootnote{\textsc{Chr. Huygens}, \cite{00212}\textit{Kort onderwijs}, Den Haag 1665 (\textit{HO} XVII, S.~199\textendash 237).}} Ejus tota vis in eo consistit. Esto Horologium\protect\index{Sachverzeichnis}{horologium}, quod exacte monstret quanto tempore domo absimus. Ut ita sciamus exacte quae nunc hora sit domi. Deinde observetur quae nunc hora sit hic, ubi nunc sumus. Et ita sciemus praecise quantum quoad longitudinem\protect\index{Sachverzeichnis}{longitudo} domo absimus. Duo igitur requiruntur (1) ut Horologium\protect\index{Sachverzeichnis}{horologium} sit exactum, tale per globos pendulos\protect\index{Sachverzeichnis}{pendulum} optime habetur, qui exactissime confecti nec in momento deficiunt, (2) ut nunc observetur quae hic sit hora. Hoc vero per solem\protect\index{Sachverzeichnis}{sol} vel stellas\protect\index{Sachverzeichnis}{stella} fieri potest. Egregium hoc praeclarumque inventum est, si est quale describitur. Sed ex eo tempore nihil de eo inauditum. Interim absolutum perfectumque non est, pendet enim ex alieno, nempe observatione Horae per solem\protect\index{Sachverzeichnis}{sol} vel stellas\protect\index{Sachverzeichnis}{stella}. Iam vero saepe contingit navem\protect\index{Sachverzeichnis}{navis} multos dies obnubilato coelo nec solem\protect\index{Sachverzeichnis}{sol} nec stellas\protect\index{Sachverzeichnis}{stella} videre. Et ita nec horam exacte observare posse. Eo igitur casu qui saepissimus est, haeremus rursum incerti. Ego cum comperissem haberi Instrumenta quaedam gestatilia quae deambulantis passus numerant, commode ab iis adhibenda, qui vallum civitatis \edtext{obambulant, quibus adjunctus}{\lemma{civitatis}\Afootnote{ \textit{ (1) }\ adjuncto \textit{ (2) }\ obambulant, quibus adjunctus \textit{ L}}} comes suspiciosus quietam numerationem non permittit: ideo venit in mentem, an non haberi machina posset, quae totum navis\protect\index{Sachverzeichnis}{navis} cursum, omnes declinationes\protect\index{Sachverzeichnis}{declinatio}, et quod amplius est etiam cursus celeritatem\protect\index{Sachverzeichnis}{celeritas} nobis repraesentaret. Hoc si haberemus perfecta esset prorsus navigationis pars \edtext{[\pgrk{limenereutik'h}]}{\lemma{\pgrk{liminereutik`h}}\Afootnote{\textit{\ L \"{a}ndert Hrsg.}}}. Hoc ita concepi: deberet ea machina in subjecta charta vel alia materia tot puncta facere, quot minuta prima \edtext{(vel satis, si horam) seu}{\lemma{}\Afootnote{(vel satis, si horam) seu $\frac{1}{60}$ horae \textit{ erg.} \textit{ L}}} $\displaystyle\frac{1}{60}$\rule[-4mm]{0mm}{10mm} horae progressa est navis\protect\index{Sachverzeichnis}{navis}. Minuta tantum proportionaliter (nach dem verj\"{u}ngten masstab) inter se distent, quantum navis\protect\index{Sachverzeichnis}{navis} a loco priori. Eosdem etiam angulos retineant. Ea ratione exhiberi perfecte et celeritas\protect\index{Sachverzeichnis}{celeritas} et vestigium cursus poterit. \edtext{ Situs}{\lemma{poterit.}\Afootnote{ \textit{ (1) }\ Figura \textit{ (2) }\ Situs \textit{ L}}} enim punctorum vestigium, \edtext{distantia}{\lemma{vestigium,}\Afootnote{ \textit{ (1) }\ distantia \textit{ (2) }\ multitudo seu distantia \textit{ (3) }\ distantia \textit{ L}}} celeritatem\protect\index{Sachverzeichnis}{celeritas} cursus dabit. Hoc solo instrumento perpetuo haberetur et longitudo\protect\index{Sachverzeichnis}{longitudo} et latitudo\protect\index{Sachverzeichnis}{latitudo}, et \edtext{praeter Magnetis declinationes nihil amplius dubitari posset}{\lemma{et}\Afootnote{ \textit{ (1) }\ corrigerentur Magnetis\protect\index{Sachverzeichnis}{magnes|textit} quoque declinationes\protect\index{Sachverzeichnis}{declinatio|textit} nec quicquam \textit{ (2) }\ praeter [...] posset \textit{ L}}}. Eadem ratione perfici posset, geographia, et locorum \edtext{[distantiae]}{\lemma{distantias}\Afootnote{\textit{\ L \"{a}ndert Hrsg.}}} situsque perfecte determinari. Idem artificium posset deinde in terra exhiberi, et ea ratione perfecte delineari via quam ivimus cum omnibus anfractibus per horae minuta \edtext{vel horam}{\lemma{}\Afootnote{vel horam \textit{ erg.} \textit{ L}}} durantibus. Eadem arte possent delineari figurae sylvarum, templorum, cryptarum, hoc esset, vere pantometrum plus quam Kircherianum.\edtext{}{\lemma{Kircherianum.}\Bfootnote{\textsc{A. Kircher, }\cite{00067}\textit{Magnes}, Rom 1654, S.~174\textendash176. Vgl. auch \textsc{C. Schott}, \cite{00121}\textit{Pantometrum Kircherianum}, W\"{u}rzburg 1660. }}\pend \pstart Hujus rei modum tandem ali$\langle$--$\rangle$ $\langle$faci$\rangle$le videor reperisse multis praetentatis cogitata. Enim initio an non posset aliquid in ipsa navi\protect\index{Sachverzeichnis}{navis} contrarie motum navis\protect\index{Sachverzeichnis}{navis} tum impetum tum flexum significare. Sed comperi tandem, cum omnia quae in navi\protect\index{Sachverzeichnis}{navis} sint habeant idem cum nave\protect\index{Sachverzeichnis}{navis} centrum gravitatis\protect\index{Sachverzeichnis}{gravitas}, frustraneum hoc esse nec contrarium aliquem nisum sentiri posse. Circumspiciendum igitur erat de machina in navi\protect\index{Sachverzeichnis}{navis} ad aliquid firmum stabileque extrinsecum alliganda. Et quidem primo coelum solque\protect\index{Sachverzeichnis}{sol} in mentem venit, et succurrit repertum ingeniosum Cornelii Drebelii\protect\index{Namensregister}{\textso{Drebbel} (Drebelius, Drebel), Cornelius 1572\textendash 1633}, qui \edtext{organon sponte sonans effecerat, solo haud dubie liquore}{\lemma{qui}\Afootnote{ \textit{ (1) }\ liquore quodam \textit{ (2) }\ organon [...] liquore \textit{ L}\ \hspace{10mm} 22 \textit{ (1) }\ est tum \textit{ (2) }\ Id \textit{ L}}} aliquo, qui solis\protect\index{Sachverzeichnis}{sol} ortu excitatus salutabat quasi orientem Lucem, elegantibus motu suo in organo modis musicis, qua ratione solis\protect\index{Sachverzeichnis}{sol} ortus etiam die Nubiloso, quanquam\footnote{\textit{Interlinear \"{u}ber} quanquam tum: Imo secus} tum sonus debilior erat, haberi poterat.\edtext{}{\lemma{poterat.}\Bfootnote{Vermutl. \textsc{C. Drebbel, }\cite{00041}\textit{Ein kurzer Tractat von der Natur der Elementen}, Hamburg 1619, [S.~17f.]. \hspace{10mm} 22 \hspace{3mm} poem.: \textsc{H. Grotius}, \cite{00054}\textit{In organum motus perpetui}, Leiden 1645, S.~270.}} Id instrumentum Jacobo Regi\protect\index{Namensregister}{\textso{England: Jakob I.} (Rex Jacobus, R. Jac.), K\"{o}nig von England 1603\textendash 1625} a Drebelio\protect\index{Namensregister}{\textso{Drebbel} (Drebelius, Drebel), Cornelius 1572\textendash 1633} donatum\footnote{\textit{Interlinear \"{u}ber} a Drebelio\protect\index{Namensregister}{\textso{Drebbel} (Drebelius, Drebel), Cornelius 1572\textendash 1633} donatum: \edtext{Id}{\lemma{evanescere.}\Afootnote{ \textit{ (1) }\ est tum \textit{ (2) }\ Id \textit{ L}}} carmen H. Grotii\protect\index{Namensregister}{\textso{Grotius,} Hugo 1583\textendash 1654} in poem.} Digbaeus\protect\index{Namensregister}{\textso{Digby} (Digbaeus), Kenelm 1603\textendash 1665} memorat. Eo instrumento posset ortus solis\protect\index{Sachverzeichnis}{sol} et horarum \edtext{[numeri]}{\lemma{numeros}\Afootnote{\textit{\ L \"{a}ndert Hrsg.}}} etiam sine sole\protect\index{Sachverzeichnis}{sol} die quolibet sciri, et esset perfectum Hugenii\protect\index{Namensregister}{\textso{Huygens} (Hugenius, Vgenius, Hugens, Huguens), Christiaan 1629\textendash 1695} inventum. Sed cum ignota sit Drebelii\protect\index{Namensregister}{\textso{Drebbel} (Drebelius, Drebel), Cornelius 1572\textendash 1633} ars flectenda alio vela fuere. Duo jam restabant, quibus alligari machina in nave\protect\index{Sachverzeichnis}{navis} posset: aer et aqua. Et aqua\footnote{\textit{Interlinear \"{u}ber} aqua nimis: Imo contra} nimis crassa et perpetuo instabilis, nec facile recludenda in cancellos, \edtext{mature}{\lemma{cancellos,}\Afootnote{ \textit{ (1) }\ facile \textit{ (2) }\ mature \textit{ L}}} displicuit, solus tentamento \edtext{aer supererat}{\lemma{aer}\Afootnote{ \textit{ (1) }\ restabat \textit{ (2) }\ supererat \textit{ L }}}. Hunc continue venti instabilem fluctuantemque reddunt, qua ratione scopo nostro ineptus futurus videbatur. Sed mihi altius omnia agitanti remedium aliquod sese obtulit: inclaudatur globo machina. Sit utrinque foramen exiguum. Idque ita ut ingressus aeri difficilis, egressus ab alia parte promtus sit, quod fiet si utrinque foramen in tubi formam assurgat. Foramen non sit majus quam acicula faciat. Tubus ipse qua aditus aeri ad foramen flecti contorqueri et ita multis modis emuniri potest, ut non nisi subtilissimum et quasi minimum aeris ad machinam veniat, nullo ventorum fluctuumque extrinsecorum sensu, eo saltem qui ad sensum notabilis et mutationis visu digno scibilis causa esse possit. Flecti etiam in eam partem foramen potest, ut ab extrinsecus oppositis defensum nunquam\hspace{1pt} directe\hspace{1pt} vento\hspace{1pt} occurrat. \hspace{1pt}Et\hspace{1pt} si\hspace{1pt} nondum\hspace{1pt} fidis\hspace{1pt} possunt\hspace{1pt} multiplicari\hspace{1pt} toties mutari foramina, ut non possit tandem non aeris extrinsecus impetus evanescere.\pend