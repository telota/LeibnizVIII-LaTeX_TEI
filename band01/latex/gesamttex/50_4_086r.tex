[86~r\textsuperscript{o}] etiam radii illi congregari debent, inveniatur magnitudo. Quod ut fiat supponatur \textit{K} esse illud punctum quod  longissime ab \textit{N} aut \textit{D} distat, ad quod radius aliquis refractus  tendat, sitque \textit{I} punctum ad quod exterior radius cylindri, hic  per \textit{AB} designatus tendit. Deinde ducta sit \textit{BA}, sitque \textit{IM},  perpendicularis ad axem. Manifestum itaque est, omnes  radios praedicti Cylindri occurrere debere illi Circulo, qui ab \textit{IM} circa axem \textit{DNK} rotata, describitur, circulumque  hunc etiam longe majorem esse quam minimum illud planum,  in quo radii hi congregantur. Ut jam est \textit{KF} ad \textit{FB}, ita \textit{KI} ad \textit{IM}. Cumque \textit{IM} major evadat ex eo quod \textit{KI}  major supponatur eadem tamen remanente \textit{BF}, sequitur \textit{KF} sive \textit{KN} + \textit{NF} esse ad \textit{FB}, ut \textit{KI} + alia quad: lin: ad \textit{IM} + alia quad: lin:\rule[-9mm]{0mm}{9mm}\\
 \renewcommand{\arraystretch}{2.3}
$\begin{array}{cccccccc}
%\rule[0mm]{0mm}{10mm}
 \displaystyle1.&\displaystyle \frac{429}{231} &\hspace{-3mm}+&\hspace{-3mm} \displaystyle\frac{4}{5} &\hspace{-3mm}-&\hspace{-3mm} \displaystyle\frac{3}{5} &\hspace{-3mm}-& \hspace{-3mm}\displaystyle\frac{271}{5,231}\\
%\rule[0mm]{0mm}{10mm}
\displaystyle2.&\displaystyle \frac{429}{231} &\hspace{-3mm}+&\hspace{-3mm} \displaystyle\frac{12}{13} &\hspace{-3mm}-&\hspace{-3mm} \displaystyle\frac{5}{13} &\hspace{-3mm}-&\hspace{-3mm} \displaystyle\frac{277}{13,231}\\
%\rule[0mm]{0mm}{10mm}
\displaystyle3.&\displaystyle \frac{429}{231} &\hspace{-3mm}+&\hspace{-3mm} \displaystyle\frac{24}{25} &\hspace{-3mm}-&\hspace{-3mm} \displaystyle\frac{7}{25} &\hspace{-3mm}-&\hspace{-3mm} \displaystyle\frac{278}{25,231}\\
%\rule[0mm]{0mm}{10mm}
\displaystyle4.&\displaystyle \frac{429}{231} &\hspace{-3mm}+&\hspace{-3mm} \displaystyle\frac{40}{41} &\hspace{-3mm}-&\hspace{-3mm} \displaystyle\frac{9}{41} &\hspace{-3mm}-&\hspace{-3mm} \displaystyle\frac{279}{41,231}\\
%\rule[0mm]{0mm}{10mm}
\displaystyle5.&\displaystyle \frac{429}{231} &\hspace{-3mm}+&\hspace{-3mm} \displaystyle\frac{480}{481} &\hspace{-3mm}-&\hspace{-3mm} \displaystyle\frac{31}{481} &\hspace{-3mm}-&\hspace{-3mm} \displaystyle\frac{279}{481,231}\\
%\rule[0mm]{0mm}{10mm}
\displaystyle6.&\displaystyle \frac{429}{231} &\hspace{-3mm}+&\hspace{-3mm} \displaystyle\frac{1200}{1201} &\hspace{-3mm}-&\hspace{-3mm} \displaystyle\frac{49}{1201} &\hspace{-3mm}-&\hspace{-3mm} \displaystyle\frac{271}{1201,231}
\end{array}\left\{\begin{array}{c}
%\rule[0mm]{0mm}{10mm}
\displaystyle\frac{816}{15345}\\
%\rule[0mm]{0mm}{10mm}
\displaystyle\frac{1385}{108537}\\
%\rule[0mm]{0mm}{10mm}
\displaystyle\frac{1946}{406725}\\
%\rule[0mm]{0mm}{10mm}
\displaystyle\frac{2511}{1099989}\\
%\rule[0mm]{0mm}{10mm}
\displaystyle\frac{8649}{152587149}\\
%\rule[0mm]{0mm}{10mm}
\displaystyle\frac{1361}{951707229}
\end{array}\right\}$\parbox{1.5cm}{quae minor est}$
\left\{\begin{array}{c}
%\rule[0mm]{0mm}{10mm}
\displaystyle\frac{1}{8}\\
%\rule[0mm]{0mm}{10mm}
\displaystyle\frac{1}{78}\\
%\rule[0mm]{0mm}{10mm}
\displaystyle\frac{1}{209}\\
%\rule[0mm]{0mm}{10mm}
\displaystyle\frac{1}{438}\\
%\rule[0mm]{0mm}{10mm}
\displaystyle\frac{1}{17642}\\
%\rule[0mm]{0mm}{10mm}
\displaystyle\frac{1}{69615}
\end{array}\right\}.
$
\advanceline{5}