\pstart 
[101 v\textsuperscript{o}] \textso{Gerick. lib. 5 cap. 4 }\edtext{}{\lemma{4}\Bfootnote{\textsc{O. v. Guericke, }\cite{00055}a.a.O., S.~157.}} Jam motus terrae\protect\index{Sachverzeichnis}{terra} diurnus quam annuus tractu temporis fit tardior, ipsa terra\protect\index{Sachverzeichnis}{terra} scilicet crescente ut alia vegetabilia hinc haud dubie praecessio aequinoctiorum.\pend \pstart Cap. 5.\edtext{}{\lemma{5.}\Bfootnote{\textsc{O. v. Guericke}, \cite{00055}a.a.O., S.~157f.}} \textit{Semper in }\textit{tellure}\protect\index{Sachverzeichnis}{tellus}\textit{ duae sunt intumescentiae a partibus oppositis, tempus vero inter }\edtext{\textit{duas}}{\lemma{}\Afootnote{\textit{duas} \textit{ erg.} \textit{ L}}}\textit{ ejusmodi intumescentias ut et detumescentias est 12 horarum cum }$\displaystyle 24\hspace{-6pt}\raisebox{9pt}{,}\hspace{6pt}\frac{3}{8}\rule[-4mm]{0mm}{10mm}$\textit{ minutis. Duo haec tempora faciunt 24 horas cum }$\displaystyle 48\hspace{-7pt}\raisebox{9pt}{,}\hspace{6pt}\frac{3}{8}\rule[-4mm]{0mm}{10mm}$\textit{ min. quae fere constituunt 25 horas atque ideo singulis diebus intumescentia summa integra fere hora ($\displaystyle50\hspace{-7pt}\raisebox{9pt}{,}$ minutis) tardius ad eundem meridianum}\protect\index{Sachverzeichnis}{meridianus}\textit{ redit.} \edtext{Intumescentia maxima cum luna}{\lemma{\textit{redit.}}\Afootnote{ \textit{ (1) }\ Lunae \protect\index{Sachverzeichnis}{luna|textit} \textit{ (2) }\ Intumescentia maxima cum luna \textit{ L}}} meridianum\protect\index{Sachverzeichnis}{meridianus} occupat supremum vel imum, \edtext{in}{\lemma{imum,}\Afootnote{ \textit{ (1) }\ ut \textit{ (2) }\ in \textit{ L}}} pleniluniis et noviluniis itidem major, item intra Tropicos\protect\index{Sachverzeichnis}{tropicus} quam versus polos\protect\index{Sachverzeichnis}{polus}, item tempore aequinoctiorum. Sed quia simul intumescit ab opposita quoque Lunae\protect\index{Sachverzeichnis}{luna} parte ergo non a sola Luna\protect\index{Sachverzeichnis}{luna}. Ideo advocando telluris\protect\index{Sachverzeichnis}{tellus} circumgyratio ut magni vasis. Motio autem duplex \edtext{mare}{\lemma{duplex}\Afootnote{ \textit{ (1) }\ ob terram\protect\index{Sachverzeichnis}{terra|textit} \textit{ (2) }\ mare \textit{ L}}} bisectum per Americam quando apud nos est intumescentia meridie est apud Antipodes\protect\index{Sachverzeichnis}{antipodes} eodem quoque tempore, scilicet media ipsorum nocte (+ \Denarius~+).\pend 
\pstart \textso{Cap. 7 }\edtext{}{\lemma{\textso{7}}\Bfootnote{\textsc{O. v. Guericke}, \cite{00055}a.a.O., S.~160f.}}\textit{ Solis}\protect\index{Sachverzeichnis}{sol}\textit{ profunditas tempore crepusculi 24 gradus non excedit, sed ad summum }$\displaystyle21\frac{3}{4}\rule[-4mm]{0mm}{10mm}$ \textit{grad.}\edtext{}{\lemma{\textit{grad.}}\Bfootnote{Bei Guericke: $\displaystyle 21\frac{1}{2}$ grad.}}\textit{ terminatur, plerumque tamen 18. nonnunquam 16 tantum est grad.} (+ experimenta instituenda quando crepuscula longiora +) ideo aer sensibilis circiter ex Riccioli\protect\index{Namensregister}{\textso{Riccioli} (Ricciolus), Giovanni Battista 1598\textendash 1671} calculo miliarium Germ. 24.\edtext{}{\lemma{24.}\Bfootnote{\textsc{G. Riccioli, }\cite{00086}\textit{Almagestum novum}, Bologna 1651, S.~39, 63.}}\pend 
\pstart \textso{Cap. 7} Odor maximarum urbium sentitur per 4 miliaria Germanica et odor terrae\protect\index{Sachverzeichnis}{terra} a navigantibus per 6.\pend
 \pstart \textso{Cap. 8 }\edtext{}{\lemma{\textso{8}}\Bfootnote{\textsc{O. v. Guericke}, \cite{00055}a.a.O., S.~161f.}} Relatio Fr\"{o}lichii\protect\index{Namensregister}{\textso{Fr\"{o}lich} (Frolichius, Fr\"{o}lichius), David 1595\textendash 1648} de Carpathio monte\protect\index{Ortsregister}{Karpaten (Mons Carpathius)}\edtext{}{\lemma{de}\Bfootnote{\textsc{D. Fr\"{o}lich, }\cite{00047}a.a.O., S.~268\textendash288.}}, et regionibus aeris (+ adde aliam Morini\protect\index{Namensregister}{\textso{Morin} (Morinus), Jean-Baptiste 1583\textendash 1656} de regionibus subterraneis cujus meminit Gassendus\protect\index{Namensregister}{\textso{Gassendi} (Gassendus), Pierre 1592\textendash 1655} in \edtext{vita}{\lemma{vita}\Bfootnote{\textsc{P. Gassendi, }\cite{00051}\textit{Viri illustris Nicolai Claudii Fabricii de Pereiesc}, Den Haag 1651, S.~174f.}} Peireskii\protect\index{Namensregister}{\textso{Peiresc} (Peireskius), Nicolas Claude Fabri de 1580\textendash 1637}. +) Memorabile sclopetum\protect\index{Sachverzeichnis}{sclopetum} in summo aere explosum nullum pene sonum\protect\index{Sachverzeichnis}{sonus} edidisse, at sonum\protect\index{Sachverzeichnis}{sonus} mox in convalles delatum horrendum dedisse murmur. Vidit nubes sub pedibus, in quibus antea velut in nebulis inerat.\pend \pstart \textso{Cap. 9 }\edtext{}{\lemma{\textso{9}}\Bfootnote{\textsc{O. v. Guericke}, \cite{00055}a.a.O., S.~163.}} Frigiditas non \edtext{in montibus}{\lemma{non}\Afootnote{ \textit{ (1) }\ est in \textit{ (2) }\ in montibus \textit{ L}}} quasi media regio aeris sit frigida, sed a montium effluentiis, terraeque\protect\index{Sachverzeichnis}{terra}. Hinc Zabarella\protect\index{Namensregister}{\textso{Zabarella,} Giacomo 1533\textendash 1589}\edtext{}{\lemma{Hinc}\Bfootnote{\textsc{G. Zabarella, }\cite{00111}\textit{De rebus naturalibus}, K\"{o}ln 1597, S.~554.}} in monte Veneris agri Patavini aestivo tempore eundem qui in fundo expertus calorem\protect\index{Sachverzeichnis}{calor}.\pend \pstart \textso{Gerick. lib. 5 cap. 10 }\edtext{}{\lemma{\textso{10}}\Bfootnote{\textsc{O. v. Guericke}, \cite{00055}a.a.O., S.~164f.}} Quando aer plus aquei humoris recepit tunc se magis extendit majusque spatium occupat: quo vero magis humore \edtext{constituitur,}{\lemma{constituitur,}\Bfootnote{Bei Guericke: destituitur}} eo magis se contrahit seu minus spatium occupat. Unde semper post pluviam \textit{astra}\protect\index{Sachverzeichnis}{astrum}\textit{ apparent clariora et minora, minoresque habent }\textit{refractiones}\protect\index{Sachverzeichnis}{refractio}\textit{ et id tempus ad observanda sidera longe aptissimum.} Hinc sidera nobis ob aerem apparent ab horizonte elevatiora, quam revera sunt; imo videri esse supra horizontem cum sunt infra. \textit{Hinc omnia eorum puncta visibilia immutantur, elevationes, ascensiones, descensiones, longitudines, latitudines, declinationes, ipsa aequinoctia et solstitia.} Hinc diametri apparentes quoque in horizonte majores \edtext{cernuntur, quanto}{\lemma{}\Afootnote{cernuntur, \textbar\ et \textit{ gestr.}\ \textbar\ quanto \textit{ L}}} elevatiora magis restringuntur; hinc et in horizonte soli\protect\index{Sachverzeichnis}{sol} et lunae\protect\index{Sachverzeichnis}{luna} Elliptica vel ovalis forma, \textit{ita ut diameter horizontalis sit $30\hspace{-6pt}\raisebox{9pt}{,}$ verticalis
 $\vspace{1cm}36\hspace{-6pt}\raisebox{9pt}{,}$ minutorum. Hinc astra propter aeris sphaeram semper nonnihil elevantur a vero suo loco. Ideo }\textit{Tycho}\protect\index{Namensregister}{\textso{Brahe} (Tycho), Tycho 1546\textendash 1601}\textit{ duas construxit refractionum tabulas unam pro }\textit{planetis}\protect\index{Sachverzeichnis}{planeta}\textit{ et }\textit{fixis}\protect\index{Sachverzeichnis}{stella!fixa}\textit{ alteram pro sole, quam et }\textit{Lunae}\protect\index{Sachverzeichnis}{luna}\textit{ quodammodo applicari posse censet. In }\textit{sole}\protect\index{Sachverzeichnis}{sol}\textit{ ad 45 grad. in }\textit{fixis}\protect\index{Sachverzeichnis}{stella!fixa}\textit{ ad gradus altitudinis 20 has }\textit{refractiones}\protect\index{Sachverzeichnis}{refractio}\textit{ exhibet. Si sublimiores sunt }\textit{stellae}\protect\index{Sachverzeichnis}{stella}\textit{ superfluum putavit aestimare }\textit{refractiones}\protect\index{Sachverzeichnis}{refractio}\textit{ quia tunc radii }\textit{stellarum}\protect\index{Sachverzeichnis}{stella}\textit{ vapores rectius penetrent. }\textit{Fixae}\protect\index{Sachverzeichnis}{stella!fixa}\textit{ quomodocumque nonnihil amittunt de }\textit{refractione}\protect\index{Sachverzeichnis}{refractio}\textit{ }\textit{solis}\protect\index{Sachverzeichnis}{sol}\textit{ tam in horizonte quam aliis altitudinibus, quia de nocte et praesertim hyeme observantur, }\textit{sol}\protect\index{Sachverzeichnis}{sol}\textit{ autem de die et aestate, tamen secundum leges opticas }\textit{refractio}\protect\index{Sachverzeichnis}{refractio}\textit{ }\textit{fixarum}\protect\index{Sachverzeichnis}{stella!fixa}\textit{ debet esse eadem quae }\textit{solis}\protect\index{Sachverzeichnis}{sol}\textit{, et non tantum in elevatione ad 20 vel 45 grad. sed et in omni elevatione} (+ poli\protect\index{Sachverzeichnis}{polus} puto +) \textit{quia ubivis gentium }\textit{aeris sphaera}\protect\index{Sachverzeichnis}{sphaera!aeris}\textit{ est consequenter }\textit{refractio}\protect\index{Sachverzeichnis}{refractio}\textit{, quod et }\textit{Tycho}\protect\index{Namensregister}{\textso{Brahe} (Tycho), Tycho 1546\textendash 1601}\textit{ ipse anno 1600 cum esset in }\textit{Bohemia}\protect\index{Ortsregister}{Bohmen@B\"{o}hmen (Bohemia)}\textit{ et observationes continuaret expertus est; invenit enim altitudinem Spicae Virginis}\protect\index{Sachverzeichnis}{Virgo}\textit{ sub 30 grad. 1. scrupulum proxime refringi.} Sed et propter variationem altitudinis et ideo sphaerae aeris\protect\index{Sachverzeichnis}{sphaera!aeris}, virtute densitatis refractiones\protect\index{Sachverzeichnis}{refractio} quotidie variant. In horizonte quoque sidus per copiosiorem aerem videtur, id est quod non tantum super sed et circa nos est. Hinc major species praesertim quia aer terrae\protect\index{Sachverzeichnis}{terra} proprior compressior. 
 \pend