\pend \pstart [p.~184] XVII. Hinc penultinus Cometes\protect\index{Sachverzeichnis}{cometa} altissimus censendus esset, si nulla omnino cauda\protect\index{Sachverzeichnis}{cauda cometae} visa esset in ipsa oppositione; vel circa illam; visa autem fuit die scilicet 28. decembris a P. Ignatio Regis\protect\index{Namensregister}{\textso{Regis,} Ignatio SJ ?\textendash 1651 oder 1669}, versus Boream proiecta,  quo die, cometes\protect\index{Sachverzeichnis}{cometa} fuit soli oppositus, saltem proxime;  quod certe ad praesens institutum sufficit; vnde tanta forte altitudo cometae\protect\index{Sachverzeichnis}{cometa} non fuit, quantam aliqui praedicant;\footnote{\textit{Leibniz unterstreicht}: vnde tanta [...] praedicant} [...].\selectlanguage{latin}