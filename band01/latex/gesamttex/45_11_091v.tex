 \pstart[91 v\textsuperscript{o}] \textso{Gerick. lib. 6. cap. 12}\edtext{}{\lemma{\textso{12}}\Bfootnote{\textsc{O. v. Guericke, }\cite{00055}a.a.O., S.~211f.}} \textit{Quanto majora  corpora tanto tardior eorum vertigo, et  contra.} Unde Jupiter\protect\index{Sachverzeichnis}{Jupiter} se citius vertit Saturno\protect\index{Sachverzeichnis}{Saturnus} Mars\protect\index{Sachverzeichnis}{Mars} Jove\protect\index{Sachverzeichnis}{Jovis}, terra\protect\index{Sachverzeichnis}{terra} Marte\protect\index{Sachverzeichnis}{Mars}. Sociae motum  vertiginis non habent.\pend \pstart \textso{Cap. 13.}\edtext{}{\lemma{\textso{13.}}\Bfootnote{\textsc{O. v. Guericke}, \cite{00055}a.a.O., S.~212.}} De planetarum\protect\index{Sachverzeichnis}{planeta}  distantiis potest quodammodo ex oscillationibus\protect\index{Sachverzeichnis}{oscillatio}  perpendiculorum judicari. \textit{Cum autem duorum  perpendiculorum altitudo minor ad majorem  se habeat ut quadratum vibrationum minoris  altitudinis ad quadratum vibrationum majoris} (+ dubito +) \textit{altitudinis.} Et latio periodica  telluris circa solem\protect\index{Sachverzeichnis}{sol} fit uno anno, periodi autem semidiameter est 2644 semidiam. terrae\protect\index{Sachverzeichnis}{terra}, quarum quadratum 6990736. Jam Saturni\protect\index{Sachverzeichnis}{Saturnus} fit $\displaystyle29\frac{1}{2}\rule[-4mm]{0mm}{10mm}$ annis ergo per  regulam \edtext{auream. Ut}{\lemma{auream.}\Afootnote{ \textit{ (1) }\ Ergo \textit{ (2) }\ Ut \textit{ L}}} Latio unius  anni ad quadratum distantiae terrae\protect\index{Sachverzeichnis}{terra} a sole\protect\index{Sachverzeichnis}{sol}, ita latio $\displaystyle29\frac{1}{2}\rule[-4mm]{0mm}{10mm}$. annorum ad  quadratum distantiae \saturn\textsuperscript{ni}. (+ Ego calculum  istum minime justum puto, sunt enim  altitudines, ut quadrata vibrationum, non  vibrationes, ut quadrata altitudinum; et  vero, aliud vibratio aliud gyratio.  Erit sic potius calculandum periodi  sunt ut 1. ad $\displaystyle29\frac{1}{2}\rule[-4mm]{0mm}{10mm}$ seu $\displaystyle\frac{59}{2}\rule[-4mm]{0mm}{10mm}$.  Ergo distantiae, ut 1 ad $\displaystyle\frac{59^\smallfrown 59}{2^\smallfrown 2}=\frac{3481}{4}$\footnote{\textit{Nebenrechnung zu} $\displaystyle\frac{3481}{4}$:\newline$\protect\begin{array}{r}59\\59\\\overline{531}\\295\protect\rule[0pt]{5.5pt}{0pt}\\\overline{3481}
\protect\end{array}$}  vel ut 1. ad
%\pend\clearpage\pstart\noindent
$\protect\begin{array}{ccc}\cancel{2}&&\\\cancel{3}\cancel{4}\cancel{8}\cancel{1}&f&870.\\\cancel{4}\cancel{4}&&\end{array}$  Ergo Saturnus\protect\index{Sachverzeichnis}{Saturnus} abesset tantum  a sole\protect\index{Sachverzeichnis}{sol} 870 vicibus amplius  quam terra\protect\index{Sachverzeichnis}{terra} a sole\protect\index{Sachverzeichnis}{sol}, et Jupiter\protect\index{Sachverzeichnis}{Jupiter} 144  vicibus, ergo multipliciter distantia terrae\protect\index{Sachverzeichnis}{terra} 2644. per 870 habes distantiam Saturni\protect\index{Sachverzeichnis}{Saturnus} per 144 habes Jovis\protect\index{Sachverzeichnis}{Jovis}, si  huic methodo insistendum est. Sed Autor  aliam sequitur simul quadratum  distantiae terrae\protect\index{Sachverzeichnis}{terra} et multiplicat per  periodum, inde extrahit radicem, Saturnus\protect\index{Sachverzeichnis}{Saturnus} ei distat a sole\protect\index{Sachverzeichnis}{sol} \edtext{semidiametris}{\lemma{sole}\Afootnote{ \textit{ (1) }\ vicibus \textit{ (2) }\ semidiametris \textit{ L}}} terrae\protect\index{Sachverzeichnis}{terra} 14360. Jupiter\protect\index{Sachverzeichnis}{Jupiter} [9159]\edtext{}{\Afootnote{9195\textit{\ L \"{a}ndert Hrsg. } }}.\edtext{}{\lemma{[9159].}\Bfootnote{Korrektur nach \textsc{O. v. Guericke}, \cite{00055}a.a.O., S.~213.}} Mars\protect\index{Sachverzeichnis}{Mars} 3739. Terra\protect\index{Sachverzeichnis}{terra} 2644. Venus\protect\index{Sachverzeichnis}{Venus} 2049. Mercurius\protect\index{Sachverzeichnis}{Mercurius} 1296.  Ego aliter Saturnus\protect\index{Sachverzeichnis}{Saturnus} $2644^\smallfrown 870.\hspace{5.5pt}f.\hspace{5.5pt}2300280$.\footnote{\textit{Nebenrechnung zu} 2300280:\newline$\protect\begin{array}{l}\hspace{11pt}2644\\\hspace{22pt}870\\\hspace{5.5pt}\overline{185080}\\21152\\\overline{2300280}
 \protect\end{array}$} sem. terrae\protect\index{Sachverzeichnis}{terra}. Jupiter\protect\index{Sachverzeichnis}{Jupiter} 12 annis minus 50  diebus 
 12\hspace{5.5pt}$\displaystyle\frac{50}{365}$\hspace{2.25pt}\vrule\hspace{2.25pt}$\displaystyle\frac{10}{73}$\hspace{2.25pt}\vrule\hspace{2.25pt}$\displaystyle\frac{1}{7}\rule[-4mm]{0mm}{10mm}$
\newline%\rule[-4mm]{0mm}{40mm}
%\protect\newpage
% $\begin{array}{rrrrr}
%                    & \cancel{3}2\hspace{16.5pt}&&&\\
%                     &\cancel{7}\cancel{8}\hspace{16.5pt}&&&\\
%                     &\cancel{2}\cancel{3}\cancel{6}2&&2644&\\
%                   \displaystyle \frac{85}{7}^\smallfrown\frac{85}{7}[\sqcap]\displaystyle\frac{7225}{49} 
%                                     f 147  
%           &\cancel{7}\cancel{2}\cancel{2}\cancel{5}&f&147&\\
%                     &\cancel{4}\cancel{9}\cancel{9}\cancel{9}&&\overline{\hspace{5.5pt}18508}&f. 388668\\
%                     &\cancel{4}\cancel{4}\hspace{5.5pt}&&10576\hspace{5.5pt}&\\
%                     &&&2644\hspace{11pt}&\\
%                     &&&\overline{388668}&
%                     \end{array}$

\edtext{}{\lemma{388668}\Bfootnote{\selectlanguage{ngerman}Das richtige Ergebnis muss 372804 Radien der Erde lauten. Leibniz' Fehler beruht darauf, dass er eine Zeile zuvor 50 Tage addiert, anstatt sie zu subtrahieren. Eine offensicht\-liche Flüchtigkeit bei der Multiplikation, die zu 288668 Halbmessern als Produkt führte, wurde stillschweigend korrigiert.\selectlanguage{latin}}}
\raisebox{2,2mm}{$ \displaystyle \frac{85}{7}^\smallfrown\frac{85}{7}[\sqcap]\displaystyle\frac{7225}{49}$ \protect\footnote{\textit{Nebenrechnung zu} $\displaystyle\frac{7225}{49}$:\newline 
  $\protect\begin{array}{l}\hspace{11pt}85\\\hspace{11pt}85\\\overline{\hspace{5.5pt}425}\\680\hspace{16.5pt}\\7225                    \protect\end{array}$}   
               $       f\hspace{5.5pt}147  $}
$\begin{array}{lllr}             
\hspace{5.5pt}\cancel{3}2&&&\\
\hspace{5.5pt}\cancel{7}\cancel{8}&&&\\
\cancel{2}\cancel{3}\cancel{6}2&&\hspace{11pt}2644&\\
\cancel{7}\cancel{2}\cancel{2}\cancel{5}&f&\hspace{16.5pt}147&f.\hspace{5.5pt}388668\text{ semidiam. terrae}\rule[0mm]{5mm}{0mm}\\
\cancel{4}\cancel{9}\cancel{9}\cancel{9}&&\overline{\hspace{5.5pt}18508}&\\
\hspace{5.5pt}\cancel{4}\cancel{4}&&10576&\\
  &&2644&\\
  &&\overline{388668}&
 \end{array}$  
%
%\newline
\protect\index{Sachverzeichnis}{terra}pro Jovis\protect\index{Sachverzeichnis}{Jovis} dist. a sole\protect\index{Sachverzeichnis}{sol}. Martis\protect\index{Sachverzeichnis}{Mars} \edtext{periodus}{\lemma{Martis}\Afootnote{ \textit{ (1) }\ distantia a s \textit{ (2) }\ periodus \textit{ L}}} est 2 annis fere  seu 98 septimanarum, ponatur esse 2  annorum ergo periodus ejus 
                              $\begin{array}{r}2644\\4\\\overline{10576}
                                       \end{array}$
 sem. ter.\protect\index{Sachverzeichnis}{terra}\rule[-2cm]{0cm}{0cm} Venus\protect\index{Sachverzeichnis}{Venus} lationem absolvit 32 septimanis  seu $\displaystyle\frac{32}{52}$ anni $\displaystyle\frac{16}{26}$ vel $\displaystyle \frac{1}{6}-\frac{3}{26}$ Mercurius\protect\index{Sachverzeichnis}{Mercurius} septimanis $\displaystyle12\frac{1}{2}\hspace{11pt}\frac{25}{102}\hspace{2.25pt}\vrule\hspace{2.25pt}\frac{1}{4}\rule[-4mm]{0mm}{10mm}$. fere anni. Sed mihi tota haec  hypothesis non satis benefundata videtur.\edtext{}{\lemma{}\Afootnote{videtur.  \textbar\ Marte\protect\index{Sachverzeichnis}{Mars} \textit{ gestr.}\ \textbar\ +)\textit{ L}}} +)
 \pend
 \pstart \textso{Cap. 14 }\edtext{}{\lemma{\textso{14}}\Bfootnote{\textsc{O. v. Guericke}, \cite{00055}a.a.O., S.~214.}} Putat Gerickius\protect\index{Namensregister}{\textso{Guericke} (Gerickius, Gerick.), Otto v. 1602\textendash 1686}  corpora planetarum\protect\index{Sachverzeichnis}{planeta} esse ut periodos, Saturnum\protect\index{Sachverzeichnis}{Saturnus} esse terra\protect\index{Sachverzeichnis}{terra} majorem vicibus $\rule[-4mm]{0mm}{10mm}\displaystyle29\frac{1}{2}$ Jovem\protect\index{Sachverzeichnis}{Jovis} 12. Martem\protect\index{Sachverzeichnis}{Mars} 2. Venerem\protect\index{Sachverzeichnis}{Venus}  minorem $\displaystyle\frac{2}{3}\rule[-4mm]{0mm}{10mm}$ Mercurium\protect\index{Sachverzeichnis}{Mercurius} $\displaystyle\frac{1}{2}\rule[-4mm]{0mm}{10mm}$. \pend \pstart \textso{Cap. 15.}\edtext{}{\lemma{\textso{15.}}\Bfootnote{\textsc{O. v. Guericke}, \cite{00055}a.a.O., S.~214.}} Rheita\protect\index{Namensregister}{\textso{Roberval} (Robervallius), Gilles Personne de 1602\textendash 1675} huc valde  inclinat, \textit{Oculo} lib. 4. c. 1. fol. 179,\edtext{}{\lemma{179,}\Bfootnote{\textsc{A. M. Schyrl, }\cite{00129}a.a.O., S.~179.}}  ubi et refert Cusani\protect\index{Namensregister}{\textso{Cusanus,} Nicolaus 1401\textendash 1464} locum%\edtext{}{\lemma{locum}\Bfootnote{\textsc{Referenz????}\cite{00128}}}
, saltem  in Jove\protect\index{Sachverzeichnis}{Jovis} esse incolas.\pend \pstart  Cap. 16.\edtext{}{\lemma{16.}\Bfootnote{\textsc{O. v. Guericke}, \cite{00055}a.a.O., S.~218.}} Scripturam sacram  loqui ad captum vulgi exemplum 1. Reg. cap. 7  vers. 23\edtext{}{\lemma{23}\Bfootnote{1 K\"{o} 7,23}} ubi laci fusi dicitur diameter 10  cuborum, et filum 30. quasi diametri  ad circumferentiam proportio sit ut 1. ad  3. quae tamen non nisi popularis et  mechanica. Ita et de motu terrae\protect\index{Sachverzeichnis}{motus!terrae}.\pend \pstart  Cap. 16. Inexplicabile, quomodo ex Tychonica Hypothesi  quae circa solem feruntur terram\protect\index{Sachverzeichnis}{terra} tam  exiguam secum non capiant circa eundem solem\protect\index{Sachverzeichnis}{sol}. Tellus juxta Tychonicos non est  in centro solis motus\protect\index{Sachverzeichnis}{motus!solis}, modo enim is apogaeus modo perigaeus, nullum ergo centrum  habet sol motus\protect\index{Sachverzeichnis}{motus!solis} sui. Ex quolibet planeta\protect\index{Sachverzeichnis}{planeta} posset systema fabricari quale Tychonicum,  et cujuslibet planetae\protect\index{Sachverzeichnis}{planeta} incolae possent  idem dicere imo Saturnus\protect\index{Sachverzeichnis}{Saturnus} concinnissime pro centro locaretur, posito caetera  omnia circa ipsum ferri.\pend \pstart \textso{Gerick. lib. 7. De stell. fix. c. 3.}\edtext{}{\lemma{\textso{3.}}\Bfootnote{\textsc{O. v. Guericke}, \cite{00055}a.a.O., S.~230.}} Aristarchus\protect\index{Namensregister}{\textso{Aristarch v. Samos,} um 230 v. Chr.} in Gallia\protect\index{Ortsregister}{Frankreich (Gallia, Francia)} redivivus apud Mersennum\protect\index{Namensregister}{\textso{Mersenne} (Mersennus), Marin 1588\textendash 1648} tom. 3. \textit{Novarum obs. physico-math.} pag. 8.\edtext{}{\lemma{8.}\Bfootnote{\textsc{Aristarch von Samos}, \cite{00211}a.a.O., S. 8.}} quamlibet  fixam videri caput alicujus systematis ut solem\protect\index{Sachverzeichnis}{sol}  nostri.\pend 