      
               
                \begin{ledgroupsized}[r]{120mm}
                \footnotesize 
                \pstart                
                \noindent\textbf{\"{U}berlieferung:}   
                \pend
                \end{ledgroupsized}
            
              
                            \begin{ledgroupsized}[r]{114mm}
                            \footnotesize 
                            \pstart \parindent -6mm
                            \makebox[6mm][l]{\textit{L}}Exzerpt: LH XXXVII 2 Bl. 11. 1 Bl. rechteckig beschnitten, 14 x 6 cm. 4/5 S., R\"{u}ckseite leer.\\Kein Eintrag in KK 1 oder Cc 2. \pend
                            \end{ledgroupsized}
                %\normalsize
                \vspace*{5mm}
                \begin{ledgroup}
                \footnotesize 
                \pstart
            \noindent\footnotesize{\textbf{Datierungsgr\"{u}nde}: Wir ordnen dieses St\"{u}ck in das Textcorpus der fr\"{u}hen Auseinandersetzung mit der Cartesischen \cite{00038}\textit{La dioptrique} ein und \"{u}bernehmen die Datierung aus N. 21.}
                \pend
                \end{ledgroup}
            
                \vspace*{8mm}
                \pstart 
                \normalsize
            [11 r\textsuperscript{o}] Cartesius\protect\index{Namensregister}{\textso{Descartes} (Cartesius, des Cartes, Cartes.), Ren\'{e} 1596\textendash 1650} quidem in suis \textit{Dioptricis} \edtext{artic. 22 asserit}{\lemma{\textit{Dioptricis}}\Afootnote{ \textit{ (1) }\ ostendit \textit{ (2) }\ artic. 22 asserit \textit{ L}}} \textit{speculum conburens cujus diameter non multo major est centesima circiter parte distantiae quae }\edtext{\textit{est}}{\lemma{}\Afootnote{\textit{est} \textit{ erg.} \textit{ L}}}\textit{ inter illum locum in quo radios solis colligere debet, id est cujus eadem sit ratio ad hanc distantiam quae diametri solis ad eam quae inter nos et solem licet angeli manu expoliatur non magis calefaciet illum locum in quo radios quam maxime colliget, quam illi radii qui ex nullo speculo reflexi directe a sole manant.}\edtext{}{\lemma{\textit{manant}.}\Bfootnote{\textsc{R. Descartes, }\cite{00037}\textit{Specimina philosophiae}, Amsterdam 1650, Teil 2, S.~168. }} Memini me aliquando rem considerare \edtext{et similes quasdam proportiones prodiisse, visam tamen rem}{\lemma{considerare}\Afootnote{ \textit{ (1) }\ et visam rem \textit{ (2) }\ et [...] rem \textit{ L}}} paulo latius patere.\pend 