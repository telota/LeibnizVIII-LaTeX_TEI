[133 r\textsuperscript{o}] mais comme nous n'avons point de vuide\protect\index{Sachverzeichnis}{vide} parfait, et comme il y a tousjour de l'air quoyque \edtext{extremement}{\lemma{quoyque}\Afootnote{ \textit{ (1) }\ tres \textit{ (2) }\ extremement \textit{ L}}} rarifi\'{e} entre deux: il arrivera ce que nous voyons arriver quand la liqueur purg\'{e}e\protect\index{Sachverzeichnis}{liqueur!purg\'{e}e} tombe \`{a} cause de l'interposition d'une petite bulle, car cette petite bulle \edtext{qui est \`{a} peine sensible}{\lemma{}\Afootnote{qui est \`{a} peine sensible \textit{ erg.} \textit{ L}}} \edtext{estant oblig\'{e}e de remplir tout le Matras quand la liqueur tombe, se dilate extremement}{\lemma{sensible}\Afootnote{ \textit{ (1) }\ se rari \textit{ (2) }\  dilate extremement  \textit{(a)}\ quan \textit{(b)}\ \`{a} cause qu'elle remplit \textit{ (3) }\ estant [...] extremement \textit{ L}}} et peut estre autant que l'air qui reste dans le Recipient, et neantmoins elle fait cesser l'effort \edtext{que la nature fait}{\lemma{l'effort}\Afootnote{ \textit{ (1) }\ \`{a} \textit{ (2) }\ de la nature \`{a} la join \textit{ (3) }\ que la nature fait \textit{ L}}} contre la discontinuation des corps: donc l'air \edtext{qui reste}{\lemma{}\Afootnote{qui reste \textit{ erg.} \textit{ L}}} dans le recipient fera le même. \edtext{\`{A} cause qu'il n'y a}{\lemma{même.}\Afootnote{ \textit{ (1) }\ Cela \textit{ (2) }\  La Raison est, parce qu'il n'y \textit{ (3) }\ \`{A} cause qu'il n'y a \textit{ L}}} point de discontinuation des corps\protect\index{Sachverzeichnis}{corps!sensible} grossiers ou sensibles, \edtext{quand la liqueur se separe du verre.}{\lemma{sensibles,}\Afootnote{ \textit{ (1) }\ par \textit{ (2) }\ quand [...] verre. \textit{ L}}} L'air quoyque dilat\'{e} estant entre deux. \edtext{(Mais il me semble, qu'il y reste, quelque difficult\'{e} car supposons tout plein d'air même ordinaire neantmoins le mouuement general se trouuera mieux, en joignant deux corps grossiers, que l'air seulement.)}{\lemma{}\Afootnote{(Mais [...] en \textit{ (1) }\ separant \textit{ (2) }\ joignant deux [...] seulement.) \textit{ erg.} \textit{ L}}}\pend
\clearpage
 \pstart \textso{Object. 3.} Mais \edtext{comment pourra-t-on}{\lemma{Mais}\Afootnote{ \textit{ (1) }\ on ne pourra pas \textit{ (2) }\ comment pourra-t-on \textit{ L}}} rendre raison par cette Hypothese, pourquoy un grand choc, donn\'{e} contre le Tuyau fait tomber la liqueur, \edtext{phaenom. 5.}{\lemma{}\Afootnote{phaenom. 5. \textit{ erg.} \textit{ L}}} La Response est facile, s\c{c}avoir qu'il est tres ais\'{e} d'en rendre raison, parce que \edtext{le choc produit}{\lemma{que}\Afootnote{ \textit{ (1) }\ par le choc\protect\index{Sachverzeichnis}{choc|textit} il s'engendre \textit{ (2) }\ le choc produit \textit{ L}}} des bulles d'air.\pend 
 \pstart \textso{Object. 4.} Pourquoy est il donc plus difficile de separer la liqueur purg\'{e}e\protect\index{Sachverzeichnis}{liqueur!purg\'{e}e}, du verre, par le choc\protect\index{Sachverzeichnis}{choc}, quand elle a est\'{e} longtemps en repos \edtext{selon le phenomene 8.}{\lemma{}\Afootnote{selon le phenomene 8. \textit{ erg.} \textit{ L}}}? Pour r\'{e}ponse on peut dire que la liqueur \edtext{ayant est\'{e} longtemps}{\lemma{liqueur}\Afootnote{ \textit{ (1) }\ estant \textit{ (2) }\ ayant est\'{e} longtemps \textit{ L}}} fort contrainte et ayant quitt\'{e} la facilit\'{e} de produire des bulles, veu qu'il y a long-temps qu'elle n'en a pas produit, il faut de la force pour luy rendre cette disposition par le choc\protect\index{Sachverzeichnis}{choc}.\pend
  \pstart \textso{Object. 5.} S'il y a une pression comme nous venons de dire, il seroit même difficile de faire glisser une placque sur l'autre \edtext{dans le vuide\protect\index{Sachverzeichnis}{vide}}{\lemma{}\Afootnote{dans le vuide\protect\index{Sachverzeichnis}{vide} \textit{ erg.} \textit{ L}}}, comme si l'on les pressoit l'une contre l'autre entre deux doigts, \edtext{on ne pourroit pas les remuer ais\'{e}ment}{\lemma{}\Afootnote{on ne pourroit pas   \textbar\ les \textit{ erg.}\ \textbar\  remuer ais\'{e}ment \textit{ erg.} \textit{ L}}}. Cette objection est generalle, contre toutes les pressions par lesquelles on est accoustum\'{e} \edtext{d'expliquer les phenomenes que les anciens expliquoient par le vuide}{\lemma{accoustum\'{e}}\Afootnote{ \textit{ (1) }\ d'appliquer \`{a} ce phenomene soit qu'on a \textit{ (2) }\ d'expliquer [...] vuide \textit{ L}}} car on demande pourquoy le poids de l'air\protect\index{Sachverzeichnis}{poids!de l'air} ne nous incommode pas, et pourquoy un plongeur n'est pas \'{e}cras\'{e} entre l'eau et le fonds. Et il faut seulement appliquer icy la response des Messieurs \edtext{Pascal\protect\index{Namensregister}{\textso{Pascal} (Pascalius), Blaise 1623\textendash 1662}}{\lemma{Pascal}\Bfootnote{\textsc{B. Pascal}, \cite{00081}\textit{Traitez de l'équilibre des liqueurs}, Paris 1663, S. 38\textendash44 (\textit{PO} III, S.~167\textendash192).}}, 
  \edtext{Boyle\protect\index{Namensregister}{\textso{Boyle} (Boylius, Boyl., Boyl), Robert 1627\textendash 1691}}{\lemma{Boyle}\Bfootnote{\textsc{R. Boyle}, \cite{00015}\textit{New experiments physico-mechanicall}, Oxford 1660, S. 3f. (\textit{BW} I, S.~158).}}, \edtext{Guericke\protect\index{Namensregister}{\textso{Guericke} (Gerickius, Gerick.), Otto v. 1602\textendash 1686}}{\lemma{Guericke}\Bfootnote{\textsc{O. v. Guericke}, \cite{00055}\textit{Experimenta nova}, Amsterdam 1672, S.~72.}} par laquelle ils ont sauu\'{e} la pression de l'atmosphere\protect\index{Sachverzeichnis}{atmosph\`{e}re} \edtext{et l'equilibre des liqueurs}{\lemma{}\Afootnote{et l'equilibre des liqueurs \textit{ erg.} \textit{ L}}} contre cette sorte \edlabel{objectionsstart}\edtext{d'objections.}{\lemma{d'objections.}\xxref{objectionsstart}{objectionsend}\Afootnote{ \textit{ (1) }\ \textso{Object. 5.} Il reste une objection tres difficile, \textit{(a)}\ de la liqu \textit{(b)}\ du \textit{(c)}\ de l'experience \textit{ (2) }\ \textso{Object. 6.} \textit{ L}}} 
\pend 