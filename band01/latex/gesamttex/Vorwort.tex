\thispagestyle{empty}
{\vrule height 0mm depth 30mm width 0mm}
\begin{flushright}
Habent sua fata manuscripta Leibnitiana\\(frei nach Terentianus Maurus)
\end{flushright}
\par
\vspace{2.0ex}
Am 1. August 1976 nahmen Walter S. Contro und ich die Arbeiten an der Reihe VII Mathematische Schriften der Leibniz-Edition auf. Fast genau zwanzig Jahre sp\"{a}ter, am 29. August 1996, zeitgleich mit meiner Wahl zum ordentlichen Mitglied der Berlin-Brandenburgischen Akademie der Wissenschaften (BBAW), f\"{u}hrten der damalige Pr\"{a}sident der BBAW, Dieter Simon, und ich ein erstes Gespr\"{a}ch \"{u}ber die M\"{o}glichkeit, Reihe VIII Naturwissenschaftliche, medizinische, tech\-nische Schriften an der BBAW zu verwirklichen. Wir verabredeten, dass ich bis Ende 1996 einen vorl\"{a}ufigen Projektplan ausarbeitete. Der vorliegende erste Band der Reihe VIII gibt Anlass, dankbar in gebotener K\"{u}rze den Weg nachzuzeichnen, der von Dieter Simons Initiative zu diesem Ergebnis gef\"{u}hrt hat.\par
Von Anbeginn geh\"{o}rte zu den Zielen, die deutsch-franz\"{o}sische Zusammenarbeit wieder zu beleben, die 1901 zum Beginn der Leibniz-Edition gef\"{u}hrt hatte. In seinem Anruf vom 16. Januar 1998 \"{a}usserte Dieter Simon mir gegen\"{u}ber den Wunsch, auch Russland einzubeziehen. Eine derartige, internationale und dezentralisierte Zusammenarbeit von Editoren und Forschern an weit auseinander gelegenen Orten erforderte eine neuartige Konzeption und Organisation der wissenschaftlichen Editionsarbeit. Die technische L\"{o}sung lag in der Nutzung der M\"{o}glichkeiten des Internet. Die Handschriften mussten im Internet zug\"{a}nglich gemacht werden. Die Edition wurde nicht nur im Druck, sondern auch f\"{u}r das Internet vorbereitet.\par
Das Finden geeigneter Mitarbeiter war und blieb schwierig. Die zur\"{u}ckliegenden Jahre sind von zahlreichen Mitarbeiterwechseln, teilweise verursacht durch zwei Todesf\"{a}lle, gekennzeichnet. Die erforderlichen finanziellen Mittel mussten eingeworben bzw. bereit gestellt werden, um die internationale Zusammenarbeit zu erm\"{o}glichen und in Berlin an der BBAW eine neue Arbeitsstelle zu gr\"{u}nden. Am Anfang gab es weder eine personelle noch eine wissenschaftliche Infrastruktur, weder eine Handbibliothek noch Kataloge.\par \vspace{2.0ex}

Die Organisation der Arbeitsgruppen\par\vspace{1.0ex}
Die franz\"{o}sisch-deutsche Zusammenarbeit begann am 30. September 1997 in Paris mit einem Treffen, an dem insbesondere Guy Ourisson, der damalige Vizepr\"{a}sident der Acad\'{e}mie des Sciences, und Manfred Bierwisch, der damalige Vizepr\"{a}sident der BBAW, teilnahmen. Mein Ansprechpartner auf franz\"{o}sischer Seite wurde der membre de l'Institut Claude Debru. Die franz\"{o}sische freie Mitarbeiterin ist zur Zeit Anne-Lise Rey, Universit\"{a}t Lille I.\par
Die russisch-deutsche Zusammenarbeit wurde in einem Kooperations-Ver\-trag beschlossen, der am 20. November 1998 von den Pr\"{a}sidenten der Russischen Akademie der Wissenschaften, Jurij Ossipov, und der BBAW, Dieter Simon, in Mos\-kau unterzeichnet wurde. Die russischen Mitarbeiter wurden die Wissenschafts\-historiker Vladimir Kirsanov (gest. am 12. Mai 2007) am 1. Mai 2000 in Moskau, unterst\"{u}tzt von der Latinistin Olga Fedorova, und Alena Kuznetsova (gest. am  25. September 2005) am 1. Juni 2000 in St. Petersburg, unterst\"{u}tzt von der Latinistin Ekaterina Basargina. Zur Zeit gibt es Bem\"{u}hungen, die Zusammenarbeit mit anderen russischen Wissenschaftshistorikern fortzusetzen.\par
An der BBAW stimmten deren ordentliche Mitglieder am 16. M\"{a}rz 2000 f\"{u}r die Aufnahme der Reihe VIII der Leibniz-Edition unter die Langzeitvorhaben der Akademie. Die neue Arbeitsstelle mit zun\"{a}chst einer und einer drittel Besch\"{a}ftigungsposition wurde zum 1. Januar 2001 eingerichtet. Editor und Arbeitsstellenleiter wurde der Physiker und Philosoph Hartmut Hecht. Die Linguistin und Philosophin Simone Rieger wurde f\"{u}r die Schaffung der elektronischen Arbeits\-umgebung zust\"{a}ndig. Diese Stelle wurde sp\"{a}ter von Lutz Sattler wahrgenommen. Der langj\"{a}hrige Vorsitzende der Leibniz-Kommission, J\"{u}rgen Mittelstra{\ss}, hat ent\-scheidenden Anteil daran, dass der Arbeitsstelle schlie{\ss}lich eine zweite volle Editorenstelle zur Verf\"{u}gung gestellt wurde, die heute der Chemiker und Theologe Sebastian Stork innehat.\par\vspace{2.0ex}

Die Schaffung der Arbeitsvoraussetzungen\par\vspace{1.0ex}
Die Hermann und Elise geborene Heckmann Wentzel-Stiftung bewilligte am 28. Juni 1999 Mittel zur Ausstattung der k\"{u}nftigen Berliner Arbeitsstelle mit Arbeitsmaterialien. Die Digitalisierung der etwa 4500 Blatt umfassenden Leibniz-Handschriften begann am 26. August 1999 in der Gottfried Wilhelm Leibniz Bibliothek in Hannover. Von Anbeginn unterst\"{u}tzten die Bibliotheks\-direktoren Wolfgang Ditt\-rich, sp\"{a}ter Georg Ruppelt das Unternehmen. Daf\"{u}r geb\"{u}hrt ihnen gr\"{o}{\ss}ter Dank. Die Deutsche Forschungsgemeinschaft (DFG) bewilligte Dittrich am 28. November 2000 die Mittel zur Digitalisierung der Handschriften. Die Konzeption der Datenbank wurde von Simone Rieger und Peter Cassiers ausgearbeitet.\par
Am 19. Januar 2000 bewilligte die DFG die Mittel, um \"{u}ber f\"{u}nf Jahre mit den zwei russischen Arbeitsgruppen in Moskau und St. Petersburg einen Werkvertrag abzuschlie{\ss}en. Danach \"{u}bernahm die BBAW diese finanzielle Verpflichtung.
Einen hohen Betrag gew\"{a}hrte die Stiftung der VGH Versicherungen Landschaft\-liche Brandkasse Hannover am 29. August  2001. Sie wurde 1750 gegr\"{u}ndet und sieht sich ideell in der Nachfolge von Leibniz, der nachdr\"{u}cklich f\"{u}r die Gr\"{u}ndung von Versicherungsgesellschaften zum Schutz des Einzelnen eingetreten war. Die bewilligten Mittel dienten u. a. zur Bezahlung des freien Mitarbeiters Peter Cassiers, der die Digitalisate bearbeitete, ins Internet stellte und die Formalismen f\"{u}r die Internetedition ausarbeitete. Sie waren als eingeworbene Drittmittel nicht an Haushaltsjahre gebunden und halfen bis ins Jahr 2008, freie Mitarbeiter bei der Fertigstellung des vorliegenden Bandes zu bezahlen.\par
Als sehr hilfreich auch f\"{u}r diese Reihe der Leibniz-Edition erwies sich Martin Gr\"{o}tschels Gr\"{u}ndung einer Arbeitsgruppe \textit{Elektronisches Publizieren}  (TELOTA) am 19. Dezember 2000 an der BBAW. Ein ermutigendes Zeichen der Kooperations\-bereitschaft setzte der damalige stellvertretende Direktor der Herzog August Bibliothek Wolfenb\"{u}ttel, Ulrich Schneider. Die ihm von der DFG bewilligten Mittel zur retrospektiven Digitalisierung fr\"{u}hneuzeitlicher Werke dienten ihm 2002 u. a. dazu, auf Vorschlag der Berliner Arbeitsstelle Leibniz-relevante Originalliteratur des 17. Jahrhunderts  zu digitalisieren. Entsprechende von Leibniz zitierte Literatur ist in der elektronischen Edition der Reihe VIII mit den Digitalisaten des Wolfenb\"{u}tteler Servers verlinkt.\par \vspace{2.0ex}

Das Arbeitsziel\par\vspace{1.0ex}
Ziel der Editionsarbeit ist es, alle \"{u}berlieferten Leibniz'schen Schriften, Auf\-zeichnungen und Marginalien zu naturwissenschaftlichen, medizinischen und tech\-nischen Themen in Abstimmung mit den inhaltlich verwandten Schriftenreihen VI Philosophische Schriften und VII Mathematische Schriften\rule[-5mm]{0mm}{0mm} zu ver\"{o}ffent\-lichen. Wie in diesen Schriftenreihen \"{u}blich werden die Texte chronologisch angeordnet, aber innerhalb eines Zeitraums in Themengruppen zusammengefasst.\par
Die Texte werden in der Druckversion voraussichtlich acht bis neun B\"{a}nde erfordern. Thematisch lassen sie sich \"{u}berwiegend vier Gebieten zuordnen:
\newcounter{vorwort}
\begin{list}{\arabic{vorwort}.\hfill}{\usecounter{vorwort}%
   \setlength{\labelwidth}{3mm}
   \setlength{\labelsep}{1mm}
   \setlength{\itemindent}{5mm}
   \setlength{\leftmargin}{0cm}
   \setlength{\itemsep}{-0.5ex}}
\item Naturwissenschaften (ca. 1500 Blatt) mit Astronomie, Botanik, Zoologie, Chemie, Geographie, Geologie, Physik;
\item Technik (ca. 450 Blatt) mit Hydraulik, M\"{u}hlen, Planetarien, Schiffsbau, Rechenmaschine, Transportwesen, Zeitmessung;
\item Militaria (ca. 300 Blatt) mit Artillerie, Fortifikation, Kriegsf\"{u}hrung;
\item Medizin und Pharmazie (ca. 1400 Blatt) mit Arzneimittelkunde, Di\"{a}tetik, Kosmetik, Medizinalwesen, Pathologie.
\end{list}
Die Reihe VIII ist ein transdisziplin\"{a}res, innerhalb der BBAW klassen\"{u}bergreifendes Editionsprojekt, das im Wesentlichen Neuland betritt: Von den betroffenen Handschriften sind bis heute nur sehr wenige bekannt.\par
Allen privaten und \"{o}ffentlichen Geldgebern, allen Unterst\"{u}tzern sei an dieser Stelle noch einmal von Herzen gedankt. Ohne sie h\"{a}tte die Reihe VIII der Leibniz-Edition nicht begonnen werden k\"{o}nnen.
Neben den bereits genannten, ist an dieser Stelle Alexandra Lewendoski, Sabine Seifert, vor allem aber Nele-Hendrikje Lehmann zu danken, die u. a. durch Transkriptionen, die Aufnahme der Marginalien, durch Wasserzeichenrecherchen, die Anfertigung von Zeichnungen und Lektorierungsarbeiten das Projekt \"{u}ber Jahre begleitet haben. In der Endphase hat Sabine Sellschopp einige St\"{u}cke hinsichtlich der Formalien ihrer Pr\"{a}sentation im Druck durchgesehen und bei der Transkription von Teilen aus N. 70 geholfen. Besonderer Dank gilt dem TELOTA-Team der BBAW unter der Leitung von Gerald Neumann, vor allem Markus Schn\"{o}pf, das eine tech\-nische L\"{o}sung zur weitgehend automatischen Transformation der XML-Daten der Internet-Edition in das Layout der Druckfassung erarbeitet hat. Markus Schn\"{o}pf war dar\"{u}ber hinaus an dem Schreiben des Satzprogramms in \LaTeX und der manuellen Eingabe automatisch nicht konvertierbarer XML-Daten beteiligt. Die Arbeiten am Band VIII, 1 haben auf unterschiedliche Weise durch die Leibniz-Arbeitsstellen in Hannover, M\"{u}nster und Potsdam Unterst\"{u}tzung erfahren. Dies betraf die M\"{o}glichkeit der Einsicht in Kataloge, den Austausch von Erfahrungen bei der Pr\"{a}sentation komplizierter Textpassagen, Wasserzeichenrecherchen oder das \"{U}berlassen von Satzhilfen. Daf\"{u}r sei den betreffenden Kollegen herzlich gedankt.\par \vspace{1.0ex}
Berlin, im Fr\"{u}hjahr 2009 \hfill Eberhard Knobloch\par