[151 r\textsuperscript{o}] \edtext{\textso{Conseq. 6}. On pourroit bien expliquer \textso{le phaenomene 9}. ou l'attachement de deux placques dans le vuide, par une liqueur ou matiere fluide dans laquelle on suppose un mouuement en tous sens;\edlabel{hypo150v2}}
{\lemma{Hypothese.}\xxref{hypo150v1}{hypo150v2}\Afootnote{ \textit{ (1) }\ \textso{Consequ. 6}. J'ose \textbar~même \textit{ gestr.}\ \textbar\ dire, d'avantage que le phenomene 7.  \textbar\ (et par consequent les autres non plus) \textit{ erg.}\ \textbar\ ne  peut pas estre expliqu\'{e} par un mouuement d'une matiere subtile, mais seulement par un effort compens\'{e} \textit{(a)}\ et insensible, sans \textit{(b)}\ comme est celuy de  \textbar\ l'adversaire \textit{ erg. u. gestr.}~\textbar l'atmosphaere sur nous, ou de la coulomne de la mer sur les plongeurs: dont on ne s'appercoit pas, sinon quand on leur donne \textit{ (2) }\ \textso{Conseq.} [...] expliquer  \textbar\ \textso{le phaenomene 9.} ou \textit{ erg.}\ \textbar\ l'attachement [...] liqueur  \textbar\ ou matiere fluide \textit{ erg.}\ \textbar\ dans [...] sens; \textit{ L}}} 
dont les vagues frappent les superficies exterieures des placques:\edlabel{plaq151r1} \pend \pstart\edtext{\edlabel{plaq151r2}Mais}{\lemma{placques:}\xxref{plaq151r1}{plaq151r2}\Afootnote{ \textit{ (1) }\ pourveu que ces placques  \textit{(a)}\ ne soient pas fort \textit{(b)}\ soient moins poreuses, que solides car   \textbar\ autrement \textit{ erg.}\ \textbar\  la liqueur passant par les pores frapperoit les interieures aussi. \textit{ (2) }\   Mais \textit{ L}}} on aura de la peine d'expliquer par ce mouuement d'une \edtext{matiere subtile}{\lemma{d'une}\Afootnote{ \textit{ (1) }\ liqueur \textit{ (2) }\ matiere subtile \textit{ L}}} en tous sens le phaenomene de la liqueur purg\'{e}e\protect\index{Sachverzeichnis}{liqueur!purg\'{e}e} d'air. Car le mouuement de \edtext{cette matiere subtile}{\lemma{cette}\Afootnote{ \textit{ (1) }\ liqueur \textit{ (2) }\ matiere subtile \textit{ L}}} continuera, \edtext{même}{\lemma{continuera}\Afootnote{ \textit{ (1) }\ ces \textit{ (2) }\ ses coups, qu \textit{ (3) }\ , même \textit{ L}}} quand il y aura de l'air engendr\'{e} dans la liqueur, et comme il \edtext{est capable de presser la liqueur vers la surface du verre, malgr\'{e} sa pesanteur, il sera aussi capable d'empecher}{\lemma{il}\Afootnote{ \textit{ (1) }\ a \textit{ (2) }\ suffit de so \textit{ (3) }\ est\'{e} assez fort \`{a} soûtenir les liqueurs, il sera aussi assez fort \`{a} empe \textit{ (4) }\ est [...] aussi \textit{(a)}\ assez fort pour \textit{(b)}\ capable d'empecher \textit{ L}}} qu'une petite bulle d'air se mette entre deux, et se dilate comme nous voyons qu'elle fait.%
\footnote{\textit{In der rechten Spalte}: %\edtext{Comme nous}{\lemma{d'air.}\Afootnote{ \textit{ (1) }\ La masse \textit{ (2) }\ Il restera pourtant cette difficult\'{e} \textit{ (3) }\ Comme nous \textit{ L}}} 
Comme nous l'experimentons dans l'air, dont le mou\-uement n'est pas en tous sens, et dans lequel il n'y a point de vagues, pour cet effect %  \textbar\ et [...] effect \textit{ erg.}\ \textbar\  
quoque l'effort soit en tous sens, que si l'on explique le mouuement en tous sens de cette fa\c{c}on par un simple effort, %\textbar\ par un simple effort \textit{ erg.}\ \textbar\ 
l'approuue entierement, et je m'en servira, par apres moy même. Mais je crois de n'avoir pas besoin d'un autre que celuy de l'air, dont nous sommes persuadez partout d'experiences, sans %\edtext{employer une}{\lemma{nous}\Afootnote{ \textit{ (1) }\ avoir besoin d'un \textit{ (2) }\ employer une \textit{ L}}}
employer une matiere purement suppos\'{e}e, qui passe par les pores du verre.\par%\edtext{Je crois}{\lemma{une}\Afootnote{ \textit{ (1) }\ Mais si \textit{ (2) }\ Je crois \textit{ L}}} 
Je crois même que l'Hypothese %\edtext{du mouuement en}{\lemma{crois}\Afootnote{ \textit{ (1) }\ de ceux qui supposeront  \textit{(a)}\ une matiere subtile\protect\index{Sachverzeichnis}{mati\`{e}re!subtile|textit} m\"{u} \textit{(b)}\ le mouuement en \textit{(c)}\ que le mouuement \textit{ (2) }\ du mouuement en \textit{ L}}}
du mouuement en tous sens de la %\edtext{matiere lequel passant}{\lemma{en}\Afootnote{ \textit{ (1) }\ liqueur qui passe \textit{ (2) }\ matiere lequel passant \textit{ L}}} 
matiere lequel passant les pores du verre (pour y remplir la place quand on tire l'air) renferm\'{e} dans la petite bulle soit capable d'\'{e}galer %\edtext{tous les autres}{\lemma{passant}\Afootnote{ \textit{ (1) }\ le reste \textit{ (2) }\ tous les autres \textit{ L}}} 
tous les autres coups %\edtext{de la même matiere, se combatte}{\lemma{autres}\Afootnote{ \textit{ (1) }\ que la même liqueur recoit par dehors, et qui la pressent vers la superficie interieure du verre \textit{ (2) }\ de [...] combatte \textit{ L}}}
de la même matiere, se combatte elle même car s'il y a des pores, %\edtext{le dit mouuement en tous sens, passant par le verre fera tomber la liqueur purg\'{e}e qui est suspend\"{u}e dans le tuyau, \`{a} cause que la liqueur suspend\"{u}e est press\'{e}e}{\lemma{combatte}\Afootnote{ \textit{ (1) }\ la matiere pressante pressera \textit{ (2) }\ le [...] press\'{e}e \textit{ L}}} 
le dit mouuement en tous sens, passant par le verre fera tomber la liqueur purg\'{e}e qui est suspend\"{u}e dans le tuyau, \`{a} cause que la liqueur suspend\"{u}e est press\'{e}e de deux costez comme cela arrive, quand on donne %\edtext{l'entr\'{e}e \`{a} l'air per\c{c}ant en haut le tuyau}{\lemma{press\'{e}e}\Afootnote{ \textit{ (1) }\ ouuerture \textit{ (2) }\ entr\'{e}e [...] tuyau \textit{ L}}} 
l'entr\'{e}e \`{a} l'air per\c{c}ant en haut le tuyau de Torricelli\protect\index{Namensregister}{\textso{Torricelli} (Torricellius), Evangelista 1608\textendash 1647}.}%
%folgende Afootnotes beziehen sich auf die footnote
\edtext{}{\lemma{}\linenum{|2|||2|}\Afootnote{ \textit{ (1) }\ La masse \textit{ (2) }\ Il restera pourtant cette difficult\'{e} \textit{ (3) }\ Comme nous \textit{ L}}}\edtext{}{\lemma{}\linenum{|3|||4|}\Afootnote{et [...] effect \textit{ erg. L}}}\edtext{}{\lemma{}\linenum{|5|||5|}\Afootnote{par un simple effort \textit{ erg. L}}}\edtext{}{\lemma{sans}\linenum{|7|||8|}\Afootnote{ \textit{ (1) }\ avoir besoin d'un \textit{ (2) }\ employer une \textit{ L}}}\edtext{}{\lemma{verre.}\linenum{|8|||9|}\Afootnote{ \textit{ (1) }\ Mais si \textit{ (2) }\ Je crois \textit{ L}}}\edtext{}{\lemma{l'Hypothese}\linenum{|9|||9|}\Afootnote{ \textit{ (1) }\ de ceux qui supposeront  \textit{(a)}\ une matiere subtile\protect\index{Sachverzeichnis}{mati\`{e}re!subtile|textit} m\"{u} \textit{(b)}\ le mouuement en \textit{(c)}\ que le mouuement \textit{ (2) }\ du mouuement en \textit{ L}}}\edtext{}{\lemma{la}\linenum{|9|||10|}\Afootnote{ \textit{ (1) }\ liqueur qui passe \textit{ (2) }\ matiere lequel passant \textit{ L}}}\edtext{}{\lemma{d'\'{e}galer}\linenum{|11|||11|}\Afootnote{ \textit{ (1) }\ le reste \textit{ (2) }\ tous les autres \textit{ L}}}\edtext{}{\lemma{coups}\linenum{|11|||12|}\Afootnote{ \textit{ (1) }\ que la même liqueur recoit par dehors, et qui la pressent vers la superficie interieure du verre \textit{ (2) }\ de [...] combatte \textit{ L}}}%
\edtext{}{\lemma{pores,}\linenum{|12|||14|}\Afootnote{ \textit{ (1) }\ la matiere pressante pressera \textit{ (2) }\ le [...] press\'{e}e \textit{ L}}}\edtext{}{\lemma{donne l'}\linenum{|15|||15|}\Afootnote{ \textit{ (1) }\ ouuerture \textit{ (2) }\ entr\'{e}e [...] tuyau \textit{ L}}} %Ende der zu revidierenden Fussnoten
Et il ne suffit pas de dire, que cette matiere subtile, trouuant de la place dans la bulle, frappe \edtext{ainsi}{\lemma{frappe}\Afootnote{ \textit{ (1) }\ aussi \textit{ (2) }\ ainsi \textit{ L}}} la liqueur suspend\"{u}e de deux costez, et la repousse \edtext{autant}{\lemma{repousse}\Afootnote{ \textit{ (1) }\ ainsi \textit{ (2) }\ autant \textit{ L}}} qu'il l'a pouss\'{e} vers le verre. Car sans insister sur ce que même \edtext{cette pression empechera}{\lemma{même}\Afootnote{ \textit{ (1) }\ la bulle n'aura pas le pouu \textit{ (2) }\ cette pression empechera \textit{ L}}} la generation de la bulle, et surtout, qu'elle ne suffrira pas que la bulle se place entre la liqueur et la surface interieure du verre; il faut considerer que le peu de coups \edtext{du mouuement en tous sens de la matiere subtile insinu\'{e}e}{\lemma{}\Afootnote{du [...] insinu\'{e}e \textit{ erg.} \textit{ L}}} dans une petite bulle, ne peut pas \'{e}galer \edtext{ny d\'{e}truire}{\lemma{}\Afootnote{ny d\'{e}truire \textit{ erg.} \textit{ L}}} tous les autres que la liqueur recoit de tous costez, et par lesquels elle est pouss\'{e}e vers \edtext{la surface interieure du verre}{\lemma{vers}\Afootnote{ \textit{ (1) }\ le verre \textit{ (2) }\ la surface interieure du verre \textit{ L}}}. \edtext{Et il faut remarquer, qu'il y a en cela une grande difference, entre la pression universelle d'une Masse, comme est celle de l'Atmosphere, et entre la pression du mouuement d'une liqueur en tous sens. Car la pression universelle est \'{e}galle, quoyqu'elle trouue seulement un petit passage, comme nous voyons, que le Mercure suspendu dans le tuyau de Torricelli tombe si l'on perce le haut du tuyau avec une \'{e}pingle. Parce que la Masse a un effort general de partager \'{e}galement les forces partout. Mais le mouuement d'une liqueur en tous sens, est particulier \`{a} chaque partie de la masse si ce n'est pas un effort comme celuy de la force elastique, de l'air.}{\lemma{}\Afootnote{Et [...] \'{e}galle, \textit{ (1) }\ quoyque l'entr\'{e}e d'elle partout \textit{ (2) }\ quoyqu'elle  \textit{(a)}\ aye seulement une petite ouverture \textit{(b)}\ trouue [...] elastique, \textit{(aa)}\ qui est asseurement en tous sens, mais qui provient d'un \textit{(aaa)}\ mouuement universel de toute \textit{(bbb)}\ mouvement ou effort universel \textit{(bb)}\ de l'air.  \textit{ erg.} \textit{ L}}}\pend 
\pstart \textso{Conseq. 7}. Il semble qu'on peut tirer de ces phaenomenes ensemble \edtext{une observation}{\lemma{une}\Afootnote{ \textit{ (1) }\ Regle \textit{ (2) }\ observation \textit{ L}}} generalle, s\c{c}avoir \textso{que la Nature tache d'empecher la discontinuation des corps sensibles.} Car même dans le vuide o\`{u} il n'y a point de corps sensible, \edtext{deux corps solides ne se separent pas}{\lemma{sensible,}\Afootnote{ \textit{ (1) }\ il fau \textit{ (2) }\ en observe qu \textit{ (3) }\ deux corps solides  \textbar\ bien \textit{ erg. u.}\  \textit{ gestr.}\ \textbar\  ne se separent pas \textit{ L}}}, ais\'{e}ment, comme on voit par le phaenomene 9. des placques; \edtext{ny}{\lemma{placques;}\Afootnote{ \textit{ (1) }\ de même \textit{ (2) }\ ny \textit{ L}}} deux liquides, par le phaenomene 10. du siphon \`{a} jambes in\'{e}gales; ny un solide d'un liquide, par les phaenomenes \edtext{5. et 7.}{\lemma{}\Afootnote{5. et 7. \textit{ erg.} \textit{ L}}} de la liqueur purg\'{e}e\protect\index{Sachverzeichnis}{liqueur!purg\'{e}e} d'air.