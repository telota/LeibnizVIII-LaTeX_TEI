[151 v\textsuperscript{o}] Mais sitost qu'un corps sensible\protect\index{Sachverzeichnis}{corps!sensible}, \edtext{}{\lemma{}\Afootnote{sensible,  \textbar\ comme l'air \textit{ gestr.}\ \textbar\ qui \textit{ L}}} qui se peut \'{e}tendre ou prendre un plus grand volume, est interpos\'{e}, alors cet attachement cesse, et les corps attachez se separent, parce que toute la place entre deux peut estre remplie par l'air qui \edtext{s'\'{e}tend}{\lemma{qui}\Afootnote{ \textit{ (1) }\ s'en fl \textit{ (2) }\ s'\'{e}tend \textit{ L}}} pour cet effect.\pend \pstart  Partant celuy qui rendroit raison de cette regle, ou \edtext{de cette loix de la Nature}{\lemma{ou}\Afootnote{ \textit{ (1) }\ loix \textit{ (2) }\ de [...] Nature \textit{ L}}} rendroit raison en même temps de tous ces phenomenes.\pend \pstart Pour rendre raison d'un phenomene \edtext{de la nature}{\lemma{}\Afootnote{de la nature \textit{ erg.} \textit{ L}}} il faut tousjours tacher \edtext{de l'expliquer}{\lemma{tacher}\Afootnote{ \textit{ (1) }\ d'en \textit{ (2) }\ s'il est possible   \textbar\ de \textit{ erg.} \textit{ Hrsg. }\  l'expliquer \textit{ (3) }\ de l'expliquer \textit{ L}}} par d'autres phenomenes \edtext{et se garder des hypotheses}{\lemma{phenomenes}\Afootnote{ \textit{ (1) }\ plus connûs et plus universels \textit{ (2) }\ d\'{e}j\`{a} commun \textit{ (3) }\ dont la raison est d\'{e}j\`{a} conn\"{u}e, ou dont \textit{ (4) }\ et se garder des hypotheses \textit{ L}}} autant qu'il est possible pour cet effect. J'ay tach\'{e} de rendre raison de tous les effects de la nature (au moins en gros) sans me servir d'une Hypothese, \edtext{ou}{\lemma{Hypothese,}\Afootnote{ \textit{ (1) }\ et sa \textit{ (2) }\ ou \textit{ L}}} d'un autre principe, que de ce phenomene incontestable, du mouuement de la lumiere \edtext{du soleil}{\lemma{}\Afootnote{du soleil \textit{ erg.} \textit{ L}}} \`{a} l'entour de la terre\protect\index{Sachverzeichnis}{terre} dans l'equateur et dans les cercles parallels \`{a} l'equateur: dont je tire la consequence d'un autre mouuement, vers les poles par les meridiens, parce que la matiere plus grossiere que \edtext{celle de}{\lemma{}\Afootnote{celle de \textit{ erg.} \textit{ L}}} la lumiere, mais moins grossiere que les corps sensibles\protect\index{Sachverzeichnis}{corps!sensible} estant rejett\'{e}s de l'equateur et paralleles par le mouuement de la lumiere, et ne pouuant pas aller vers le centre \`{a} cause des corps plus grossiers, et chass\'{e}es vers les poles.\pend \pstart  Car la lumiere par sa rapidit\'{e}, \edtext{tache ou de dissiper ou de}{\lemma{}\Afootnote{BITTE UEBERPRUEFEN!!! tache [...] de \textit{ erg.} \textit{ L}}} rejetter tous les obstacles, et tous les corps heterogenes \edtext{ou trop grossiers}{\lemma{}\Afootnote{ou trop grossiers \textit{ erg.} \textit{ L}}} qui troublent son mouuement, vers l'endroit o\`{u} le mouuement est moins rapide, c'est \`{a} dire vers le centre, et (en cas qu'il ne peut pas vers le centre) vers le pole.\pend \pstart  \edtext{Mais en cas}{\lemma{?LEMMA?:pole.}\Afootnote{ \textit{ (1) }\ Dont je ren \textit{ (2) }\ Mais en cas \textit{ L}}} que les corps ne peuuent pas estre \edtext{chassez}{\lemma{estre}\Afootnote{ \textit{ (1) }\ rejettez \textit{ (2) }\ chassez \textit{ L}}} \edtext{ny même dissipez }{\lemma{}\Afootnote{ny même dissipez   \textbar\ en une subtilit\'{e} plus grande, (pour moins empecher par leur grossieret\'{e}) \textit{ gestr.}\ \textbar\   \textit{ erg.} \textit{ L}}}, le mouuement general les fait prendre au moins la place \edtext{et la situation}{\lemma{}\Afootnote{et la situation \textit{ erg.} \textit{ L}}} la plus propre selon la bienseance universelle, pour estre empech\'{e} le moins qu'il est possible. De ces consequences necessaires d'un phenomene \edtext{general}{\lemma{}\Afootnote{general \textit{ erg.} \textit{ L}}} incontestable je tache de rendre raison des phenomenes \edtext{plus particuliers}{\lemma{}\Afootnote{plus particuliers \textit{ erg.} \textit{ L}}} de la pesanteur, du ressort, et de l'aimant; et je crois \edtext{d'en pouuoir}{\lemma{crois}\Afootnote{ \textit{ (1) }\ d'avoir \textit{ (2) }\ d'en pouuoir \textit{ L}}} tirer même quelque consequence, \textso{sans faire aucune Hypothese nouuelle,} pour rendre raison de ces phenomenes de l'attachement dans le vuide, ou de \textso{la loix de la continuation des corps sensibles.} \edtext{Car selon}{\lemma{?LEMMA?:sensibles.}\Afootnote{ \textit{ (1) }\ Dont voicy la maniere \textit{ (2) }\ Car selon \textit{ L}}} ce que \edtext{j'espere de montrer}{\lemma{que}\Afootnote{ \textit{ (1) }\ j'ay montr\'{e} \textit{ (2) }\ j'espere de montrer \textit{ L}}} ailleurs plus amplement, il s'ensuit de ce mouuement \edtext{publique,}{\lemma{mouuement}\Afootnote{ \textit{ (1) }\ general, \textit{ (2) }\ publique, \textit{ L}}} la Regle generalle de \textso{l'equilibre universel,} c'est \`{a} dire, \edtext{[qu'ils]}{\lemma{}\Afootnote{qu'ils \textit{ erg.} \textit{ Hrsg. }\ }} se trouuent des \'{e}gales forces partout. De sorte que la pesanteur compense le ressort, la vitesse compense la \edtext{petitesse}{\lemma{la}\Afootnote{ \textit{ (1) }\ matiere petite \textit{ (2) }\ petitesse \textit{ L}}} \edtext{le lieu compense le temps}{\lemma{}\Afootnote{le lieu compense le temps \textit{ gestr. und wieder g\"ultig gemacht} \textit{ L}}} \edtext{La fermet\'{e} de l'obstacle l'effort arrest\'{e}. Par consequent}{\lemma{?LEMMA?:temps}\Afootnote{ \textit{ (1) }\ . Les consequences de cette regle sont tres importantes \textit{ (2) }\ La fermet\'{e}  \textit{(a)}\ du vase, le effort de la matiere enferm\'{e}e \textit{(b)}\ de l'obstacle l'effort  \textit{(aa)}\ empech\'{e} \textit{(bb)}\ arrest\'{e}. Par consequent \textit{ L}}}, s'il y a un lieu mal pourveu de forces, et qui n'a pas assez de resistance pour equilibrer les corps ambiants, toute la nature s'efforcera \`{a} luy faire justice, et d\'{e}tachera autant que luy faut, de toutes les autres parties du monde, et cela en un moment. \edtext{Il est ais\'{e}}{\lemma{moment.}\Afootnote{ \textit{ (1) }\ Par consequent \textit{ (2) }\ Il est ais\'{e} \textit{ L}}} d'appliquer cela \`{a} nostre propos, car aussitost qu'on separe deux corps, comme deux placques, il faut qu'il se trouue un corps \edtext{de quelqu'effort}{\lemma{}\Afootnote{de quelqu'effort \textit{ erg.} \textit{ L}}} entre deux. \edtext{ Pas \`{a} cause}{\lemma{deux.}\Afootnote{ \textit{ (1) }\ Autrement \textit{ (2) }\  Pas \`{a} cause \textit{ L}}} de la crainte du vuide\protect\index{Sachverzeichnis}{vide}, mais parce que toute la masse agit, contre un lieu oû il n'y a point d' effort. Car il y a point d'air. \edtext{ S\c{c}avoir}{\lemma{d'air.}\Afootnote{ \textit{ (1) }\ Mais on dira, qu'il y a de l' effort la dedans en effect \textit{ (2) }\  S\c{c}avoir \textit{ L}}} celuy du poids qui tache de separer les corps, et qui peut bien \'{e}galer celuy du ressort d'un peu d'air, qui s'y met par apres. Je responds \`{a} cette objection assez difficile en apparence; que la force de la pesanteur, ou d'un ressort (comme de fer) est finie. Et se repose ayant atteint son terme, celle de l'air est infinie. Car il s'ouuriroit tousjours\edtext{ . Il faut}{\lemma{tousjours}\Afootnote{ \textit{ (1) }\ , de sorte \textit{ (2) }\  . Il faut \textit{ L}}} donc \edtext{dans ce lieu}{\lemma{}\Afootnote{dans ce lieu \textit{ erg.} \textit{ L}}} ou de l'air, ou d'un corps qui resiste \`{a} la pression generalle. \edtext{}{\lemma{}\Afootnote{generalle.  \textbar\ Et par consequent tous les corps qui n'ont pas un effort de s'\'{e}tendre se gelent, estant \textit{ gestr.}\ \textbar\  \textit{ L}}}