[70 r\textsuperscript{o}] Sed cum eorum omnium quae hactenus dicta sunt ad Longitudines\protect\index{Sachverzeichnis}{longitudo} applicatio ex horologio\protect\index{Sachverzeichnis}{horologium} exacto supposito pendeat: Et vero supersint quaedam circa usum penduli \protect\index{Sachverzeichnis}{pendulum} in mari difficultates; operae pretium est exponi \textso{rationem novam, qua defectus }\textso{Horologii}\protect\index{Sachverzeichnis}{horologium}\textso{ ejusmodi accurati, de tempore in tempus suppleri possit}, invenirique quae nunc sit hora, aut quod horae (a revolutione primi mobilis aestimatae) minutum in loco cognitae longitudinis\protect\index{Sachverzeichnis}{longitudo}, seu in universum in mundo. Nam invenire quae sit hora loci dati v. g. si in  eo sit meridies, nec nosse tamen loci longitudinem\protect\index{Sachverzeichnis}{longitudo} usum habet nullum, non enim ideo cognoscetur hora universalis in mundo, quia nec ejus loci situs in mundo plene cognitus est. Quanquam ergo, horologio \protect\index{Sachverzeichnis}{horologium} exacto deficiente, inventio horae universalis inventionem longitudinum\protect\index{Sachverzeichnis}{longitudo} supponere videatur,  venit tamen mihi in mentem haec, quam nunc exponam ratio, quae nec longitudine\protect\index{Sachverzeichnis}{longitudo} loci praesentis praecognita, nec horologio\protect\index{Sachverzeichnis}{horologium} exacto (in loco cognitae longitudinis\protect\index{Sachverzeichnis}{longitudo}, in hora minutove debito \edtext{primum}{\lemma{}\Afootnote{primum \textit{ erg.} \textit{ L}}} constituto, et hactenus currente) indigeat, etsi non semper possit adhiberi, quod nec necesse est \edtext{dummodo, ut dixi, \textso{de tempore in tempus} usurpetur, ut, exempli causa de mense in mensem. intermedium enim tempus horologio\protect\index{Sachverzeichnis}{horologium} aliquo vulgari arenario, cujus etiam in navibus\protect\index{Sachverzeichnis}{navis} usus esse solet (praesertim si certa quadam ratione emendentur) sic satis accurate scietur, nec paucis diebus error adeo grandis intervenire potest, praesertim si horologia\protect\index{Sachverzeichnis}{horologium} plura eorundem et plurium generum inter se conferantur. Ut taceam spem esse Horologia Elastica\protect\index{Sachverzeichnis}{horologium!elasticum} a jactatione navis\protect\index{Sachverzeichnis}{navis} independentia, sic satis accurata, diligentia adhibita, construendi}{\lemma{}\Afootnote{BITTE UEBERPRUEFEN!!! dummodo, ut dixi, \textso{de tempore in tempus}usurpetur, [...] tempushorologio\protect\index{Sachverzeichnis}{horologium}aliquo [...] innavibus\protect\index{Sachverzeichnis}{navis}usus [...] sihorologia\protect\index{Sachverzeichnis}{horologium} plura   \textbar\ eorundem et \textit{ erg.}\ \textbar\  plurium generum inter se conferantur. Ut taceam spem esse Horologia Elastica\protect\index{Sachverzeichnis}{horologium!elasticum} a jactatione navis\protect\index{Sachverzeichnis}{navis} independentia, sic satis accurata, diligentia adhibita, construendi \textit{ erg.} \textit{ L}}}.\pend \pstart Problema ita concipitur: \textso{Data per observationem unicam, seu eodem tempore factam in loco incognitae }\textso{longitudinis}\protect\index{Sachverzeichnis}{longitudo}\textso{ saltem latitudine}\protect\index{Sachverzeichnis}{latitudo}\textso{ loci seu elevatione poli, ac praeterea elevatione lunae et cujusdam alterius }\textso{fixae}\protect\index{Sachverzeichnis}{stella!fixa}\textso{ super horizontem loci dati; invenire tempus universale, a primi mobilis revolutione pendens, seu loci cognitae }\textso{longitudinis}\protect\index{Sachverzeichnis}{longitudo}\textso{, et per consequens ope problematis praecedentis, invenire longitudinem}\protect\index{Sachverzeichnis}{longitudo}\textso{ loci dati; seu }\textso{latitudine}\protect\index{Sachverzeichnis}{latitudo}\textso{ jam praecognita, locum }\textso{navis}\protect\index{Sachverzeichnis}{navis}\textso{ verum.}\pend \pstart Hoc problema solvi potest, quoties simul et Polus\protect\index{Sachverzeichnis}{polus}, et Luna\protect\index{Sachverzeichnis}{luna}, et alia quaedam fixa\protect\index{Sachverzeichnis}{stella!fixa} cognita non nimis Polo\protect\index{Sachverzeichnis}{polus} vicina, videri possunt. Quod certe frequentissime continget. Cum contra ii qui observationes suas sideri in meridiano\protect\index{Sachverzeichnis}{meridianus} existenti, aliisque definitis circumstantiis alligare coguntur, raro occasionem observandi reperiant. Ut enim definitum aliquod momentum noctis, aut spatium coeli serenum 