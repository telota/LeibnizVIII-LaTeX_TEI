\newpage
[101 v\textsuperscript{o}] At si liquor aere purgatus est, nihil est quod locum descensu  liquoris in tubo relinquendum implere possit, nisi id scilicet aut ex corpore ipsius liquoris eliciatur, aut ab externo per ejus  poros intret. Ergo si majore vi opus sit ad  talis corporis interni expressionem aut externi comminutionem  in partes tam subtiles quae penetrationi sufficiant;  quam est liquoris ultra altitudinem ordinariam  27 pollicum in Tubo comprehensi (qui enim  infra illam altitudinem est ejus pondus aeris externi aequipondio destruitur, praeterquam in Recipiente exhausto)  necesse est liquorem suspensum manere. Ratio difficultatis  est, quia corpus ejusmodi subtile ad locum implendum  in satis magna quantitate cogi non potest, quin  alia corpora quibus exprimitur tantundem comprimantur;  at huic compressioni insuetae, circulatio universalis,  omnia continue ad uniformitatem \edtext{[sollicitans],}{\lemma{solicitans}\Afootnote{\textit{L ändert Hrsg.}}} resistit. \edlabel{permi101r}%{\lemma{}\Afootnote{ \textit{ (1) }\  \textit{ (2) }\  \textit{ (3) }\ At [...] possit, \textit{(a)}\ supposito  quod liquor ipse nihil aliud Elasticum nobis compertum  contineat, \textit{(b)}\ nisi [...] liquoris  \textbar\ maxima vi \textit{ gestr.}\ \textbar\ eliciatur, [...] intret. \textit{(aa)}\ Sed cur \textit{(bb)}\ Ergo [...] corporis  \textbar\ interni \textit{ erg.}\ \textbar\ expressionem [...] est \textit{(aaa)}\ aeris aequipondio in  aere \textit{(bbb)}\ ejus [...] insuetae, \textit{(aaaa)}\ aetheris\protect\index{Sachverzeichnis}{aether|textit} circulatio unif \textit{(bbbb)}\ circulatio [...] resistit. \textit{ L}}} 
Etsi haec resistentia possit \edtext{pondere liquoris}{\lemma{}\Afootnote{pondere liquoris \textit{ erg.} \textit{ L}}} superari, quoniam vero  maximis Tubis opus esset, ut proprio pondere  liquor avellatur; poterit compendium \edtext{sumtuum}{\lemma{compendium}\Afootnote{ \textit{ (1) }\ saltuum \textit{ (2) }\ sumtuum \textit{ L}}}  taediique fieri ad postremum altitudinis gradum inveniendum \edtext{\textso{hoc experimento}}{\lemma{\textso{hoc}}\Afootnote{\textso{experimento} \textit{ erg.} \textit{ L}}},\footnote{\textit{In der rechten Spalte}: \textso{Experiment. 4}} si pro liquore embolus\protect\index{Sachverzeichnis}{embolus} exacte adaptatus tubo;  pro pondere liquoris ac tubi longitudine pondus embolo\protect\index{Sachverzeichnis}{embolus}  appensum, aut si id magnum nimis esse deberet, vis Mechanica\protect\index{Sachverzeichnis}{mechanica} ut cochlea\protect\index{Sachverzeichnis}{cochlea} adhibeatur. Ita enim \edtext{avulsione facta}{\lemma{}\Afootnote{avulsione facta \textit{ erg.} \textit{ L}}} facile  supputari potest quanta altitudine Tubi opus futurum  fuisset ad divulsionem procurandam.\pend \pstart  Hactenus dicta sufficiunt ad rationes eorum omnium  facile reddendas, quae novissime ab Illustri Hugenio\protect\index{Namensregister}{\textso{Huygens} (Hugenius, Vgenius, Hugens, Huguens), Christiaan 1629\textendash 1695}  detecta sunt; alia postea experimenta \edtext{afferemus, quae}{\lemma{afferemus,}\Afootnote{ \textit{ (1) }\ quibus res videtur indubitabilis reddi posse \textit{ (2) }\ quae \textit{ L}}}  Hypotheseos nostrae veritatem prorsus convincent.\pend \pstart  Cur in Experim. 1 \edtext{in proximo diario}{\lemma{1}\Afootnote{ \textit{ (1) }\ ab eo \textit{ (2) }\ in proximo diario \textit{ L}}} memorato  liquor \edtext{aere purgatus}{\lemma{}\Afootnote{aere purgatus \textit{ erg.} \textit{ L}}} in Recipiente manserit suspensus dictum est.\edtext{}{\lemma{est.}\Bfootnote{\textsc{Chr. Huygens, }\cite{00062}a.a.O., S.~134f. (\textit{HO} VII, S.~202).}}  Nec major difficultas in Experimento 2\textsuperscript{do}.\edtext{}{\lemma{2\textsuperscript{do}.}\Bfootnote{\textsc{Chr. Huygens, }\cite{00062}a.a.O., S.~135f. (\textit{HO} VII, S.~202f.).}} Nam liquor  descendere conatur nisu propriae gravitatis\protect\index{Sachverzeichnis}{gravitas}. Haec in  fundo seu prope orificium tubi major, ergo et major  ibi pressio. Ergo et maxima aeris \edtext{post purgationem}{\lemma{post}\Afootnote{ \textit{ (1) }\ pressionem \textit{ (2) }\ purgationem \textit{ L}}} quantamcunque residui ad locum in summo implendum  expressio. Cum vero bullae aeris nimis exiguae  ab aquae crassitie retineantur, necesse  est eas colligi in unam majorem, \edtext{ut}{\lemma{majorem,}\Afootnote{ \textit{ (1) }\ ad \textit{ (2) }\ ut \textit{ L}}} ascendere  possint, confluereque in eam canalibus insensibilibus seu  rimis in aqua actis. Bulla semel nata, aqua  levior, satisque virium ad perrumpendum nacta  ascendet, reliquumque aerem in itinere colliget,  ac proinde continue augebitur. Ubi vero fissura  sibi aperta parte sui ad summum usque seu supra,  liquorem emicuerit, liquor descendet, gravitateque\protect\index{Sachverzeichnis}{gravitas}  sua bullam ita distendet, ut spatio toti implendo  sufficiat. Tubi vero concussionem ad primam  bullae generationem multis modis conferre posse  manifestum est, facere enim potest, ut partes aeris 