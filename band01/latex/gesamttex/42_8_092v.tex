[92 v\textsuperscript{o}] ascendit. Quod ip⟨sum⟩ ⟨su⟩fficere videtur ad has aeris columnas  vanitatis arguend⟨as⟩. Quid enim inquiunt? Levis ac vix  sensibilis digiti calor\protect\index{Sachverzeichnis}{calor} solo tactu aquam thermometri\protect\index{Sachverzeichnis}{thermometrum} elevans,  quomodo totam atmosphaerae\protect\index{Sachverzeichnis}{atmosphaera} incumbentis gravitatem\protect\index{Sachverzeichnis}{gravitas}  vincat? Sed sciendum est \edtext{atmosphaeram}{\lemma{est}\Afootnote{ \textit{ (1) }\ nihil ea  \textit{(a)}\ gravitate\protect\index{Sachverzeichnis}{gravitas|textit} opus \textit{(b)}\ a \textit{ (2) }\ atmosphaeram \textit{ L}}} totam hic nec reniti, nec vinci: digitus ille utcunque  calidus tantum \edtext{antea}{\lemma{}\Afootnote{antea \textit{ erg.} \textit{ L}}} rarefaciebat aerem liberum sibi circumfusum,  quantum nunc rarefacit aerem Thermometri\protect\index{Sachverzeichnis}{thermometrum}, ergo cum in  aerem thermometri\protect\index{Sachverzeichnis}{thermometrum} agit, in aerem  liberum agere desinit, quantum ergo aer thermometri\protect\index{Sachverzeichnis}{thermometrum}  expanditur, tantum aer liber contrahitur, quare aqua ascendens  obstaculum invenit ab aeris pressione\protect\index{Sachverzeichnis}{pressio!aeris} nullum. At cur aerem liberum \edtext{circumfusum tam facile rarefacit}{\lemma{liberum}\Afootnote{ \textit{ (1) }\ circumfacientem tam facile refrigerat \textit{ (2) }\ circumfusum tam facile rarefacit \textit{ L}}}  digitus? Quia scilicet vis caloris\protect\index{Sachverzeichnis}{vis!caloris} maxima est, et innumerabilibus  exiguis sclopetis\protect\index{Sachverzeichnis}{sclopetum} displosis comparari potest. Negari tamen \textso{non} potest aeris gravitatem\protect\index{Sachverzeichnis}{gravitas!aeris} obsistere nonnihil, ac proinde Thermometrum\protect\index{Sachverzeichnis}{thermometrum} apertum duobus dominis obedire aeris gravitati\protect\index{Sachverzeichnis}{gravitas!aeris},  et caloris\protect\index{Sachverzeichnis}{calor} displosi Elaterio\protect\index{Sachverzeichnis}{elaterium}.\pend \pstart  Tertiam Aeris qualitatem dixi esse \textso{Cohaesionem,} et ab  hac phaenomenon istud nobile Liquoris\protect\index{Sachverzeichnis}{liquor!purgatus} aere purgati,  ultra horizontem solito altius assurgentis repetendum  puto. \edtext{Observandum enim est}{\lemma{puto.}\Afootnote{ \textit{ (1) }\ Sciendum est ergo \textit{ (2) }\ Observandum enim est \textit{ L}}}  tria haec liquorum genera: aerem, aquam et Mercurium\protect\index{Sachverzeichnis}{mercurius}  ea in re mire differre: \edtext{Etsi}{\lemma{differre:}\Afootnote{ \textit{ (1) }\ aqua  \textit{(a)}\ facillime diffluit in omnes  et \textit{(b)}\ diffluit in  \textit{(aa)}\ omnes partes, et \textit{(bb)}\ omne latus, et dissilit in partes  plures. \textit{ (2) }\ aqua facillime diffluit et dissilit, \textit{ (3) }\ aquae partes quoque  minimae, quantum nobis consequi licet, habent fluiditatem  totius  \textit{(a)}\ dicunturque \textit{(b)}\ diffluuntque in angulos omnes, et si opus sit  a toto separantur. Hi \textit{ (4) }\ Etsi \textit{ L}}} enim aer sit rarior aqua,  et aqua Mercurio\protect\index{Sachverzeichnis}{mercurius}, certum est tamen aquam nonnunquam facilius  in exiguas corporum rimas penetrare, quam aerem, et Mercurium\protect\index{Sachverzeichnis}{mercurius}  quam aquam. De Mercurio\protect\index{Sachverzeichnis}{mercurius} habemus experimenta multa: ne  memorem decantatam illam \edtext{suspectamque nonnihil vaporis mercurialis}{\lemma{illam}\Afootnote{ \textit{ (1) }\ Mercurii\protect\index{Sachverzeichnis}{mercurius|textit} \textit{ (2) }\ suspectamque nonnihil vaporis mercurialis \textit{ L}}} per  ipsum corpus humanum penetrationem; quo annulum \edtext{ore contentum}{\lemma{ore}\Afootnote{\textbar\ clauso \textit{ gestr.}\ \textbar\ contentum \textit{ L}}} infici ajunt, si nudum tantum  pedem Mercurio\protect\index{Sachverzeichnis}{mercurius} imponas, quod quidam ad  \edtext{Mercurium\protect\index{Sachverzeichnis}{mercurius} nescio quem}{\lemma{Mercurium}\Afootnote{ \textit{ (1) }\ quendam \textit{ (2) }\ nescio quem \textit{ L}}} Antimonialem restringunt.  Cogitemus tantum miram illam Mercurii\protect\index{Sachverzeichnis}{mercurius} per corium expressionem:  consideremus metalla\protect\index{Sachverzeichnis}{metallum} gigni in  lapidibus minerarum\protect\index{Sachverzeichnis}{minera}, vapore quodam Mercuriali penetratis,  quod nulla aqua possit, cogitemus amalgama\protect\index{Sachverzeichnis}{amalgama} Mercurii\protect\index{Sachverzeichnis}{mercurius}  ipsum compactissimum aurum\protect\index{Sachverzeichnis}{aurum} dissolvere: adde  odorem plumbi\protect\index{Sachverzeichnis}{plumbum} argentique\protect\index{Sachverzeichnis}{argentum}, \edtext{qui certe mercurialis quidam vapor est}{\lemma{}\Afootnote{qui [...] est \textit{ erg.} \textit{ L}}} mira penetrandi vi praeditam  esse, \edtext{plumbum\protect\index{Sachverzeichnis}{plumbum} lebetes optime munitos perforat}{\lemma{}\Afootnote{plumbum\protect\index{Sachverzeichnis}{plumbum} lebetes optime munitos perforat \textit{ erg.} \textit{ L}}}. Est quoddam caementationis genus quo cinnabaris\protect\index{Sachverzeichnis}{cinnabaris} odore argenti\protect\index{Sachverzeichnis}{argentum}  pervaditur tingiturque \edtext{rarissime}{\lemma{}\Afootnote{rarissime \textit{ erg.} \textit{ L}}}, etsi tantundem argenti\protect\index{Sachverzeichnis}{argentum} vicissim perdatur; \edtext{idem argenti odor per  fortissimum cementum penetrat in stannum}{\lemma{perdatur;}\Afootnote{ \textit{ (1) }\ etiam stannum\protect\index{Sachverzeichnis}{stannum|textit} certo modo ar \textit{ (2) }\ argentum\protect\index{Sachverzeichnis}{argentum|textit} quoque \textit{ (3) }\ idem [...] stannum \textit{ L}}} sibi immersum. Et  sunt rationes efficiendi ut metalla\protect\index{Sachverzeichnis}{metallum} ipsum vitri corpus penitus \edtext{pervadant tingantque}{\lemma{penitus}\Afootnote{ \textit{ (1) }\ penetrent \textit{ (2) }\ tingantur \textit{ (3) }\ pervadant tingantque \textit{ L}}}.\pend \pstart  Aqua vicissim ipso aere penetrantior est quod vulgaribus  utique experimentis constat corium \edtext{enim}{\lemma{}\Afootnote{enim \textit{ erg.} \textit{ L}}} (nulla pinguedine munitum)  aerem facilius continet, quam aquam: aqua in \edtext{plumam tenuissimam}{\lemma{in}\Afootnote{ \textit{ (1) }\ canna \textit{ (2) }\ plumam tenuissimam \textit{ L}}} ultra suum horizontem assurgit,  quia aeri introitus tam facilis non est. Aer in experimento 
 Torricelliano\protect\index{Sachverzeichnis}{experimentum!Torricellianum} per medium Mercurium\protect\index{Sachverzeichnis}{mercurius} non penetrat, etiamsi  longissimo tempore suspensus relinquatur. \edtext{At communicatum}{\lemma{At}\Afootnote{ \textit{ (1) }\ compertum \textit{ (2) }\  communicatum \textit{ L}}} est mihi experimentum amici  notabile, \edtext{ex quo sequitur}{\lemma{notabile,}\Afootnote{ \textit{ (1) }\ quo constat \textit{ (2) }\ ex quo sequitur \textit{ L}}} aquam per ipsum Mercurii\protect\index{Sachverzeichnis}{mercurius} corpus, aut  per vitri Mercuriique\protect\index{Sachverzeichnis}{mercurius} commissuram penetrare.  Is siphonem\protect\index{Sachverzeichnis}{sipho} sumsit bicrurum eumque ita statuit,  ut crura sursum verterentur. Fuit autem alterum siphonis\protect\index{Sachverzeichnis}{sipho} crus altero multo longius. Cruri minori \edtext{Mercurium\protect\index{Sachverzeichnis}{mercurius}, infudit}{\lemma{Mercurium,}\Afootnote{ \textit{ (1) }\ immisit \textit{ (2) }\ infudit \textit{ L}}}, majori aquam, ita ut aqua tanto esset altior Mercurio\protect\index{Sachverzeichnis}{mercurius}, quanto Mercurius\protect\index{Sachverzeichnis}{mercurius} gravior aqua, atque ita  hinc Mercurius\protect\index{Sachverzeichnis}{mercurius} illinc aqua in bilance \edtext{librabantur. Cum aliquandiu ita}{\lemma{bilance}\Afootnote{ \textit{ (1) }\ tenebantur.  Cum aliquot  \textit{(a)}\ noctibus \textit{(b)}\ diebus ita \textit{ (2) }\ librabantur. Cum aliquandiu ita \textit{ L}}} \edtext{reliquisset,}{\lemma{reliquisset,}\Afootnote{  \textbar\ paulatim \textit{ gestr.}\ \textbar\ aqua \textit{ L}}} aqua  tandem in alterum latus per medium Mercurium\protect\index{Sachverzeichnis}{mercurius} evasit, Mercurio\protect\index{Sachverzeichnis}{mercurius} in locum suum naturalem id est fundum  seu commissuram crurum delapso. Quid futurum esset oleo \edtext{quodam aqua graviore aliove liquore  medio}{\lemma{oleo}\Afootnote{ \textit{ (1) }\ medio aliove liquore \textit{ (2) }\ quodam [...] medio \textit{ L}}} interposito inter aquam et Mercurium\protect\index{Sachverzeichnis}{mercurius}, experimento non indignum foret.