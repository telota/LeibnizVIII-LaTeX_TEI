[100 v\textsuperscript{o}] \edtext{Restant ex}{\lemma{succedit.}\Afootnote{ \textit{ (1) }\ Ex eo \textit{ (2) }\ Restant ex \textit{ L}}} effectibus fugae Vacui\protect\index{Sachverzeichnis}{fuga vacui} ascriptis initio enumeratis suspensio aquae in vase unius tantum aperturae, difficultas in embolo\protect\index{Sachverzeichnis}{embolus}, ex tubo ab opposito latere clauso, extrahendo, aut in folle clauso diducendo; ac denique liquoris per antliam\protect\index{Sachverzeichnis}{antlia} elevatio. Haec omnia ego assero quemadmodum \edtext{Tabularum politarum}{\lemma{quemadmodum}\Afootnote{ \textit{ (1) }\ laminarum \textit{ (2) }\ Tabularum politarum \textit{ L}}} cohaerentium Siphonisque\protect\index{Sachverzeichnis}{sipho} aequicruri phaenomenon, \edtext{non}{\lemma{\hspace{2mm}phaenomenon,}\Afootnote{\textbar\ non \textit{ erg.} \textbar\ a \textit{ L}}} a sola aeris pressione\protect\index{Sachverzeichnis}{pressio!aeris} non pendere, non magis quam a Fuga Vacui\protect\index{Sachverzeichnis}{fuga vacui} veteribus decantata, sed veram horum omnium effectuum rationem esse Repugnantiam naturae contra difformitatem, qua scilicet ferre \edtext{nisi per vim,}{\lemma{}\Afootnote{nisi per vim, \textit{ erg.} \textit{ L}}} non potest ut in loco aliquo plus sit materiae subtilioris\protect\index{Sachverzeichnis}{materia!subtilis}, \edtext{aut}{\lemma{}\Afootnote{aut \textit{ erg.} \textit{ L}}} crassioris quam in caeteris circumjacentibus. Quod a circulatione Aetheris\protect\index{Sachverzeichnis}{aether} universali\protect\index{Sachverzeichnis}{circulatio!universalis} \edtext{per difformitatem turbata}{\lemma{}\Afootnote{per difformitatem turbata \textit{ erg.} \textit{ L}}} oriri, et hinc Gravitatis\protect\index{Sachverzeichnis}{gravitas} Elateriique\protect\index{Sachverzeichnis}{elaterium} causas, et omnem omnino conatum\protect\index{Sachverzeichnis}{conatus} ad aequilibrium\protect\index{Sachverzeichnis}{aequilibrium} pendere, in Hypothesi a me aliquando publicata satis opinor clare ostensum est.\edtext{}{\lemma{est.}\Bfootnote{\textsc{G. W. Leibniz, }\cite{00256}\textit{Hypothesis physica nova}, Mainz 1671, § 27 (\cite{00256}\textit{LSB} VI, 2, N. 40 § 27).}} Ubi \edtext{illud quoque annotavi}{\lemma{Ubi}\Afootnote{ \textit{ (1) }\ ostendi illud quoque \textit{ (2) }\ illud quoque \textit{(a)}\ annotatum est \textit{(b)}\ annotavi \textit{ L}}}, quicquid denique de Gravitate\protect\index{Sachverzeichnis}{gravitas} aut Elaterio\protect\index{Sachverzeichnis}{elaterium} aeris dicamus aut experiamur, necessario eundum esse ad causam altiorem ipsius Gravitatis\protect\index{Sachverzeichnis}{gravitas} Elateriique\protect\index{Sachverzeichnis}{elaterium} in universum\protect\index{Sachverzeichnis}{universum}, \edtext{cujus}{\lemma{universum,}\Afootnote{ \textit{ (1) }\ Gravitati Aeris\protect\index{Sachverzeichnis}{gravitas!aeris|textit} soli ex iis \textit{ (2) }\ cujus \textit{ L}}} ea quam in aere sentimus non nisi consequentia est.\pend 
\clearpage
\pstart Ex his igitur phaenomenis quae nunc communiter Aeris gravitati\protect\index{Sachverzeichnis}{gravitas!aeris}\edtext{}{\lemma{gravitati}\Afootnote{ \textbar\ Elateriove\protect\index{Sachverzeichnis}{elaterium} \textit{ gestr.}\ \textbar\ ascribuntur, \textit{ L}}} ascribuntur, solum \edtext{Baroscopium}{\lemma{solum}\Afootnote{ \textit{ (1) }\ Experimentum Torricellianum\protect\index{Sachverzeichnis}{experimentum!Torricellianum|textit} \textit{ (2) }\ Baroscopium \textit{ L}}} ei integre deberi puto, et praeterea aliud sed quod ei hactenus quantum meminerim ascriptum non est, \edtext{\textso{frigus} scilicet}{\lemma{est,}\Afootnote{ \textit{ (1) }\ Frigus\protect\index{Sachverzeichnis}{frigus|textit} scilicet \textit{ (2) }\ omnem frigoris sensum\protect\index{Sachverzeichnis}{sensus!frigoris|textit} tale \textit{ (3) }\ \textso{frigus} scilicet \textit{ L}}} aeris \edtext{refrigerationem, ubi}{\lemma{aeris}\Afootnote{ \textit{ (1) }\ ubi fateor clarissimum Petitum\protect\index{Namensregister}{\textso{Petit} (Petitus), Pierre 1598\textendash 1677|textit} nuper docuisse, et recte frigus\protect\index{Sachverzeichnis}{frigus|textit} ab aere esse \textit{ (2) }\ refrigerationem, ubi \textit{ L}}} \edtext{a calore}{\lemma{ubi}\Afootnote{ \textit{ (1) }\ scilicet a sole igneve\protect\index{Sachverzeichnis}{ignis|textit} \textit{ (2) }\ a calore \textit{ L}}} rarefieri desiit. Cum enim aer corporibus\edtext{}{\lemma{}\Afootnote{corporibus \textbar\ nostro \textit{ gestr.}\ \textbar\ circumjacens \textit{ L}}} circumjacens calore\protect\index{Sachverzeichnis}{calor} rarefit, etiam corporum ipsorum aer evocatur, etiamsi \edtext{enim}{\lemma{enim}\Afootnote{ \textit{ erg.} \textit{ L}}} ipse non rarefiat \edtext{a circumjacente, rarefacit tamen ipse sese.}{\lemma{}\Afootnote{a circumjacente, rarefacit tamen ipse sese. \textit{ erg.} \textit{ L}}} Hinc in nobis caloris sensus\protect\index{Sachverzeichnis}{sensus!caloris}, in aliis corporibus maturatio putredo, aliique effectus. Ut enim \edtext{vesica flaccida in}{\lemma{enim}\Afootnote{ \textit{ (1) }\ aer in \textit{ (2) }\ vesica flaccida in \textit{ L}}} Vacuo Magdeburgico dilatatur, \edtext{ita}{\lemma{dilatatur,}\Afootnote{ \textit{ (1) }\ ubi \textit{ (2) }\ ita \textit{ L}}} necesse est aerem corporum sanguini\protect\index{Sachverzeichnis}{sanguis} 
\edtext{succisve}{\lemma{sanguini}\Afootnote{ \textit{ (1) }\ eorum \textit{ (2) }\ succisve \textit{ L}}} per infinitas bullas inclusum, Elaterio\protect\index{Sachverzeichnis}{elaterium} proprio se dilatare, ubi ab aere circumjacente quippe a calore\protect\index{Sachverzeichnis}{calor} dilatato, comprimi desiit. Hinc apertura pororum, et sanguinis\protect\index{Sachverzeichnis}{sanguis} ebullitio, et spirituum\protect\index{Sachverzeichnis}{spiritus} evocatio, et sensus caloris\protect\index{Sachverzeichnis}{sensus!caloris} blandus, aut dolor intolerabilis. Ut enim vesica in Vacuo Magdeburgico mediocri Recipientis evacuatione inflatur, at nimia rumpi potest, ita nimia rarefactio aeris circumjacentis dat nostro nimiam se tendendi ac denique claustra perrumpendi libertatem, unde dolor. \edtext{Contra}{\lemma{dolor.}\Afootnote{ \textit{ (1) }\ Nec puto \textit{ (2) }\ Contra \textit{ L}}} ubi cessat calor\protect\index{Sachverzeichnis}{calor} aerem dilatans, aer ipse proprio totius massae\protect\index{Sachverzeichnis}{massa} pondere comprimitur in statum priorem, \edtext{hinc}{\lemma{priorem,}\Afootnote{ \textit{ (1) }\ unde sequitur: calorem\protect\index{Sachverzeichnis}{calor|textit} esse natu \textit{ (2) }\ hinc \textit{ L}}} corpora quoque \edtext{aut potius}{\lemma{quoque}\Afootnote{ \textit{ (1) }\ et \textit{ (2) }\ aut potius \textit{ L}}} inclusus ipsis aer \edtext{denuo}{\lemma{aer}\Afootnote{ \textit{ (1) }\ denique \textit{ (2) }\ denuo \textit{ L}}} comprimuntur; nam revera nihil pene \edtext{aliud quam}{\lemma{pene}\Afootnote{ \textit{ (1) }\ nisi \textit{ (2) }\ aliud quam \textit{ L}}} inclusus corporibus aer comprimitur; constat enim aquam ipsam et adhuc magis terram \edtext{compressionis vix}{\lemma{terram}\Afootnote{ \textit{ (1) }\ nullius pene \textit{ (2) }\ rarefacti \textit{ (3) }\ compressionis  \textbar\ vix \textit{ erg.}\ \textbar\ \textit{ L}}} esse capaces nisi quatenus ipsis inest aer. Unde tentari vellem \edtext{hoc \textso{Experimentum}}{\lemma{}\Afootnote{hoc \textso{Experimentum} \textit{ erg.} \textit{ L}}}\footnote{\textit{In der rechten Spalte}: \textso{Experimentum 2}} an et quatenus aqua aere purgata geletur, quod discere poterimus si maximi frigoris\protect\index{Sachverzeichnis}{frigus} tempore aquam in Recipiente Magdeburgico\protect\index{Sachverzeichnis}{Recipiens!Magdeburgicum} exhausto aliquandiu relinquamus. Ergo in nobis quoque aliisque corporibus nihil aliud a frigore\protect\index{Sachverzeichnis}{frigus} quam aer inclusus, comprimetur. Ex his colligo Calorem\protect\index{Sachverzeichnis}{calor} esse naturalem singulis aeris partibus sibi relictis, Frigus\protect\index{Sachverzeichnis}{frigus} vero esse naturale massae\protect\index{Sachverzeichnis}{massa} aereae sibi relictae, versus fundum; aerem enim illic positum a massa\protect\index{Sachverzeichnis}{massa} superincumbente comprimi; a se ipso autem \edtext{alias}{\lemma{}\Afootnote{alias \textit{ erg.} \textit{ L}}} proprio Elaterio\protect\index{Sachverzeichnis}{elaterium} dilatari necesse est. Unde patet quo sensu aer primum frigidum appellari possit. Patet etiam cur corpora omnia pressa angustorumque pororum sint contactu frigida, quia aer in his angustiis minus liber magisque pressus est; quemadmodum venti in angiportibus sunt frigidiusculi; et aqua in recessibus angustis, sinubusque exiguis minus agitata, et aer ipse frigidior in recessibus umbrosis.\pend 
\pstart Quod attinet vero effectus\edtext{}{\lemma{}\Afootnote{effectus \textbar\ reliquos \textit{ erg. u.}\ \textit{ gestr.}\ \textbar\ soli \textit{ L}}} soli aeris pressioni\protect\index{Sachverzeichnis}{pressio!aeris} \edtext{communiter nostro tempore}{\lemma{pressioni}\Afootnote{ \textit{ (1) }\ vulgo \textit{ (2) }\ communiter nostro tempore \textit{ L}}} sed non recte ascriptos, \edtext{praeter}{\lemma{ascriptos,}\Afootnote{ \textit{ (1) }\ qui scilicet post remotas \textit{ (2) }\ praeter \textit{ L}}} tractatos jam duos Laminarum \edtext{politarum, et Siphonis iniquicruri}{\lemma{Laminarum}\Afootnote{ \textit{ (1) }\ cohaesionem, et Siphonem\protect\index{Sachverzeichnis}{sipho|textit} iniquicrurum \textit{ (2) }\ politarum, et Siphonis iniquicruri \textit{ L}}}; ajo de caeteris idem esse judicandum, aeris \edtext{gravitatem}{\lemma{aeris}\Afootnote{ \textit{ (1) }\ compressionem \textit{ (2) }\ gravitatem \textit{ L}}} conferre\edtext{}{\lemma{conferre}\Afootnote{\textbar\ multum \textit{ gestr.}\ \textbar\ , sed \textit{ L}}}, sed non esse causam\edtext{}{\lemma{}\Afootnote{causam \textbar\ solam \textit{ gestr.}\ \textbar\ . Ac \textit{ L}}}. Ac primum quod attinet liquoris aut alterius corporis gravis (ut emboli\protect\index{Sachverzeichnis}{embolus} exacte intrantis) in vase\edtext{}{\lemma{}\Afootnote{vase \textbar\ clauso \textit{ gestr.}\ \textbar\ unius \textit{ L}}} unius tantum aperturae suspensionem hanc certum est itidem in Vacuo Magdeburgico evenire, quando scilicet liquor est ab aere purgatus\edtext{, ut}{\lemma{purgatus}\Afootnote{ \textit{ (1) }\ . Imo amplius accidit \textit{ (2) }\ , ut \textit{ L}}} ab Hugenio\protect\index{Namensregister}{\textso{Huygens} (Hugenius, Vgenius, Hugens, Huguens), Christiaan 1629\textendash 1695} primum observatum est,\edtext{}{\lemma{est,}\Bfootnote{\textsc{Chr. Huygens, }\cite{00062}a.a.O., S.~134f. (\textit{HO} VII, S.~202).}} observavit et Boylius\protect\index{Namensregister}{\textso{Boyle} (Boylius, Boyl., Boyl), Robert 1627\textendash 1691} idem evenire Mercurio\protect\index{Sachverzeichnis}{mercurius} ab aere purgato in Tubo Torricelliano\protect\index{Sachverzeichnis}{Tubus!Torricellianus} seu Baroscopio\protect\index{Sachverzeichnis}{baroscopium} ut plane non descenderit, etsi in 75 pollicum altitudine experimentum sit captum, cum alioqui ex 27 \edtext{}{\lemma{}\Afootnote{ex 27 \textbar\ aut \textit{ gestr.}\ \textbar\ [(30)] \textit{ L}}}[(30)]\edtext{}{\lemma{30}\Afootnote{\textit{\ L \"{a}ndert Hrsg.}}} pollicum altitudine tantum suspendatur %\edtext{
\edlabel{suspendaturstart}in aere ordinario,\edtext{}{\lemma{ordinario}\Bfootnote{\textsc{R. Boyle}, \cite{00015}a.a.O., S.~68, 84 (\textit{BW} I, S.~192, 200).}} et in Recipiente Magdeburgico, ex\edlabel{suspendaturend} \edtext{nulla}{\lemma{}\xxref{suspendaturstart}{suspendaturend}\Afootnote{in aere ordinario, et in \textit{ (1) }\ Vacuo \textit{ (2) }\ Recipiente Magdeburgico, ex nulla \textit{ erg.} \textit{ L}}}. Ratio diversitatis inter \edtext{liquorem aere}{\lemma{inter}\Afootnote{ \textit{ (1) }\ aerem \textit{ (2) }\ liquorem aere \textit{ L}}} purgatum et non purgatum facilis captu est. Liquor enim aerem quem continet gravitate\protect\index{Sachverzeichnis}{gravitas} sua exprimit, \edtext{ad locum quem descensu suo relicturus est implendum.}{\lemma{exprimit,}\Afootnote{ \textit{ (1) }\ et locum quem descensu suo relicturus est implere cogit, et si in statu naturali ad implendum non sufficit, dilatat eum ac rarefacit, prorsus ut funis pondere appenso diducitur ut adeo hic demum locum habeat quiddam Francisci Lini\protect\index{Namensregister}{\textso{Linus,} Franciscus 1595\textendash 1675|textit} funiculo\protect\index{Sachverzeichnis}{funiculus|textit} simile \textit{ (2) }\ ad [...] implendum. \textit{ L}}} Et quia aer tendi seu dilatari potest,\rule[-100mm]{0mm}{0mm} hinc bulla etiam exigua Tubum implere potest unde Liquori descensus permittitur.\edlabel{permi100v}
%{\lemma{permittitur.}\xxref{permi100v}{permi101r}\Afootnote{ \textit{ (1) }\ Hujus rei experimentum capi poterit hoc modo: \textit{(a)}\ Sumatur \textit{(aa)}\ Mercurius\protect\index{Sachverzeichnis}{mercurius|textit} \textit{(bb)}\ liquor aere purgatus, \textit{(aaa)}\ introducatur \textit{(bbb)}\ immittatur ei exigua aeris bullula, sit tubi altitudo quanta maxima commode haberi potest  \textbar\ quam tamen tanto minorem esse sufficit, quanto liquor est gravior, et minimam pro Mercurio\protect\index{Sachverzeichnis}{mercurius|textit} \textit{ erg.}\ \textbar\ ; bulla introducta a liquore expressa liquoris descendere nitentis gravitate\protect\index{Sachverzeichnis}{gravitas|textit} velut elisa locum a liquore replendum implere tentabit \textit{(b)}\ Sumatur Tubus quam longissimus (etsi pro Mercurio\protect\index{Sachverzeichnis}{mercurius|textit} sufficiat minor quam pro aqua) \textit{AB} \textit{(aa)}\ plenus liquo \textit{(bb)}\ clausus in \textit{A} in quem liquor (ut Mercurius\protect\index{Sachverzeichnis}{mercurius|textit}) immittatur per \textit{(aaa)}\ vas apertum \textit{(bbb)}\ orificium apertum sursum conversum \textit{B} ita tamen ut tubus multum absit a pleno si invertatur Tubus ita ut orificium  \textbar\ \textit{B} \textit{ erg.}\ \textbar\ intret in vas eodem liquore stagnans \textit{C}. 
%Si Mercurius\protect\index{Sachverzeichnis}{mercurius|textit} \textit{(aaaa)}\ fuit aqua purgatus \textit{(bbbb)}\ aere purgatus non est, delabetur in vas \textit{C} nec nisi 27 circiter pollicibus  \textbar\ qui repraesententur altitudine \textit{BD} \textit{ erg.}\ \textbar\ ultra ejus superficiem eminebit, \textit{(aaaaa)}\ sin aere purgatus sit totus suspensus manebit, nisi forte Tubi \textit{(bbbbb)}\ locus autem in Tubo \textit{AD} ad sensum vacuus, reapse aere ex corpore Mercurii\protect\index{Sachverzeichnis}{mercurius|textit} (ere non purgati) expresso impletus erit, sed eo dilatato seu rarefacto, quod ex eo colligi potest quia aer externus \textbar\ mox \textit{ gestr.}\ \textbar\ foramine in \textit{A} aperto, (ut si vesica obligatam sit, quae acicula perforetur) irrumpit; tum quia aeris externi pressio\protect\index{Sachverzeichnis}{pressio!aeris|textit} Mercurium\protect\index{Sachverzeichnis}{mercurius|textit} ad locum replendum sursum repellere conatur, quod non faceret si is a tergo aeque seu in Tubo ac ante se seu extra tubum aerem aequalis pressionis seu Elaterii\protect\index{Sachverzeichnis}{elaterium|textit} haberet. Hinc sequitur etiam non posse Mercurium\protect\index{Sachverzeichnis}{mercurius|textit} suspensum praecise aequiponderare massae\protect\index{Sachverzeichnis}{massa|textit} aereae in aere libero, aut Elaterio\protect\index{Sachverzeichnis}{elaterium|textit} seu pressioni aeris\protect\index{Sachverzeichnis}{pressio!aeris|textit} clausi, \textit{(aaaaa-a)}\ nisi ei adjiciatur \textit{(bbbbb-b)}\ sed detrahendam esse ab ejus pondere vim funiculi\protect\index{Sachverzeichnis}{funiculus|textit} seu \textit{(aaaaa-aa)}\ vim qua aer \textit{(bbbbb-bb)}\ Elaterium\protect\index{Sachverzeichnis}{elaterium|textit} quo \textbar\ aeris \textit{ gestr.}\ \textbar\  in \textit{(aaaaa-aaa)}\ Tubo \textit{(bbbbb-bbb)}\ loco per descensum vacuato nimis dilatatus se contrahere nititur, \textit{(ccccc)}\ nisi ei addatur pondus aeris\protect\index{Sachverzeichnis}{pondus!aeris|textit} in Tubo post tergum relicti \textit{DA}. Quod si jam aer Elaterio\protect\index{Sachverzeichnis}{elaterium|textit} proprio se expandere potest in infinitum, (modo scilicet nihil sit quod eum comprimat,) ita ut \textit{(aaaaa-a)}\ gutta \textit{(bbbbb-b)}\ bulla exigua aeris implere seu  \textbar\ (in Recipiente exhausto) \textit{ erg.}\ \textbar\  tendere possit vesicam diametri quantaecunque; eadem evenient, quantacunque sit Tubi altitudo, et quantulacunque bulla aeris in liquore relicta, aut ei immissa sit, sufficiet enim loco in Tubo Vacuo relicto quantocunque implendo et Mercurius\protect\index{Sachverzeichnis}{mercurius|textit} semper descendet ad altitudinem usque consuetam. Quodsi aliquando aer ad terminos pervenit ultra quos expandi non potest  \textbar\ aut non facile potest, id est si aer resistit dilatanti, (quod hactenus deprehendi non potuit) uti resistit comprimenti \textit{ erg.}\ \textbar\ , potest tubus tam longus cogitari (etsi incertum hactenus an et opere obtineri) ut Mercurius\protect\index{Sachverzeichnis}{mercurius|textit} descendere non possit, sed ita suspensus maneat  \textbar\ altius solito \textit{ erg.}\ \textbar\ ,  \textit{(aaaaa-aa)}\ ne aerem ultra debitam dilatet \textit{(bbbbb-bb)}\ ne longius descendendo spatium justo majus \textit{(aaaaa-aaa)}\ a tergo relinquat aeremve \textit{(bbbbb-bbb)}\ in Tubo aeremque qui implere debet nimis dilatet. \textit{ (2) }\  Quod aeri quantulocunque facile est, quia quantulacunque aeris guttula in Vacuo \textit{(a)}\ maximam \textit{(b)}\ satis exhausto quantamcunque vesicam tendere potest. Aer enim resistit quidem comprimenti sed non dilatanti, ac dilatari potest in infinitum  \textbar\ quantum sensu judicari queat \textit{ erg.}\ \textbar\ , Elaterio\protect\index{Sachverzeichnis}{elaterium|textit} proprio, id est circulationis generalis omnia in summam subtilitatem  \textbar\ qualis aetheris\protect\index{Sachverzeichnis}{aether|textit} circulantis est \textit{ erg.}\ \textbar\ si possit disjicere conantis, vi, modo scilicet non tantundem alibi comprimatur, nihil enim sine compensatione fieri potest. \textit{ (3) }\ At [...] possit, \textit{(a)}\ supposito quod liquor ipse nihil aliud Elasticum nobis compertum contineat, \textit{(b)}\ nisi [...] liquoris \textbar\ maxima vi \textit{ gestr.}\ \textbar\ eliciatur, [...] intret. \textit{(aa)}\ Sed cur \textit{(bb)}\ Ergo [...] corporis \textbar\ interni \textit{ erg.}\ \textbar\ expressionem [...] est \textit{(aaa)}\ aeris aequipondio in aere \textit{(bbb)}\ ejus [...] insuetae, \textit{(aaaa)}\ aetheris\protect\index{Sachverzeichnis}{aether|textit} circulatio unif \textit{(bbbb)}\ circulatio [...] resistit. \textit{ L}}}