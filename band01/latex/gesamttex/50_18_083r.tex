      
               
                \begin{ledgroupsized}[r]{120mm}
                \footnotesize 
                \pstart                
                \noindent\textbf{\"{U}berlieferung:}   
                \pend
                \end{ledgroupsized}
            
              
                            \begin{ledgroupsized}[r]{114mm}
                            \footnotesize 
                            \pstart \parindent -6mm
                            \makebox[6mm][l]{\textit{LiA}}Marginalien und Erg\"{a}nzungen in \textsc{J. Hudde}, \cite{00125}\textit{Specilla circularia}: LH XXXVII 2 Bl. 83\textendash 91. 5 Bog. 4\textsuperscript{o}. 17 1/3 S. Bl. 92 leer. Bl. 84 r\textsuperscript{o} im unteren Drittel der Seite die Zeichnungen fig. 1, fig. 2 und \textit{[Fig. 3]}. In der Hannoveraner Abschrift (siehe Datierungsbegr\"{u}ndung) fehlen gegen\"{u}ber der Londoner Abschrift Teile des Textes. Dem Schreiber ist bei der Herstellung der Kopie zudem die Seitenfolge durcheinander geraten. Diese und andere Unkorrektheiten werden von Leibniz durch Marginalien und kleinere Texteingriffe korrigiert. Da das Verst\"{a}ndnis der Marginalien die Kenntnis gr\"{o}ßerer Textteile erfordert, wird im Folgenden der gesamte Text abgedruckt.\pend
                            \end{ledgroupsized}
              
                            \begin{ledgroupsized}[r]{114mm}
                            \footnotesize 
                            \pstart \parindent -6mm
                            \makebox[6mm][l]{\textit{E}}\textsc{R. Vermij / E. Atzema}, \cite{00194}\textit{Specilla circularia: an unknown work by Johannes Hudde}, in: \textit{Studia Leibnitiana}, Bd. XXVII/1 (1995) S.~104\textendash 121. Huddes Text S.~113\textendash121. \pend
                            \end{ledgroupsized}
                %\normalsize
                \vspace*{5mm}
                \begin{ledgroup}
                \footnotesize 
                \pstart
            \noindent\footnotesize{\textbf{Datierungsgr\"{u}nde}: Wie die Herausgeber von \textit{E} mitteilen, ist der Text \textit{Specilla circularia}, dessen Druck offenbar verloren gegangen ist, in zwei Abschriften \"{u}berliefert. Eine Abschrift befindet sich in der Royal Society in London, die zweite ist unsere Druckvorlage. Die Londoner Abschrift enth\"{a}lt am Ende den Zusatz: Huddenius\protect\index{Namensregister}{\textso{Hudde} (Huddenius), Jan 1628\textendash 1704} Consul Amstelodamensis\protect\index{Ortsregister}{Amsterdam}, aus dem die Herausgeber schliessen, dass diese Abschrift 1672 oder sp\"{a}ter angefertigt worden sein muss. Die Datierung deckt sich mit dem  Wasserzeichen des Texttr\"{a}gers unserer Druckvorlage, das f\"{u}r M\"{a}rz 1672 nachgewiesen ist.}
                \pend
                \end{ledgroup}
            
                \vspace*{8mm}
                \pstart 
                \normalsize
            [83 r\textsuperscript{o}] Specilla Circularia\protect\index{Sachverzeichnis}{specillum!circulare}, sive quomodo per solas  Circulares figuras fieri possint omnis generis specilla\protect\index{Sachverzeichnis}{specillum}  tam Microscopia\protect\index{Sachverzeichnis}{microscopium} quam telescopia\protect\index{Sachverzeichnis}{telescopium}, etc: eundem plane  effectum habentia, aut saltem quam proxime accedentem  ad eorum, quae per ellipticas aut hyperbolicas figuras  fieri possent. \pend \pstart  Notum jam omnibus satis est, quanta sit specillorum\protect\index{Sachverzeichnis}{specillum}  utilitas: myopes\protect\index{Sachverzeichnis}{myops} alias et senes novaque, post inventa microscopia\protect\index{Sachverzeichnis}{microscopium}, et telescopia\protect\index{Sachverzeichnis}{telescopium}, tam in coelis quam, \edtext{hic}{\lemma{}\Afootnote{hic \textit{ erg.} \textit{ L}}},\pend \pstart\noindent in terra, circa  nos magna copia detecta objecta, luculentum sunt testimonium  sed multa adhuc magis admiranda quam ea quae hactenus detecta sunt promittere nobis videntur, imo procul omni  dubio horum ope ab astronomis motuum coelestium, a  Philosophis naturae corporum mixtorum; a Medicis  naturae et virium herbarum, et corporis humani, perfectior longe notitia haberi poterit, quam unquam absque his  expectanda foret. Cumque hoc publice constaret, plurimi  fuere jam brevi, qui maxima cum diligentia specilla\protect\index{Sachverzeichnis}{specillum}  haec ad summam perfectionem perducere conati sunt. Sed  nulli id, meo judicio, melius successit quam incomparabili  viro Renato Descartes\protect\index{Namensregister}{\textso{Descartes} (Cartesius, des Cartes, Cartes.), Ren\'{e} 1596\textendash 1650}, cujus labori nihil plane superaddi 