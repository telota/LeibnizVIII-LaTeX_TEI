[74~r\textsuperscript{o}]  $\langle$--$\rangle$ a fortissima, quae efficiat, $\langle$et ut$\rangle$ $\langle$r$\rangle$etineat versus septentrionem, $\langle$n$\rangle$avem \protect\index{Sachverzeichnis}{navis} ratione et globus et caetera omnia retineant eundem situm, et ipsa quoque ab iis pendens tabula. Quod ut fiat possunt et caetera omnia ubique aciculis similiter frictis adimpleri, aut potius si non superius globus, saltem inferius tabula. Quam superius globus sequetur perpetuo in septentrionem\footnote{\textit{Interlinear \"{u}ber} perpetuo in septentrionem: (Imo etsi superius globus non sequatur.)} dirigatur magnete\protect\index{Sachverzeichnis}{magnes} vel acicula, et per consequens toti machinae erit idem situs. Hoc jam praestito tota res confecta est. Nam nave\protect\index{Sachverzeichnis}{navis} flexa flectetur simul id quod punctum imprimit, machina non flectitur, et per consequens in oppositam flexionis partem flectetur tabula, et punctum impressum similiter declinabit, et per consequens in tabula totus navis\protect\index{Sachverzeichnis}{navis} motus, sed inverse spectandus exhibebitur nam quae plaga in navi\protect\index{Sachverzeichnis}{navis} est oriens in tabula erit occidens, quae septentrio ea meridies. In tabula praeterea designentur gradus tum longitudinis \protect\index{Sachverzeichnis}{longitudo} tum latitudinis\protect\index{Sachverzeichnis}{latitudo}, ut appareat quae sit declinatio\protect\index{Sachverzeichnis}{declinatio}. Sit praeterea aliquis qui diligenter qualibet septimana tabulae ascribat horas, et redigat multas tabulas in unam agglutinando, et deinde ex iis extractum delineationis \selectlanguage{ngerman}nach dem verj\"{u}ngtem maßstab\selectlanguage{latin} faciat. \edtext{Tabulae}{\lemma{faciat.}\Afootnote{ \textit{ (1) }\ Ea \textit{ (2) }\ Tabulae \textit{ L}}} vero in margine ascribi potest, quae tunc litora, quam promontoriorum, currentium faciem longitudinem\protect\index{Sachverzeichnis}{longitudo} diei, elevationem poli\protect\index{Sachverzeichnis}{elevatio!poli} observaverint \edtext{per cameras obscuras semper perpetuo situs mutatio designetur.}{\lemma{observaverint}\Afootnote{\textit{ (1) }\ per cam. obscur. statim designet \textit{ (2) }\ per [...] designetur. \textit{ erg.} \textit{ L}}} Ea ratione ad summam perfectionem veniet tandem Hydrographia\protect\index{Sachverzeichnis}{hydrographia}, et si idem in terra fiat, geographia perfecte loca designabuntur. Nec erit quicquam quod nos amplius miretur, praeter  unam magnetis\protect\index{Sachverzeichnis}{magnes} declinationem\protect\index{Sachverzeichnis}{declinatio}. Sed tamen et huic rei multorum observationes facile medebuntur. Quaelibet ita navis\protect\index{Sachverzeichnis}{navis} \pend\pstart\noindent  perfectam non solum formae, sed et celeritatis\protect\index{Sachverzeichnis}{celeritas} sui cursus delineationem exhibere poterit. Idem poterit facere currus in terra. Imo et homo simplex simile instrumentum portans, et \edtext{delineaturus cryptas}{\lemma{et}\Afootnote{ \textit{ (1) }\ mensuraturus \textit{ (2) }\ delineaturus cryptas \textit{ L}}}, aliaque loca non facile accessa. Tibi DEUS grates ago, \edtext{tuam}{\lemma{ago,}\Afootnote{ \textit{ (1) }\ quod \textit{ (2) }\ tuam \textit{ L}}} erga me misericordiam providentiamque agnosco, qui rem generi humano tam utilem mihi potissimum in mentem venire voluisti.\pend \pstart  NB. Utinam inveniretur alioquando magnes\protect\index{Sachverzeichnis}{magnes} aliquis  perpetuo se vertens ad solem\protect\index{Sachverzeichnis}{sol}, uti magnes\protect\index{Sachverzeichnis}{magnes} ad polum\protect\index{Sachverzeichnis}{polus}. Qua ratione possent longitudines\protect\index{Sachverzeichnis}{longitudo} sola supputatione inveniri, inveniretur enim quantitas diei, qua determinata nullo amplius horologio\protect\index{Sachverzeichnis}{horologium}  opus esset. \edtext{Hoc fortasse unus praestare potuisset Drebelius\protect\index{Namensregister}{\textso{Drebbel} (Drebelius, Drebel), Cornelius 1572\textendash 1633}. Sed etsi}{\lemma{esset.}\Afootnote{ \textit{ (1) }\ Sed interim etsi \textit{ (2) }\ Hoc [...] etsi \textit{ L}}} hoc haberetur, instrumenti tamen proxime delineati utilitates nondum exhauriret, quippe quod non tantum aliquando consulenti locum monstrat, sed perfecte totum navis\protect\index{Sachverzeichnis}{navis} cursum exhibet. Chymistae ferunt tincturam ejusmodi magnetem\protect\index{Sachverzeichnis}{magnes} esse et ad solem\protect\index{Sachverzeichnis}{sol} perpetuo converti. Sane Drebelium\protect\index{Namensregister}{\textso{Drebbel} (Drebelius, Drebel), Cornelius 1572\textendash 1633} in arcana chymica penetrasse ne⟨ga⟩ri non potest.\pend \pstart $\langle$Con$\rangle$jicies aerem cum sit instabilis mutationem etsi initio non sit sensibilis progressu temporis fore. Sed minime, quia quaelibet insen⟨sibilis⟩ declinatio\protect\index{Sachverzeichnis}{declinatio} non alteri⟨us⟩ cau⟨sae est⟩, sed est quaelibet per se ab altera 