      
               
                \begin{ledgroupsized}[r]{120mm}
                \footnotesize 
                \pstart                
                \noindent\textbf{\"{U}berlieferung:}   
                \pend
                \end{ledgroupsized}
            
              
                            \begin{ledgroupsized}[r]{114mm}
                            \footnotesize 
                            \pstart \parindent -6mm
                            \makebox[6mm][l]{\textit{LiH}}Marginalien, An- und Unterstreichungen in \textsc{G. Desargues}, \cite{00034}\textit{Mani\`{e}re universelle pour pratiquer la perspective}, Paris 1648: Leibn. Marg. 175. Mehrere Unterstreichungen mit Bleistift, die nicht eindeutig Leibniz zugeordnet werden k\"{o}nnen und daher keine Ber\"{u}cksichtigung finden. \pend
                            \end{ledgroupsized}
                %\normalsize
                \vspace*{5mm}
                \begin{ledgroup}
                \footnotesize 
                \pstart
            \noindent\footnotesize{\textbf{Datierungsgr\"{u}nde}: F\"{u}r die Datierung beziehen wir uns auf die Gespr\"{a}chsnotiz N. 28. Es ist anzunehmen, dass dieser ein Gespr\"{a}ch Leibniz' mit Mariotte\protect\index{Namensregister}{\textso{Mariotte,} Edme, Seigneur de Chazeuil ca. 1620\textendash 1684} vorausging, in dem Leibniz \"{u}ber seine eigene Desargues-Lekt\"{u}re berichtete. Die Entstehungszeit der Marginalien zu Desargues d\"{u}rfte sich daher mit dem Entstehungszeitraum von N. 28 decken.}
                \pend
                \end{ledgroup}
            
                \vspace*{8mm}
                \pstart 
                \normalsize
          \selectlanguage{ngerman} [Vakatseite: \textit{Notiz von Leibniz}] \selectlanguage{french}Figure fautive p. 86. de la perspective  cette methode n'est pas ass\'{e}s propre \`{a} eclairer l'esprit, parce qu'elle ne nous fait connoistre qu'\`{a} la fin les raisons pourquoy l'auteur nous mene comme cela. Elle n'est pas \edtext{si}{\lemma{}\Afootnote{si \textit{ erg.} \textit{ L}}} propre \`{a} l'invention mais elle a l'avantage de surprendre les lecteurs, quand ils se trouuent men\'{e}s \`{a} quelque chose sans y penser; et on retient mieux les choses qu'on admire. V. p. 57. 58. p. 83. fin. p. 84 fin.\pend \pstart  Dans la page 28 on ne voit pas bien \edtext{encore}{\lemma{}\Afootnote{encore \textit{ erg.} \textit{ L}}} la raison, pourquoy \textit{CZ} et \textit{EL} doiuent estre prises telles qu'il dit.\pend \pstart J'ay adjout\'{e} quelque chose, (marqu\'{e} de NB) p. 86.