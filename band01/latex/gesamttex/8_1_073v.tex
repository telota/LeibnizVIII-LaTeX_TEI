\pstart [73 v\textsuperscript{o}] \textso{Astronomica}. Hookii\protect\index{Namensregister}{\textso{Hooke} (Hookius, Hook), Robert 1635\textendash 1703} designatio observandi, Tellus\protect\index{Sachverzeichnis}{tellus} ne aliquando sensibiliter accederet \edtext{abscederetque fixis}{\lemma{accederet}\Afootnote{ \textit{ (1) }\ astris fixis\protect\index{Sachverzeichnis}{stella!fixa|textit} \textit{ (2) }\ abscederetque fixis \textit{ L}}}, unde probaretur, eam non esse in centro Mundi eum in finem tubum perpendiculariter erexit, et stellas observavit, quae sunt verticales. Ipse dorso supinus incumbens exactissime magnitudines observabat. Theoria planetarum\protect\index{Sachverzeichnis}{theoria planetarum}, Ludimagistri cujusdam Londinensis, eaque non inepta. Praedictiones tempestatum\protect\index{Sachverzeichnis}{praedictio tempestatum} ex coelo, cum ipsis plagis ventorum, Londinum, urbs observationibus inepta\edtext{}{\lemma{inepta}\Bfootnote{Zu Wetterbeobachtungen durch Leibniz vgl. N. 37, \cite{00251}\textit{LSB} III, 1, S.~41 und durch die Royal Society \cite{00154}\textit{BH} III, S.~75.}}. Praedictio cometarum\protect\index{Sachverzeichnis}{praedictio cometarum}. Hevelii\protect\index{Namensregister}{\textso{Hevelius,} Johannes 1611\textendash 1687}\edtext{}{\lemma{cometarum}\Bfootnote{Vgl. \cite{00215}\textit{The motion of the late Comet praedicted}, \textit{PT} 1 (1665), S.~3\textendash 8, einen Bericht \"{u}ber eine Sendung des A. Auzout\protect\index{Namensregister}{\textso{Auzout} (Auzutus), Adrien 1622\textendash 1603|textit} mit der ersten Beschreibung einer Kometenbahn. Wegen des nachfolgenden Hinweises vielleicht auch \textsc{J. Hevelius, }\cite{00152}\textit{Prodromus Cometicus}, Danzig 1665, und auch \cite{00220}\textit{Extract of a Letter of M. Hevelius}, \textit{PT} 7 (1672), S.~4017f.}} organica coelestis\edtext{}{\lemma{coelestis}\Bfootnote{\textsc{J. Hevelius, }\cite{00151}\textit{Machina coelestis}, Danzig 1673.}}. Instrumentum 2\textsuperscript{da} minuta temporis inveniendi sole. S.H)\edtext{}{\lemma{sole.}\Bfootnote{Vgl. \textsc{Th. Sprat, }\cite{00098}a.a.O., S.~246.}} add. Geometr. Lunae mappa\protect\index{Sachverzeichnis}{mappa lunae} in relievo\edtext{}{\lemma{relievo}\Bfootnote{Vgl. \textsc{Th. Sprat, }\cite{00098}a.a.O., S.~315.}}.\pend \pstart \textso{Mechanica }\protect\index{Sachverzeichnis}{mechanica} Hook\protect\index{Namensregister}{\textso{Hooke} (Hookius, Hook), Robert 1635\textendash 1703} de aequiresistentibus \edtext{figuris\protect\index{Sachverzeichnis}{figurae aequiresistentes}}{\lemma{figuris}\Bfootnote{Vgl. dazu \cite{00251}\textit{LSB} III, 1, S.~41f.}} demonstratio. Horologium annuum\protect\index{Sachverzeichnis}{horologium!annuum} et ultra. Motus perennis\label{DrebInst} Drebelii\protect\index{Namensregister}{\textso{Drebbel} (Drebelius, Drebel), Cornelius 1572\textendash 1633} sub Jacobo\protect\index{Namensregister}{\textso{England: Jakob I.} (Rex Jacobus, R. Jac.), K\"{o}nig von England 1603\textendash 1625}\edtext{}{\lemma{Jacobo}\Bfootnote{C. Drebbel hatte zwischen 1606 und 1609 f\"{u}r James I.\protect\index{Namensregister}{\textso{England: Jakob I.} (Rex Jacobus, R. Jac.), K\"{o}nig von England 1603\textendash 1625} eine astronomische Uhr konstruiert (beschrieben durch \textsc{Th. Tymme, }\cite{00162}\textit{Dialogue}, London 1612, S.~60\textendash 63), die durch steigendes oder fallendes Wasser in einer Glaskugel auch die Gezeiten anzeigte. Die Beschreibung hebt die perpetual motion der Uhr hervor.}}. Horologium pendulum\protect\index{Sachverzeichnis}{horologium!pendulum} conveniens Soli.\pend 
\pstart Experimenta de gradibus resistentiae\protect\index{Sachverzeichnis}{resistentia!lignorum} et flexibilitatis\protect\index{Sachverzeichnis}{flexibilitas lignorum} summae, variorum lignorum\edtext{}{\lemma{lignorum}\Bfootnote{Vgl. \textsc{Th. Sprat, }\cite{00098}a.a.O., S.~227.}}. S.H) De ligno\footnote{\textit{Oberhalb} de ligno quod semestri: Locust Tree, arcus} quod semestri tensione nihil virium amittit\edtext{}{\lemma{amittit}\Bfootnote{Vgl. \textsc{Th. Sprat, }\cite{00098}a.a.O., S.~198.}} S.H)\pend 
\pstart $\langle$\edtext{Movet}{\lemma{Movet}\Afootnote{\textit{Lesung unsicher}}}$\rangle$ $\langle$−$\rangle$ wie eine kammer. M. Coterel\protect\index{Namensregister}{\textso{Coterel}}, pendulum sine sono, et quod noctu igne monstrat horam, luce quadam tenui, dixit M. Coterel\protect\index{Namensregister}{\textso{Coterel}}. De recipiendis et praeservandis viribus pulveris pyrii\protect\index{Sachverzeichnis}{pulvis!pyrius}\edtext{}{\lemma{viribus}\Bfootnote{Vgl. \textsc{Th. Sprat, }\cite{00098}a.a.O., S.~250.}} S.H) Drebels\protect\index{Namensregister}{\textso{Drebbel} (Drebelius, Drebel), Cornelius 1572\textendash 1633} m. p. se restituens\edtext{}{\lemma{restituens}\Bfootnote{Das Buch \textsc{Th. Tymme,} \cite{00162}a.a.O., verspricht im Titel eine \textit{Artificiall perpetuall motion}. Drebbels Konstruktion wird jedoch nicht als perpetuum mobile pr\"{a}sentiert, sondern als Antrieb wird ein \textit{fierie spirit out of the minerall matter, ioyn[ed] with his proper Aire} genannt. (vgl. \textsc{Th. Tymme}, \cite{00162}a.a.O., S.~60\textendash 63). Die wiederholte Notiz Leibniz' zu diesem Gegenstand stimmt mit seinem anhaltenden Interesse an einem motus perennis \"{u}berein.}}; corruptus a R. Jac.\protect\index{Namensregister}{\textso{England: Jakob I.} (Rex Jacobus, R. Jac.), K\"{o}nig von England 1603\textendash 1625} ipso absente. Ein Franzoß unseres Chevalier zu francfort hat etwas, so wie ein Bliz\protect\index{Sachverzeichnis}{Blitz} schrecken soll dixit \edtext{Sch\protect\index{Namensregister}{\textso{Schroeter} (Schr., Sch.), Wilhelm 1644\textendash 1688}}{\lemma{Sch.}\Bfootnote{Die mit Sch. bezeichnete Person oder Quelle ist nicht eindeutig gekl\"{a}rt. Ein Gro{\ss}teil der so gekennzeichneten Nachrichten ist in Deutsch festgehalten und enth\"{a}lt mehrfach Informationen zu Ereignissen in Deutschland. Zudem lautet die Abk\"{u}rzung ein Mal Schr. Daher wird mit dem K\"{u}rzel Sch. vermutlich Schroeter\protect\index{Namensregister}{\textso{Schroeter} (Schr., Sch.), Wilhelm 1644\textendash 1688|textit} bezeichnet. Diese Annahme wird durch mehrere Erw\"{a}hnungen Schroeters in Leibniz' Briefen (\cite{00252}\textit{LSB} III, 1, S.~38 und 48), die anf\"{a}nglich auch Austausch mit Schroeter nahelegen, unterst\"{u}tzt.}}.\pend \pstart \textso{Pneumatica }\protect\index{Sachverzeichnis}{pneumatica} Boylii\protect\index{Namensregister}{\textso{Boyle} (Boylius, Boyl., Boyl), Robert 1627\textendash 1691} experimenta de relatione aeris et flammae\protect\index{Sachverzeichnis}{flamma}\edtext{}{\lemma{flammae}\Bfootnote{\textsc{R. Boyle}, \textit{New experiments, touching the relations betwixt flame and air} vermutl. von Leibniz gelesen als Teil der \cite{00156}\textit{Tracts}, London 1672 (\cite{00017}\textit{BW} 7, S.~73\textendash 131). Zu diesem Buch vgl. S.~\pageref{tracts}.}}. Steph. ab Angelis\protect\index{Namensregister}{\textso{Angeli,} Stefano degli 1623\textendash 1697} del peso dell' aria\edtext{}{\lemma{aria}\Bfootnote{\textsc{St. degli Angeli, }\cite{00165}\textit{Gravit\`{a}}, Padua 1672.}}.\pend \pstart \textso{Meteorologica}\protect\index{Sachverzeichnis}{meteorologica}: Bohun\protect\index{Namensregister}{\textso{Bohun,} Ralph 1639\textendash 1716} de ventis observationes Nautarum\edtext{}{\lemma{Nautarum}\Bfootnote{\textsc{R. Bohun, }\cite{00166}\textit{Dis\-course}, Oxford 1671.}}; \edtext{Wetter Glock\protect\index{Sachverzeichnis}{Wetterglocke}}{\lemma{Wetter Glock}\Bfootnote{Vgl. \textsc{Th. Sprat, }\cite{00098}a.a.O., S.~313.}} Wrennii\protect\index{Namensregister}{\textso{Wren} (Wrennius), Christopher 1632\textendash 1723} et Hookii\protect\index{Namensregister}{\textso{Hooke} (Hookius, Hook), Robert 1635\textendash 1703} Wallisius\protect\index{Namensregister}{\textso{Wallis} (Wallisius), John 1616\textendash 1703} observat.\pend \pstart Stellarum cadentium\protect\index{Sachverzeichnis}{stella!cadentia} examinatio: materia mucilaginosa dicta staar-shoot\edtext{}{\lemma{staar\textendash shoot}\Bfootnote{Vgl. \textsc{Th. Sprat, }\cite{00098}a.a.O., S.~227.}} S.H).\pend 
\pstart \textso{Hydrostatica }\protect\index{Sachverzeichnis}{hydrostatica} \edtext{navis}{\lemma{navis}\Bfootnote{Drebbel hatte zwischen 1620\textendash 1624 unter James I.\protect\index{Namensregister}{\textso{England: Jakob I.} (Rex Jacobus, R. Jac.), K\"{o}nig von England 1603\textendash 1625} mehrere Ruderboote durch Lederüberzug und Schnorchel zu Unterwasserfahrzeugen ver\"{a}ndert und durch die Themse fahren lassen. Dazu ein Bericht bei Boyle 1662.}} Drebelii\protect\index{Namensregister}{\textso{Drebbel} (Drebelius, Drebel), Cornelius 1572\textendash 1633}, \edtext{ejus}{\lemma{Drebelii,}\Afootnote{ \textit{ (1) }\ remi \textit{ (2) }\ ejus \textit{ L}}} mirabiles\edtext{}{\lemma{mirabiles}\Bfootnote{Vgl. dazu \cite{00253}\textit{LSB} II, 1, S.~263.}}, ideo sub Carolo\protect\index{Namensregister}{\textso{England: Karl I.} (Carolus), K\"{o}nig von England und Schottland 1625\textendash 1649}\protect\index{Namensregister}{\textso{England: Karl II.} (Carolus), K\"{o}nig von England 1649\textendash 1685}.\pend \pstart  Boylius\protect\index{Namensregister}{\textso{Boyle} (Boylius, Boyl., Boyl), Robert 1627\textendash 1691} quaedam Hydrostatica \edtext{publicabit}{\lemma{publicabit}\Bfootnote{Nach M\"{a}rz 1673 ist ein Titel Boyles mit dem Stichwort Hydrostatica nicht bekannt (vgl. \cite{00017}\textit{BW} 7, S.~335f.; 11, S.~189\textendash 196). Entweder kannte Leibniz ein nicht realisiertes Projekt Boyles, oder er bekam Zugang zu Boyles \textit{Hydrostatical Discourse} und \textit{Hydrostatical Letter} in den \cite{00156}\textit{Tracts} (\cite{00017}\textit{BW} 7, S.~73\textendash 184) erst zu einem sp\"{a}teren Zeitpunkt. Daf\"{u}r spricht das Exzerpt dieser Publikation als Nachtrag, vgl. S.~\pageref{tracts}. Vgl. auch \cite{00216}\textit{An Accompt of two Books}, \textit{PT} 8 (1673), S.~5197\textendash 6006.}}. Dusonius\protect\index{Namensregister}{\textso{Du Sonius} (Dusonius), Aigmont 1604\textendash ?} nunc in fodinarum aquis amoliendis exercetur.\pend 
\pstart (S.H de figuris corporum, ita accommodandis, ut per diversa media simul fundum attingant. \textso{Experim}. machina 1000 tonnen waßer in einer stunde aus\-zupumpen. (an modo momentum.) dixit Sch. Modulum esse in Soc. Repositorium. Einer der mit einem inst. so eine cheminee aus dem waßer raus gehend hat, etliche stunden kan unter waßer seyn. Relatio de \edtext{Cingulo aere pleno, quo}{\lemma{de}\Afootnote{ \textit{ (1) }\ homine qu \textit{ (2) }\ Cingulo aere pleno, quo \textit{ L}}} in aqua iri potest (coram duce Florentiae\protect\index{Ortsregister}{Florenz (Fiorenze, Florentia)}) S.H) adde Christianum IV.\protect\index{Namensregister}{\textso{D\"{a}nemark:} Christian IV., K\"{o}nig von D\"{a}nemark 1588\textendash 1648} Daniae\protect\index{Ortsregister}{Danemark@D\"{a}nemark (Dania)}.
\pend\newpage
\pstart \textso{Nautica}. Experimenta Brunckeri\protect\index{Namensregister}{\textso{Brouncker} (Brunckerus), William Viscount 1620\textendash 1684}, apparatus ejus, canal artificiel cum tot navium formis periit me praesente. Trinity-house\protect\index{Sachverzeichnis}{Trinity\textendash house} [at]\edtext{}{\Afootnote{a\textit{\ L \"{a}ndert Hrsg.}}} London\protect\index{Ortsregister}{London (Londinum)}, ibi relationes nautarum omnes. Societas licentiam sperat percurrendi. Aqua marina dulcis. Dictionarium Nauticum. $\langle$At$\rangle$las Anglicanus. Mensura terrae\protect\index{Sachverzeichnis}{mensura terrae} vera.\pend 
\pstart \textso{Magnetica}\protect\index{Sachverzeichnis}{magnetica}. Observationes Dantiscanae, item in Hudsonsbay\edtext{}{\lemma{Hudsonsbay}\Bfootnote{Vgl. \cite{00251}\textit{LSB} III, 1, S.~43.}}. Collectanea Boylii\protect\index{Namensregister}{\textso{Boyle} (Boylius, Boyl., Boyl), Robert 1627\textendash 1691}. Ejus modus mutandi polum magnetis\protect\index{Sachverzeichnis}{polus!magnetis}. Ejus modus, acus\protect\index{Sachverzeichnis}{acus!magnetica} praeservandi. Bond\protect\index{Namensregister}{\textso{Bond,} Henry 1600?\textendash 1678}.\pend \pstart Diversio virium attractivarum\protect\index{Sachverzeichnis}{vis attractiva} interposito ferro\protect\index{Sachverzeichnis}{ferrum}. NB. Kircher\protect\index{Namensregister}{\textso{Kircher} (Kircherus), Athanasius SJ 1602\textendash 1680} de p. m.\edtext{}{\lemma{de p. m.}\Bfootnote{\textsc{A. Kircher}, \cite{00067}\textit{Magnes}, Rom 1654, S.~239\textendash 245.}} si qui divertere possit. \edtext{tum}{\lemma{possit.}\Afootnote{ \textit{ (1) }\ probatio \textit{ (2) }\ tum \textit{ L}}} per pulverem chalybeum\protect\index{Sachverzeichnis}{pulvis!chalybeus}, tum per acus\protect\index{Sachverzeichnis}{acus}, manifestare. Lineas directionis magneticae, contrarias theoriae Cartesii\protect\index{Namensregister}{\textso{Descartes} (Cartesius, des Cartes, Cartes.), Ren\'{e} 1596\textendash 1650} (Wren\protect\index{Namensregister}{\textso{Wren} (Wrennius), Christopher 1632\textendash 1723}) detegere easdem lineas in composita variorum magnetum influentia\edtext{}{\lemma{influentia}\Bfootnote{Vgl. \textsc{Th. Sprat, }\cite{00098}a.a.O., S.~221f.}}. S.H \pend 
\pstart \edtext{Magnetica\protect\index{Sachverzeichnis}{magnetica!terrella} terrella in assere plano, semiextans polis incumbens horizonti, asser respersus armatura at furrows, that flud like a sorte of helix quasi exiens ex uno polo et ad alium rediens, planum totum figuratum quasi circulis \edtext{hemisphaerii}{\lemma{hemisphaerii}\Bfootnote{Zu diesem Experiment vgl. \textsc{Th. Sprat, }\cite{00098}a.a.O., S.~315f.}}. Boylii\protect\index{Namensregister}{\textso{Boyle} (Boylius, Boyl., Boyl), Robert 1627\textendash 1691} experimenta Magnetico-chymica\edtext{}{\lemma{Magnetico-chymica}\Bfootnote{Vermutl. \textsc{R. Boyle, }\cite{00167}\textit{Specimen unum}, London 1661.}}. Bond\protect\index{Namensregister}{\textso{Bond,} Henry 1600?\textendash 1678} laßet uberal observirn trifft ziemblich zu. Boyl.\protect\index{Namensregister}{\textso{Boyle} (Boylius, Boyl., Boyl), Robert 1627\textendash 1691} se habere rationes cur desperet rem unquam regulari posse. Nautae diversi retulere Boylio\protect\index{Namensregister}{\textso{Boyle} (Boylius, Boyl., Boyl), Robert 1627\textendash 1691} summam declinationem australem\protect\index{Sachverzeichnis}{declinatio!Australis} et summam Borealem\protect\index{Sachverzeichnis}{declinatio!Borealis} fere   congruere. Boyl.\protect\index{Namensregister}{\textso{Boyle} (Boylius, Boyl., Boyl), Robert 1627\textendash 1691} hat ein groß recueil observationum de declinatione. Hudsonsbay\protect\index{Ortsregister}{Hudsonbay (Hudsonsbay)} fahrer experti daß die Nadel 25 bis 30 grad declinirt.}{\lemma{}\Afootnote{magnetica terrella [...] horizonti, asser \textit{ (1) }\ planus \textit{ (2) }\ respersus armatura [...] grad declinirt. \textit{ erg.} \textit{ L}}}\pend 
\pstart \textso{Physica}. \edtext{Boyl.\protect\index{Namensregister}{\textso{Boyle} (Boylius, Boyl., Boyl), Robert 1627\textendash 1691} diamant\protect\index{Sachverzeichnis}{Diamant} so bisweilen schwarz wird intus.}{\lemma{}\Afootnote{Boyl. diamant [...] wird intus. \textit{ erg.} \textit{ L}}} Boyl.\protect\index{Namensregister}{\textso{Boyle} (Boylius, Boyl., Boyl), Robert 1627\textendash 1691} de \edtext{Gemmis}{\lemma{Gemmis}\Bfootnote{\textsc{R. Boyle}, \cite{00240}\textit{Gems}, London 1672 (\cite{00017}\textit{BW} 7, S.~3\textendash 72), oder die lateinische Fassung \cite{00168}\textit{Exercitatio de origine gemmarum}, London 1673.}}. Boyl.\protect\index{Namensregister}{\textso{Boyle} (Boylius, Boyl., Boyl), Robert 1627\textendash 1691} mox de effluviis. Power. Glisson\protect\index{Namensregister}{\textso{Glisson,} Francis 1597\textendash 1677} de vita naturae\edtext{}{\lemma{naturae}\Bfootnote{\textsc{F. Glisson, }\cite{00170}\textit{Tractatus}, London 1672.}}. \edtext{Micrographiae   supplementa.}{\lemma{naturae.}\Afootnote{ \textit{ (1) }\ Relatio de ignivor \textit{ (2) }\ Micrographiae   supplementa. \textit{ L}}}\pend \pstart   Instr. mesurandi corporis descensus et motus ad duas tertias temporis\edtext{}{\lemma{temporis}\Bfootnote{Vgl. \textsc{Th. Sprat, }\cite{00098}a.a.O., S.~225 und 248.}}. Ignivorus in manu lavans plumbo fuso $\langle$−$\rangle$ liquida, carbones devorans ignitos, liquor Drebels\protect\index{Namensregister}{\textso{Drebbel} (Drebelius, Drebel), Cornelius 1572\textendash 1633}, so warhafftig Ebbe und flut zeigte\edtext{}{\lemma{zeigte}\Bfootnote{Zu Drebbels Instrument vgl. S.~\pageref{DrebInst}.}}.\pend \pstart \textso{Botanica }\protect\index{Sachverzeichnis}{botanica} Grey\protect\index{Namensregister}{\textso{Grew,} Nehemiah 1641\textendash 1712} \textit{Anatome}\edtext{}{\lemma{\textit{Anatome}}\Bfootnote{Kurz vor der Teilnahme Leibniz' an einigen Sitzungen der Royal Society hatte diese ihren Auftrag an N. Grew\protect\index{Namensregister}{\textso{Grew,} Nehemiah 1641\textendash 1712}, eine Anatomie der Pflanzen zu schreiben, best\"{a}tigt (vgl. \cite{00154}\textit{BH} III, S.~69), aus der Grew\protect\index{Namensregister}{\textso{Grew,} Nehemiah 1641\textendash 1712} Ausz\"{u}ge vortrug (vgl. \cite{00154}\textit{BH} III, S.~72). Anatome ist eine Referenz auf \textsc{N. Grew}, \cite{00171}\textit{An Idea of a Phytological History}, London 1673 oder Vorgriff auf \textsc{N. Grew, }\cite{00172}\textit{The Anatomy of Plants}, London 1682.}}, Malpighii\protect\index{Namensregister}{\textso{Malpighi} (Malpighius), Marcello 1628\textendash 1694} anatome\edtext{}{\lemma{anatome}\Bfootnote{Malpighii anatome bezieht sich entweder auf die fr\"{u}heren anatomischen Arbeiten des Malpighius oder Leibniz hatte wie bei Grew\protect\index{Namensregister}{\textso{Grew,} Nehemiah 1641\textendash 1712} Einsicht in Vorarbeiten zu \cite{00200}\textsc{M. Malpighi,} \textit{Anatome Plantarum}, London 1675.}} sucus duplex\protect\index{Sachverzeichnis}{sucus duplex} alter aqueus alter lacteus: et hic incongelabilis. \edtext{Morison}{\lemma{incongelabilis.}\Afootnote{ \textit{ (1) }\ Hartlieb\protect\index{Namensregister}{\textso{Hartlib} (Hartlieb), Samuel 1600\textendash 1662|textit} resp. apum Butler\protect\index{Namensregister}{\textso{Butler,} Charles 1560?\textendash 1647|textit} Monarchia foeminina \textit{ (2) }\ Morison \textit{ L}}} plantarum series\edtext{}{\lemma{series}\Bfootnote{\textsc{R. Morison, }\cite{00175}\textit{Plantarum Distributio}, Oxford 1672.}} Rey\protect\index{Namensregister}{\textso{Ray} (Rey), John 1627\textendash 1705} Itinerarium Botanicum\edtext{}{\lemma{Botanicum}\Bfootnote{Titelangabe nicht eindeutig, entweder \textsc{J. Ray, }\cite{00208}\textit{Observations made in a journey}, London 1673, oder \cite{00207}\textsc{J. Ray, }\textit{Catalogus plantarum Angliae}, London 1670.}}. Agricult. et pomicult. secretum verwant Herr Boyle\protect\index{Namensregister}{\textso{Boyle} (Boylius, Boyl., Boyl), Robert 1627\textendash 1691}.  \pend \pstart Reunio corticis arborum separati\edtext{}{\lemma{separati}\Bfootnote{Vgl. \textsc{Th. Sprat, }\cite{00098}a.a.O., S.~223.}}. S.H) De gramine (exotico) funibus fortissimis aptiore, quam cannabis\protect\index{Sachverzeichnis}{Cannabis sativa}\edtext{}{\lemma{cannabis}\Bfootnote{Vgl. \textsc{Th. Sprat, }\cite{00098}a.a.O., S.~197.}} S.H) De planta quadam mire propagativa pene ineradicabili S.H) Pfeffer\protect\index{Sachverzeichnis}{Pfeffer} aus jamaica\protect\index{Ortsregister}{Jamaika (Jamaica)} so recht wie \edtext{n\"{a}gelchen}{\lemma{n\"{a}gelchen}\Bfootnote{Vgl. \textsc{R. Boyle, }\cite{00155}\textit{Usefulness}, Teil I, S.~12 (\cite{00017}\textit{BW} 3, S.~206f.).}}. U.) Napellus sine veneno in Polonia Annus 2\textsuperscript{dus} Med. Germ.)\edtext{}{\lemma{Germ.)}\Bfootnote{\textsc{M. B. v. Berniz,} \textit{Napellus in Polonia non venenosus}. \cite{00279}\textit{Miscellanea curiosa} 2 (1671), S.~79\textendash 82.}}\pend
\pstart \textso{Anatomica }\protect\index{Sachverzeichnis}{anatomica} Willughby\protect\index{Namensregister}{\textso{Willughby,} Francis 1635\textendash 1672} itinerarium Zoicum\edtext{}{\lemma{Zoicum}\Bfootnote{\textsc{F. Willughby, }\cite{00176}\textit{Voyage through Spain}, London 1673.}}. Malphighii\protect\index{Namensregister}{\textso{Malpighi} (Malpighius), Marcello 1628\textendash 1694} \textit{De Bombyce}\edtext{}{\lemma{\textit{Bombyce}}\Bfootnote{\textsc{M. Malpighi, }\cite{00177}\textit{De bombyce}, London 1669.}}. De pullo in ovo\edtext{}{\lemma{ovo}\Bfootnote{W\"{a}hrend der Sitzung der Royal Society am 22. Januar 1673 wurde ein Brief Malpighis \"{u}ber Beobachtungen an Eiern verlesen, vgl. \cite{00154}\textit{BH} III, S.~73. Vgl. auch \textsc{R. Boyle,} \cite{00155}\textit{Usefulness}, Teil I, S.~54f. (\cite{00017}\textit{BW} 3, S.~236).}}. Butler\protect\index{Namensregister}{\textso{Butler,} Charles 1560?\textendash 1647} de apibus\edtext{}{\lemma{apibus}\Bfootnote{Vgl. \textsc{Ch. Butler, }\cite{00174}\textit{Monarchia Foeminina}, London 1673, Widmungsgedicht.}}. Schwammerdam\protect\index{Namensregister}{\textso{Swammerdam} (Schwammerdam), Jan 1637\textendash 1680} conditura uteri\protect\index{Sachverzeichnis}{uterus}\edtext{}{\lemma{conditura}\Bfootnote{\textsc{J. Swammerdam,} \cite{00201}\textit{Miraculum naturae sive uteri muliebris fabrica}, Leiden 1672. Vgl. auch \cite{00154}\textit{BH} III, S.~52.}}. Ejusdem restitutio hepatis\protect\index{Sachverzeichnis}{hepar}\edtext{}{\lemma{restitutio}\Bfootnote{Kurz vor Leibniz' Teilnahme hatte Swammerdam der Royal Society mehrere anatomische Pr\"{a}parate geschenkt, darunter einen uterus humanus und eine Arteria Hepatica, vgl. \cite{00154}\textit{BH} III, S.~71 und 76.}}. Willis\protect\index{Namensregister}{\textso{Willis,} Thomas 1621\textendash 1675} de anima brutorum seu sensitiva\edtext{}{\lemma{sensitiva}\Bfootnote{\textsc{Th. Willis, }\cite{00178}\textit{De Anima}, Amsterdam 1672.}}. \edtext{Biceps}{\lemma{sensitiva.}\Afootnote{ \textit{ (1) }\ Ejusdem de passione hyst \textit{ (2) }\ Biceps \textit{ L}}} Gallus Indicus\protect\index{Sachverzeichnis}{Gallus Indicus} in spiritu vini\protect\index{Sachverzeichnis}{spiritus!vini} conservatus. Formatio loquelae.\pend \pstart Musculi artificiales\protect\index{Sachverzeichnis}{musculi artificiales} tenduntur Elaterio pyrio pulvere\protect\index{Sachverzeichnis}{pulvis!pyrii}\edtext{}{\lemma{Musculi}\Bfootnote{Vgl. \textsc{Th. Sprat, }\cite{00098}a.a.O., S.~226.}} S.H) Caro reducta menstruo in liquorem sanguini similem\edtext{}{\lemma{similem}\Bfootnote{Vgl. \textsc{Th. Sprat, }\cite{00098}a.a.O., S.~226, und \textsc{R. Boyle,} \cite{00155}\textit{Usefulness}, Teil II, S.~20 (\cite{00017}\textit{BW} 3, S.~306).}}. S.H) Dentes lupi marini esse id quod pro Krotenstein\protect\index{Sachverzeichnis}{Kr\"{o}tnstein} venditatur, et annulis includitur\edtext{}{\lemma{includitur}\Bfootnote{Vgl. \textsc{Th. Sprat, }\cite{00098}a.a.O., S.~242.}}. S.H) Liming of the ground vermehrt der Schaff\protect\index{Sachverzeichnis}{Schaf} fettigkeit, verderbt die wolle\protect\index{Sachverzeichnis}{Wolle}\edtext{}{\lemma{wolle}\Bfootnote{Vgl. \textsc{Th. Sprat, }\cite{00098}a.a.O., S.~242.}}. Tarantulae fabulosae Transact.\edtext{}{\lemma{Transact.}\Bfootnote{Vgl. \textit{An Account of some books}, \cite{00273}\textit{PT} 3 (1668), S.~660\textendash 604 [664], darin Rezension zu I. W. Sangwerdius PD, \cite{00281}\textit{De Tarantula}, Leiden 1668.}} \edtext{Schwammerdam\protect\index{Namensregister}{\textso{Swammerdam} (Schwammerdam), Jan 1637\textendash 1680} tract. de Sanguificatione\protect\index{Sachverzeichnis}{sanguificatio}\edtext{}{\lemma{Sanguificatione}\Bfootnote{Vermutl. Referenz auf \textit{Extracts of two Letters of Dr. Swammerdam}, \cite{00224}\textit{PT} 8 (1673), S.~6040\textendash 6042, hier S.~6041.}} reddet hepati\protect\index{Sachverzeichnis}{hepar} officium promittens sibi applausum tanto majorem, quia nemo ostenderit chylum\protect\index{Sachverzeichnis}{chylus} vehi ad vasa lactea\protect\index{Sachverzeichnis}{vasa lactea} primi generis ut vocat. Quod faciat eum credere, nihil esse in illis nisi lympham albam\protect\index{Sachverzeichnis}{lympha alba} in venis lactenis apparentem, et exeuntem glandulis\protect\index{Sachverzeichnis}{glandula} intestinarum.}{\lemma{}\Afootnote{Schwammerdam\protect\index{Namensregister}{\textso{Swammerdam} (Schwammerdam), Jan 1637\textendash 1680} [...] intestinarum. \textit{ erg.} \textit{ L}}} Infans sine cerebro annus 2\textsuperscript{dus} Med. Germ.\edtext{}{\lemma{Germ.}\Bfootnote{\textsc{J. J. Wepfer,} \textit{De puella sine cerebro nata, historia}, \cite{00280}\textit{Miscellanea curiosa} 3 (1672), S.~205\textendash 208.}} 
\pend
\newpage
\pstart   \textso{Chym}\protect\index{Sachverzeichnis}{chymica}. Catal. commutandorum Lampadographia experimentalis Haakii\protect\index{Namensregister}{\textso{Haak} (Haakius), Theodor 1605\textendash 1690}. Fornax Multituba. Drebelii\protect\index{Namensregister}{\textso{Drebbel} (Drebelius, Drebel), Cornelius 1572\textendash 1633} petarda marina. Drebelii\protect\index{Namensregister}{\textso{Drebbel} (Drebelius, Drebel), Cornelius 1572\textendash 1633} fixio Mercurii\protect\index{Sachverzeichnis}{Mercurius} ipsius manu. Mollitio et induratio ferri\protect\index{Sachverzeichnis}{ferrum}, ejusdem elaboratio in cuprei\protect\index{Sachverzeichnis}{cuprum} coloris faciem. Liquor indurans subito. \pend 
\pstart Menstruum stanni\protect\index{Sachverzeichnis}{menstruum stanni}\edtext{}{\lemma{stanni.}\Bfootnote{Vgl. \cite{00251}\textit{LSB} III, 1, S.~39.}}\footnote{\textit{Oberhalb} stanni: salz spir [\textit{bricht ab}]}.\pend \pstart   Nova species metalli\edlabel{metallistart}.\pend \pstart    \edtext{Experimentum\edlabel{metalliend}}{{\xxref{metallistart}{metalliend}}\lemma{metalli.}\Afootnote{ \textit{ (1) }\ Saxa quae \textit{ (2) }\ Experimentum \textit{ L}}} certa liquoris gutta immissa indurandi aquam in lapidem instar tophi. Principis Roberti\protect\index{Namensregister}{\textso{Pfalz-Simmern: Ruprecht} (Prinz Ruppert, princeps Robertus), Pfalzgraf von Pfalz-Simmern 1619\textendash 1682} pulvis pyrius\protect\index{Sachverzeichnis}{pulvis!pyrius} ordinario fortior \edtext{vicies}{\lemma{vicies}\Bfootnote{Vgl. \textsc{Th. Sprat, }\cite{00098}a.a.O., S.~258, und \cite{00254}\textit{LSB} II, 1, S.~256, sowie III, 1, S.~378.}}: Duratio ferri ut possit scindere porphyritem, et rursus ejus mollitio ut possit laborari. De molliendo metallo quod durescit accepta impressione\edtext{}{\lemma{impressione}\Bfootnote{Vgl. \textsc{Th. Sprat, }\cite{00098}a.a.O., S.~250 und \cite{00251}\textit{LSB} III, 1, S.~39.}}, deque ratione reducendi has impressiones in tam exiguam proportionem quam desideratur in metallo duriore \rightmoon. \pend 
\pstart Thermometer\protect\index{Sachverzeichnis}{Thermometer} fur graden der hize der flammen\protect\index{Sachverzeichnis}{Flamme} sogar schmelzen\edtext{}{\lemma{schmelzen}\Bfootnote{Vgl. \textsc{Th. Sprat, }\cite{00098}a.a.O., S.~249.}} S.H) \pend 
\pstart $\langle$−$\rangle$ Butlers\protect\index{Namensregister}{\textso{Butler,} Charles 1560?\textendash 1647} contubernalis. vid. Medica.
\pend 
\pstart Saturnus\edlabel{saturnusstart} fulminans\protect\index{Sachverzeichnis}{Saturnus!fulminans} \edtext{Kief.\protect\index{Namensregister}{\textso{Kiefler} (Kieflerus, Kief.)}}{\lemma{Kief.}\Bfootnote{Nicht eindeutig zu identifizieren. Einer der Br\"{u}der Johannes Sibertus\protect\index{Namensregister}{\textso{Kiefler,} Johannes Sibertus 1595\textendash 1677}, Abraham\protect\index{Namensregister}{\textso{Kiefler,} Abraham ?\textendash 1657} und Jacob Kiefler\protect\index{Namensregister}{\textso{Kiefler,} Jacob } (auch Kuffler, K\"{u}ffler), von denen zwei mit Drebbels T\"{o}chtern Anna und Catharina verheiratet waren, und die Drebbels Erfindungen zu verbreiten suchten.}} pro aceto\protect\index{Sachverzeichnis}{acetum} \protect\includegraphics[width=0.025\textwidth]{images/salpeter.pdf}, siccaret calcem, distillato aceto\protect\index{Sachverzeichnis}{acetum}, perdidit ein $\langle$−$\rangle$ auge.) Zement eines d\"{a}nen, das aurum mixtum in superficiem kommen.) Joel Lancellot\protect\index{Namensregister}{\textso{Langelott} (Lancellot), Joel 1617\textendash 1680} sal \protect\includegraphics[width=0.02\textwidth]{images/taros.pdf}\textsuperscript{ri} \edtext{volatile}{\lemma{volatile}\Bfootnote{Eine Beschreibung der Gewinnung von Weins\"{a}ure durch Langelott in \cite{00237}\textit{An Extract of a Latin Epistle of Dr. Joel Langelot}, \textit{PT} 7 (1672), S.~5052\textendash 5069, hier S.~5052.}} praeced. \edtext{digesti}{\lemma{digesti}\Bfootnote{Vgl. \textsc{J. Langelott},  \cite{00282}\textit{Epistola de quibusdam in chymia praetermissis}, \textit{Miscellanea curiosa} 3 (1672), S.~96\textendash 106, hier S.~97, und \cite{00221}\textit{An excerpt of a letter, Written by David von der Becke}, \textit{PT} 8 (1673), S.~5185\textendash 5193.}}. Ejus Tinctura corallii\protect\index{Sachverzeichnis}{tinctura corallii} vera digesti cum quodam oleo per integram hyemem superfuso postea spiritu vini\protect\index{Sachverzeichnis}{spiritus!vini}. Ejus extractio spiritus salis tartari\protect\index{Sachverzeichnis}{spiritus!salis tartari} volatilis. Deque opii\protect\index{Sachverzeichnis}{opium} essentia rubra instar rubini praecedente fermentatione. De vero auro potabili\protect\index{Sachverzeichnis}{aurum!potabile} per Triturationem seu Molendinum verus processus \earth. Facti ab ipso Jo\"{e}le Gottorpie\protect\index{Ortsregister}{Gottorp}\edtext{}{\lemma{Jo\"{e}le Gottorpie}\Bfootnote{Nicht eindeutig identifiziert. Der Vorname Joele ist wahrscheinlich auf den vorher genannten Joele Langelott zur\"{u}ckzuf\"{u}hren. Die Belegstelle  (\cite{00282}\textit{Miscellanea curiosa} 3 (1672), S.~96\textendash 102, hier S.~102) nennt die \textit{acta Ducalis Laboratorii Gottorpiensis}, und nicht eine Person.}} et Joh. \edtext{Knichelio}{\lemma{Knichelio}\Bfootnote{Eine Person mit diesem Namen konnte nicht identifiziert werden. Der zweifache Ansatz zeigt, dass Leibniz sich des Namens nicht sicher ist. Die Belegstelle (loc. cit.) nennt Johannes Kunckelius\protect\index{Namensregister}{\textso{Kunckel} (Kunckelius), v. L\"{o}wenstern Johann 1638\textendash 1703}.}} in Saxonia. Theodori Severi qui nuper ex Anglia\protect\index{Ortsregister}{England (Anglia)} per Galliam\protect\index{Ortsregister}{Frankreich (Gallia, Francia)} petiit in Germaniam\protect\index{Ortsregister}{Deutschland (Germania, Duitsland)}, sal \protect\includegraphics[width=0.02\textwidth]{images/taros.pdf} volat.\protect\index{Sachverzeichnis}{sal!tartaris volatile} Julii Kiefler\protect\index{Namensregister}{\textso{Kiefler} (Kieflerus, Kief.)} \astrosun\ ex \saturn. Gallus adeptus fortasse ipso suspicante Basilio\protect\index{Namensregister}{\textso{Basilius}}, fuit in Africa\protect\index{Ortsregister}{Afrika (Africa)} et Arabia\protect\index{Ortsregister}{Arabien (Arabia)}, varii communis)\pend 
\pstart \textso{Sch}. mit einen tropfen liquoris, quicum Saxa in aqua similia topho. Aus 10 \saturn, 2. unz \rightmoon. Zinn in \saturn\ umb es zu cupelliren. Zinn in der mine plenum \astrosun\ et \rightmoon. \mercury\ in $\bigtriangleup$ communi crescit injecto certo sale, et quod crevit est \rightmoon. Sed parum lucriferum. M. Bonnet\protect\index{Namensregister}{\textso{Bonnet,} Abraham 1623\textendash 1685} solle adeptus seyn. Manuscriptum cum quo Kellius invenit process. apud unum amicorum Sch. Blauenfeld\protect\index{Namensregister}{\textso{Blauenfeld,} Sch.(?)} Germanus hat Prinz Rupperten\protect\index{Namensregister}{\textso{Pfalz-Simmern: Ruprecht} (Prinz Ruppert, princeps Robertus), Pfalzgraf von Pfalz-Simmern 1619\textendash 1682} gebracht die kunst eisen zu weichen und h\"{a}rten.\pend 
\pstart Es ist ein Cement so er weich machet als wenn er im feuer unterm hammer were. Seine st\"{u}ck gießerey ist zu Windsor\protect\index{Ortsregister}{Windsor}. Schr. ait se posse vitrum obducere rubro folio in fluxu, cum ante fluxum totum fuerit rubrum.  Eisen schmelzen ohne \earth\ oder bley, daß mans ausgießen kan ausm crucibulo wie bley, et postea erat durius quam ante \astrosun. Man hatte also aus Eisen pulver machen k\"{o}nnen wie man macht aus bley. U) Malleable Soder\protect\index{Sachverzeichnis}{Soda} desideratum der Buchsen und Kupferschmiede der guthe Silber Soder\protect\index{Sachverzeichnis}{Soda} approximirt ihm sehr U) Gewiß Pulver damit er bley Erz ohne ofen \edtext{geschmelzet.\edlabel{saturnusend}}{{\xxref{saturnusstart}{saturnusend}}\lemma{}\Afootnote{Saturnus fulminans [...] Gottorpie et Joh. \textit{ (1) }\ Kircher\protect\index{Namensregister}{\textso{Kircher} (Kircherus), Athanasius SJ 1602\textendash 1680|textit} \textit{ (2) }\ Knichelio in Saxonia. \textit{ (a) }\ Drebels fixio \textit{ (b) }\ Theodori Severi qui nuper ex [...] adeptus seyn. Manuscriptum  \textit{ (aa) }\ Kelleri ex quo \textit{ (bb) }\ cum quo Kellius invenit [...] und h\"{a}rten. \textbar\ Es ist ein Cement [...] ofen geschmelzet. \textit{ erg.} \textbar\ \textit{ erg. } \textit{ L}}}\pend 
\pstart \textso{Medica}\protect\index{Sachverzeichnis}{medica}: Willis\protect\index{Namensregister}{\textso{Willis,} Thomas 1621\textendash 1675} cum Highmoro\protect\index{Namensregister}{\textso{Highmore} (Highmorus), Nathaniel 1613\textendash 1685} de passione hysterica\protect\index{Sachverzeichnis}{passio!hysterica} et hypochondriaca\protect\index{Sachverzeichnis}{passio!hypochondriaca}\edtext{}{\lemma{hypochondriaca}\Bfootnote{\textsc{N. Highmore, }\cite{00179}\textit{De Passione Hysterica}, Amsterdam 1670.}}. Bettus\protect\index{Namensregister}{\textso{Betts} (Bettus), John ca. 1623\textendash 1695} \textit{de ortu et natura} \edtext{\textit{sanguinis}}{\lemma{sanguinis}\Bfootnote{\textsc{J. Betts, }\cite{00180}\textit{De Ortu et Natura Sanguinis}, London 1669.}}. Medela Medicinae\edtext{}{\lemma{Medicinae}\Bfootnote{\textsc{M. Nedham}, \cite{00181}\textit{Medela Medicinae}, London 1665.}}. Propositio de Balneis veterum reducendis. De viperae morsu curato. De Morbillis, de venaesectionis\protect\index{Sachverzeichnis}{venaesectio} usu, de sedanda sanguinis emissione per nares. Lapis Butleri\protect\index{Sachverzeichnis}{lapis!Butleri}\edtext{}{\lemma{Butleri}\Bfootnote{Fr\"{u}he Erw\"{a}hnung eines weit verbreiteten (z. B. \textsc{H. Boerhave}, \cite{00202}\textit{Institutiones}, Leiden 1724, S.~91, \textsc{E. G. Stahl}, \cite{00203}\textit{Fundamenta}, N\"{u}rnberg 1747, S.~480), aber wenig beschriebenen Therapeutikum.}}, ejusque compositio. Jejunium annuum. \edtext{Curatio}{\lemma{annuum.}\Afootnote{ \textit{ (1) }\ Curationes \textit{ (2) }\ Curatio \textit{ L}}} per attactum. \pend 
\pstart Certa curatio hydropis\protect\index{Sachverzeichnis}{hydrops} cum viscera incorrupta.\pend 
\pstart Paronychia folio rubaceo\edtext{}{\lemma{rubaceo}\Bfootnote{Paronychia folio rubaceo, oder Saxafragis tridactylites, vgl. \textsc{Caspar Bauhin, }\cite{00182}\textit{Catalogus Plantarum}, Basel 1671, Nr. 84a/84b.}} simplici infusione cerevisiae (beer) curat das \edtext{kings evil\protect\index{Sachverzeichnis}{king's evil}}{\lemma{king's evil}\Bfootnote{Name f\"{u}r Scrofula, die als heilbar durch Ber\"{u}hrung des frischgesalbten K\"{o}nigs angesehen wurde. Vgl. \cite{00183}\textit{Book of Common Prayer}, London 1662. Vgl. auch Boyle, \cite{00155}\textit{Usefulness}, Teil I, S.~47 und Teil II, S.~121 (\cite{00017}\textit{BW} 3, S.~231, 366).}}. U) Persicaria\edtext{}{\lemma{Persicaria}\Bfootnote{Pflanzenart der Familie der Polygonaceae.}}, wie man rosenwaßer destillirt; waßer curirt sogar lapidem vesicae\protect\index{Sachverzeichnis}{lapis!vesicae}. Item Millepedibus Horatius Augenius\protect\index{Namensregister}{\textso{Augenio} (Augenius), Orazio 1527\textendash 1603} in Laurenbergius liberavere\edtext{}{\lemma{liberavere}\Bfootnote{Vgl. \textsc{Boyle, }\cite{00155}\textit{Usefulness}, Teil II, S.~154f. (\cite{00017}\textit{BW} 3, S.~386).}}. Sectioni paratos. Pillen aus \rightmoon\ vor serum und hydropem\protect\index{Sachverzeichnis}{hydrops}\edtext{}{\lemma{hydropem}\Bfootnote{Vgl. \textsc{R. Boyle, }\cite{00155}\textit{Usefulness}, Teil II, S.~120f. (\cite{00017}\textit{BW} 3, S.~366)}}. Tranck wodurch exulcerierte doch nicht cancrose br\"{u}ste curirt werden\edtext{}{\lemma{werden}\Bfootnote{Vgl. \textsc{R. Boyle, }\cite{00155}\textit{Usefulness}, Teil II, S.~122 und 156 (\cite{00017}\textit{BW} 3, S.~367, 387).}}.\pend \pstart Curatio fistulae ohne ofnung der Brust\edtext{}{\lemma{Brust}\Bfootnote{Vgl. \textsc{R.~Boyle, }\cite{00155}\textit{Usefulness}, Teil II, S.~123 (\cite{00017}\textit{BW} 3, S.~368). Das dazugeh\"{o}rende Rezept \cite{00155}a.a.O., S.~319\textendash 321 (\cite{00017}\textit{BW} 3, S.~490f.).}}.\pend 
\pstart Correctio mira opii et aliorum venenorum\edtext{}{\lemma{venenorum}\Bfootnote{Vgl. \textsc{R. Boyle, }\cite{00155}\textit{Usefulness}, Teil II, S.~134 (\cite{00017}\textit{BW} 3, S.~375).}}. Sali \protect\includegraphics[width=0.02\textwidth]{images/taros.pdf}ri certo   modo praeparato, opii\protect\index{Sachverzeichnis}{opium} it. digestione et fermentatione cum certis vegetabilibus seu simplicibus appropriatis\edtext{}{\lemma{appropriatis}\Bfootnote{Vgl. \textsc{R. Boyle, }\cite{00155}\textit{Usefulness}, Teil II, S.~136 (\cite{00017}\textit{BW} 3, S.~376).}}. Idem sal Tartari cum simplicibus eorum virtutes exaltat ultra vim croci Metall. et Mercur.\protect\index{Sachverzeichnis}{Mercurius} vitae, et tum sine vi Emetica\protect\index{Sachverzeichnis}{vis emetica}   et cathartica\protect\index{Sachverzeichnis}{vis cathartica}. \pend 
\pstart  Extractio salium et sulphurum ex simplicibus, crasin eorum plus solito retinentium. De ratione reducendi animalium consistentia in liquorem, sine igne violento, sine additione, qui liquor dimittat partes sulphureas\protect\index{Sachverzeichnis}{partes!sulphureae} et salinas\protect\index{Sachverzeichnis}{partes!salinae} ante phlegma. Tinctura \earth\ rubra, quae ex menstruo non praecipitatur, ut vulgaris spiritu urinae vel alcalica   solutione. Pleuresis in juvene quadam curata sine venae sectione dato medicamento omnia U). Curatio hydropis\protect\index{Sachverzeichnis}{hydrops} \edtext{K.}{\lemma{K.}\Afootnote{\textit{doppelt unterstrichen}}} ex fundamento emendato hepate\protect\index{Sachverzeichnis}{hepar} dum nondum corrupto, per pilulas purgirt waßer ab per sedes et urinam. Ein ostindisch \edtext{semen}{\lemma{semen}\Bfootnote{Vermutl. Cassia fistula oder Pudding Pipe Tree.}}, so nur auff die Zunge genommen ohne violenz purgirt \edtext{per sedes}{\lemma{}\Afootnote{per sedes \textit{ erg.} \textit{ L}}}, ein Chirurgus nahmens Schmitt\protect\index{Namensregister}{\textso{Schmitt}}, so iezo zu Amsterdam, memorabat \edtext{[Kieflerum]\protect\index{Namensregister}{\textso{Kiefler} (Kieflerus, Kief.)}}{\lemma{}\Afootnote{Kieflerus \textit{\ L \"{a}ndert Hrsg.}}}. \pend 
\pstart Durante peste podagra\protect\index{Sachverzeichnis}{podagra} cessavit, ut et le haut mal\protect\index{Sachverzeichnis}{haut mal} Transact. Kieflers\protect\index{Namensregister}{\textso{Kiefler} (Kieflerus, Kief.)} Vater von einem becker in holland\protect\index{Ortsregister}{Holland (Hollandia)} vor 40 jahren a podagra\protect\index{Sachverzeichnis}{podagra} liberirt, putant per genus quoddam \astrosun\ diaphoretici\protect\index{Sachverzeichnis}{diaphoreticum}, princeps Amaicus honorem quaesierat, laborans et ipse, sed jam obierat. \pend 
\pstart Claiton\protect\index{Namensregister}{\textso{Clayton} (Claiton), Thomas 1612?\textendash 1693} Oxonii\protect\index{Ortsregister}{Oxford (Oxonium)} liberavit Dominum de Schonborn\protect\index{Namensregister}{\textso{Sch\"{o}nborn}, Melchior Friedrich v. 1644\textendash 1717}\edtext{}{\lemma{Dominum}\Bfootnote{Es ist nicht eindeutig gekl\"{a}rt, welches Mitglied der Familie Sch\"{o}nborn hier genannt wird. Da die Behandlung zu Oxford stattfand, ist wahrscheinlich Melchior Friedrich von Sch\"{o}nborn\protect\index{Namensregister}{\textso{Sch\"{o}nborn}, Melchior Friedrich v. 1644\textendash 1717|textit} gemeint.}} a quartana\protect\index{Sachverzeichnis}{quartana} per fortissimas ligationes tempore appetentis paroxysmi. \pend 
\pstart Sydenham\protect\index{Namensregister}{\textso{Sydenham,} Thomas 1624\textendash 1689} historia 20 circiter morborum\edtext{}{\lemma{morborum}\Bfootnote{Bis 1673 hatte Sydenham nur eine Schrift mit dem Titel \cite{00185}\textit{Methodus curandi febres}, London 1666, 2nd ed. London 1668, ver\"{o}ffentlicht. Sie wurde von Boyle besprochen (\cite{00223}\textsc{R. Boyle}, \textit{An Account Of Dr. Sydenham's Book}, \textit{PT} 1 (1665), S.~210\textendash 213). Diese Besprechung und Leibniz' Notizen hier zeigen einige \"{U}bereinstimmung.}}, quales nullae extent, nisi forte in Hippoc. Galeni et aliis hypoth. miscere: tempestates anni addendae historiae morborum. Morbi phaenomena alia   aeterna, alia a phaenomenis ipsaque curatione. Specifica vero esse pauca, plerasque etiam traditiones autorum Europaeorum a schola infectas. Corticem peruvianum\protect\index{Sachverzeichnis}{cortex Peruvianum} utilem in declinatione quartanae\protect\index{Sachverzeichnis}{quartana}, non ante sirupum spicae cervinae egregium contra serosos humores, hydropicos ea a se restitutos. Potest morbus curari sine cognitione causae exemplo morbillorum. Distinctas optimas, tardius erumpentes, non turbandam secretionem medicamentis nec nimis maturandam calore si flaccae sint, alium natura exitum quaerit, unde in infantibus diarrhoea\protect\index{Sachverzeichnis}{diarrhoea}, ex non impedienda haec diarrhoea\protect\index{Sachverzeichnis}{diarrhoea}, ut congrua naturae. Filium ipse suum ita curavit ex promisso.\pend 
\pstart \textso{Miscellanea}. Vernix Chinensium\protect\index{Sachverzeichnis}{vernix Chinensium}, aquae calidae resistens. Tapetes impressis figuris ingentibus insignes. \edtext{Dentelli, ex charta, iidem ex}{\lemma{insignes.}\Afootnote{ \textit{ (1) }\ Dentella, ex charta, eaedem, ex \textit{ (2) }\ Dentelli, ex charta, iidem ex \textit{ L}}} taffetas per instrumentum. Copia sigillorum. Acidulae artificiales. Impressio Sigilli in metallum. Duo liquores incolores, quorum mixtione pulchrum Caeruleum\protect\index{Sachverzeichnis}{caeruleum} existit\edtext{}{\lemma{existit}\Bfootnote{Vermutl. Notiz \"{u}ber eine Mitteilung Boyles an die Royal Society auf der Sitzung vom 29. Januar 1673 (vgl. \cite{00154}\textit{BH} III, S.~73). Vgl. auch die Darstellung von Kupferaminen aus Kupferchlorid und Urin, Boyle, Workdiary 21,14). Das Interesse an der Herstellung blauer Pigmente gr\"{u}ndet in der geringen Zahl blauer Farbstoffe aus nat\"{u}rlichen Quellen.}}.\pend 
\pstart Wilkinsii\protect\index{Namensregister}{\textso{Wilkins} (Wilkinsius), John 1614\textendash 1672} character universalis cum figuris\edtext{}{\lemma{figuris}\Bfootnote{\textsc{J. Wilkins, }\cite{00184}\textit{An Essay towards a real character, and a philosophical language}, London 1668.}}. Ejusd. prodromus Grammatica rationis. Alterius Ars signorum seu character universalis. Ars substituendi sibi perpetuo \edtext{diversi}{\lemma{perpetuo}\Afootnote{ \textit{ (1) }\ diversas \textit{ (2) }\ diversi \textit{ L}}} gradus literas ex hoc postremo characteris universalis autore.\pend 
\pstart Substituere ova oleo in pictura S.H)\edtext{}{\lemma{S.H)}\Bfootnote{Vgl. \textsc{Th. Sprat}, \cite{00098}a.a.O., S.~199.}}. Ars \textso{vitraria} par aut melior Veneta, allata sumtibus ducis Buckinhamii\protect\index{Namensregister}{\textso{Villiers} (Dux Buckinhamii), George, 2nd Duke of Buckingham 1628\textendash 1687}\edtext{}{\lemma{ducis}\Bfootnote{Vgl. \textsc{Th. Sprat}, \cite{00098}a.a.O., S.~250f.}}.\pend 
\pstart Wilkinsii\protect\index{Namensregister}{\textso{Wilkins} (Wilkinsius), John 1614\textendash 1672} series omnium specierum\edtext{}{\lemma{specierum}\Bfootnote{Vgl. \textsc{Th. Sprat, }\cite{00098}a.a.O., S.~251.}}.\pend \pstart Schr. dicit quod K\"{a}se\protect\index{Sachverzeichnis}{K\"{a}se} mit ungeleschten kalch gemischt ist der beste Kitt\protect\index{Sachverzeichnis}{Kitt}, so bekant beßer als hausenblasen. Die alte weiber schienen gebrochene topfe mit K\"{a}se\protect\index{Sachverzeichnis}{K\"{a}se} und ruß, kalck beßer als ruß.\pend \pstart Boylius\protect\index{Namensregister}{\textso{Boyle} (Boylius, Boyl., Boyl), Robert 1627\textendash 1691} refert quendam in Anglia\protect\index{Ortsregister}{England (Anglia)} recuperasse artem Musaicam\protect\index{Sachverzeichnis}{ars Musaica}, eundem in itinerariis infinitos libros et alia collegisse de l'art de la charte tanerie. \pend \pstart L'Art d'Emailler sur le verre verlohren. Boyl\protect\index{Namensregister}{\textso{Boyle} (Boylius, Boyl., Boyl), Robert 1627\textendash 1691} hats auf gl\"{a}sern der vor 100 jahr gemachten uhren gesehen. Boyl\protect\index{Namensregister}{\textso{Boyle} (Boylius, Boyl., Boyl), Robert 1627\textendash 1691} sagt zu \edtext{Salesbury}{\lemma{Salesbury}\Bfootnote{Vgl. \cite{00222}\textit{Directions for Inquiries concerning Stones}, \textit{PT} 8 (1673), S.~6010.}} sey ein altar von Marbel\protect\index{Sachverzeichnis}{Marbel} dem ansehen nach, in veritate holz, sehr wohl nach gemacht. Caement gebrochene statuen ganz und damit marbel nach zu machen U) man kan in England\protect\index{Ortsregister}{England (Anglia)} nicht porphyr\protect\index{Sachverzeichnis}{Porphyr} wurchen, wohl aber zu Rom. Stahl\protect\index{Sachverzeichnis}{Stahl} ausleschen in aqua cortice arboris impraegnata contra rubrig.\pend