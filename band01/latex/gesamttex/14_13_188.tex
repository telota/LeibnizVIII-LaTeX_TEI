\pend \pstart [p.~188] [...] igitur longe inferior sole, vt patet; cometa\protect\index{Sachverzeichnis}{cometa}  etiam anni 1665. die 5. Inanuarij infra solem fuisse\footnote{\textit{Leibniz unterstreicht}: cometa etiam [...] solem fuisse} videtur;  cum enim sol esset in grad. 13. Capric.\protect\index{Sachverzeichnis}{Capricornus} circiter, et cometa\protect\index{Sachverzeichnis}{cometa}  in 8. gr. Tauri\protect\index{Sachverzeichnis}{Taurus}, distabat a sole arcu circuli maximi grad.  circiter 115. sit ergo sol in B, \textit{fig} 144. centrum mundi\protect\index{Sachverzeichnis}{mundus} A,  angulus BAC grad. 115. ducantur BC, AC, AG, parallela BC;\footnote{\textit{Gedruckte Marginalie}: Fig. 144.} [...].