[159~r\textsuperscript{o}] Witsen\protect\index{Namensregister}{\textso{Witsen,} Nicolaes 1641\textendash 1717} \textit{Shipsbow en Bestier} \edtext{fol.}{\lemma{\textit{Bestier}}\Afootnote{ \textit{ (1) }\ pag \textit{ (2) }\ fol. \textit{ L}}} 119. \edtext{Hoedaenigkeit\edlabel{hoedanigkeitstart}}{\lemma{fol. 119.}\Afootnote{ \textit{ (1) }\ Hodaenigkeit \textit{ (2) }\ Hoedaenigkeit \textit{ L}}}\edtext{}{{\xxref{hoedanigkeitstart}{hoedanigkeitend}}\lemma{}\Bfootnote{Hoedaenigkeit: Von Hoedaenigkeit bis goet te zijn vgl. \textsc{N. Witsen}, \cite{00153}a.a.O., S.~119.}} van yzer\protect\index{Sachverzeichnis}{ijzer}, stael\protect\index{Sachverzeichnis}{staal}, en \edtext{koolen}{\lemma{en}\Afootnote{ \textit{ (1) }\ kohlen \textit{ (2) }\ koolen \textit{ L}}}. \textit{De platte en} vier kande \textit{staven die met letter F gemerckt zijn, werden voor d'alderbeste gehouden, waer de gekroonde H navolgt; en het} merck \textit{d'ongekroonde H, verdient in deught de derde} plaetse.
\pend 
\pstart \textit{ Het }\textit{\textso{Orgel-gront-yzer}}\protect\index{Sachverzeichnis}{Orgel\textendash gront\textendash ijzer}\textit{ werd soo good gehouden, als dat} geene \textit{'t welck} \edtext{\textit{met}}{\lemma{\textit{welck}}\Afootnote{ \textit{ (1) }\ \textit{med} \textit{ (2) }\ \textit{met} \textit{ L}}} \textit{de letter F is} gemerckt.
\pend 
\pstart \textit{ Het }\textit{Stockholmer yzer}\protect\index{Sachverzeichnis}{Stockholmer ijzer}\textit{, is in'tgemeen zoo good niet als het} Spanish \textit{yzer}\protect\index{Sachverzeichnis}{Spaans ijzer}\textit{. Het }\textit{yzer}\protect\index{Sachverzeichnis}{ijzer}\textit{ dat tot }\textit{Dantzig}\protect\index{Ortsregister}{Danzig (Dantzig)}\textit{ werd gearbeit en van lange staven is, valt beter als het geene tot }\textit{Stockholm}\protect\index{Ortsregister}{Stockholm}\textit{ is bearbeit. Gelijk de} vierkandte \textit{en lange Gottenburger} staven \textit{mede beter} zyn \textit{als die van }\textit{Stockholm}\protect\index{Ortsregister}{Stockholm}.\pend 
\pstart \textso{Haarts-}\textit{\textso{yzer}}\protect\index{Sachverzeichnis}{Haartz yzer}\textit{ is in minder waerdigkeit en niet het best, om }\edtext{\textit{tot}}{\lemma{\textit{om}}\Afootnote{ \textit{ (1) }\ \textit{tos} \textit{ (2) }\ \textit{tot} \textit{ L}}} Schips \textit{tuigen te} verwercken.
\pend 
\pstart \textit{Het }\textit{\textso{Neurenberger stael}}\protect\index{Sachverzeichnis}{Neurenburger staal}\textit{ daer en denneboom op staet, is het best; welcken} Keur \textit{het merk de }\textit{\textso{Zantloper}}\protect\index{Sachverzeichnis}{Zantloper staal} \edtext{\textit{volght,}}{\lemma{\textit{\textso{Zantloper}}}\Afootnote{ \textit{ (1) }\ \textit{voght,} \textit{ (2) }\ \textit{volght,} \textit{ L}}}\textit{ dit de }\textit{\textso{Hellebart}}\protect\index{Sachverzeichnis}{Helle\textendash baert staal}\textit{, en daer na het }\textit{Klaverblat}\protect\index{Sachverzeichnis}{Klaverblat staal} reght \textit{of crom. Het }\textit{\textso{Zweetsche}}\textit{ }\textit{stael}\protect\index{Sachverzeichnis}{Zweetsche staal}\textit{ werd tien ten hondert minder van waerdigkeit gehouden als het }\textit{Neurenberger stael}\protect\index{Sachverzeichnis}{Neurenburger staal}\textit{. Het }\textit{stael}\protect\index{Sachverzeichnis}{staal}\textit{, 't geen hier uit }\textit{\textso{Bergsland}}\textit{ werdt gebracht, en met een of twee raederen is gemerckt: }\textit{Hans Musiker stael}\protect\index{Sachverzeichnis}{Hans Musiker staal}\textit{ genaemt, werd in Deught tusschen sweets en Neurenberger gehouden. }
\pend 
\pstart \textit{ Von grof }\textit{yzer}\protect\index{Sachverzeichnis}{ijzer}\textit{ werden de }\textit{scheep-ankers}\protect\index{Sachverzeichnis}{Anker}\textit{ gemaekt, met eenig} Spaans \textit{yzer}\protect\index{Sachverzeichnis}{Spaans ijzer}\textit{ vermenght: welcke twee, te saemen zigh zeer wel laeten mengen. }\textit{Spaens yzer}\protect\index{Sachverzeichnis}{Spaans ijzer} \textit{alleen }\edtext{\textit{valt}}{\lemma{\textit{alleen}}\Afootnote{ \textit{ (1) }\ \textit{felt} \textit{ (2) }\ \textit{valt} \textit{ L}}}\textit{ }\edtext{\textit{te slap}}{\lemma{\textit{valt}}\Afootnote{ \textit{ (1) }\ \textit{the} \textit{ (2) }\ \textit{te slap} \textit{ L}}} (+ nimis puto flaccidum +) \textit{en heeft geen stiifte.}\textit{}
\pend
\pstart \textit{ De }\textit{koolen}\protect\index{Sachverzeichnis}{kolen}\textit{ waer mede }\textit{yzer}\protect\index{Sachverzeichnis}{ijzer}\textit{ en }\textit{stael}\protect\index{Sachverzeichnis}{staal}\textit{ gezmeet en heet} werd gemaeckt, \textit{werden ons van }\textit{Nieu-Kaesteel}\protect\index{Ortsregister}{Newcastle (Nieu-Kaesteel)}\textit{ gebragt, die men voor de beste hout: de meeste }\textit{Scotze koolen}\protect\index{Sachverzeichnis}{Schotze kolen}\textit{, zyn in't gemeen slimmer als de} Neucastelsche\protect\index{Sachverzeichnis}{Nieuw\textendash kasteelsche kolen}\textit{, doch de sommige syn wel so goet.} Luyksche \textit{koolen}\protect\index{Sachverzeichnis}{Luicksche kolen}\textit{ zoo got als de} Neucasteelsche\protect\index{Sachverzeichnis}{Nieuw\textendash kasteelsche kolen}. \pend 
\pstart \textit{Om te }\textit{koolen}\protect\index{Sachverzeichnis}{kolen}\textit{ te proeven of te goet zyn: so} neemt eene \textit{hand vol, duwtze sterktoe, en opent dan de hant; zoo ze op te hant van den andern vallen}, zyn ze \textit{goet, maer blijven ze in een klomp, zo} zyn ze \textit{onbequaem. Of in't water geworpen, zoze datelijk zincken, zijn ze niet goet, maer moeten, als vet} (ut pinguedo) \textit{in 'teerst op't water drijven} (+ postea cadent +) \textit{om goet te zijn}\edlabel{hoedanigkeitend}.
\pend 
\clearpage 
\pstart \textso{Witsen}\protect\index{Namensregister}{\textso{Witsen,} Nicolaes 1641\textendash 1717}\textso{ pag.} 157. \edtext{Zum\edlabel{zummodelstart}}{{\xxref{zummodelstart}{zummodelend}}\lemma{Zum}\Bfootnote{Von Zum model bis het Galioen vgl. \textsc{N. Witsen}, \cite{00153}a.a.O., S.~157.}} model des Schifbows\protect\index{Sachverzeichnis}{Schiffbau} kan man nehmen ein Pinaß schiff\protect\index{Sachverzeichnis}{Pynas-Schiff}, \textit{lang oversteven 134 voeten}, ist nur \textit{middelbare} gr\"{o}ße, und k\"{o}nnen sowohl kleinere als gr\"{o}ßere schiff\protect\index{Sachverzeichnis}{Schiff} nach diesen eingerichtet werden. Ein solch schiff\protect\index{Sachverzeichnis}{Schiff} solte man leichtlich in 4 monathen mit 20 oder 22 mannen bauen. Ein Schiff\protect\index{Sachverzeichnis}{Schiff} von 180 oder 185 fuß k\"{o}nnen hierzulande 50 man in 5 monathen \edtext{bequemlich volziehen}{\lemma{monathen}\Afootnote{ \textit{ (1) }\ bauen \textit{ (2) }\ bequemlich volziehen \textit{ L}}}. Von den unkosten ist schwehr etwas gewißes zu sagen, wegen ver\"{a}nderung der Zeit und arbeits Leute; nichts destominder sagt unser autor, will ich umb zur Nachricht folgends beyf\"{u}gen: \textit{Onkost Certer} (+ Certer, puto nachricht, \edtext{Estat.}{\lemma{nachricht,}\Afootnote{ \textit{ (1) }\ et scheda, \textit{ (2) }\ Estat. \textit{ L}}}) \textit{von en schip\protect\index{Sachverzeichnis}{schip} lang 165 voet, wijt 43 voet, hohl 16 voet daer boven acht en dan noch 7 voet}:
\setlength\LTleft{0pt} \setlength\LTright{0pt} \setcounter{LTchunksize}{90}
\begin{longtable}{p{105mm}r}
 & \edtext{gulden}{\lemma{}\Afootnote{ \textit{ (1) }\ fl Gul \textit{ (2) }\ gulden \textit{ L}}} \\
\textit{de kiel\protect\index{Sachverzeichnis}{kiel} zall kosten} & \edtext{1000}{\lemma{1000}\Bfootnote{Bei Witsen: 2000}}\\
\textit{de Vorsteven}\protect\index{Sachverzeichnis}{voorsteven} & \textit{300}\\
\textit{de achtersteven\protect\index{Sachverzeichnis}{achtersteven} 120, de Rantzoenhouten 200, de Heck-balck 60,} & \textit{t'zaem 380}\\
\textit{2 Wulpen 80, 't broeckstuck 15, de heckstutten 36} & t'saem \textit{131}\\
\textit{7 Gangen in't Vlack, 4 1/4-planck lanck} & t'samen \textit{2460}\\
\textit{5 Kimgangen van 5 gang, syn 50 plancken} & \textit{2100}\\
\textit{95 Buickst\"{u}cken tot 40 gl. 200 zitters en leggers tot 20 gl.} & \textit{komt 7800}\\
\textit{220 oplangen tot 18 gl.} & \textit{3960}\\
\textit{Het Kolzem\protect\index{Sachverzeichnis}{kolzem} 200, drie kim-wagers van 5 duim 600, sum.} & \textit{800}\\
\textit{Voorde Wegers in't Vlack en in de Kimmen} & \textit{2200}\\
\textit{voor twee balck-wegers onder malkandere 7 duim} & \textit{500}\\
\textit{13 Kattespoors, tot 45, en 26 Zitters tot 30 gl.} & \textit{1365}\\
\textit{Voor 7 banden, 4 Spooren achter in't Zogh, tot 40 guld. en 18 Zitters tot 30} & \textit{980}\\
\textit{30 overloops balcken 85, en 60 knies tot 60 gl.} & \textit{6050}\\
\textit{Voor kloßen en lijfhouten} & \textit{450}\\
\textit{Voor ribben en karviel-houten} & \textit{150}\\
\textit{Voor balck en kim-wegers} & \textit{560}\\
\textit{Voor wegers tusschen beyde} & \textit{560}\\
\textit{32 Verdeckbalcken 50 gl. 64} \edtext{\textit{Knies}}{\lemma{\textit{64}}\Afootnote{ \textit{ (1) }\ Kieds \textit{ (2) }\ \textit{Knies} \textit{ L}}} \textit{tot 25,} & \textit{3200}.\\
\textit{twee banden voor in de boeg} & \textit{150}.\\
\textit{achter aen de Spiegel 6 knies} & \textit{400}\\
\textit{Vorschaer-stocken} & \textit{160}\\
\textit{4 gang en Huit-dicht 5 planck lanck, zijn 40 plancken} & \textit{1400}\\
\textit{2 Spant-barck-houten dick 9 duim.} & \textit{1000}\\
\textit{90. overloops-plancken tot 9 fl.} & \textit{810}\\
\textit{de} zetgang & \textit{250}\\
\textit{230 Stutten tot 15 gl.} & \textit{3500}\\
\textit{Voorhout tot de Breegang en poorten} & \textit{300}\\
\textit{Een Spante barckhout 300, en een spant-vollinge te samen} & \textit{550}.\\
\textit{Noch een spant-barck-hout met 2 breet vollingen} & \textit{280}.\\
\textit{Een Raehoudt en} Zetgang & \textit{200}.\\
\textit{Vor regelingen en waageschot} & \textit{70}.\\
\textit{Voor Rusten, Hals houten, en klampen} & \textit{120}\\
\textit{Vor het} Galioen\edlabel{zummodelend} & \textit{300}\\
\edtext{\textit{Vor}\edlabel{vorhoutstart}}{{\xxref{vorhoutstart}{vorhoutend}}\lemma{\textit{Vor}}\Bfootnote{Von Vor Hout bis unkosten 74152 vgl. \textsc{N. Witsen}, \cite{00153}a.a.O., S. 158.}} \textit{Hout tot al het beeltwerck zo binnen als buyten} & \textit{400}\\
\textit{Vor al het binne-werck}\protect\index{Sachverzeichnis}{binnenwerk}, schut \textit{en Wulfften} & \textit{150}\\
\textit{Hout to kamers, kotten, kombuis enz} & \textit{1600}\\
\textit{Voor 2 duims deelen} & \textit{900}\\
\textit{Voor 2 1/2 duims} \edtext{\textit{deelen}}{\lemma{}\Afootnote{\textit{deelen} \textbar\ en ringels \textit{ gestr.}\ \textbar\ \textit{700} \textit{ L}}} & \textit{700}\\
\textit{Voor 1 1/2 duims deelen en ringels} & \textit{700}\\
\textit{38 derde deck-balcken tot 25 gl. en 76 knies tot 8 gl.} & \textit{1558}.\\
\textit{vor balckwegers en kim-wegers} & \textit{296}\\
\textit{Vor schaer-stocken, op het tweede deck, en Lijffhouten} & \textit{300}\\
\textit{Voor't Rooster-werck en hoofden} & \textit{150}\\
\textit{Voor de Rooster boven en Hoofden} & \textit{190}\\
\textit{Voor water-borden en kim-wegers} & \textit{192}.\\
\textit{Voor Hut-balcken en knies} & \textit{120}\\
\textit{Voor alle knechten groot en klein} & \textit{200}\\
\textit{Voor alle Betings, betings-balcken en knies} & \textit{200}\\
\textit{Voor de Scheen-breecker en ander Lijfhout} & \textit{60}.\\
\textit{Voor kruis houten, klampen, bos-bancken en schandecken} & \textit{400}\\
\textit{2 Spiellen en't roer} & \textit{200}\\
\textit{voor arbeits} & \textit{15000}\\
\textit{voor de Masten}\protect\index{Sachverzeichnis}{mast} & \textit{4100}\\
\textit{voor een Spant-dicke Vollinge} & \textit{200}\\
\textit{Voor peck, teeren werck} & \textit{500}\\
\textit{Nagels en stelling-houdt} & \textit{600}\\
\textit{Summa alle dießer unkosten} & \textit{74152}\edlabel{vorhoutend}
\end{longtable}
\vspace{0.5ex}
[159~v\textsuperscript{o}, rechte Sp.]  \edtext{Dieses\edlabel{diesesallesstart}}{{\xxref{diesesallesstart}{diesesallesend}}\lemma{Dieses}\Bfootnote{Von Dieses alles bis 9365 g\"{u}lden vgl. \textsc{N. Witsen}, \cite{00153}a.a.O., S. 158.}} alles war nichts als holzwerck, aber dieses war vom besten holze\protect\index{Sachverzeichnis}{Holz}, und wann man wolte holz\protect\index{Sachverzeichnis}{Holz} von schlechteren wert nehmen k\"{o}nte man 11070 fl. abziehen. In ubrigen \"{u}ber die obgedachte holzwerck w\"{u}rde das Eisenwerck erfodern \hfill 7784 \edtext{gl.}{\lemma{7784}\Afootnote{ \textit{ (1) }\ Rfl. \textit{ (2) }\ gl. \textit{ L}}}\\
Die Kochsgereitschafft\protect\index{Sachverzeichnis}{Kochsgereitschaft} in einen schiff\protect\index{Sachverzeichnis}{Schiff} von dießen großen wert erfodert \hfill 352.\\
Uyt de Lijnbaen heefft \textit{men noodig 35261 pont} tow, \textit{tot }\edtext{\textit{45} g\"{u}lden}{\lemma{\textit{tot 45}}\Afootnote{ \textit{ (1) }\ fl. \textit{ (2) }\ glden \textit{ (3) }\ g\"{u}lden \textit{ L}}} \textit{het} \edtext{\textit{schip\protect\index{Sachverzeichnis}{schip} pont}}{\lemma{\textit{schip}}\Afootnote{ \textit{ (1) }\ vont \textit{ (2) }\ \textit{pont} \textit{ L}}} tesammen \hfill \edtext{5289}{\lemma{tesammen}\Afootnote{ \textit{ (1) }\ 5298 \textit{ (2) }\ 5289 \textit{ L}}}\\
Die Seegel\protect\index{Sachverzeichnis}{Segel} kosten auffs wenigste \hfill 2827\\
Die ankers\protect\index{Sachverzeichnis}{Anker} wegen zusammen 6450 lb. Zu 3 st (+ puto stubers +) das lb,\newline thut \hfill 967 gl.\\
Und voor schips-noodige kleidinge wird man vonn\"{o}then haben \hfill 2264.\\
Also das obgemeldtes schiff\protect\index{Sachverzeichnis}{Schiff} ohne kriegs n\"{o}thige ausr\"{u}stung, und \edtext{mund-kosten ehe}{\lemma{mund-kosten}\Afootnote{ \textit{ (1) }\ machen wird, \textit{ (2) }\ ehe \textit{ L}}} es in see gehen kan, erfodern wird zum wenigsten \hfill \edtext{9365\edlabel{diesesallesend}}{\lemma{9365}\Bfootnote{Bei Witsen: 93635 gulden}} \edtext{g\"{u}lden}{\lemma{9365}\Afootnote{ \textit{ (1) }\ fl \textit{ (2) }\ g\"{u}lden \textit{ L}}}.
\pend
\pstart Ein also gebautes schiff\protect\index{Sachverzeichnis}{Schiff} kan lange jahre dauern, wie ich dann sagt Witsen\protect\index{Namensregister}{\textso{Witsen,} Nicolaes 1641\textendash 1717} ein Englisch schiff\protect\index{Sachverzeichnis}{Schiff!englisches} gesehen, so 70 jahr alt. Und wo em einen schiff\protect\index{Sachverzeichnis}{Schiff} nichts ungemeines Vorst\"{o}st, kan es 20, 30 bis 50 jahr dauern. Alleine die meisten fahrzeuge kommen vor der Zeit durch wind, und wetter, ungl\"{u}ck und feinde, umb den hals.
\pend 
\clearpage
\pstart \edtext{Artic 19 ist pag 280 bis 285 ist eine liste von allen beweglichen dingen ein schiff\protect\index{Sachverzeichnis}{Schiff}, und gereitschaft zum schiffs gebrauchs, auch defension und speise, vor ein jahr vor ein schiff\protect\index{Sachverzeichnis}{Schiff} vor 100 mannen so etwa nach Curacao\protect\index{Ortsregister}{Cura\c{c}ao (Curacao)}[,] Aleppo\protect\index{Ortsregister}{Aleppo} und Guinea\protect\index{Ortsregister}{Guinea} gehet. Dießes meritirte ganz copiiret zu werden}{\lemma{Artic 19}\Bfootnote{Von Artic 19 bis zu werden vgl. \textsc{N. Witsen}, \cite{00153}a.a.O., S.~280}}.
\pend 
\pstart[159 r\textsuperscript{o} Mitte rechts] Witsen\protect\index{Namensregister}{\textso{Witsen,} Nicolaes 1641\textendash 1717} \edtext{parle expr\`{e}s}{\lemma{Witsen}\Afootnote{ \textit{ (1) }\ a un chapitre expr\`{e}s \textit{ (2) }\ parle expr\`{e}s \textit{ L}}} du bastiment des vaisseaux\protect\index{Sachverzeichnis}{bâtiment des vaisseaux} fran\c{c}ois\edtext{}{\lemma{fran\c{c}ois}\Bfootnote{Zur Darstellung des Schiffbaus in Frankreich durch Witsen vgl. \textsc{N. Witsen}, \cite{00153}a.a.O., S.~195ff.}}. Item des bastimens\protect\index{Sachverzeichnis}{b\^{a}timent des vaisseaux} des \edtext{Anglois}{\lemma{Anglois}\Bfootnote{Zur Beschreibung des Schiffbaus in England durch Witsen vgl. \textsc{N. Witsen}, \cite{00153}a.a.O., S.~200ff.}}. Il loue les anglois de ce qu'ils marquent exactement sur le papier les proportions du vaisseau\protect\index{Sachverzeichnis}{vaisseau} qu'ils font bastir. \edtext{Die\edlabel{englanderstart}}{{\xxref{englanderstart}{englanderend}}\lemma{Die}\Bfootnote{Von Die Engl\"{a}nder bis diefte zusammen vgl. \textsc{N. Witsen}, \cite{00153}a.a.O., S.~200.}} Engl\"{a}nder wenn sie vor bekand angenommen haben die lange des kiels\protect\index{Sachverzeichnis}{Kiel}, die dieffe in die [h\"{o}he]\edtext{}{\Afootnote{h\"{o}hle\textit{\ L \"{a}ndert Hrsg. } }}, und die Breite vom Boden. Der große mast\protect\index{Sachverzeichnis}{Mast} ist bey ihnen 2$\displaystyle\frac{1}{2}$\rule[-4mm]{0mm}{10mm} die L\"{a}nge vom Boden, oder sie nehmen auchwohl die breidte und dieffe vom schiff\protect\index{Sachverzeichnis}{Schiff} zusammen; solches duppelt, und was komt durch 3 getheilt, vor die L\"{a}nge des großen masts\protect\index{Sachverzeichnis}{Mast}; andre nehmen die breite und diefte \edlabel{englanderend}zusammen, \edtext{so}{\lemma{zusammen,}\Afootnote{ \textit{ (1) }\ deßen \textit{ (2) }\ so \textit{ L}}} \edtext{verduppelt\edlabel{verduppeltstart}}{{\xxref{verduppeltstart}{verduppeltend}}\lemma{so ver\-duppelt}\Bfootnote{Von so verduppelt bis reckenen etc. vgl. \textsc{N. Witsen}, \cite{00153}a.a.O., S.~201.}} die L\"{a}nge des großen masts\protect\index{Sachverzeichnis}{Großmast}. Andre f\"{u}gen die l\"{a}nge vom kiel\protect\index{Sachverzeichnis}{Kiel}, die breite vom boden nebens der diefte zusammen, und zur sum f\"{u}gen sie das verschill\footnote{\textit{\"{U}ber verschill:} (an differentia)} zwischen der diefte und breite. Dießes multiplicirt durch die breite vom boden, das product darvon gedeelt durch die lezte Sum, davon die uytkomst verdoppelt, ist das begerte. Par exemple: l\"{a}nge von kiel\protect\index{Sachverzeichnis}{Kiel} 86. Breite vom boden 33, die diefte 15. Summa 134. Adde deren unterschied zwischen breite und diefte, so ist 18 die summa ist 152, multiplicirt durch 33 giebt 5016 getheilt durch 152 \edtext{blijft}{\lemma{152}\Afootnote{ \textit{ (1) }\ giebt \textit{ (2) }\ blijft \textit{ L}}} de uytkomst 33, en verduppelt 66, die begehrte l\"{a}nge. 
\pend 
\pstart Focken mast\protect\index{Sachverzeichnis}{Fockmast} $\displaystyle\frac{8}{19}$\rule[-4mm]{0mm}{10mm} theil von großen mast\protect\index{Sachverzeichnis}{Großmast} de groote steng\protect\index{Sachverzeichnis}{groote steng} $\displaystyle\frac{1}{2}$\rule[-4mm]{0mm}{10mm} \edtext{des}{\lemma{}\Afootnote{des \textit{ erg.} \textit{ L}}} großen masts\protect\index{Sachverzeichnis}{Großmast}. Die große bramseegels stang\protect\index{Sachverzeichnis}{Bramstenge} $\displaystyle\frac{1}{4}$\rule[-4mm]{0mm}{10mm} \textit{de voor of }\textit{fockesteng}\protect\index{Sachverzeichnis}{fockesteng}\textit{ de helfft van de }\textit{voormast}\protect\index{Sachverzeichnis}{voormast}\textit{, en de }\textit{voor Bramzeils steng}\protect\index{Sachverzeichnis}{Voor Bramzeils steng}\textit{ de helfft van de }\textit{fockesteng}\protect\index{Sachverzeichnis}{fockesteng}\textit{, de }\textit{Boeg spriet}\protect\index{Sachverzeichnis}{Boegspriet} \textit{hebbe de lengte van de fockemast}\protect\index{Sachverzeichnis}{fockesteng}\textit{. De Bezaen of Mißne Mast}\protect\index{Sachverzeichnis}{Besan}\textit{ hefft de hoogte van de groote Mars} \edtext{\textit{seegels}}{\lemma{\textit{Mars}}\Afootnote{ \textit{ (1) }\ \textit{zeils} \textit{ (2) }\ \textit{ seegels} \textit{ L}}}\textit{ mast van het bovenste Verdeck af te reckenen\edlabel{verduppeltend} etc}. \edtext{De Englander nehmen ihre Seegel\protect\index{Sachverzeichnis}{Segel} hoher als wir in}{\lemma{De Englander}\Bfootnote{Von De Englander bis als wir in vgl. \textsc{N. Witsen}, \cite{00153}a.a.O., S.~205.}}  \edtext{Holland\protect\index{Ortsregister}{Holland (Hollandia)}\edlabel{hollandstart}}{{\xxref{hollandstart}{hollandend}}\lemma{in Holland}\Bfootnote{Von Holland bis zu gebrauchen vgl. \textsc{N. Witsen}, \cite{00153}a.a.O., S.~206.}} thun, verdoppelen offt ihre Segel\protect\index{Sachverzeichnis}{Segel} welches bey uns unnothige \edtext{Kost geacht}{\lemma{unnothige}\Afootnote{ \textit{ (1) }\ Kost gem \textit{ (2) }\  Kost geacht \textit{ L}}} werd. Sie ubertreffen auch die onsre in \textit{veel voudig voeren}, von Stagseegelen\protect\index{Sachverzeichnis}{Stagsegel}. Das auffbauen ihrer Schiff\protect\index{Sachverzeichnis}{Schiff} geschieht nicht wie bey uns \textit{op vlacke}-wercken \textit{maer in docken, 't geen} vacken seyn, 't \textit{welcke door opgelatene schut-deuren water} k\"{o}nnen in nehmen wenn die Schiff\protect\index{Sachverzeichnis}{Schiff} gebauet seyn, \textit{om die doen} rijsen en Zeel-reete bringen. Sie sind gewont ihre Schiffkielen\protect\index{Sachverzeichnis}{Kiel} aus mehr stucken zusammen zu sezen als die Holl\"{a}nder. Sezen alle das werck auff \textit{palen} oder \textit{stocken} anstatt daß men hier das kiel\protect\index{Sachverzeichnis}{Kiel} auff \textit{balcken legt}. Bauen ihre schiff\protect\index{Sachverzeichnis}{Schiff} \textit{in de boeg het zwaerst, en de} plancken zeyn \textit{daer't} dickst, wegen des großen anstoßes den sie leyden vom anstoesenden \textit{water, en om} uytbarsting by steven vor zu kommen. Zwischen die leggers maecken sie schleuven \textit{tot schot voor 't water waer kettings of towen door komen, die na de pomp-put strecken, welcke bewegelijck seyn om 't water te konnen roeren, en alle verv\"{u}ilinge te weeren.} 't Kolzen\protect\index{Sachverzeichnis}{kolzem} wird bey ihnen gemacht \textit{'t eenenmahl van enen gestalte binnewarts, als de} \textit{Kiel}\protect\index{Sachverzeichnis}{kiel} \textit{buyten waerts, waer tegen Slaep Balcken} weder seyts \textit{van afleggen na de Schips}\protect\index{Sachverzeichnis}{schip} Siiden \textit{toe. Achter} zyn \textit{hun scheppen}\protect\index{Sachverzeichnis}{schip} \textit{alle meest onder} rond \textit{en niet plat. De} plancken \textit{die zo voor als achter in }\textit{steven}\protect\index{Sachverzeichnis}{steven}\textit{ kommen werden bey haer met} schwaere \textit{bouts daer an gekloncken, 't geen men hier meest mit wel te voegen te wege bringt}. Es ist ein großer irrthumb bey ihnen (+ den Englischen +) daß bey ihnen offt die schiffe\protect\index{Sachverzeichnis}{Schiff} [159 v\textsuperscript{o}, linke Sp.] gemacht werden oben scharff und unten breit deßen das gegentheil geschehen, soll derweil unden scharff und schmahl schneit die See, \textit{en geeft schot an 't schip}\protect\index{Sachverzeichnis}{schip}, oben breit vergr\"{o}st den inhalt \edtext{und Ladung.}{\lemma{}\Afootnote{und \textbar\ große \textit{ gestr.}\ \textbar\ Ladung. \textit{ L}}} Darumb man hier zulande die Plancken \textit{voor zoo veel buigt als 't moeglijck is}, wie wohl dat im die mittelmaß macht zunehmen, dann die scharffe und schmahle schiff\protect\index{Sachverzeichnis}{Schiff} tragen wenig, die schwere wollen nicht vort, und sein \textit{loom als} \edtext{\textit{koeien}}{\lemma{\textit{als}}\Afootnote{ \textit{ (1) }\ koeifen \textit{ (2) }\ \textit{koeien} \textit{ L}}}. D'Engelsche \textit{seilen mede daer in, dat zy hunne }\textit{schepen}\protect\index{Sachverzeichnis}{schip}\textit{ tr\"{o}gsche wijß} zaer breid \textit{en onbeesneden vaak maken; achter al te plat, en lootlijnig}. Und wie wohl sie wohl wißen, daß ein schiff\protect\index{Sachverzeichnis}{Schiff} zu schmahl und zu schmuig wenig tragt, zu dick und zu groß nicht vort will, so sieet man sie doch\ \textendash\ mehr als die hollander, von dießen regeln abweichen; hingegen vint man offt daß sie beßer seegeln als die Hollander, dieweil sie sich mehr befleißen, \textit{plat laeg, en lang te timmeren}, als man hier thut. Allein, trachtende die schiff\protect\index{Sachverzeichnis}{Schiff} zu schnell lauffend zu machen, machen sie selbige zu scharff, daß sie oft umbfallen; wenn man sie von ballast und schwehre entbl\"{o}st. Sie sind sonst wackere seeleute und wißen sich wohl zu retten in Zeit von noth. Sparen ihre gort-zacken\protect\index{Sachverzeichnis}{gort\textendash zacken} nicht umb die L\"{o}cher under waßer zu f\"{u}llen. Dieweil sie \edtext{wißen}{\lemma{sie}\Afootnote{ \textit{ (1) }\ wollen, \textit{ (2) }\ wißen \textit{ L}}} daß solche genezt schwellen und den plaz f\"{u}llen. Schmelten \textit{hun teer}\protect\index{Sachverzeichnis}{Teer}\textit{ met gloeiende kogels om brant te schouwen en} schlaen \textit{zeilen}\protect\index{Sachverzeichnis}{zeil}\textit{ voor reten en openinge in tijd} von \textit{storm om te kalefaten en te} kluzen \textit{bey stillte. Ja ombinden de }\edtext{\textit{schepen}}{\lemma{\textit{de}}\Afootnote{ \textit{ (1) }\ \textit{schippen}\protect\index{Sachverzeichnis}{schip|textit} \textit{ (2) }\ \textit{schepen} \textit{ L}}}\textit{ mit towen in hooge noth en weten bey gebreck van }\textit{anckers}\protect\index{Sachverzeichnis}{Anker}\textit{ gevulde kisten} mit loot \edtext{bley}{\lemma{}\Afootnote{bley \textit{ erg.} \textit{ L}}} of eisen\protect\index{Sachverzeichnis}{Eisen} in ihre stelle zu gebrauchen\edlabel{hollandend}. \edtext{Keine\edlabel{keinenationstart}}{{\xxref{keinenationstart}{keinenationend}}\lemma{Keine}\Bfootnote{Von Keine nation bis en Hantbuß vgl. \textsc{N. Witsen}, \cite{00153}a.a.O., S.~207.}} nation hat so viel eisenwerck und eiserne Negel\protect\index{Sachverzeichnis}{Eisennagel} an ihren schiffen\protect\index{Sachverzeichnis}{Schiff} als sie. \textit{Hare pompen\protect\index{Sachverzeichnis}{pomp} seyn ketting pompen\protect\index{Sachverzeichnis}{ketting pomp}, die} mitten im schiff\protect\index{Sachverzeichnis}{Schiff} \textit{staen}, welches loblicher als die \textit{hierlantsche} pompen\protect\index{Sachverzeichnis}{pomp}, denn sie nicht so balld unreinen werden, aber hingegen hinderlich in des schiffs\protect\index{Sachverzeichnis}{Schiff} raum, und einen unlieblichen gelaut geben. Hier waßer und bier wird mit pompens\protect\index{Sachverzeichnis}{pomp} \edtext{oben ausgezapt}{\lemma{oben}\Afootnote{ \textit{ (1) }\ ausgedapt \textit{ (2) }\ ausgezapt \textit{ L}}}: dient zur erhaltung das es nicht verderbe.
\pend 
\pstart Sie schmieren ihre schiff\protect\index{Sachverzeichnis}{Schiff} von außen mit \textit{seep\protect\index{Sachverzeichnis}{seep} en talck}\protect\index{Sachverzeichnis}{talk}, die Hollander allein mit schmeer\protect\index{Sachverzeichnis}{smeer}, \textit{met gekalckt} kannefaß \textit{en dat over gotten mit heeten pick, breuwen sie in de reeten tegeens 't }\textit{ongediert}\protect\index{Sachverzeichnis}{ongedierte}, \textit{'t is bey haer en gebruyck in 't schlaen, de }\textit{schepen}\protect\index{Sachverzeichnis}{schip}\textit{ rondom mit roode }\textit{schans-kleeden}\protect\index{Sachverzeichnis}{schans\textendash kleed}\textit{ te bedecken, de wijse van haer enteren is op 't hooghst van 't }\textit{schip}\protect\index{Sachverzeichnis}{schip}\textit{. 't Zy aens hut of back wel vorsien van Enterbyl, Sabel en Hantbuß\edlabel{keinenationend}}. \edtext{Er sezte eine Englische Schips Instruction\protect\index{Sachverzeichnis}{Schiffsinstruktion} oder ordre\protect\index{Sachverzeichnis}{Schiffsordnung} unterm Nahmen des Herzogs von Jorck\protect\index{Namensregister}{\textso{England: Jakob II.,} K\"{o}nig von England 1685\textendash 1688} Hoch Hohen admirals\protect\index{Sachverzeichnis}{Hochadmiral} von England\protect\index{Ortsregister}{England (Anglia)}. Ist an den Capitain gericht, and sehr notabel}{\lemma{Er}\Bfootnote{Von Er sezte bis sehr notabel vgl. \textsc{N. Witsen}, \cite{00153}a.a.O., S.~209.}}. \edtext{Ist schohn dazu gedacht, das das Britannische mar\protect\index{Ortsregister}{Armelkanal@\"{A}rmelkanal (Britannische mar)} gehet bis Cap finisterrae\protect\index{Ortsregister}{Kap Finisterre (Cap finisterrae)}}{\lemma{Ist}\Bfootnote{Von Ist schohn bis finisterrae vgl. \textsc{N. Witsen}, \cite{00153}a.a.O., S.~212.}}.\pend 