
\pstart  Non est tamen negandum Gravitatem\protect\index{Sachverzeichnis}{gravitas} \edtext{individualem alicujus massae}{\lemma{Gravitatem}\Afootnote{ \textit{ (1) }\ totius massae\protect\index{Sachverzeichnis}{massa|textit} \textit{ (2) }\ individualem alicujus massae \textit{ L}}} Elaterium\protect\index{Sachverzeichnis}{elaterium} habentis, qualis est aerea, plurimum  influere in specificam, pondus\protect\index{Sachverzeichnis}{pondus} enim molis incumbentis superioris,  auget inferioris compressionem seu densitatem,  ac proinde gravitatem specificam\protect\index{Sachverzeichnis}{gravitas!specifica}. At vero  connexio ista utriusque gravitatis\protect\index{Sachverzeichnis}{gravitas}, neque ita perpetua,  neque ita universalis, neque ita secura est ut contenti  esse cum ratione possimus instrumento uno ad  mensurandam utramque. Idque in aere  speciatim multis indiciis\edtext{}{\lemma{}\Afootnote{indiciis  \textbar\ mihi \textit{ gestr.}\ \textbar\ videtur \textit{ L}}} videtur comprobari.  Quae \edtext{mox commodius}{\lemma{Quae}\Afootnote{ \textit{ (1) }\ postea \textit{ (2) }\ mox commodius \textit{ L}}} exponam, ubi disquirendum erit,  plusne ad aeris tempestatem indicandam gravitas\protect\index{Sachverzeichnis}{gravitas}\edtext{}{\lemma{}\Afootnote{gravitas  \textbar\ ejus \textit{ gestr.}\ \textbar\ massae \textit{ L}}} massae\protect\index{Sachverzeichnis}{massa} an speciei conferat, ne  idem bis dicere necesse sit.
\pend
 \pstart  II\textsuperscript{da} ergo pars est tractationis hujus, quaerere \textso{an tempestates aeris magis cum }\textso{gravitate}\protect\index{Sachverzeichnis}{gravitas}\textso{ ejus specifica quam cum individuali, seu  cum pondere totius }\textso{massae}\protect\index{Sachverzeichnis}{massa}\textso{ connectantur.}  Duo sunt quae potissimum per aeris gravitatem\protect\index{Sachverzeichnis}{gravitas} praenosci posse sperantur, pluviae  (sub quibus grandines quoque, et pluvias hyemales  seu nives comprehendo), et venti. Pluviae quia  aerem onerant exonerantve; venti, quia sustinent. \edtext{Et a vaporibus quidem pluvialibus  manifestum est specificam aeris gravitatem}{\lemma{sustinent.}\Afootnote{ \textit{ (1) }\ Ac de pluviis quidem manifestum est, specificam aeris gravitatem\protect\index{Sachverzeichnis}{gravitas!aeris|textit} \textit{ (2) }\ Et [...] gravitatem \textit{ L}}}  semper immutari. Nam aer vaporibus oneratus differt  a puro ut aqua sale\protect\index{Sachverzeichnis}{sal} aliquo dissoluto impraegnata  a dulci: quare ut aquam ita et aerem hac impraegnatione incrassari, ac proinde in eadem mole seu expansione  graviorem fieri necesse est. At vero Cylindri aerei  pondus non \edtext{semper  augetur,}{\lemma{non}\Afootnote{ \textit{ (1) }\ habet necessariam cum aer \textit{ (2) }\ semper  augetur, \textit{ L}}} ob aerem vaporibus magis oneratum:  potest enim fieri \edtext{et saepe, credo, fit,}{\lemma{et}\Afootnote{ saepe, credo, fit, \textit{ erg.} \textit{ L}}} ut aere in imo cylindri id est  prope nos onerato et ad pluviam disposito,  aer in summo sit tanto liberior, et exoneratior, \edtext{et rarior, et motibus quibusdam sustentatus}{\lemma{}\Afootnote{et rarior, et motibus quibusdam sustentatus \textit{ erg.} \textit{ L}}}, ac proinde cylindri totius pondus  non ideo \edtext{nec congruenter}{\lemma{}\Afootnote{nec congruenter \textit{ erg.} \textit{ L}}} sit auctum. At vero non a summi  sed nostri aeris statu capiendae sunt de futuro  aeris nostri statu, conjecturae. Adde quod massa\protect\index{Sachverzeichnis}{massa} aeris dato loco incumbentis, sit sector
  [106~v\textsuperscript{o}]  sphaerae potius quam cylinder. Unde illa  amplissima in summo spatia multarum  mutationum capacia sunt, quibus nihil \edtext{aut parum}{\lemma{}\Afootnote{aut parum \textit{ erg.} \textit{ L}}} aeris nostri status, attamen cylindri  aerei pondus, Mercuriique\protect\index{Sachverzeichnis}{mercurius} in Torricelliano Tubo\protect\index{Sachverzeichnis}{Tubus!Torricellianus} \edtext{suspensi}{\lemma{Tubo}\Afootnote{ \textit{ (1) }\ ascensi,  \textit{ (2) }\ suspensi \textit{ L}}} altitudo plurimum immutetur.
\pend 