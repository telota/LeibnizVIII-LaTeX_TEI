 \pstart [p.~39] IV. Verum extra casum hunc, et particulares alios nonnullos (quos hic certe nil attinet commemorare) generatim et illimitate conceptum. Problema solidum est, pluresque duabus solutiones admittit; id quod facile perspicietur concipiendo punctum datum (puta X) in primo casu extra angulum ABF jacere (vel intra eundem, in secundo) quo posito liquet e praecedentibus obtingere posse nonnunquam, ut duorum\footnote{\textit{Gedruckte Marginalie}: Fig. 50.} ad partes BF incidentium refracti concurrant ad X; quin et alterius unius ad partes BE incidentis refractum etiam per idem X trans\-ire quod cum subinde, dico contingere possit, inde certo consequetur \textit{Problema} solidum esse.\footnote{\textit{Hierzu im Anschluss}: Haec consequentia non est firma. Nam etsi ad aequationem necessario ascendat plus quam quadraticam, nihil tamen prohibet eam aequationem esse divisibilem.}
 \pend