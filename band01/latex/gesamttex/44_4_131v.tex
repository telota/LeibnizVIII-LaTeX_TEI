(3) \textso{Exper. Florent. de Glaciatione Naturali, p. 172} aqua distillata\protect\index{Sachverzeichnis}{aqua distillata} semper conglaciata est limpidior ordinaria, magisque transparens tantum in medio quasi nucleus apparuit glaciei opacioris et albicantioris unde exeunt in omnes  partes \selectlanguage{italian}\textit{come tante reste d'un ghiaccio della medesima qualita,}\selectlanguage{latin}\edtext{}{\lemma{\textit{qualita,}}\Bfootnote{\cite{00143}\textsc{L. Magalotti}, a.a.O., S.~CLXXII.}}  quasi inclusus esset glaciei \selectlanguage{italian}\textit{un riccio di castagno diacciato in un  pezzo di cristal di monte. }\edtext{}{\lemma{\textit{monte.}}\Bfootnote{\cite{00143}\textsc{L. Magalotti}, a.a.O., S.~CLXXII.}}\selectlanguage{latin}Circa \textso{aquam marinam} notandum cum \edtext{vesperi}{\lemma{cum}\Afootnote{ \textit{ (1) }\ nocte \textit{ (2) }\ vesperi \textit{ L}}} duo cyathi ea pleni expositi essent tempore, quo thermometrum\protect\index{Sachverzeichnis}{thermometrum} 50 graduum erat in novem. \selectlanguage{italian}\textit{In capo a un ora}\selectlanguage{latin}\edtext{}{\lemma{\textit{ora}}\Bfootnote{\cite{00143}\textsc{L. Magalotti}, a.a.O., S.~CLXXII.}} invenimus  unum horum, qui erat minor, incepisse gelu, sed modo ab aqua ordinaria  differente, \selectlanguage{italian}\textit{mentre in esso pareva che fossero state messe in gran  copia scagliuole di talco sottilissimamente sminuzzato. }\edtext{}{\lemma{\textit{sminuzzato.}}\Bfootnote{\cite{00143}\textsc{L. Magalotti}, a.a.O., S.~CLXXII.}}\selectlanguage{latin}Haec auferebant  transparentiam aquae, et ei dabant debilissimam consistentiam \selectlanguage{italian}\textit{qual'a  il sorbetto, che si piglia in gielo la state, allorche mancandogli esteriormente la neve si va }\edtext{\textit{struggendo.}}{\lemma{\textit{struggendo.}}\Bfootnote{\cite{00143}\textsc{L. Magalotti}, a.a.O., S.~CLXXII.}}\selectlanguage{latin} Postea continuantes observare invenimus gelu nonnihil firmius, in quantum multiplicatio delle scagliuole (puto  squamarum,) comminuerat partem fluidam aquae. Mane durior erat, sed  nunquam pervenit ad duritiem glaciei ordinariae, nam quantulumcunque  agitata glacies in aquam abibat. Figura squamarum erat longa,  et parcissime \selectlanguage{italian}\textit{larga e tra esse v'erano tuttavia di moltissime parti  fluide quindi la massa era affatto distaccata dal vaso, girandosi in esso  liberamente. La superficie era piana senza alcuna prominenza, e  in somma tutta la diversita consisteva in un'orditura piu rada, ed in  un ripieno assai piu fine che non e quello del ghiaccio ordinario.}\selectlanguage{latin}\edtext{}{\lemma{\textit{ordinario.}}\Bfootnote{\cite{00143}\textsc{L. Magalotti}, a.a.O., S.~CLXXIII.}}\pend
\pstart \textso{P. 173.} glacies ad alia congelanda maxime operatur cum sale\protect\index{Sachverzeichnis}{sal} aliquo aspergitur, ut constat. Jam prae caeteris sal armoniacum\protect\index{Sachverzeichnis}{sal!armoniacum}  intendit ejus virtutem. Nam vidimus aequalem quantitatem ejusdem aquae  aequalis temperiei in vasis vitreis similis figurae capacitatis et subtilitatis,  aequali quantitate glaciei pulverisatae circumdatis, \selectlanguage{italian}\textit{onde ne rimanessero  fasciati ugualmente,}\selectlanguage{latin}\edtext{}{\lemma{\textit{ugualmente,}}\Bfootnote{\cite{00143}\textsc{L. Magalotti}, a.a.O., S.~CLXXIII.}} \edtext{unius [glacies]\edtext{}{\Afootnote{glace\textit{\ L \"{a}ndert Hrsg. } }} sale armoniaco alterius}{\lemma{\textit{ugualmente,}}\Afootnote{ \textit{ (1) }\ altero \textit{ (2) }\ uninus [glacies] sale armoniaco alterius \textit{ L}}} aequali  quantitate salis nitri\protect\index{Sachverzeichnis}{sal!nitri} aspersa. Nam cum Thermometrum\protect\index{Sachverzeichnis}{thermometrum} 100 graduum  immersum aquae quae cum nitro\protect\index{Sachverzeichnis}{nitrum} gelari debebat erat grad. $\displaystyle7\frac{1}{2}\rule[-4mm]{0mm}{10mm}$ aliud simile  immersum in aquam sale armoniaco\protect\index{Sachverzeichnis}{sal!armoniacum} circumdatam, \selectlanguage{italian}\textit{postovi come l'altro a  g. 20. era gia sotto ai 5 e l'acqua avea cominciato a velare.}\selectlanguage{latin}\edtext{}{\lemma{\textit{velare.}}\Bfootnote{\cite{00143}\textsc{L. Magalotti}, a.a.O., S.~CLXXIII.}}  Sed et aqua ardens mire juvat operationem glaciei, cui si addatur sal\protect\index{Sachverzeichnis}{sal}  fiet operatio efficacissima. Facit et saccarum aliquid sed parum in  comparatione salium\protect\index{Sachverzeichnis}{sal}.\pend \pstart  Glacies quantum nonnihil ex multis observationibus colligi potuit, conservatur diutissime in plumbo\protect\index{Sachverzeichnis}{plumbum}, sic satis in stanno\protect\index{Sachverzeichnis}{stannum},  parum in cupro\protect\index{Sachverzeichnis}{cuprum}, et ferro\protect\index{Sachverzeichnis}{ferrum}, minus in auro\protect\index{Sachverzeichnis}{aurum}, et adhuc minus in argento\protect\index{Sachverzeichnis}{argentum} quanquam  id nonnunquam fefellerit. Quare id non datur ut Experientia certa.  Ait \edtext{Gassendus\protect\index{Namensregister}{\textso{Gassendi} (Gassendus), Pierre 1592\textendash 1655}}{\lemma{Gassendus}\Bfootnote{\cite{00143}\textsc{L. Magalotti}, a.a.O., S.~CLXXIV.}} laminam glaciei sale\protect\index{Sachverzeichnis}{sal} aspersam largiter, fortissime Tabulae adhaerere. Idque verum experti sumus. Sed idem noluit succedere sale\protect\index{Sachverzeichnis}{sal} nitro\protect\index{Sachverzeichnis}{nitrum}.\pend \pstart  Velum illud vitrorum aqua frigida aut glacie plenorum  nonnunquam in glaciem abit, et hoc accidit, quando glacies aut nix  contenta alterata est sale\protect\index{Sachverzeichnis}{sal} aut aqua ardente. Tunc similiter exhalat fumus  nebulosus et humidus, qui plerumque apparet derivari ex fundo vasis \selectlanguage{italian}\textit{di doue muove un soffio d'aura gelata}\selectlanguage{latin}\edtext{}{\lemma{\textit{gelata}}\Bfootnote{\cite{00143}\textsc{L. Magalotti}, a.a.O., S.~CLXXV.}} quae praeterquam quod sensibiliter  recognoscitur si manu accedas, adhuc apparet magis ex motu quem in  flamma candelae apposita producit. Non refert \edtext{cujus}{\lemma{refert}\Afootnote{ \textit{ (1) }\ quae \textit{ (2) }\ cujus \textit{ L}}} materiae  sint vasa; quantum ad figuram nonnihil refert, nam in cyathis, subito  fumare coepit \selectlanguage{italian}\textit{di sotto, al contrario le tazze sparse prima di fumar  dal fondo fumino per qualche breve spatio di tempo gagliardamente  par insu. In una tazza di oro sparsa,}\selectlanguage{latin}\edtext{}{\lemma{\textit{sparsa,}}\Bfootnote{\cite{00143}\textsc{L. Magalotti}, a.a.O., S.~CLXXVI.}} observatum, quod et in aliis \edtext{vasis}{\lemma{}\Afootnote{vasis \textit{ erg.} \textit{ L}}} evenire debet. Scilicet:  cessante fumo crusta glaciei incipiebat \selectlanguage{italian}\textit{a piovere a mo' di ruggiada, un gielo finissimo,  come poluere di vetro pesto, e dur\`{o} infinattanto che risoluto il ghiaccio nella tazza  anche quel sottil  panno esteriormente  gelato, finì di liquefarsi.}\selectlanguage{latin}\edtext{}{\lemma{\textit{liquefarsi.}}\Bfootnote{\cite{00143}\textsc{L. Magalotti}, a.a.O., S.~CLXXVI.}}\pend \pstart  Fumus iste ex  glacie diversus ab  eo qui ex rebus  ardentibus. Est  enim similis  nebulae matutinae surgenti.\pend \pstart  Voluimus observare an speculum concavum\protect\index{Sachverzeichnis}{speculum!concavum} expositum massae quingentarum librarum glaciei  sensibilem haberet repercussionem frigoris in thermometrum\protect\index{Sachverzeichnis}{thermometrum} sensibilissimum 400 graduum  in foco\protect\index{Sachverzeichnis}{focus} sphaerae collocatum, verum est subito descendere incepisse, sed vicinitas glaciei  reddidit dubiam experientiam. \edtext{At tecto speculo}{\lemma{experientiam.}\Afootnote{ \textit{ (1) }\ Sed tamen tecto sp \textit{ (2) }\ At tecto speculo \textit{ L}}} subito resurrexit spiritus vini\protect\index{Sachverzeichnis}{spiritus!vini} in thermometro\protect\index{Sachverzeichnis}{thermometrum}. Non audemus tamen hoc frigoris reverberium nimis  fidenter asseverare, cum multis cautelis opus sit.\pend 