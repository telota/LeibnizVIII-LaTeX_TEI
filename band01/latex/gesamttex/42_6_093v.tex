[93 v\textsuperscript{o}] \edtext{His}{\lemma{His}\Afootnote{\textit{ erg.} \textit{ L}}} ausim praedicere etiam suctionem elevationemve  aeris per antliam\protect\index{Sachverzeichnis}{antlia} communem in \edtext{Recipiente  exhausto}{\lemma{in}\Afootnote{ \textit{ (1) }\ eo Recipiente \textit{ (2) }\ Recipiente  exhausto \textit{ L}}} non minus quam in aere ordinario eventura esse.  Cujus rei experimentum ita sumi potest. \edtext{Inspice figuram Diarii novissimi  pag. 134\edtext{}{\lemma{134}\Bfootnote{\textsc{Chr. Huygens,} \cite{00062}a.a.O., S.~134 (\textit{HO} VII, S.~202). Vgl. auch [Fig. 1].}} in eo Vas Recipiens pneumaticum \textit{B}}{\lemma{potest.}\Afootnote{ \textit{ (1) }\ Mittatur  in Recipientem exhauriendum, \textit{ (2) }\ Vasi  \textit{(a)}\ aqua \textit{(b)}\ in Recipiente  Pneumatico posito \textit{(c)}\ in pneumatico Recipiente posito aqua infundatur, in aquam intret orificium antliae\protect\index{Sachverzeichnis}{antlia|textit} communis, Embolo\protect\index{Sachverzeichnis}{embolus|textit} instructae. \textit{ (3) }\ Vasi pneumatico ad  vacuum exhibendum, ut vocant comparato ut in Diar. proximo  pag. 134 depictum est \textit{ (4) }\ Vas Recipiens pneumaticum \textit{ (5) }\ Inspice [...] \textit{B}. \textit{ L}}} Vas  aquam continens \edtext{sed ita ut plenum non sit, \textit{D}}{\lemma{continens}\Afootnote{ \textit{ (1) }\ \textit{D} \textit{ (2) }\ sed [...] sit, \textit{D} \textit{ L}}} ampulla aqua plena \edtext{\textit{C} applicetur}{\lemma{\textit{C}}\Afootnote{ \textit{ (1) }\ addatur  ei \textit{ (2) }\ applicetur \textit{ L}}} antlia\protect\index{Sachverzeichnis}{antlia} vulgaris, orificio \edtext{suo}{\lemma{}\Afootnote{suo \textit{ erg.} \textit{ L}}} in vas \textit{D} ita \edtext{descendens}{\lemma{ita}\Afootnote{ \textit{ (1) }\ intrans \textit{ (2) }\ descendens \textit{ L}}},  ut aquae superficiem non attingat, nisi tum demum cum \edtext{vas \textit{D}}{\lemma{cum}\Afootnote{ \textit{ (1) }\ ea  vas \textit{ (2) }\ vas \textit{D} \textit{ L}}} aqua (aere non purgata) ex  ampulla \textit{C} descendente pene repletur. Natet in eadem  aqua vasis \textit{D} corpus leve \edtext{ut lignum}{\lemma{}\Afootnote{ut lignum \textit{ erg.} \textit{ L}}}, quod eodem tempore \edtext{ascendat quo aqua ex \textit{C} descendit,}{\lemma{}\Afootnote{ascendat quo aqua ex \textit{C} descendit, \textit{ erg.} \textit{ L}}} \edtext{ascendendoque liberet pondus suspensum embolo alligatum.}{\lemma{descendit,}\Afootnote{ \textit{ (1) }\ ascendens  pondus aliquod embolo\protect\index{Sachverzeichnis}{embolus|textit} antliae\protect\index{Sachverzeichnis}{antlia|textit} alligatum et alicubi sed \textit{ (2) }\ ascendendoque [...] alligatum. \textit{ L}}} 
Pondus  liberatum embolum\protect\index{Sachverzeichnis}{embolus} antliae\protect\index{Sachverzeichnis}{antlia} extrahet, et aquam per antliam\protect\index{Sachverzeichnis}{antlia}  prorsus ut in aere ordinario attollet. Hoc si verum comperietur, \edtext{certum erit}{\lemma{comperietur,}\Afootnote{ \textit{ (1) }\ sequitur \textit{ (2) }\ certum erit \textit{ L}}} suctionem quoque \edtext{antliarum\protect\index{Sachverzeichnis}{antlia} ab ea causa quae Mercurium in experimento Torricellii suspendit}{\lemma{antliarum}\Afootnote{ \textit{ (1) }\ ab aeris pressione\protect\index{Sachverzeichnis}{pressio!aeris|textit} minime ori \textit{ (2) }\  Nulla enim in recipiente aeris pressio\protect\index{Sachverzeichnis}{pressio!aeris|textit} est, alioquin \textit{ (3) }\ ab [...] suspendit \textit{ L}}},  seu ab aeris liberi gravitate\protect\index{Sachverzeichnis}{gravitas}, inclusive Elaterio\protect\index{Sachverzeichnis}{elaterium}, oriri non posse,  quia haec  \edtext{Mercurii\protect\index{Sachverzeichnis}{mercurius!purgatus} aere non purgati, suspensio}{\lemma{Mercurii}\Afootnote{ \textit{ (1) }\ suspensio (modo aere non purgetur) \textit{ (2) }\ aere non purgati, suspensio \textit{ L}}} in \edtext{aere exhausto cessat}{\lemma{in}\Afootnote{ \textit{ (1) }\ Tubo cessat \textit{ (2) }\ aere exhausto cessat \textit{ L}}}, antliae\protect\index{Sachverzeichnis}{antlia} autem suctio non cessaret. Ajunt Galilaeum\protect\index{Namensregister}{\textso{Galilei} (Galilaeus, Galileus), Galileo 1564\textendash 1642} cum  ab Artificibus audisset, aquam per antliam\protect\index{Sachverzeichnis}{antlia} non attolli in infinitum,\edtext{}{\lemma{infinitum,}\Bfootnote{\textsc{G. Galilei, }\cite{00050}\textit{Discorsi}, Leiden 1638, S.~17 (\textit{GO} VIII, S.~64).}}  hinc de aeris  \edtext{pressione\protect\index{Sachverzeichnis}{pressio!aeris} tandem superata}{\lemma{pressione}\Afootnote{ \textit{ (1) }\ ratiocinatum \textit{ (2) }\ tandem superata \textit{ L}}} conjecturam cepisse; Torricellium\protect\index{Namensregister}{\textso{Torricelli} (Torricellius), Evangelista 1608\textendash 1647} Mercurio\protect\index{Sachverzeichnis}{mercurius} tuborum \edtext{praelongorum}{\lemma{tuborum}\Afootnote{ \textit{ (1) }\ non ita \textit{ (2) }\ praelongorum \textit{ L}}} minus  indigo, famosum illud quod vocant vacui \edtext{experimentum}{\lemma{experimentum}\Bfootnote{\textsc{E. Torricelli, }\cite{00107}\textit{Brief an Ricci vom 11. Juni 1644}, in: C. Dati, \textit{Lettera a Filaleti}, Florenz 1663, S.~20f. (\textit{TO} III, S.~186\textendash188).}} \edtext{exhibuisse}{\lemma{experimentum}\Afootnote{ \textit{ (1) }\ ex quo liquor a \textit{ (2) }\ exhibuisse \textit{ L}}}. 
\edtext{Pascalius\protect\index{Namensregister}{\textso{Pascal} (Pascalius), Blaise 1623\textendash 1662} postea demonstrasse sibi visus est, laminarum cohaesionem, siphonis item antliaeque phaenomenon, aliaque id genus, quae vulgo fugae vacui adscribebantur ab aeris \edtext{gravitate,}{\lemma{gravitate,}\Bfootnote{\textsc{B. Pascal, }\cite{00081}\textit{Traitez de l'\'{e}quilibre des liqueurs}, Paris 1663, S. 6\textendash15 (\textit{PO} III, S.~201\textendash223).}}
Boylius\protect\index{Namensregister}{\textso{Boyle} (Boylius, Boyl., Boyl), Robert 1627\textendash 1691} Elaterium\protect\index{Sachverzeichnis}{elaterium} summa cum ratione \edtext{adjunxit.}{\lemma{adjunxit.}\Bfootnote{\textsc{R. Boyle, }\cite{00015}\textit{New experiments physico-mechanicall}, Oxford 1660, S.~22 (\textit{BW} I, S.~165).}}}{\lemma{}\Afootnote{Pascalius\protect\index{Namensregister}{\textso{Pascal} (Pascalius), Blaise 1623\textendash 1662} postea demonstrasse sibi visus est,  \textit{ (1) }\ omnes illos naturae effectus \textit{ (2) }\ laminarum [...] aeris \textit{(a)}\ pressione pendere \textit{(b)}\ gravitate, Boylius\protect\index{Namensregister}{\textso{Boyle} (Boylius, Boyl., Boyl), Robert 1627\textendash 1691} Elaterium\protect\index{Sachverzeichnis}{elaterium} summa cum ratione adjunxit. \textit{ erg.} \textit{ L}}} Sed ego ut Torricellianum experimentum\protect\index{Sachverzeichnis}{experimentum!Torricellianum} aeris pressionem\protect\index{Sachverzeichnis}{pressio!aeris}\edtext{}{\lemma{}\Afootnote{pressionem  \textbar\ omnino \textit{ gestr.}\ \textbar\ indicare \textit{ L}}} indicare pro demonstrato  habeo; ita asserere ausim Antliae\protect\index{Sachverzeichnis}{antlia} longe alias esse rationes. \edtext{}{\lemma{}\Afootnote{rationes.  \textbar\ Quod ut experimentis confirmetur \textit{ gestr.}\ \textbar\ Et \textit{ L}}} Et nihilominus concedo antliae\protect\index{Sachverzeichnis}{antlia} elevationem non ituram in infinitum, quemadmodum laminarum\protect\index{Sachverzeichnis}{laminae politae} quoque cohaesio maximis ponderibus \edtext{tandem vincitur}{\lemma{tandem}\Afootnote{ \textit{ (1) }\ divelli \textit{ (2) }\ vincitur \textit{ L}}},  etsi hanc quoque ostensum sit non oriri ab aeris \edtext{pressione\protect\index{Sachverzeichnis}{pressio!aeris}. Idem  ut amplioribus experimentis confirmetur, loco antliarum  praelongarum, quarum difficilis  usus est adhibendae sunt mediocres quidem,}{\lemma{pressione}\Afootnote{ \textit{ (1) }\ Ut ergo hujus rei experimentum capiatur vel longissimis  \textit{(a)}\ tubis  aqua plenis utendum est \textit{(b)}\ antliis\protect\index{Sachverzeichnis}{antlia|textit} utendum est, vel potius mediocribus  sugendus est. \textit{ (2) }\ Ut autem determinetur quam nihil h \textit{ (3) }\ Idem [...] quarum \textit{(a)}\ incommodissimus \textit{(b)}\ difficilis [...] quidem, \textit{ L}}} sed  aquae loco Mercurio\protect\index{Sachverzeichnis}{mercurius} applicandae. Nam ita apparebit, in quantam altitudinem per antlias\protect\index{Sachverzeichnis}{antlia} attolli possit Mercurius\protect\index{Sachverzeichnis}{mercurius}.  Si eadem antliae\protect\index{Sachverzeichnis}{antlia} et Experimenti  Torricelliani\protect\index{Sachverzeichnis}{experimentum!Torricellianum} ratio \edtext{est}{\lemma{est}\Afootnote{ \textit{ (1) }\ nulla antliae\protect\index{Sachverzeichnis}{antlia|textit}  vi attolli poterit \textit{ (2) }\ Mercurius \textit{ L}}} \edtext{Mercurius aere non purgatus per antliam in tantam praecise altitudinem ultra attolletur, quantam in Tubo Torricelliano suspensus manet  ultra horizontem vasis subjecti stagnantis.}{\lemma{Mercurius}\Afootnote{ \textit{ (1) }\ ultra \textit{ (2) }\ supra \textit{ (3) }\ nisi 30. ad summum pollices \textit{ (4) }\ aere [...] stagnantis. \textit{ L}}}  At si \edtext{diversa erunt eventa antliae et Experimenti Torricelliani, seu Barometri \edlabel{confi93va}confirmabitur}{\lemma{si}\Afootnote{ \textit{ (1) }\ diversae sunt rationes sequetur Mercurium\protect\index{Sachverzeichnis}{mercurius!purgatus|textit}  etiam aere non purgatum  \textit{(a)}\ attolli \textit{(b)}\ tantundem attolli  \textit{(c)}\ per antliam\protect\index{Sachverzeichnis}{antlia|textit}, quantum aere purgatus suspensus teneri  \textit{(d)}\ per antliam\protect\index{Sachverzeichnis}{antlia|textit} ultra 30. pollices attolli; quod si experimento sumto  (quod praescripta ratione facile est) experiemur a nullo  amplius dubi \textit{ (2) }\ diversa [...] confirmabitur \textit{ L}}}%
\edtext{}{\lemma{confirmabitur}\xxref{confi93va}{confi93vb}\Afootnote{ \textit{ (1) }\ quod paulo \textit{ (2) }\ Antliae\protect\index{Sachverzeichnis}{antlia|textit} et  ratio \textit{ (3) }\ quod in Recipiente exhausto  \textit{(a)}\ eventurum praesumo \textit{(b)}\ demonstratum iri praesumo rationes quoque Antliae et Barometri esse diversas, id quidem credo, \textit{(aa)}\ aerem non \textit{(bb)}\ Mercurium aere non purgatum eousque per Antliam attolli non posse, quousque purgatus; qui eousque attolli \textit{(aaa)}\ potest \textit{(bbb)}\ utique poterit, quousque purgatus in Tubo Torricelliano poterit suspensus teneri illud tamen vicissim \textit{(aaaa)}\ arbitror \textit{(bbbb)}\ apparebit,  \textit{(aaaaa)}\ solam causam quae \textit{(bbbbb)}\ causam solam \textit{(ccccc)}\ causam, quae Antliam in infinitum attollere neget fore eandem quae laminas duas separat, \textit{(aaaaa-a)}\ Antliam\protect\index{Sachverzeichnis}{antlia|textit} \textit{(bbbbb-b)}\  Embolum enim a Mercurio\protect\index{Sachverzeichnis}{mercurius|textit} \textbar\ aere non \textit{ erg. und gestr.}\ \textbar\ se esse separaturam, aeremque ex Mercurio\protect\index{Sachverzeichnis}{mercurius|textit}  \textit{(aaaaa-aa)}\ esse \textbar\ credo \textit{ gestr.} \textbar\ prodituram \textit{(bbbbb-bb)}\ elicitum spatium inter Antliam\protect\index{Sachverzeichnis}{antlia|textit} et Mercurium\protect\index{Sachverzeichnis}{mercurius|textit} esse impleturum;  \textit{(aaaaa-aaa)}\ et auguror Mercurium\protect\index{Sachverzeichnis}{mercurius|textit} nihilominus embolum\protect\index{Sachverzeichnis}{embolus|textit} ducentem, debilius \textit{(bbbbb-bbb)}\ quo facto loco Mercurii\protect\index{Sachverzeichnis}{mercurius|textit} attrahendi   aer interjectus ab Embolo\protect\index{Sachverzeichnis}{embolus|textit} protracto in summam raritatem   distendetur. Et si Epistomium\protect\index{Sachverzeichnis}{epistomium|textit} \textit{ (4) }\    Credo etiam eventurum esse, ut Mercurius\protect\index{Sachverzeichnis}{mercurius|textit} in Tubo amplo,   non aeque alte per antliam\protect\index{Sachverzeichnis}{antlia|textit} attollatur, ac in Tubo angusto, quod tamen   in  \textit{(a)}\ Tubo Torricelliano\protect\index{Sachverzeichnis}{Tubus!Torricellianus|textit} nihil refert; ratio est quia Mercurius\protect\index{Sachverzeichnis}{mercurius|textit}   in Tubo Torricelliano\protect\index{Sachverzeichnis}{Tubus!Torricellianus|textit} \textit{(b)}\ Barometro\protect\index{Sachverzeichnis}{barometrum|textit} nihil refert; ratio est quia Mercurius\protect\index{Sachverzeichnis}{mercurius|textit}   in Barometro\protect\index{Sachverzeichnis}{barometrum|textit} amplo etiam columnam aeris  \textit{(aa)}\ amplam \textit{(bb)}\ crassam   sustinet, at in Antlia\protect\index{Sachverzeichnis}{antlia|textit} ampla exacte tamen adaptata,   porro an \textit{ (5) }\ solamque [...] Mercurio  \textbar\ antea non purgato \textit{ erg.}\ \textbar\  eliciti, inter  \textit{(a)}\ antliam\protect\index{Sachverzeichnis}{antlia|textit} et \textit{(b)}\ Embolum [...] postea \textit{(aa)}\ aquam \textit{(bb)}\ aerem [...] at \textit{(aaa)}\ causam \textit{(bbb)}\ in [...] cur \textit{(aaaa)}\ Barometrum\protect\index{Sachverzeichnis}{barometrum|textit} \textit{(bbbb)}\ Mercurius [...] delapsus \textit{(aaaaa)}\ aliam ob causam \textit{(bbbbb)}\ denuo suspensus maneat \textit{(aaaaa-a)}\ , quod ob aeris contraprementis aequilibrium\protect\index{Sachverzeichnis}{aequilibrium|textit} contingit. Quod \textit{(bbbbb-b)}\    . Quod ob aeris  \textit{(aaaaa-aa)}\ contrariam \textit{(bbbbb-bb)}\ contraprementis aequilibrium contingere \textit{ L}}}% quod in Recipiente exhausto demonstratum iri praesumo  rationes quoque Antliae et Barometri esse diversas, id quidem credo, Mercurium aere non purgatum  eousque per Antliam attolli non posse,  quousque purgatus; qui eousque attolli utique poterit,  quousque purgatus in Tubo Torricelliano poterit suspensus teneri  illud tamen vicissim apparebit, causam, quae Antliam in infinitum attollere neget  fore eandem quae laminas duas separat,  Embolum
