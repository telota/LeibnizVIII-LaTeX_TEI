\edtext{Ampulla paulo antequam aqua ad summam illam saltus periodum  adveniet, glacie}{\lemma{brevissimum}\Afootnote{ \textit{ (1) }\ Aqua gla \textit{ (2) }\ Ampulla [...] glacie \textit{ L}}} exemta, etiam extra eam saltum fecit, et subito quasi instanti  fluxum perdidit et glacie astricta est, et quidem tota. Hinc jam cum  repetitis experimentis easdem semper periodos teneri observaremus, voluimus  in variis liquidis experimenta instituere, et notare hos status: statum naturalem\protect\index{Sachverzeichnis}{status naturalis},  statum primae levationis, statum contractionis, statum quietis, statum elevationis, statum saltus in puncto conglaciationis. Nam status motus post saltum prosecuti \edtext{rationem}{\lemma{prosecuti}\Afootnote{ \textit{ (1) }\ calculum \textit{ (2) }\ rationem \textit{ L}}} habendam non putavimus, cum non sit nisibus prosecutio  rarefactionis a gelu dum induratio penitus absolvitur. Caeterum Glacies istae  artificiales non habent totam illam duritiem, et praeterea rupturam pilae\protect\index{Sachverzeichnis}{pila}  vitreae metuimus, exeruimusque antequam eo res pervenire posset. In aqua fontana  ita circiter evenit, aqua in statu naturali\protect\index{Sachverzeichnis}{status naturalis} attingebat gradum vasis 143.  in saltu immersionis attigit gradum 145. \edtext{aut 146.}{\lemma{}\Afootnote{aut 146. \textit{ erg.} \textit{ L}}} postea descendit ad gradum $\displaystyle119\frac{1}{2}\rule[-4mm]{0mm}{10mm}$ ibique  quievit, inde ad 131. \edtext{129.}{\lemma{}\Afootnote{129. \textit{ erg.} \textit{ L}}} assurrexit, ac denique ad 170 (in alio vase ita longo,  ut effluere exiliendo non posset) exiliit.\edtext{}{\lemma{exiliit.}\Bfootnote{\cite{00143}\textsc{L. Magalotti}, a.a.O., S.~CLVII.}} \edtext{Saltus}{\lemma{exiliit.}\Afootnote{ \textit{ (1) }\ Facta quidem sunt experimenta \textit{ (2) }\  Saltus \textit{ L}}} maxime differebat in diversis fluidis, cum caetera sic satis congruerent. Saltus autem erat altior et velocior, in iis quae fortius  conglaciabantur. In aqua di canella stillata nullus plane observatus  saltus, sed ejus loco subito transiit ad motum elevationis paulo velociorem  solito, \edtext{quo tempore et gelu}{\lemma{solito,}\Afootnote{ \textit{ (1) }\ ubi et gelu a \textit{ (2) }\ quo tempore et gelu \textit{ L}}} corripiebatur. Aqua di neve  stratta lentius et differenter admodum ab aliis liquoribus conglaciabatur. Ea  enim cum saltus tempus erat accelerationem\protect\index{Sachverzeichnis}{acceleratio} tantum sumsit, sed lentam si motui caeterorum liquorum compararetur. Glacies producta non erat aequalis,  ut in aliis sed interrupta venis disordinatis in omnem partem. Vin rosso di  Chianti etiam saltu caruit, substituta acceleratione\protect\index{Sachverzeichnis}{acceleratio} motus. In Moscadello  bianco non fuit saltus, sed nec allevatio lenta, sed loco allevationis lentae  et saltus simul allevatio nonnihil solito celerior. In aceto\protect\index{Sachverzeichnis}{acetum} \edtext{albo}{\lemma{}\Afootnote{albo \textit{ erg.} \textit{ L}}} saltus agglaciationis  altissimus quidem, sed minoris velocitatis quam aquae majoris quam Moscadelli  aquae canellae et aceti\protect\index{Sachverzeichnis}{acetum} non distillati. Agro di limone lentissimam tantum elevationem habuit. Spiritus vitrioli\protect\index{Sachverzeichnis}{spiritus!vitrioli} lentissime assurrexit et uniformiter,  agglaciatus interim de loco in locum in diversis planis, ut aqua naturalis  messa in vasi di vetro ad agghiacciarsi in sereno. Oleum ecce condensatum tantum, totumque in pilam\protect\index{Sachverzeichnis}{pila} absorptum, atque ita conglaciatum est, sine omni rarefactione,  unde fit, ut oleum conglaciatum solum summergatur in fluido. Spiritus vini\protect\index{Sachverzeichnis}{spiritus!vini} mire condensatur frigido, sed nec rarefit, nec conglaciatur.\edtext{}{\lemma{conglaciatur.}\Bfootnote{\cite{00143}\textsc{L. Magalotti}, a.a.O., S.~CLXI--CLXV.}} Quod  jam pag. 168 sqq. attinet conglaciationes Naturales ibi sciendum  ipsam materiam, etiam insubtilissimis illis velis filisve duriorem, solidiorem et quasi magis cristallinam. Caeterum vasis variis vitreis terreis metallicis,  variae figurae magnitudinis, plenitatis adhibitis nunc huic nunc illi vento  expositis, nil nisi summa irregularitas reperta est, modo his modo illis primo  glaciantibus, eodem tempore. Nisi quod vasa terrea prae caeteris conglaciationem promoveant. Ordo congelationis naturalis hic est: primum  aqua \edtext{superior commincia}{\lemma{superior}\Afootnote{ \textit{ (1) }\ recipit se \textit{ (2) }\ commincia \textit{ L}}} \selectlanguage{italian}\textit{a rappigliarsi in giro, e da quel primo nastro  di gielo che ricorre la circomferentia del vaso}\selectlanguage{latin}\edtext{}{\lemma{\textit{vaso}}\Bfootnote{\cite{00143}\textsc{L. Magalotti}, a.a.O., S.~CLXIX.}} incipit mittere versus partes  medii subtilissima fila postea per omnem profunditatem, et ex his in omnem  partem postea fila ista incipiunt obtundi schiacciarsi manentia nihilominus   alia parte crassiora quam altera forma cultellorum \selectlanguage{italian}\textit{dalle costole dei  quali }\selectlanguage{latin}reincipiunt excire alia fila subtilissima, \selectlanguage{italian}\textit{ma fitti e spessi a guisa della  piuma, o delle foglie della palma e questi a quel primo ordito fanno  per modo di dire un ripieno scompigliato e confuso, finche crescendo per ogni  parte il lavoro si va compiendo la tela col totale agghiacciamento dell aqua.  La superficie poi di essa si vede tutta graffiata in varie diritture come  un cristallo intagliato a bulino finissimo.}\edtext{}{\lemma{\textit{finissimo.}}\Bfootnote{\cite{00143}\textsc{L. Magalotti}, a.a.O., S.~CLXIX.}} \selectlanguage{latin}Superficies apparet primum plana,  sed postea cum perficienda est fit colma, et irregularis. Glacies in vacuo magis  aequalis, magis dura, minus transparens, minus porosa, magis gravis in specie. [131~v\textsuperscript{o}]