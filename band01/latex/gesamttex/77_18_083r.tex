      
               
                \begin{ledgroupsized}[r]{120mm}
                \footnotesize 
                \pstart                
                \noindent\textbf{\"{U}berlieferung:}   
                \pend
                \end{ledgroupsized}
            
              
                            \begin{ledgroupsized}[r]{114mm}
                            \footnotesize 
                            \pstart \parindent -6mm
                            \makebox[6mm][l]{\textit{A}}Abschrift eines Textes von Schreiberhand. Autor nicht ermittelt: LH XXXVII 2 Bl. 83\textendash91, 5 Bl. 2\textsuperscript{o}, in der Mitte gefaltet. 17 1/3 S., 2 S. leer. Bl. 84 r\textsuperscript{o} im unteren Drittel der Seite 3 Zeichnungen. \pend
                            \end{ledgroupsized}
              
                            \begin{ledgroupsized}[r]{114mm}
                            \footnotesize 
                            \pstart \parindent -6mm
                            \makebox[6mm][l]{\textit{L}}Marginalien zu A auf Bl. 87 v\textsuperscript{o}, Bl. 88 r\textsuperscript{o} und Bl. 89 v\textsuperscript{o} sowie Bl. 90 r\textsuperscript{o}, v\textsuperscript{o} am linken Rand. Eine weitere Marginalie auf Bl. 91 r\textsuperscript{o} zwischen der 1. und 2. Zeile. \pend
                            \end{ledgroupsized}
                %\normalsize
                \vspace*{5mm}
                \begin{ledgroup}
                \footnotesize 
                \pstart
            \noindent\footnotesize{\textbf{Datierungsgr\"{u}nde}: Das vorliegende St\"{u}ck enth\"{a}lt die Kopie eines Textes, dessen Grundlage die Cartesianische Optik bildet. Wir ordnen das St\"{u}ck den Nummern 18 bis 22 unseres Bandes zu, in denen Leibniz die \cite{00038}\textit{Optik} Descartes' entweder referiert oder sich mit ihr auseinandersetzt. Die Zuordnung wird durch das f\"{u} M\"{a}rz 1672 nachgewiesene Wasserzeichen des Textr\"{a}gers best\"{a}tigt.}
                \pend
                \end{ledgroup}
            
                \vspace*{8mm}
                \pstart 
                \normalsize
            [83 r\textsuperscript{o}] Specilla Circularia\protect\index{Sachverzeichnis}{specillum!circulare}, sive quomodo per solas  Circulares figuras fieri possint omnis generis specilla\protect\index{Sachverzeichnis}{specillum}  tam Microscopia\protect\index{Sachverzeichnis}{microscopium} quam telescopia\protect\index{Sachverzeichnis}{telescopium}, etc.: eundem plane  effectum habentia, aut saltem quam proxime accedentem  ad eorum, quae per ellipticas aut hyperbolicas figuras  fieri possent. \pend \pstart  Notum jam omnibus satis est, quanta sit specillorum\protect\index{Sachverzeichnis}{specillum}  utilitas: myopes\protect\index{Sachverzeichnis}{myops} alias et senes novaque, post inventa microscopia\protect\index{Sachverzeichnis}{microscopium}, et telescopia\protect\index{Sachverzeichnis}{telescopium}, tam in coelis quam, \edtext{hic}{\lemma{}\Afootnote{hic \textit{ erg.} \textit{ L}}}, in terra, circa  nos magna copia detecta objecta, luculentum sunt testimonium  sed multa adhuc magis admiranda quam ea quae hactenus  detecta sunt promittere nobis videntur, imo procul omni  dubio horum ope ab astronomis motuum coelestium, a  Philosophis naturae corporum mixtorum; a Medicis  naturae et virium herbarum, et corporis humani, perfectior  longe notitia haberi poterit, quam unquam absque his  expectanda foret. Cumque hoc publice constaret, plurimi  fuere jam brevi, qui maxima cum diligentia specilla\protect\index{Sachverzeichnis}{specillum}  haec ad summam perfectionem perducere conati sunt. Sed  nulli id, meo judicio, melius successit quam incomparabili  viro Renato Descartes\protect\index{Namensregister}{\textso{Descartes} (Cartesius, des Cartes, Cartes.), Ren\'{e} 1596\textendash 1650}, cujus labori nihil plane superaddi 