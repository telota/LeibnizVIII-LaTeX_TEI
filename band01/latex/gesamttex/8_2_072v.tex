\pstart [72 v\textsuperscript{o}] \edtext{Boyl.\protect\index{Namensregister}{\textso{Boyle} (Boylius, Boyl., Boyl), Robert 1627\textendash 1691}\edlabel{tractsstart}}{{\xxref{tractsstart}{tractsend}}\lemma{Boyl.}\Bfootnote{Der Abschnitt Boyl. \textit{De relatione} bis \textit{ponderabilitate flammae} ist auf der ansonsten leeren Seite von Leibniz exakt dem Textfeld der Einleitung auf Bl. 72 r\textsuperscript{o} angepasst. Wegen des deutlich unterschiedenen Inhalts und der fehlenden Rubriken\"{u}berschrift ist er dennoch hier als eigenst\"{a}ndiger Eintrag aufgef\"{u}hrt. Auf die hier erw\"{a}hnten Titel Boyles wird in den Rubriken Pneumatica und Hydrostatica verwiesen.}} \textit{De\edlabel{fluidorumstart}\label{tracts} relatione aeris et flammae}\protect\index{Sachverzeichnis}{flamma}\edtext{}{{\xxref{fluidorumstart}{fluidorumend}}\lemma{\textit{flammae}}\Bfootnote{\textsc{R. Boyle, }\cite{00156}\textit{Tracts} (\textit{BW} 7, S.~73\textendash 226). Vgl. auch \cite{00216}\textit{An Accompt of two Books}, \textit{PT} 8 (1673), S.~5197\textendash 6006 (Fehler bei Sei\-tenz\"{a}hlung im Original) und \cite{}\textit{BW} 7, S.~73\textendash 226. Leibniz' Notizen entsprechen den \"{U}berschriften der Abschnitte und der drei Anh\"{a}nge in Boyles Werk. Die Annahme, Leibniz habe die lateinische Ausgabe, Rotterdam 1669, gelesen (\cite{00251}\textit{LSB} III, 1, S.~41), kann trotz der lateinischen Wiedergabe durch Leibniz nicht best\"{a}tigt werden, da diese Ausgabe das Material ganz anders ordnet, statt der Antwort auf H. Moore\protect\index{Namensregister}{\textso{Moore,} Henry 1614\textendash 1687} eine auf F. Linus\protect\index{Namensregister}{\textso{Linus,} Franciscus 1595\textendash 1675} enth\"{a}lt und 1673 auch nicht mehr \textit{novissimum} (\textit{LSB} III, 1, S.~\cite{00251}42, \cite{00249}86) ist.}}: de difficultate producendi flammam\protect\index{Sachverzeichnis}{flamma} sine aere; de difficultate propagandi flammam\protect\index{Sachverzeichnis}{flamma} actualem in vacuo Boyliano\protect\index{Sachverzeichnis}{vacuum!Boylianum}: Nova experimenta de relatione inter aerem et flammam vitalem animalium\protect\index{Sachverzeichnis}{flamma!vitalis animalium}. Conatus producendi animalia in vacuo Boyliano\protect\index{Sachverzeichnis}{vacuum!Boylianum}. Nova experimenta de explosionibus\protect\index{Sachverzeichnis}{explosio}. Nova experimenta de positiva et relativa levitate corporum\protect\index{Sachverzeichnis}{levitas!corporum} sub aquis. Nova experimenta de pressione Elaterii aeris in corpora sub aquis. Nova experimenta de differenti pressione corporum \edtext{gravium}{\lemma{gravium}\Afootnote{ \textit{ (1) }\ flu \textit{ (2) }\ solidorum \textit{ L}}} solidorum et fluidorum. Dissertatio Hydrostatica confutens objectiones Mori\protect\index{Namensregister}{\textso{Moore,} Henry 1614\textendash 1687}\edlabel{fluidorumend}. \edtext{\textit{De Effluviis corporum}\protect\index{Sachverzeichnis}{effluvium corporum}, \textit{de ponderabilitate flammae}\protect\index{Sachverzeichnis}{ponderabilitas flammae}}{\lemma{\textit{flammae}}\Bfootnote{Diese Titel stimmen weitgehend mit Boyles' auf die \cite{00156}\textit{Tracts} folgendem Werk \textsc{R. Boyle, }\cite{00238}\textit{Essays Effluviums Fire and Flame}, London 1673 (\cite{}\textit{BW} 7, S.~227\textendash 336), \"{u}berein.}}\edlabel{tractsend}.\pend
\pstart\noindent\raggedleft
\textso{Arithm.}\\\vspace{1.5ex}
Algeb.\\
Geomet.\\
\textso{Mus.}\\
\textso{Optic.}\\