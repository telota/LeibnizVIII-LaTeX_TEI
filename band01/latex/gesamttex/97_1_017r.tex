\pstart
[17 r\textsuperscript{o}] \edtext{Equilibre de M. Pascal\protect\index{Namensregister}{\textso{Pascal} (Pascalius), Blaise 1623\textendash 1662}}{\lemma{Equilibre}\Afootnote{de M. Pascal\protect\index{Namensregister}{\textso{Pascal} (Pascalius), Blaise 1623\textendash 1662} \textit{ erg.} \textit{ L}}} \textso{Observations de Mons. }\textso{Perrier}\protect\index{Namensregister}{\textso{P\'{e}rier} (Perrier, Perier, Perierius), Florin 1605\textendash 1702}\textso{.}\edtext{}{\lemma{\textso{Perrier}.}\Bfootnote{\textsc{F. P\'{e}rier, }\cite{00127}\textit{Recit des observations}, Paris 1663, S.~195\textendash203 (\textit{PO} II, S.~441\textendash451).}} Faisant des observations, pendant cinq ou 6 mois je trouuay \textit{que d'ordinaire  et communement le }\textit{viv argent}\protect\index{Sachverzeichnis}{vif argent}\textit{ se haussoit dans les tuyaux en temps froid et humide ou couuert, et s'abaissoit en temps chaud et sec; mais que quelque fois  aussi le }\textit{viv argent}\protect\index{Sachverzeichnis}{vif argent}\textit{ s'abaissoit le temps devenant plus froid et plus humide, et se haussoit quand le temps devenoit plus chaud ou plus sec.}\edtext{}{\lemma{\textit{sec.}}\Bfootnote{\textsc{F. P\'{e}rier, }\cite{00127}a.a.O., S.~199 (\textit{PO} II, S.~442).}} Les observations de Messieurs Canut\protect\index{Namensregister}{\textso{Chanut} (Canut, Canutus), Hector Pierre 1604\textendash 1662} et des Cartes\protect\index{Namensregister}{\textso{Descartes} (Cartesius, des Cartes, Cartes.), Ren\'{e} 1596\textendash 1650} \`{a} Stockholm\protect\index{Ortsregister}{Stockholm} confirmerent la même chose. \textit{Je crois pourtant qu'on pourroit faire cette regle avec quelque  certitude que le }\textit{viv argent}\protect\index{Sachverzeichnis}{vif argent}\textit{ se hausse toutes les fois que ces deux choses arrivent tout ensemble, s\c{c}avoir que le temps se refroidist, et qu'il  se charge ou couure, et qu'il s'abaisse au contraire, toutes les fois que ces deux choses arrivent aussi ensemble, que le temps devienne  plus chaud et qu'il se decharge par la pluye ou par la neige;}\edtext{}{\lemma{\textit{neige;}}\Bfootnote{\textsc{F. P\'{e}rier, }\cite{00127}a.a.O., S.~199f. (\textit{PO} II, S.~444).}} quand l'une de \edtext{deux d'une cost\'{e} se rencontre  avec}{\lemma{deux}\Afootnote{ \textit{ (1) }\ choses ar \textit{ (2) }\ d'une cost\'{e}  \textit{(a)}\ et \textit{(b)}\ se rencontre  avec \textit{ L}}} une de deux de l'autre celle qui prevaut l'emporte. \textit{L'}\textit{Argent viv}\protect\index{Sachverzeichnis}{vif argent}\textit{ baisse et hausse \`{a} toutes sortes de vents en toutes  saisons quoyque il soit ordinairement plus haut en hyver, qu'en est\'{e}, quoyque cette regle ne soit pas seure. Car par exemple je l'ay veu \`{a} }\textit{Clermont}\protect\index{Ortsregister}{Clermont}\textit{  le 16}\textsuperscript{\textit{me}}\textit{ de Janvier 1651 \`{a} 25. poulces 11. lignes et le 17 \`{a} 25 poulces dix lignes, qui est presque son plus bas estat. Il faisoit ces jours  l\`{a} un calme doux et un grand ouest. Et on l'a veu \`{a} }\textit{Paris}\protect\index{Ortsregister}{Paris (Parisii)}\textit{ le 9. d'Aoust 1649 \`{a} 28 poulces 2 lignes, qui est un Estat, qu'il ne passe  gueres.}\edtext{}{\lemma{\textit{gueres.}}\Bfootnote{\textsc{F. P\'{e}rier}, \cite{00127}a.a.O., S.~200f. (\textit{PO} II, S.~444).}} \textit{\`{A} }\textit{Clermont}\protect\index{Ortsregister}{Clermont}\textit{ le plus haut 26 poulces 11 lignes et demie, le 14 Fevr. 1651. Nort bien gel\'{e} et assez beau. Cela n'est arriv\'{e} que ce jour  l\`{a} car en beaucoup d'autres durant le même hyver, il a eu 26 poulces, 10 lignes ou 9 lignes, et même XI lignes le 5 Novemb. 1649. Le plus  bas 25 poulces 8 lignes, le 5 Octobr. 1649}\edtext{}{\lemma{\textit{1649}}\Bfootnote{\textsc{F. P\'{e}rier, }\cite{00127}a.a.O., S.~201 (\textit{PO} II, S.~445).}} \textit{quelques autres \`{a} 25 poulces 9 lignes ou 10 ou 11. La difference \`{a} }\textit{Clermont}\protect\index{Ortsregister}{Clermont}\textit{ entre le plus bas et le plus  haut 1. poul. 3. lign.}
$\displaystyle\frac{1}{2}\rule[-4mm]{0mm}{10mm}$.
% \begin{wrapfigure}{l}{0.4\textwidth}                    
                %\includegraphics[width=0.4\textwidth]{../images/Aus+Blaise+Pascal%2C+Traitez+de+l%27equilibre+des+liqueurs/LH035%2C15%2C01_017r/files/100194.png}
                        %\caption{Bildbeschreibung}
                        %\end{wrapfigure}
                        %@ @ @ Dies ist eine Abstandszeile - fuer den Fall, dass mehrere figures hintereinander kommen, ohne dass dazwischen laengerer Text steht. Dies kann zu einer Fahlermeldung fuehren. @ @ @ \\
                    \textit{\`{A} }\textit{Paris}\protect\index{Ortsregister}{Paris (Parisii)}\textit{ le plus haut 28 p. 7. l. le 3 et 5. 9bre 49, le plus bas 27 poulces 3 lignes et demie 4 octobr. 49}\edtext{}{\lemma{\textit{49}}\Bfootnote{\textsc{F. P\'{e}rier, }\cite{00127}a.a.O., S.~202 (\textit{PO} II, S.~445).}}. \textit{\`{A} }\textit{Stockholm}\protect\index{Ortsregister}{Stockholm}\textit{  le plus haut 28 poulces 7 l. le plus haut 8 Decembre 49 auquel jour M. }\textit{des Cartes}\protect\index{Namensregister}{\textso{Descartes} (Cartesius, des Cartes, Cartes.), Ren\'{e} 1596\textendash 1650}\textit{ remarque qu'il faisoit fort froid, le plus bas 26 poulces 4 lignes et $\displaystyle\frac{3}{4}\rule[-4mm]{0mm}{10mm}$ 6 May  1650 vent soit d'ouest temps trouble et doux.}\edtext{}{\lemma{\textit{doux.}}\Bfootnote{\textsc{F. P\'{e}rier, }\cite{00127}a.a.O., S.~202 (\textit{PO} II, S.~445).}} Les inegalitez \`{a} Stockholm\protect\index{Ortsregister}{Stockholm} plus grandes, et aussi plus promptes, comme en deux jours en Xbre, un pouce de  difference \`{a} Paris\protect\index{Ortsregister}{Paris (Parisii)} en dix jours. Xbre M. Perier\protect\index{Namensregister}{\textso{P\'{e}rier} (Perrier, Perier, Perierius), Florin 1605\textendash 1702} dit de pouuoir faire imprimer ces observations si l'on desire. Mons. Chanut\protect\index{Namensregister}{\textso{Chanut} (Canut, Canutus), Hector Pierre 1604\textendash 1662} ecrit, qu'il ne faut pas seulement observer le froid chaud humide et sec, trouble et serain, mais aussi celles des vents regnants, qui causent une diminution ou augmentation  uniforme et quasi reguliere.\edtext{}{\lemma{reguliere.}\Bfootnote{\textsc{P. Chanut, }\cite{00117}\textit{Copie d'une lettre \'{e}crite par Monsieur Chanut \`{a} Monsieur Perier}, Paris 1663, S.~205 (\textit{PO} II, S.~414).}}\pend