\pend \pstart [p.~98] [...] si vero specillum  admotum sit sphaericum cavum, et centrum illius in foco\protect\index{Sachverzeichnis}{focus} A statuatur, repercussi radij iterum in puncto A  colligentur, vt patet; redeunt enim singuli per lineam  perpendicularem, per quam in cavum sphaericum inciderunt, hinc duplum caloris incrementum, et dupla  vis puncti vstorij\protect\index{Sachverzeichnis}{punctum!ustorium}\footnote{\textit{Leibniz unterstreicht}: hinc duplum [...] puncti vstorij}: sed in his nulla est difficultas: denique  si radius PD cadat in conuexum parabolae, reflexus  ibit per DT; nempe anguli PDE, TDP aequales  esse constat ex dictis.