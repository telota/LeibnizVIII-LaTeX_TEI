\pstart \textso{Venti,} seu aeris motus magni,  aeque gravitatem specificam\protect\index{Sachverzeichnis}{gravitas!specifica}, ac individualem\protect\index{Sachverzeichnis}{gravitas!individualis}  mutant. Nam cum ventus aerem loco incumbentem fert, minor est gravitas cylindri; sed et  ideo quoque minor est aeris densitas, cum  enim minus \edtext{prematur, aperit}{\lemma{prematur,}\Afootnote{\textbar\ aer \textit{ gestr.}\ \textbar\ aperit \textit{ L}}} expanditque sese  quantum potest, ac proinde fit rarior, ergo et  speciei levioris. Hanc apparet omnem mutationem in cylindro, seu gravitate \edtext{individuali\protect\index{Sachverzeichnis}{gravitas!individualis}, derivari quoque in}{\lemma{individuali}\Afootnote{ \textit{ (1) }\ conferre quoque ad \textit{ (2) }\ derivari quoque in \textit{ L}}} raritatis  densitatisque differentias seu \edtext{Gravitatem individualem}{\lemma{seu}\Afootnote{ \textit{ (1) }\ differentiam \textit{ (2) }\ Gravitatem individualem \textit{ L}}}; non contra: ac proinde  quicquid individualis habet utile, id in specifica  largius inesse.
\pend 
\pstart  Sed et discrimen aliquod observandum  est: nimirum variatio in cylindro parum \edtext{efficit}{\lemma{parum}\Afootnote{ \textit{ (1) }\ sentiri pot \textit{ (2) }\ efficit \textit{ L}}} ad specificae variationem  nisi fiat in loco nobis vicino. Pone enim  summum Atmosphaerae\protect\index{Sachverzeichnis}{atmosphaera} motibus quibusdam  agitati, at imum vaporibus onerari,  sane plus ad nos haec oneratio, quam allevatio  illa pertinebit: at motus massae nobis vicinae, \edtext{seu venti}{\lemma{}\Afootnote{seu venti \textit{ erg.} \textit{ L}}}  sine dubio non pondus tantum, sed et densitatem  mutant.
\pend 
\pstart  At hoc loco removenda est difficultas  quadam quae nasci posset, scilicet Gravitatem specificam\protect\index{Sachverzeichnis}{gravitas!specifica} \edtext{calore}{\lemma{specificam}\Afootnote{ \textit{ (1) }\ densitate \textit{ (2) }\ calore \textit{ L}}} quoque et frigore\protect\index{Sachverzeichnis}{frigus}, variari,  non individualem\protect\index{Sachverzeichnis}{gravitas!individualis}: calorem\protect\index{Sachverzeichnis}{calor} autem et frigus\protect\index{Sachverzeichnis}{frigus}  nihil ad pluviam ventosque, et caeteras\edtext{}{\lemma{}\Afootnote{caeteras  \textbar\ temporis \textit{ gestr.}\ \textbar\ mutatione, \textit{ L}}}  mutatione, similes pertinere. Hic respondendum  est, primum non adeo esse exploratum, quod  ajunt, frigus\protect\index{Sachverzeichnis}{frigus} in densitate, calorem\protect\index{Sachverzeichnis}{calor}  in raritate aeris consistere;  nam calor\protect\index{Sachverzeichnis}{calor} frigusque\protect\index{Sachverzeichnis}{frigus} etiam vitra  penetrant, etiam in loco aere evacuato  efficaciam exercent \edtext{quod certis experimentis constat}{\lemma{}\Afootnote{quod certis experimentis constat \textit{ erg.} \textit{ L}}}: deinde fateor calore\protect\index{Sachverzeichnis}{calor}  expandi aerem, at frigore\protect\index{Sachverzeichnis}{frigus} aerem densari, \edtext{nondum}{\lemma{nondum}\Afootnote{\textbar~usque \textit{ gestr.}\ \textbar\ certum \textit{ L}}} certum est, nam et aquam non minus frigore\protect\index{Sachverzeichnis}{frigus}  quam calore\protect\index{Sachverzeichnis}{calor} ebullire experientias habemus. \edtext{Sed}{\lemma{habemus.}\Afootnote{ \textit{ (1) }\ Deinde \textit{ (2) }\ Etsi   \textbar\ vero \textit{ erg.}\ \textbar\  negari non possit aerem reddi per  calorem\protect\index{Sachverzeichnis}{calor|textit} specifice leviorem \textit{ (3) }\ Sed \textit{ L}}}  non est cur huc confugiamus, nam illud potius  verisimile est ad aeris judicia plurimum etiam caloris\protect\index{Sachverzeichnis}{calor} frigorisque\protect\index{Sachverzeichnis}{frigus} rationes pertinere: quare Gravitas specifica\protect\index{Sachverzeichnis}{gravitas!specifica} in se una collectas  habet differentias caeteras pene omnes, Thermometri\protect\index{Sachverzeichnis}{thermometrum}, Hygrometri\protect\index{Sachverzeichnis}{hygrometrum} et Barometri\protect\index{Sachverzeichnis}{barometrum} communis, quatenus  maxime vim habent ad aeris mutationes.  Et vero observandum est an non aliquid a caloris\protect\index{Sachverzeichnis}{calor}  excessu et Barometra\protect\index{Sachverzeichnis}{barometrum} communia sentiant.  Sed ut disquisitionem istam sane difficilem  finiam, \edtext{relinquamus}{\lemma{finiam,}\Afootnote{ \textit{ (1) }\ ponamus Barometrum\protect\index{Sachverzeichnis}{barometrum|textit} Torricelli\protect\index{Namensregister}{\textso{Torricelli} (Torricellius), Evangelista 1608\textendash 1647|textit}  \textit{(a)}\ plus ad \textit{(b)}\ magis esse \textit{ (2) }\ relinquamus \textit{ L}}} in medio  utra gravitas specifica\protect\index{Sachverzeichnis}{gravitas!specifica} an individualis\protect\index{Sachverzeichnis}{gravitas!individualis} majoris sit ad nostrum \edtext{institutum momenti}{\lemma{nostrum}\Afootnote{ \textit{ (1) }\ instrumenti \textit{ (2) }\ institutum momenti \textit{ L}}} imo si placet  individuali, ac proinde Barometro\protect\index{Sachverzeichnis}{barometrum}  communi  palmam concedamus: attamen \edtext{ut}{\lemma{attamen}\Afootnote{ \textit{ (1) }\ non nihil  negligenda nobis, \textit{ (2) }\ ut \textit{ L}}}  in posterum certius de tota re statuere possimus,  quod prohibet etiam de altera institui  observationes, cum enim diversae sunt  causae utriusque diversae utriusque mensurandi  rationes,\rule[-2cm]{0cm}{0.5cm} quid prohibet effectus quoque diversos,  conclusionesque differentes expectare, plurimum  ex collatione mutua lucis \edtext{\edlabel{alla106v1}allaturas.}{\lemma{allaturas.}\xxref{alla106v1}{alla106v2}
\Afootnote{\textbar\ Quod si appareret  \textit{ (1) }\ nihil differe \textit{ (2) }\ Gravitatem specificam   \textit{(a)}\ illa \textit{(b)}\ individuali perpetuo  \textit{(aa)}\ \protect\pgrk{sumpaj'izein}
%συμπαθιζειν\hspace{-18.5pt}´\hspace{18.5pt}
%\selectlanguage{polutonikogreek}συμπαθζειν
%ί$\iota$
 \textit{(bb)}\  conspirare, jam concludendum erit fortasse,  ipsum Mercurium\protect\index{Sachverzeichnis}{mercurius} etiam in Tubo Torricelliano\protect\index{Sachverzeichnis}{Tubus!Torricellianus}  non tam a pondere totius cylindri aerei  quam a densitate seu gradu Elaterii\protect\index{Sachverzeichnis}{elater} compressioni ulteriori  renitentis, pendere. \textit{ gestr.}\ \textbar\ Denique \textit{ L}}}% }}
\pend 
\pstart  Denique\edlabel{alla106v2}
 illud saltem manifestum  est, pondus cylindri  minus esse posse, in loco scilicet  altiore, eodem licet aeris in eo  loco densitate existente, ut  si tempus sit pluviosum, quae  alibi, in loco depressione, ubi  majus scilicet cylindri pondus, et Mercurii\protect\index{Sachverzeichnis}{mercurius} altitudo: at gravitatem\protect\index{Sachverzeichnis}{gravitas!specifica} aeris\protect\index{Sachverzeichnis}{gravitas!aeris} specificam, in diversae altitudinis  locis, eodem tempestatis statu eandem  in summa fore. Quare  in tali casu plus gravitati specificae\protect\index{Sachverzeichnis}{gravitas!specifica}  fidendum esse. Quodsi alii sunt casus,  ubi plus \edtext{tribuendum sit individuali}{\lemma{plus}\Afootnote{ \textit{ (1) }\ fidendum erit \textit{ (2) }\ tribuendum sit individuali \textit{ L}}} hoc saltem ad  summum \edtext{inde colligetur}{\lemma{}\Afootnote{inde colligetur \textit{ erg.} \textit{ L}}}; utriusque collatione  ad perficiendas Meteorologicas praedictiones opus esse.  
 \pend 