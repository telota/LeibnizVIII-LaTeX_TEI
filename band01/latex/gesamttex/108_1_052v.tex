[52 v\textsuperscript{o}] in Austris certo et infallibiliter nisi vibrata fuerint, se disponent, quod de lateribus quibusdam ferrugineis certo quoque \edtext{fieri mihi}{\lemma{quoque}\Afootnote{ \textit{ (1) }\ mihi \textit{ (2) }\ fieri mihi \textit{ L}}} constat. Et Ferrum quod longo tempore perpendicularem situm habuit inferiore parte boream superiore austrum petit, et pars inferior versorii partem australem, altera borealem trahit. Haec Kircherus\protect\index{Namensregister}{\textso{Kircher} (Kircherus), Athanasius SJ 1602\textendash 1680} ubi experimento dignum, an etiam habeant declinationem\protect\index{Sachverzeichnis}{declinatio}. Item experimentandum \edtext{si}{\lemma{experimentandum}\Afootnote{ \textit{ (1) }\ an \textit{ (2) }\ si \textit{ L}}} \edtext{terrellae meridianus}{\lemma{si}\Afootnote{ \textit{ (1) }\ ex \textit{ (2) }\ terrellae meridianus \textit{ L}}} juxta Grandamicum\protect\index{Namensregister}{\textso{Grandami} (Grandamicus), Jacques SJ 1588\textendash 1672} inveniatur, et caetera abundantur retento solo orbe meridiani\protect\index{Sachverzeichnis}{meridianus}, vel etiam solo ejus diametro in acus\protect\index{Sachverzeichnis}{acus} formam ruditer redacto, eadem maneat versus polos\protect\index{Sachverzeichnis}{polus} perfecta \edtext{directio}{\lemma{perfecta}\Afootnote{ \textit{ (1) }\ inclinatio\protect\index{Sachverzeichnis}{inclinatio|textit} \textit{ (2) }\ directio \textit{ L}}} ego puto omnino nullam fore, sed tamen res experimento digna si inclinaret, sublata esset omnis de \edtext{terrellae}{\lemma{}\Afootnote{terrellae \textit{ erg.} \textit{ L}}} libratione difficultas. Sed ut dixi non puto ob Analysin 12.\edtext{}{\lemma{Analysin 12.}\Bfootnote{\textsc{A. Kircher, }\cite{00067}a.a.O., S.~105.}} Kircheri\protect\index{Namensregister}{\textso{Kircher} (Kircherus), Athanasius SJ 1602\textendash 1680} d. l. quod partes polares in toto sint fortiores aequinoctialibus, sed non separatae. Sed haec forte de attractione intellexit, non de verticitate. Libratio in aqua non est constans, quia aqua tandem suber penetrante globus subsidit. Idem Kircherus\protect\index{Namensregister}{\textso{Kircher} (Kircherus), Athanasius SJ 1602\textendash 1680} haec habet: Ferrum oblongum \textit{AB} hactenus non magneticum si applicetur versorio librato in puncto \textit{A} \edtext{ex loco superiore rapiet}{\lemma{\textit{A}}\Afootnote{ \textit{ (1) }\ rapietur borea sic ut \textit{A} si \textit{ (2) }\ ex loco superiore \textit{(a)}\ trahet \textit{(b)}\ rapiet \textit{ L}}} partem Boream acus.\protect\index{Sachverzeichnis}{acus} Si vero applicetur eodem puncto a sursum verso trahet partem australem lib.1 parte 2. prop. 1. Experimento 1.\edtext{}{\lemma{Experimento 1.}\Bfootnote{\textsc{A. Kircher}, \cite{00067}a.a.O., S.~26f. }} ubi et rursus de verticitate instrumentorum igniariorum confirmat, aitque nihil esse eo genere tritius.\pend \pstart Cum non solum ferrum ad magnetem\protect\index{Sachverzeichnis}{magnes}, sed et magnes\protect\index{Sachverzeichnis}{magnes} ad ferrum rapiatur, satius est aenea fila esse, quibus tabula designatoria sustinetur. Libratio et fortasse fieri posset more Kircheriano\protect\index{Namensregister}{\textso{Kircher} (Kircherus), Athanasius SJ 1602\textendash 1680} in liquore homogeneo, ita plane liber penderet. Suber, si adhibetur lamina obducendum est, ne aquam imbibat et tandem subsidat. Posset forte constantius libratio in oleo fieri, quod ob tenacitatem non ita facile ut aqua agitaretur, sed dubito, an \edtext{}{\lemma{}\Afootnote{an \textbar\ non \textit{ gestr.}\ \textbar\ libera \textit{ L}}}libera satis hoc modo gyratio terrellae futura sit. Globis pendulis ita fortasse fieri possunt Horologia\protect\index{Sachverzeichnis}{horologium}, ut ipsa \edtext{numerent vibrationes}{\lemma{ipsa}\Afootnote{ \textit{ (1) }\ ictus \textit{ (2) }\ numerent vibrationes \textit{ L}}}, quod non venit in mentem Kirchero\protect\index{Namensregister}{\textso{Kircher} (Kircherus), Athanasius SJ 1602\textendash 1680}, et hac forsan arte suos Hugenius\protect\index{Namensregister}{\textso{Huygens} (Hugenius, Vgenius, Hugens, Huguens), Christiaan 1629\textendash 1695} construxit.\pend \pstart Fortasse etiam in navi sine difficultate et metu jactationis \edtext{poterit librari}{\lemma{jactationis}\Afootnote{ \textit{ (1) }\ retineri \textit{ (2) }\ poterit librari \textit{ L}}} terrella. Fiat cylinder, parum amplior subere terrellae. In eo libretur terrella. Cylinder sit aqua plenus ad summum[,] terrella igitur libretur in media aqua. Ne suber aquam imbibat obducatur vitro tenui. Magnes\protect\index{Sachverzeichnis}{magnes} in aquam non amittit sed servat vires. Sed superius in hemisphaerium cylinder fastigietur, \edtext{per foramen}{\lemma{fastigietur,}\Afootnote{ \textit{ (1) }\ cujus summum foramen \textit{(a)}\ introrsus sit \textit{(b)}\ extra sit ut infundat per foramen \textit{ (2) }\ foramini \textit{ (3) }\ per foramen \textit{ L}}} rotundum egrediens e terrella filum tabulam impactoriam sustinens ita perfecte accommodetur, ut vix quicquam \edtext{}{\lemma{}\Afootnote{quicquam \textbar\ in \textit{ gestr.}\ \textbar\ medium \textit{ L}}} medium exire aut intrare possit, sit tamen libera filo gyratio. Ea ratione non apparet quomodo cylinder jactata quantumcunque nave concutere aquam quam continet possit. Quia aqua in angusto non potest se circumrotare\edtext{, multo minus in pleno, et}{\lemma{circumrotare}\Afootnote{ \textit{ (1) }\ . Et q \textit{ (2) }\ , multo minus in pleno, et \textit{ L}}} cujus aer accedere nullus potest. Porro hic cylinder collocetur in media aqua, ne expiret aqua aere aut calore, et fiat \edtext{patens. Tota}{\lemma{patens.}\Afootnote{ \textit{ (1) }\ Suus \textit{ (2) }\ Tota \textit{ L}}} Machina filo alligetur, sed ita ut rigidum uno tantum modo firmatum sit pensile sine vibratione. \pend 