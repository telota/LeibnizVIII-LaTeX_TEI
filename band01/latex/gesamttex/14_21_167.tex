\pend \pstart [p.~167] IX. Vt autem puncta O et V pro 22. hora in tropico\protect\index{Sachverzeichnis}{tropicus} Cancri\protect\index{Sachverzeichnis}{Cancer}, habeo in analemmate signata, ita habeo puncta  pro aliis horis; itemque in tropico\protect\index{Sachverzeichnis}{tropicus} Capric.\protect\index{Sachverzeichnis}{Capricornus} ac proinde citra  vllum calculum, aut descriptionem hyperbolae, aut  operam acus magneticae\protect\index{Sachverzeichnis}{acus!magnetica}, cum praedicto analemmate,  signato quolibet puncto vmbrae, in quolibet plano verticali, horologium\protect\index{Sachverzeichnis}{horologium}, describi potest, Italicum quidem,  vt dictum est; Astronomicum vero, longe facilius, idque per solam applicationem horizontalis;\footnote{\textit{Leibniz unterstreicht}: ac proinde [...] horizontalis} sed de horologiis\protect\index{Sachverzeichnis}{horologium} in Gnomonica ex professo agam.\selectlanguage{latin}