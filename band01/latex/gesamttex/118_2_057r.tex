      
               
                \begin{ledgroupsized}[r]{120mm}
                \footnotesize 
                \pstart                
                \noindent\textbf{\"{U}berlieferung:}   
                \pend
                \end{ledgroupsized}
            
              
                            \begin{ledgroupsized}[r]{114mm}
                            \footnotesize 
                            \pstart \parindent -6mm
                            \makebox[6mm][l]{\textit{L}}Konzept: LH XXXV 15, 6 Bl. 57. Rechteckig beschnittenes Blatt, 10 x 20 cm. 1~1/4 S. Vorderseite ganz, R\"{u}ckseite zu 1/4 gegenl\"{a}ufig beschrieben.\\KK 1, Nr. 193 J \pend
                            \end{ledgroupsized}
                %\normalsize
                \vspace*{5mm}
                \begin{ledgroup}
                \footnotesize 
                \pstart
            \noindent\footnotesize{\textbf{Datierungsgr\"{u}nde}: Am Ende des Textes bezieht sich Leibniz auf eine Stelle des 4. Bandes der \cite{00044}\textit{Physik} von Fabri. Wie aus \textit{LSB} I, 2 N. 436 hervorgeht, hat Leibniz die komplette Ausgabe der \cite{00044}\textit{Physik} am 4. Januar 1672 erhalten. Da es sich bei dem Textzeugen um Papier aus der Zeit vor Leibniz' Parisaufenthalt handelt, gehen wir von einer Entstehungszeit des St\"{u}ckes Anfang 1672 aus.}
                \pend
                \end{ledgroup}
            
                \vspace*{8mm}
                \pstart 
                \normalsize
            [57 r\textsuperscript{o}] An non effici potest ope Tuborum, ut liceat semper videre ubi sit \edtext{sol\protect\index{Sachverzeichnis}{sol}, etiam}{\lemma{sol,}\Afootnote{ \textit{ (1) }\ vel luna\protect\index{Sachverzeichnis}{luna|textit} aut dua stella\protect\index{Sachverzeichnis}{stella|textit} etiam \textit{ (2) }\ etiam \textit{ L}}} die pluviosissimo, et semper ejus umbram habere sic posse, et ita perpetua horologia\protect\index{Sachverzeichnis}{horologium} solis\protect\index{Sachverzeichnis}{sol}. Item \edtext{an liceat interdiu}{\lemma{Item}\Afootnote{ \textit{ (1) }\ noctu \textit{ (2) }\ an liceat interdiu \textit{ L}}} videre stellas\protect\index{Sachverzeichnis}{stella} item an liceat hoc applicare ad longitudines\protect\index{Sachverzeichnis}{longitudo} ita ut sol\protect\index{Sachverzeichnis}{sol} ea ratione semper aliquid secum rotet. Hugenii\protect\index{Namensregister}{\textso{Huygens} (Hugenius, Vgenius, Hugens, Huguens), Christiaan 1629\textendash 1695} invento deesse $\langle$dei$\rangle$ quod non retinet se in perpendiculo, jactata nave\protect\index{Sachverzeichnis}{navis}. An non hoc efficere licebit ope tensionis se restituentis, quippe quae tam fortis ut \edtext{jactationi}{\lemma{ut}\Afootnote{ \textit{ (1) }\ navem\protect\index{Sachverzeichnis}{navis|textit} \textit{ (2) }\ jactationi \textit{ L}}} resistat, et NB utcunque jactetur nihilominus aget semper eodem modo in suum globulum\protect\index{Sachverzeichnis}{globulus}. Fateor tamen nondum ne mihi satisfacere, et etsi detur globus qui se ipse in aere circumvertat, tamen \edtext{nullo}{\lemma{tamen}\Afootnote{ \textit{ (1) }\ non si \textit{ (2) }\ nullo \textit{ L}}} modo alligatus ad navem\protect\index{Sachverzeichnis}{navis} quomodo eam sequetur. Et etsi superius teneatur a magnete\protect\index{Sachverzeichnis}{magnes}, tamen magnete\protect\index{Sachverzeichnis}{magnes} jactato jactabitur. Gyrus facit rem recta semper tendere sursum, sane gravitas\protect\index{Sachverzeichnis}{gravitas} deorsum, sed quomodo resistet impressae jactationi. Illud optimum remedium videtur si res natet in aqua modo Kircheriano qua plenum vas sigillatum,\edtext{}{\lemma{sigillatum,}\Bfootnote{\textsc{A. Kircher, }\cite{00067}\textit{Magnes}, Rom 1654, S.~310. }} ita repletum ut nulla concussione turbetur aqua, quia nullus in ea locus vacuus ad sensum, et in ea sit horologium\protect\index{Sachverzeichnis}{horologium}, sed ita nil effecerit tensio\protect\index{Sachverzeichnis}{tensio}. Optime et hoc forte erit, si res pendeat a magnete\protect\index{Sachverzeichnis}{magnes} ita tamen ut eum non tangat, ita enim etsi cum eo moveatur tamen non recipiet ab eo impressiones crassas. Sed quod hoc efficiemus. Res sane bene natabit in lampadi-\pend\pstart\noindent bus quae non effunduntur sed nescio an non contingat error turbans succussionibus aliquot minuta, et porro, quod deinde grandem parit confusionem. Si numerabiles essent vibrationes tensionum\protect\index{Sachverzeichnis}{tensio} quia et ipsae aequidiuturnae, res forte esset effectu facilior. Nam hoc demonstravit Honoratus Fabri\protect\index{Namensregister}{\textso{Fabri} (Hon. Fab.), Honor\'{e} 1607\textendash 1688}\edtext{}{\lemma{demonstravit}\Bfootnote{\textsc{H. Fabri, }\cite{00044}\textit{Physica}, Bd. 4, Lyon 1671, S.~44. }}. Sed vereor ne hae vibrationes sint nimis subtiles quam ut sint numerabiles, et vidibiles. An ita res instituenda, ut tensione\protect\index{Sachverzeichnis}{tensio} horologiolorum applicata ad horologium pendulum\protect\index{Sachverzeichnis}{horologium!pendulum}, statim resistat tendens si minimum a motu perpendiculari abeat.
            %\footnote{\textit{Am oberen Rand von Bl. 57 v\textsuperscript{o}}: NB. Hic modus forte optimus \Denarius \hspace{3pt}NB.} 