\pend \pstart [p.~85] XIV. Hinc quaedam ludicra caui speculi\protect\index{Sachverzeichnis}{speculum} opera praestari queunt nam 1. si penna ita applicetur in axe speculi\protect\index{Sachverzeichnis}{axis!speculi}, vt rostrum speculi\protect\index{Sachverzeichnis}{speculum} centro adhaereat, videtur ad instar arboris, ab oculo\protect\index{Sachverzeichnis}{oculus}, in eodem axe collocato; ratio patet ex dictis. 2. manus apposita manum quasi proiectam extra speculum\protect\index{Sachverzeichnis}{speculum} stringere videtur. 3. Duo item quasi rudibus ludere videntur, micatque gladius extra speculum\protect\index{Sachverzeichnis}{speculum}. 4. eadem arte in scena multa ludicra representari possunt, sed amplissimum speculum\protect\index{Sachverzeichnis}{speculum} esse oportet; quae omnia ex praemissis demonstratis facile petuntur.\footnote{\textit{Am Rand angestrichen:} Hinc quaedam [...] petuntur.}