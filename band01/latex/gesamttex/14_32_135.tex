\pend \pstart [p.~135] VII. Maior item campus, vt vocant, obiecti apparet,\footnote{\textit{Leibniz unterstreicht}: Maior item [...] apparet} cum longe plures radij laterales, id est ab extremitatibus obiecti profecti, qui alioquin post decussationem  in foco\protect\index{Sachverzeichnis}{focus}, in lentem\protect\index{Sachverzeichnis}{lens} vtrimque conuexam, vel nullo modo,  vt dixi, vel obliquius iusto illaberentur, in nouae lentis\protect\index{Sachverzeichnis}{lens} planum incidant, idque subduplo inclinationis\protect\index{Sachverzeichnis}{inclinatio} angulo; ex quo certe campus amplificatur: praeterea maiorem obiectiui\protect\index{Sachverzeichnis}{objectivum} aperturam sustinet; quia scilicet radios\protect\index{Sachverzeichnis}{radius}  etiam obliquius in obiectiuum\protect\index{Sachverzeichnis}{objectivum} illapsos post decussationem in foco\protect\index{Sachverzeichnis}{focus} ad minus obliquum inclinationis angulum\protect\index{Sachverzeichnis}{angulus!inclinationis}  reducit; vnde, quod ex eo timendum erat, non sequitur refractionum\protect\index{Sachverzeichnis}{refractio} confusio: Deinde hinc etiam concludo,  obiectum illustrius exhiberi, ex hoc saltem capite; quia  plures a singulis obiecti punctis radij excipiuntur\footnote{\textit{Leibniz unterstreicht}: obiectum illustrius [...] excipiuntur}, ob  maiorem scilicet aperturam; ex hoc saltem, inquam, cum  [p.~136] ex alio capite obscurius euadat, ex eo scilicet, quod  maius appareat; sic enim radij valde distrahuntur.\selectlanguage{latin}