[73 r\textsuperscript{o}] terrae definiendas accuratius quam hactenus habemus, insigni usui esse possint.\pend \pstart Sed ut ad rem nostram redeamus dicet aliquis, aetate Lunae\protect\index{Sachverzeichnis}{luna} definita (aut periodo siderum circumjovialium) definietur non ideo tempus praesens mundi, quia ignoramus quoto mense aut \edtext{quota }{\lemma{}\Afootnote{quota  \textit{ erg.} \textit{ L}}}periodo nunc a patria seu loco cognitae longitudinis\protect\index{Sachverzeichnis}{longitudo} absimus, sciemus ergo aetatem mensis (aut periodi) sed non cujus mensis aut cujus periodi. Sed facilis responsio est non posse Nautas in tempore suo etiam sine omnibus horologiis\protect\index{Sachverzeichnis}{horologium} numerando, ita aberrare ut \edtext{pluribus diebus integris}{\lemma{ut}\Afootnote{ \textit{ (1) }\ una die integra \textit{ (2) }\ pluribus diebus integris \textit{ L}}}, nedum ut mense toto (idem est de circumjoviali periodo, etsi mense minori) fallantur; nec numeratio tantum dierum et noctium, sed et qualecunque Horologium\protect\index{Sachverzeichnis}{horologium} eos ab hoc periculo expediet.\footnote{NB. Malum in eo est, quod nondum restitutus est motus lunaris, fatente ipso Bullialdo\protect\index{Namensregister}{\textso{Boulliau} (Bullialdus), Ismael 1605\textendash 1694}. Sed errari potest tertia horae parte, seu 72\textsuperscript{ma} circuli @@@ G R A F I K @@@ id est gradibus 5. Ac nescio an aliorum planetarum motus sit praecise exploratus. NB. autem fortasse tertia horae pars, seu error motus lunaris intelligi debet non in circulo diurno, sed in Zodiaco\protect\index{Sachverzeichnis}{zodiacus}, quem\rightmoon absolvit mense, id est diebus circiter 30. seu horis 720. Ergo error tabularum est 3 ◠ 720\textsuperscript{ma}\edtext{seu 2160\textsuperscript{ma} pars circuli}{\lemma{?LEMMA?:potest.}\Afootnote{ \textit{ (1) }\ pars circuli seu \textit{ (2) }\ seu 2160@@@SUPSUBON@@@ma@@@SUPSUBOFF@@@ pars circuli \textit{ L}}}Zodiaci\protect\index{Sachverzeichnis}{zodiacus}. Cumque Zodiacus\protect\index{Sachverzeichnis}{zodiacus}inclinatus sit ad aequatorem\protect\index{Sachverzeichnis}{aequator} et parallelos\protect\index{Sachverzeichnis}{circulus parallelus}, error in illis adhuc erit minor. Ergo vix futurus esset error 2 miliarium Germanicorum.An fortasse ratio erroris, in determinando futuro Lunae\protect\index{Sachverzeichnis}{luna} loco provenit ex variante distantia\rightmoon\textsuperscript{nae} a terra unde varia ejus parallaxis\protect\index{Sachverzeichnis}{parallaxis}. Distantiae ergo\rightmoon\textsuperscript{naris} a terra notitia nobis opus est, sine qua nec parallaxes\protect\index{Sachverzeichnis}{parallaxis}haberi possunt.}\pend \pstart Reperto ergo tandem tempore Mundi, longitudines\protect\index{Sachverzeichnis}{longitudo} loci habemus in potestate, secundum methodum \textso{problematis prioris,} perinde scilicet ac si Horologium\protect\index{Sachverzeichnis}{horologium} exactum haberemus, Meridiano\protect\index{Sachverzeichnis}{meridianus} \edtext{scilicet}{\lemma{?LEMMA?:Meridiano}\Afootnote{ \textit{ (1) }\ scil. \textit{ (2) }\ scilicet \textit{ L}}} primo sphaerae artificialis \edtext{supra descripto}{\lemma{}\Afootnote{supra descripto \textit{ erg.} \textit{ L}}} sideri\protect\index{Sachverzeichnis}{sidus} cuidam, cujus cognita nobis ab eo praesens in Mundo distantia est, debite admoto: ita differentia Meridiani\protect\index{Sachverzeichnis}{meridianus} loci et Meridiani\protect\index{Sachverzeichnis}{meridianus} primi in aequatore\protect\index{Sachverzeichnis}{aequator} numeranda, seu gradus longitudinis\protect\index{Sachverzeichnis}{longitudo} apparebit. Quodsi diversae aut simul, aut diversis temporibus ex cognitae distantiae locis observationes instituantur ac inter se comparentur, quales occasiones utique subinde offerentur, sic satis secura reddetur navigatio, quantum ab observationibus coelestibus sperari potest.\pend 