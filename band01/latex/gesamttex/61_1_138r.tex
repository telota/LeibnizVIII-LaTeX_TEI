      
               
                \begin{ledgroupsized}[r]{120mm}
                \footnotesize 
                \pstart                
                \noindent\textbf{\"{U}berlieferung:}   
                \pend
                \end{ledgroupsized}
            
              
                            \begin{ledgroupsized}[r]{114mm}
                            \footnotesize 
                            \pstart \parindent -6mm
                            \makebox[6mm][l]{\textit{L}}Konzept: LH XXXVIII Bl. 138, 1 Bl. 17 x 18 cm, gleichm\"{a}ssig beschnitten. 1 S., R\"{u}ckseite leer. Datierung in der linken oberen Ecke. Zeichnung und Nebenrechnung am linken Rand.\\Cc 2, Nr. 966 C \pend
                            \end{ledgroupsized}
                \vspace*{8mm}
                \pstart 
                \normalsize
            [138 r\textsuperscript{o}] \selectlanguage{french}Maji \begin{center}1675 Probleme:\end{center} \pend \pstart \textso{Un bâton estant fich\'{e} dans le fonds d'un foss\'{e} plein d'eau, et sortant  tant soit peu hors de l'eau juger de la profondeur de l'eau ou du foss\'{e} sans  tirer le bâton et sans en s\c{c}avoir la longueur, pourveu qu'on aye la libert\'{e} }\textso{ de le remuer.}\pend \pstart \textso{Ou si vous voulez une sonde estant jett\'{e}e }\textso{ dans la mer et touchant fonds, juger de la profondeur de la mer,  sans retirer la sonde, et sans s\c{c}avoir la longueur de la corde.}\edtext{}{\lemma{corde.}\Bfootnote{Markierung durch Anf\"{u}hrungszeichen am linken Rand.}} \pend \pstart  Soit le bâton ou la sonde, \textit{AB} touchant fonds au bout \textit{B} @@@ G R A F I K @@@% \begin{wrapfigure}{l}{0.4\textwidth}                    
                %\includegraphics[width=0.4\textwidth]{../images/D%26eacute%3Btermination+de+la+profondeur+de+l%27eau.+Deuxi%26egrave%3Bme+tentative/LH038_138r/files/100071.png}
                        %\caption{Bildbeschreibung}
                        %\end{wrapfigure}
                        %@ @ @ Dies ist eine Abstandszeile - fuer den Fall, dass mehrere figures hintereinander kommen, ohne dass dazwischen laengerer Text steht. Dies kann zu einer Fahlermeldung fuehren. @ @ @ \\
                     sortant de  l'eau \textit{DE}, d'une partie \textit{\textso{AC}}\textso{ dont la longueur nous est connue  par exemple 1. pouce} lorsque le baston  (ou la corde) est perpendiculaire \`{a} l'horison ou au niveau de l'eau \textit{DE}. \textit{A} present puisque nous avons la libert\'{e} de remuer, laissons le bout \textit{B}  immobile, et \edtext{remuons le bout \textit{A}}{\lemma{et}\Afootnote{ \textit{ (1) }\ trouuons le bout \textit{A} de la \textit{ (2) }\ remuons le bout \textit{A} \textit{ L}}} jusqu'\`{a} ce qu'il \edtext{ entre dans}{\lemma{qu'il}\Afootnote{ \textit{ (1) }\ touche le \textit{ (2) }\  entre dans \textit{ L}}} l'eau, en \textit{D}. Car si \textit{AB} est un bâton, il est ais\'{e} de le remuer  sans le tirer du point \textit{B}, o\`{u} il est fich\'{e} dans le fonds, et \edtext{si la  ligne \textit{AB} est}{\lemma{et}\Afootnote{ \textit{ (1) }\ s'il est \textit{ (2) }\ si la  ligne \textit{AB} est \textit{ L}}} une corde, et si \textit{B} est un plomb qui touche fonds, sa pesanteur  l'y tiendra quoyqu'on remue le point \textit{A} de la corde. Mesurons  \`{a} present la distance \edtext{\textit{DC} entre le point \textit{D} o\`{u} le bâton entre  dans l'eau \`{a} present, et le point}{\lemma{distance}\Afootnote{ \textit{ (1) }\ du point \textit{D} o\`{u} le bâton \textit{(a)}\ touche l'eau \textit{(b)}\ entre  dans l'eau \`{a} present, du point \textit{ (2) }\ \textit{DC} [...] et  \textbar\ entre \textit{ gestr.}\ \textbar\ le point \textit{ L}}} \textit{C} ou il entroit auparavant, ce qui  est ais\'{e}, en tenant une regle d'une main, \`{a} fleur d'eau, pendant qu'on  remue le bâton de l'autre: Et supposons \textso{par exemple  la longueur de }\textit{\textso{DC}}\textso{, 9. pouces.} \pend \pstart  \begin{center}\textso{Regle:}\end{center}\edtext{}{\lemma{\begin{center}Regle:\end{center}}\Afootnote{\textit{doppelt unterstrichen}}} \pend \pstart  Multipliez le nombre des pouces de la ligne \textit{AC} s\c{c}avoir 1. par soy même,  et vous aurez 1. Multipliez aussi le nombre des pouces de la ligne \textit{CD}  s\c{c}avoir 9. par soy même, et vous aurez 81. Adjoutez ces deux  produits ensemble, et vous aurez 82. Divisez cette somme par  le double du nombre des pouces de la ligne \textit{AC}, s\c{c}avoir par 2.  Et le quotient qui proviendra en divisant 82. par 2. sera 41. Ostez  de ce quotient le nombre des pouces de la ligne \textit{AC}, s\c{c}avoir 1. et il vous  restera 40. @@@ G R A F I K @@@% \begin{wrapfigure}{l}{0.4\textwidth}                    
                %\includegraphics[width=0.4\textwidth]{../images/D%26eacute%3Btermination+de+la+profondeur+de+l%27eau.+Deuxi%26egrave%3Bme+tentative/LH038_138r/files/100223.png}
                        %\caption{Bildbeschreibung}
                        %\end{wrapfigure}
                        %@ @ @ Dies ist eine Abstandszeile - fuer den Fall, dass mehrere figures hintereinander kommen, ohne dass dazwischen laengerer Text steht. Dies kann zu einer Fahlermeldung fuehren. @ @ @ \\
                     qui est le nombre des pouces de la ligne \textit{BC} ou la profondeur de l'eau. La  demonstration de cette regle se peut donner ais\'{e}ment  par l'Analyse.\selectlanguage{latin}\pend 