      
               
                \begin{ledgroupsized}[r]{120mm}
                \footnotesize 
                \pstart                
                \noindent\textbf{\"{U}berlieferung:}   
                \pend
                \end{ledgroupsized}
            
              
                            \begin{ledgroupsized}[r]{114mm}
                            \footnotesize 
                            \pstart \parindent -6mm
                            \makebox[6mm][l]{\textit{L}}Konzept: LH XXXVII 4 Bl. 71, 1 Bl. 2\textsuperscript{o}. 1 1/3 S. In der linken oberen Ecke von Bl. 71 r\textsuperscript{o} etwa 10 x 12 cm des Papiers ausgeschnitten. Am unteren Rand Papierabbruch mit Textverlusten. Die untere H\"{a}lfte von Bl. 71 v\textsuperscript{o} etwa zu 1/3 beschrieben.\\Cc 2, Nr. 28 \pend
                            \end{ledgroupsized}
                %\normalsize
                \vspace*{5mm}
                \begin{ledgroup}
                \footnotesize 
                \pstart\selectlanguage{ngerman}
            \noindent\footnotesize{\textbf{Datierungsgr\"{u}nde}: Das Wasserzeichen des Papiers weist in die Zeit des Beginns der Auseinandersetzung mit Vakuumph\"{a}nomenen in Paris. Dasselbe Wasserzeichen befindet sich auf den Texttr\"{a}gern von N. 6, N. 41 und N. 47, die sich alle auf die 2. H\"{a}lfte 1672 datieren lassen. Auch inhaltlich passt sich das St\"{u}ck gut in die Thematik der Pneumatica dieser Zeit ein.}\selectlanguage{latin}
                \pend
                \end{ledgroup}
            
                \vspace*{8mm}
                \pstart 
                \normalsize
            [71 r\textsuperscript{o}] Novum elegansque experimentum hoc institui potest, constat duas chordas \edtext{similiter tensas}{\lemma{chordas}\Afootnote{ \textit{ (1) }\ unam alia tensa, \textit{ (2) }\ tendi \textit{ (3) }\ similiter tensas \textit{ L}}}  ita comparatas esse, ut una tacta altera etiam non tacta resonet. Id facile experiemur an ab aere pendeat an non, si altera tacta, altera similiter si non resonet, id enim audire non poterimus, saltem tremat.\pend \pstart Accendantur corpora combustibilia in vacuo, speculo ardente, videndum an intus durare diu ignis possit, item an sit locum repleturus fumo seu aere ita ut aer externus tantum postea irrumpendi conatum\protect\index{Sachverzeichnis}{conatus} non habeat. Hoc experimento discemus naturam effluviorum rerum combustibilium, et an ea recolligi possint in aliam materiam combustibilem utique cum pura hic sint seu aeri non confundantur.
            \pend 
            \pstart Experiendum quae figurae quibus gradibus facilius rumpantur, item quae crassities, et cujus materiae. Inde rationes duci poterunt de corporum soliditate.\footnote{\textit{Unterhalb des Ausschnitts}: Experiendum qui liquores\protect\index{Sachverzeichnis}{liquor!purgatus} aere purgandi plus dent bullarum vel diutius.}
            \pend 
            \pstart  An corpora in vacuo putrescant, saltem per partes interiores. Et an vermes in vacuo nascantur. Spiritus vini\protect\index{Sachverzeichnis}{spiritus!vini} \edtext{vitrum}{\lemma{vini}\Afootnote{ \textit{ (1) }\ vas \textit{ (2) }\ vitrum \textit{ L}}} quoddam replens, solo manuum calore rupturus ut mihi refert dominus Dalancay\protect\index{Namensregister}{\textso{Dalanc\'{e}} (Dalancay), Joachim 1640\textendash 1707} dissiluitque in mille fragmina vitrum. Erat l'instrument pour niveller. Ex eo tempore Hubin\protect\index{Namensregister}{\textso{Hubin,} Ludion de } adhibet aquam secundam.
            \pend 
            \pstart Experimenta in Vacuo facienda: Thermometrum\protect\index{Sachverzeichnis}{thermometrum} in eo suspendendum, item Hygrometrum\protect\index{Sachverzeichnis}{hygrometrum}, ut appareat an siccitas et humiditas per ipsum vitri corpus sentiantur, ac per consequens in solis vaporibus corporeis non insint. An in Vacuo rumpatur lacryma vitri, et si non rumpitur an statim aperto vase rumpatur. Si rumpitur in ipso Vacuo, videndum an rumpatur in compresso an statim compresso vase rumpatur.
            \pend 
            \pstart  Videndum an in vacuo liquores altius ascendant in Tubum exiguum, et per consequens an contribuat aeris compressio\protect\index{Sachverzeichnis}{compressio}. Item an in vacuo quoque medium vasis aqua pleni sit altius extremo, item alia a Rohaultio\protect\index{Namensregister}{\textso{Rohault} (Rohaultius), Jacques 1620\textendash 1675} \edtext{dicta.}{\lemma{dicta.}\Bfootnote{\textsc{J. Rohault}, \cite{00087}\textit{Traité de physique}, Teil 1, Paris 1671, S. 246, 254.}} An in vacuo exerceatur succini vis Electrica, qua corpora retinet. An alia sit refractio\protect\index{Sachverzeichnis}{refractio} radii in vacuo quam in pleno, et an refringatur radius magis a perpendiculari, quam \edtext{ante. Si  vis Electrica exercetur in vacuo}{\lemma{ante.}\Afootnote{ \textit{ (1) }\ An in vacuo \textit{ (2) }\ Si [...] vacuo \textit{ L}}}, falsum erit quod dicebat Baconus\protect\index{Namensregister}{\textso{Bacon} (Baconus), Francis 1561\textendash 1626}  causam vis Electricae esse, quod corpora facilius corpori solidiori\protect\index{Sachverzeichnis}{corpus!solidum} quam aeri \edtext{adhaereant,}{\lemma{adhaerent,}\Bfootnote{\textsc{F. Bacon}, \cite{00270}\textit{Novum organum}, London 1620, S. 316.}} sin contra: verum\edtext{, in \textit{Instauratione Magna seu novo organo}, ubi et addit}{\lemma{verum}\Afootnote{ \textit{ (1) }\ . Adde Baconi\protect\index{Namensregister}{\textso{Bacon} (Baconus), Francis 1561\textendash 1626|textit} dictum \textit{ (2) }\ , in \textit{Instauratione Magna seu novo organo}, ubi et addit \textit{ L}}} \edtext{ideo bracteolas tenuissimas}{\lemma{ideo}\Afootnote{ \textit{ (1) }\ laminam tenuissimam \textit{ (2) }\ bracteolas tenuissimas \textit{ L}}} aureas digito quasi subito \edtext{adhaerere,}{\lemma{adhaerere,}\Bfootnote{\textsc{F. Bacon}, \cite{00270}a.a.O., S.~315.}} quod aerem magis fugiant. Experiendum an Gerickianus\protect\index{Namensregister}{\textso{Guericke} (Gerickius, Gerick.), Otto v. 1602\textendash 1686} Globus \edtext{sulphureus}{\lemma{sulphureus}\Bfootnote{\textsc{O. v. Guericke}, \cite{00055}\textit{Experimenta nova}, Amsterdam 1672, S. 147\textendash150.}} exerceri possit in vacuo.
            \pend 
% Zeitz auskommentiert           \pstart Experiendum an pulvis pyrius tanta vi displodatur in vacuo, quam extra ope speculi ardentis intus accendentis. Si tanta vi exploditur, non erit ejus vis ab aere exhausto nisi ⟨valde⟩ sit. Tentandum postea an et in compresso aere. Videndum et ⟨an⟩ ⟨tanta⟩ $\langle$---$\rangle$ \edlabel{displo1}⟨displosis⟩\edtext{}{\lemma{⟨displosis⟩}\xxref{displo1}{displo2}\Afootnote{ \textit{ (1) }\ An disploso plurimo pulvere musquetario per partes materia tandem vitrum sit impletura. \textit{ (2) }\  An [...] impletura. \textit{ L}}}
%            \pend