 \pstart [97 v\textsuperscript{o}] \textso{Gerick. lib. 3. cap. 15 }\edtext{}{\lemma{\textso{15}}\Bfootnote{\textsc{O. v. Guericke, }\cite{00055}a.a.O., S.~91f.}} \textit{De Sono}\protect\index{Sachverzeichnis}{sonus}\textit{ in vacuo.} Nullus est  in vacuo Tinnitus, est tamen strepitus, et strepitus  nil differt a communi. Suspenso enim in eo Horologio\protect\index{Sachverzeichnis}{horologium} sonante, \edtext{ita}{\lemma{sonante,}\Afootnote{ \textit{ (1) }\ quod \textit{ (2) }\ ita \textit{ L}}} composito ut malleoli pulsu  in tintinnabulo sonum\protect\index{Sachverzeichnis}{sonus} certis distinctum intervallis  per semihoram ederet; \edtext{extrahendo evanuit  tinnitus}{\lemma{ederet;}\Afootnote{ \textit{ (1) }\ sonus\protect\index{Sachverzeichnis}{sonus|textit} \textit{ (2) }\ extrahendo evanuit  tinnitus \textit{ L}}}, sed aure vitro admota, auditus tantum  est strepitus obtusus de campanula malleoli pulsu  ortus, velut si quis NB. campanulam manu tangat, et ita pulset. Redito aere auditus est tinnitus. Ita alterius pistilli in crotalum sonus\protect\index{Sachverzeichnis}{sonus}, seu  ictus ita auditus est versato vase, ut non differret  ab eo qui auditur in recipiente pleno. Strepitus  ergo a virtute sonante, tinnitus ab ipso aere.  (+ Explorandum an in vacuo augeatur sonus\protect\index{Sachverzeichnis}{sonus}  per Tubum Morlandi\protect\index{Sachverzeichnis}{Tubus!Morlandi} exiguo sumto. +) Caeterum  est quidam fragor, qui ab ipso aere efficitur,  ut cum Lagenae quadratae franguntur, et fragor vehemens inde causatur, aere circumstante in spatium  illud vacuum cum impetu confluente. Unde sonus\protect\index{Sachverzeichnis}{sonus}  ex ipso aerum concursu quae causa fragoris tonitruum et bombardarum. Nam quando ignis rapidissimus aerem celerrime dilatat et puncto  temporis extinguitur, spatium relictum aer vehementi concursu replet. Alioquin certissime sciendum  (\Denarius ) si horologium\protect\index{Sachverzeichnis}{horologium} sonans vitreo seu cupreo globo  inclusum sursus in recipiente suspendatur et  aer eliciatur ex recipiente, tinnitum, tamen  aeris in globo illo minore inclusi extra auditum  iri. Sed se experimentum facere noluisse, quia  nullum sit dubium globum esse rumpendum. (+ Ego  puto posse esse satis firmum ut si sit cupreus  non facile \edtext{[rumpatur].}{\lemma{rumpetur}\Afootnote{\textit{L ändert Hrsg.}}} Et dubito an ad  nos perventurus sit tinnitus. Si pervenit  demonstratur aerem ne tinnitus quidem vehiculum esse. +)\pend
 \pstart \textso{Cap. 16.}\edtext{}{\lemma{\textso{16.}}\Bfootnote{\textsc{O. v. Guericke}, \cite{00055}a.a.O., S.~93.}} Si uvae per dimidium annum  serventur in recipiente exhausto, durat illis  species, sed evanescit omnis sapor, quem attraxit recipiens. (+ Experiendum an aqua immissa id quod exhalavit capi possit, aut alio modo colligi in spiritum, quodam quasi distillationis genere. +)\pend \pstart \textso{Cap. 18.}\edtext{}{\lemma{\textso{18.}}\Bfootnote{\textsc{O. v. Guericke}, \cite{00055}a.a.O., S.~95.}} Potest ope exhausti  aeris sibilus diu durans gratusque, erumpentis vel irrumpentis in locum vacuum aut ex prae-pleno, \edtext{ exhiberi.}{\lemma{prae-pleno,}\Afootnote{ \textit{ (1) }\ construi \textit{ (2) }\  exhiberi. \textit{ L}}}\pend \pstart \textso{Cap. 19 }\edtext{}{\lemma{\textso{19}}\Bfootnote{\textsc{O. v. Guericke}, \cite{00055}a.a.O., S.~97.}} Aqua in vas exhaustum  non ascendit nisi ad altitudinem 19 ulnarum  Magedeburgicarum, et tantum scilicet ponderat sphaera aeris\protect\index{Sachverzeichnis}{sphaera!aeris} globo nostro circumfusa.\pend \pstart \textso{Cap. 20.}\edtext{}{\lemma{\textso{20.}}\Bfootnote{\textsc{O. v. Guericke}, \cite{00055}a.a.O., S.~100.}} Ad aeris gravitatem\protect\index{Sachverzeichnis}{gravitas!aeris}  quovis tempore deprehendendam exigua e ligno  virunculi specie efficta statua digito ostendit  certa puncta. Pro aeris gravitate\protect\index{Sachverzeichnis}{gravitas!aeris}, Artificium in  inferiori vitri parte non apparet, superior detecta est (+~in imagine sed non reapse apud Dn. Dalanc\'{e}\protect\index{Namensregister}{\textso{Dalanc\'{e}} (Dalancay), Joachim 1640\textendash 1707} +). Gustum rei hunc dat: Si globus evacuatus suspenderetur, globum dum aer gravior evadat  reddi leviorem (+ sed hoc fit et calore frigore +)  non est Thermoscopium quod calore quoque et frigore  alteretur. Si apertum sit Epistomium\protect\index{Sachverzeichnis}{epistomium} vasis pro calore et frigore variat pondus. Nam frigore gravius  est, plus enim aeris est in ipso, calore levius.\pend \pstart \textso{Gerick. lib. 3. cap. 21.}\edtext{}{\lemma{\textso{21.}}\Bfootnote{\textsc{O. v. Guericke}, \cite{00055}a.a.O., S.~101.}} \edtext{Non posse determinari aeris}{\lemma{\textso{21.}}\Afootnote{ \textit{ (1) }\ Aerem non posse ponderari \textit{ (2) }\ Non posse determinari aeris \textit{ L}}} proportionem specificam ad aquam  ob differentem ejus compressionem. Imo  potest fortasse designato loco et tempore, vel  rectius sic potest: aer qui tantum Elaterii\protect\index{Sachverzeichnis}{elater} seu  virium compressus habet, tantum ponderat (+).\pend 
\rule[0mm]{0mm}{3mm}
\pstart \textso{Cap. 22.}\edtext{}{\lemma{\textso{22.}}\Bfootnote{\textsc{O. v. Guericke}, \cite{00055}a.a.O., S.~102.}} Mensura Magdeburgica  habet in altitudine $\displaystyle \frac{38}{100}\rule[-4mm]{0mm}{10mm}\hspace{5.5pt}$% \begin{wrapfigure}{l}{0.4\textwidth}                    
                %\includegraphics[width=0.4\textwidth]{../images/Aus+Otto+von+Guericke%2C+Experimenta+nova/LH035%2C14%2C02_097v/files/100334.png}
                        %\caption{Bildbeschreibung}
                        %\end{wrapfigure}
                        %@ @ @ Dies ist eine Abstandszeile - fuer den Fall, dass mehrere figures hintereinander kommen, ohne dass dazwischen laengerer Text steht. Dies kann zu einer Fahlermeldung fuehren. @ @ @ \\
                     in diametro $\displaystyle\frac{19}{100}\rule[-4mm]{0mm}{10mm}\hspace{5.5pt}$% \begin{wrapfigure}{l}{0.4\textwidth}                    
                %\includegraphics[width=0.4\textwidth]{../images/Aus+Otto+von+Guericke%2C+Experimenta+nova/LH035%2C14%2C02_097v/files/100336.png}
                        %\caption{Bildbeschreibung}
                        %\end{wrapfigure}
                        %@ @ @ Dies ist eine Abstandszeile - fuer den Fall, dass mehrere figures hintereinander kommen, ohne dass dazwischen laengerer Text steht. Dies kann zu einer Fahlermeldung fuehren. @ @ @ \\
                     ulnae  Magdeburgicae,  aqua quam capit ponderat $\displaystyle4\frac{1}{8}$ % \begin{wrapfigure}{l}{0.4\textwidth}                    
                %\includegraphics[width=0.4\textwidth]{../images/Aus+Otto+von+Guericke%2C+Experimenta+nova/LH035%2C14%2C02_097v/files/100343.png}
                        %\caption{Bildbeschreibung}
                        %\end{wrapfigure}
                        %@ @ @ Dies ist eine Abstandszeile - fuer den Fall, dass mehrere figures hintereinander kommen, ohne dass dazwischen laengerer Text steht. Dies kann zu einer Fahlermeldung fuehren. @ @ @ \\
                     libras seu 2\textsuperscript{dum} numeros decimales 4, 125 (3)  et libra ponderat 16 imperiales.\pend \pstart \textso{Cap. 23.}\edtext{}{\lemma{\textso{23.}}\Bfootnote{\textsc{O. v. Guericke}, \cite{00055}a.a.O., S.~104.}} Corium si sit cera  therebintina commixta inunctum aer non  transit. Duae phialae diametrum simul  constituentes $\displaystyle\frac{3}{4}\rule[-4mm]{0mm}{10mm}\hspace{5.5pt}$% \begin{wrapfigure}{l}{0.4\textwidth}                    
                %\includegraphics[width=0.4\textwidth]{../images/Aus+Otto+von+Guericke%2C+Experimenta+nova/LH035%2C14%2C02_097v/files/100367.png}
                        %\caption{Bildbeschreibung}
                        %\end{wrapfigure}
                        %@ @ @ Dies ist eine Abstandszeile - fuer den Fall, dass mehrere figures hintereinander kommen, ohne dass dazwischen laengerer Text steht. Dies kann zu einer Fahlermeldung fuehren. @ @ @ \\
                     ulnae Magdeburgicae non poterant divelli a 16 equis, 16 equi\edtext{}{\lemma{}\Afootnote{equi  \textbar\ autem \textit{ gestr.}\ \textbar\ non \textit{ L}}}  non poterant elevare pondus aeris quod est hic  2686 librarum.  Putat ex his 16 equis 8.  onerari 2686 libris et alteros 8. itidem  (+ Ego puto, quo8vis dimidio tantum +). Etsi autem 8 equi  possint trahere currum tot libris onustum, tamen  ibi facilior\edtext{}{\lemma{facilior}\Bfootnote{Bei Guericke: difficilior}} tractio. 
                     \pend\pstart
                     \textso{Cap. 24.} \edtext{}{\lemma{\textso{24.}}\Bfootnote{\textsc{O. v. Guericke}, \cite{00055}a.a.O., S.~105.}} Duo haemisphaeria diametri 1 ulnae non potuere  distrahi 24 equis, imo non nisi a 34.  Nota ubi fit distractio, auditur sonus\protect\index{Sachverzeichnis}{sonus}  quasi sclopeti displosi, aere irruente et  confluxu sonum\protect\index{Sachverzeichnis}{sonus} edente (+ usus mechanicus horum +).\pend \pstart  Cap. 27\edtext{}{\lemma{27}\Bfootnote{\textsc{O. v. Guericke}, \cite{00055}a.a.O., S.~109.}} Ratisbonae\protect\index{Ortsregister}{Regensburg (Ratisbona)} in Comitiis \edtext{1654.}{\lemma{Comitiis}\Afootnote{ \textit{ (1) }\ vas \textit{ (2) }\ 1654. \textit{ L}}}\pend \pstart \textso{Cap. 29.}\edtext{}{\lemma{\textso{29.}}\Bfootnote{\textsc{O. v. Guericke}, \cite{00055}a.a.O., S.~112f.}} Quod sclopetum  novum attinet, longitudo proportionata esse  debet, observavi eum canalem ulnarum $\displaystyle4\frac{1}{2}\rule[-4mm]{0mm}{10mm}\hspace{5.5pt}$% \begin{wrapfigure}{l}{0.4\textwidth}                    
                %\includegraphics[width=0.4\textwidth]{../images/Aus+Otto+von+Guericke%2C+Experimenta+nova/LH035%2C14%2C02_097v/files/100462.png}
                        %\caption{Bildbeschreibung}
                        %\end{wrapfigure}
                        %@ @ @ Dies ist eine Abstandszeile - fuer den Fall, dass mehrere figures hintereinander kommen, ohne dass dazwischen laengerer Text steht. Dies kann zu einer Fahlermeldung fuehren. @ @ @ \\
                     longius quam 3 ulnarum globum ejicere,  ejaculari saepius potes uno Recipiente Epistomium\protect\index{Sachverzeichnis}{epistomium} vertendo, ut aer intret, sed subito  ut Epistomium\protect\index{Sachverzeichnis}{epistomium} rursus claudatur, interea fit  explosio.
                     \pend 
                     \newpage
                     \pstart  \textso{Cap. 31.}\edtext{}{\lemma{\textso{31.}}\Bfootnote{\textsc{O. v. Guericke}, \cite{00055}a.a.O., S.~114.}} Globum vacuum suspendi ex balance, id altius est quando aer  gravior depressius quando minus gravis. \edtext{Quia semper  gravius contentum, quo levius continens.  Sed quia hoc modo ponderatio difficulter institui potest, exacta,}{\lemma{gravis.}\Afootnote{ \textit{ (1) }\ Sed cum  haec bilanx non possit exacte, \textit{ (2) }\ Quia [...] exacta, \textit{ L}}} ideo adhibui inquit virunculum  illum ea arte in vitro suspensum, ut in aere  inferius ferretur, et ex eo pro diversa aeris  gravitate ascenderet et descenderet digito  etiam suo in quibusdam punctis variationem monstraret. (+ Ex his necesse est constructionem \edtext{meam}{\lemma{meam}\Bfootnote{Leibniz meint sehr wahrscheinlich sein Instrumentum inclinationum. Vgl. dazu \cite{00266}N. 47.}}  et ipsius coincidere. Utitur bis hac comparatione  etiam in literis ad P. Schottum\protect\index{Namensregister}{\textso{Schott} (Schottus), Caspar SJ 1608\textendash 1666}\edtext{}{\lemma{Schottum}\Bfootnote{\textsc{C. Schott, }\cite{00096}\textit{Technica curiosa}, S.~59\textendash65.}}. Sed objectio  ei fieri posset, ita aerem continentem fieri leviorem calore. Respondebit fortasse non fieri, ob compressionem a summo auctam etc. +)\pend \pstart \textso{Cap. 33.}\edtext{}{\lemma{\textso{33.}}\Bfootnote{\textsc{O. v. Guericke}\cite{00055}, a.a.O., S.~116.}} \textit{Aer in sclopeto  ventaneo ad spatium cogitur quintuplo minus.}\pend \pstart \textso{Cap. 37.}\edtext{}{\lemma{\textso{37.}}\Bfootnote{\textsc{O. v. Guericke}, \cite{00055}a.a.O., S.~122.}} Thermometrum\protect\index{Sachverzeichnis}{thermometrum}  novum Magdeburgicum in \textit{Technica Curiosa} descriptum  lib. II. c. 13. p. 871\edtext{}{\lemma{871}\Bfootnote{\textsc{C. Schott, }\cite{00096}\textit{Technica curiosa}, S.~871.}} ubi et internoscit Schottus\protect\index{Namensregister}{\textso{Schott} (Schottus), Caspar SJ 1608\textendash 1666}  quis sit \edtext{ejus}{\lemma{ejus}\Afootnote{\textit{ erg.} \textit{ L}}} anni calidissimus dies, quod alias  difficile vide ibi. Non est magni adeo momenti.
                     \pend 