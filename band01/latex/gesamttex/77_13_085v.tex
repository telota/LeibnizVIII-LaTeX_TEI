[85~v\textsuperscript{o}] ita ut omnes paralleli radii intra altitudinem perpendicularis @@@ G R A F I K @@@% \begin{wrapfigure}{l}{0.4\textwidth}                    
                %\includegraphics[width=0.4\textwidth]{../images/}
                        %\caption{Bildbeschreibung}
                        %\end{wrapfigure}
                        %@ @ @ Dies ist eine Abstandszeile - fuer den Fall, dass mehrere figures hintereinander kommen, ohne dass dazwischen laengerer Text steht. Dies kann zu einer Fahlermeldung fuehren. @ @ @ \\
                      Ex quibus patet quanto \textit{x} sine \textit{BF} minor est, tanto etiam  punctum \textit{I} longius distare ab \textit{N}, hoc est quanto radius  aliquis magis distat ab axe aut vertice \textit{D}, tanto etiam  remotius a vertice axem secare.\pend \pstart Deinde si concipiatur \textit{IDB}, circa axem \textit{DI} rotatam, figuram  vitri describere, facile etiam inveniri potest magnitudo  minimi plani ad angulos rectos ad \textit{DK} erecti, in quod omnes  radii qui \textit{DI} sunt paralleli, atque contenti intra Cylindrum illum  ab \textit{ABF}, circa axem \textit{DFN} rotata, descriptum, incidunt; (quod planum postea vocabitur focus:) sed cum non necesse  habeamus scire minimi hujus plani magnitudinem ut ad  propositum perveniamus, satis erit si tantum alterius  cujusdam, quod longe quam hoc majus est, atque in quo 