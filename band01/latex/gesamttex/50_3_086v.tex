\rule[0mm]{0mm}{10mm}
\pend\vspace{10mm}\pstart
[86 v\textsuperscript{o}] Liquet igitur ex praecedentibus, quod, si supponatur vitrum  figuram illam habere quam describit \textit{KDB}, circa axem \textit{DI}  rotata, ac semidiametrum circuli \textit{ND} aequari unitati, quod tum inquam omnes radii in Cylindro ex lineae \textit{AB} circa  axem \textit{DK} circumgyratione orto, contenti, cujus basis  semidiameter aequalis sit \textit{FB}, congregabuntur in producto axe \textit{DK}, nempe\\
\renewcommand{\arraystretch}{2.3}
\hspace{-8mm}
\parbox{1.5cm}{Cum \textit{FB} sumatur aequalis}$\left\{
\begin{tabular}{c}
$\displaystyle\frac{3}{5}$\\$\displaystyle\frac{5}{13}$\\$\displaystyle\frac{7}{25}$\\$\displaystyle\frac{9}{41}$\\$\displaystyle\frac{31}{481}$\\$\displaystyle\frac{49}{1201}$
\end{tabular}
\right\}$
\parbox{2cm}{intra lon-\\gitudinem minorem quam}
$\left\{
\begin{tabular}{c}
$\displaystyle\frac{1}{4}$\\$\displaystyle\frac{1}{10}$\\$\displaystyle\frac{1}{20}$\\$\displaystyle\frac{1}{33}$\\$\displaystyle\frac{1}{398}$\\$\displaystyle\frac{1}{994}$
\end{tabular}
\right\}$
\parbox{2cm}{eritque se-\\midiameter foci minor\\quam}
$\left\{
\begin{tabular}{c}
$\displaystyle\frac{1}{18}$\\$\displaystyle\frac{1}{78}$\\$\displaystyle\frac{1}{209}$\\$\displaystyle\frac{1}{438}$\\$\displaystyle\frac{1}{17642}$\\$\displaystyle\frac{1}{69615}$
\end{tabular}
\right\}$semidiametri \textit{ND}.\pend \pstart Apparet etiam, si in vitris quorum semidiameter aequatur $\displaystyle\frac{1}{4}\:$\rule[-4mm]{0mm}{10mm} digiti mensurae, sumatur apertura aequalis $\displaystyle\frac{7}{25}\:$\rule[-4mm]{0mm}{10mm} quartae partis  digiti, hoc est $\displaystyle\frac{14}{25}\:$ pro diametro basis praedicti Cylindri  radiorum (quae longitudo major est semisse semidiametri circuli \textit{NDB}, cujus figuram vitrum induit:) quod tum semidiameter foci minor erit quam $\displaystyle\frac{1}{209}\:$\rule[-4mm]{0mm}{10mm} quartae partis digiti. Unde constat, focum ipsum pro puncto mechanico\protect\index{Sachverzeichnis}{punctum!mechanicum} tantum habendum 