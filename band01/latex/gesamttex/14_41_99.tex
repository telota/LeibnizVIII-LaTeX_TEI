[p.~99] 4. diuidatur quilibet axis in partes aequales, ductisque per eas secundum numeros impares applicatis, quae sint vt radices quadratae segmentorum axis a praedictis applicatis sectorum v.g. sit axis AD,\footnote{\textit{Gedruckte Marginalie}: Figur 75.} diuisus in quotcumque partes aequales AB; sitque segmentum AB, 1. BC, 3. CD, 5. ducantur applicatae BE, CF, DG, haec vltima sit applicata et basis, eam diuido in 3. partes aequales, DIHG, ductisque BE, IE, item HF, CF, ibit parabola per puncta AEFG. 5.\footnote{\textit{Gedruckte Marginalie}: Figur 76.} sit quaelibet chorda in situ horizontali AB, citra tensionem,\footnote{\textit{Am Rand angestrichen}: 5. sit [...] tensionem } incuruatur in parabolam vt Galileus\protect\index{Namensregister}{\textso{Galilei} (Galilaeus, Galileus), Galileo 1564\textendash 1642} asserit, quod sic demonstro; [...].