[86 r\textsuperscript{o}]  etiam radii illi congregari debent, inveniatur magnitudo.  Quod ut fiat supponatur \textit{K} esse illud punctum quod  longissime ab \textit{N} aut \textit{D} distat, ad quod radius aliquis refractus  tendat, sitque \textit{I} punctum ad quod exterior radius cylindri, hic  per \textit{AB} designatus tendit. Deinde ducta sit \textit{BA}, sitque \textit{IM},  perpendicularis ad axem. Manifestum itaque est, omnes  radios praedicti Cylindri occurrere debere illi Circulo, qui ab \textit{IM} circa axem \textit{DNK} rotata, describitur, circulumque  hunc etiam longe majorem esse quam minimum illud planum,  in quo radii hi congregantur. Ut jam est \textit{KF} ad \textit{FB}, ita \textit{KI} ad \textit{IM}. Cumque \textit{IM} major evadat ex eo quod \textit{KI}  major supponatur eadem tamen remanente \textit{BF}, sequitur \textit{KF} sine \textit{KN} + \textit{NF} esse ad \textit{FB}, ut \textit{KI} + alia quad: lin: ad \textit{IM} + alia quad: lin:@@@ G R A F I K @@@% \begin{wrapfigure}{l}{0.4\textwidth}                    
                %\includegraphics[width=0.4\textwidth]{../images/}
                        %\caption{Bildbeschreibung}
                        %\end{wrapfigure}
                        %@ @ @ Dies ist eine Abstandszeile - fuer den Fall, dass mehrere figures hintereinander kommen, ohne dass dazwischen laengerer Text steht. Dies kann zu einer Fahlermeldung fuehren. @ @ @ \\
                    