[18~v\textsuperscript{o}] jecti omnibus, quod lentes\protect\index{Sachverzeichnis}{lens} communes axi Optico\protect\index{Sachverzeichnis}{axis!opticus} tribuunt; jam constat ne puncti quidem objecti, in axe Optico\protect\index{Sachverzeichnis}{axis!opticus} positi radios omnes ulla figura, nisi quae Pandocha esse non potest, reduniri) ideo inventum est a me remedium novum, et intactum, quo magna pars pereuntium radiorum conservatur.\pend \pstart Hoc ut verbo dicam, praestatur \textso{Tubis }\protect\index{Sachverzeichnis}{tubus!Catadioptricus}quibusdam \textso{Catadioptricis,} (ad normam tamen Lentium Pandocharum\protect\index{Sachverzeichnis}{lens!pandocha} constructis) id est, \textso{conjunctione }\textso{Dioptricae}\protect\index{Sachverzeichnis}{dioptrica}\textso{ et }\textso{Catoptricae}\protect\index{Sachverzeichnis}{catoptrica}\textso{ in unam visionem,} cujus primus omnium, quod sciam, meritissimus de re Mathematica Hevelius\protect\index{Namensregister}{\textso{Hevelius,} Johannes 1611\textendash 1687} in \textit{Polemoscopio} specimen dedit,\edtext{}{\lemma{dedit,}\Bfootnote{\textsc{J. Hevelius, }\cite{00058}\textit{Selenographia}, Danzig 1647, S.~24\textendash31. }} sed alio plane consilio fructuque.\pend \pstart Rem tanta certitudine, quanta caetera Optica omnia habemus, demonstrasse mihi videor, atque illud etiam comperisse, \textso{Hyperbolae} et \textso{Ellipsae} et aliarum id genus figurarum non\textendash pandocharum virtutes ad \textso{distinctam visionem} efficiendam tantas non fore, quantae passim habentur, nec proinde in \textso{projiciendis imaginibus} expectationi satisfacturas.\pend\pstart Ad radios autem diversorum etiam punctorum confundendos, aut in exiguum spatium contrudendos, id est ad \textso{comburendum} aut \textso{illustrandum} (qui duo sunt effectus luminis\protect\index{Sachverzeichnis}{lumen} intensi, sed confusi) magnam utique vim habebunt, ac proinde poliri eas operae pretium erit. Quod duobus tantum motibus,\pend\begin{center} recto et circulari, et utroque nonnisi semel adhibito, facili\\ negotio praestari potest.\end{center}