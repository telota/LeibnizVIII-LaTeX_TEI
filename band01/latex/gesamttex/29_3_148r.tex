      
               
                \begin{ledgroupsized}[r]{120mm}
                \footnotesize 
                \pstart                
                \noindent\textbf{\"{U}berlieferung:}   
                \pend
                \end{ledgroupsized}
            
              
                            \begin{ledgroupsized}[r]{114mm}
                            \footnotesize 
                            \pstart \parindent -6mm
                            \makebox[6mm][l]{\textit{L}}Konzept: LH XXXVII 3 Bl. 148\textendash149 1 Bog. 2\textsuperscript{o}, 2 1/2 S. zweispaltig. Bl. 149 r\textsuperscript{o} zur H\"{a}lfte beschrieben, Bl. 149 v\textsuperscript{o} leer.\\Cc 2 Nr. 491 F \pend
                            \end{ledgroupsized}
                \vspace*{8mm}
                \pstart 
                \normalsize
            [148 r\textsuperscript{o}] Toutes ces Hypotheses estant presque refut\'{e}es d\'{e}ja par des \textso{experiences toutes faites,} il reste une seule, de celles dont je crois me pouuoir \'{e}claircir entierement, par des \textso{experiences} ais\'{e}es; \edtext{\`{a} faire; qui m'oblige de balancer encor}{\lemma{ais\'{e}es;}\Afootnote{ \textit{ (1) }\ que je n'oserois encor condemner \textit{ (2) }\ \`{a} [...] encor \textit{ L}}} avec une indifference indetermin\'{e}e, jusqu'\`{a} l'evenement de l'experience probatoire. Elle me tomba dans l'esprit, en cherchant quelque moyen, qui pourroit suppleoir icy, dans le Recipient \'{e}puis\'{e} même, le defaut de la pression de l'Atmosphere\protect\index{Sachverzeichnis}{atmosph\`{e}re}. Je con\c{c}ois donc un air rafin\'{e}, ou si vous voulez \edtext{matiere plus subtile que l'air}{\lemma{voulez}\Afootnote{ \textit{ (1) }\ aether\protect\index{Sachverzeichnis}{aether|textit} c'est \`{a} dire \textit{ (2) }\ matiere plus subtile que l'air \textit{ L}}}, dont l'atmosphere\protect\index{Sachverzeichnis}{atmosph\`{e}re} \edtext{de l'air}{\lemma{}\Afootnote{de l'air \textit{ erg.} \textit{ L}}} est entrelass\'{e}e: comme nous voyons que les vapeurs sont entrelass\'{e}es d'air. Cette matiere quoyque peu grossiere, est pourtant pesante, (parce qu'elle va sans comparaison plus \edtext{en}{\lemma{}\Afootnote{en \textit{ erg.} \textit{ L}}} haut que l'air,) et compose une essence d'une autre atmosphere\protect\index{Sachverzeichnis}{atmosph\`{e}re}, outre celle de l'air; \`{a} l'entour \edtext{du globe de la terre}{\lemma{l'entour}\Afootnote{ \textit{ (1) }\ de nostre terre\protect\index{Sachverzeichnis}{terre|textit} \textit{ (2) }\ du globe de la terre \textit{ L}}}. Sa subtilit\'{e} va jusqu'\`{a} passer par les pores même des solides, comme du verre, pourveu qu'\edtext{on la force}{\lemma{qu'}\Afootnote{ \textit{ (1) }\ elle soit forc\'{e}e \textit{ (2) }\ on la force \textit{ L}}}. \edtext{Car sans cela elles les suspend, comme la placque inferieure, attach\'{e}e \`{a} la superieure dans la vuide\protect\index{Sachverzeichnis}{vide}, par le poids\protect\index{Sachverzeichnis}{poids} d'une colomne de toute la hauteur de cette nouuelle Atmosphere\protect\index{Sachverzeichnis}{atmosph\`{e}re}, mais de la largeur de la placque. Par consequent la placque dans l'air libre est suspend\"{u}e par le poids\protect\index{Sachverzeichnis}{poids} de deux atmospheres\protect\index{Sachverzeichnis}{atmosph\`{e}re}, dans le vuide\protect\index{Sachverzeichnis}{vide}, par la pression de la nouuelle seulement.}{\lemma{}\Afootnote{BITTE UEBERPRUEFEN!!! Car [...] lavuide\protect\index{Sachverzeichnis}{vide}, par le poids\protect\index{Sachverzeichnis}{poids}d'une [...] nouuelleAtmosphere\protect\index{Sachverzeichnis}{atmosph\`{e}re}, [...] lepoids\protect\index{Sachverzeichnis}{poids} de deux atmospheres\protect\index{Sachverzeichnis}{atmosph\`{e}re}, dans le vuide\protect\index{Sachverzeichnis}{vide}, [...] seulement. \textit{ erg.} \textit{ L}}} Pour le passage des liqueurs, il y a une difference tres considerable, car si les liqueurs sont entrelass\'{e}es d'air, (comme une esponge enfl\'{e}e d'eau) ce fluide subtil passe sans peine trouuant \edtext{passage par l'air}{\lemma{trouuant}\Afootnote{ \textit{ (1) }\ l'air, qui luy donne \textit{ (2) }\ passage par l'air \textit{ L}}}, son domicile ordinaire, dont elle n'est differente peut estre qu'en finesse. Mais si les liqueurs\protect\index{Sachverzeichnis}{liqueur!purg\'{e}e} en sont purg\'{e}es, elle se trouue arrest\'{e}e tout court. Les parties de la liqueur estant \`{a} present plus serr\'{e}es qu'il ne luy faut pour avoir passage. \edtext{D'o\`{u} vient que les liqueurs purg\'{e}es\protect\index{Sachverzeichnis}{liqueur!purg\'{e}e} hors du Recipient, sont suspend\"{u}es d'une hauteur plus grande que les autres non purg\'{e}es; et demeurent suspend\"{u}es dans le Recipient \'{e}puis\'{e}, quand les autres tombent.}{\lemma{}\Afootnote{BITTE UEBERPRUEFEN!!! D'o\`{u} vient que les liqueurs purg\'{e}es\protect\index{Sachverzeichnis}{liqueur!purg\'{e}e}  \textit{ (1) }\ sont suspend\"{u}es non seulement par l'atmosphere\protect\index{Sachverzeichnis}{atmosph\`{e}re|textit}, \textit{ (2) }\ hors [...] grande \textit{(a)}\ que celle de l'atmosphere\protect\index{Sachverzeichnis}{atmosph\`{e}re|textit} \textit{(b)}\ que les autres non purg\'{e}es; [...] tombent. \textit{ erg.} \textit{ L}}} Jusqu'\`{a} ce qu'une bulle d'air, quoyque petite, arrivant \edtext{\`{a} cette partie de la liqueur, qui est suspend\"{u}e par}{\lemma{arrivant}\Afootnote{ \textit{ (1) }\ \`{a} l'eau suspend\"{u}e par \textit{ (2) }\ \`{a} [...] par \textit{ L}}} la pression nouuelle,\edtext{}{\lemma{}\Afootnote{nouuelle,  \textbar\ et \textit{ gestr.}\ \textbar\ la \textit{ L}}} la pesanteur de la liqueur estant soûten\"{u}e de celle de la nouuelle colomne, pour sortir du combat, est bien ais\'{e} de se servir de cet arbitre de leurs differents, et oblige la bulle \edtext{d'air}{\lemma{}\Afootnote{d'air \textit{ erg.} \textit{ L}}} de se repandre en un moment, par toute la \edtext{liqueur}{\lemma{la}\Afootnote{ \textit{ (1) }\ matiere \textit{ (2) }\ liqueur \textit{ L}}}, pour faire \edtext{passage}{\lemma{faire}\Afootnote{ \textit{ (1) }\ place \textit{ (2) }\ passage \textit{ L}}} \`{a} ce fluide.\pend \pstart  Mais quoyque je n'ose pas encor prononcer contre cette Hypothese, je n'y trouue pas pourtant toute la satisfaction. Car \edtext{}{\lemma{Car}\Afootnote{ \textit{ (1) }\ premierement \textit{ (2) }\  \textit{ L}}}