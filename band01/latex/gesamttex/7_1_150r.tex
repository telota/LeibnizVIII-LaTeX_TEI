      
               
                \begin{ledgroupsized}[r]{120mm}
                \footnotesize 
                \pstart                
                \noindent\textbf{\"{U}berlieferung:}   
                \pend
                \end{ledgroupsized}
            
              
                            \begin{ledgroupsized}[r]{114mm}
                            \footnotesize 
                            \pstart \parindent -6mm
                            \makebox[6mm][l]{\textit{L}}Konzept: LH XXXVII 3 Bl. 150\textendash151. 1 Bog. 2\textsuperscript{o}. 4 S. zweispaltig. Rechte und linke Sp. vollst\"{a}ndig beschrieben. Im oberen Drittel von Bl. 151 r\textsuperscript{o} Flecken, die auf die R\"{u}ckseite durchschlagen und beidseitig zu geringen Textverlusten f\"{u}hren. \pend
                            \end{ledgroupsized}
              
                            \begin{ledgroupsized}[r]{114mm}
                            \footnotesize 
                            \pstart \parindent -6mm
                            \makebox[6mm][l]{\textit{$E^1$}}\textsc{M. L. Alcoba}, \cite{00189}\textit{G. W. Leibniz: Consequence de l'Hypothese generalle publi\'{e}e il y a quelque temps, pour expliquer le Phenomene de l'attachement dans le vuide, ou dans une place dont l'air a est\'{e} tir\'{e}}, in: Studia Leibnitiana, XXVIII (1996) S.~7\textendash16.\pend
                            \end{ledgroupsized}         
                            \begin{ledgroupsized}[r]{114mm}
                            \footnotesize 
                            \pstart \parindent -6mm
                            \makebox[6mm][l]{\textit{$E^2$}}\textsc{M. L. Alcoba}, \cite{00190}\textit{La ley de continuidad en G. W. Leibniz}, Dissertation Sevilla 1994, S.~342\textendash350.\\Cc 2, Nr. 491 B \pend
                            \end{ledgroupsized}
           %     \vspace*{8mm}
                \clearpage
                \pstart 
                \normalsize
            [150 r\textsuperscript{o}] \selectlanguage{french}Consequence de l'Hypothese generalle  publi\'{e}e il y a quelque temps, pour expliquer \textso{le phenomene  de l'attachement dans le Vuide, ou dans une place dont l'air a est\'{e} tir\'{e}.}\edlabel{tire150r1}\pend 
            \pstart \edtext{\edlabel{tire150r2}\textso{Phenomene 1}.}{\lemma{\textso{tir\'{e}.}}\xxref{tire150r1}{tire150r2}\Afootnote{ \textit{ (1) }\ Il a est\'{e} observ\'{e} par  \textit{(a)}\ Messieurs Boyle\protect\index{Namensregister}{\textso{Boyle} (Boylius, Boyl., Boyl), Robert 1627\textendash 1691|textit}, Huygens\protect\index{Namensregister}{\textso{Huygens} (Hugenius, Vgenius, Hugens, Huguens), Christiaan 1629\textendash 1695|textit} et Guericke\protect\index{Namensregister}{\textso{Guericke} (Gerickius, Gerick.), Otto v. 1602\textendash 1686|textit} \textit{(b)}\ Mons. Hugens\protect\index{Namensregister}{\textso{Huygens} (Hugenius, Vgenius, Hugens, Huguens), Christiaan 1629\textendash 1695|textit} dans l'air libre \textit{ (2) }\ \textso{Phenomene 1}. \textit{ L}}} Les liqueurs ne \edtext{s'\'{e}coulent}{\lemma{ne}\Afootnote{ \textit{ (1) }\ sortent \textit{ (2) }\ s'\'{e}coulent \textit{ L}}} pas \edtext{d'un}{\lemma{pas}\Afootnote{ \textit{ (1) }\ regulierement \textbar\ dans\textit{ (1) }\ l'air \textit{ (2) }\ la liqueur \textit{erg. u. gestr.} \textbar\ \textit{ (2) }\ d'un \textit{ L}}} \edtext{vase}{\lemma{d'un}\Afootnote{ \textit{ (1) }\ tuyau \textit{ (2) }\ vase \textit{ L}}} qui est ouuert seulement \edtext{par un trou}{\lemma{seulement}\Afootnote{ \textit{ (1) }\ par le bout \textit{ (2) }\ par un trou \textit{ L}}} \edtext{si l'experience se fait dans l'air libre comme par exemple si un tuyau de verre ouuert en haut et bouch\'{e} en bas, rempli d'eau et renvers\'{e} par apres adroitement, l'eau n'en pourra pas \'{e}couler}{\lemma{trou}\Afootnote{ \textit{ (1) }\ en  \textit{(a)}\ bas \textit{(b)}\ un petit endroit comme   \textbar\ par exemple \textit{ erg.}\ \textbar\  d'un tuyau de verre bouch\'{e} en haut, et ouuert en bas \textit{ (2) }\ si [...] verre \textit{(a)}\ soit ouuert d'une co \textit{(b)}\ ouuert [...] \'{e}couler \textit{ L}}}.\pend 
            \pstart \textso{Phaen. 2}. Pourveu que \edtext{la hauteur de la liqueur}{\lemma{que}\Afootnote{ \textit{ (1) }\ leur hauteur \textit{ (2) }\ la hauteur de la liqueur \textit{ L}}} ne soit pas \edtext{trop augment\'{e}e, outre une}{\lemma{pas}\Afootnote{ \textit{ (1) }\ augment\'{e}e, jusques \`{a} une \textit{ (2) }\ trop augment\'{e}e, outre une \textit{ L}}} mesure determin\'{e}e selon l'espece de la liqueur, car alors \edtext{il se trouue}{\lemma{alors}\Afootnote{ \textit{ (1) }\ on a observ\'{e} \textit{ (2) }\ il se trouue \textit{ L}}} qu'elles tombent.\footnote{\textit{In der rechten Spalte}: \selectlanguage{french}\textso{Experiences faites}\hspace{-0.2em}: les liqueurs n'\'{e}coulent pas d'un tuyau \'{e}troit,  ouuert par un bout seulement, % \edtext
            quoyque il soit renvers\'{e}. (2) Pourveu que la hauteur de la liqueur ne soit pas trop grande: car il y a des hauteurs determin\'{e}es selon espece de la liqueur (les plus pesantes n'ayant pas besoin d'une si grande) qui la font tomber comme l'eau a besoin environ de 30 pieds, le Mercure\protect\index{Sachverzeichnis}{mercure} de 27. pouces de hauteur.} \edtext{}{\lemma{seulement,}\linenum{|12|||12|}\Afootnote{ \textit{ (1) }\ de quelle maniere, \textit{ (2) }\ quoyque \textit{ L}}}\pend 
            \pstart \textso{Phenom. 3}. On a pourtant observ\'{e} \edtext{le Tuyau  estant mis dans un Recipient, dont l'air estoit tir\'{e} avec la pompe de Mons. Guericke, que la liqueur s'\'{e}couloit}{\lemma{observ\'{e}}\Afootnote{ \textit{ (1) }\ que dans un vase, dont on a tir\'{e} l'air avec la pompe de Mons. Guericke\protect\index{Namensregister}{\textso{Guericke} (Gerickius, Gerick.), Otto v. 1602\textendash 1686|textit}, la liqueur \'{e}coule \textit{ (2) }\ un Recipient  \textit{(a)}\ ferm\'{e} \textit{(b)}\ bien bouch\'{e} \textit{(c)}\ dont la liqueur s'\'{e}coule, sitost qu'on \textit{ (3) }\ que le vase ou \textit{ (4) }\ le Tuyau   \textbar\   \textbar\ renvers\'{e} \textit{ gestr.}\ \textbar\   estant \textit{ erg.}\ \textbar\ mis [...] s'\'{e}couloit \textit{ L}}}, comme s'il y avoit un trou dans le haut du tuyau.\pend 
            \pstart \textso{Phaenom. 4}. Mais comme l'eau, ou quelque autre liqueur ayant demeur\'{e} longtemps dans le vuide se purge de l'air \edtext{ou s'\'{e}puise de la matiere propre \`{a} produir l'air, en faisant}{\lemma{l'air}\Afootnote{ \textit{ (1) }\ en tenant \textit{ (2) }\ ou s'\'{e}puise de la matiere  \textit{(a)}\ d'engendrer \textit{(b)}\ propre \`{a} produir l'air, en faisant \textit{ L}}} continuellement des petites bulles; il est arriv\'{e} enfin comme on se servoit de cette même eau purg\'{e}e dans le \edtext{dit}{\lemma{}\Afootnote{dit \textit{ erg.} \textit{ L}}} tuyau, elle ne s'\'{e}couloit pas: quoyque l'experience estant faite dans le vuide il y avoit qui luy resistoit.\pend 
            \pstart \textso{Phaenom. 5}. \edtext{Neantmoins}{\lemma{5.}\Afootnote{ \textit{ (1) }\ Sinon quand \textit{ (2) }\ Neantmoins \textit{ L}}} elle avoit receu un choc, ou quand une nouuelle bulle d'air qui estoit engendr\'{e}e \edtext{au fond de l'eau, ou qu'on avoit fait entrer}{\lemma{engendr\'{e}e}\Afootnote{ \textit{ (1) }\ dans l'eau \textit{ (2) }\ au [...] entrer \textit{ L}}} estoit mont\'{e}e \`{a} une certaine hauteur du tuyau,\edtext{}{\lemma{}\Afootnote{tuyau,  \textbar\ car \textit{ gestr.}\ \textbar\ alors \textit{ L}}} alors la liqueur se d\'{e}tachoit, et tomboit \`{a} l'ordinaire.\pend 
            \pstart \edtext{\textso{Phaenom. 6}. Cette hauteur est justement la même avec celle dont par apres la liqueur (apres estre tomb\'{e}e \`{a} l'ordinaire) demeure encor suspend\"{u}e.}{\lemma{}\Afootnote{\textso{Phaenom. 6}. Cette hauteur  \textit{ (1) }\ estoit jusque la hauteur \`{a} laquelle \textit{ (2) }\ est [...] liqueur  \textit{ (a) }\ est \textit{ (b) }\ (apres estre  \textit{(aa)}\ tomb\'{e}e \textit{(bb)}\ d\'{e}tach\'{e}e \textit{(cc)}\ tomb\'{e}e \`{a} l'ordinaire)  \textit{ (aaa) }\ demeuroit \textit{ (bbb) }\ demeure encor suspend\"{u}e. \textit{ erg.} \textit{ L}}}\pend 
            \pstart \textso{Phaenom. 7}. Le même arrive avec le Mercure\protect\index{Sachverzeichnis}{mercure} hors du Recipient. Car comme \edtext{l'eau}{\lemma{comme}\Afootnote{ \textit{ (1) }\ il \textit{ (2) }\ l'eau \textit{ L}}} \edtext{ordinaire}{\lemma{}\Afootnote{ordinaire \textit{ erg.} \textit{ L}}} bien que d'une \edtext{petite pesanteur}{\lemma{d'une}\Afootnote{ \textit{ (1) }\ hauteur mediocre \textit{ (2) }\ petite pesanteur \textit{ L}}} tombe dans le Recipient \edtext{\'{e}puis\'{e}}{\lemma{}\Afootnote{\'{e}puis\'{e} \textit{ erg.} \textit{ L}}}, parce que l'obstacle de l'air en est ost\'{e}, de même le Mercure\protect\index{Sachverzeichnis}{mercure} \edtext{ordinaire}{\lemma{}\Afootnote{ordinaire \textit{ erg.} \textit{ L}}} tombe dans l'air libre, parce que sa pesanteur est grande. Mais comme \edtext{le Mercure purg\'{e}e ne tombe pas dans}{\lemma{comme}\Afootnote{ \textit{ (1) }\ (par le phaenomene precedant,) l'eau purg\'{e}e ne tombe pas dans le Recipient,   \textbar\ bien qu'\'{e}puis\'{e} \textit{ erg.}\ \textbar\ : de même \textit{ (2) }\ le [...] dans \textit{ L}}} l'air, quoyque sa hauteur soit plus grande qu'\`{a} l'ordinaire, et qu'elle aille même jusqu'\`{a} 70 pouces, au lieu de 27.\edlabel{27150r1}\pend
             \pstart \edtext{\textso{Phaenom. 8}. Si une liqueur purg\'{e}e est demeur\'{e}e longtemps dans le vuide en un certain endroit il faut que le choc soit plus fort pour l'en d\'{e}tacher.\edlabel{27150r2}}{\lemma{}\Afootnote{\textit{ (1) }\ \textso{Phaenom. 8}.  \textit{(a)}\ Sinon to \textit{(b)}\ Quand une bulle d'air s'engendre dans la liqueur, et monte en haut,  sitost \textit{ (2) }\ \textso{Phaenom.} [...] endroit \textit{(a)}\ elle s'en d\'{e}tache plus difficilement par le choc \textit{(b)}\ il [...] d\'{e}tacher. \textit{ erg.} \textit{ L}}}\pend
              \pstart \textso{Phaenom. 9}. \edtext{On avoit crû, que deux placques}{\lemma{9.}\Afootnote{ \textit{ (1) }\ Enfin deux placques \textit{ (2) }\ On [...] placques \textit{ L}}} bien unies ne se separoient pas \`{a} cause de la pression de l'air\protect\index{Sachverzeichnis}{pression de l'air}; mais on a \'{e}prouu\'{e}, \edtext{que le même attachement}{\lemma{\'{e}prouu\'{e},}\Afootnote{ \textit{ (1) }\ qu'ils \textit{ (2) }\ que le même attachement \textit{ L}}} se trouue aussi dans le vuide ou Recipient \'{e}puis\'{e}.\pend \pstart \textso{Phaenom. 10}. Le siphon\protect\index{Sachverzeichnis}{siphon} \`{a} deux jambes \edtext{inegales}{\lemma{inegales}\Afootnote{\textit{ erg.} \textit{ L}}} fait son effect aussi bien dans le vuide que dans l'air.\pend \pstart