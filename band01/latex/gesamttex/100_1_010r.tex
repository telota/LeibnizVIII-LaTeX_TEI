      
               
                \begin{ledgroupsized}[r]{120mm}
                \footnotesize 
                \pstart                
                \noindent\textbf{\"{U}berlieferung:}   
                \pend
                \end{ledgroupsized}
            
              
                            \begin{ledgroupsized}[r]{114mm}
                            \footnotesize 
                            \pstart \parindent -6mm
                            \makebox[6mm][l]{\textit{L}}Notiz: LH XXXVII 2 Bl. 10. 1 Bl. 10 x 7 cm. 1 S., 16 Zeilen. An drei Seiten beschnitten. Unterer Rand unregelm\"{a}ßig abgerissen, R\"{u}ckseite leer.\\Kein Eintrag in KK 1 oder Cc 2. \pend
                            \end{ledgroupsized}
                %\normalsize
                \vspace*{5mm}
                \begin{ledgroup}
                \footnotesize 
                \pstart
            \noindent\footnotesize{\textbf{Datierungsgr\"{u}nde}: Die Notiz bezieht sich auf denselben Sachverhalt wie N. 34. Da die beiden St\"{u}cke auch hinsichtlich der referierten Descartes-Stelle \"{u}bereinstimmen, gehen wir von gleichen Entstehungszeiten aus.}
                \pend
                \end{ledgroup}
            
                \vspace*{8mm}
                \pstart 
                \normalsize
            [10 r\textsuperscript{o}]  Quia ostendunt Hugenius\protect\index{Namensregister}{\textso{Huygens} (Hugenius, Vgenius, Hugens, Huguens), Christiaan 1629\textendash 1695} et Huddenius\protect\index{Namensregister}{\textso{Hudde} (Huddenius), Jan 1628\textendash 1704}\edtext{}{\lemma{et}\Bfootnote{\textsc{R. Descartes, }\cite{00036}\textit{ Geometria}, Teil 1, Frankfurt 1659, S.~270. }} aliquo casu radios omnes ab eodem puncto venientes, \edtext{vel ad unum punctum tendentes}{\lemma{venientes,}\Afootnote{ \textit{ (1) }\ iterum ad unum punctum colligi \textit{ (2) }\ vel ad unum punctum tendentes \textit{ L}}}, refringi posse \edtext{quasi}{\lemma{posse}\Afootnote{ \textit{ (1) }\ quaesita \textit{ (2) }\ quasi \textit{ L}}} ab uno alio puncto venirent, vel ad unum aliud punctum tenderent, hinc ducere licebit modum componendi vitrum tale, cum aliis communibus, ut ea res usum aliquem habere possit, quem in uno vitro Huddenius\protect\index{Namensregister}{\textso{Hudde} (Huddenius), Jan 1628\textendash 1704} exiguum fore ostendit. \pend 