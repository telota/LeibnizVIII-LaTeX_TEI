[68 r\textsuperscript{o}] omnibus \edtext{Geographis}{\lemma{}\Afootnote{Geographis \textit{ erg.} \textit{ L}}} usurpari. Hujus ergo meridiani\protect\index{Sachverzeichnis}{meridianus} primi tellurem totam repraesentatis loco adhibeatur  circulus quidam rigidus, Polis\protect\index{Sachverzeichnis}{polus} sphaerae artificialis affixus, sed mobiliter tamen, ita ut separatim  circa polum pro lubitu circumagi possit.  Separatim inquam hic circulus circa polum\protect\index{Sachverzeichnis}{polus}  et sphaeram; et separatim ipsa sphaera circa polum\protect\index{Sachverzeichnis}{polus}; ita ut nec Meridiano\protect\index{Sachverzeichnis}{meridianus} primo circumacto ideo circumagatur sphaera; nec sphaera  circumacta ideo circumagatur Meridianus\protect\index{Sachverzeichnis}{meridianus}  primus.\pend \pstart Quemadmodum autem sphaera, ita et Meridianus\protect\index{Sachverzeichnis}{meridianus} primus, cum polo\protect\index{Sachverzeichnis}{polus} elevabitur aut deprimetur. Elevatione ergo poli\protect\index{Sachverzeichnis}{elevatio!poli} data, ultra horizontem artificialem constituta, et sphaera artificiali circa polum, ita acta, ut sidus observatum in sphaera depictum, tantum distet ab horizonte et meridiano\protect\index{Sachverzeichnis}{meridianus} loci navis\protect\index{Sachverzeichnis}{navis} \edtext{artificiali, quantum}{\lemma{artificiali}\Afootnote{ \textbar\ immobili \textit{ gestr.}\ \textbar\ , quantum \textit{ L}}} distare observatum est ab horizonte et meridiano\protect\index{Sachverzeichnis}{meridianus} loci navis\protect\index{Sachverzeichnis}{navis} vero. Hoc \edtext{inquam}{\lemma{}\Afootnote{inquam \textit{ erg.} \textit{ L}}} facto meridianus\protect\index{Sachverzeichnis}{meridianus} primus ita circumagatur ut sidus\protect\index{Sachverzeichnis}{sidus} in globo depictum quodcunque (idem,  aliudve cum eo quod observatum est) quod momento observationis per horologium\protect\index{Sachverzeichnis}{horologium} dato, calculus, aut Ephemerides\protect\index{Sachverzeichnis}{ephemeris} nostrae Meridianum\protect\index{Sachverzeichnis}{meridianus} primum subire, aut ab eo dato graduum minutorumque numero distare monstrant, subeat etiam Meridianum\protect\index{Sachverzeichnis}{meridianus} primum artificialem, aut eodem graduum minutorumve \edtext{numero in aequatore\protect\index{Sachverzeichnis}{aequator}}{\lemma{numero}\Afootnote{ \textit{ (1) }\  in aequatore\protect\index{Sachverzeichnis}{aequator|textit} aut parallelo, quem ejus sideris\protect\index{Sachverzeichnis}{sidus|textit} latitudo\protect\index{Sachverzeichnis}{latitudo|textit} monstrat, \textit{ (2) }\ in aequatore   \textbar\ aut parallelo, quem ejus sideris latitudo monstrat, \textit{ gestr.}\ \textbar\ numerato\ \textit{ L}}} numerato, ab eo distet.\pend \pstart Hoc facto  habebimus mundum artificialem, vero similem, ac proinde eam distantiam Meridiani\protect\index{Sachverzeichnis}{meridianus} primi  artificialis, a meridiano\protect\index{Sachverzeichnis}{meridianus} loci artificiali, in globo, quae est veri primi, a vero loci in mundo et  quod hinc sequitur gradus longitudinis\protect\index{Sachverzeichnis}{longitudo} in aequatore\protect\index{Sachverzeichnis}{aequator} globi artificialis numerabiles inventos, quod erat faciendum.\pend \pstart Saepe mecum miratus sum hanc sphaerae artificialis accessionem, ad usum Geographicum,