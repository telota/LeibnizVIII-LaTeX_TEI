\pend \pstart [p.~55] Quand le plan du tableau se trouue paralelle \`{a} la figure\footnote{\textit{Leibniz unterstreicht}: Quand [...] figure} qui est le suiet, lors en quelle part que l'oeil\protect\index{Sachverzeichnis}{oeil} se trouue situ\'{e}, la figure de representation est to\^{u}jours entierement de mesme forme que celle du suiet: et de plus vne mesme grandeur sert \`{a} la mesurer toute en tout sens d'vn bout \`{a} l'autre.\\Quand l'oeil\protect\index{Sachverzeichnis}{oeil} est entendu situ\'{e} \`{a} distance infinie\footnote{\textit{Leibniz unterstreicht}: Quand [...] infinie}, ou intermin\'{e}e, en quelque sorte que le plan du tableau soit situ\'{e}, la figure de representation est de telle espece, que ces deux choses s'y trouuent; [...].\pend 