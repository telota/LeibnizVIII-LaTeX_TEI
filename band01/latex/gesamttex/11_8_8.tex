      
               
                \begin{ledgroupsized}[r]{120mm}
                \footnotesize 
                \pstart                
                \noindent\textbf{\"{U}berlieferung:}   
                \pend
                \end{ledgroupsized}
            
              
                            \begin{ledgroupsized}[r]{114mm}
                            \footnotesize 
                            \pstart \parindent -6mm
                            \makebox[6mm][l]{\textit{LiH}}Marginalien sowie eine Unterstreichung in \textsc{J. Dubreuil}, \cite{00164}\textit{La perspective practique}, Paris 1642. Der Titel enth\"{a}lt dar\"{u}ber hinaus zahlreiche An- und Unter\-streichungen, die nicht von Leibniz stammen. \pend
                            \end{ledgroupsized}
                %\normalsize
                \vspace*{5mm}
                \begin{ledgroup}
                \footnotesize 
                \pstart
            \noindent\footnotesize{\textbf{Datierungsgr\"{u}nde}: Aleaumes\protect\index{Namensregister}{\textso{Aleaume} (Alleaume), Jacques 1562\textendash 1627} Schrift \cite{00003}\textit{La perspective speculative et pratique} und Dubreuils\protect\index{Namensregister}{\textso{Dubreuil,} Jean 1602\textendash 1670} Titel \cite{00164}\textit{La perspective practique} sind in Leibniz' Handexemplar zusammengebunden, so dass wir die Datierung aus N. 29 \"{u}bernehmen.}
                \pend
                \end{ledgroup}
            
                \vspace*{8mm}
                \pstart 
                \normalsize
           \selectlanguage{french} [p.~8] [...] et si les objets estoient produits \`{a} l'infiny, ils s'approcheroient tousiours plus pres de ce rayon centrical T, iusques \`{a} ce qu'ils sembleroient ne faire qu'vn poinct qui seroit \`{a} l'infiny comme doiuent estre tous les poincts de veuë.\footnote{\textit{Leibniz unterstreicht:} qui seroit \`{a} l'infiny comme doiuent estre tous les poincts de veuë.}