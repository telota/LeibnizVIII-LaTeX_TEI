      
               
                \begin{ledgroupsized}[r]{120mm}
                \footnotesize 
                \pstart                
                \noindent\textbf{\"{U}berlieferung:}   
                \pend
                \end{ledgroupsized}
            
              
                            \begin{ledgroupsized}[r]{114mm}
                            \footnotesize 
                            \pstart \parindent -6mm
                            \makebox[6mm][l]{\textit{L}}Konzept: LH XXXVII 2 Bl. 9. 1 Bl. 19 x 7 cm. 2 S. An drei R\"{a}ndern beschnitten, unterer Rand unregelm\"{a}ßig abgerissen. Die Zeichnungen befinden sich in der oberen H\"{a}lfte von Bl. 9 r\textsuperscript{o} und teilen den Text optisch im Verh\"{a}ltnis 1:3.\\Kein Eintrag in KK 1 oder Cc 2. \pend
                            \end{ledgroupsized}
                %\normalsize
                \vspace*{5mm}
                \begin{ledgroup}
                \footnotesize 
                \pstart
            \noindent\footnotesize{\textbf{Datierungsgr\"{u}nde}: Leibniz bezieht sich in diesem St\"{u}ck auf den \cite{00124}\textit{Cursus seu mundus mathematicus} von Dechales, der 1674 in Lyon\protect\index{Ortsregister}{Lyon (Lugdunum)} erschienen ist. Wir gehen von dem Erscheinungsdatum dieses Titels als dem wahrscheinlichsten Entstehungszeitraum unseres St\"{u}cks aus.}
                \pend
                \end{ledgroup}
            
                \vspace*{8mm}
                \pstart 
                \normalsize
           \begin{center} [9 r\textsuperscript{o}] Optici phaenomeni explicatio \end{center} \pend \vspace{1.0ex} \pstart  Si plures circuli aequales ex diversis centris describantur, quo majores erunt, eo minus discrepabunt ab unica figura circulari quia quo erunt radii eorum  majores distantia centrorum, eo plures habebunt \edtext{circuli}{\lemma{}\Afootnote{circuli \textit{ erg.} \textit{ L}}} partes communes. \pend \vspace{5mm}%\begin{wrapfigure}{l}{0.4\textwidth}       
              \begin{center}             
              \begin{tabular}{ccc}
              \includegraphics[width=0.2\textwidth]
%   Zeitz           {images/37_2_9r1}
{images/37_2_9r-1}
              &\rule{10mm}{0mm}&
              \includegraphics[width=0.2\textwidth]
%              {images/37_2_9r2}
 {images/37_2_9r-2}
              \\
              \textit{[Fig. 1]}
              &&
              \textit{[Fig. 2]}
              \end{tabular}
              \end{center}
           \pstart  Itaque tandem ita augeri poterunt circuli, ut sensibiliter desinant in unum praesertim, si non tres tantum  adhibeantur, sed plures. \pend \pstart  Hinc fit ut radius solaris per foramen quodcunque transmissus quo in majore a foramine distantia plano ad ipsum radium recto excipiatur, eo magis ad circulum accedat. Quod experientia clarissimum est, sed ratio non aeque obvia pendet autem ex posita paulo ante propositione Geometrica. Cum enim radius solaris per rimulas, vel hiatus inter arborum folia transmissus in circulum efformatur, id fit quia ipse sol circularis est,