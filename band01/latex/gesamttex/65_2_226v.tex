[226 v\textsuperscript{o}] \selectlanguage{ngerman}ist als die andere. Wenn aber die andere Zahl \edtext{der Tabell}{\lemma{}\Afootnote{der Tabell \textit{ erg.} \textit{ L}}} gr\"{o}ßer w\"{a}re gewesen als die erste, \edtext{w\"{u}rde die}{\lemma{erste,}\Afootnote{ \textit{ (1) }\ m\"{u}ßte man \textit{ (2) }\ w\"{u}rde die \textit{ L}}} pendul Uhr\protect\index{Sachverzeichnis}{Pendeluhr} \edtext{langsamer}{\lemma{Uhr}\Afootnote{ \textit{ (1) }\ zu \textit{ (2) }\ langsamer \textit{ L}}} als die Sonne gehen, und m\"{u}ßte man zu  der von gedachter Pendul-Uhr\protect\index{Sachverzeichnis}{Pendeluhr}  bemerckten Zeit die erwehnte  differenz addiren; dergestalt  w\"{u}rde man die Zeit bekommen, welche die Sonnen \edtext{Uhr \protect\index{Sachverzeichnis}{Sonnenuhr} (wenn anders der Sonnenschein zu sehen w\"{a}re,) anzeigen w\"{u}rde.}{\lemma{Uhr}\Afootnote{ \textit{ (1) }\ bemercken \textit{ (2) }\ (wenn [...] w\"{u}rde. \textit{ L}}} Solte man aber die Sonne wurcklich zu sehen bekommen k\"{o}nnen, und das tr\"{a}ffe nicht ein, so w\"{a}re es ein Zeichen, daß die pendul uhr\protect\index{Sachverzeichnis}{Pendeluhr} umb soviel als es fehlet \edtext{zu geschwind oder zu  langsam gienge; und}{\lemma{fehlet}\Afootnote{ \textit{ (1) }\ unrecht  gienge und \textit{ (2) }\ zu [...] und \textit{ L}}} m\"{u}ße also  an ihr eine correction vorgenommen  werden. Welches auf die von Hugenio