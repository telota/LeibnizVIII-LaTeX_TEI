[72 v\textsuperscript{o}] fundamentum Astronomicum computandi tempus Mundi praesens, sed quod non aeque facile praxi applicetur; et primum quod \textso{solem}\protect\index{Sachverzeichnis}{sol} attinet, \edtext{difficultas objicitur}{\lemma{}\Afootnote{difficultas objicitur \textit{ erg.} \textit{ L}}} \textso{tum} quia solis\protect\index{Sachverzeichnis}{sol} locus in respectu ad alias stellas\protect\index{Sachverzeichnis}{stella} non aeque facile observationibus haberi potest, (nisi quatenus eos radiis suis involvit, item quatenus saepe fit, ut sol\protect\index{Sachverzeichnis}{sol} et luna\protect\index{Sachverzeichnis}{luna} mane, aut vesperi simul videantur, quo casu itidem habemus facilem satis computandi rationem; aut quatenus eclipses nobis monstrant situm utriusque sideris\protect\index{Sachverzeichnis}{sidus}, etsi unum eorum tantum videatur;) \textso{tum} quia solis motus a stellarum fixarum\protect\index{Sachverzeichnis}{stella!fixa} motu diversus, exiguo tempore non redditur sensibilis; cum enim Luna\protect\index{Sachverzeichnis}{luna} una primi mobilis revolutione gradibus retardet tredecim, sol\protect\index{Sachverzeichnis}{sol} praeceditur non nisi uno; quod posterius incommodum est in \textso{caeteris planetis} praeter Lunam\protect\index{Sachverzeichnis}{luna} omnibus (demptis \edtext{certo modo}{\lemma{}\Afootnote{certo modo \textit{ erg.} \textit{ L}}} Jovialibus, ut postea  dicam) et etsi priore careant, id est etsi possint simul cum fixis\protect\index{Sachverzeichnis}{stella!fixa} observari, habent tamen aliud, ut rarius appareant in coelo quam sol\protect\index{Sachverzeichnis}{sol}. Quicquid ejus tamen sit, certum est si plures simul aut diversis exigui inter se intervalli temporibus Planetae observentur, computationem redditum iri certiorem. Et speciatim in \textso{sideribus circumjovialibus} notatum est, earum revolutiones peculiares circa Jovem\protect\index{Sachverzeichnis}{Jovis} esse satis celeres, quarum Ephemerides\protect\index{Sachverzeichnis}{ephemeris}, si ut speramus ad perfectionem deducentur, poterit horum quoque siderum\protect\index{Sachverzeichnis}{sidus} telescopio\protect\index{Sachverzeichnis}{telescopium} observatorum usus esse aliquando ad tempus mundi praesens, seu ipsius revolutionis vel periodi particularis aetatem definiendam, luna\protect\index{Sachverzeichnis}{luna} non apparente; sed rarus eorum conspectus est, nec proinde magnus inde fructus in mari sperandus, etsi ad longitudines\protect\index{Sachverzeichnis}{longitudo} locorum 