\pend \pstart [p.~97] [...] hinc petitur ratio, seu demonstratio illius praxis, qua describitur parabola opera fili, cuius altera extremitas affixa est in A, altera pendulum\protect\index{Sachverzeichnis}{pendulum} sustinet, quod mouetur per rectam AB; hinc facilis ratio ducendae tangentis parabolam;\footnote{\textit{Leibniz unterstreicht}: ratio [...] parabolam} [...] habes igitur omnes radios\protect\index{Sachverzeichnis}{radius} candentes in speculum parabolicum\protect\index{Sachverzeichnis}{speculum!parabolicum} cavum parallelos scilicet, quales supponuntur radij\protect\index{Sachverzeichnis}{radius} a sole profecti, reflecti ac colligi in focum\protect\index{Sachverzeichnis}{focus} A, vbi erit punctum\protect\index{Sachverzeichnis}{punctum!ustorium} vstorium.\footnote{\textit{Am Rand angestrichen}: habes igitur [...] vstorium}\pend \pstart V. Vt autem fiat speculum\protect\index{Sachverzeichnis}{speculum} huiusmodi, voluatur semiparabola BGA, circa axem AG, speculum\protect\index{Sachverzeichnis}{speculum!ustorium} vstorium perfectius dari nequit;\footnote{\textit{Leibniz unterstreicht}: speculum vstorium [...] nequit} cum omnes omnino radij physice loquendo colligantur; dico physice; quia Geometrice secus accidit; tum quia radij non sunt omnes paralleli, tum quia superficies speculi\protect\index{Sachverzeichnis}{speculum} nunquam ita tersa est, quin aliquae salebrae restent; si autem ita apponatur aliud speculum\protect\index{Sachverzeichnis}{speculum!parabolicum} parabolicum, communi foco\protect\index{Sachverzeichnis}{focus}, A, radij ab eo reflexi paralleli erunt, vnde si minimum sit, omnes radios in lineam vrentem, vt vocant, et infinite productam colliget;\footnote{\textit{Leibniz unterstreicht}: si autem [...] colliget} lineam dico physice adaequantem scilicet minoris speculi\protect\index{Sachverzeichnis}{speculum} angustias; [...].