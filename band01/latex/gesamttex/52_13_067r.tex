[67 r\textsuperscript{o}] observatione coelesti, ac proinde aeris marisque injuria independentem, de quo \edtext{alias}{\lemma{}\Afootnote{alias \textit{ erg.} \textit{ L}}} fusius dicendi locus erit.\pend \pstart Problema jam propositum ita solvetur: \textso{Dato }\textso{Horologio}\protect\index{Sachverzeichnis}{horologium}\textso{ exacto, datur }\textso{locus sideris}\protect\index{Sachverzeichnis}{locus sideris}\textso{ in} \textso{mundo, momento dato, }\textso{locus sideris }\protect\index{Sachverzeichnis}{locus sideris}inquam, id est ejus longitudo\protect\index{Sachverzeichnis}{longitudo} et latitudo\protect\index{Sachverzeichnis}{latitudo}: \textso{Latitudo}\protect\index{Sachverzeichnis}{latitudo} quidem si sidus cum polo\protect\index{Sachverzeichnis}{polus} immobili; \textso{Longitudo,}\protect\index{Sachverzeichnis}{longitudo} si cum centro seu cum meridiano\protect\index{Sachverzeichnis}{meridianus} primo in terra ut immobili supposita, ducto comparetur. \textso{Latitudo}\protect\index{Sachverzeichnis}{latitudo} ejus si quidem sit stella \textso{fixa,}\protect\index{Sachverzeichnis}{stella!fixa} semper eadem, ac proinde cognita est. Si \textso{planeta,}\protect\index{Sachverzeichnis}{planeta} determinatur, determinato ejus loco in Zodiaco\protect\index{Sachverzeichnis}{zodiacus} quem nobis Ephemerides\protect\index{Sachverzeichnis}{ephemeris} monstrant, hora, minutove dato. \textso{Longitudo}\protect\index{Sachverzeichnis}{longitudo} sideris\protect\index{Sachverzeichnis}{sidus} cognoscitur hora data, nam si \textso{fixa}\protect\index{Sachverzeichnis}{stella!fixa} est, scimus quantum a Meridiano\protect\index{Sachverzeichnis}{meridianus} primo eam abesse, nunc necesse sit, quia fixarum\protect\index{Sachverzeichnis}{stella!fixa} revolutiones sunt semper aequales, et praecise 24. horis absolvuntur. Si ergo constet nobis, quo in loco tempore quodam cognito, aliqua fuerit fixa\protect\index{Sachverzeichnis}{stella!fixa}, constabit nobis semper quo in loco, tempore quocunque dato eadem fixa\protect\index{Sachverzeichnis}{stella!fixa}, imo alia quaecunque futura sit. Nam ipsarum fixarum\protect\index{Sachverzeichnis}{stella!fixa} motus peculiaris, a motu primi mobilis diversus, cum non nisi post multa saecula sensibilis fiat, in calculum venire nec potest, nec debet.\pend \pstart Sin \textso{Planeta }\protect\index{Sachverzeichnis}{planeta}est, constabit nobis utique per easdem Ephemerides\protect\index{Sachverzeichnis}{ephemeris} quantum nunc absit a meridiano\protect\index{Sachverzeichnis}{meridianus} primo versus ortum, occasumque, nam etsi revolutiones ejus non sint revolutionibus fixarum aequales, differentia tamen nobis cognita est quae supputari potest, potuitque. Sed in eam quidem rem planetis\protect\index{Sachverzeichnis}{planeta} praeter solem\protect\index{Sachverzeichnis}{sol} lunamque\protect\index{Sachverzeichnis}{luna} (quorum exploratus satis cursus est) opus non habemus, quoties enim caeteri planetae\protect\index{Sachverzeichnis}{planeta} videri possunt, poterunt etiam fixae\protect\index{Sachverzeichnis}{stella!fixa} videri, quarum usus certe expeditior.\pend \pstart \textso{Dato porro }\textso{Loco sideris}\protect\index{Sachverzeichnis}{locus sideris}\textso{ praesente (per }\textso{horologium}\protect\index{Sachverzeichnis}{horologium}\textso{ exactum Ephemeridesque)}\protect\index{Sachverzeichnis}{ephemeris}\textso{ dataque (per observatio-}\linebreak\textso{nem) elevatione ejus super} \edtext{\textso{horizontem,}}{\lemma{}\Afootnote{\textso{horizontem,} \textbar\ \textso{versus definitam} \textit{ erg. u.}\ \textit{ gestr.}\ \textbar\ \textit{ L}}}