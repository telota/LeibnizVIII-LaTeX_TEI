 \pstart [93 v\textsuperscript{o}] \textso{Gerick. lib. 5. c. 12 }\edtext{}{\lemma{\textso{12}}\Bfootnote{\textsc{O. v. Guericke, }\cite{00055}a.a.O., S.~168.}} Remissius  movetur aer sub Tropicis, ubi gyratio terrae\protect\index{Sachverzeichnis}{terra} celerior, hinc retardatio ejus,  quia minus capax impetus\protect\index{Sachverzeichnis}{impetus} impressi \edtext{magni}{\lemma{}\Afootnote{magni \textit{ erg.} \textit{ L}}}, quam  aqua aut alia corpora solida, at levioris est magis proportionaliter capax,  credendum et majorem esse adhuc retardationem aeris altioris.
 \pend 
 \pstart 
 \textso{Cap. 18. }\edtext{}{\lemma{\textso{18.}}\Bfootnote{\textsc{O. v. Guericke}, \cite{00055}a.a.O., S.~177.}} Corpora lucida  et luminosa, quo magis removentur,  hoc magis inclarescunt, ut nubes circa  occasum solis\protect\index{Sachverzeichnis}{sol} instar solis\protect\index{Sachverzeichnis}{sol} radiare cernimus,  et si a luna\protect\index{Sachverzeichnis}{luna} longius recederemus, fulgeret  nobis tandem quasi stella, quia omnes colores  ob distantiam evanescunt, luce\protect\index{Sachverzeichnis}{lux} ipsa  evanescente tantum omnium postrema.
 \pend 
 \pstart 
 \textso{Cap. 22.}\edtext{}{\lemma{\textso{22.}}\Bfootnote{\textsc{O. v. Guericke}, \cite{00055}a.a.O., S.~180f.}} Quod ad Lunae\protect\index{Sachverzeichnis}{luna}  distantiam, Parallaxis\protect\index{Sachverzeichnis}{parallaxis} ejus communior est $53\hspace{-6pt}\raisebox{9pt}{,}\protect\hspace{4pt}$.  minutorum hinc sequitur ex Tabula sinuum  distare a terra\protect\index{Sachverzeichnis}{terra} 64 semidiametris terrae\protect\index{Sachverzeichnis}{terra}.  $53\hspace{-6pt}\raisebox{9pt}{,}\protect\hspace{4pt}$ minutorum parallaxis\protect\index{Sachverzeichnis}{parallaxis} observatur cum luna\protect\index{Sachverzeichnis}{luna} est in signis Borealibus,  id est elevatur et ideo minus subjecta refractionibus\protect\index{Sachverzeichnis}{refractio}. Diameter lunae\protect\index{Sachverzeichnis}{luna} apparens ut plurimum $30\hspace{-6pt}\raisebox{9pt}{,}\protect\hspace{4pt}$ min. semidiameter Lunae\protect\index{Sachverzeichnis}{luna} 42 milliarium Germanicorum.
 \pend 
 \pstart 
 \textso{Cap. 25.}\edtext{}{\lemma{\textso{25.}}\Bfootnote{\textsc{O. v. Guericke}, \cite{00055}a.a.O., S.~183.}} Petrus\protect\index{Namensregister}{\textso{Petrus,} der Apostel} in Ep.  priorem coelum et terram\protect\index{Sachverzeichnis}{terra} fuisse ex aqua  et diluvio periisse. Ideo intelligenda in  scriptura sacra coeli voce: aer.
 \pend 
 \pstart 
 Ad Gerickii\protect\index{Namensregister}{\textso{Guericke} (Gerickius, Gerick.), Otto v. 1602\textendash 1686} lib. 5. Appendix de Cometis\protect\index{Sachverzeichnis}{cometa}  sunt Epistolae inter Gerickium\protect\index{Namensregister}{\textso{Guericke} (Gerickius, Gerick.), Otto v. 1602\textendash 1686}  et Stanislaum Lubieniecium de Lubieniez \protect\index{Namensregister}{\textso{Lubienietzki} (Lubieniecius), Stanislaus 1623\textendash 1675} Eq. Polonum, insertae \textit{Theatro Cometico} Amst. 1668. a fol. 453 ad  fol. 465.\edtext{}{\lemma{465.}\Bfootnote{\textsc{S. Lubienietzki, }\cite{00133}\textit{Theatrum cometicum}, Amsterdam 1668, S.~453\textendash465.}}\edtext{}{\lemma{465.}\Bfootnote{\textsc{O. v. Guericke}, \cite{00055}a.a.O., S.~184.}} Hypothesis est Gerickii\protect\index{Namensregister}{\textso{Guericke} (Gerickius, Gerick.), Otto v. 1602\textendash 1686}  de Cometis\protect\index{Sachverzeichnis}{cometa}. Tubos opticos\protect\index{Sachverzeichnis}{tubus!opticus} ostendere nil  esse Cometas\protect\index{Sachverzeichnis}{cometa} quam nubem conglobatam  solaribus radiis illuminatam, tempestas  aliquando avulsa instar pyroboli, superata pondere\protect\index{Sachverzeichnis}{pondus} suo in tertiam usque aeris regionem  eluctatur.\edtext{}{\lemma{eluctatur.}\Bfootnote{\textsc{O. v. Guericke}, \cite{00055}a.a.O., S.~189.}}\edtext{}{\lemma{}\Afootnote{eluctatur.  \textbar\ Ibique \textit{ gestr.}\ \textbar\ \textit{Cauda} \textit{ L}}} \textit{Cauda}\protect\index{Sachverzeichnis}{cauda cometae}\textit{ variat apparentiam secundum solis et oculi dispositionem.}\edtext{}{\lemma{\textit{dispositionem.}}\Bfootnote{\textsc{O. v. Guericke}, \cite{00055}a.a.O., S.~189.}} (+ Haec cum motu difficulter conciliabuntur nisi is rectus seu trajectionis. +) Parallaxis\protect\index{Sachverzeichnis}{parallaxis} hic difficilis, ob motum cometae\protect\index{Sachverzeichnis}{cometa}, \edtext{qui non}{\lemma{cometae,}\Afootnote{ \textit{ (1) }\ qui non \textit{ (2) }\ et ipsius \textit{ (3) }\ qui non \textit{ L}}}  potest constanter in uno loco observari.\edtext{}{\lemma{observari.}\Bfootnote{\textsc{O. v. Guericke}, \cite{00055}a.a.O., S.~189.}}  (+ Imo possunt conferri\edtext{}{\lemma{}\Afootnote{conferri  \textbar\ diversorum \textit{ gestr.}\ \textbar\ momentorum \textit{ L}}} momentorum eorundem observationes. Potest etiam  aliquid colligi ex motu cometae\protect\index{Sachverzeichnis}{cometa} de  motu terrae\protect\index{Sachverzeichnis}{terra} +). Cometae\protect\index{Sachverzeichnis}{cometa} non videntur, ut Luna\protect\index{Sachverzeichnis}{luna} per totum terrarum orbem.  Possunt attolli ad 100 forte miliaria  caeterum non egredi aereas regiones,  quia ipsorum caudae\protect\index{Sachverzeichnis}{cauda cometae} sunt umbrae luminosae seu refractiones\protect\index{Sachverzeichnis}{refractio} radiorum \astrosun\textsuperscript{larium}, nam in aethere\protect\index{Sachverzeichnis}{aether} non essent  caudati, sed criniti seu rosae; quibuscum aliter comparatio est, quique  probabiliter quiescunt etsi ob motum annuum \edtext{terrae}{\lemma{}\Afootnote{terrae \textit{ erg.} \textit{ L}}} moveri videantur, nisi  forte a sole\protect\index{Sachverzeichnis}{sol} attrahuntur, ignis enim  attrahet aerem.\edtext{}{\lemma{aerem.}\Bfootnote{\textsc{O. v. Guericke}, \cite{00055}a.a.O., S.~189f.}} (+ Notandum quod \edtext{princeps Robertus}{\lemma{Robertus}\Bfootnote{Nicht sicher bestimmt, vermutlich Ruprecht von Pfalz-Simmern\protect\index{Namensregister}{\textso{Pfalz-Simmern: Ruprecht} (Prinz Ruppert, princeps Robertus), Pfalzgraf von Pfalz-Simmern 1619\textendash 1682|textit}}} dicebat, ut mihi retulit der domscholaster zu Maynz\protect\index{Ortsregister}{Mainz (Maynz)}\protect\index{Namensregister}{\textso{Karl Heinrich Freih. von Metternich-Winneburg und Beilstein}, Domscholaster zu Mainz 1622\textendash 1679}, in America\protect\index{Ortsregister}{Amerika (America)} quandam  quasi stellam orkani periodici  post aliquot annos redeuntis  praesagam haberi. +)
 \pend 
\pstart  
Gerick. ad lib. 5. append. p. 189.\edtext{}{\lemma{189.}\Bfootnote{\textsc{O. v. Guericke}, \cite{00055}a.a.O., S.~189.}} De homunculo\protect\index{Sachverzeichnis}{homunculus} ita loquitur. \textit{In }\edtext{\textit{oblongo angustoque vasculo vitreo}}{\lemma{\textit{In}}\Afootnote{ \textit{ (1) }\ \textit{vitro} \textit{ (2) }\ \textit{oblongo angustoque vasculo vitreo} \textit{ L}}}\textit{ instar  canalis facto imaguncula quaedam viri  ex ligno artificiose ita facta, ut  aere sustineatur, et ab eo libere moveatur  et digito mutationem aeris ejusque  ponderis pro diverso tempore indicet} si imaguncula solito inferius se  demittat, indicio id est aerem praeter solitum leviorem esse factum, et tum  quoque experientia probat magnas  et horrendas extitisse tempestates. Lubieniecius\protect\index{Namensregister}{\textso{Lubienietzki} (Lubieniecius), Stanislaus 1623\textendash 1675} in respondendo assumit vas esse aeris vacuum,  quod tum Gerickius\protect\index{Namensregister}{\textso{Guericke} (Gerickius, Gerick.), Otto v. 1602\textendash 1686} ei non scri$\uppsi$erat.  Id Gerickius\protect\index{Namensregister}{\textso{Guericke} (Gerickius, Gerick.), Otto v. 1602\textendash 1686} in respondendo  quasi assumit cum ait p. 195\edtext{}{\lemma{195}\Bfootnote{\textsc{O. v. Guericke}, \cite{00055}a.a.O., S.~195.}}  quod attinet homullum ligneum  quem in vase vitreo oblongo \textso{aere vacuo} constitutum (+~forte intellexit: \textso{super,} insistere scilicet  columnae vitreae aere vacuae~+)  eum primo ea intentione non  confeci, plurimum assurgit index  si aer multum aerem attraxit.  Nondum periculum sui virunculi in oceano factum.\pend \pstart \textso{Gerick. lib. 6. cap. 4.}\edtext{}{\lemma{\textso{4.}}\Bfootnote{\textsc{O. v. Guericke}, \cite{00055}a.a.O., S.~201.}} Eos qui fixas circa terram\protect\index{Sachverzeichnis}{terra} agunt cum Mich. Havemanno\protect\index{Namensregister}{\textso{Havemann} (Havemannus), Michael 1597\textendash 1672} comparat,  ei rustico comparat, qui ut Panem asse emtum  domum deduceret, currum commodato sumtum  trahebat.\pend \pstart \textso{Cap. 5.}\edtext{}{\lemma{\textso{5.}}\Bfootnote{\textsc{O. v. Guericke}, \cite{00055}a.a.O., S.~202.}} In spatio aethereo\protect\index{Sachverzeichnis}{spatium!aethereum} non  potest esse motus corporis extra suum centrum, sine  alterius corporis adminiculo, etsi possit mente  fingi.
\pend 