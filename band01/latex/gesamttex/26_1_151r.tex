[151 r\textsuperscript{o}] \edtext{\textso{Conseq. 6.} On pourroit bien expliquer \textso{le phaenomene 9.} ou l'attachement de deux placques dans le vuide, par une liqueur ou matiere fluide dans laquelle on suppose un mouuement en tous sens;}{\lemma{}\Afootnote{ \textit{ (1) }\  \textit{(a)}\  \textit{(b)}\ de la mer sur les plongeurs: dont on ne s'appercoit pas, sinon quand on leur donne \textit{ (2) }\ @@@UNDERLINEON@@@Conseq. [...] expliquer  \textbar\ \textso{le phaenomene 9.} ou \textit{ erg.}\ \textbar\ l'attachement [...] liqueur  \textbar\ ou matiere fluide \textit{ erg.}\ \textbar\ dans [...] sens; \textit{ L}}} dont les vagues frappent les superficies exterieures des placques: \pend \pstart\edtext{  Mais}{\lemma{placques:}\Afootnote{ \textit{ (1) }\ pourveu que ces placques  \textit{(a)}\ ne soient pas fort \textit{(b)}\ soient moins poreuses, que solides car   \textbar\ autrement \textit{ erg.}\ \textbar\  la liqueur passant par les pores frapperoit les interieures aussi. \textit{ (2) }\   Mais \textit{ L}}} on aura de la peine d'expliquer par ce mouuement d'une \edtext{matiere subtile}{\lemma{d'une}\Afootnote{ \textit{ (1) }\ liqueur \textit{ (2) }\ matiere subtile \textit{ L}}} en tous sens le phaenomene de la liqueur purg\'{e}e\protect\index{Sachverzeichnis}{liqueur!purg\'{e}e} d'air. Car le mouuement de cette \edtext{matiere subtile}{\lemma{cette}\Afootnote{ \textit{ (1) }\ liqueur \textit{ (2) }\ matiere subtile \textit{ L}}} continuera \edtext{, même}{\lemma{continuera}\Afootnote{ \textit{ (1) }\ ces \textit{ (2) }\ ses coups, qu \textit{ (3) }\ , même \textit{ L}}} quand il y aura de l'air engendr\'{e} dans la liqueur, et comme il \edtext{est capable de presser la liqueur vers la surface du verre, malgr\'{e} sa pesanteur, il sera aussi capable d'  empecher}{\lemma{il}\Afootnote{ \textit{ (1) }\ a \textit{ (2) }\ suffit de so \textit{ (3) }\   \textbar\ a \textit{ erg.} \textit{ Hrsg. }\  est\'{e} assez fort \`{a} soûtenir les liqueurs, il sera aussi assez fort \`{a} empe \textit{ (4) }\ est [...] aussi \textit{(a)}\ assez fort pour \textit{(b)}\ capable d'  empecher \textit{ L}}} qu'une petite bulle d'air se mette entre deux, et se \footnote{\edtext{Comme nous}{\lemma{d'air.}\Afootnote{ \textit{ (1) }\ La masse \textit{ (2) }\ Il restera pourtant cette difficult\'{e} \textit{ (3) }\ Comme nous \textit{ L}}} l'experimentons dans l'air, dont le mouuement n'est pas en tous sens,   \textbar\ et [...] effect \textit{ erg.}\ \textbar\  quoque l'effort soit en tous sens, que si l'on explique le mouuement en tous sens de cette fa\c{c}on   \textbar\ par un simple effort \textit{ erg.}\ \textbar\ , l'approuue entierement, et je m'en servira, par apres moy même. Mais je crois de n'avoir pas besoin d'un autre que celuy de l'air, dont nous sommes persuadez partout d'experiences, sans \edtext{employer une}{\lemma{nous}\Afootnote{ \textit{ (1) }\ avoir besoin d'un \textit{ (2) }\ employer une \textit{ L}}}matiere purement suppos\'{e}e, qui passe par les pores du verre.\edtext{Je crois}{\lemma{une}\Afootnote{ \textit{ (1) }\ Mais si \textit{ (2) }\ Je crois \textit{ L}}} même que l'Hypothese \edtext{du mouuement en}{\lemma{crois}\Afootnote{ \textit{ (1) }\ de ceux qui supposeront  \textit{(a)}\ une matiere subtile\protect\index{Sachverzeichnis}{mati\`{e}re!subtile|textit} m\"{u} \textit{(b)}\ le mouuement en \textit{(c)}\ que le mouuement \textit{ (2) }\ du mouuement en \textit{ L}}} tous sens de la \edtext{matiere lequel passant}{\lemma{en}\Afootnote{ \textit{ (1) }\ liqueur qui passe \textit{ (2) }\ matiere lequel passant \textit{ L}}} les pores du verre (pour y remplir la place quand on tire l'air) renferm\'{e} dans la petite bulle soit capable d'\'{e}galer \edtext{tous les autres}{\lemma{passant}\Afootnote{ \textit{ (1) }\ le reste \textit{ (2) }\ tous les autres \textit{ L}}} coups \edtext{de la même matiere, se combatte}{\lemma{autres}\Afootnote{ \textit{ (1) }\ que la même liqueur recoit par dehors, et qui la pressent vers la superficie interieure du verre \textit{ (2) }\ de [...] combatte \textit{ L}}}elle même car s'il y a des pores, \edtext{le dit mouuement en tous sens, passant par le verre fera tomber la liqueur purg\'{e}e qui est suspend\"{u}e dans le tuyau, \`{a} cause que la liqueur suspend\"{u}e est press\'{e}e}{\lemma{combatte}\Afootnote{ \textit{ (1) }\ la matiere pressante pressera \textit{ (2) }\ le [...] press\'{e}e \textit{ L}}} de deux costez comme cela arrive, quand on donne l'\edtext{entr\'{e}e \`{a} l'air per\c{c}ant en haut le tuyau}{\lemma{press\'{e}e}\Afootnote{ \textit{ (1) }\ ouuerture \textit{ (2) }\ entr\'{e}e [...] tuyau \textit{ L}}} de Torricelli\protect\index{Namensregister}{\textso{Torricelli} (Torricellius), Evangelista 1608\textendash 1647}.}dilate comme nous voyons qu'elle fait. Et il ne suffit pas de dire, que cette matiere subtile, trouuant de la place dans la bulle, frappe \edtext{ainsi}{\lemma{frappe}\Afootnote{ \textit{ (1) }\ aussi \textit{ (2) }\ ainsi \textit{ L}}} la liqueur suspend\"{u}e de deux costez, et la repousse \edtext{autant}{\lemma{repousse}\Afootnote{ \textit{ (1) }\ ainsi \textit{ (2) }\ autant \textit{ L}}} qu'il l'a pouss\'{e} vers le verre. Car sans insister sur ce que même \edtext{cette pression empechera}{\lemma{même}\Afootnote{ \textit{ (1) }\ la bulle n'aura pas le pouu \textit{ (2) }\ cette pression empechera \textit{ L}}} la generation de la bulle, et surtout, qu'elle ne suffrira pas que la bulle se place entre la liqueur et la surface interieure du verre; il faut considerer que le peu de coups \edtext{du mouuement en tous sens de la matiere subtile insinu\'{e}e}{\lemma{}\Afootnote{BITTE UEBERPRUEFEN!!! du [...] insinu\'{e}e \textit{ erg.} \textit{ L}}} dans une petite bulle, ne peut pas \'{e}galer \edtext{ny d\'{e}truire}{\lemma{}\Afootnote{ny d\'{e}truire \textit{ erg.} \textit{ L}}} tous les autres que la liqueur recoit de tous costez, et par lesquels elle est pouss\'{e}e vers \edtext{la surface interieure du verre}{\lemma{vers}\Afootnote{ \textit{ (1) }\ le verre \textit{ (2) }\ la surface interieure du verre \textit{ L}}}. \edtext{Et il faut remarquer, qu'il y a en cela une grande difference, entre la pression universelle d'une Masse, comme est celle de l'Atmosphere, et entre la pression du mouuement d'une liqueur en tous sens. Car la pression universelle est \'{e}galle, quoyqu'elle trouue seulement un petit passage, comme nous voyons, que le Mercure suspendu dans le tuyau de Torricelli tombe si l'on perce le haut du tuyau avec une \'{e}pingle. Parce que la Masse a un effort general de partager \'{e}galement les forces partout. Mais le mouuement d'une liqueur en tous sens, est particulier \`{a} chaque partie de la masse si ce n'est pas un effort comme celuy de la force elastique,  }{\lemma{}\Afootnote{BITTE UEBERPRUEFEN!!! Et [...] \'{e}galle, \textit{ (1) }\ quoyque l'entr\'{e}e d'elle partout \textit{ (2) }\ quoyqu'elle  \textit{(a)}\ aye seulement une petite ouverture \textit{(b)}\ trouue [...] elastique, \textit{(aa)}\ qui [...] d'un \textit{(aaa)}\ mouuement universel de toute \textit{(bbb)}\ mouvement ou effort universel \textit{(bb)}\ de l'air.  \textit{ erg.} \textit{ L}}}\pend \pstart \textso{Conseq. 7.} Il semble qu'on peut tirer de ces phaenomenes ensemble une \edtext{observation}{\lemma{une}\Afootnote{ \textit{ (1) }\ Regle \textit{ (2) }\ observation \textit{ L}}} generalle, s\c{c}avoir \textso{que la Nature tache d'empecher la discontinuation des corps sensibles.} Car même dans le vuide o\`{u} il n'y a point de corps sensible, \edtext{deux corps solides ne se separent pas}{\lemma{sensible,}\Afootnote{ \textit{ (1) }\ il fau \textit{ (2) }\ en observe qu \textit{ (3) }\ deux corps solides  \textbar\ bien \textit{ erg. u.}\  \textit{ gestr.}\ \textbar\  ne se separent pas \textit{ L}}}, ais\'{e}ment, comme on voit par le phaenomene 9. des placques; \edtext{ny}{\lemma{placques;}\Afootnote{ \textit{ (1) }\ de même \textit{ (2) }\ ny \textit{ L}}} deux liquides, par le phaenomene 10. du siphon \`{a} jambes in\'{e}gales; ny un solide d'un liquide, par les phaenomenes \edtext{5. et 7.}{\lemma{}\Afootnote{5. et 7. \textit{ erg.} \textit{ L}}} de la liqueur purg\'{e}e\protect\index{Sachverzeichnis}{liqueur!purg\'{e}e} d'air.  {weiter mit fol. 151v}