[128 v\textsuperscript{o}] dans le Recipient, bien qu'\'{e}puis\'{e}; de même le Mercure purg\'{e}\protect\index{Sachverzeichnis}{mercure!purg\'{e}} ne tombe pas dans l'air libre, quoyque il soit plus pesant, et d'une hauteur plus grande, qu'\`{a} l'ordinaire, s\c{c}avoir de 70 pouces et \edtext{apparemment, d'avantage au lieu de 27., quoyque on ne s\c{c}ache pas encor les bornes de la force suspendante;}{\lemma{et}\Afootnote{ \textit{ (1) }\ (apparemment, quoyque on ne s\c{c}ache pas encor les bornes de cette force suspendante) d'avantage; au lieu de 27. \textit{ (2) }\ apparemment, [...] suspendante; \textit{ L}}}\pend 
\pstart \textso{Phenom. 8.} on a observ\'{e}, \edtext{qu'il faut un choc plus fort}{\lemma{observ\'{e},}\Afootnote{ \textit{ (1) }\ que \textit{ (2) }\ qu'il [...] fort \textit{ L}}} pour d\'{e}tacher la liqueur purg\'{e}e\protect\index{Sachverzeichnis}{liqueur!purg\'{e}e}, suspend\"{u}e, ou laiss\'{e}e long temps en repos, \edtext{\`{a}}{\lemma{\`{a}}\Afootnote{ \textit{ erg.} \textit{ L}}} un même endroit du Tuyau dans le Vuide\protect\index{Sachverzeichnis}{vide}.\pend \pstart \textso{Phenomen. 9.} \edtext{La 1. et 2. experience}{\lemma{\textso{9.}}\Afootnote{ \textit{ (1) }\ Le 1. et 2. Phenomene \textit{ (2) }\ La 1. et 2. experience \textit{ L}}} avoient fait croire aux philosophes de nostre temps, que \edtext{ce phenomene de}{\lemma{ce}\Afootnote{ phenomene de \textit{ erg.} \textit{ L}}} deux placques bien unies, \edtext{qui}{\lemma{}\Afootnote{qui \textit{ erg.} \textit{ L}}} ne se separent pas \edtext{aisement}{\lemma{}\Afootnote{aisement \textit{ erg.} \textit{ L}}}, (si non \edtext{quand l'une peut glisser}{\lemma{non}\Afootnote{ \textit{ (1) }\ en faisant glisser l'une \textit{ (2) }\ quand l'une peut glisser \textit{ L}}} sur l'autre) \edtext{provenoit seulement}{\lemma{l'autre)}\Afootnote{ \textit{ (1) }\ a aussi \textit{ (2) }\ provenoit seulement \textit{ L}}} de la pression de l'air\protect\index{Sachverzeichnis}{pression de l'air}; mais \`{a} present on a \'{e}prouu\'{e}, contre toute leur attente, \edtext{que cela}{\lemma{que}\Afootnote{ \textit{ (1) }\ le même attachement \textit{ (2) }\ cela \textit{ L}}} se rencontre aussi dans le Vuide\protect\index{Sachverzeichnis}{vide} ou Recipient \'{e}puis\'{e}.\pend 
\pstart \textso{Phenom. 10.} Le même sentiment mis en avant par \textso{Torricelli}\protect\index{Namensregister}{\textso{Torricelli} (Torricellius), Evangelista 1608\textendash 1647}, \edtext{auteur en partie de l'experience}{\lemma{auteur}\Afootnote{ \textit{ (1) }\ du phenomene \textit{ (2) }\ en partie de l'experience \textit{ L}}} 2\textsuperscript{me}, et confirm\'{e} par \textso{Mons. Pascal}\protect\index{Namensregister}{\textso{Pascal} (Pascalius), Blaise 1623\textendash 1662}, passoit pour incontestable \`{a} l'\'{e}gard de tous les autres phenomenes, que les anciens attribuoient \`{a} l'horreur du Vuide\protect\index{Sachverzeichnis}{vide}; et neantmoins celuy du siphon\protect\index{Sachverzeichnis}{siphon} \edtext{\`{a} jambes inegales}{\lemma{siphon}\Afootnote{ \textit{ (1) }\   \textbar\ renvers\'{e} \textit{ erg.}\ \textbar\  \`{a} jambes inegales de \textit{ (2) }\ \`{a} jambes inegales \textit{ L}}} reussit aussi dans le Vuide\protect\index{Sachverzeichnis}{vide} et fit couler l'eau purg\'{e}e d'un \edtext{vase dans lequel la jambe la plus courte trempoit; le bout de la longue hors du vase estant au dessous du niveau de l'eau du vase}{\lemma{d'un}\Afootnote{ \textit{ (1) }\ vaisseau plein dans lequel la jambe la plus courte trempoit; le bout de la longue hors du vaisseau estant au dessous du niveau de l'eau dans le  \textit{(a)}\ vaisseau \textit{(b)}\ vase \textit{ (2) }\ vase [...] vase \textit{ L}}}.\pend \pstart Ces phenomenes m\'{e}ritent assur\'{e}ment d'estre bien consider\'{e}s; le 2\textsuperscript{me} et 3\textsuperscript{me} ayant \edtext{monstr\'{e}}{\lemma{ayant}\Afootnote{ \textit{ (1) }\ d\'{e}couuert \textit{ (2) }\ monstr\'{e} \textit{ L}}} la pression de l'atmosphere\protect\index{Sachverzeichnis}{atmosph\`{e}re}, auparavant inconn\"{u}e; et les autres nous menants \`{a} la connoissance experimentelle d'une Matiere ou au moins d'une action ou pression toute nouuelle; et comme dans un nouueau monde\protect\index{Sachverzeichnis}{monde} subtil, dont le nostre grossier, est en quelque fa\c{c}on entrelass\'{e} partout.\pend \pstart Outre que l'union des corps sensibles\protect\index{Sachverzeichnis}{corps!sensible}, et peut estre particulierement celles des liqueurs\protect\index{Sachverzeichnis}{liqueur} gel\'{e}es, avec le vase, et dans leur parties, \edtext{vient aussi apparemment, en partie}{\lemma{parties,}\Afootnote{ \textit{ (1) }\ semble  \textit{(a)}\ dependre \textit{(b)}\ aussi dependre en quelque fa\c{c}on \textit{ (2) }\ vient aussi apparemment, en partie \textit{ L}}} de la raison\edtext{}{\lemma{}\Afootnote{raison  \textbar\ cach\'{e}e \textit{ gestr.}\ \textbar\ de \textit{ L}}} de nos phenomenes.\pend \pstart Laquelle n'estant pas encor assez d\'{e}couuerte ou \'{e}tablie, m'a convi\'{e} \`{a} une recherche si importante: j'ay consider\'{e} toutes les hypotheses qui sont parven\"{u}es, \`{a} ma connoissance, j'ay \edtext{propos\'{e}}{\lemma{j'ay}\Afootnote{ \textit{ (1) }\ projett\'{e} \textit{ (2) }\ propos\'{e} \textit{ L}}} \textso{des experiences \`{a} faire}, \edtext{de l'iss\"{u}e des\-quelles semble dependre le destin}{\lemma{\textso{faire},}\Afootnote{ \textit{ (1) }\ du destin \textit{ (2) }\ desquelles  \textit{(a)}\ semble le destin depend \textit{(b)}\ nous pouuons le destin \textit{ (3) }\ de [...] destin \textit{ L}}} de ces hyotheses. Enfin j'ay produit ma \edtext{pens\'{e}e laquelle}{\lemma{pens\'{e}e}\Afootnote{ \textit{ (1) }\ qui me semble \textit{ (2) }\ \`{a} l'a \textit{ (3) }\ laquelle \textit{ L}}}, si elle n'est pas \edtext{asseur\'{e}e}{\lemma{pas}\Afootnote{ \textit{ (1) }\ demonstr\'{e}e \textit{ (2) }\ asseur\'{e}e \textit{ L}}}; est au moins \`{a} l'\'{e}preuue de ces \'{e}cueils, qui peuuent faire \'{e}cho\"{u}er les autres, \edtext{et peut subsister, sans se mettre en compromis,}{\lemma{}\Afootnote{et [...] compromis, \textit{ erg.} \textit{ L}}} quelque puisse estre l'evenement des experiences projett\'{e}es.\footnote{\textit{Unter dem Text rechts}: Recherche}\pend 