   
        
        \begin{ledgroupsized}[r]{120mm}
        \footnotesize 
        \pstart        
        \noindent\textbf{\"{U}berlieferung:}  
        \pend
        \end{ledgroupsized}
      
       
              \begin{ledgroupsized}[r]{114mm}
              \footnotesize 
              \pstart \parindent -6mm
              \makebox[6mm][l]{\textit{L}}Konzept: LH XXXV 15, 6 Bl. 49\textendash50. 1 Bog. 2\textsuperscript{o}. 4 S. zweispaltig. Linke Spalte fortlaufender Text, rechte Spalte Erg\"{a}nzungen und Referenzen.\\KK 1, Nr. 193 D \pend
              \end{ledgroupsized}
        \vspace*{8mm}
        \pstart 
        \normalsize
      \begin{center}[49 r\textsuperscript{o}] \edlabel{3start}\edtext{Longit. 3.}{\lemma{}\Afootnote{Longit. 3. \textit{ erg.} \textit{ L}}}\end{center}
      \pend 
      \vspace{1.0ex} 
      \pstart \edtext{Non}{\lemma{3.}\xxref{3start}{3end}\Afootnote{ \textit{ (1) }\ Omnis \textit{ (2) }\ Non \textit{ L}}} potest generari circulus, nec motus circularis, nisi \edlabel{3end}per \edtext{ductum lineae}{\lemma{per}\Afootnote{ \textit{ (1) }\ rei \textit{ (2) }\ ductum lineae \textit{ L}}} circa centrum immotum. Ideo nec equorum in gyrum circumeuntium motus, aut navium\protect\index{Sachverzeichnis}{navis} aliorumve \edtext{}{\lemma{}\Afootnote{aliorumve \textbar\ non \textit{ gestr.}\ \textbar\ est \textit{ L}}}est circularis, sed rectus constans multis angulis. Ita ut aliquandiu mota in lineam rectam res, quiescat et ita se contorqueat circa centrum paulum ut angulum faciat ad priorem lineam, deinde pergit porro moveri, et ea ratione non potest objici contra versoriam nostram, quia nave se flectente circulari motu eccentrico, declinatio\protect\index{Sachverzeichnis}{declinatio} observari non possit.
      \pend 
      \pstart Ut instrumenta sint libera et tamen non pendula modum habeo. Libera esse debent \edtext{ut disponant}{\lemma{debent}\Afootnote{ \textit{ (1) }\ ne fluctuent \textit{ (2) }\ ut disponant \textit{ L}}} se semper paralleliter ad horizontem, firma et minime pendula, \edtext{ne}{\lemma{pendula,}\Afootnote{ \textit{ (1) }\ ut \textit{ (2) }\ ne \textit{ L}}} fluctuatione situm debitum perturbent, id ita fiet. Sit globus ex quo firmus axis \textit{bc} descendat. In puncto \textit{c}\edtext{}{\lemma{puncto \textit{c}}\Bfootnote{Vgl. \textit{[Fig. 3]}, S.~\pageref{fig3}.}}
      lineae \textit{ik} sit foramen rotundum paulo minus globo. Huic superimponatur, ita poterit se flectere \edtext{paralleliter}{\lemma{flectere}\Afootnote{ \textit{ (1) }\ quolibet \textit{ (2) }\  paralleliter \textit{ L}}} ad horizontem nave\protect\index{Sachverzeichnis}{navis} inclinata ventis. Globus iste debet esse tanti ponderis, quanti est globus \textit{b} cum omnibus ab eo pendentibus et ipsa tabula, ne forte ipse ab iis in contrario nisu flectatur. Videtur et firmari superius posse, ut quomodocunque non tamen circa proprium axem flectatur, idque fiet si axis ejus sit immobilis funi alligatus, vel ipse in loco axis (nempe si \textit{ac} continuaretur) ansa aliqua in fune sit firmatus sed de hoc posteriore dubito. Sed nihil ad rem. Si verum est inventum\label{inventum} P. Grandamici\protect\index{Namensregister}{\textso{Grandami} (Grandamicus), Jacques SJ 1588\textendash 1672}, quod examinatum a se et verum deprehensum asserunt P. Zucchius\protect\index{Namensregister}{\textso{Zucchi} (Zucchius), Niccol\`{o} SJ 1586\textendash 1670}, P. Kircherus \protect\index{Namensregister}{\textso{Kircher} (Kircherus), Athanasius SJ 1602\textendash 1680} et P. Schottus\protect\index{Namensregister}{\textso{Schott} (Schottus), Caspar SJ 1608\textendash 1666}\edtext{}{\lemma{P. Schottus}\Bfootnote{\textsc{C. Schott}, \cite{00094}\textit{Magia universalis}, Frankfurt 1659, S.~334. Der Hinweis auf Grandami, Zucchi und Kircher bei Schott.}}, declinationibus\protect\index{Sachverzeichnis}{declinatio} Magneticis adhibitum est remedium, et haberi potest perpetuo linea meridiana. Porro hoc supposito, jam inventum meum etiam per magnetem\protect\index{Sachverzeichnis}{magnes} exhiberi potest, si per eum tabula circumagatur, etsi enim ob loxodromias\protect\index{Sachverzeichnis}{loxodromia} canones linea quae designatur \edtext{magnete\protect\index{Sachverzeichnis}{magnes}}{\lemma{}\Afootnote{magnete\protect\index{Sachverzeichnis}{magnes} \textit{ erg.} \textit{ L}}} non est vera, potest tamen ex calculo perfecte corrigi. Cum sciamus semper ubi sumus; servandae igitur regulae loxodromicae\protect\index{Sachverzeichnis}{regula loxodromica}, et constituenda Tabula, \edtext{cujus ope}{\lemma{Tabula,}\Afootnote{ \textit{ (1) }\ in qua omn \textit{ (2) }\ cujus ope \textit{ L}}} continue nautae differentias habere possint verticalis locorum dati et quaesiti a Loxodromia\protect\index{Sachverzeichnis}{loxodromia}, quae tanto major erit, quanto obliquior est loxodromia. Cum ergo sciant initio praecise ubi sint \edtext{[sequentes]}{\lemma{sequentem}\Afootnote{\textit{\ L \"{a}ndert Hrsg.}}} primam Loxodromiam\protect\index{Sachverzeichnis}{loxodromia} certo tempore corrigentur tot \edtext{gradus}{\lemma{tot}\Afootnote{ \textit{ (1) }\ arcu \textit{ (2) }\ gradus \textit{ L}}} reflectentes quot \edtext{differt angulus}{\lemma{quot}\Afootnote{ \textit{ (1) }\ declinat jam angu \textit{ (2) }\ differt angulus \textit{ L}}} loxodromiae\protect\index{Sachverzeichnis}{loxodromia} ad meridianum\protect\index{Sachverzeichnis}{meridianus} \edtext{loci}{\lemma{}\Afootnote{loci \textit{ erg.} \textit{ L}}} in quo sumus ab angulo loxodromiae\protect\index{Sachverzeichnis}{loxodromia} ad meridianum\protect\index{Sachverzeichnis}{meridianus} loci eundem. Et pro eodem constituentur tabulae ut liceat perpetuo rectam magnetis\protect\index{Sachverzeichnis}{magnes} lineam mutare in curvam et contra. Et hac ratione tabulae Loxodromicae\protect\index{Sachverzeichnis}{tabula loxodromica} Stevini\protect\index{Namensregister}{\textso{Stevin} (Stevinus), Simon 1548\textendash 1620}\edtext{}{\lemma{ratione}\Bfootnote{\textsc{S. Stevin, }\cite{00099}\textit{La Cosmographie}, Leiden 1634, S.~150\textendash160. }} et Herigoni\protect\index{Namensregister}{\textso{H\'{e}rigone} (Herigonus), Pierre 1580\textendash 1643}\edtext{}{\lemma{et}\Bfootnote{\textsc{P. H\'{e}rigone, }\cite{00057}\textit{La doctrine de la sphere du monde}, Paris 1644, S.~426\textendash 450.}} perficiendae et ad istud institutum applicandae sunt, quod non difficulter fiet. Dummodo quoties gradus differentia est, toties gradu uno reflectantur. 
      [49 v\textsuperscript{o}] Commodius ad praxin res in globo poterit determinari, ubi semper apparet in horizonte \edtext{quem angulum}{\lemma{horizonte}\Afootnote{ \textit{ (1) }\ quot \textit{(a)}\ angulos \textit{(b)}\ gradus \textit{ (2) }\ quem angulum \textit{ L}}} faciat Loxodromia\protect\index{Sachverzeichnis}{loxodromia} ad meridianum\protect\index{Sachverzeichnis}{meridianus}. Facit ac semper eundem cum priore. Et ita eundem cum plaga initiali navigationis. Sed verticalis locorum relicti et quaesiti mutat semper angulos ad meridianos\protect\index{Sachverzeichnis}{meridianus} locorum navigationis. Et hi perfecte possunt ex sphaera determinari. Dummodo in ea \edtext{sint 360}{\lemma{ea}\Afootnote{ \textit{ (1) }\ tot \textit{ (2) }\ sint 360 \textit{ L}}} meridiani\protect\index{Sachverzeichnis}{meridianus}. Sed cum eae non sint, Res calculo ex canone Triangulorum efficietur. Ita perfecte res \edtext{haberi, nec errari}{\lemma{haberi,}\Afootnote{ \textit{ (1) }\ et narrari \textit{ (2) }\ nec errari \textit{ L}}} ullo pacto potest, et modo verum sit inventum P. Grandamici\protect\index{Namensregister}{\textso{Grandami} (Grandamicus), Jacques SJ 1588\textendash 1672}, quod P. Schott\protect\index{Namensregister}{\textso{Schott} (Schottus), Caspar SJ 1608\textendash 1666} depraedicat \cite{00093}\textit{Curs. Mathem.} lib. 13. sub finem part. 1. fol. 384.\edtext{}{\lemma{fol. 384.}\Bfootnote{\textsc{C. Schott, }\cite{00093}\textit{Cursus mathematicus}, W\"{u}rzburg 1661, S.~384. }} Et describit. \cite{00094}\textit{Mag. Nat.} part. 4. lib. 3. synt. ult. pragmatia 1. pag. 334.\edtext{}{\lemma{pag. 334.}\Bfootnote{\textsc{C. Schott, }\cite{00094}\textit{Magia universalis}, S.~334. Hier der Hinweis auf Grandami.}} Cursus navis\protect\index{Sachverzeichnis}{navis} extra Meridianum\protect\index{Sachverzeichnis}{ meridianus}, \edtext{Aequatorem\protect\index{Sachverzeichnis}{aequator} et ejus parallelos}{\lemma{Meridianum,}\Afootnote{ \textit{ (1) }\ parallelum\protect\index{Sachverzeichnis}{circulus parallelus|textit} et \textit{ (2) }\ Aequatorem et ejus parallelos \textit{ L}}} est compositus ex variarum Loxodromiarum\protect\index{Sachverzeichnis}{loxodromia} particulis. Si loca sita sunt in eodem meridiano\protect\index{Sachverzeichnis}{meridianus} dirigatur navis\protect\index{Sachverzeichnis}{navis} perpetuo in septentrionem vel austrum, et rhombus seu via navis\protect\index{Sachverzeichnis}{navis} erit meridianus\protect\index{Sachverzeichnis}{meridianus} loci. \edtext{}{\lemma{}\Afootnote{loci. \textbar\ Si \textit{streicht Hrsg.}\ \textbar\ \textit{ L}}}
\pend 