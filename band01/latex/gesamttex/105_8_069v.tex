[69 v\textsuperscript{o}] fixae\protect\index{Sachverzeichnis}{stella!fixa} (id est saltem una earum, nam una bene posita, caeterae omnes bene sunt positae) in vera  quam tunc habent a meridiano\protect\index{Sachverzeichnis}{meridianus} primo distantia initio sunt constitutae, veram distantiam  supposita horologii\protect\index{Sachverzeichnis}{horologium} exactitudine in ipsa sphaera artificiali semper monstrabunt.\pend \pstart Sed hoc curiosum magis et elegans, quam utile fore existimo pro fixarum\protect\index{Sachverzeichnis}{stella!fixa} quidem usu, calculi tam facilis qui intra horae minutum  nullo negotio fieri possit evitandi causa  machinam adhibere velle. At habebit fortasse usum cum Luminaribus observatis opus erit.\pend \pstart Quod si ergo sole\protect\index{Sachverzeichnis}{sol} Lunave\protect\index{Sachverzeichnis}{luna} (quia interdiu sidera\protect\index{Sachverzeichnis}{sidus} \edtext{non lucent; noctu aliquando, Luna\protect\index{Sachverzeichnis}{luna} lucente, caetera sidera}{\lemma{}\Afootnote{BITTE UEBERPRUEFEN!!! non lucent; noctu aliquando, Luna\protect\index{Sachverzeichnis}{luna} lucente, caetera sidera \textit{ erg.} \textit{ L}}} non lucent) uti velimus, similiter nulla alia re opus est, quam ut praesentem locum horum planetarum in Zodiaco\protect\index{Sachverzeichnis}{zodiacus}, seu latitudinem\protect\index{Sachverzeichnis}{latitudo} et  distantiam a meridiano\protect\index{Sachverzeichnis}{meridianus} primo seu longitudinem \protect\index{Sachverzeichnis}{longitudo} sciamus.\pend \pstart Hoc autem ex Ephemeridibus \protect\index{Sachverzeichnis}{ephemeris} facile sciri posse constat, aut certe Ephemerides\protect\index{Sachverzeichnis}{ephemeris} solis\protect\index{Sachverzeichnis}{sol} Lunaeve\protect\index{Sachverzeichnis}{luna} ita facile accommodari possunt ut hoc monstrent.\pend \pstart Quodsi Ephemeridibus\protect\index{Sachverzeichnis}{ephemeris} carere velis sufficiet machinam horologii \protect\index{Sachverzeichnis}{horologium} ipsi sphaerae artificiali, eo quem supra dixi modo applicari, et duo luminaria in eadem machina Zodiacum\protect\index{Sachverzeichnis}{zodiacus} suis quodque spatiis percurrere, uti jam tum ab Automato\textendash poeo quodam Germano procuratum vidi. Ita Machina semel recte constituta, latitudinem\protect\index{Sachverzeichnis}{latitudo}, longitudinemque\protect\index{Sachverzeichnis}{longitudo} solis\protect\index{Sachverzeichnis}{sol} et Lunae\protect\index{Sachverzeichnis}{luna}, et tandem unica elevationis eorum ultra horizontem loci navis\protect\index{Sachverzeichnis}{navis}, observatione ac per consequens Horizontis meridianique\protect\index{Sachverzeichnis}{meridianus} loci in sphaera artificiali simili tum ad aequatorem\protect\index{Sachverzeichnis}{aequator}, tum ad locum  lunae\protect\index{Sachverzeichnis}{luna} solisve\protect\index{Sachverzeichnis}{sol} accommodatione accedente, distantiam Meridiani\protect\index{Sachverzeichnis}{meridianus} loci navis\protect\index{Sachverzeichnis}{navis}, a Meridiano\protect\index{Sachverzeichnis}{meridianus} primo, seu longitudinem\protect\index{Sachverzeichnis}{longitudo} loci navis\protect\index{Sachverzeichnis}{navis}, vel [latitudine\protect\index{Sachverzeichnis}{latitudo}  quippe jam praecognita]\edtext{}{\lemma{praecognita]}\Bfootnote{Klammern von Leibniz}} locum navis\protect\index{Sachverzeichnis}{navis} praecise, definiri posse manifestum est. \pend \pstart Habes ergo rationem \textso{Mechanicam} facilem \edtext{et\edtext{}{\lemma{et}\Bfootnote{Unterstreichung Leibniz}}}{\lemma{}\Afootnote{et \textit{ erg.} \textit{ L}}} ad usum sufficientem dato horologio\protect\index{Sachverzeichnis}{horologium} exacto cognitatur loci navis\protect\index{Sachverzeichnis}{navis} latitudine\protect\index{Sachverzeichnis}{latitudo}; unica, eaque propemodum indefinita ac rarissime per plures dies continuos defutura observatione, Longitudines\protect\index{Sachverzeichnis}{longitudo} seu locum navis\protect\index{Sachverzeichnis}{navis} verum praecise inveniendi. Unde \edtext{\textso{Geometrica}}{\lemma{Unde}\Afootnote{ \textit{ (1) }\ Geometrica  \textit{ (2) }\ \textso{Geometrica} \textit{ L}}} per calculum Trigonometricum definiendi ratio ignorari non potest, quam prolixe exponere, hujus loci non est.\pend \pstart 