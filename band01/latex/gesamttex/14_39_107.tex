\pend \pstart[p.~107] VIII. Dato angulo refractionis\protect\index{Sachverzeichnis}{angulus!refractionis}, radio incidente ex  raro in densum, dari potest, incidente ex denso in rarum; si enim vt Keplerus\protect\index{Namensregister}{\textso{Kepler} (Keplerus), Johannes 1571\textendash 1630} ait, supra arcum 30. angulus\protect\index{Sachverzeichnis}{angulus!refractionis}\footnote{\textit{Am Rand angestrichen}: Dato angulo [...] 30. angulus}  refractionis est $\displaystyle \frac{1}{3}$\rule[-4mm]{0mm}{10mm}. anguli complementi, quando cadit  ex raro in densum, erit $\displaystyle \frac{1}{2}$\rule[-4mm]{0mm}{10mm}. cadente ex denso in rarum\footnote{\textit{Gedruckte Marginalie}: Fig. 84. 2.}  v.g. sit medium densum BCL, centrum A, arcus 30.  BC, sit DCK incidens ex raro in densum, perpendicularis ACE, sint arcus ED, AK, descripti ex  centro C, sit angulus KCI $\displaystyle \frac{1}{3}$\rule[-4mm]{0mm}{10mm}. anguli ACK, aequali  ECD, producatur ICF, certe si IC cadat ex denso in rarum, radius refractus erit CD, vt \textit{supra} dictum est; sed angulus refractionis\protect\index{Sachverzeichnis}{angulus!refractionis} FCD est subduplus  anguli ICA, vel FCE reliqua de refractione\protect\index{Sachverzeichnis}{refractio} \textit{infra}  dicentur. 