      
               
                \begin{ledgroupsized}[r]{120mm}
                \footnotesize 
                \pstart                
                \noindent\textbf{\"{U}berlieferung:}   
                \pend
                \end{ledgroupsized}
            
              
                            \begin{ledgroupsized}[r]{114mm}
                            \footnotesize 
                            \pstart \parindent -6mm
                            \makebox[6mm][l]{\textit{LiH}}Unterstreichungen in \textsc{H. Philippes}, \cite{00002}\textit{The Sea-man's Kalender}, London 1672.\pend
                            \end{ledgroupsized}
                %\normalsize
                \vspace*{5mm}
                \begin{ledgroup}
                \footnotesize 
                \pstart
            \noindent\footnotesize{\textbf{Datierungsgr\"{u}nde}: Die Datierung ergibt sich aus dem Brief vom 13. Mai 1676 an Henry Bond\protect\index{Namensregister}{\textso{Bond,} Henry 1600?\textendash 1678} (\textit{LSB} III, 1 N. 80), in dem Leibniz mitteilt, er habe jetzt auch den von Henricus Philippus\protect\index{Namensregister}{\textso{Philippes} (Philippus), Henry ?\textendash 1677} herausgegebenen \cite{00002}\textit{Sea-man's Kalender} gelesen.}
                \pend
                \end{ledgroup}
            
                \vspace*{8mm}
                \pstart 
                \normalsize
            [p.~103] \selectlanguage{english}And whatsoever many may expect some excellent way for it from Foreign parts, by certain small Stars\protect\index{Sachverzeichnis}{star} near \textit{Jupiter}\protect\index{Sachverzeichnis}{Jupiter}, and that some here at home would have the World conceited of a way by Celestial Observation; yet it is without doubt, the Longitude must be found by Observation made of something below the Moon\protect\index{Sachverzeichnis}{moon}\footnote{\textit{Leibniz unterstreicht}: Moon}: for I do truly affirm, that there are Magnetical Poles\protect\index{Sachverzeichnis}{pole}\footnote{\textit{Leibniz unterstreicht}: Magnetical Poles}, whose Latitude and Longitude I do as certainly know, as concurrent Observations and Arithmetical Calculations can discover them; and their Annual motion\footnote{\textit{Leibniz unterstreicht}: their Annual motion} I know, and by consequence the time of their Revolution\footnote{\textit{Leibniz unterstreicht}: the time of their Revolution}. It may be objected; that the Variation in many, nay in most places, are very irregular, and not according to such Magnetical Poles\protect\index{Sachverzeichnis}{pole} as I speak of; for in some places on the same Parallel in equal spaces, it altereth much swifter than in other\footnote{\textit{Leibniz unterstreicht}: in some places [...] in other}; besides in the Parallel of \textit{London}\protect\index{Ortsregister}{London (Londinum)}, there is 2 Degrees 00 Easterly variation to the Eastwards of \textit{London}\protect\index{Ortsregister}{London (Londinum)}, and 2 Degrees 00 Easterly variation to the Westwards of \textit{London}\protect\index{Ortsregister}{London (Londinum)}; and yet both these places are to the Eastward of the first Meridian of the World, within 45 Degr. 0 Min. of Longitude. It is true; but all this I can  very well resolve, and I doubt not but to do it for all places: Moreover, there are some places\hfill within a\pend \newpage \pstart\noindent certain Longitude, whose variations continue constant for hundreds of Years, and yet afterwards do vary as ours here at \textit{London}\protect\index{Ortsregister}{London (Londinum)} doth now; but at \textit{London}\protect\index{Ortsregister}{London (Londinum)} it is never constant, although in former time the variation of it was not sensible, it is now in its swiftest motion.\footnote{\textit{Leibniz unterstreicht}: there are some [...] swiftest motion}\selectlanguage{latin}\pend 