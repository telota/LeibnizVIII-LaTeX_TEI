[87 r\textsuperscript{o}] esse. Et si in circulo cujus semidiameter sit 12 pedum, praedicta \textit{FB} sumatur aequalis $\displaystyle\frac{49}{1201}\:$\rule[-4mm]{0mm}{10mm}
semidiametri, hoc est, pro diametro aperturae vitri, seu basis Cylindri plus quam mensura $\displaystyle11\frac{1}{4}\:$\rule[-4mm]{0mm}{10mm}
digitorum; quod tum iste radiorum Cylindrus efficiet focum, cujus semidiameter minor erit $\displaystyle\frac{1}{69615}\:$\rule[-4mm]{0mm}{10mm}
duodecim pedum, hoc est, minor quam $\displaystyle\frac{1}{483}\:$
digiti. Unde etiam sequitur, focum hunc pro puncto mechanico\protect\index{Sachverzeichnis}{punctum!mechanicum} habendum esse.\pend \pstart Atque hoc non solummodo locum habet in ipso foco, sed etiam in illa axis longitudine intra quam radii hi incidunt: longitudo enim illa aeque ac focus, ita parva reddi potest; servata tamen pro radiorum transitu magna satis apertura; ut pro puncto mechanico\protect\index{Sachverzeichnis}{punctum!mechanicum} etiam sit habenda. Nam si sumamus ex: gr: vitrum aliquod ex minimis figuram habens circuli, cujus semidiameter aequalis sit mensurae $\displaystyle\frac{1}{8}$\rule[-4mm]{0mm}{10mm} digiti, sitque \textit{FB} aequalis $\displaystyle\frac{5}{13}\:$\rule[-4mm]{0mm}{10mm} octavae partis digiti, erit diameter aperturae $\displaystyle\frac{10}{13}$ ipsius \textit{ND} semidiametri circuli vitri, radiique congregabuntur in ipso axe intra longitudinem $\displaystyle \frac{1}{33}$\rule[-4mm]{0mm}{10mm} octavae digiti partis. Eodem modo, si sumatur vitrum cujus exterior superficies figuram habet Circuli cujus semidiameter sit ut antea 12 pedum, cujus ope fieri 