[94 v\textsuperscript{o}] \edtext{Duriuscule}{\lemma{determinare.}\Afootnote{ \textit{ (1) }\ Iniquius \textit{ (2) }\  Duriuscule \textit{ L}}} mihi ⟨vi⟩detur cum Scholis actum esse, quas  hactenus ob Fugam Vacui\protect\index{Sachverzeichnis}{fuga vacui} suggilavimus; quasi\edtext{}{\lemma{}\Afootnote{quasi  \textbar\ scilicet \textit{ gestr.}\ \textbar\ Naturae \textit{ L}}} Naturae \edtext{appetitum quendam tribuentes}{\lemma{Naturae}\Afootnote{ \textit{ (1) }\ habeat \textit{ (2) }\ appetitum quendam tribuentes \textit{ L}}}. Illud enim ni fallor  dicere voluere: talia phaenomena evenire, quia Vacuum  (acervatum) non possit evenire in natura. Quidni  autem? Quia omnia jam sint plena, ac proinde non possit  evenire Vacuum, nisi aliquod corpus \edtext{annihiletur}{\lemma{corpus}\Afootnote{ \textit{ (1) }\ destruatur \textit{ (2) }\ annihiletur \textit{ L}}},  aut in alterius locum subeat seu dimensiones ejus penetret.  Hinc corpus aliquod ab alio copore aut aliquo  loco divelli non posse, nisi aliud corpus possit in ejus locum  subire. Hactenus illi recte ratiocinati sunt. Sed in eo male,  quod his phaenomenis probari posse credidere nullum absolute  vacuum dari posse; cum tantum \edtext{vacuo sensibili apud nos obsistere naturam hinc probetur}{\lemma{tantum}\Afootnote{ \textit{ (1) }\ posse dari vacuum sensibile\protect\index{Sachverzeichnis}{vacuum!sensibilis|textit} \textit{ (2) }\ vacuo [...] probetur \textit{ L}}}.  Sed qui fit ergo, inquies, quod ille contra vacuum  nisus naturae tandem vincitur, quod Antliae\protect\index{Sachverzeichnis}{antlia} vires in nimia  altitudine deficiunt, quod ponderibus appensis duae laminae\protect\index{Sachverzeichnis}{laminae politae}  tandem divelluntur. Hujus rei ratio est, quia \edtext{aucta vi}{\lemma{aucta}\Afootnote{ \textit{ (1) }\ pressione \textit{ (2) }\  vi \textit{ L}}} \edtext{tandem necesse est aut omnia rumpi, incurvari, porosve aperiri, aut materiam  circumjacentem attenuari in summam subtilitatem}{\lemma{tandem}\Afootnote{ \textit{ (1) }\ corpora circumjacentia \textit{ (2) }\ corporum circumjacentium pars in \textit{ (3) }\  necesse est aut omnia rumpi,   \textbar\ incurvari, \textit{ erg.}\ \textbar\ porosve [...] subtilitatem \textit{ L}}}  aut plurimum materiae subtilis\protect\index{Sachverzeichnis}{materia!subtilis}, ut aeris  attenuari, aut etiam aetheris\protect\index{Sachverzeichnis}{aether} in unum congregari,  ad poros pervadendos locumque implendum.  At aer attenuationi suae, aether\protect\index{Sachverzeichnis}{aether} si quis est,  collectioni in unum notabilis sui quantitatis resistit.\pend \pstart  Ita enim necesse est turbari aequilibrium\protect\index{Sachverzeichnis}{aequilibrium} circulationis,  ut in Hypothesi \edtext{nostra}{\lemma{nostra}\Bfootnote{\textsc{G. W. Leibniz, }\cite{00256}\textit{Hypothesis physica nova}, Mainz 1671, § 27 (\cite{00256}\textit{LSB} VI, 2, N. 40 § 27).}} ostensum est, dum \edtext{hic}{\lemma{dum}\Afootnote{ \textit{ (1) }\ alibi \textit{ (2) }\ hic \textit{ L}}} in unum corporum  subtilium collectum est, et proinde corpora alibi tanto sunt  compressiora. Ergo causa horum phaenomenorum est, repugnantia  materiae circumjacentis ad partium suarum subtiliationem, aut  subtilium expressionem seu segregationem; poris pervadendis  locoque \edtext{quem corpus aliquod}{\lemma{locoque}\Afootnote{ \textit{ (1) }\ a corpore dato \textit{ (2) }\ quem corpus aliquod \textit{ L}}} deseret implendo  necessariam, repugnantiam autem istam tandem ad aequilibrium\protect\index{Sachverzeichnis}{aequilibrium} circulationis universalis\protect\index{Sachverzeichnis}{circulatio!universalis} revocandam.\pend \pstart  Haec vera esse satis ex ipsis phaenomenis apparet. Nam quod Mercurius\protect\index{Sachverzeichnis}{mercurius!purgatus} aere purgatus in Tubo Torricelliano\protect\index{Sachverzeichnis}{Tubus!Torricellianus}, \edtext{aut aqua in Recipiente exhausto plane non descendit,}{\lemma{aut aqua in Recipiente exhausto}\Afootnote{\textbar\ plane \textit{ erg.}\ \textbar\  non descendit, \textit{ erg.} \textit{ L}}}  manifestissima ratio est, quia nihil suppetit quod locum  ejus implere possit. Cum enim aer inest\edtext{ liquori,}{\lemma{inest}\Afootnote{ \textit{ (1) }\ , is ex eo segregatur \textit{ (2) }\  liquori, \textit{ L}}} ipso pondere liquoris exprimitur, locumque implet, aut si non  sufficit ad totum locum implendum, dilatatur. Unde \edtext{exigua etiam}{\lemma{Unde}\Afootnote{ \textit{ (1) }\ non dubito  si quis foramen \textit{ (2) }\ exigua etiam \textit{ L}}} \edtext{ac pene  invisibilis bulla aeris liquoris immissa}{\lemma{etiam}\Afootnote{ \textit{ (1) }\ bulla Tubo immissa \textit{ (2) }\ ac [...] immissa \textit{ L}}} locum dilatata implet,  descensumque procurat, quia aer \edtext{facile}{\lemma{}\Afootnote{facile \textit{ erg.} \textit{ L}}} expandi potest, at Mercurius\protect\index{Sachverzeichnis}{mercurius}  ipse aut aqua non aeque. Hinc arbitror si quis eo casu perforet  Tubi \edtext{summitatem, sibilum}{\lemma{summitatem,}\Afootnote{ \textit{ (1) }\ stridorem \textit{ (2) }\ sibilum \textit{ L}}} aeris intrantis sensuram \edtext{quemadmodum aperto Recipiente exhausto. Et ut id fiat commodius poterit Tubus ipse vitreus esse in summo apertus, vesicaque agglutinata ac postea perforanda clausus.}{\lemma{quemadmodum}\Afootnote{[...] poterit \textit{ (1) }\ caput seu summitas Tubi \textit{ (2) }\ Tubus ipse  \textit{(a)}\ Torricellianus\protect\index{Sachverzeichnis}{Tubus!Torricellianus|textit} \textit{(b)}\ vitreus esse [...] clausus. \textit{ erg.} \textit{ L}}} Ex  his salvatur Experimentum Hugenianum\protect\index{Sachverzeichnis}{experimentum!Hugenianum} primum.\edtext{}{\lemma{primum.}\Bfootnote{\textsc{Chr. Huygens, }\cite{00062}a.a.O., S.~134f.  (\textit{HO} VII, S.~202).}} Nec major  difficultas in secundo.\edtext{}{\lemma{secundo.}\Bfootnote{\textsc{Chr. Huygens, }\cite{00062}a.a.O., S.~135f. (\textit{HO} VII, S.~202f.).}} \edtext{Nam aqua descendere}{\lemma{Nam}\Afootnote{ \textit{ (1) }\ aer des \textit{ (2) }\ aqua descendere \textit{ L}}} conatur  nisu propriae gravitatis\protect\index{Sachverzeichnis}{gravitas}. Haec in fundo seu prope orificium  tubi major. Ergo et maxima ibi pressio. Ergo et maxima  aeris \edtext{post purgationem quantamcunque residui}{\lemma{}\Afootnote{post purgationem   \textbar\ quantamcunque \textit{ erg.}\ \textbar\  residui \textit{ erg.} \textit{ L}}} ad locum in summo implendum expressio.  Ergo aeris cujusdam in fundo constituti \edtext{et undiquaque pressi}{\lemma{}\Afootnote{et undiquaque pressi \textit{ erg.} \textit{ L}}} collectio  congregatioque in unam bullam, ad quam per insensibiles canales  ab omni parte confluxit. Bulla semel nata quippe  aqua levior ascendet, et in via reliquum aerem colliget, \edtext{magisque ac magis dilatabitur}{\lemma{magisque}\Afootnote{ac magis dilatabitur \textit{ erg.} \textit{ L}}}  ac proinde continue augebitur cumque continue ab aquae gravitate\protect\index{Sachverzeichnis}{gravitas} ad ascendendum \edtext{sollicitetur seu}{\lemma{sollicitetur seu}\Afootnote{\textit{ erg.} \textit{ L}}} prematur tandem \edtext{velut cuneo facto satis virium nanciscetur ad}{\lemma{velut [...] ad}\Afootnote{\textit{ erg.} \textit{ L}}} fissuram sibi \edtext{aperiendam}{\lemma{sibi}\Afootnote{ \textit{ (1) }\ aperiet repertis sine dubio \textit{ (2) }\ aperiendam \textit{ L}}} per \edtext{aquam}{\lemma{per}\Afootnote{ \textit{ (1) }\ aerem \textit{ (2) }\ aquam \textit{ L}}}, ut non expectato  ulteriore ascensu summitatem Tubi attingere possit  maxima sui ad \edtext{summitatem}{\lemma{ad}\Afootnote{ \textit{ (1) }\ locum \textit{ (2) }\ summitatem \textit{ L}}} implendam necessaria  dilatatione. \edtext{Cumque}{\lemma{dilatatione.}\Afootnote{ \textit{ (1) }\ Nec illud difficile est \textit{ (2) }\ Cur \textit{ (3) }\ Cumque \textit{ L}}} ipsa Tubi  concussio conferat ad primae cujusdam bullae generationem,  (facit enim, ut partes aeris \edtext{in aqua sparsa}{\lemma{aeris}\Afootnote{ \textit{ (1) }\ interspersae, nec ad se uniendum \textit{ (2) }\  aquae interspersae \textit{ (3) }\ in aqua sparsa \textit{ L}}} collidantur, aut inter concussionem rimam  quandam communicationis inveniant ad se colligendum  assequendamque magnitudinem ad perrumpendum necessariam, vel  etiam immittit ei aeris aliquid ex ipsis sui lateribus ictu\protect\index{Sachverzeichnis}{ictus} excessum)  eadem postea sequentur. Cur spiritus vini\protect\index{Sachverzeichnis}{spiritus!vini} plus bullarum dederit \textso{Exper. 3}.\edtext{}{\lemma{\textso{Exper. 3}.}\Bfootnote{\textsc{Chr. Huygens, }\cite{00062}a.a.O., S.~136f. (\textit{HO} VII, S.~203f.).}} alterius loci quaestio est, caeterae experimenti 3. circumstantiae patent. Cui exper. 4.\edtext{}{\lemma{exper. 4.}\Bfootnote{\textsc{Chr. Huygens, }\cite{00062}a.a.O., S.~137\textendash139 (\textit{HO} VII, S.~204f.).\protect\rule[0cm]{6cm}{0cm}}} bulla resorbetur, ubi  primum ratio ejus experimendae cessavit, per se patet. 