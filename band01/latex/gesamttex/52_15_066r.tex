      
               
                \begin{ledgroupsized}[r]{120mm}
                \footnotesize 
                \pstart                
                \noindent\textbf{\"{U}berlieferung:}   
                \pend
                \end{ledgroupsized}
            
              
                            \begin{ledgroupsized}[r]{114mm}
                            \footnotesize 
                            \pstart \parindent -6mm
                            \makebox[6mm][l]{\textit{Lil}}Korrekturen und Unterstreichungen in einer Abschrift von Schreiberhand: LH XXXV 15, 6 Bl. 66\textendash 73. 4 Bog. 2\textsuperscript{o}. 15 S. zweispaltig. Linke Spalte fortlaufender Text, rechts geringf\"{u}gige Erg\"{a}nzungen sowie auf Bl. 67 v\textsuperscript{o} eine l\"{a}ngere Erg\"{a}nzung von Schreiberhand. Wir gehen davon aus, dass es sich bei den Texteingriffen des Kopisten um die nachtr\"{a}glichen Korrekturen von Unachtsamkeiten w\"{a}hrend des Abschreibens handelt und weisen diese daher im Apparat nicht aus. Auf allen Seiten mit Ausnahme von Bl. 66 r\textsuperscript{o} meist kleinere Korrekturen und Unterstreichungen von Leibniz' Hand. Darunter etwa in der Mitte von Bl. 71 v\textsuperscript{o} eine Rech\-nung. Bl. 71 r\textsuperscript{o} enth\"{a}lt in der Mitte der rechten Spalte die Zeichnung \textit{[Fig.~1]}. Bl. 73 v\textsuperscript{o} leer.\\Cc 2, Nr. 484 B \pend
                            \end{ledgroupsized}
                \vspace*{8mm}
                \pstart 
                \normalsize
             \centering[66 r\textsuperscript{o}] X\pend \pstart Ex quo horologium\protect\index{Sachverzeichnis}{horologium} fune-pendulo\protect\index{Sachverzeichnis}{funependulum} animatum, omnibus hactenus cognitis accuratius detectum est, in magnam omnes spem erecti sumus negotii longitudinum\protect\index{Sachverzeichnis}{longitudo} praecise aliquando penitus conficiendi, quantum ab observatione coeli sperari potest. Horologio\protect\index{Sachverzeichnis}{horologium} jam accurato supposito variae propositae sunt loci navis\protect\index{Sachverzeichnis}{navis} per observationes coelestes inveniendi rationes, alia alia commodior, ex quibus una mihi in mentem venit universalis admodum et simplex, satisque ut credo accurata.\pend \pstart \textso{Simplex,} quia non nisi una observatione coelesti transigitur, cum contra ubi duabus pluribusque observationibus \textso{diverso tempore} factis opus est, interea navi\protect\index{Sachverzeichnis}{navis} provecta, difficillima reddatur computatio. \textso{Universalis,} quia nulli fere tempori, non diei, non nocti, non certis siderum altitudinibus alligata est; sed solis\protect\index{Sachverzeichnis}{sol}, Lunaeve\protect\index{Sachverzeichnis}{luna}, aut stellae\protect\index{Sachverzeichnis}{stella!fixa} cujusdam fixae conspectu contenta est, qui raro per notabile tempus deesse potest. Cum contra solutiones quae ex \textso{Lunae }\protect\index{Sachverzeichnis}{luna}observatione pendent, dimidio fere mensis tempore, ante et post novilunium, conspectu scil: Lunae\protect\index{Sachverzeichnis}{luna} negato, cessent. Et quae \textso{Sole}\protect\index{Sachverzeichnis}{sol} indigent, noctu fieri nequeant, et eae in quibus duabus aequalibus ejusdem sideris\protect\index{Sachverzeichnis}{altitudo sideris} altitudinibus observatis opus est, hoc incommodum habeant, ut priore observatione facta, posterior aeris marisve injuria facile intercipiatur, ac proinde prior reddatur inutilis. Et eae quae meridianam praecise altitudinem desiderant, cum certo quasi momento sint alligatae, saepissime frustrentur, cum fortasse tempore meridianae, seu summae altitudinis, non eandem quam aliis quibusdam indefinitis, antea posteave momentis, serenitatem simus habituri. De quibus aliisque in hoc negotio observandis legi possunt tum quae ab Illustri Hugenio\protect\index{Namensregister}{\textso{Huygens} (Hugenius, Vgenius, Hugens, Huguens), Christiaan 1629\textendash 1695}, horologii penduli\protect\index{Sachverzeichnis}{horologium!pendulum} inventore, circa applicationem ejus ad longitudines\protect\index{Sachverzeichnis}{longitudo} sunt scripta,\edtext{}{\lemma{scripta,}\Bfootnote{\textsc{Chr. Huygens}, \cite{00212}\textit{Kort onderwijs}, Den Haag 1665 (\textit{HO} XVII, S.~199\textendash 237).}} tum quae transactionibus Anglicanis num: 47. sunt inserta.\edtext{}{\lemma{inserta.}\Bfootnote{\textsc{Chr. Huygens, }\cite{00064}\textit{Instructions concerning the use of pendulum-watches}, \textit{PT} 4 (1669), S.~937\textendash 953 (\textit{HO} VI, S.~446\textendash 459).}} \textso{Accurata} denique satis est quam propono, ratio, tum quia calculo exiguo, aut facillimo, ac ne nautas quidem turbaturo indiget, tum quia sola 