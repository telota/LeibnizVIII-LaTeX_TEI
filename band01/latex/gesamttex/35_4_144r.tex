      
               
                \begin{ledgroupsized}[r]{120mm}
                \footnotesize 
                \pstart                
                \noindent\textbf{\"{U}berlieferung:}   
                \pend
                \end{ledgroupsized}
            
              
                            \begin{ledgroupsized}[r]{114mm}
                            \footnotesize 
                            \pstart \parindent -6mm
                            \makebox[6mm][l]{\textit{L}}Konzept: LH XXXVII 3 Bl. 144\textendash145. 1 Bog. 2\textsuperscript{o}. 3 2/3 S. zweispaltig. Bl. 145 v\textsuperscript{o} 2/3 beschrieben. Auf dieser Seite oben rechts die Zeichnung fig. 4. Nur wenige Korrekturen am Text und ein ungew\"{o}hnlich sauberes Schriftbild verweisen auf den in gewisser Hinsicht abschließenden Charakter des St\"{u}cks.\\Cc 2, Nr. 491 D \pend
                            \end{ledgroupsized}
                %\normalsize
                \vspace*{5mm}
                \begin{ledgroup}
                \footnotesize 
                \pstart
            \noindent\footnotesize{\textbf{Datierungsgr\"{u}nde}: Auch dieses St\"{u}ck geh\"{o}rt inhaltlich zu den Systematisierungsversuchen, die Leibniz hinsichtlich der Vakuumph\"{a}nomene vornimmt, um zu einer alle diese Ph\"{a}nomene erkl\"{a}renden Hypothese zu gelangen. Insbesondere gibt es \"{U}bereinstimmungen mit N. 51, die sich u.~a. darin ausdr\"{u}cken, dass sich die Zeichnung fig. 4 von Bl. 145 v\textsuperscript{o} unseres St\"{u}cks auch auf Bl. 142 v\textsuperscript{o} von N. 51 findet, wo sie unter der Bezeichnung fig. 5 firmiert. Wir nehmen dies als Indiz f\"{u}r einen vergleichbaren Entstehungszeitraum und st\"{u}tzen diesen Befund durch das mit den Texttr\"{a}gern von N. 50 und N. 51 \"{u}bereinstimmende Wasserzeichen.}
                \pend
                \end{ledgroup}
            
                \vspace*{8mm}
                \pstart 
                \normalsize
            [144 r\textsuperscript{o}] Pour rendre Raison d'un phaenomene il faut tacher de se garder de toutes les hypotheses, autant qu'il est possible: et en effect, je crois pouuoir rendre raison \edtext{des experiences}{\lemma{raison}\Afootnote{ \textit{ (1) }\ de tous les phenomenes \textit{ (2) }\ de toutes les experiences \textit{ (3) }\ des experiences \textit{ L}}}  de l'attachement des corps dans le Recipient  \'{e}puis\'{e}, et de leur suspension quand  même leur hauteur ou pesanteur semble surpasser les forces de la pression de l'Atmosphere\protect\index{Sachverzeichnis}{atmosph\`{e}re}, en n'y employant que le Ressort de l'air (dont nous sommes convaincus par une infinit\'{e} d'experiences), s'il en reste tant soit dans le Recipient \'{e}puis\'{e}, \edtext{puisque asseurement il en reste tousjours.}{\lemma{}\Afootnote{puisque [...] tousjours. \textit{ erg.} \textit{ L}}} Et \`{a} fin que ceux qui prendront la peine de lire ces lignes, ne soyent pas rebut\'{e}s par une proposition qui leur semblera d'abord si peu croyable, s\c{c}avoir que le peu d'air qui reste dans le Recipient, et qui est incapable de soûtenir l'eau \edtext{ordinaire}{\lemma{}\Afootnote{ordinaire \textit{ erg.} \textit{ L}}} de la  hauteur d'un pouce, puisse soûtenir  dix ou douze de la purg\'{e}e; je les prie de considerer, que si une bulle d'air \`{a} peine visible, dans la liqueur purg\'{e}e\protect\index{Sachverzeichnis}{liqueur!purg\'{e}e}, \edtext{la peut faire tomber comme nous voyons;}{\lemma{purg\'{e}e,}\Afootnote{ \textit{ (1) }\ peut faire tomber; qu \textit{ (2) }\ la [...] voyons; \textit{ L}}} il y a de l'apparence qu'aussi peu d'air, dans le Recipient, hors de la liqueur purg\'{e}e\protect\index{Sachverzeichnis}{liqueur!purg\'{e}e}, la puisse soûtenir, comme je croy. Mais il est un peu plus difficile, de montrer la maniere de cette pression, et comment le ressort d'un peu d'air, dont les forces semblent si petites, puisse \edtext{soûtenir un si grand poids}{\lemma{puisse}\Afootnote{ \textit{ (1) }\ faire  \textit{(a)}\ des effects si grands \textit{(b)}\ un si grand effect \textit{ (2) }\ soûtenir un si grand poids \textit{ L}}}; parce que cela depend d'un theoreme assez paradoxe, que vous allez voir d\'{e}montr\'{e}; s\c{c}avoir \textso{que le Ressort de la moindre bulle d'air quand elle a toute la libert\'{e} d'agir est d'une force presqu'infinie, et capable d'elever un poids} \edtext{\textso{grandissime;}}{\lemma{\textso{grandissime;}}\Afootnote{ \textbar\ car \textit{ gestr.}\ \textbar\ Monsieur \textit{ L}}} Monsieur Boyle\protect\index{Namensregister}{\textso{Boyle} (Boylius, Boyl., Boyl), Robert 1627\textendash 1691} a prouu\'{e} par des experiences, que l'air est capable d'une rarefaction incroyable; et il y a de l'apparence, que la moindre bulle pourroit remplir toute la place de l'atmosphere\protect\index{Sachverzeichnis}{atmosph\`{e}re} si le reste de l'air venoit \`{a} estre reduit en rien. Mais pour venir \`{a} une demonstration, voyla comme je croy m'y pouuoir prendre. Je dis donc, que la moindre bulle d'air dans sa constitution naturelle, comme nous l'appellons \`{a} nostre \'{e}gard, c'est \`{a} dire \edtext{dans un degrez de rarefaction conforme \`{a}}{\lemma{dire}\Afootnote{ \textit{ (1) }\ comme elle \textit{ (2) }\  dans le degrez de rarefaction qu'elle a, quand \textit{ (3) }\ dans [...] \`{a} \textit{ L}}} celuy de l'air ambient, soûtient effectivement \edtext{}{\lemma{}\Afootnote{effectivement~\textbar~pour ainsi dire \textit{ gestr.}\ \textbar\ l'effort \textit{ L}}}l'effort de tout le monde, ou au moins (pour ne mêler pas les conjectures, quoyque vraysemblables, avec la demonstration), de toute l'atmosphere\protect\index{Sachverzeichnis}{atmosph\`{e}re}. Je ne dis pas le poids\protect\index{Sachverzeichnis}{poids} de l'atmosphere\protect\index{Sachverzeichnis}{atmosph\`{e}re}, mais son effort. Car le poids\protect\index{Sachverzeichnis}{poids} de la colomne d'air qui presse la bulle donn\'{e}e, est peu de chose, et \'{e}gale environ le poids\protect\index{Sachverzeichnis}{poids} d'une colomne de vif argent\protect\index{Sachverzeichnis}{vif argent} de la hauteur, de 27 pouces, et de la largeur de la Bulle donn\'{e}e.\pend