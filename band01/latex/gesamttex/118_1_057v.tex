[57 v\textsuperscript{o}] Ad\footnote{\textit{Am oberen Rand von Bl. 57 v\textsuperscript{o}}: NB. Hic modus forte optimus \Denarius \hspace{3pt}NB.} nostram delineationem longitudinum\protect\index{Sachverzeichnis}{longitudo} opus, quod et efficitur ut aliquid nach proportion \selectlanguage{ngerman}im verjungten masstab soviel zur\"{u}ckgehe, als das schiff\protect\index{Sachverzeichnis}{Schiff} vor sich, und denn etwas, das stets\selectlanguage{latin} ad certum locum \selectlanguage{ngerman}weise,\selectlanguage{latin} a quo cognita distantia nostra a data re, et modis flexionis situs \edtext{portus a quo abiimus vel ad quem tendimus, et omnino locus noster cognoscitur}{\lemma{situs}\Afootnote{ \textit{ (1) }\ rei cogno \textit{ (2) }\ portus [...] cognoscitur \textit{ L}}}.\pend