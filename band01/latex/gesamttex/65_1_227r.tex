[227 r\textsuperscript{o}] dicto loco\edtext{}{\lemma{loco}\Bfootnote{\textsc{Chr. Huygens}, \cite{0123}a.a.O., S.~16 (\textit{HO} XVIII, S. 115).}}\selectlanguage{ngerman} pag. 16 vorgeschriebene Weise also geschehen muß.\pend \pstart Gesezt die pendel-Uhr\protect\index{Sachverzeichnis}{Pendeluhr} \edtext{solle in iedem schlag eine secunde machen, und irre}{\lemma{pendel-Uhr}\Afootnote{ \textit{ (1) }\ irre umb \textit{ (2) }\ solle [...] irre \textit{ L}}} in einem tag 3 oder 4 minuten, so muß man die pendul\protect\index{Sachverzeichnis}{Pendel} l\"{a}nger machen, wenn sie zu geschwind gehet; und k\"{u}rzer wenn sie zu langsam; und \edtext{zwar}{\lemma{}\Afootnote{zwar \textit{ erg.} \textit{ L}}} also daß ohngefehr $\displaystyle\frac{7}{144}$\rule[-4mm]{0mm}{10mm} eines zolles, den unterschied einer minute betragen; und soviel minuten der irrthum in sich halt, so viel mahl muß man $\displaystyle\frac{7}{144}$\rule[-4mm]{0mm}{10mm} eines Zolles der l\"{a}nge der pendul\protect\index{Sachverzeichnis}{Pendel} geben oder nehmen. Fehlet leztens noch ein weniges an der richtigkeit, so will Herr Hugenius\protect\index{Namensregister}{\textso{Huygens} (Hugenius, Vgenius, Hugens, Huguens), Christiaan 1629\textendash 1695}, daß man solche vollends erhalte vermittelst eines kleinen gewichtleins, mit \pgrk{D} in seiner figur \edtext{ad pag. 5}
{\lemma{pag. 5}\Bfootnote{\textsc{Chr. Huygens, }\cite{0123}a.a.O., Fig. I. (\textit{HO} XVIII, S. 71). Zu dieser Figur vgl. Leibniz' Marginalien in \cite{00262}\textit{LSB} VII, 4, S.~29f.}}\edtext{}{\lemma{}\Afootnote{ad pag. 5 \textit{ erg.} \textit{ L}}} bezeichnet, und an der stange der Pendul\protect\index{Sachverzeichnis}{Pendel} befindlich, so mit einer \edtext{daran befestigten}{\lemma{daran}\Afootnote{ \textit{ (1) }\ befindlichen \textit{ (2) }\ befestigten \textit{ L}}} schraube \edtext{etwas}{\lemma{}\Afootnote{etwas \textit{ erg.} \textit{ L}}} \edtext{auf oder nieder}{\lemma{etwas}\Afootnote{ \textit{ (1) }\ auf und nieder \textit{ (2) }\ auf oder nieder \textit{ L}}} gebracht werden kan; und zwar \edtext{wenn}{\lemma{zwar}\Afootnote{ \textit{ (1) }\ wird \textit{ (2) }\ wenn \textit{ L}}} die pendul\protect\index{Sachverzeichnis}{Pendel} etwas zu langsam, wird das gewichtlein aufwarts; und wo sie zu geschwind, abwarts geschraubt. Und hat er die maaße der figur\edtext{}{\lemma{}\Afootnote{figur \textbar\ pag. 5 \textit{ erg. u.}\ \textit{ gestr.}\ \textbar\ der \textit{ L}}} der uhr\protect\index{Sachverzeichnis}{Uhr} \edtext{ad pag. 5}{\lemma{pag. 5}\Bfootnote{\textsc{Chr. Huygens}, \cite{0123}a.a.O., Fig. I. (\textit{HO} XVIII, S. 71). Zu dieser Figur vgl. Leibniz' Marginalien in \cite{00262}\textit{LSB} VII, 4, S. 29f.}}\edtext{}{\lemma{}\Afootnote{ad pag. 5 \textit{ erg.} \textit{ L}}} bey gesezet, und dict. p. 16\edtext{}{\lemma{16}\Bfootnote{\textsc{Chr. Huygens}, \cite{0123}a.a.O., S.~16 (\textit{HO} XVIII, S. 115).}} ercl\"{a}ret, woraus zu sehen, umb wie viel das gewichtlein zu ver\"{a}ndern.\selectlanguage{latin}\pend 