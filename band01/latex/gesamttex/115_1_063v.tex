\edtext{mari, [63 v\textsuperscript{o}] nam interdum}{\lemma{mari,}\Afootnote{ \textit{ (1) }\ ideo a puppi \textit{ (2) }\ nam interdum \textit{ L}}} gubernaculo\protect\index{Sachverzeichnis}{gubernaculum} quiescente caetera circumaguntur, interdum media pars navis\protect\index{Sachverzeichnis}{navis}, locusque ubi malus est, interdum ipsa \edtext{prora}{\lemma{ipsa}\Afootnote{ \textit{ (1) }\ propra \textit{ (2) }\ prora \textit{ L}}} quiescit puppi circumacta, ideo \textso{Angulus flexionis}\protect\index{Sachverzeichnis}{angulus!flexionis}\edtext{}{\lemma{\textso{Angulus flexionis}}\Afootnote{\textit{doppelt unterstrichen}}}, \edtext{\textit{ABb} id est quem faciunt duae \edtext{\textso{lineae navigationis}}{\lemma{\textso{lineae navigationis}}\Afootnote{\textit{doppelt unterstrichen}}}\protect\index{Sachverzeichnis}{linea!navigationis} \textso{seu} motus navis \protect\index{Sachverzeichnis}{navis} \textit{AB}, \textit{Ab}}{\lemma{}\Afootnote{\textit{ABb} id est quem faciunt duae \textso{lineae} \textbar\ \textso{navigationis}\protect\index{Sachverzeichnis}{linea!navigationis} \textso{seu} \textit{ erg.}\ \textbar\ motus navis \protect\index{Sachverzeichnis}{navis} \textit{AB}, \textit{Ab} \textit{ erg.} \textit{ L}}} eo casu \edtext{quo}{\lemma{casu}\Afootnote{ \textit{ (1) }\ quando \textit{ (2) }\ quo \textit{ L}}} \edtext{\textso{centrum verticitatis}\protect\index{Sachverzeichnis}{centrum!verticitatis}\textso{ navis}\protect\index{Sachverzeichnis}{navis}\textso{ et magnetis}\protect\index{Sachverzeichnis}{magnes}}{\lemma{\textso{centrum verticitatis navis et magnetis}}\Afootnote{\textit{doppelt unterstrichen}}} non coincidunt, non est differentia angulorum directionis\protect\index{Sachverzeichnis}{angulus!directionis}. \pend \pstart Esto enim centrum verticitatis\protect\index{Sachverzeichnis}{centrum!verticitatis} in navi\protect\index{Sachverzeichnis}{navis} \textit{A} in magnete\protect\index{Sachverzeichnis}{magnes} \textit{B} \edtext{aut post flexionem in \textit{b}}{\lemma{}\Afootnote{aut post flexionem in \textit{b} \textit{ erg.} \textit{ L}}} tunc si polus\protect\index{Sachverzeichnis}{polus} esset in \textit{C} \textso{linea directionis }\protect\index{Sachverzeichnis}{linea!directionis}\edtext{}{\lemma{\textso{linea directionis}}\Afootnote{\textit{doppelt unterstrichen}}} magneticae erit \textit{BC} aut \textit{bc}. Sed quia polus\protect\index{Sachverzeichnis}{polus} \edtext{non est in \textit{C} verum in recta \textit{AC}}{\lemma{polus}\Afootnote{ \textit{ (1) }\ est quidem in recta \textit{AC} sed non in \textit{ (2) }\ non est in \textit{C} \textit{(a)}\ et si sit r \textit{(b)}\ sed \textit{(c)}\ verum in recta \textit{AC} \textit{ L}}} ultra \textit{C} producta in tantam longitudinem\protect\index{Sachverzeichnis}{longitudo} ut \textso{angulus}\edtext{}{\lemma{\textso{angulus magneticus}}\Afootnote{\textit{doppelt unterstrichen}}} \edtext{\textso{ magneticus} \textit{ACB} vel \textit{ACb} (si \textit{C} interim polus esse fingatur) seu quem faciunt duae}{\lemma{\textso{angulus}}\Afootnote{ \textit{ (1) }\ \textit{ACB}, si \textit{ (2) }\ \textit{C} fingatur du \textit{ (3) }\ \textso{magneticus} \textbar\ \textit{ACB} [...] fingatur) \textit{ erg.}\ \textbar\ seu quem faciunt duae \textit{ L}}} lineae \edtext{directionum}{\lemma{lineae}\Afootnote{ \textit{ (1) }\ sit nullus \textit{ (2) }\ \textit{AC} p \textit{ (3) }\ directionum \textit{ L}}} ex diversis centris verticitatum\protect\index{Sachverzeichnis}{centrum!verticitatis} magneticarum\protect\index{Sachverzeichnis}{magnes}, \edtext{ob nimiam exilitatem haberi possit}{\lemma{magneticarum,}\Afootnote{ \textit{ (1) }\ sit \textit{ (2) }\ ob nimiam exilitatem haberi possit \textit{ L}}} pro nullo, ac proinde lineae directionum\protect\index{Sachverzeichnis}{linea!directionis} ex diversis centris verticitatum\protect\index{Sachverzeichnis}{centrum!verticitatis} magneticarum pro parallelis: ideo posito polo\protect\index{Sachverzeichnis}{polus} longe ultra \textit{C} et linea directionis\protect\index{Sachverzeichnis}{linea!directionis} ex centro verticitatis\protect\index{Sachverzeichnis}{centrum!verticitatis} \textit{A} posita \textit{AC} utcunque producta, erit linea directionis ex centro verticitatis\protect\index{Sachverzeichnis}{centrum!verticitatis} \textit{B}, \textit{BD} \edtext{producta ultra \textit{D} et}{\lemma{\textit{BD}}\Afootnote{ \textit{ (1) }\ parallela et \textit{ (2) }\ producta ultra \textit{D} et \textit{ L}}} ex centro \textit{b} erit \textit{bd} producta ultra \textit{d} saltem ad sensum eruntque parallelae \textit{AC}, \textit{BD}, \textit{bd}. Eruntque \edtext{duo}{\lemma{}\Afootnote{ \textit{ (1) }\ hi quatuor \textit{ (2) }\ duo \textit{ erg.} \textit{ L}}} anguli directionis\protect\index{Sachverzeichnis}{angulus!directionis} \edtext{posita eadem linea navigationis\protect\index{Sachverzeichnis}{linea!navigationis}, ubicunque sit magnes\protect\index{Sachverzeichnis}{magnes}, sive etiam si plures sint magnetes\protect\index{Sachverzeichnis}{magnes}}{\lemma{}\Afootnote{posita [...] magnetes\protect\index{Sachverzeichnis}{magnes} \textit{ erg.} \textit{ L}}} semper aequales (quamdiu magnes \protect\index{Sachverzeichnis}{magnes} non mutat directionem superveniente forte declinatione, de quo postea) \edtext{ et si linea navigationis sit \textit{AB} et duo magnetes alius in \textit{A} alius in \textit{B} erunt aequales anguli \textit{DBA} et \textit{CAB}}{\lemma{postea)}\Afootnote{ \textit{ (1) }\ \textit{CAB}, \textit{C} \textit{ (2) }\ sive magnes\protect\index{Sachverzeichnis}{magnes|textit} sit in \textit{A} nempe duo: \textit{CAB} et \textit{CAb} sive sit in \textit{B} vel \textit{b} nempe \textit{DBA}, \textit{dba} \textit{ (3) }\ et [...] \textit{CAB} \textit{ L}}} similiter si linea navigationis\protect\index{Sachverzeichnis}{linea!navigationis} sit [\textit{Ab}]\edtext{}{\Afootnote{\textit{AB}\textit{\ L \"{a}ndert Hrsg. } }} et magnes\protect\index{Sachverzeichnis}{magnes} sit in \textit{A} vel \textit{b} vel utroque simul erunt anguli aequales \textit{CAb}, \textit{dbA} posito autem magnete\protect\index{Sachverzeichnis}{magnes} extra centrum verticitatis\protect\index{Sachverzeichnis}{centrum!verticitatis} ipsius navis\protect\index{Sachverzeichnis}{navis}, duorum diversorum situum \edtext{anguli}{\lemma{situum}\Afootnote{ \textit{ (1) }\ lineae \textit{ (2) }\ anguli \textit{ L}}} directionis\protect\index{Sachverzeichnis}{angulus!directionis} \textit{DBA} et \textit{dbA} inter se different. Differentiae quantitas ita computabitur: finge acum verticitate\protect\index{Sachverzeichnis}{vertex} propria carere, seu verticitate\protect\index{Sachverzeichnis}{vertex} navis\protect\index{Sachverzeichnis}{navis} circumagi, id est transfer \textit{BD} in \textit{bD-D} et \textit{AC} in \textit{AC-C}, ut \edtext{quem lineae \textit{CA}, \textit{DB} faciunt angulum ad \textit{AB} eum}{\lemma{ut}\Afootnote{ \textit{ (1) }\ lineae \textit{CA}, \textit{DB} eundem facian \textit{ (2) }\ quem [...] eum \textit{ L}}} faciant ad \textit{Ab}. Erit angulus \textit{D-DbA} aequalis angulo \textit{DBA} ac proinde differentia \edtext{duorum angulorum directionis}{\lemma{differentia}\Afootnote{ \textit{ (1) }\ Anguli \textit{ (2) }\ duorum angulorum directionis \textit{ L}}} \textit{DBA} et \textit{dbA} erit \textit{D-DbD}. Restat determinare angulum flexionis\protect\index{Sachverzeichnis}{angulus!flexionis} \textit{BAb}. \edtext{Id fiet si centrum verticitatis}{\lemma{\textit{BAb}.}\Afootnote{ \textit{ (1) }\ Id ita fiet: datur linea navigationis\protect\index{Sachverzeichnis}{linea!navigationis|textit} \textit{AB} in situ posteriori aequalis \textit{Ab} in situ \textit{ (2) }\ Id fiet si centrum \textit{(a)}\ motus s \textit{(b)}\ verticitatis \textit{ L}}} navis\protect\index{Sachverzeichnis}{navis} seu punctum \textit{A} vel quantitatem lineae \textit{AB} habebimus. \pend 