[103 v\textsuperscript{o}]  poterit praeter Experimentum~8. etiam hic convinci. Tubus \textit{AAB} esto in \textit{AA} apertilis prominensque  extra vas \textit{D} in aerem liberum.\footnote{\textit{In der rechten Spalte}: \textso{Experim. 10.}} Sint item partes Tubi \textit{AAE} et \textit{BE} conjunctae Epistomio\protect\index{Sachverzeichnis}{epistomium}, quod claudi  ac proinde communicatio aperiri potest. His positis Mercurio\protect\index{Sachverzeichnis}{mercurius} delapso ex \textit{AAE} in vas subjectum \textit{C}  claudatur\edtext{}{\lemma{}\Afootnote{claudatur  \textbar\ eo ipso \textit{ gestr.}\ \textbar\ Epistomium \textit{ L}}} Epistomium\protect\index{Sachverzeichnis}{epistomium} \edtext{\textit{E}}{\lemma{\textit{E}}\Afootnote{ \textit{ erg.} \textit{ L}}}, ita communicatio  inter \textit{AAE} et \textit{BE} interrumpetur. Fieri non difficili  artificio potest, ut ipse Mercurius\protect\index{Sachverzeichnis}{mercurius} labendo Epistomium\protect\index{Sachverzeichnis}{epistomium}  claudat. Vas \textit{D} enim aperiri non debet. Clauso Epistomio\protect\index{Sachverzeichnis}{epistomium} aperiatur manu apertura \textit{AA} extra  vas \textit{D} prominens\edtext{. Sumatur Embolus Tubi \textit{FG} ab altero  latere \textit{F} aperti forma, ita tamen ut fundus ejus clausus sit exiguo  foramine \textit{G} aperibili claudibilique pertusus}{\lemma{prominens}\Afootnote{ \textit{ (1) }\ , et embolo\protect\index{Sachverzeichnis}{embolus|textit} cavo  \textbar\ immisso \textit{ gestr.}\ \textbar\  foramen  in fundo habente \textit{ (2) }\ . Sumatur Embolus Tubi   \textbar\ \textit{FG} \textit{ erg.}\ \textbar\  ab altero  latere   \textbar\ \textit{F} \textit{ erg.}\ \textbar\ aperti [...] fundus  \textbar\ ejus \textit{ erg.}\ \textbar\  clausus sit exiguo  foramine   \textbar\ \textit{G} \textit{ erg.}\ \textbar\  aperibili claudibilique  \textit{(a)}\ perforatus \textit{(b)}\ pertusus \textit{ L}}}.  Hoc foramine aperto emboloque\protect\index{Sachverzeichnis}{embolus} in \textit{AAE} intruso, aer  per foramen \textit{G} exibit in embolum\protect\index{Sachverzeichnis}{embolus}, claudatur  foramen \textit{G} embolusque\protect\index{Sachverzeichnis}{embolus} per vim extrahatur, et  statim \edtext{apertura \textit{AA}}{\lemma{statim}\Afootnote{ \textit{ (1) }\ vas \textit{ (2) }\ apertura \textit{AA} \textit{ L}}} denuo claudatur,  quo facto aperiatur iterum Epistomium\protect\index{Sachverzeichnis}{epistomium} \textit{E} patet \textit{AAE} esse omni aere quantum per artem possibile  est vacuum, si ergo \edtext{funiculus aeris Mercurium  in \textit{BE} sustinet}{\lemma{ergo}\Afootnote{ \textit{ (1) }\ sustentatio funiculi\protect\index{Sachverzeichnis}{funiculus|textit} \textit{ (2) }\ funiculus [...] sustinet \textit{ L}}}, eum nunc labi necesse est, quia  iste funiculus\protect\index{Sachverzeichnis}{funiculus} aere expulso non potest non \edtext{evanuisse}{\lemma{non}\Afootnote{ \textit{ (1) }\ esse \textit{ (2) }\  evanuisse \textit{ L}}}. \edtext{At vero non labetur sed suspensus manebit.}{\lemma{}\Afootnote{At [...] manebit. \textit{ erg.} \textit{ L}}} Facilius experimentum hoc erit ad  idem fortius ostendendum;\footnote{\textit{In der rechten Spalte}: \textso{Experim. 11.}} quod demonstrationis vim  habebit ad controversiam de funiculo\protect\index{Sachverzeichnis}{funiculus} penitus dirimendam.  Esto in vase \textit{D} vas aliud \textit{I} clausum nisi  quod per canalem \textit{IH} cum Tubo \textit{AE} communicat, Mercurio\protect\index{Sachverzeichnis}{mercurius} \textit{AE} delapso aer in vase \textit{I} se partietur  in vas \textit{I} et \edtext{Tubum}{\lemma{et}\Afootnote{ \textit{ (1) }\ canalem \textit{ (2) }\ Tubum \textit{ L}}} \textit{AE}. Ergo aer Tubi \textit{AE} minus quam ante \edtext{erit dilatatus}{\lemma{erit}\Afootnote{ \textit{ (1) }\ vacuus \textit{ (2) }\ dilatatus \textit{ L}}}, et \edtext{tanto propior}{\lemma{tanto}\Afootnote{ \textit{ (1) }\ minus  pleno \textit{ (2) }\ propior \textit{ L}}} ordinario, quanto majus est vas \textit{I} etsi loco vasis \edtext{\textit{I} Tubus \textit{IL}}{\lemma{\textit{I}}\Afootnote{ \textit{ (1) }\ canalis \textit{ (2) }\ Tubus \textit{IL} \textit{ L}}} exiret extra  vas \textit{D} in liberum aerem,\footnote{\textit{In der rechten Spalte}: \textso{Exp. 12.} NB.} quo nullum vas capacius,  canalis, in effectu locus \textit{AE} haberi  poterit pro pleno, nulla ergo in eo aeris tensio\protect\index{Sachverzeichnis}{tensio}.  Ergo si \edtext{a funiculo}{\lemma{si}\Afootnote{ \textit{ (1) }\ ab aeris \textit{ (2) }\ a funiculo \textit{ L}}} materiae in \textit{AE}  tensae pendet Mercurii\protect\index{Sachverzeichnis}{mercurius} sustentatio, necesse est Mercurium\protect\index{Sachverzeichnis}{mercurius} delabi sustentaculo scilicet cessante.  At suspensus manebit. \edtext{Aer}{\lemma{manebit.}\Afootnote{ \textit{ (1) }\ Tensio\protect\index{Sachverzeichnis}{tensio|textit} \textit{ (2) }\ Aer \textit{ L}}} \edtext{ergo delapsu}{\lemma{ergo}\Afootnote{ \textit{ (1) }\ dilatatione \textit{ (2) }\  delapsu \textit{ L}}} Mercurii\protect\index{Sachverzeichnis}{mercurius} ex \textit{AE} dilatatur a se ipso,  non vi a Mercurio\protect\index{Sachverzeichnis}{mercurius} labente \edtext{adhibita}{\lemma{labente}\Afootnote{ \textit{ (1) }\ contra \textit{ (2) }\ adhibita \textit{ L}}}, ac  proinde Funiculus\protect\index{Sachverzeichnis}{funiculus} est \edlabel{104v1}
nullus.\pend \pstart
\edtext{Sed}{\lemma{nullus.}\xxref{104v1}{104r1}\Afootnote{ \textit{ (1) }\ Pondus au \textit{ (2) }\ Pressionem \textit{ (3) }\ Compressionem autem aeris in vase \textit{D} oriri a pondere Mercu \textit{ (4) }\  Sed \textit{ L}}} \edlabel{104r1}miretur aliquis, cur altitudo  Tubi nihil ad rem pertineat; nam \edtext{cum}{\lemma{nam}\Afootnote{ \textit{ (1) }\ certum est \textit{ (2) }\   cum \textit{ L}}} Mercurius\protect\index{Sachverzeichnis}{mercurius} ex Tubo \textit{AB} labatur in Vas \textit{D} quanto altior est lapsus ex Tubo, tanto magis  comprimitur aer in \textit{D} ac \edtext{proinde si Tubus est altior, deberet Mercurius altius}{\lemma{proinde}\Afootnote{ \textit{ (1) }\ deberet magis \textit{ (2) }\ si [...] altius \textit{ L}}} suspendi. Sed 