[159 v\textsuperscript{o}]  [S.~206] gemacht werden oben scharff und unten breit deßen das gegentheil geschehen,  soll derweil unden scharff und schmahl schneit die See, en geeft schot an 't schip\protect\index{Sachverzeichnis}{schip} ,  oben breit vergr\"{o}st den inhalt und \edtext{}{\lemma{}\Afootnote{und  \textbar\ große \textit{ gestr.}\ \textbar\ Ladung. \textit{ L}}} Ladung. Darumb man hier  zulande die Plancken voor zoo veel buigt als 't moeglijck is, wie  wohl dat im die mittelmaß macht zunehmen, dann die scharffe und  schmahle schiff\protect\index{Sachverzeichnis}{Schiff} tragen wenig, die schwere wollen nicht vort, und  sein loom als \edtext{koeien}{\lemma{als}\Afootnote{ \textit{ (1) }\ koeifen \textit{ (2) }\ koeien \textit{ L}}}. \textit{D'Engelsche feilen mede daer in, dat zy  hunne }\textit{schepen}\protect\index{Sachverzeichnis}{schip}\textit{ tr\"{o}gsche wijß zaer breid en onbeesneden vaak  maken; achter al te plat, en lootlijnig}. Und wie wohl  sie wohl wißen, daß ein schiff\protect\index{Sachverzeichnis}{Schiff} zu schmahl und zu schmuig wenig  tragt, zu dick und zu groß nicht vort will, so sieet man sie dochmehr als die hollander, von dießen regeln abweichen; hingegen vint  man offt da§ sie beßer seegeln als die Hollander, dieweil sie sich mehr befleißen, plat laeg, en lang te timmeren, als man hier thut. Allein, trachtende die schiff\protect\index{Sachverzeichnis}{Schiff}e zu schnell lauffend zu machen, machen sie selbige zu  scharff, daß sie oft umbfallen; wenn man sie von ballast und schwehre entbl\"{o}st. Sie sind sonst wackere seeleute und wißen sich wohl zu retten in  Zeit von noth. Sparen ihre gort\textendash zacken\protect\index{Sachverzeichnis}{gort\textendash zakken} nicht umb die L\"{o}cher under  waßer zu f\"{u}llen. Dieweil sie \edtext{wißen}{\lemma{sie}\Afootnote{ \textit{ (1) }\ wollen, \textit{ (2) }\ wißen \textit{ L}}} daß solche genezt schwellen  und den plaz f\"{u}llen. \textit{Schmelten hun }\textit{teer}\protect\index{Sachverzeichnis}{Teer}\textit{ met gloeiende kogels  om brant te schouwen en schlaen }\textit{zeilen}\protect\index{Sachverzeichnis}{zeil}\textit{ voor reten en openinge  in tijd von storm om te kalefaten en te kluzen bey stillte.  Ja ombinden de }\edtext{\textit{schepen}}{\lemma{de}\Afootnote{ \textit{ (1) }\ \textit{schippen}\protect\index{Sachverzeichnis}{schip|textit} \textit{ (2) }\ \textit{schepen} \textit{ L}}}\textit{ mit towen in hooge noth en weten bey  gebreck van }\textit{anckers}\protect\index{Sachverzeichnis}{Anker}\textit{ gevulde kisten} mit loot \edtext{bley}{\lemma{}\Afootnote{bley \textit{ erg.} \textit{ L}}} of eisen\protect\index{Sachverzeichnis}{Eisen} in  ihre stelle zu gebrauchen [S.~207] Keine nation hat so viel eisenwerck und eiserne Negel\protect\index{Sachverzeichnis}{Eisennagel} an ihren schiffen\protect\index{Sachverzeichnis}{Schiff} als sie. Har pompen\protect\index{Sachverzeichnis}{pomp}  seyn ketting pompen\protect\index{Sachverzeichnis}{ketting pomp}, die mitten im schiff\protect\index{Sachverzeichnis}{Schiff} staen, welches loblicher  als die hierlantsche pompen\protect\index{Sachverzeichnis}{pomp}, denn sie nicht so balld unreinen werden,  aber hingegen hinderlich in des schiffs\protect\index{Sachverzeichnis}{Schiff} raum, und einen unlieblichen  gelaut geben. Hier waßer und bier wird mit pompens\protect\index{Sachverzeichnis}{pomp} oben \edtext{ausgezapt}{\lemma{oben}\Afootnote{ \textit{ (1) }\ ausgedapt \textit{ (2) }\ ausgezapt \textit{ L}}}: dient zur erhaltung das es nicht verderbe.\pend \pstart  Sie schmieren ihre schiff\protect\index{Sachverzeichnis}{Schiff} von außen mit seep\protect\index{Sachverzeichnis}{seep} en talck\protect\index{Sachverzeichnis}{talk}, die Hollander  allein mit schmeer\protect\index{Sachverzeichnis}{smeer}, \textit{met gekalckt kannefaß en dat over  gotten mit heeten pick, breuwen sie in de reeten tegeens 't }\textit{ongediert}\protect\index{Sachverzeichnis}{ongedierte}, \textit{'t is bey haer en gebruyck in 't schlaen, de }\textit{schepen}\protect\index{Sachverzeichnis}{schip}\textit{ rondom mit  roode }\textit{schans\textendash kleeden}\protect\index{Sachverzeichnis}{schans\textendash kleed}\textit{ te bedecken, de wijse van haer enteren is op  't hooghst van 't }\textit{schip}\protect\index{Sachverzeichnis}{schip}\textit{. 't zy aens hut of back wel vorsien van Enterbyl,  Sabel en Hantbuß}. [S.~209] Er sezte eine Englische Schips Instruction\protect\index{Sachverzeichnis}{Schiffsinstruktion} oder ordre\protect\index{Sachverzeichnis}{Schiffsordnung} unterm Nahmen des Herzogs von Jorck\protect\index{Namensregister}{\textso{England: Jakob II.}, K\"{o}nig von England, Herzog von Yorck 1685\textendash 1688 \textdagger 1701} Hoch Hohen admirals\protect\index{Sachverzeichnis}{Hochadmiral}  von England\protect\index{Ortsregister}{England}. Ist an den Capitain\protect\index{Sachverzeichnis}{Kapit\"{a}n} gericht, and sehr notabel. [S.~212] Ist schohn dazu  gedacht, das das Britannische mar\protect\index{Ortsregister}{Armelkanal@\"{A}rmelkanal (Britannische mar)} gehet bis Cap finisterrae\protect\index{Ortsregister}{Kap Finisterre (Cap finisterrae)}.\pend \pstart  [?] Dieses alles war nichts als holzwerck, aber dieses  war vom besten holze\protect\index{Sachverzeichnis}{Holz}, und wann man wolte holz\protect\index{Sachverzeichnis}{Holz} von  schlechteren wert nehmen k\"{o}nte man 11070 R. abziehen.  In ubrigen \"{u}ber die obgedachte holzwerck w\"{u}rde das  Eisenwerck erfodern 7784 \edtext{gl.}{\lemma{7784}\Afootnote{ \textit{ (1) }\ R. \textit{ (2) }\ gl. \textit{ L}}} \pend \pstart  Die Kochsgereitschafft\protect\index{Sachverzeichnis}{Kochsgereitschaft} in einen schiff\protect\index{Sachverzeichnis}{Schiff} von dießen großen wert erfodert 352  uyt. De Lijnbaen heefft men noodig 35261 pont tow,  tot 45 fl. \edtext{g\"{u}lden}{\lemma{fl.}\Afootnote{ \textit{ (1) }\ glden \textit{ (2) }\ g\"{u}lden \textit{ L}}} daß het schip\protect\index{Sachverzeichnis}{schip} vont tesammen \edtext{5289}{\lemma{tesammen}\Afootnote{ \textit{ (1) }\ 5298 \textit{ (2) }\ 5289 \textit{ L}}}  Die Seegel\protect\index{Sachverzeichnis}{Segel} kosten auffs wenigste 2827  Die ankers\protect\index{Sachverzeichnis}{Anker} wegen zusammen 6450 lb. Zu 3 st (+ puto stubers +)  das lb, thut 967 gl. Und voor schips\textendash noodige  kleidinge wird man vonn\"{o}then haben 2264.  Also das obgemeldtes schiff\protect\index{Sachverzeichnis}{Schiff} ohne kriegs n\"{o}thige  ausr\"{u}stung, und mund\textendash kosten \edtext{ehe}{\lemma{mund\textendash kosten}\Afootnote{ \textit{ (1) }\ machen wird, \textit{ (2) }\ ehe \textit{ L}}} es in see gehen kan, erfodern wird zum wenigsten 9365 \edtext{g\"{u}lden}{\lemma{9365}\Afootnote{ \textit{ (1) }\ fl \textit{ (2) }\ g\"{u}lden \textit{ L}}}  Ein also gebautes schiff\protect\index{Sachverzeichnis}{Schiff} kan lange jahre dauern,  wie ich dann sagt Witsen\protect\index{Namensregister}{\textso{Witsen,} Nicolaes 1641\textendash 1717} ein Englisch schiff\protect\index{Sachverzeichnis}{Schiff!englisches} gesehen, so 70 jahr alt. Und wo em einen schiff\protect\index{Sachverzeichnis}{Schiff}  nichts ungemeines Vorst\"{o}st, kan es 20, 30 bis 50  jahr dauern. Alleine die meisten fahrzeuge kommen  vor der Zeit durch wind, und wetter, ungl\"{u}ck  und feinde, umb den hals.\pend \pstart  Artic 19 ist pag 280 bis 285 ist eine liste von allen  beweglichen dingen ein schiff\protect\index{Sachverzeichnis}{Schiff}, und gereitschaft zum  schiffs gebrauchs, auch defension und speise, vor ein jahr  vor ein schiff\protect\index{Sachverzeichnis}{Schiff} vor 100 mannen so etwa nach Curacao\protect\index{Ortsregister}{Cura\c{c}ao} Aleppo\protect\index{Ortsregister}{Aleppo} und Guinea\protect\index{Ortsregister}{Guinea} gehet. Dießes meritirte ganz copiiret  zu werden.\pend \pstart 