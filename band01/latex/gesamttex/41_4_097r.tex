      
               
                \begin{ledgroupsized}[r]{120mm}
                \footnotesize 
                \pstart                
                \noindent\textbf{\"{U}berlieferung:}   
                \pend
                \end{ledgroupsized}
            
              
                            \begin{ledgroupsized}[r]{114mm}
                            \footnotesize 
                            \pstart \parindent -6mm
                            \makebox[6mm][l]{\textit{L}}Konzept: LH XXXVII 3 Bl. 97\textendash98. 1 Bog. 2\textsuperscript{o}. 4 S. zweispaltig. Die Seiten 97~v\textsuperscript{o}, 98~r\textsuperscript{o} und 98~v\textsuperscript{o} enthalten in der rechten Spalte jeweils eine Zeichnung. Die Zeichnung auf Bl. 98~v\textsuperscript{o} wurde gestrichen.\\Cc 2, Nr. 486 C \pend
                            \end{ledgroupsized}
                %\normalsize
                \vspace*{5mm}
                \begin{ledgroup}
                \footnotesize 
                \pstart
            \noindent\footnotesize{\textbf{Datierungsgr\"{u}nde}: Bei diesem Text handelt es sich um Leibniz' fr\"{u}he, insbesondere durch Huygens und Guericke angeregte Auseinandersetzung mit Vakuumexperimenten. In Zeile 8 von Bl. 97 r\textsuperscript{o} befindet sich hinter quosdam ein Strich, der als Auslassungszeichen f\"{u}r einen Satzteil steht, dessen Anfang identisch ist mit LH XXXVII, 3 Bl. 96 r\textsuperscript{o}, Zeile 3 (N. 63). %Vermutlich hatte Leibniz diesen Satz aus N. 63 noch im Kopf, als er mit der Niederschrift unseres St\"{u}ckes begann. Wir gehen daher von dem gleichen Entstehungszeitraum aus.\\
            Vermutlich erinnerte sich Leibniz an diesen Satz aus N. 63, als er mit der Niederschrift unseres St\"{u}ckes begann. Wir gehen daher von dem gleichen Entstehungszeitraum aus.}
                \pend
                \end{ledgroup}
            
                \vspace*{8mm}
                \pstart 
                \normalsize
            [97 r\textsuperscript{o}] \selectlanguage{latin}Experimenta Pneumatica\protect\index{Sachverzeichnis}{experimentum!pneumaticum} \edtext{circa Vacuum}{\lemma{}\Afootnote{circa Vacuum \textit{ erg.} \textit{ L}}}\edtext{}{\lemma{Vacuum}\Bfootnote{\textsc{Chr. Huygens, }\cite{00062}\textit{Extrait d'une lettre}, \textit{JS} (1672), S.~133\textendash140 (\textit{HO} VII, S.~201\textendash206).}} quibus Illustris Hugenius\protect\index{Namensregister}{\textso{Huygens} (Hugenius, Vgenius, Hugens, Huguens), Christiaan 1629\textendash 1695} nuper  rem literariam auxit \edtext{admonuere}{\lemma{auxit}\Afootnote{ \textit{ (1) }\ occasio \textit{ (2) }\ admonuere \textit{ L}}} ne \edtext{quaerundam}{\lemma{quaerundam}\Afootnote{ \textbar\ ratiocinationum atque \textit{ gestr.}\ \textbar\ Experientiarum, \textit{ L}}} Experientiarum, quas  in eam rem sum dudum meditatus, et quibus sumtis  omnes in hoc negotio controversiae mihi firma  demonstratione dirimi posse videntur. \edtext{Has nunc publice proponere operae pretium mihi videtur, sed ut earum intelligatur vis ac ratio}{\lemma{videntur.}\Afootnote{ \textit{ (1) }\ Quod ut intelligatur \textit{ (2) }\ Has [...] ratio \textit{ L}}},  res altius nonnihil repetenda est. Recepta fuit in  Scholis sententia effectus quosdam \rule[1mm]{1cm}{1pt}  in infinitum.
            \pend 
            \pstart  Discipulus Galilaei\protect\index{Namensregister}{\textso{Galilei} (Galilaeus, Galileus), Galileo 1564\textendash 1642} Torricellius\protect\index{Namensregister}{\textso{Torricelli} (Torricellius), Evangelista 1608\textendash 1647} phaenomeno quod ab autore  \edtext{Torricellianum, aliis Baroscopium a mensuranda aeris gravitate}{\lemma{Torricellianum,}\Afootnote{ \textit{ (1) }\ quibusdam Baroscopium\protect\index{Sachverzeichnis}{baroscopium|textit}, (quod aeris gravitatem\protect\index{Sachverzeichnis}{gravitas!aeris|textit} mensurare \textit{ (2) }\ aliis Baroscopium a mensuranda aeris gravitate \textit{ L}}}  appellatur, detecto viam aperuit ad investigandam horum effectuum causam. Mercurius\protect\index{Sachverzeichnis}{mercurius} enim 27 \edtext{minimum aut 30 pollices altus,}{\lemma{minimum}\Afootnote{aut 30 pollices altus, \textit{ erg.} \textit{ L}}}  in Tubo Torricelliano\protect\index{Sachverzeichnis}{Tubus!Torricellianus} non tantum a summitate  avellitur, etsi nihil sensibile in ejus locum extrinsecus  succedere possit, \edtext{quemadmodum}{\lemma{possit,}\Afootnote{ \textit{ (1) }\ sed et \textit{ (2) }\ quemadmodum \textit{ L}}} Emboli\protect\index{Sachverzeichnis}{embolus} antliarum\protect\index{Sachverzeichnis}{antlia}  ab aqua nimis elevata tandem avelluntur; sed etiam  si longe altior sit his 27 aut 30 pollicibus, residuum  delabitur ex Tubo, solis his pollicibus\edtext{}{\lemma{}\Afootnote{pollicibus  \textbar\ in Tubo \textit{ gestr.}\ \textbar\ suspensis \textit{ L}}} suspensis  manentibus. Si minor sit 27. pollicibus ne avellitur  quidem. Haec \edtext{duo phaenomena}{\lemma{duo}\Afootnote{ \textbar\ ergo \textit{ erg. u.}\  \textit{ gestr.}\ \textbar\ phaenomena \textit{ L}}} antliae\protect\index{Sachverzeichnis}{antlia} et Tubi Torricelliani\protect\index{Sachverzeichnis}{Tubus!Torricellianus}  cum exacte ipsis etiam numeris concordarent, est enim Mercurius\protect\index{Sachverzeichnis}{mercurius} quaterdecies altior aqua, et proinde  etiam altitudo ejus quaterdecies aquae altitudine  minor id est 30 pollices Mercurii\protect\index{Sachverzeichnis}{mercurius}, loco 31 pedum aquae,  ad avulsionem sufficere debent, quod et experientia  confirmavit, \edtext{argumento  fuere}{\lemma{confirmavit,}\Afootnote{ \textit{ (1) }\ suspicionem rabidam inje \textit{ (2) }\ argumento  fuere \textit{ L}}} eandem utrobique subesse causam. Cumque postea  Clarus Perierius\protect\index{Namensregister}{\textso{P\'{e}rier} (Perrier, Perier, Perierius), Florin 1605\textendash 1702} \edtext{experimento}{\lemma{}\Afootnote{experimento \textit{ erg.} \textit{ L}}} ingeniosissimi Pascalii\protect\index{Namensregister}{\textso{Pascal} (Pascalius), Blaise 1623\textendash 1662} monitu \edtext{in celso quodam Arverniae Monte\protect\index{Ortsregister}{Puy de Dome@Puy de D\^{o}me (Mons Averniae, Puy de domme)} facto}{\lemma{}\Afootnote{in celso quodam Arverniae Monte\protect\index{Ortsregister}{Puy de Dome@Puy de D\^{o}me (Mons Averniae, Puy de domme)|textit} facto \textit{ erg.} \textit{ L}}}, ostendisset, minorem \edtext{altitudinem Mercurii ad avulsionem sufficere}{\lemma{minorem}\Afootnote{ \textit{ (1) }\ esse \textit{ (2) }\ altitudinem Mercurii ad avulsionem sufficere \textit{ L}}}, et ultra horizontem \edtext{alterius}{\lemma{}\Afootnote{alterius \textit{ erg.} \textit{ L}}} in subjecto vase stagnantis \edtext{eminere, cum}{\lemma{eminere,}\Afootnote{ \textit{ (1) }\ quanto \textit{ (2) }\  cum \textit{ L}}} locus est altior, jam pro certo haberi coepit \edtext{quod antea conjectura invaluerat}{\lemma{}\Afootnote{quod antea conjectura invaluerat \textit{ erg.} \textit{ L}}}, \edtext{aeris contrapondium tum aquam in Siphone\protect\index{Sachverzeichnis}{sipho}, tum Mercurium\protect\index{Sachverzeichnis}{mercurius} in Tubo Torricelliano\protect\index{Sachverzeichnis}{Tubus!Torricellianus} elevare.}{\lemma{}\Afootnote{aeris contrapondium tum aquam in Siphone\protect\index{Sachverzeichnis}{sipho}, tum Mercurium\protect\index{Sachverzeichnis}{mercurius} in Tubo Torricelliano\protect\index{Sachverzeichnis}{Tubus!Torricellianus} elevare. \textit{ erg.} \textit{ L}}}\edtext{}{\lemma{elevare.}\Bfootnote{\textsc{F. P\'{e}rier, }\cite{00127}\textit{Brief an Pascal vom 22. September 1648}, in: B. Pascal, \cite{00081}\textit{Traitez de l'\'{e}quilibre des liqueurs}, Paris 1663, S.~176\textendash188 (\textit{PO} II, S.~151\textendash158).}}
            \pend 
            \pstart  Hanc sententiam Otto Gerickius\protect\index{Namensregister}{\textso{Guericke} (Gerickius, Gerick.), Otto v. 1602\textendash 1686} in Germania\protect\index{Ortsregister}{Deutschland (Germania, Duitsland)} egregiis  experimentis confirmavit, ostendit enim non  aeris tantum\edtext{}{\lemma{}\Afootnote{tantum  \textbar\ densitatem aut \textit{ gestr.}\ \textbar\ raritatem, \textit{ L}}} raritatem,\edtext{}{\lemma{raritatem,}\Bfootnote{\textsc{O. v. Guericke, }\cite{00055}\textit{Experimenta nova}, Amsterdam 1672, S.~101.}} sed et ventos Baroscopii\protect\index{Sachverzeichnis}{baroscopium} \edtext{altitudinem minuere}{\lemma{altitudinem}\Afootnote{ \textit{ (1) }\ variare \textit{ (2) }\ minuere \textit{ L}}}.\edtext{}{\lemma{minuere.}\Bfootnote{\textsc{O. v. Guericke, }\cite{00055}a.a.O., S.~100.}} Aerem enim ventorum  motu sustentari, quemadmodum in aqua agitata videmus  corpora natare, quae in quiescente subsidunt. Idem  corpora \edtext{quaedam}{\lemma{}\Afootnote{quaedam \textit{ erg.} \textit{ L}}} sola aeris gravitate\protect\index{Sachverzeichnis}{gravitas!aeris} cohaerentia nec  24 equis divelli posse monstravit,\edtext{}{\lemma{monstravit,}\Bfootnote{\textsc{O. v. Guericke}, \cite{00055}a.a.O., S.~105.}} et Machinam  illam admirabilem Recipientis Magdeburgici\protect\index{Sachverzeichnis}{Recipiens!Magdeburgicum}\edtext{}{\lemma{admirabilem}\Bfootnote{\textsc{O. v. Guericke, }\cite{00055}a.a.O., S.~94.}} (Magdeburgi\protect\index{Ortsregister}{Magdeburg (Magdeburgum)} enim degit autor) nomine notam primus detexit qua aer ex vase aliquo hauriri potest, cujus usum  postea Illustres viri, Boylius\protect\index{Namensregister}{\textso{Boyle} (Boylius, Boyl., Boyl), Robert 1627\textendash 1691} primum in Anglia\protect\index{Ortsregister}{England (Anglia)}, deinde Hugenius\protect\index{Namensregister}{\textso{Huygens} (Hugenius, Vgenius, Hugens, Huguens), Christiaan 1629\textendash 1695} in Batavis Galliaque\protect\index{Ortsregister}{Frankreich (Gallia, Francia)} egregie promovere. Boylius\protect\index{Namensregister}{\textso{Boyle} (Boylius, Boyl., Boyl), Robert 1627\textendash 1691}\edtext{}{\lemma{promovere.}\Bfootnote{\textsc{R. Boyle, }\cite{00015}\textit{New experiments physico-mechanicall}, Oxford 1660, S.~22 (\textit{BW} I, S.~165).}} ostendit praeter Gravitatem aeris\protect\index{Sachverzeichnis}{gravitas!aeris} vim  ejus Elasticam\protect\index{Sachverzeichnis}{vis!elastica} ad effectus fugae vacui\protect\index{Sachverzeichnis}{fuga vacui} vulgo ascriptos  esse exhibendam; nam \edtext{quod}{\lemma{nam}\Afootnote{ \textit{ (1) }\ ut descensu \textit{ (2) }\ quod \textit{ L}}} in aere  libero totam columnam atmosphaerae\protect\index{Sachverzeichnis}{atmosphaera} elevat,  id aerem clausum ultra quam columna atmosphaerae\protect\index{Sachverzeichnis}{atmosphaera}  potuit, comprimere potest.\pend 
            