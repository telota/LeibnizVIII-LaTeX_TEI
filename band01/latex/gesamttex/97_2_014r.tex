      
               
                \begin{ledgroupsized}[r]{120mm}
                \footnotesize 
                \pstart                
                \noindent\textbf{\"{U}berlieferung:}   
                \pend
                \end{ledgroupsized}
            
              
                            \begin{ledgroupsized}[r]{114mm}
                            \footnotesize 
                            \pstart \parindent -6mm
                            \makebox[6mm][l]{\textit{L}}Konzept: LH XXXV 15, 1 Bl. 14, 17. 2 Bl. 9 x 23 cm und 10 x 23 cm. R\"{u}ckseiten jeweils leer. Bl. 14 r\textsuperscript{o} oberer Teil eines Folioblatts, Bl. 17 r\textsuperscript{o} unterer Teil eines Folioblatts. Beide Teile leicht schr\"{a}g abgeschnitten. Rechter Seitenrand von Bl. 17 r\textsuperscript{o} besch\"{a}digt. Dadurch geringe Textverluste.\\Cc 2 Nr. 494 A, B \pend
                            \end{ledgroupsized}
                %\normalsize
                \vspace*{5mm}
                \begin{ledgroup}
                \footnotesize 
                \pstart
            \noindent\footnotesize{\textbf{Datierungsgr\"{u}nde}: Ausz\"{u}ge aus Pascals \cite{00081}\textit{Trait\'{e}} kehren fast w\"{o}rtlich in den Texten wieder, die unmittelbar nach dem Erscheinen des Huygens'schen Briefes im \cite{00062}\textit{Journal des S\c{c}avans} vom 25. Juli 1672 entstanden sind. Dies trifft insbesondere auf den Bericht \"{u}ber Chanut zu, auf den Leibniz in N. 42 und N. 46 Bezug nimmt. Wir gehen daher von derselben Entstehungszeit aus.}
                \pend
                \end{ledgroup}
            
                \vspace*{8mm}
                \pstart 
                \normalsize
            [14 r\textsuperscript{o}] \selectlanguage{french}Trait\'{e} de l'Equilibre, des liqueurs et de la pesanteur de la masse de l'air par M. Pascal\protect\index{Namensregister}{\textso{Pascal} (Pascalius), Blaise 1623\textendash 1662}. Paris\protect\index{Ortsregister}{Paris (Parisii)} chez Guillaume Desprez\protect\index{Namensregister}{\textso{Desprez,} Guillaume 1630\textendash 1708} 1663. 12\textsuperscript{o}.\pend 
            \pstart \textso{Pascal}\protect\index{Namensregister}{\textso{Pascal} (Pascalius), Blaise 1623\textendash 1662}\textso{ de l'equilibre des liqueurs preface de sa vie.} \textit{Une fois lorsqu'il n'avoit qu'onze ans quelqu'un ayant \`{a} table sans y penser frapp\'{e}  un plat de fayence avec un cousteau, il prit garde que cela rendoit un grand }\edtext{[\textit{son}\protect\index{Sachverzeichnis}{sonus}]}{\lemma{}\Afootnote{\textit{son}\protect\index{Sachverzeichnis}{sonus} \textit{ erg.} \textit{ Hrsg. }\ }}\textit{, mais qu'aussi tost, qu'on mettoit la main dessus  il s'arrestoit. Il voulut en même temps en s\c{c}avoir la cause, et cette experience l'ayant port\'{e}, a en faire beaucoup d'autres sur les  sons, il y remarqua tant de choses, qu'il en fit un petit trait\'{e} qui fut jug\'{e} tres ingenieux et tres solide.}\edtext{}{\lemma{\textit{solide.}}\Bfootnote{\textsc{B. Pascal, }\cite{00081}\textit{Traitez de l'\'{e}quilibre des liqueurs}, Paris 1663, Preface, %\hspace{5.5pt}
             o.S. (\textit{PO} III, S.~270f.).}}\pend 
            \pstart  Mons. Pascal\protect\index{Namensregister}{\textso{Pascal} (Pascalius), Blaise 1623\textendash 1662} \textit{en l'aage de 16 ans fit un trait\'{e} de Coniques,}\edtext{}{\lemma{\textit{Coniques,}}\Bfootnote{\textsc{B. Pascal, }\cite{00083}\textit{Essay pour les coniques}, Paris 1640 (\textit{PO} I, S.~251\textendash260).}}\textit{ qui passa au jugement des plus habiles pour un des plus grands efforts d'esprit  qu'on se puisse imaginer.}\edtext{}{\lemma{\textit{imaginer.}}\Bfootnote{\textsc{B. Pascal, }\cite{00081}\textit{Traitez de l'\'{e}quilibre des liqueurs}, Preface, o.S. (\textit{PO} III, S.~273).}} \pend \clearpage
            \pstart \`{A} l'aage de 19 ans il inventa cette belle machine d'Arithmetique\protect\index{Sachverzeichnis}{machina!arithmetica}. \`{A} l'aage de 23 ans il commenca \`{a} r\'{e}ver sur l'experience de Torricelli\protect\index{Sachverzeichnis}{experimentum!Torricellianum}\protect\index{Namensregister}{\textso{Torricelli} (Torricellius), Evangelista 1608\textendash 1647}. Et il trouua enfin quelque chose sur la roulette sous le nom d'Etonville\protect\index{Namensregister}{\textso{Pascal} (Pascalius), Blaise 1623\textendash 1662}.\edtext{}{\lemma{d'Etonville.}\Bfootnote{\textsc{B. Pascal, }\cite{00116}\textit{Lettre de A. Dettonville} [d.i. Blaise Pascal] \textit{\`{a} Monsieur de Carcavy}, Paris 1658 (\textit{PO} VIII, S.~334\textendash 382 u. \textit{PO} IX, S.~3\textendash 133).}}\pend
             \pstart L'Experience des petits tuyaux est de\"{u}e \`{a} Mons. Rho.\protect\index{Namensregister}{\textso{Roannez} (Rho.), Arthur-Gouffier Duc de 1627\textendash 1696}, et la regle de Mons. Pascal\protect\index{Namensregister}{\textso{Pascal} (Pascalius), Blaise 1623\textendash 1662}, que  liqueurs pesent selon leur hauteur, \edtext{de}{\lemma{hauteur,}\Afootnote{ \textit{ (1) }\ sans \textit{ (2) }\ de \textit{ L}}} quelque largeur puisse estre le tuyau, se doit entendre: pourveu que ces tuyaux demeurent  tousjours un peu plus gros, comme de deux ou trois lignes de diametre. Car si de deux tuyaux ayant communication ensemble  l'un estoit fort menu comme de la grosseur d'une epingle, ou même un peu plus, l'eau se tiendroit plus haut dans le menu, que  dans le plus gros. Et quand même ces tuyaux fort menus sont separez l'un de l'autre en les mettant dans l'eau, on voit que  l'eau y monte et y demeure suspend\"{u}e aux uns plus haut, et aux autres plus bas, selon, qu'ils sont plus ou moins  menus quoque ils soient ouuerts par enhaut aussi bien que par enbas.\pend \pstart \textso{Pascal}\protect\index{Namensregister}{\textso{Pascal} (Pascalius), Blaise 1623\textendash 1662}\textso{. equilib. des liqu. chap. 2.} \textit{J'ay demontr\'{e} par cette methode} (que \edtext{jamais un corps}{\lemma{(que}\Afootnote{ \textit{ (1) }\ les  \textit{ (2) }\ jamais un corps \textit{ L}}} se meut par son propre poids sans  que son centre de gravit\'{e}\protect\index{Sachverzeichnis}{centre de gravit\'{e}}\protect\index{Sachverzeichnis}{centre de gravit\'{e}|see{centrum gravitatis}} descende) \textit{dans un petit trait\'{e} de Mechanique la raison de toute la multiplication des forces.}\edtext{}{\lemma{\textit{forces.}}\Bfootnote{Vgl. \textsc{B. Pascal, }\cite{00081}\textit{Traitez de l'\'{e}quilibre des liqueurs}, S.~11 (\textit{PO} III, S.~167).}}\pend