      
               
                \begin{ledgroupsized}[r]{120mm}
                \footnotesize 
                \pstart                
                \noindent\textbf{\"{U}berlieferung:}   
                \pend
                \end{ledgroupsized}
            
              
                            \begin{ledgroupsized}[r]{114mm}
                            \footnotesize 
                            \pstart \parindent -6mm
                            \makebox[6mm][l]{\textit{L}}Konzept: LH XXXVII 3 Bl. 146\textendash147. 1 Bog. 2\textsuperscript{o}. 1 1/5 S. zweispaltig. Auf Bl. 146 v\textsuperscript{o} 4 Zeilen, Bl. 147 leer.\\Cc 2, Nr. 491 E \pend
                            \end{ledgroupsized}
                \vspace*{8mm}
                \pstart 
                \normalsize
            [146 r\textsuperscript{o}] \selectlanguage{french}J'ay rapport\'{e} \edtext{et examin\'{e}}{\lemma{}\Afootnote{et examin\'{e} \textit{ erg.} \textit{ L}}} jusqu'icy des Hypotheses qui ne me semblent pas estre \`{a} l'\'{e}preuve \edtext{de l'evenement}{\lemma{}\Afootnote{de l'evenement \textit{ erg.} \textit{ L}}} des experiences, ou faites, ou ais\'{e}es \`{a} faire. Il reste \`{a} present de produire celles, que j'ay forg\'{e}es moy même \`{a} force de r\'{e}ver sur cette matiere, que je crois n'estre pas sujettes \`{a} cet inconvenient. Je confesse pourtant qu'elles sont toutes imparfaites, et qu'il y reste tousjours quelque chose d'inexplicable par ce, que nous s\c{c}avons, jusqu'\`{a} l\`{a}, \edtext{de la Physique; sans cela, leur resolution parfaite en des premiers Elements d'une Hypothese generalle est purement mechanique, nous les feroit peut estre reduire toutes \`{a} une seule.}{\lemma{l\`{a},}\Afootnote{ \textit{ (1) }\ de la nature \textit{ (2) }\ de [...] nous \textit{(a)}\ monstreroit \textit{(b)}\ les [...] seule. \textit{ L}}} Mais ce n'est pas merveille, la Nature ne montrant pas tout \`{a} la fois toutes les maximes de sa conduite \edtext{mysterieuse}{\lemma{conduite}\Afootnote{ \textit{ (1) }\ cach\'{e}e \textit{ (2) }\ mysterieuse \textit{ L}}}. Il vaut \edtext{pourtant}{\lemma{}\Afootnote{pourtant \textit{ erg.} \textit{ L}}} mieux d'avancer seur\'{e}ment de deux pieds, que de s'\'{e}garer de plusieurs lie\"{u}es; de s'emparer pour tousjours d'une forteresse, quoyque peu considerable que de ravager en vain des provinces. Un autre aura le bonheur de passer au dela de nos \edlabel{conq146r1}conquêtes.\pend
             \pstart \edtext{\edlabel{conq146r2}}{\lemma{}\xxref{conq146r1}{conq146r2}\Afootnote{conquêtes.  \textbar\ Au reste je crois, que la derniere Resolution des Hypotheses, que je vais proposer \textit{ erg. u.}\  \textit{ gestr.}\ \textbar\ Pour \textit{ L}}} Pour le dire en general, il est \`{a} croire que \edtext{ce sera peut estre}{\lemma{}\Afootnote{ce sera peut estre \textit{ erg.} \textit{ L}}} le dernier Ressort en cette Matiere, apres avoir essay\'{e} tout en vain, d'avoir recours en partie \`{a} l'explication des anciens, qui estoit en usage devant qu'on ait entendu parler de l'Experience\protect\index{Sachverzeichnis}{exp\'{e}rience de Torricelli}  de Torricelli\protect\index{Namensregister}{\textso{Torricelli} (Torricellius), Evangelista 1608\textendash 1647}, et de la machine de M. Guericke\protect\index{Namensregister}{\textso{Guericke} (Gerickius, Gerick.), Otto v. 1602\textendash 1686}. S\c{c}avoir que deux placques ne peuuent estre separ\'{e}es, \edtext{même dans le Recipient \'{e}puis\'{e},}{\lemma{}\Afootnote{même dans le Recipient \'{e}puis\'{e}, \textit{ erg.} \textit{ L}}} ny des liqueurs purg\'{e}es\protect\index{Sachverzeichnis}{liqueur!purg\'{e}e}, detach\'{e}es du Tuyau, sans qu'une matiere remplisse en même temps l'espace entre deux placques ou entre la \textso{liqueur et la superficie} interieure du Tuyau, qui demeureroit vuide sans cela. Car quand les liqueurs\protect\index{Sachverzeichnis}{liqueur!purg\'{e}e} ne sont pas purg\'{e}es d'air; l'air se pouuant dilater (de quelque maniere que se fasse cette dilatation) remplira l'espace \edtext{qui doit estre}{\lemma{}\Afootnote{qui doit estre \textit{ erg.} \textit{ L}}} quitt\'{e}, et le corps tombera, \`{a} moins que la pesanteur\protect\index{Sachverzeichnis}{pesanteur} de l'atmosphere\protect\index{Sachverzeichnis}{atmosph\`{e}re} ne l'empeche. Cela pos\'{e}, \edtext{il se presentera deux chemins, pour aller plus outre}{\lemma{pos\'{e},}\Afootnote{ \textit{ (1) }\ il y a deux facons de passer plus outr \textit{ (2) }\ il [...] pour \textit{(a)}\ passer \textit{(b)}\ aller plus outre \textit{ L}}} \edtext{selon les manieres differentes d'expliquer la dilatation}{\lemma{}\Afootnote{selon [...] dilatation \textit{ erg.} \textit{ L}}}, car %\edtext{quelques uns croyent, qu'un corps peut remplir plus au moins de place, sans recevoir ou perdre qui que ce soit, que la place: et ceux pensent de n'avoir point besoin,}{\lemma{car}\Afootnote{ \textit{ (1) }\ o\`{u} l'on a besoin de faire passer  \textit{(a)}\ que \textit{(b)}\ par la solidit\'{e} du vase des autre \textit{ (2) }\ quelques [...] recevoir \textit{(a)}\ quelque chose \textit{(b)}\ ou [...] ceux \textit{(aa)}\ n'ont \textit{(bb)}\ pensent de n'avoir point besoin,  \textit{ L}}}