   
        
        \begin{ledgroupsized}[r]{120mm}
        \footnotesize 
        \pstart        
        \noindent\textbf{\"{U}berlieferung:}  
        \pend
        \end{ledgroupsized}
      
       
              \begin{ledgroupsized}[r]{114mm}
              \footnotesize 
              \pstart \parindent -6mm
              \makebox[6mm][l]{\textit{LiA}}Marginalien zu einer Abschrift von Schreiberhand: LH XXXV 15, 6 Bl. 20\textendash 21. 1 Bog. 8\textsuperscript{o}. 1~1/4 S. auf Bl. 20. Bl. 21 r\textsuperscript{o} N. 8, Bl. 21 v\textsuperscript{o} leer. In der rechten unteren Ecke von Bl. 20 r\textsuperscript{o} ein unaufgel\"{o}stes K\"{u}rzel, m\"{o}glicherweise das Signet des Schreibers.\\Cc 2, Nr. 1556 A, B \pend
              \end{ledgroupsized}
        %\normalsize
        \vspace*{5mm}
        \begin{ledgroup}
        \footnotesize 
        \pstart
      \noindent\footnotesize{\textbf{Datierungsgr\"{u}nde}: Es handelt sich um eine Abschrift derselben Schreiberhand, von der auch der Text N. 6\raisebox{-0.5ex}{\tiny{2}} \textit{De longitudinum determinatione. Scheda secunda} \"{u}berliefert ist. Da zudem inhaltlich verwandte Themen diskutiert werden, d\"{u}rften die St\"{u}cke N. 6 und N. 7 etwa in derselben Zeit entstanden sein.}
        \pend
        \end{ledgroup}
      
        \vspace*{8mm}
        \pstart 
        \normalsize
      [20 r\textsuperscript{o}] \selectlanguage{french}\textso{1.} S\c{c}achant la hauteur du Pole\protect\index{Sachverzeichnis}{hauteur du pole} et de la declinaison du soleil\protect\index{Sachverzeichnis}{declinaison du soleil}, ou son lieu dans le Zodiaque\protect\index{Sachverzeichnis}{zodiaque}, trouuer quelle heure il est aux rayons du soleil, et sur mer\protect\index{Sachverzeichnis}{mer}, et sur terre\protect\index{Sachverzeichnis}{terre}.\footnote{\textit{Am oberen Rand:} \selectlanguage{french}\`{A} M. Piget\protect\index{Namensregister}{\textso{Piget,} Sim\'{e}on, bekannt seit 1639} Marchand libraire. L'auteur demande, qu'il face voir cela aux Mathematiciens de Paris\protect\index{Ortsregister}{Paris (Parisii)}, pour dire si l'instrument qu'il a invent\'{e}, et sur lequel on pourra practiquer cela merite d'estre mis au jour. [\textit{Von Leibniz' Hand.}]}\pend
      \pstart \textso{2.} La declinaison de l'aiguille Aymant\'{e}e\protect\index{Sachverzeichnis}{aiguille aimant\'{e}e} estant conn\"{u}e auec la declinaison du soleil\protect\index{Sachverzeichnis}{declinaison du soleil}, ou son etc. trouuer la hauteur\protect\index{Sachverzeichnis}{hauteur du pole} du Pole, et sur mer\protect\index{Sachverzeichnis}{mer}, et sur terre\protect\index{Sachverzeichnis}{terre}, par une seule obseruation, auant, et apres midy, aussi bien qu'\`{a} midy.\pend \pstart \textso{3.} Ayant la hauteur du Pole\protect\index{Sachverzeichnis}{hauteur du pole}, auec la declinaison du soleil\protect\index{Sachverzeichnis}{declinaison du soleil} etc. par une seule obseruation, trouuer sur mer\protect\index{Sachverzeichnis}{mer} et sur terre\protect\index{Sachverzeichnis}{terre} la declinaison de l'aiguille Aymant\'{e}e\protect\index{Sachverzeichnis}{aiguille aimant\'{e}e}, et l'heure courante. On trouue aussy en mesme temps l'heure courante par la 2. proposition.\pend \newpage \pstart J'obmets autres trois propositions qui ne laissent pas d'estre belles et curieuses quoy qu'elles ne soyent pas si importantes que cellescy. Je supplie les Messieurs qui prendront la peine de lire cecy, de mettre cy dessoubs leur tesmoignage aussy briefvement qu'il leur plaira. J'ay encore un autre instrument uniuersel pour cognoistre partout quelle heure il est, lequel J'inuentay en mesme temps que le premier, et que je pretends donner ensemble, auec quelques autres inuentions astronomiques. Le nom de mon premier Instrument sera tel.\pend \pstart La M\^{o}ntre\protect\index{Sachverzeichnis}{montre} Uniuerselle Equinoctialle et Polaire.\protect\index{Sachverzeichnis}{aiguille aimant\'{e}e}\footnote{\textit{Auf der R\"{u}ckseite:} Si l'auteur a trouu\'{e} une maniere de regler les declinaisons de l'\'{e}guille aimant\'{e}e, ce seroit une chose de grande importance. Autrement trouuer la declinaison de l'aimant par l'observation du ciel, et trouuer la ligne meridienne, c'est la même chose. [\textit{Von Leibniz' Hand.}]}\pend \pstart La M\^{o}ntre\protect\index{Namensregister}{\textso{La Montre,} Abb\'{e} de}. $\langle$bre.$\rangle$\edtext{}{\linenum{|11|||11|}\lemma{importance}\Afootnote{ \textit{ (1) }\ Mais s'il trouue l \textit{ (2) }\ Autrement trouuer la \textit{ L}}}\selectlanguage{latin}\pend 