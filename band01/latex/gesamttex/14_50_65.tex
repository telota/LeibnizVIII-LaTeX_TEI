\pend \pstart \selectlanguage{latin} [p.~65] IV. Alij ex hoc praedictam rationem deducunt, quod scilicet vnus tantum angulus aequalis angulo incidentiae\protect\index{Sachverzeichnis}{angulus!incidentiae}\protect\index{Sachverzeichnis}{angulus!incidentiae|see{angolo di incidenza}}\protect\index{Sachverzeichnis}{angulus!incidentiae|see{angle d'incidence}} respondeat, sint vero infiniti inaequales; cur autem potius per vnum inaequalem, quam per alium? ab vno igitur fit determinatio; nempe quod vnum est, determinatum est, sed in dicto puncto C, vna tantum perpendicularis surgit, et aliae infinitae; cur igitur per illam quae vna est, radius reflexus\protect\index{Sachverzeichnis}{radius!reflexus} non ibit?\footnote{\textit{Am Rand mit Tinte}: Responsio manifesta, ne diversae causae eundem faciant effectum.} praesertim cum in motu reflexo\protect\index{Sachverzeichnis}{motus!reflexus}, noua determinatio, quae a puncto reflectente accedit, in ipsa perpendiculari fiat, ex qua et priore componitur mixta, vt suo loco demonstratum est.\selectlanguage{latin}