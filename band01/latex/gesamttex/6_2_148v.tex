 \edtext{Car [148 v\textsuperscript{o}] je}
 {\lemma{Car}\Afootnote{ \textit{ (1) }\ premierement \textit{ (2) }\  je \textit{ L}}} ne comprends pas \edtext{assez pourquoy}{\lemma{pas}\Afootnote{ \textit{ (1) }\ pourquoy si \textit{ (2) }\ assez pourquoy \textit{ L}}} cette matiere, ne passe ais\'{e}ment, que par l'air; et encor moins, pourquoy un fluide plus subtil que l'air, sans comparaison, ne puisse passer, o\`{u} une bulle d'air trouuera passage. Car si la liqueur purg\'{e}e\protect\index{Sachverzeichnis}{liqueur!purg\'{e}e} est assez serr\'{e}e, pour exclure ce fluide nouueau, elle \edtext{ne sera pas assez}{\lemma{elle}\Afootnote{ \textit{ (1) }\ sera aussi \textit{ (2) }\  ne sera pas assez \textit{ L}}} ouuerte pour une bulle d'air. Je s\c{c}ais bien qu'on se pourroit servir de l'exemple de l'eau, de huyle, et du Mercure même, qui passent \edtext{par certains corps avec plus de facilit\'{e}}{\lemma{passent}\Afootnote{ \textit{ (1) }\ plus  \textit{(a)}\ ais\'{e} \textit{(b)}\ faci \textit{ (2) }\ par [...] facilit\'{e} \textit{ L}}} que l'air. Mais cela n'est pas satisfaire l'esprit. \edtext{Outre qu'il est peu vraysemblable qu'une grande quantit\'{e} de Mercure purg\'{e}\protect\index{Sachverzeichnis}{mercure!purg\'{e}} suspendu se laisse penetrer et comme enfler en un moment par une petite bulle d'air, qui passe par tous ses pores, comme un \'{e}clair: estant constant d'ailleurs qu'il n'est pas facile \`{a} l'air de passer par le Mercure.}{\lemma{}\Afootnote{Outre [...] suspendu  \textit{ (1) }\ dans \textit{ (2) }\ se [...] moment \textit{ (a) }\ en forme d'éclair qui passe par tous ses \textit{ (b) }\ par [...] pas \textit{ (aa) }\ ais\'{e} \textit{ (bb) }\ facile [...] Mercure. \textit{ erg.} \textit{ L}}} Pourtant si je ne trouuois rien \`{a} \edtext{redire que cela}{\lemma{\`{a}}\Afootnote{ \textit{ (1) }\ dire \textit{ (2) }\ redire  \textit{(a)}\ d'avantage \textit{(b)}\ que cela \textit{ L}}}, je mettrois cette Hypothese au rang de celles, dont je parleray bientost, lesquelles, quoyque imparfaites, sont pourtant au dessus de l'evenement des Experiences \edtext{ais\'{e}es}{\lemma{}\Afootnote{ais\'{e}es \textit{ erg.} \textit{ L}}} dont je me puis aviser \edtext{pour les convaincre entierement}{\lemma{}\Afootnote{pour les convaincre entierement \textit{ erg.} \textit{ L}}}. 
 \edtext{\edlabel{mais148v1}}{\lemma{\textit{Von} \textso{Mais} \textit{bis} \textso{suspend\"{u}e.}}\xxref{mais148v1}{mais148v2}\Afootnote{\textit{Markierung am Rand}}} 
 \textso{Mais cela ne se fera pas, que quand elle aura soûtenu le choc de deux experiences}\edtext{\textso{. La premiere est}}{\lemma{\textso{experiences}}\Afootnote{ \textit{ (1) }\ \textso{, que je m'en vais proposer dont la premiere} \textit{ (2) }\ \textso{. J'ay} \textit{ (3) }\ \textso{. La premiere est} \textit{ L}}}\textso{, de faire monter la bulle }\edtext{\textso{dans la}}{\lemma{\textso{bulle}}\Afootnote{ \textit{ (1) }\ \textso{au milieu de} \textit{ (2) }\ \textso{ dans la} \textit{ L}}}\textso{ phiole }\textit{\textso{A-CD}}\edtext{\textso{ (fig. 1.)}}{\lemma{\textso{fig. 1}}\Afootnote{\textit{unterstrichen}}}\textso{ avec cette precaution, qu'elle ne touche pas la superficie interieure de la phiole, en montant. Ce qui }\edtext{\textso{se pourra}}{\lemma{\textso{qui}}\Afootnote{ \textit{ (1) }\ \textso{sera} \textit{ (2) }\ \textso{se pourra} \textit{ L}}}\textso{ practiquer ais\'{e}ment \`{a} ce que j'ay dit cy dessus, en plantant un fil de fer ou petit bâton mince au milieu de la phiole, auquel la bulle soit appuy\'{e}e en montant. Alors si la liqueur ne tombe pas, quoyque }\edtext{\textso{la bulle}}{\lemma{\textso{la}}\Afootnote{ \textit{ (1) }\ \textso{liqueur} \textit{ (2) }\ \textso{bulle} \textit{ L}}}\textso{ ait pass\'{e} }\textit{\textso{B}}\textso{ jusqu'\`{a} ce qu'elle rencontre la superficie interieure du tuyau, nous pourrons estre asseurez de la fausset\'{e} de l'Hypothese: la bulle estant au milieu de la liqueur serr\'{e}e, suspend\"{u}e par la force nouuelle, sans ouurir pourtant le passage au fluide\linebreak nouueau, }\edtext{\textso{contre ce qu'on avoit suppos\'{e}.}}{\lemma{}\Afootnote{\textso{contre ce qu'on avoit suppos\'{e}.} \textit{ erg.} \textit{ L}}}\textso{ La deuxieme Experience n'est pas moins convainquante. La }\textso{liqueur purg\'{e}e}\protect\index{Sachverzeichnis}{liqueur!purg\'{e}e}\textso{ estant tomb\'{e}e par l'arriv\'{e}e de la bulle, ayons soins de reprendre la même liqueur }\edtext{\textso{pr\'{e}cisement, qui s'est}}{\lemma{\textso{liqueur}}\Afootnote{ \textit{ (1) }\ \textso{incontinent} \textit{ (2) }\ \textso{pr\'{e}cisement, qui s'est} \textit{ L}}}\textso{ ecoul\'{e}e en partie, de la phiole, ou du tuyau, et qui reste en partie la dedans, et de l'employer incontinent, pour refaire l'experience de la }\textso{liqueur purg\'{e}e}\protect\index{Sachverzeichnis}{liqueur!purg\'{e}e}\textso{ suspend\"{u}e: que si elle sert encor une fois }\edtext{\textso{sans estre purg\'{e}e de nouueau}}{\lemma{}\Afootnote{\textso{sans estre purg\'{e}e de nouueau} \textit{ erg.} \textit{ L}}}\textso{, l'Hypothese est refut\'{e}e, car si }\edtext{\textso{la liqueur}}{\lemma{\textso{si}}\Afootnote{ \textit{ (1) }\ \textso{les pores} \textit{ (2) }\ \textso{la liqueur} \textit{ L}}}\textso{ a est\'{e} ouuerte par la bulle, elle sera encor\linebreak ouuerte, parce que rien l'a serr\'{e}e de nouueau. Et si elle n'est pas serr\'{e}e, le fluide suppos\'{e} passera sans peine, sans la soûtenir suspend\"{u}e.}\edlabel{mais148v2}