[149 r\textsuperscript{o}] Devant que de passer aux Hypotheses, qui me semblent au dessus de l'evenement des experiences ais\'{e}es, je proposeray une Experience, mais un peu difficile, laquelle nous en pourroit faire juger, si elle \edtext{se trouueroit faisable}{\lemma{elle}\Afootnote{ \textit{ (1) }\ estoit faisab \textit{ (2) }\ se trouueroit faisable \textit{ L}}} pour cet effect, \edtext{ }{\lemma{effect,}\Afootnote{ \textit{ (1) }\ employons un tuyau  \textit{(a)}\ etroit, \`{a} l'ordinaire des barometres\protect\index{Sachverzeichnis}{barom\`{e}tre} \textit{(b)}\ de la largeur \textit{(c)}\ assez large \textit{(d)}\ d'une hauteur  \textit{(aa)}\ practicable \textit{(bb)}\ considerable, \`{a} laquelle l' ouvrier puisse arriver, ou pour le soûlager de cette peine, \textit{ (2) }\ \textbar~en \textit{erg.}\ \textbar\ faisons joindre deux ou trois \`{a} la flamme de la lampe   \textbar\ ou avec un bon ciment \textit{ erg.}\ \textbar\  \textit{ (3) }\ remplissons ce tuyau ouvert par un bout de Mercure purg\'{e}\protect\index{Sachverzeichnis}{mercure!purg\'{e}|textit} \textbar\ d'air \textit{erg.}\ \textbar\ dont il nous faut une provision notable et en le renversant prenons garde, qu'il ne se casse pas par la pesanteur du mercure\protect\index{Sachverzeichnis}{mercure|textit}. Ce qui s'evitera ais\'{e}ment, en le tenant couch\'{e} \textbar\ ou \textit{erg. u. gestr.}\ \textbar\ sur un ais \`{a} l'ordinaire des barometres\protect\index{Sachverzeichnis}{barom\`{e}tre|textit}. Cependant jusqu'\`{a} ce qu'il soit renvers\'{e} et debout, \textbar\ \textit{(1)}\ et \textit{(2)}\ trempant par le bout ouuert dans un vase p \textit{erg. u. gestr.} \textbar  , on bouchera  \textit{(a)}\ l'ouuerture \textit{(b)}\ le bout ouuert avec la main, ou autrement,  \textit{(aa)}\ et relachant un peu \textit{(bb)}\ et par apres ostant la main, voyons si le Mercure purg\'{e}\protect\index{Sachverzeichnis}{mercure!purg\'{e}|textit} tombe par sa pesanteur \textit{ (4) }\    \textbar\ faisons premierement l'experience de Mons. Boyle\protect\index{Namensregister}{\textso{Boyle} (Boylius, Boyl., Boyl), Robert 1627\textendash 1691}, du Mercure purg\'{e} d'air suspendu dans un tuyau plein, de la  hauteur de 75. pouces, et par apres tachons d'augmenter la hauteur sans changer de Tube \textit{ gestr.}\ \textbar\  \textit{ L}}} [\textit{Satz bricht ab}] \pend 