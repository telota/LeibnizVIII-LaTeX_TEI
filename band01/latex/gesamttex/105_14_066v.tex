[66 v\textsuperscript{o}] sideris cujusdam elevatione ultra Horizontem loci navis\protect\index{Sachverzeichnis}{navis} observata, quae certe instrumentis bonis, \edtext{(qualia in casu necessitatis fieri possent ad quartas usque minutorum partes, et ultra divisa, et vero ratio est instrumenta communia ab uno observatore tractabilia, ad decimas usque minutorum secundorum partes subdividendi)}{\lemma{bonis,}\Afootnote{ \textit{ (1) }\ ad quartas usque minutorum partes, et ultra divisis, \textit{ (2) }\ (qualia in casu necessitatis fieri possent ad quartas usque minutorum secundorum partes, et ultra divisa) \textit{ (3) }\ (qualia [...] subdividendi) \textit{ L}}} satis recte sumi potest, perficitur; neque enim nisi angulo indigemus, quem radius e sidere\protect\index{Sachverzeichnis}{sidus} dato ductus facit ad loci horizontem, quanquam exactitudine tanta ne opus quidem sit, et tanto quidem minus, quanto locus est ab aequatore\protect\index{Sachverzeichnis}{aequator} remotior, cum in aequatore ipso error minuti \edtext{gradus longitudinum\protect\index{Sachverzeichnis}{longitudo}}{\lemma{}\Afootnote{gradus longitudinum\protect\index{Sachverzeichnis}{longitudo} \textit{ erg.} \textit{ L}}} non sit major quam miliarii Italici, qualis profecto in mari est nullius momenti. Nec a \textso{refractionibus}\protect\index{Sachverzeichnis}{refractio} metuere nobis magnopere debemus, praeterquam enim quod sideris\protect\index{Sachverzeichnis}{sidus} ultra horizontem satis evecti refractio\protect\index{Sachverzeichnis}{refractio} minus turbat, et tabula etiam computandarum refractionum\protect\index{Sachverzeichnis}{refractio}, ex crepusculorum quantitate aliisque indiciis ab Astronomis condita est; praeter inquam haec omnia en facilem occurrendi rationem. Si eodem tempore duo pluraque sidera\protect\index{Sachverzeichnis}{sidus} \edtext{diversae super horizontem elevationis}{\lemma{}\Afootnote{diversae super horizontem elevationis \textit{ erg.} \textit{ L}}} (semper enim plures fixae\protect\index{Sachverzeichnis}{stella!fixa} simul videntur) observentur, cum enim earum refractionem\protect\index{Sachverzeichnis}{refractio} necesse sit esse diversam, sese mutuo corrigent observationes, quae cum eodem tempore fiant, \textso{uni observationi} aequipollent. Utile autem est sidus eligi prae caeteris, quod alte supra horizontem loci assurget\edtext{, ejus enim Refractio\protect\index{Sachverzeichnis}{refractio} exigua est; et stellae\protect\index{Sachverzeichnis}{stella} verticalis, nulla}{\lemma{}\Afootnote{BITTE UEBERPRUEFEN!!! , ejus enim Refractio\protect\index{Sachverzeichnis}{refractio} exigua est; et   \textbar\ stellae\protect\index{Sachverzeichnis}{stella} \textit{ erg.}\ \textbar\  verticalis, nulla \textit{ erg.} \textit{ L}}}. \pend \pstart Problema ergo meum ita concipitur: \textso{Data }\textso{latitudine}\protect\index{Sachverzeichnis}{latitudo}\textso{ }\edtext{\textso{ loci }\textso{navis}\protect\index{Sachverzeichnis}{navis}}{\lemma{}\Afootnote{\textso{ loci }\textso{navis}\protect\index{Sachverzeichnis}{navis}  \textbar\ \textso{ac linea meridiana} \textit{ gestr.}\ \textbar\   \textit{ erg.} \textit{ L}}}\textso{ }\textso{Horologioque}\protect\index{Sachverzeichnis}{horologium}\textso{ exacto, et accedente unica quacunque, }\textso{sideris}\protect\index{Sachverzeichnis}{sidus}\textso{ (motus explorati) cujuscunque (in horizonte }\textso{navis}\protect\index{Sachverzeichnis}{navis}\textso{, motum satis sensibilem habentis, }\edtext{\textso{et}}{\lemma{habentis,}\Afootnote{ \textit{ (1) }\ \textso{seu} \textit{ (2) }\ \textso{et} \textit{ L}}}\textso{ }\textso{ polo}\protect\index{Sachverzeichnis}{polus}\textso{ non nimis vicini) observatione, }\textso{Longitudines}\protect\index{Sachverzeichnis}{longitudo}\textso{, ac per consequens locum }\textso{navis}\protect\index{Sachverzeichnis}{navis}\textso{ reperire.}\pend \pstart Turbare non debet quod hoc loco ad inventionem longitudinum\protect\index{Sachverzeichnis}{longitudo}, latitudines\protect\index{Sachverzeichnis}{latitudo} inventas praerequirimus; nam alioquin latitudinum\protect\index{Sachverzeichnis}{latitudo} quoque inventio jam tum necessaria est ad cursum navis\protect\index{Sachverzeichnis}{navis} gubernandum, et ut latitudo\protect\index{Sachverzeichnis}{latitudo} sine longitudine\protect\index{Sachverzeichnis}{longitudo}, ita contra longitudo\protect\index{Sachverzeichnis}{longitudo} quoque sine latitudine\protect\index{Sachverzeichnis}{latitudo} non sufficit. Ut taceam hoc loco spem esse magnam, inveniri posse, aut ad perfectionem deduci quam primum inventionem latitudinis\protect\index{Sachverzeichnis}{latitudo} seu elevationis Poli\protect\index{Sachverzeichnis}{elevatio!poli} universalem, ab omni  