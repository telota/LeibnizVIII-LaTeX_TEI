[16 v\textsuperscript{o}] \textso{ptricas} appellavi, quibus objecta aequiapparentia, superficies \edtext{ordinate}{\lemma{}\Afootnote{ordinate \textit{ erg.} \textit{ L}}} refringentes aut reflectentes, denique \edtext{imago}{\lemma{denique}\Afootnote{ \textit{ (1) }\ loco \textit{ (2) }\ foco \textit{ (3) }\ imago \textit{ L}}} ejusdem \edtext{objecti; focique ejusdem puncti}{\lemma{ejusdem}\Afootnote{ \textit{ (1) }\ puncti, \textit{ (2) }\ objecti; focique ejusdem puncti \textit{ L}}} (nullius enim puncti focus\protect\index{Sachverzeichnis}{focus} unus, nullius objecti unica imago est) circumscriberentur aut connecterentur. Ita differentia punctorum ejusdem objecti inter se, quod ad focum\protect\index{Sachverzeichnis}{focus} projiciendum attinet, ingens, discriminisque ratio atque illud simul apparuit, \edtext{cujus apertura}{\lemma{apparuit,}\Afootnote{ \textit{ (1) }\ quod nos \textit{ (2) }\ aucta \textit{ (3) }\ apertura \textit{ (4) }\ cujus apertura \textit{ L}}} \edtext{}{\lemma{}\Afootnote{apertura  \textbar\ tam \textit{ gestr.}\ \textbar\ exigua \textit{ L}}}exigua reddita, tot radios velut inutiles excludere cogamur. \pend \pstart  Reperto mali fonte, remedium sponte patuit, \textso{inventumque est a me }\textso{Lentium}\protect\index{Sachverzeichnis}{lens}\textso{, quas quia }\edtext{\textso{quantamcumque}}{\lemma{\textso{quia}}\Afootnote{ \textit{ (1) }\ \textso{maximam} \textit{ (2) }\ \textso{quantamcumque} \textit{ L}}}\textso{ aperturam ferunt,  Pandochas appellare }soleo \edtext{novum nec ab ullo quod sciam tactum genus,}{\lemma{soleo}\Afootnote{ \textit{ (1) }\ novum genus \textit{ (2) }\ quae certis \textit{ (3) }\ genus pene profecto \textit{ (4) }\ novum [...] genus, \textit{ L}}} cujus species variae una \edtext{}{\lemma{}\Afootnote{una  \textbar\ omnium \textit{ gestr.}\ \textbar\ simplicissima \textit{ L}}}simplicissima figurae \edtext{sic satis}{\lemma{figurae}\Afootnote{ \textit{ (1) }\ pars facile \textit{ (2) }\ sic satis \textit{ L}}} parabilis, ex qua caeterae pro commoditate mutilatae. \pend \pstart   Omnium autem commune est, nullo distantiae figuraeque objecti aut fundi excipientis discrimine, ut omnia objecti puncta non minus distincte repraesententur ac si unumquodque eorum in axe optico \protect\index{Sachverzeichnis}{axis!opticus} esset, quod hactenus in mentem venit nulli. \pend \pstart  Quantus \edtext{sit hujus inventi}{\lemma{Quantus}\Afootnote{ \textit{ (1) }\ ingens \textit{ (2) }\ incredibilis autem \textit{ (3) }\ sit hujus inventi \textit{ L}}} fructus \edtext{neminem}{\lemma{fructus}\Afootnote{ \textit{ (1) }\ unicuique \textit{ (2) }\ neminem \textit{ L}}} Opticae intelligentem latet. Constat enim magnitudinem quidem apparentem posse \edtext{vitris}{\lemma{}\Afootnote{vitris \textit{ erg.} \textit{ L}}} augeri in infinitum, sed ea aucta deminui lucem\protect\index{Sachverzeichnis}{lux}. \edtext{Unde ut nunc sunt lentes defectu lucis, in augenda magnitudine apparente parci esse debemus. Si vero aperturas maximas adhibere liceret,}{\lemma{lucem.}\Afootnote{ \textit{ (1) }\ Sed \textit{ (2) }\ Sed si aperturas maximas facere liceret \textit{ (3) }\ Unde [...] liceret, \textit{ L}}} cum radii quoque futuri sint proportione plures, ac proinde lux\protect\index{Sachverzeichnis}{lux} major, poterunt radii quoque in majorem amplitudinem imaginis salva luce\protect\index{Sachverzeichnis}{lux} et distinctione impune refringi. \pend 