\pend \pstart[90 v\textsuperscript{o}] Hoc sic ostendo, augeatur (vel minuatur) utrobique aequaliter pondus \textit{ab}  vel \textit{fd} ajo volumina post absolutam pressionem seu post aequilibrium\protect\index{Sachverzeichnis}{aequilibrium}  producta, fore semper proportionalia,  quantacunque aut quantulacunque sint pondera dummodo  utrobique aequalia.\edlabel{prod90v2}
Ac proinde Volumina esse,  mensuram constantem differentiae specierum sumtis ponderibus\protect\index{Sachverzeichnis}{pondus} aequalibus.\edlabel{aequ90v1}\pend 
\pstart \edtext{\edlabel{aequ90v2}Nam si  duae sint potentiae aequales}{\lemma{aequalibus.}\xxref{aequ90v1}{aequ90v2}\Afootnote{ \textit{ (1) }\ Est enim Vis Elastica\protect\index{Sachverzeichnis}{vis!elastica|textit} conatus\protect\index{Sachverzeichnis}{conatus|textit} ad mutandum Volumen, \textit{ (2) }\  Ergo si aequalis \textit{ (3) }\ Si duae sint po \textit{ (4) }\ Nam [...] aequales \textit{ L}}} (hoc loco materia \textit{bc}  et \textit{de}. Sustinent enim aequalia pondera\protect\index{Sachverzeichnis}{pondus} \textso{per hypoth.}  et sunt ut pondera\protect\index{Sachverzeichnis}{pondus} per prop. 1.) et effectus \edtext{(hoc loco Volumina producta)}{\lemma{}\Afootnote{(hoc loco Volumina producta) \textit{ erg.} \textit{ L}}} caeteris  licet omnibus paribus sint \edtext{inaequales}{\lemma{inaequales}\Afootnote{\textbar\ hoc loco Voluminis \textit{ erg. u.}\  \textit{ gestr.}\ \textbar\ , necesse \textit{ L}}}, necesse est conatus\protect\index{Sachverzeichnis}{conatus} potentiarum inaequales esse. Erunt proinde conatus\protect\index{Sachverzeichnis}{conatus} ut effectus. Semper enim effectus sunt  in ratione conatuum\protect\index{Sachverzeichnis}{conatus} caeteris paribus; \edtext{quia ducuntur}{\lemma{paribus;}\Afootnote{ \textit{ (1) }\ ducuntur enim \textit{ (2) }\ quia ducuntur \textit{ L}}} conatus\protect\index{Sachverzeichnis}{conatus} per omnia actionis momenta\protect\index{Sachverzeichnis}{momentum}. Jam conatus\protect\index{Sachverzeichnis}{conatus}  mutandi.\footnote{\textit{Unterhalb des mit} Jam conatus mutandi \textit{endenden Absatzes}: Cogitandum an non \edtext{aequalitas}{\lemma{spatia.}\Afootnote{ \textit{ (1) }\ ratio \textit{ (2) }\ aequalitas \textit{ L}}} sumenda a Volumine mutando  aequali ad mutatum.} \edtext{Vires Elasticae diversarum specierum, sunt ut  Volumina earum naturalia}{\lemma{mutandi.}\Afootnote{ \textit{ (1) }\ Si corpora Elastica\protect\index{Sachverzeichnis}{corpus!elasticum|textit} habent  \textit{(a)}\ conatus\protect\index{Sachverzeichnis}{conatus|textit} \textit{(b)}\ Volumina naturalia \textit{ (2) }\ corpora Elastica\protect\index{Sachverzeichnis}{corpus!elasticum|textit} \textit{ (3) }\ Vires [...] naturalia \textit{ L}}}, seu quibus acquisitis  quiescunt. \edtext{Si}{\lemma{quiescunt.}\Afootnote{ \textit{ (1) }\ Si quae habent, caeteris paribus \textit{ (2) }\  Si \textit{ L}}} duo corpora\protect\index{Sachverzeichnis}{corpus!elasticum} \edtext{Elastica}{\lemma{}\Afootnote{Elastica \textit{ erg.} \textit{ L}}} \edtext{eodem momento}{\lemma{Elastica}\Afootnote{ \textit{ (1) }\ occupent idem Volumen, et eodem  tempore \textit{ (2) }\ eodem momento \textit{ L}}}  relaxentur, \edtext{erunt Vires Elasticae  ut tempora restitutionum}{\lemma{relaxentur,}\Afootnote{ \textit{ (1) }\ nec  \textit{(a)}\ ullo impedimento graventur: \textit{(b)}\ ullum impe \textit{ (2) }\  ab omni restitutionis impedimento \textit{ (3) }\ motus  restitutionum erunt proportionales, seu si unum \textit{ (4) }\  Volumina   \textbar\ continue quaesita \textit{ erg.}\ \textbar\  erunt proportionalia durante toto tempore  restitutionis, seu ut sunt \textit{ (5) }\ erunt [...] restitutionum \textit{ L}}}.\pend \pstart  Si duo corpora Elastica\protect\index{Sachverzeichnis}{corpus!elasticum} ferant pondus\protect\index{Sachverzeichnis}{pondus} idem eodem  tempore in spatia diversa, erunt ut spatia. \edtext{}{\lemma{non}\linenum{|8|||8|}\Afootnote{\textit{ (1) } ratio \textit{ (2) } aequalitatis \textit{L}}} \pend 