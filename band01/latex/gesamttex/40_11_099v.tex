[99 v\textsuperscript{o}]
\edtext{confirmavit;}{\lemma{}\Afootnote{confirmavit; \textit{ erg.} \textit{ L}}} qua ostensum est in loco altiore ubi scilicet minus  aeris \edtext{pondus}{\lemma{}\Afootnote{pondus \textit{ erg.} \textit{ L}}} incumbit, altitudinem Mercurii\protect\index{Sachverzeichnis}{mercurius} supra \edtext{vasis}{\lemma{supra}\Afootnote{ \textit{ (1) }\ suum \textit{ (2) }\  vasis \textit{ L}}} in quo stagnat horizontem assurgentis, esse  minorem, \edtext{et ut inter ascendendum continue decrescet, ita  crescet inter descendendum.}{\lemma{et}\Afootnote{ \textit{ (1) }\ inter ascensum continue decrescere,  crescereque inter descensum \textit{ (2) }\ ut [...] descendendum. \textit{ L}}}\edtext{}{\lemma{descendendum.}\Bfootnote{\textsc{F. P\'{e}rier, }\cite{00287}\textit{Brief an Pascal vom 22. September 1648}, Paris 1663, S.~176\textendash188 (\textit{PO} II, S.~151\textendash158).}}\pend \clearpage
\pstart  Idem Pascalius\protect\index{Namensregister}{\textso{Pascal} (Pascalius), Blaise 1623\textendash 1662}  ex pila \edtext{quae}{\lemma{}\Afootnote{quae \textit{ erg.} \textit{ L}}} semiinflata in \edtext{valle}{\lemma{in}\Afootnote{ \textit{ (1) }\ fundo \textit{ (2) }\ valle \textit{ L}}}, inflat ipsa sese  proportione ascensus in montem,\edtext{}{\lemma{montem,}\Bfootnote{\textsc{B. Pascal, }\cite{00081}\textit{Traitez de l'\'{e}quilibre des liqueurs}, Paris 1663, S.~50 (\textit{PO} III, S.~198).}} conjecit  aerem nostrum \edtext{inferiorem  incumbentis}{\lemma{nostrum}\Afootnote{ \textit{ (1) }\ ab eo \textit{ (2) }\ ab incumbente ita co \textit{ (3) }\ inferiorem  incumbentis \textit{ L}}} superioris pondere compressum, si  altius evehatur, ac proinde prematur minus explicare  sese.\footnote{\textit{Nebenrechnungen zur wieder gestrichenen Ergänzung:}\\$\protect\begin{array}{r}31\\\protect\underline{\protect\hspace{5.5pt}12}\\62\\\protect\underline{31\protect\hspace{5.5pt}}\\372\protect\end{array}$\protect\rule[0cm]{1cm}{0cm}$\protect\begin{array}{lrr}\hspace{5.5pt}2&&\\\cancel{1}\cancel{0}&&\\\cancel{3}\cancel{7}2&f&13\\\cancel{2}\cancel{7}7&&\\\hspace{5.5pt}2&&\protect\end{array}$\\\textit{Leibniz hat die Überwärtsdivision mit 14 ausgeführt, dann aber gemerkt, dass die Lösung 13 sein muss. Die Rechnung wurde daraufhin abgebrochen und das Ergebnis korrigiert.}} \edtext{\edlabel{99vse1}}{\lemma{sese.}\xxref{99vse1}{99vse2}\Afootnote{\textbar\ Accedebat calculus  \textit{ (1) }\   \textbar\ ea collatione \textit{ erg.}\ \textbar\  altitudinis \textit{ (2) }\ ea collata altitudine antliarum\protect\index{Sachverzeichnis}{antlia} et Baroscopii\protect\index{Sachverzeichnis}{baroscopium},  \textit{ (a) }\ plane \textit{ (b) }\ sic satis consentiens.  \textit{ (aa) }\ Sipho\protect\index{Sachverzeichnis}{sipho|textit} \textit{ (bb) }\ Aqua enim per antliam\protect\index{Sachverzeichnis}{antlia} ad 31 circiter pedes   \textbar\ seu 372   \textbar\ pollicibus \textit{ erg. Hrsg.}\ \textbar\  \textit{ erg.}\ \textbar\  attollitur, Mercurius\protect\index{Sachverzeichnis}{mercurius} in Baroscopio\protect\index{Sachverzeichnis}{baroscopium} ad 27 pollices, qui in 372 pollicibus pene quaterdecies continentur, nam et Mercurius\protect\index{Sachverzeichnis}{mercurius} aqua quaterdecies gravior est. \textit{ erg. u.}\  \textit{ gestr.}\ \textbar\ His \textit{ L}}}%
\pend
%\vspace{-1.5ex}
\pstart  \edlabel{99vse2}His ita stabilitis credidit vir doctissimus  posse se tuto concludere \edtext{et ut ait demonstrare}{\lemma{}\Afootnote{et ut ait demonstrare \textit{ erg.} \textit{ L}}} caeteros effectus fugae vacui\protect\index{Sachverzeichnis}{fuga vacui}  ascriptos a solo aeris pondere\protect\index{Sachverzeichnis}{pondus!aeris} pendere.\pend\clearpage
  \pstart  Eodem tempore in Germania\protect\index{Ortsregister}{Deutschland (Germania, Duitsland)} Otto Gerickius\protect\index{Namensregister}{\textso{Guericke} (Gerickius, Gerick.), Otto v. 1602\textendash 1686}  vir in philosophia Experimentali\protect\index{Sachverzeichnis}{philosophia experimentalis} versatissimus \edtext{ex iisdem principiis}{\lemma{versatissimus}\Afootnote{ \textit{ (1) }\ suis ipse \textit{ (2) }\ ex iisdem principiis \textit{ L}}} experimenta produxit, plane \edtext{admiranda. Deprehendit}{\lemma{admiranda.}\Afootnote{\textbar\ Construxit enim Barometri\protect\index{Sachverzeichnis}{barometrum} genus plane novum  quod solo aere pondus in eo natans   \textbar\ (virunculi speciem habet gradus digito designantis) \textit{ erg.}\ \textbar\  nunc elevante  nunc deprimente peragitur; \textit{ gestr.}~\textbar\ Deprehendit \textit{ L}}} \edtext{enim}{\lemma{}\Afootnote{enim \textit{ erg.} \textit{ L}}} evidenter  (quod Canutus\protect\index{Namensregister}{\textso{Chanut} (Canut, Canutus), Hector Pierre 1604\textendash 1662} et Cartesius\protect\index{Namensregister}{\textso{Descartes} (Cartesius, des Cartes, Cartes.), Ren\'{e} 1596\textendash 1650} in Suecia\protect\index{Ortsregister}{Schweden (Suecia)} suspicati fuerant,\edtext{}{\lemma{fuerant,}\Bfootnote{\textsc{F. P\'{e}rier}, \cite{00127}\textit{Recit des observations}, in: B. Pascal, \cite{00081} a.a.O., S.~200 (\textit{PO} II, S. 442).}} 
   etsi Pascalio\protect\index{Namensregister}{\textso{Pascal} (Pascalius), Blaise 1623\textendash 1662} non satis placuerit) aeris gravitatem\protect\index{Sachverzeichnis}{gravitas!aeris}  non tantum \edtext{humiditate}{\lemma{tantum}\Afootnote{ \textit{ (1) }\ loci \textit{ (2) }\ humiditate \textit{ L}}} ejus \edtext{et}{\lemma{ejus}\Afootnote{ \textit{ (1) }\ aut \textit{ (2) }\ et \textit{ L}}} condensatione,  sed et a ventis \edtext{variari.}{\lemma{variari.}\Bfootnote{\textsc{O. v. Guericke}, \cite{00055}\textit{Experimenta nova}, Amsterdam 1672, S.~100.}} \edtext{Aerem enim ventorum motu velut}{\lemma{variari.}\Afootnote{ \textit{ (1) }\ Ventos enim aerem motum  ita \textit{ (2) }\ Aerem enim ventorum motu velut \textit{ L}}} sustentari,  ut minus ponderet, quemadmodum in aqua agitata corpora,  in quiescente subsidentia, natant. Idem primus \edtext{excogitavit Machinam}{\lemma{primus}\Afootnote{ \textit{ (1) }\ invenit Ma \textit{ (2) }\ excogitavit Machinam \textit{ L}}} illam \edtext{\edlabel{admirastart} admirabilem}{\lemma{admirabilem}\Bfootnote{\textsc{O. v. Guericke, }\cite{00055}a.a.O.  S.~94. }}%
\edtext{}{\lemma{admirabilem}\xxref{admirastart}{admiraend} \Afootnote{ \textit{ (1) }\ ab Illustribus viris Boylio\protect\index{Namensregister}{\textso{Boyle} (Boylius, Boyl., Boyl), Robert 1627\textendash 1691|textit} primum, deinde et Hugenio\protect\index{Namensregister}{\textso{Huygens} (Hugenius, Vgenius, Hugens, Huguens), Christiaan 1629\textendash 1695|textit}  promotam \textit{ (2) }\ qua aer   \textbar\ sensibilis \textit{ erg.}\ \textbar\  Recipiente quodam  \textbar\ seu \textit{ gestr.}\ \textbar\ vase, [...] solet) \textit{(a)}\ penitus exhauritur) \textit{(b)}\ singulari momento penitus exhauritur.  \textit{(aa)}\ Eadem arte \textit{(bb)}\ Idem [...] ut \textit{(aaa)}\ duo \textit{(bbb)}\ 24 [...] sponte.  Eadem ratione tum novum [...] pondera  \textbar\ subito \textit{ gestr.}\ \textbar\  elevandi exhibuit.  \textit{(aaaa)}\ Haec a \textit{(bbbb)}\  Ut ex  \textit{(aaaaa)}\ libro no \textit{(bbbbb)}\ opere [...] apparebit. \textit{(aaaaa-a)}\ Horum experimentorum pars \textit{(bbbbb-b)}\ Haec [...] promota \textit{ L}}}%
 qua aer sensibilis Recipiente quodam vase,  (quod ideo Recipiens Magdeburgicum appellari solet) singulari momento penitus exhauritur.
 \pend \pstart %
\# Idem duo Hemisphaeria cuprea ita  composuit, ut 24 et amplius equis divelli  non possint, etsi nullo vinculo sensibili contineantur,  admisso aere dilabantur \edtext{sponte.}{\lemma{sponte.}\Bfootnote{\textsc{O. v. Guericke, }\cite{00055}a.a.O., S.~105.}}%
\pend \pstart%
Eadem ratione tum  novum sclopeti ventanei genus, quod scilicet aere non  exeunte sed irrumpente animatur;\edtext{}{\lemma{animatur;}\Bfootnote{\textsc{O. v. Guericke, }\cite{00055}a.a.O., S.~112.}} tum rationem  ingentia pondera elevandi exhibuit.  Ut ex opere de spatio Vacuo, ab autore  novissime publicato, amplius apparebit. Haec Machina Recipientis Magdeburgici jam olim rogatu autoris a P. Gaspare Schotto S. J. publicata,\edtext{}{\lemma{publicata,}\Bfootnote{\textsc{C. Schott, }\cite{00096}\textit{Technica curiosa}, Würzburg 1664, S. 8\textendash11. }} ab illustribus  viris, Boylio primum in Anglia, deinde et Hugenio  in Batavis Galliaque promota\edlabel{admiraend}
%{\lemma{admirabilem}\xxref{admirastart}{admiraend} \Afootnote{ \textit{ (1) }\ ab Illustribus viris Boylio\protect\index{Namensregister}{\textso{Boyle,} Robert (1627\textendash1691)|textit} primum, deinde et Hugenio\protect\index{Namensregister}{\textso{Huygens} (Hugenius, Vgenius, Hugens, Huguens), Christiaan 1629\textendash 1695|textit}  promotam \textit{ (2) }\ qua aer   \textbar\ sensibilis \textit{ erg.}\ \textbar\  Recipiente quodam  \textbar\ seu \textit{ gestr.}\ \textbar\ vase, [...] solet) \textit{(a)}\ penitus exhauritur) \textit{(b)}\ singulari momento penitus exhauritur.  \textit{(aa)}\ Eadem arte \textit{(bb)}\ Idem [...] ut \textit{(aaa)}\ duo \textit{(bbb)}\ 24 [...] sponte.  Eadem ratione tum novum [...] pondera  \textbar\ subito \textit{ gestr.}\ \textbar\  elevandi exhibuit.  \textit{(aaaa)}\ Haec a \textit{(bbbb)}\  Ut ex  \textit{(aaaaa)}\ libro no \textit{(bbbbb)}\ opere [...] apparebit. \textit{(aaaaa-a)}\ Horum experimentorum pars \textit{(bbbbb-b)}\ Haec [...] promota \textit{ L}}}
et egregiis \edtext{experimentis adhibita est.}{\lemma{experimentis}\Afootnote{ \textit{ (1) }\ illustris reddita est a \textit{ (2) }\ adhibita est. \textit{ L}}} \edtext{Boylius\protect\index{Namensregister}{\textso{Boyle} (Boylius, Boyl., Boyl), Robert 1627\textendash 1691}}{\lemma{ursit.}\Bfootnote{\textsc{R. Boyle, }\cite{00015}\textit{New experiments}, Oxford 1660, S.~15 (\textit{BW} I, S.~165).}} non Gravitatem\protect\index{Sachverzeichnis}{gravitas} tantum aeris sed et vim Elasticam\protect\index{Sachverzeichnis}{vis!elastica} \edtext{ursit.}{\lemma{Elasticam}\Afootnote{ \textit{ (1) }\ , quam  jam initio hujus seculi \textit{ (2) }\ ursit. \textit{ L}}} Quanquam enim vim aeris Elasticam\protect\index{Sachverzeichnis}{vis!elastica} sclopeta\protect\index{Sachverzeichnis}{sclopetum} ventanea  (dont l'invention est de\"{u}e a Marin Bourgeois de Lisieux\protect\index{Namensregister}{\textso{Bourgeois,} Marin ca. 1550\textendash 1634} qui la presenta a Henry IV.\protect\index{Namensregister}{\textso{Frankreich: Heinrich IV.}, K\"{o}nig von Fankreich 1589\textendash 1610} en  l'an 1605, si clarissimo Petito\protect\index{Namensregister}{\textso{Petit} (Petitus), Pierre 1598\textendash 1677} credimus, et  memini sane me librum de iis vidisse, ante annum  hujus seculi 10\textsuperscript{mum} impressum) egregie illustrarint, aerem 