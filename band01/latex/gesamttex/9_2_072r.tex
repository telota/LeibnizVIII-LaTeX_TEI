   
        
        \begin{ledgroupsized}[r]{120mm}
        \footnotesize 
        \pstart        
        \noindent\textbf{\"{U}berlieferung:}  
        \pend
        \end{ledgroupsized}
      
       
              \begin{ledgroupsized}[r]{114mm}
              \footnotesize 
              \pstart \parindent -6mm
              \makebox[6mm][l]{\textit{L}}Konzept: LH IV 8, 22 Bl. 72 \textendash 73. 1 Bog. 4\textsuperscript{o}. Insgesamt ca. 2 S. Textfolge 72 r\textsuperscript{o}, 73 v\textsuperscript{o}, 72 v\textsuperscript{o}, 73 r\textsuperscript{o}. Nach dem Beschreiben hat wahrscheinlich zu Transportzwecken eine weitere Faltung des Bogens stattgefunden, wodurch die untere H\"{a}lfte des Bl. 72 r\textsuperscript{o} zur Außenseite wurde, wie F\"{a}rbung und Abrieb zeigen. Die beiden \"{a}ußeren Seiten (72 r\textsuperscript{o}, 73 v\textsuperscript{o}) sind zweispaltig eng beschrieben. Der Text ist in Rubriken unterteilt, die mit den am Anfang aufgelisteten Wissenschafts-Bezeichnungen \"{u}berschrieben sind. Die Eintr\"{a}ge zu unterschiedlichen Rubriken wurden nachtr\"{a}glich durch Tintenstriche voneinander abgesetzt. Am oberen und rechten Rand auf Bl. 72 r\textsuperscript{o} sind mehrere Nachtr\"{a}ge zu den Rubriken auf Bl. 73 v\textsuperscript{o}. Auf den beiden inneren Seiten (72 v\textsuperscript{o}, 73 r\textsuperscript{o}) wird die Einteilung der Vorderseiten durch Spiegelung an der Papierebene exakt reproduziert, einschließlich der damit einhergehenden unregelm\"{a}ßigen Aufteilung der Seiten und der rechtsb\"{u}ndigen Auff\"{u}hrung der Bezeichnungen der Rubriken. Die R\"{u}ckseiten sind jedoch bis auf je eine Eintragung leer. Da die Einteilung nach Rubriken das Ordnungsprinzip dieser Aufzeichnungen ist, und auf Grund der sp\"{a}teren Benutzung der R\"{u}ckseiten, ergibt sich die Textfolge Bl. 72 r\textsuperscript{o}, 73 v\textsuperscript{o} mit Einf\"{u}gungen, 72 v\textsuperscript{o}, 73 r\textsuperscript{o}. Der vorliegende Text hat nicht den geplanten Umfang erreicht. Das ergibt sich aus den vorbereiteten, aber nicht ausgef\"{u}llten Rubriken auf den R\"{u}ckseiten (72 v\textsuperscript{o}, 73 r\textsuperscript{o}), und aus einem Verweis in den Chymica auf einen Eintrag in den Medica, der dort nicht ausgef\"{u}hrt worden ist.\pend
              \end{ledgroupsized}
       
              \begin{ledgroupsized}[r]{114mm}
              \footnotesize 
              \pstart \parindent -6mm
              \makebox[6mm][l]{\textit{E}}\textsc{K. I. Gerhardt}, \cite{00241}\textit{Leibniz in London}, in: \textit{Sitzungsberichte der Preußischen Akademie der Wissenschaften} X (1891) S. 157\textendash 176, darin S. 165f. (Teildruck). Englische \"{U}bersetzung in: \textsc{J. M. Child}, \cite{00242}\textit{The Early Mathematical Manuscripts of Leibniz}, Chicago 1920, S. 184\textendash 186. Nachdruck Mineola 2005.\\Cc 2, Nr. 344\pend
              \end{ledgroupsized}
        %\normalsize
        \vspace*{5mm}
        \begin{ledgroup}
        \footnotesize 
        \pstart
      \noindent\footnotesize{\textbf{Datierungsgr\"{u}nde}: Der Zeitpunkt der Anfertigung ist am Beginn des Textes zwei Mal als initium anni 1673 festgehalten. Weiterhin besteht eine auffallende Korrespondenz der Themen in einem Brief an Oldenburg vom 8. M\"{a}rz 1673 (\textit{LSB} III, 1 N. 9) zu den hier festgehaltenen Observata. Vermutlich sind Teile des vorliegenden Textes in direkter Umgebung des Briefes entstanden. Diese Datierung wird unterst\"{u}tzt durch das Wasserzeichen im Bl. 72, das eine eindeutige Konzentration auf M\"{a}rz bis Mai 1673 zeigt (vgl. \textit{LSB} VII, 3 N. 19). Jedoch ist der Text, wie an den stark variierenden Schriftbildern und einigen Wiederholungen erkennbar wird, bei unterschiedlichen Gelegenheiten geschrieben worden. Daher erfolgt die Datierung nicht genauer als M\"{a}rz 1673.}
        \pend
        \end{ledgroup}
      
        \vspace*{8mm}
        \pstart 
        \normalsize
        \centering[72 r\textsuperscript{o}] Observata Philosophica\\in itinere Anglicano\\sub initium anni 1673.\pend \vspace{1.0ex} \pstart Cum initio anni 1673. Ablegato Moguntino illustrissimo Baroni Sch\"{o}nbornio\protect\index{Namensregister}{\textso{Mainz: Johann Philipp von Sch\"{o}nborn} (Elector Moguntini), Bischof von W\"{u}rzburg u. Worms, Kurf\"{u}rst 1647\textendash 1673} Electoris Moguntini ex fratre nepoti, Parisiis\protect\index{Ortsregister}{Paris (Parisii)} Londinum\protect\index{Ortsregister}{London (Londinum)} comes ivissem, etsi vix permissa in Anglia\protect\index{Ortsregister}{England (Anglia)} mensis mora, inter varias interturbationes, operam dedi tamen philosophiae quoque incrementis cognoscendis, quando nunc ea potissimum fama gens illa floret. \pend \pstart Diarium condere taediosum et minutum, et ipsa inaequalitate, ingratum, neque enim eadem omnium dierum fortuna est, et nunc acervatur materia annotandi, nunc ingenti vacuo hiat. Quare fortasse satius fuerit, ire per capita rerum, observatione observationem velut vocante.\pend \pstart Annotandorum haec summa capita fieri possunt: Arithmetica, Geometrica, Musica\protect\index{Sachverzeichnis}{musica}, Optica, Astronomica, Mechanica, Pneumatica, Meteorologica\protect\index{Sachverzeichnis}{meteorologica}, Hydrostatica, Magnetica, Nautica, Botanica\protect\index{Sachverzeichnis}{botanica}, Anatomica\protect\index{Sachverzeichnis}{anatomica}, Chymica\protect\index{Sachverzeichnis}{chymica}, Medica\protect\index{Sachverzeichnis}{medica}, Miscellanea.\pend \pstart In \edtext{\textso{Arithmeticis}}{\lemma{\textso{Arithmeticis}}\Afootnote{\textit{doppelt unterstrichen}}} \edtext{Linea}{\lemma{\textso{Arithmeticis}}\Afootnote{ \textit{ (1) }\ notabilis est \textit{Trigonometria Britannica} fol. in qua Logarithmi\protect\index{Sachverzeichnis}{logarithmus|textit} computati sunt ad centesimam usque gradus partem, \textit{(a)}\ ad \textit{(b)}\ ac proinde pene ad minutum secundum \textit{ (2) }\ Linea \textit{ L}}} proportionum, \edtext{seu}{\lemma{}\Afootnote{seu \textit{ erg.} \textit{ L}}} Gunters\protect\index{Namensregister}{\textso{Gunter,} Edmund 1581\textendash 1626} Lini\protect\index{Sachverzeichnis}{Gunters Linea}, aliis \edtext{doublescale}{\lemma{doublescale}\Bfootnote{Vermutl. Verweis auf Gunters\protect\index{Namensregister}{\textso{Gunter,} Edmund 1581\textendash 1626|textit} Rechenst\"{a}be\protect\index{Sachverzeichnis}{Rechenstab}, vgl. \cite{00275}\textit{LSB} III, 1, S.~678f.}}. \edtext{\textit{Logarithmotechnia}}{\lemma{}\Afootnote{\textit{Logarithmotechnia} \textbar\ Mercatoris\protect\index{Namensregister}{\textso{Mercator,} Nicolaus 1620\textendash 1687} Pellii\protect\index{Namensregister}{\textso{Pell} (Pellius), John 1611\textendash 1685} terminationes numerorum quadratorum\protect\index{Sachverzeichnis}{numeri quadrati}. \textit{ gestr.}\ \textbar\ seu \textit{ L}}}
        \edtext{}{\lemma{\textit{Logarithmotechnia}}\Bfootnote{\cite{00141}\textsc{N. Mercator}, \textit{Logarithmotechnia}, London 1668. Leibniz' Kenntnis dieses Buches vermutl. durch \textsc{I. Wallis}, \cite{00236}\textit{Logarithmotechnia Nicolai Mercatoris}, \textit{PT} 3 (1668), S.~753\textendash 764. Zu dem gestrichenen Teil der Notiz \textsc{J. Pell}, \cite{00229}\textit{Tabulae}, London 1672.}} seu compendium calculandi Logarithmos\protect\index{Sachverzeichnis}{logarithmus}; dignoscere numeros quadratos\protect\index{Sachverzeichnis}{numeri quadrati} a non quadratis ex terminationibus. \pend \pstart Machina\protect\index{Sachverzeichnis}{machina!Morlandi} Morlandi\protect\index{Namensregister}{\textso{Morland,} Samuel 1625\textendash 1695}\edtext{}{\lemma{Morlandi.}\Bfootnote{Zu Leibniz' Kenntnis der Rechenmaschine Morlands\protect\index{Namensregister}{\textso{Morland,} Samuel 1625\textendash 1695|textit} vgl. \cite{00247}\textit{LSB} III, 1, S.~21, und Marginalie in \cite{00205}\textsc{S. Morland, }\textit{Arithmetick Instruments}, London 1673, vgl. dazu \cite{00217}\textit{An Accompt of some Books}, \textit{PT} 8 (1673), S.~6048f.; auch \cite{00276}\textit{LSB} III, 6, S.~330.}}.\pend \pstart \textso{Algebra}. Corpus Algebrae Anglicum \edtext{opus}{\lemma{opus}\Afootnote{ \textit{ (1) }\ 20 \textit{ (2) }\ 27 \textit{ L}}} 27 annorum. Algebra Pellii\protect\index{Namensregister}{\textso{Pell} (Pellius), John 1611\textendash 1685}\edtext{}{\lemma{Algebra}\Bfootnote{\textsc{J. H. Rahn, }\cite{00139}\textit{Algebra}, London 1668.}}. In priore, parum regularum, multum exemplorum selectorum. Renaldinus\protect\index{Namensregister}{\textso{Renaldini} (Renaldinus), Carlo 1615\textendash 1698} non aestimatur in Anglia\protect\index{Ortsregister}{England (Anglia)}.\pend \pstart \textso{Geometrica}\protect\index{Sachverzeichnis}{geometrica}. Tangentes\protect\index{Sachverzeichnis}{tangens} omnium figurarum\edtext{}{\lemma{figurarum}\Bfootnote{Vermutl. Bemerkung zu einem Brief von R. Sluse\protect\index{Namensregister}{\textso{Sluse,} Ren\'{e} Fran\c{c}ois Walter de 1622	\textendash 1685|textit} vom 17. Januar 1673 an die Royal Society; vgl. \cite{00248}\textit{LSB} III, 1, S.~32 und \cite{00154}\textit{BH} III, S. 74.}}. Figurarum geometricarum explicatio per motum puncti in moto lati.\pend \pstart Quadrantes\protect\index{Sachverzeichnis}{quadrans}\edtext{}{\lemma{Quadrantes}\Bfootnote{F\"{u}r die Quadranten und die beiden folgenden Instrumente vgl. \textsc{Th. Sprat, }\cite{00098}\textit{History}, London 1667, S.~246.}} 18 pollicum meliores omnibus hactenus usitatis, terra, pro angulis sumendis. Instrumentum sumendi angulum\protect\index{Sachverzeichnis}{instrumentum!sumendi angulum} per reflex. ita ut oculus simul videat duo obiecta, both as touching in the same point, quanquam vel semicirculo distent, magni usus in observ. marit.\protect\index{Sachverzeichnis}{observatio maritima} $\langle$Stay$\rangle$\edtext{}{\lemma{$\langle$Stay$\rangle$}\Bfootnote{In Vorlage Staff (\textsc{Th. Sprat}, \cite{00098}a.a.O., S.~246).}} pro \astrosun\ altit., wiewohl schatten auff 3 fuß distance so ist doch keine penumbra, und der schatten kan distinguirt werden ad quartam partem minuti. Niveau sive linea horizontalis sine errore 2\textsuperscript{dorum} aliquot.\edtext{}{\lemma{aliquot.}\Bfootnote{Vgl. \textsc{Th. Sprat, }\cite{00098}a.a.O., S.~248.}} (S.H)\edtext{}{\lemma{(S.H)}\Bfootnote{Abk\"{u}rzung f\"{u}r \textsc{Th. Sprat, }\cite{00098}\textit{History of the Royal Society}, London 1667.}}\pend \pstart \textso{Musica}\protect\index{Sachverzeichnis}{musica}. Character ejus universalis. Systema de \edtext{Birthinena}{\lemma{Birthinena}\Bfootnote{Vgl. dazu die Voranzeige in \textit{Another advertisement}, \cite{00278}\textit{PT} 7 (1672), S.~5153f. eines Buches \textsc{J. Birchensha}\protect\index{Namensregister}{\textso{Birchensha,} John 1605?\textendash 1681|textit}, \textit{Syntagma Musicae}. Trotz Besuch bei der Royal Society (vgl. \cite{00154}\textit{BH} I, S.~457f.) und Voranzeige ist das Buch nicht erschienen.}}. Vossius\protect\index{Namensregister}{\textso{Vossius} (Voss.), Isaac 1618\textendash 1689} Musica edet\edtext{}{\lemma{edet.}\Bfootnote{\textsc{I. Voss, }\cite{00161}\textit{De poematum cantu}, Oxford 1673; vgl. \cite{00249}\textit{LSB} III, 1, S.~87 und \cite{00218}\textit{An Accompt of Two Books}, \textit{PT} 8 (1673), S.~6024\textendash 6030.}}.\pend \pstart \textso{Optica}\protect\index{Sachverzeichnis}{optica}. Locuti sunt mihi de phaenomeno quodam quod Barrovius\protect\index{Namensregister}{\textso{Barrow} (Barrovius), Isaac 1630\textendash 1677} fatetur se solvere non posse\edtext{}{\lemma{posse}\Bfootnote{Vgl. \cite{00250}\textit{LSB} III, 1, S. 87f. und das St\"{u}ck N. 26 in diesem Band.}}. Neutonii\protect\index{Namensregister}{\textso{Newton} (Neuton, Neutonus), Isaac 1642\textendash 1727} difficultas soluta hactenus non est\edtext{}{\lemma{est}\Bfootnote{Vermutl. Kommentar zu der anhaltenden Auseinandersetzung \"{u}ber Newtons\protect\index{Namensregister}{\textso{Newton} (Neuton, Neutonus), Isaac 1642\textendash 1727|textit} Farbenlehre, vgl. \cite{00251}\textit{LSB} III, 1, S.~44.}}. P. Pardies\protect\index{Namensregister}{\textso{Pardies,} Ignace Gaston SJ 1636\textendash 1673} manus dante\edtext{}{\lemma{dante}\Bfootnote{Zu Pardies\protect\index{Namensregister}{\textso{Pardies,} Ignace Gaston SJ 1636\textendash 1673|textit} vgl. \cite{00251}\textit{LSB} III, 1, S.~43.}}. Hookius\protect\index{Namensregister}{\textso{Hooke} (Hookius, Hook), Robert 1635\textendash 1703} \edtext{Instrumento}{\lemma{Hookius}\Afootnote{ \textit{ (1) }\ spem \textit{ (2) }\ Instrumento \textit{ L}}} Catadioptrico\protect\index{Sachverzeichnis}{instrumentum!catadioptricum}\edtext{}{\lemma{Catadioptrico}\Bfootnote{Vgl. \cite{00251}\textit{LSB} III, 1, S.~43.}} 9 pedum praestabit, quod alii 50. motus eos incommodat. Secretum aperturae maximae, quae tanta inprimis Microscopiis\protect\index{Sachverzeichnis}{microscopium} dari possit, quanta objecti distantia est. Materia speculorum\protect\index{Sachverzeichnis}{speculum}, quae aeris injuriis resistat, cujus politura \edtext{pura}{\lemma{politura}\Afootnote{ \textit{ (1) }\ pulchra \textit{ (2) }\ pura \textit{ L}}} ut vitri oleum\protect\index{Sachverzeichnis}{vitri oleum}, quo inungenda specula\protect\index{Sachverzeichnis}{speculum}, ut rubigini resistant\edtext{}{\lemma{resistant}\Bfootnote{Vgl. zu Hookes\protect\index{Namensregister}{\textso{Hooke} (Hookius, Hook), Robert 1635\textendash 1703|textit} Catadioptricum \cite{00251}\textit{LSB} III, 1, S.~43 und \cite{00154}\textit{BH} III, S. 72 und 74.}}. Cock\protect\index{Namensregister}{\textso{Cock,} Christopher} \edtext{microscopia}{\lemma{microscopia}\Bfootnote{Christopher Cock\protect\index{Namensregister}{\textso{Cock,} Christopher|textit} (Lebensdaten unbekannt) hatte mehreren Mitgliedern der Royal Society Mikroskope gebaut, darunter auch das Mikroskop, mit dem Hooke\protect\index{Namensregister}{\textso{Hooke} (Hookius, Hook), Robert 1635\textendash 1703|textit} die Beobachtungen zur Micrographia durchf\"{u}hrte.}}\protect\index{Sachverzeichnis}{microscopium}, \edtext{sabuli}{\lemma{microscopia,}\Afootnote{ \textit{ (1) }\ sabulum \textit{ (2) }\ sabuli \textit{ L}}} granum instar ovi columbini pediculus ut capra et capellus ut chorda. Schmethwick\protect\index{Namensregister}{\textso{Smethwick} (Schmethwick), Francis}\edtext{}{\lemma{Schmethwick}\Bfootnote{Die Lebens\-daten des F. Smethwick\protect\index{Namensregister}{\textso{Smethwick} (Schmethwick), Francis|textit} sind unbekannt. Er ist nachgewiesen durch einen Bericht \"{u}ber seine Vorf\"{u}hrung nicht-sph\"{a}risch geschliffener Gl\"{a}ser in einer Sitzung der Royal Society, \cite{00219}\textit{An Account of the Invention of Grinding Optick and Burning Glasses, of a figure not-Spherical, produced before the Royal Society}, \textit{PT} 3 (1668), S.~631\textendash 632.}} sectionis Conicae\protect\index{Sachverzeichnis}{sectio conica} vitrum \edtext{fodert vor ein perspectiv 10 \Pfund\ sterlings}{\lemma{}\Afootnote{fodert vor ein perspectiv 10 \Pfund\ sterlings \textit{ erg.} \textit{ L}}}; non sunt tanti, putat ipse esse hyperbolam\edtext{}{\lemma{hyperbolam}\Bfootnote{Zu misslungenen Versuchen, hyperbolische Linsen herzustellen, vgl. \textit{An account of two books. I. Renati Franc. Slusii Mesolabum}, \cite{00284}\textit{PT} 4 (1669), S.~903\textendash 912, bes. S.~904 und \textit{An answer written to the publisher}, \cite{00285}\textit{PT} 8 (1673), S.~6112.}}. Cock\protect\index{Namensregister}{\textso{Cock,} Christopher} nunc \edtext{telescopium\protect\index{Sachverzeichnis}{telescopium}}{\lemma{telescopium}\Bfootnote{Die Royal Society beauftragte Cock\protect\index{Namensregister}{\textso{Cock,} Christopher|textit}, ein Teleskop nach dem Konzept Newtons\protect\index{Namensregister}{\textso{Newton} (Neuton, Neutonus), Isaac 1642\textendash 1727|textit} zu bauen, vgl. \cite{00154}\textit{BH} III, S.~19, 43, 49 und 57.}} 50 pedum Drebelii\protect\index{Namensregister}{\textso{Drebbel} (Drebelius, Drebel), Cornelius 1572\textendash 1633} \edlabel{telescopstart}Telescop.\pend \pstart \edtext{vergroßern\edlabel{telescopend}}{{\xxref{telescopstart}{telescopend}}\lemma{Telescop.}\Afootnote{ \textit{ (1) }\ Unter \textit{ (2) }\ vergroßernbrillen \textit{ L}}}brillen \pend \pstart Flexible Gl\"{a}ser mit $\langle$−$\rangle$ Selenitis. U)\edtext{}{\lemma{U)}\Bfootnote{Vermutl. K\"{u}rzel f\"{u}r \textsc{R. Boyle,} \cite{00155}\textit{Usefulness}, London 1663.}} ohne Glaser foliiren und also Spigl\protect\index{Sachverzeichnis}{Spiegel} draus machen, desideratum.\pend