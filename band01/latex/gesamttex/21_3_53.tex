\pstart \selectlanguage{italian}[p.~53] [...] esperienza di questo h\`{o} fatta io, che con vno Specchio\protect\index{Sachverzeichnis}{specchio} sferico di piombo\protect\index{Sachverzeichnis}{piombo} ancor mal polito h\`{o} acceso il fuoco nella materia arida al fuoco di carboni\footnote{\textit{Leibniz unterstreich}t: al [...] carboni. \textit{Daneben rechts am Rand}: add. p. seq.}; [...].\pend \pstart Se adunque prenderemo di questa superficie quella parte, che \`{e} intorno alla cima, questa abbrucier\`{a} tra'l corpo focoso, e lo Specchio\protect\index{Sachverzeichnis}{specchio}; ma se vogliamo, che l'incendio sia di dietro dello Specchio\protect\index{Sachverzeichnis}{specchio}\footnote{\textit{Leibniz unterstreicht:} l'incendio [...] Specchio\protect\index{Sachverzeichnis}{specchio}.{\vrule height 0pt depth 10mm width 0pt}}, bisogner\`{a} pigliare vna parte di quella, discosta dalla cima, tanto che lasci fuori di se il foco di tal superficie verso la cima, come per essempio; [...].\selectlanguage{latin} \pend \newpage