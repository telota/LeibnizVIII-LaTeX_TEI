\pend \pstart [p.~190] Hanc hypothesim indicaui \textit{in Dialogis Physicis}\edtext{}{\lemma{Physicis}\Bfootnote{\textsc{H. Fabri, }\cite{00187}\textit{Dialogi physici}, Lyon 1665.}} in lucem editis; sed  nouis obseruationibus a me factis, comperi Veneris\protect\index{Sachverzeichnis}{Venus} apogaeae distantiam a terra\protect\index{Sachverzeichnis}{terra} esse plusquam duplam Veneris\protect\index{Sachverzeichnis}{Venus} perigaeae, vnde plusquam duplo citius orbem suum ab ortu  ad occasum decurreret; quod certe obseruationibus repugnat; quare aliam hypothesim statuam, \textit{in tomo Physicae} qui proxime succedet,\footnote{\textit{Leibniz unterstreicht}: nouis obseruationibus [...] proxime succedet} in quo de corpore coelesti ex  professo agam,\edtext{}{\lemma{agam,}\Bfootnote{\textsc{H. Fabri, }\cite{00044}\textit{Physica}, Bd. 4, 3. Buch, Lyon 1671.}} et in sequenti appendice; [...].