      
               
                \begin{ledgroupsized}[r]{120mm}
                \footnotesize 
                \pstart                
                \noindent\textbf{\"{U}berlieferung:}   
                \pend
                \end{ledgroupsized}
            
              
                            \begin{ledgroupsized}[r]{114mm}
                            \footnotesize 
                            \pstart \parindent -6mm
                            \makebox[6mm][l]{\textit{L}}Konzept: LH XXXVII 2 Bl. 122\textendash 123. 1 Bog. 2\textsuperscript{o}. 1 1/2 S. zweispaltig. Bl. 122 r\textsuperscript{o}, etwa 1/3 der linken sowie die rechte Spalte dieser Text. Der verbleibende Teil N. 33. R\"{u}ckseite leer. Vorliegender Text wird auf Bl. 123 v\textsuperscript{o} linke Spalte fortgesetzt und endet mit dem unteren F\"{u}nftel der rechten Spalte. Der verbleibende Teil dieser Spalte wird unter dem Titel \textit{De animalibus} in \textit{LSB} VIII, 2 gedruckt. Geringer Textverlust durch Papierschaden am linken unteren Rand von Bl. 123~v\textsuperscript{o}.\\Kein Eintrag in KK 1 oder Cc 2.\pend
                            \end{ledgroupsized}
                %\normalsize
                \vspace*{5mm}
                \begin{ledgroup}
                \footnotesize 
                \pstart
            \noindent\footnotesize{\textbf{Datierungsgr\"{u}nde}: F\"{u}r die Datierung sind inhaltliche Gr\"{u}nde ausschlaggebend. LH XXXVII 2 Bl. 123 v\textsuperscript{o} enth\"{a}lt die Mitteilung, dass Herr Tenzel an Tschirnhaus eine Relation mit der Bitte um Weiterleitung an den Geheimen Stadtrat geschickt haben soll. Da es sich um einen sehr detaillierten Bericht handelt, ist anzunehmen, dass Leibniz diese Information von Tschirnhaus pers\"{o}nlich erhalten hat, den er Ende August 1675 in Paris kennenlernte.}
                \pend
                \end{ledgroup}
            
                \vspace*{8mm}
                \pstart 
                \normalsize
            [122 r\textsuperscript{o}] \selectlanguage{ngerman}Feuer\protect\index{Sachverzeichnis}{Feuer} so von oben nieder gehet, und mit solcher gewalt ziehet, daß auch der ofen\protect\index{Sachverzeichnis}{Ofen} selbst schmelzet, und die ziegel herunter tr\"{u}pfen. \edtext{Es ist ein langer Zug}{\lemma{tr\"{u}pfen.}\Afootnote{ \textit{ (1) }\ Das feuer\protect\index{Sachverzeichnis}{Feuer|textit} gehet von oben nieder. \textit{ (2) }\ Es ist ein langer Zug \textit{ L}}}, braucht keines gebl\"{a}ses\protect\index{Sachverzeichnis}{Gebl\"{a}se}. Man hat in 12 stunden 50 tonnen Sole\protect\index{Sachverzeichnis}{Sole} gesotten aus 7 \`{a} 8 Ellen lang, 7 \`{a} 6 Ellen breiter pfanne, etwa mit 4 oder 5 klafftern Holz. Zum brauwesen\protect\index{Sachverzeichnis}{Brauwesen} treflich, wohl mit einem drittheil Holzes auszukomen. Die Sole\protect\index{Sachverzeichnis}{Sole} da man es probiret, ist zu Ketschau\protect\index{Ortsregister}{Koetzschau}, einem dorff ohnweit Leipzig\protect\index{Ortsregister}{Leipzig}. Es gibt auf solche weise treflich sch\"{o}hn und fein Salz\protect\index{Sachverzeichnis}{Salz}, beßer als das Hallische. Es ist auch die Sole\protect\index{Sachverzeichnis}{Sole} in copia da. \pend \pstart  \edtext{Die anbr\"{u}che der Edlen Steine oder des durchsichtigen jaspis so die franzosen Agathe oriental nennen sind}{\lemma{da.}\Afootnote{ \textit{ (1) }\ Die Steinbr\"{u}che sind \textit{ (2) }\ Die [...] sind \textit{ L}}} nicht weit von Dreßden\protect\index{Ortsregister}{Dresden}, Churf. Augustus\protect\index{Namensregister}{\textso{August I.}, S\"{a}chsischer Kurf\"{u}rst (1526\textendash 1586)} hat schohn alda gesch\"{u}rffet. Sind nicht uber 5 lachter \edtext{unterm}{\lemma{lachter}\Afootnote{ \textit{ (1) }\ unter der \textit{ (2) }\ unterm \textit{ L}}} tage ein schwebender gang, so sehr weit gehet und zimlich uniform verbleibet. Es sind darinn 3 farben gemeiniglich, so unterschiedlichen spielen. Blau ist arth vom Amethyst, roth ist coraallus, und weiß ist Chalcedonier. \pend \pstart  Das braun glas\protect\index{Sachverzeichnis}{Glas} so nach Paris\protect\index{Ortsregister}{Paris (Parisii)} kommen ist 7 schuch im diametro, Homberg\protect\index{Namensregister}{\textso{Homberg,} Wilhelm (1652\textendash 1715)} macht damit trefliche Experimenta, unter anderen sol er gold\protect\index{Sachverzeichnis}{Gold} genommen, und \edtext{lange Zeit}{\lemma{und}\Afootnote{ \textit{ (1) }\ damit \textit{ (2) }\ lange Zeit \textit{ L}}} in foco\protect\index{Sachverzeichnis}{focus} gehalten, so sol erstlich etwas glaß geben, dann immer mehr und mehr, biß endtlich alles zu einem rothen glas\protect\index{Sachverzeichnis}{Glas} worden, darauß es nicht mehr zu reduciren.  H. von Tch.  \protect\index{Namensregister}{\textso{Tschirnhaus} (H.v.Tch., H.v.Tsch.), Ehrenfried Walther v. 1651\textendash 1708} hat es schon \edtext{in}{\lemma{schon}\Afootnote{ \textit{ (1) }\ ein \textit{ (2) }\ in \textit{ L}}} etwas probirt gehabt, wenn er das gold\protect\index{Sachverzeichnis}{Gold} auf porcellan\protect\index{Sachverzeichnis}{Porzellan} getragen. \pend \pstart   Die Salia\protect\index{Sachverzeichnis}{Salz} geben dem Sande den fluß, asche laßet man beßer davon. Wenn man nun das glas\protect\index{Sachverzeichnis}{Glas} lange im feuer\protect\index{Sachverzeichnis}{Feuer} laßet stehen, so gehen endtlich die Salia\protect\index{Sachverzeichnis}{Salz} weg, denn ist es bestandig. \pend \pstart  Die Blasen oder gischen, wie es die glasarbeiter nennen, gehen auch endtlich weg, wenn das glas\protect\index{Sachverzeichnis}{Glas} lange im fluß stehet, und sich sezen kan. \pend \pstart Glaß\protect\index{Sachverzeichnis}{Glas} wenn es einer zimlichen dicke ist viel fester als Eisen\protect\index{Sachverzeichnis}{Eisen}, und keinem rost oder ander angelegenheit unterworffen \edtext{(Meines}{\lemma{unterworffen}\Afootnote{ \textit{ (1) }\ so ist \textit{ (2) }\ (Meines \textit{ L}}} erachtens die f\"{a}rtig\edtext{keit von den Edelsteinen}{\lemma{f\"{a}rtig}\Afootnote{ \textit{ (1) }\ keit vom glaß \textit{ (2) }\ keit von den Edelsteinen \textit{ L}}} komt daher, daß sie gar lange in unterirrdischen feuer\protect\index{Sachverzeichnis}{Feuer} gestanden, also sich vollkommentlich gesezet, und \edtext{die theile sich}{\lemma{}\Afootnote{die theile sich \textit{ erg.} \textit{ L}}} wohl zusammen gef\"{u}get, daß die plana allerdings zusammenpaßen.) H. v. Tsch.\protect\index{Namensregister}{\textso{Tschirnhaus} (H.v.Tch., H.v.Tsch.), Ehrenfried Walther v. 1651\textendash 1708} meinet der diamant sey genus Talci\protect\index{Sachverzeichnis}{Talcum} purissimi, habe etwas duplicis refractionis\protect\index{Sachverzeichnis}{refractio duplex}.\pend \pstart Talck\protect\index{Sachverzeichnis}{Talk} und alle diamanten werden durch das Sonnenfeuer zu bloßen kieselsteinen, verlieren also die h\"{a}rte.