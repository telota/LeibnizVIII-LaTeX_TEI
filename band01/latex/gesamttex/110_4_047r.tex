   
        
        \begin{ledgroupsized}[r]{120mm}
        \footnotesize 
        \pstart        
        \noindent\textbf{\"{U}berlieferung:}  
        \pend
        \end{ledgroupsized}
      
       
              \begin{ledgroupsized}[r]{114mm}
              \footnotesize 
              \pstart \parindent -6mm
              \makebox[6mm][l]{\textit{L}}Konzept: LH XXXV 15, 6 Bl. 47\textendash48. 1 Bog. 2\textsuperscript{o}. 4 S. zweispaltig. Linke Spalte fortlaufender Text, rechte Spalte Erg\"{a}nzungen. Bl. 48 r\textsuperscript{o} rechte Spalte oben die drei graphischen Skizzen von \textit{[Fig. 1]}. Bl. 48 v\textsuperscript{o} rechte Spalte die Zeichnungen \textit{[Fig. 2]} und \textit{[Fig. 3]}, \textit{[Fig. 2]} gestrichen. Die gestrichene Zeichnung wird im Druck wiedergegeben, da sie sich erheblich von \textit{[Fig. 3]} unterscheidet.\\KK 1, Nr. 193 C \pend
              \end{ledgroupsized}
        \vspace*{8mm}
        \pstart 
        \normalsize
    \begin{center}[47 r\textsuperscript{o}] \edtext{Longit. 2.}{\lemma{}\Afootnote{Longit. 2. \textit{ erg.} \textit{ L}}}\end{center}\pend \vspace{1.0ex} \pstart Circa Instrumentum Longitudinum\protect\index{Sachverzeichnis}{longitudo} danda opera est, ne rota\protect\index{Sachverzeichnis}{rota} ab aere circumagenda, nimis oneretur; cogitavi igitur Linea simplice careri posse, si acicula firma sit in serrata, et malleolus impingens in serratam adigat aciculam in chartam subjectam\footnote{\textit{Am Rand:} Drebelii\protect\index{Namensregister}{\textso{Drebbel} (Drebelius, Drebel), Cornelius 1572\textendash 1633} instrumentum per radios solis sonorum erat per spiritum roris majalis. Referente P. Gasp. Schotto \protect\index{Namensregister}{\textso{Schott} (Schottus), Caspar SJ 1608\textendash 1666}\cite{00094}\textit{Mag.} part. IV. lib. 2. cap. 6. fin. pag. 156.}. Sed in eo difficultas est, \edtext{acicula}{\lemma{est,}\Afootnote{ \textit{ (1) }\ duo \textit{ (2) }\ acicula \textit{ L}}} non potest perpendiculariter subjectam chartam ferire, quia \edtext{linea serrata impingens, quanto ea magis ab hypomochlio recessit, tanto movebitur in descensu}{\lemma{quia}\Afootnote{ \textit{ (1) }\ malleolus lineam serratam impingens, quanto ea magis ab hypomochlio recessit, tanto eam \textit{ (2) }\ linea [...] descensu \textit{ L}}} circulo majore, quanto propior est tanto minore, similiter et acicula impactoria lineae serratae infixa. Ergo variabitur impactio, atque ita distantia punctorum. \edtext{Liceret}{\lemma{punctorum.}\Afootnote{ \textit{ (1) }\ Verum facile effici potest, ut ea \textit{ (2) }\ Liceret \textit{ L}}} ei difficultati mederi, si malleoli duo percutere et deprimere lineam serratam possent, sed nec hoc fieri potest, quia subjecta rota\protect\index{Sachverzeichnis}{rota} impedit. Illud igitur observandum puto. Quando Linea serrata demissa est a Portatricibus
    %Bnote hier eingefuegt als Annaeherung an tatsaechliche Platzierung
    \edtext{}{\linenum{|21|||22|}\lemma{pag. 156.}\Bfootnote{\textsc{C. Schott}, \cite{00094}\textit{Magia universalis}, Würzburg 1659, pars 4, lib. 2, cap. 6, S.~156.}} et incumbit rotae\protect\index{Sachverzeichnis}{rota} secundae, tunc acicula impactoria tam prope accedat ad chartam, ut prope attingat, et vix duarum acicularum crassities desit. Malleolus ergo postquam quadrans 4\textsuperscript{tus} sonuit impingens  deprimat aciculam in chartam adeo propinquam, ita variatio a perpendiculari, in tantilla distantia erit insensibilis. Porro id quoque praecavendum est, ne linea simplex rotae\protect\index{Sachverzeichnis}{rota} simplici incumbens ea parte qua incipit protrahi praegravet, ideo debebit in altero latere paulum esse crassior, ut casu quolibet aequilibrium\protect\index{Sachverzeichnis}{aequilibrium} servetur. Sed cum ea ratione fiat nimis crassa vel longa, et ita gravis possunt portatrices rem moderari, tam vicini, ut si paulum inclinet incumbat portatrici ejus lateris, ea ratione non opus erit esse longiorem fere, quam tabula est. \edtext{Portatrix autem}{\lemma{est.}\Afootnote{ \textit{ (1) }\ Sed quando \textit{ (2) }\ Portatrix autem \textit{ L}}} lateris ad quod, erit tam remota quam tabula, similiter et portatrix lateris a quo; ne illa aciculae progressum impediat. Sed ita debebit esse duplo longior tabula, ut incumbat simul utrique portatrici. Ergo fiat in latere a quo duplex portatrix altera prope rotam\protect\index{Sachverzeichnis}{rota} secundam, altera ab ea aequaliter fere cum tabula distans, aut aliquantum minus, satis forte si dimidio tabulae distet. Erit linea serrata $\displaystyle1\frac{1}{2}=^{lis}$\rule[-4mm]{0mm}{10mm} tabulae, ut quando egressa est remotiorem lateris a quo portatricem, \edtext{attingat portatricem}{\lemma{attingat}\Afootnote{ \textit{ (1) }\ rem \textit{ (2) }\ portatricem \textit{ L}}} lateris ad quod. In eo autem summa cura ponenda est ut aer ad quemcunque navis\protect\index{Sachverzeichnis}{navis} aut currus motum rotae\protect\index{Sachverzeichnis}{rota} et omnibus annexis circumagendis sufficiat, quare omnia summae tenuitatis, et subtilitatis esse debent, sic tamen ut flecti frangique ad attactum non possint, ideo ferro, seu chalybe, qui neque nimis gravis, neque mollis est, optime fiet, \edtext{nam}{\lemma{fiet,}\Afootnote{ \textit{ (1) }\ nisi assiculis \textit{ (2) }\ nam \textit{ L}}} ossa facile franguntur. Ea autem ab aere circumagenda erunt: 1) Rota\protect\index{Sachverzeichnis}{rota} primata, 2) axis communis rotae\protect\index{Sachverzeichnis}{rota} primatae et simplici et cura iis mobilis, utrinque parieti averso incumbens \edtext{quantulaecunque crassitiei et longitudinis\protect\index{Sachverzeichnis}{longitudo}}{\lemma{}\Afootnote{quantulaecunque crassitiei et longitudinis\protect\index{Sachverzeichnis}{longitudo} \textit{ erg.} \textit{ L}}}, 3) rota\protect\index{Sachverzeichnis}{rota} \edtext{serrata quantulacunque neque enim opus esse aequalem pinnatae, 4) Linea serrata}{\lemma{rota}\Afootnote{ \textit{ (1) }\ pinnata \textit{ (2) }\ simplex, 4) Linea serrata \textit{ (3) }\ serrata [...] serrata \textit{ L}}} quantulaecunque crassitiei longitudine\protect\index{Sachverzeichnis}{longitudo} $\displaystyle1\frac{1}{2}$\rule[-4mm]{0mm}{10mm} tabulae, 5) acicula impactoria crassitiei quantulaecunque, longitudinis\protect\index{Sachverzeichnis}{longitudo} quanta est fere perpendicularis ex summo rotae serratae in tabulam. Porro haec omnia ob situm non sunt graviora quam alias solent, non sunt enim eorum extrema a fulcris remota. \pend \pstart Nam et Axis rotarum, utrinque incumbit\edtext{, et linea serrata in qua acicula impactoria}{\lemma{}\Afootnote{, et [...] impactoria \textit{ erg.} \textit{ L}}}