      
               
                \begin{ledgroupsized}[r]{120mm}
                \footnotesize 
                \pstart                
                \noindent\textbf{\"{U}berlieferung:}   
                \pend
                \end{ledgroupsized}
            
              
                            \begin{ledgroupsized}[r]{114mm}
                            \footnotesize 
                            \pstart \parindent -6mm
                            \makebox[6mm][l]{\textit{L}}Konzept: LH XXXVII 2 Bl. 122\textendash123. 1 Bog. 2\textsuperscript{o}. 2/3 S. auf Bl. 122 r\textsuperscript{o}. Der verbleibende Teil enth\"{a}lt als N. 70 \"{U}berlegungen zu unterschiedlichen technischen Projekten. Diese \"{U}berlegungen werden auf Bl. 123 v\textsuperscript{o} fortgesetzt. 4/5 der rechten Spalte von Bl. 123 v\textsuperscript{o} werden unter dem Titel \textit{De animalibus} in \textit{LSB} VIII, 2 gedruckt. In der linken Spalte von Bl. 122 r\textsuperscript{o} oben befand sich vor der Niederschrift unseres Textes der Zweizeiler: \selectlanguage{french}On porroit procurer un grand amend, \selectlanguage{german}den Leibniz gestrichen hat, um das Papier erneut zu verwenden. Bl. 122 v\textsuperscript{o} und Bl.123 r\textsuperscript{o} leer.\\Kein Eintrag in KK 1 oder Cc 2. \pend
                            \end{ledgroupsized}
                %\normalsize
                \vspace*{5mm}
                \begin{ledgroup}
                \footnotesize 
                \pstart
            \noindent\footnotesize{\textbf{Datierungsgr\"{u}nde}: F\"{u}r die Datierung beziehen wir uns auf N. 70. Unser St\"{u}ck befindet sich zusammen mit N. 70 auf einem Bogen. Dieser enth\"{a}lt auf Bl. LH XXXVII 2 Bl. 123 v\textsuperscript{o} die Mitteilung, dass Herr Tenzel\protect\index{Namensregister}{\textso{Tenzel}} an Tschirnhaus\protect\index{Namensregister}{\textso{Tschirnhaus} (H.v.Tch., H.v.Tsch.), Ehrenfried Walther v. 1651\textendash 1708} eine Relation mit der Bitte um Weiterleitung an den Geheimen Stadtrat geschickt haben soll. Da es sich um einen sehr detaillierten Bericht handelt, ist anzunehmen, dass Leibniz diese Information von Tschirnhaus\protect\index{Namensregister}{\textso{Tschirnhaus} (H.v.Tch., H.v.Tsch.), Ehrenfried Walther v. 1651\textendash 1708} pers\"{o}nlich erhalten hat, den er Ende August 1675 in Paris\protect\index{Ortsregister}{Paris (Parisii)} kennenlernte.}
                \pend
                \end{ledgroup}
            
                \vspace*{8mm}
                \pstart 
                \normalsize
            [122 r\textsuperscript{o}] \selectlanguage{french}Il se peut qu'une lunette\protect\index{Sachverzeichnis}{lunette} paroisse meilleure et ne le soit pas, que Jupiter\protect\index{Sachverzeichnis}{Jupiter} paroisse plus grand, \edtext{et bien clair, et que cependant on}{\lemma{grand,}\Afootnote{ \textit{ (1) }\ et qu'on \textit{ (2) }\ et bien clair, et que cependant on \textit{ L}}} n'en voye point les satellites\protect\index{Sachverzeichnis}{satellite}; au lieu qu'on les peut voir avec une autre qui ne paroist pas si bonne. En voicy le principe. Quand l'oculaire\protect\index{Sachverzeichnis}{oculaire} est petit, et a son foyer\protect\index{Sachverzeichnis}{foyer} eloign\'{e}, l'image qu'on voit dans ce foyer\protect\index{Sachverzeichnis}{foyer}, paroist plus petite \`{a} l'oeil, car elle est plus distante. Cependant si elle est bien distincte dans ce foyer\protect\index{Sachverzeichnis}{foyer} en elle même et marque les satellites\protect\index{Sachverzeichnis}{satellite}, on les y verra; au lieu que lorsque \edtext{les satellites}{\lemma{lorsque}\Afootnote{ \textit{ (1) }\ la lunette\protect\index{Sachverzeichnis}{lunette|textit} \textit{ (2) }\ les satellites \textit{ L}}} ne sont pas marqu\'{e}s dans \edtext{cette image}{\lemma{dans}\Afootnote{ \textit{ (1) }\ ce foyer\protect\index{Sachverzeichnis}{foyer|textit} \textit{ (2) }\ cette image \textit{ L}}}, \edtext{on ne les y verra pas, quoyque l'image}{\lemma{image,}\Afootnote{ \textit{ (1) }\ l'image pourra neantmoi \textit{ (2) }\ il \textit{ (3) }\ on [...] l'image \textit{ L}}} soit aggrandie par l'avoisinement de l'oeil au \edlabel{foyerstart}
            foyer\protect\index{Sachverzeichnis}{foyer}.\pend \pstart  \edtext{Les\edlabel{foyerend}}{{\xxref{foyerstart}{foyerend}}\lemma{foyer.}\Afootnote{ \textit{ (1) }\ Il \textit{ (2) }\ On \textit{ (3) }\ Les \textit{ L}}} lunettes\protect\index{Sachverzeichnis}{lunette} \edtext{ont}{\lemma{lunettes}\Afootnote{ \textit{ (1) }\ qui aggran \textit{ (2) }\ ont \textit{ L}}} eu jusqu'icy peu d'ouuerture, \`{a} cause de la petitesse, mais\pend\newpage\pstart\noindent  ayant des grands \edtext{verres}{\lemma{}\Afootnote{verres \textit{ erg.} \textit{ L}}} objectifs, \edtext{par le moyen des grandes pieces de verre bien claires}{\lemma{}\Afootnote{par [...] claires \textit{ erg.} \textit{ L}}} \edtext{on peut aussi leur donner des grands oculaires}{\lemma{claires}\Afootnote{ \textit{ (1) }\ et aussi des grands oculaires\protect\index{Sachverzeichnis}{oculaire|textit} \textit{ (2) }\ on [...] oculaires \textit{ L}}} et une grande ouuerture. Et par ce moyen on obtient aussi une grande clart\'{e}, jusqu'\`{a} pouuoir voir des choses eloign\'{e}es la nuit, \`{a} cause de la multiplication des images, chacune portant sa clart\'{e} avec elle.\pend \pstart  On peut \edtext{bien}{\lemma{}\Afootnote{bien \textit{ erg.} \textit{ L}}} voir \`{a} travers de ces grands verres comme \textit{AB} \pend        \pstart 
             %\begin{wrapfigure}{l}{0.4\textwidth}                    
                \includegraphics[width=0.2\textwidth]{images/37_2_122r}
                        %\caption{Bildbeschreibung}
                       % \end{wrapfigure}
                        %@ @ @ Dies ist eine Abstandszeile - fuer den Fall, dass mehrere figures hintereinander kommen, ohne dass dazwischen laengerer Text steht. Dies kann zu einer Fahlermeldung fuehren. @ @ @ \\
                               quand même le soleil \textit{C} y fait passer ce rayon de travers; et generalement dans ces grandes pieces lors même qu'une infinit\'{e} d'images paroissent, il en reste toutjours assez.\selectlanguage{latin}\pend 