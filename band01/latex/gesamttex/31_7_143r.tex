\pstart
[143 r\textsuperscript{o}] Les raisons qu'on apporte pour la confirmation de cette hypothese sont sophistiques. Car le ressort\protect\index{Sachverzeichnis}{ressort} \edtext{de l'air}{\lemma{}\Afootnote{de l'air \textit{ erg.} \textit{ L}}} agissant contre un poids\protect\index{Sachverzeichnis}{poids} n'agit pas contre un rien\protect\index{Sachverzeichnis}{rien}: et \edtext{qu'une petite portion d'air pourroit}{\lemma{et}\Afootnote{ \textit{ (1) }\ que la moindre bulle pourroit \textit{ (2) }\ qu'une petite portion d'air pourroit \textit{ L}}} remplir un grandissime espace, ne prouue rien, puisqu'alors il ne trouueroit point de resistence. Mais que la moindre bulle d'air so\^{u}tienne l'effort de toute l'atmosphere\protect\index{Sachverzeichnis}{atmosph\`{e}re}, c'est un paralogisme specieux mais sans solidit\'{e}. Car une bulle donn\'{e}e estant \'{e}galle en forces \`{a} une autre toute semblable ne peut pas \'{e}galer \edtext{en forces}{\lemma{}\Afootnote{en forces \textit{ erg.} \textit{ L}}} toute l'atmosphere\protect\index{Sachverzeichnis}{atmosph\`{e}re}, dont l'autre bulle est une partie; puisqu'\'{e}galer en forces une partie d'un autre c'est estre \textso{moindre} en forces. Mais comment peut elle donc \edtext{combattre}{\lemma{donc}\Afootnote{ \textit{ (1) }\ so\^{u}tenir \textit{ (2) }\ combattre \textit{ L}}} l'effort de toute l'atmosphere\protect\index{Sachverzeichnis}{atmosph\`{e}re}, avec des armes in\'{e}gales? Je responds: parce que l'atmosphere\protect\index{Sachverzeichnis}{atmosph\`{e}re} n'employe pas toutes ses forces \edtext{contre elle ayant \`{a} employer autant d'autres forces ailleurs contre}{\lemma{forces}\Afootnote{ \textit{ (1) }\ pour cela ayant autant d'autres forces \`{a} employer contre au \textit{ (2) }\ contre [...] contre \textit{ L}}} une infinit\'{e} d'autres bulles semblables.\pend \pstart J'ay parl\'{e} \`{a} des autres, qui pretendent, qu'il y a une je ne s\c{c}ay quelle amour\protect\index{Sachverzeichnis}{amour des particules} ou congruence des particules\protect\index{Sachverzeichnis}{congruence des particules} des corps sensibles\protect\index{Sachverzeichnis}{corps!sensible}, en sorte, qu'estant bien joints sans que l'air se trouue entre deux, ils ne se quittent qu'avec peine. Le chancelier Bacon\protect\index{Namensregister}{\textso{Bacon} (Baconus), Francis 1561\textendash 1626} me semble avoir eu \`{a} peu pr\^{e}s cette pens\'{e}e, fond\'{e}e sur quelques autres apparences. Mais si cela estoit, ils ne \edtext{permettroient}{\lemma{ne}\Afootnote{ \textit{ (1) }\ suffriroient \textit{ (2) }\ permettroient \textit{ L}}} pas \`{a} une petite bulle d'air de rompre leur contiguit\'{e} et de se mettre entre la liqueur et la superficie interieure de la phiole. Il faut donc plustost \edtext{que la bulle d'air fasse}{\lemma{}\Afootnote{que la bulle d'air fasse \textit{ erg.} \textit{ L}}} cesser l'effort, de quelque autre force exterieure dont la liqueur purg\'{e}e\protect\index{Sachverzeichnis}{liqueur!purg\'{e}e} est so\^{u}ten\"{u}e. Mais cette force doit apparemment faire un effort plustost que l'avoir un mouuement \edtext{effectif\edlabel{autr143r1},}{\lemma{, autrement}\xxref{autr143r1}{4143r1}\Afootnote{[...] attach\'{e}e  \textbar\ \`{a} la superieure \textit{ erg.}\ \textbar\ , mais [...] \textso{fig. 4.} \textit{ erg.} \textit{ L}}} autrement elle ne so\^{u}tiendroit pas seulement la placque d\'{e}ja attach\'{e}e \`{a} la superieure,\rule[-3.5cm]{0cm}{0cm} mais elle l'eleveroit aussi, et l'y joindroit, quand elle se trouue un peu au dessous, comme dans la \rule[-0.2cm]{0cm}{0cm}\edtext{\textso{fig. 4.}\edlabel{4143r1}%\pend\pstart
}{\lemma{\textso{4.}}\xxref{4143r1}{4143r2}\Afootnote{ \textit{ (1) }\ Pour moy s'il m'est permis, de dire ma pens\'{e}e, quoyque toutes les experiences projett\'{e}es ne soyent pas faites encor \textit{ (2) }\ Un de mes amis avoit cette pensée, que \textit{(a)}\ la pesanteur \textit{(b)}\ l'atmosphere pese sur les placques dans le vuide même faisant entrer par les pores du verre, des ruisseaux d'un air subtil et rafiné. Mais \textit{(aa)}\ \`{a} \textit{(bb)}\ si cela \textit{(cc)}\ cela expliqueroit seulement le phenomene des placques, car si cette pression de l'atmosphere estoit veritable les liqueurs non purg\'{e}es ne tomberoient pas dans le vuide, et une petite bulle d'air dont l'effort n'est point considerable, \`{a} l'\'{e}gard de l'atmosphere, ne les d\'{e}tacheroit non plus.  \textit{ (3) }\ Ce sera peut estre \textit{ (4) }\ Le [...] estre \textit{ L}}}\pend