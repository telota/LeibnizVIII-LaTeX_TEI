   
        
        \begin{ledgroupsized}[r]{120mm}
        \footnotesize 
        \pstart        
        \noindent\textbf{\"{U}berlieferung:}  
        \pend
        \end{ledgroupsized}
      
       
              \begin{ledgroupsized}[r]{114mm}
              \footnotesize 
              \pstart \parindent -6mm
              \makebox[6mm][l]{\textit{L}}Konzept: LH XXXVIII Bl. 17\textendash 18. 1 Bog. 2\textsuperscript{o}. 4 S. zweispaltig. Bl. 17 v\textsuperscript{o} rechts oben die Zeichnung \textit{[Fig. 1]}. Sie ist dem Teil von Pkt. (2) unter der Zwischen\"{u}berschrift: Constructio Machinae zugeordnet, der sp\"{a}ter gestrichen wurde. Die Zeichnung selbst wurde davon ausgenommen und wird im folgenden als g\"{u}ltig wiedergegeben. Die Zeichnungen \textit{[Fig. 2]} und \textit{[Fig. 3]} befinden sich in der unteren rechten H\"{a}lfte von Bl. 18 v\textsuperscript{o}. Sie sind jeweils einer Marginalie zugeordnet. Eine kleinere Zeichnung innerhalb des Textes.\pend
              \end{ledgroupsized}
       
              \begin{ledgroupsized}[r]{114mm}
              \footnotesize 
              \pstart \parindent -6mm
              \makebox[6mm][l]{\textit{E}}\cite{00243}\textsc{Gerland} 1906, S.~203\textendash207.\\Cc 2, Nr. 476 \pend
              \end{ledgroupsized}
        \vspace*{8mm}
       \pstart \begin{center}
        \normalsize
      [17 r\textsuperscript{o}] Propositio Machinae \edtext{Hydrographicae}{\lemma{Machinae}\Afootnote{ \textit{ (1) }\ Limenereuticae\protect\index{Sachverzeichnis}{machina!limenereutica|textit} \textit{ (2) }\ Hydrographicae \textit{ L}}} \end{center} \pend \vspace{1.0ex} \pstart\noindent\hangindent=5mm
      Machinae \edtext{Hydrogra$\upvarphi$icae}{\lemma{Machinae}\Afootnote{ \textit{ (1) }\ Limenereuticae\protect\index{Sachverzeichnis}{machina!limenereutica|textit} \textit{ (2) }\ Hydrogra$\upvarphi$icae \textit{ L}}}, \edtext{si perficiatur}{\lemma{}\Afootnote{si perficiatur \textit{ erg.} \textit{ L}}} fructus erunt \newline (1) inventio loci navis\protect\index{Sachverzeichnis}{navis} \newline (2) delineatio cursus navis\protect\index{Sachverzeichnis}{navis} \newline (3) \edtext{emendatio}{\lemma{(3)}\Afootnote{ \textit{ (1) }\ perfectio \textit{ (2) }\ emendatio \textit{ L}}} Hydrographiae\protect\index{Sachverzeichnis}{hydrographia}, mapparumque nauticarum \edlabel{nauticarumstart}\newline
      (4) \edtext{Supplementum impatientiae ignaviaeque nautarum, pro quibus machina delineandi officium facit.\edlabel{nauticarumend}}{{\xxref{nauticarumstart}{nauticarumend}}\lemma{nauticarum}\Afootnote{ \textit{ (1) }\ (4) navigatio non in rhombo, sed linea recta, \textbar\ (seu accuratius loquendo non in linea spirali sed circulari) \textit{ erg.}\ \textbar\  quantum scilicet, venti, currentes, litora et brevia permittunt. \textit{ (2) }\ (4) [...] facit. \textit{ L}}} \newline \edtext{Quare sequitur.}{\lemma{}\Afootnote{Quare  \textbar\ sequitur. \textit{ erg.}\ \textbar\ }} (5) \edtext{Etsi longitudines\protect\index{Sachverzeichnis}{longitudo} inventae supponerentur, nihilominus summum hujus machinae usum fore ad Geographiam Hydrogra$\upvarphi$iamque\protect\index{Sachverzeichnis}{hydrographia} \edlabel{perficiendasstart}perficiendas}{\lemma{}\Afootnote{(5) Etsi [...] perficiendas. \textit{ erg.} \textit{ L}}}\edlabel{perficiendasend}\edtext{}{{\xxref{perficiendasstart}{perficiendasend}}\lemma{perficiendas.}\Afootnote{ \textit{ (1) }\ Constructio \textit{ (2) }\ Requisita \textit{ L}}}. \pend \clearpage \pstart \begin{center} Requisita\end{center} \pend \vspace{1.0ex} \pstart\noindent\hangindent=10mm Ut cursus navis\protect\index{Sachverzeichnis}{navis} quantum fieri potest exacte delineetur \edtext{(unde caetera sequuntur)}{\lemma{}\Afootnote{(unde caetera sequuntur) \textit{ erg.} \textit{ L}}} opus est haberi, \pend \pstart\noindent\hangindent=10mm (1) \textso{quantitatem cursus navis,}\protect\index{Sachverzeichnis}{navis} seu \edtext{quantae longitudinis}{\lemma{seu}\Afootnote{ \textit{ (1) }\ quantum iter \textit{ (2) }\ quantae longitudinis \textit{ L}}} futura esset chorda per omnia ejus vestigia ducta. \edlabel{ductastart}\pend\pstart\noindent\hangindent=20mm\hspace{20mm}\edtext{Hanc quantitatem cursus navis non difficulter habebimus\edlabel{ductaend}}{\lemma{ducta.}{\xxref{ductastart}{ductaend}}\Afootnote{ \textit{ (1) }\ Hanc \textbar\ quantitatem \textit{ erg.}\ \textbar\ habebimus non difficulter \textit{ (2) }\ Hanc [...] habebimus \textit{ L}}}, \edtext{applicata (loco debito)}{\lemma{}\Afootnote{applicata (loco debito) \textit{ erg.} \textit{ L}}} Rota\protect\index{Sachverzeichnis}{rota}, conversiones suas numerante. \pend 
    \pstart\noindent\hangindent=30mm\hspace{30mm}Numerabit applicatis aliis rotis\protect\index{Sachverzeichnis}{rota} Decadicis, ut in Instrumento Passuum\protect\index{Sachverzeichnis}{instrumentum!passuum}, aut machina Arithmetica\protect\index{Sachverzeichnis}{machina!arithmetica}. \newline 
    Haec Rota\protect\index{Sachverzeichnis}{rota} non est \edtext{usque adeo magnae}{\lemma{est}\Afootnote{ \textit{ (1) }\ tantae \textit{ (2) }\ usque adeo magnae \textit{ L}}} difficultatis\edtext{, et jam aliis in mentem venit. Sed peculiari et hactenus non observata industria opus est, ad efficiendum ne numerus regularitasque conversionum a currentibus maris turbetur,}{\lemma{}\Afootnote{, et jam [...] a \textit{ (1) }\ navis\protect\index{Sachverzeichnis}{navis|textit} \textit{ (2) }\ currentibus maris turbetur, \textit{ erg.} \textit{ L}}} \pend \pstart\noindent\hangindent=10mm (2) \textso{flexus navis}\protect\index{Sachverzeichnis}{navis} \textso{omnes.} \newline Ad hos habendos opus est Re, quae vehatur navi\protect\index{Sachverzeichnis}{navis}, nec tamen flectatur cum navi\protect\index{Sachverzeichnis}{navis} ita enim in navi \protect\index{Sachverzeichnis}{navis} vehentibus flecti videbitur in contrariam partem, ac proinde designabit illis flexus Navis\protect\index{Sachverzeichnis}{navis}. \pend
    \pstart\noindent\hangindent=20mm\hspace{20mm}Corpus quod hoc praestat, una voce, magneticum est. Magnes\protect\index{Sachverzeichnis}{magnes} scilicet, aut acus magnete\protect\index{Sachverzeichnis}{acus!magnetica} imbuta, \pend \pstart\noindent\hangindent=10mm (3) \textso{complicationem quantitatis et flexuum} ut scilicet constet quantum iter intercesserit inter quemlibet flexum. \newline Hoc fieri potest vel homine perpetuo annotante, vel rectius \edlabel{mach1start}Machina. \pend
    \pstart\noindent\hangindent=20mm\hspace{20mm}\edtext{Machina enim nec labore fatigatur, nec negligentia labitur.\edlabel{mach1end}}{\lemma{Machina.}\xxref{mach1start}{mach1end}\Afootnote{ \textit{ (1) }\ Machina ita comparata esse debet ut (1) stylus acus magneticae\protect\index{Sachverzeichnis}{acus!magnetica|textit}, ductus faciat in subjecta mappa (2) mappa subjecta sit mobilis \textit{(a)}\ proportionaliter ad \textit{(b)}\ celeritate\protect\index{Sachverzeichnis}{celeritas|textit} proportionali ad motum navis\protect\index{Sachverzeichnis}{navis|textit}, quod fiet si rota\protect\index{Sachverzeichnis}{rota|textit} cursus cylindrum \textit{ (2) }\ Machina [...] labitur. \textit{ L}}} \pend \clearpage \pstart\centering Constructio Machinae \edlabel{machinaestart} \pend \vspace{1.0ex} \pstart\noindent\edtext{Constabit machina\edlabel{machinaeend}}{{\xxref{machinaestart}{machinaeend}}\lemma{Machinae}\Afootnote{ \textit{ (1) }\ Opus est. \textit{ (2) }\ Constabit machina \textit{ L}}}\pend
    \pstart\noindent\hangindent=10mm (1) ex rota\protect\index{Sachverzeichnis}{rota} primaria \edtext{seu}{\lemma{}\Afootnote{seu \textit{ erg.} \textit{ L}}} cursoria, cujus omnes conversiones simul sumtae, aequant lineam motus navis\protect\index{Sachverzeichnis}{navis} \pend