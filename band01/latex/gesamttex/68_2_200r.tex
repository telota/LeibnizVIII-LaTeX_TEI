      
               
                \begin{ledgroupsized}[r]{120mm}
                \footnotesize 
                \pstart                
                \noindent\textbf{\"{U}berlieferung:}   
                \pend
                \end{ledgroupsized}
            
              
                            \begin{ledgroupsized}[r]{114mm}
                            \footnotesize 
                            \pstart \parindent -6mm
                            \makebox[6mm][l]{\textit{L}}Konzept: LH XXXVIII Bl. 200. 1 Bl. 26 x 21 cm. 2 S. Am rechten und oberen Rand beschnitten. Eine Marginalie am linken Rand von Bl. 200 r\textsuperscript{o} oben, quer zum Text. Die Zeichnung auf Bl. 200 v\textsuperscript{o} in der Mitte rechts, Text umlaufend.\\KK 1, Nr. 1163 \pend
                            \end{ledgroupsized}
                %\normalsize
                \vspace*{5mm}
                \begin{ledgroup}
                \footnotesize 
                \pstart
            \noindent\footnotesize{\textbf{Datierungsgr\"{u}nde}: Die Datierung ergibt sich aus inhaltlichen Gr\"{u}nden. Leibniz entwickelt in dem vorausgehenden St\"{u}ck die Grundidee einer Maschine, die er in dem vorliegenden St\"{u}ck beibeh\"{a}lt, jedoch technisch anders realisiert. Er versucht offenbar, ein und dieselbe Aufgabenstellung durch Anwendung unterschiedlicher Wirkprinzipien zu l\"{o}sen. Die Texte sind so auf\-einander bezogen, dass eine zeitnahe Entstehung wahrscheinlich ist.}
                \pend
                \end{ledgroup}
            
                \vspace*{8mm}
                \pstart 
                \normalsize
            [200 r\textsuperscript{o}] In\footnote{\textit{Am linken Blattrand, quer zur Schreibrichtung}: NB. M. P. sine contrapondio} Machina mea considerandum est attente quantum aeris lignum ascendendo attrahere possit. Pone ligni aut alterius levis ascendentis gravitatem specificam ad aquam collatam esse ut 2. ad 1.  Erit ergo ligni levitas ut 1. orta ab inaequali aquae pressione\protect\index{Sachverzeichnis}{pressio!aquae} se restituere conante. Jam lignum  assurgit a\"{e}ra in follem attrahit. \edtext{Videamus}{\lemma{attrahit.}\Afootnote{ \textit{ (1) }\ Ponamus \textit{ (2) }\ Videamus \textit{ L }\ }} an eum attrahere possit, cum plus \edtext{sic turbetur aqua restituendo,}{\lemma{plus}\Afootnote{ \textit{ (1) }\ turbet se  restitu \textit{ (2) }\ sic turbetur aqua restituendo, \textit{ L }\ }} quam si rem in priore statu reliquisset. Ante omnia poterit  aerem attrahere in tantum spatium paulo minus, quanta est ipsius \edtext{ levis amplitudo seu}{\lemma{}\Afootnote{levis amplitudo seu \textit{ erg.} \textit{ L }\ }} densitas\protect\index{Sachverzeichnis}{densitas}. Videndum an aequilibrio\protect\index{Sachverzeichnis}{aequilibrium} obtento pendeat, ita ut plus non attrahatur nec amplius assurgat. Item an si vi amplius assurgat rursus ab aqua deprimatur. Quod erit novum et notabile experimentum. An potius assurget in distantiam quantumcunque. Quia aerem attrahendo per partes vincit. Nam quolibet momento attrahit se toto, aliquid se minus. Et aqua semel attractionem passa novam [tandem]\edtext{}{\Afootnote{tantandem\textit{\ L \"{a}ndert Hrsg.}}}  quolibet momento patietur. Pone \edtext{usque}{\lemma{}\Afootnote{usque  \textbar\ ad \textit{ gestr.}\ \textbar\ aerem \textit{ L }\ }} aerem paris spatii propemodum intraxisse. Aqua potius paulo altius rursus elevabit, spe totum aqua expellendi. Neque enim natura distinguit quantum ei restet altitudinis, et in hoc consistit ars ipsam eludendi, quaelibet superficies ei pro summa est. Hoc posito vicimus. Et certe pressio aetheris\protect\index{Sachverzeichnis}{pressio!aetheris} etsi lignum non possit elevare extra aquam, conabitur tamen elevare sursum tentandum an si,  quid in aquam diductum mittatur aequalis cum ea circiter gravitatis\protect\index{Sachverzeichnis}{gravitas}, ipso ejus nisu  comprimatur aere per tubum ejecto. Vereor tamen ne contra sit, etsi quaelibet superficies videatur summa. Nam quantum aeris \edtext{in ligno}{\lemma{}\Afootnote{in ligno \textit{ erg.} \textit{ L }\ }} elevabit, tantum altius intraxit. Ergo cum lucrum cessat desistet. Si propriis viribus ageret, semper attraheret, sed quia pressione  exterioris agitur negotium, desistet. \edtext{Unde}{\lemma{desistet.}\Afootnote{ \textit{ (1) }\ Quo \textit{ (2) }\ Unde \textit{ L }\ }} successus illius experimenti erit probatio  quaedam quod gravitas\protect\index{Sachverzeichnis}{gravitas} sit ab \edlabel{externostart}externo. \pend \pstart \edtext{Ponamus ergo\edlabel{externoend}}{{\xxref{externostart}{externoend}}\lemma{externo.}\Afootnote{ \textit{ (1) }\ Sed \textit{ (2) }\ Ponamus ergo \textit{ L }\ }} quod deterrimum est leve non posse attrahere plus aeris una vice, quam quantus est ipsius aer. Unde pro  ligno computationis causa vas cavum adhibebitur. Hoc inquam ponamus. Manifestum est  totum liberatum sic esse leve, aer enim attractus lignum juvat contra pondus baseos.  Ac proinde superabit pondus se majus ut lubet, ob vires \edtext{staterae}{\lemma{vires}\Afootnote{ \textit{ (1) }\ librae \textit{ (2) }\ staterae \textit{ L }\ }} seu vectis\protect\index{Sachverzeichnis}{vectis}. Ut alibi ostensum. Sed fortasse iniri potest nova ratio sine statera\protect\index{Sachverzeichnis}{statera} et vecte\protect\index{Sachverzeichnis}{vectis} sine principiis staticis\protect\index{Sachverzeichnis}{principium staticum}, solis Hydrostaticis\protect\index{Sachverzeichnis}{hydrostatica}, adhibita industria divisionis. Nimirum si semel attraxerit aeris quantum satis est, firmetur follis prior\edtext{, ita ut nec regredi nec progredi ejus tegumentum possit, sed prodeat ex eo tegumentum novum}{\lemma{prior}\Afootnote{ \textit{ (1) }\ novusque \textit{ (2) }\ , ita [...] novum \textit{ L }\ }}, iterum tantundem attrahens. Et sit amplitudo in latitudine non in altitudine. Ut pondus deinde  oppositum parum descendere deprimendo cogatur. Habebit quilibet follis  separatus proprium canalem. Qui ne priori communicet claudatur superior  tubus semel \edtext{attracto}{\lemma{semel}\Afootnote{ \textit{ (1) }\ attractus \textit{ (2) }\ attracto \textit{ L }\ }} per eum aere quanquam fortasse opus non sit, sed prodest  adhibere omnes cautelas. Ita multiplicatio erit in nostra potestate, et habebimus facile levitatem 100 librarum et ultra; tametsi pondus deprimens iterum follem non debeat esse fortius quam est levitas levis appensi attrahentis. Imo minus: posito enim obstaculo quod non patiatur attrahere \edtext{ultra}{\lemma{attrahere}\Afootnote{ \textit{ (1) }\ continue \textit{ (2) }\ ultra \textit{ L }\ }} determinatam quantitatem. Ipsismet  tubis apertis sese sponte contrahet follis. Et forte sic posset \edtext{\textso{pondere opposito omnino careri:}}{\lemma{\textso{pondere opposito omnino careri}}\Afootnote{\textit{doppelt unterstrichen}}} NB.\footnote{\textit{Daneben am Rand gr\"{o}ßer wiederholt}: NB.} folle se sponte ita contrahente, remotis impedimentis ut fiat rursus gravior  sed si sponte se non contrahit, signum est aerem attractioni nullo modo obstitisse ac