 \pstart [100 r\textsuperscript{o}] \textso{Gerick. lib. 5. cap. 10 }\edtext{}{\lemma{\textso{10}}\Bfootnote{\textsc{O. v. Guericke, }\cite{00055}a.a.O., S.~165.}} putat Ricciolus\protect\index{Namensregister}{\textso{Riccioli} (Ricciolus), Giovanni Battista 1598\textendash 1671}  refrangentem aeris sphaeram\protect\index{Sachverzeichnis}{sphaera!aeris} circiter  ad 4 mil. (Germ.) se extendere. \textit{Sol}\protect\index{Sachverzeichnis}{sol}\textit{  et }\textit{Luna}\protect\index{Sachverzeichnis}{luna}\textit{ ad quancunque in }\textit{meridiano}\protect\index{Sachverzeichnis}{meridianus}\textit{  altitudinem suas variant magnitudines} apparentes, et ideo quando  elevatiores sunt seu in signis  borealibus minores videntur, humiliores  vero in meridiano\protect\index{Sachverzeichnis}{meridianus}, seu in signis australibus, majores.\pend 
\pstart \textso{Cap. 11. }\edtext{}{\lemma{\textso{11.}}\Bfootnote{\textsc{O. v. Guericke}, \cite{00055}a.a.O., S.~165\textendash167.}} Non dantur Apogaea\protect\index{Sachverzeichnis}{apogaeum} et perigaea\protect\index{Sachverzeichnis}{perigaeum} solis\protect\index{Sachverzeichnis}{sol} et Lunae\protect\index{Sachverzeichnis}{luna},  nec datur terrae\protect\index{Sachverzeichnis}{terra} aphelium\protect\index{Sachverzeichnis}{aphelium} et perihelium\protect\index{Sachverzeichnis}{perihelium}, \edtext{et}{\lemma{perihelium,}\Afootnote{ \textit{ (1) }\ ad \textit{ (2) }\ et \textit{ L}}} datur motus Eccentricus sed quod solem\protect\index{Sachverzeichnis}{sol} et Lunam\protect\index{Sachverzeichnis}{luna} nunc majores nunc  minores videmus causa est aer (+  haec controversia definiri posset, si sol\protect\index{Sachverzeichnis}{sol} simul observaretur ex diversis  locis, ubi uni est in signis depressioribus, alteri in altioribus, et conferantur  diametri apparentes, notatis caeteris  circumstantiis factisque saepe experimentis, unde determinari posset  an haec a refractione, an a  majore propinquitate +). Exemplum  momentaneae alterationis aeris habes  in \textit{Optica Astronomica} Kepleri\protect\index{Namensregister}{\textso{Kepler} (Keplerus), Johannes 1571\textendash 1630}\edtext{}{\lemma{in}\Bfootnote{\textsc{J. Kepler, }\cite{00066}\textit{Astronomiae pars optica}, Frankfurt 1604, S.~348 (\textit{KGW} II, S.~298).}}  qui asserit anno 1588. die 2 Mart.  captam differentiam in meridiano\protect\index{Sachverzeichnis}{meridianus} altitudinis marginum lunae\protect\index{Sachverzeichnis}{luna} fuisse eodem die modo $31\hspace{-6pt}\raisebox{9pt}{,}\protect\hspace{3pt}$, modo $32\hspace{-6pt}\raisebox{9pt}{,}\protect\hspace{6pt}\displaystyle \frac{1}{3}\rule[-2mm]{0mm}{8mm}$, modo $30\hspace{-6pt}\raisebox{9pt}{,}\protect\hspace{6pt}\displaystyle \frac{1}{4}$ \edtext{}
%{\lemma{modo}\Bfootnote{\protect$30\hspace{-3pt}\raisebox{9pt}{,}\displaystyle \frac{1}{4}$ ist ein Kopierfehler von Guericke. Vgl. das Original von Kepler \protect$30\hspace{-3pt}\raisebox{9pt}{,}\displaystyle \frac{3}{4}$, Vgl. auch: O. v. Guericke, \cite{00056}\textit{Neue Magdeburger Versuche}\textit{, hg. von H. Schimank, D\"{u}sseldorf 1968, S.~291.}}}
{\lemma{modo}\Bfootnote{$30\protect\hspace{-6pt}\protect\raisebox{9pt}{,}\protect\hspace{6pt}\protect\displaystyle \protect\frac{1}{4}$ ist ein Kopierfehler von Guericke. Vgl. das Original von Kepler $30\protect\hspace{-6pt}\protect\raisebox{9pt}{,}\protect\hspace{6pt}\protect\displaystyle \protect\frac{3}{4}$  , vgl. auch: O. v. Guericke, \cite{00056}\textit{Neue Magdeburger Versuche}, hg. von H. Schimank, D\"{u}sseldorf 1968, S.~291.}}
minut. et die praecedente $33\hspace{-6pt}\raisebox{9pt}{,}\protect\hspace{3pt}$. Deinde Luna\protect\index{Sachverzeichnis}{luna} existente in media  longitudine anno 1591 die 22 Febr. observata, bis $31\hspace{-6pt}\raisebox{9pt}{,}\protect\hspace{3pt}$ sexies $32\hspace{-6pt}\raisebox{9pt}{,}\protect\hspace{3pt}$ septies  $33\hspace{-6pt}\raisebox{9pt}{,}\protect\hspace{3pt}$, sexies $34\hspace{-6pt}\raisebox{9pt}{,}\protect\hspace{3pt}$ minut. Ecce variationes unius diei. Et Gemma Frisius\protect\index{Namensregister}{\textso{Gemma Frisius van den Steen,} Rainer 1508\textendash 1555}\edtext{}{\lemma{Et}\Bfootnote{\textsc{R. Gemma Frisius, }\cite{00052}\textit{De radio astronomico}, Paris 1558, S.~36.}}  testatur in \textit{Radio Astronomico} observatam sibi anno 1542. die 15 Xbr. paulo  post quadraturam Lunae\protect\index{Sachverzeichnis}{luna} diametrum apparentem $30\hspace{-6pt}\raisebox{9pt}{,}\protect\hspace{3pt}$ min. quo  tempore debuit juxta Ptolemaeum\protect\index{Namensregister}{\textso{Ptolemaeus,} Claudius v. Alexandria 85?\textendash 165?}  (quia Lunae\protect\index{Sachverzeichnis}{luna} perigaeae\protect\index{Sachverzeichnis}{perigaeum} in quadraturis  dat distantiam semid. terrae\protect\index{Sachverzeichnis}{terra} $33\hspace{-6pt}\raisebox{9pt}{,}\protect\hspace{6pt}\displaystyle\frac{1}{2}\rule[-4mm]{0mm}{10mm}$ unde sequeretur diametrum apparentem  tunc fuisse $56\hspace{-6pt}\raisebox{9pt}{,}\protect\hspace{3pt}$ min. hoc est pene  duplo majorem, quam $\rightmoon^{nae}$\protect\index{Sachverzeichnis}{luna} Apogaeae\protect\index{Sachverzeichnis}{apogaeum}  in copulis esse) $50\hspace{-6pt}\raisebox{9pt}{,}\protect\hspace{3pt}$ min et plus. Sed  alii longe minorem diametrum Lunae\protect\index{Sachverzeichnis}{luna}  observarunt in omni quadra. Ricciol.\protect\index{Namensregister}{\textso{Riccioli} (Ricciolus), Giovanni Battista 1598\textendash 1671} lib. 4 cap. 14. n. 2.\edtext{}{\lemma{2.}\Bfootnote{\textsc{G. Riccioli, }\cite{00086} a.a.O., S.~223.}} Sol\protect\index{Sachverzeichnis}{sol} ex dictis  debet major apparere nobis  in signo Australi Capricorno\protect\index{Sachverzeichnis}{Capricornus}, \edtext{ seu tempore hyberno, quam}{\lemma{Capricorno,}\Afootnote{ \textit{ (1) }\ quam \textit{ (2) }\  seu tempore hyberno, quam \textit{ L}}} alias  tempore aestivo, in Cancro\protect\index{Sachverzeichnis}{Cancer}, signo  boreali. Nec unquam conveniunt  Astronomi in assignandis Apogaeis\protect\index{Sachverzeichnis}{apogaeum}  et perigaeis\protect\index{Sachverzeichnis}{perigaeum} et apparentibus in iis  diametris luminarium. Et quod solis\protect\index{Sachverzeichnis}{sol}  et Lunae\protect\index{Sachverzeichnis}{luna} Apogaeum\protect\index{Sachverzeichnis}{apogaeum} ex Eclipsibus\protect\index{Sachverzeichnis}{eclipsis}  probare volunt Ricciolus\protect\index{Namensregister}{\textso{Riccioli} (Ricciolus), Giovanni Battista 1598\textendash 1671} \textit{Alm.} lib. 3 cap. 10 schol. num. 2 et lib 4. cap.  16 in schol. incertum fatetur. Quod  vero allegant majorem minoremque  inventam Lunae\protect\index{Sachverzeichnis}{luna} parallaxin in  eadem distantia a vertice, opus est  observationibus et forte fieri potest, ut nonnunquam Terra\protect\index{Sachverzeichnis}{terra} Lunam\protect\index{Sachverzeichnis}{luna} nonnihil  attrahat ut globus plumulam. Etiam  istae parallaxium differentiae ex diversa  aeris constitutione esse possunt. Copernicus\protect\index{Namensregister}{\textso{Copernicus,} Nicolaus 1473\textendash 1543} et Tycho\protect\index{Namensregister}{\textso{Brahe} (Tycho), Tycho 1546\textendash 1601} quippe septentrionaliores minorem  parallaxin quam Ptolemaeus\protect\index{Namensregister}{\textso{Ptolemaeus,} Claudius v. Alexandria 85?\textendash 165?} invenere  et Ptolemaicam contrahendam putavere. \textit{Dicunt Astronomi ex observationibus  aequinoctiorum constare }\textit{solem}\protect\index{Sachverzeichnis}{sol}\textit{ in  semicirculo boreali (ab initio scilicet }\edtext{\textit{Arietis}}{\lemma{\textit{scilicet}}\Afootnote{ \textit{ (1) }\ \textit{aeris} \textit{ (2) }\ \textit{Arietis} \textit{ L}}}\textit{ per }\textit{Cancrum}\protect\index{Sachverzeichnis}{Cancer}\textit{ ad }\textit{Libram}\protect\index{Sachverzeichnis}{libra}\textit{)  versari diebus 186 vel 187, at  in semicirculo Australi ab initio }\textit{Librae}\protect\index{Sachverzeichnis}{libra}\textit{ per }\textit{Capricornum}\protect\index{Sachverzeichnis}{Capricornus}\textit{ ad Arietem}\protect\index{Sachverzeichnis}{Aries}\textit{ diebus duntaxat ab 178 vel 179. Diversitatem hanc itaque tribuunt }\textit{apogaeo}\protect\index{Sachverzeichnis}{apogaeum}\textit{  et }\textit{perigaeo}\protect\index{Sachverzeichnis}{perigaeum}\textit{ }\textit{solis}\protect\index{Sachverzeichnis}{sol}\textit{, quod scilicet in eo  quadrante in quo longiores moras efficit  sit }\textit{Apogaeum}\protect\index{Sachverzeichnis}{apogaeum}\textit{ }\textit{solis}\protect\index{Sachverzeichnis}{sol}\textit{ et contra. Sed ab }\textit{Hipparcho}\protect\index{Namensregister}{\textso{Hipparchos,} ca. 190\textendash 120 v. Chr.}\textit{  usque ad }\textit{Albategnium}\protect\index{Namensregister}{\textso{Al-Battani} (Albategnius), ca. 850/869\textendash 929}\textit{ morabatur in primo  quadrante }\textit{Ecliptico}\protect\index{Sachverzeichnis}{eclipsis}\textit{ pluribus diebus et horis  quam in 2}\textsuperscript{\textit{do}}\textit{ quadrante,  et in secundo pluribus quam in quarto,  et in quarto pluribus quam in tertio  ita ut in semicirculo ascendente a Capricorni}\protect\index{Sachverzeichnis}{Capricornus}\textit{ scilicet initio per Arietem}\protect\index{Sachverzeichnis}{Aries}\textit{ ad  initium }\textit{Cancri}\protect\index{Sachverzeichnis}{Cancer}\textit{ pluribus moraretur,  quam in semicirculo descendente  ab initio }\textit{Cancri}\protect\index{Sachverzeichnis}{Cancer}\textit{ per }\textit{Libram}\protect\index{Sachverzeichnis}{libra}\textit{ ad initium }\textit{Capricorni}\protect\index{Sachverzeichnis}{Capricornus}\textit{, consequenter cursus }\textit{solis}\protect\index{Sachverzeichnis}{sol}\textit{ jam  foret inversus et moram quam habuit  tempore }\textit{Hipparchi}\protect\index{Namensregister}{\textso{Hipparchos,} ca. 190\textendash 120 v. Chr.}\textit{ a }\textit{Libra }\protect\index{Sachverzeichnis}{libra}\textit{per }\textit{Capricornum}\protect\index{Sachverzeichnis}{Capricornus}\textit{  ad Arietem}\protect\index{Sachverzeichnis}{Aries}\textit{, jam habet ab Ariete}\protect\index{Sachverzeichnis}{Aries}\textit{ per Capricornum}\protect\index{Sachverzeichnis}{Capricornus}\textit{ ad }\textit{Libram}\protect\index{Sachverzeichnis}{libra}\textit{.} Ergo tunc  quoque ab Hipparcho\protect\index{Namensregister}{\textso{Hipparchos,} ca. 190\textendash 120 v. Chr.} ad Albategnium\protect\index{Namensregister}{\textso{Al-Battani} (Albategnius), ca. 850/869\textendash 929} Apogaeum\protect\index{Sachverzeichnis}{apogaeum} fuit in signis Australibus quod  non concedent [Astronomi]\edtext{}{\Afootnote{Astrologi\textit{\ L \"{a}ndert Hrsg.}}}.\edtext{}{\lemma{Astrologi\textit{\ L }}\Bfootnote{Korrektur nach: \textsc{O. v. Guericke}, \cite{00055}a.a.O., S.~167.}} Ergo res ista  omnis incerta. Adde quod motus in Tabulis uniformis apparentiae ab aere consentit Claramont.\protect\index{Namensregister}{\textso{Chiaramonti} (Claramont), Scipione 1565\textendash 1652}\edtext{}{\lemma{Claramont}\Bfootnote{\textsc{S. Chiaramonti}, \cite{00271}\textit{Opus de universo}, Köln 1644, 3. Buch.}} \textit{de universo} lib. 3 et Fracastorius\protect\index{Namensregister}{\textso{Fracastoro} (Fracastorius), Girolamo 1478\textendash1553}. 
\pend