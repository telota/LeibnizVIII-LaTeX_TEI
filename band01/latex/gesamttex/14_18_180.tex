\pend \pstart[p.~180] P. Francisco de la Chaise\protect\index{Namensregister}{\textso{La Chaise,} Fran\c{c}ois de SJ 1624\textendash 1709} Societatis nostrae,\footnote{\textit{Leibniz unterstreicht}: Francisco [...] nostrae} qui pro sua humanitate, Lugdunensium, Aquentium, et Parisiensium obseruationum me participem fecit; [...].\pend \pstart X. Huc reuoca cometas crinitos\protect\index{Sachverzeichnis}{cometa!crinitus}; imo licet vltimus caudatus esset, adhibita tamen telescopij\protect\index{Sachverzeichnis}{telescopium} opera, caput illius visum est albicante, eoque densissimo capillitio inuolutum die 2. Aprilis; vt Lugduno\protect\index{Ortsregister}{Lyon (Lugdunum)} ad me scripsit die 7. Aprilis, idem qui supra, sed nunquam satis laudatus, P. Franciscus de la Chaise\protect\index{Namensregister}{\textso{La Chaise,} Fran\c{c}ois de SJ 1624\textendash 1709}:\footnote{\textit{Leibniz unterstreicht}: nunquam [...] de la Chaise} cogita magnam et longe lateque dispersam lignorum diuersi generis struem accensam, et numera, si potes, tot flammulas, tot pyramidas, et vt analogia melius congruat, cogita totam orbis terraquei superficiem ita conflagrantem,\footnote{\textit{Leibniz unterstreicht}: totam orbis [...] conflagrantem\\ \textit{Daneben am Rand}: Vossius\protect\index{Namensregister}{\textso{Vossius} (Voss.), Isaac 1618\textendash 1689}} sparsa huc illuc heterogenea materia, seu ignis pabulo; [...]. 