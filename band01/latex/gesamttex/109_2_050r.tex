\pstart Pater Kircher praefatio partis 5. in libro II. \textit{Geographiae Ma}
[50 r\textsuperscript{o}] \cite{00067}\textit{gneticae} refert, quendam S. J.\edtext{}{\lemma{S. J.}\Bfootnote{\textsc{A. Kircher}, \cite{00067}a.a.O., S.~292.}} Colonia\protect\index{Ortsregister}{Koln@K\"{o}ln (Colonia)} ad se misisse tabulam de longitudine\protect\index{Sachverzeichnis}{longitudo} et Latitudine\protect\index{Sachverzeichnis}{latitudo} Romae\protect\index{Ortsregister}{Rom (Roma)} et Coloniae\protect\index{Ortsregister}{Koln@K\"{o}ln (Colonia)}, in qua 15 de ea autorum discrepantes sententias ostendit. Ibidem inquit, \textit{non ita pridem in }\textit{Brasilia}\protect\index{Ortsregister}{Brasilien (Brasilia)}\textit{ vastissimus fluv. }\textit{origliana}\protect\index{Ortsregister}{Amazonas (fluv. origliana)}\textit{ innumera utrinque hominum multitudine habitatus, quique totam }\textit{latitudinem}\protect\index{Sachverzeichnis}{latitudo}\textit{ }\textit{Americae}\protect\index{Ortsregister}{Amerika (America)} \textit{in } \textit{Quitum}\protect\index{Ortsregister}{Quito (Quitum)}\textit{ usque subtendit, non sine ingenti Hispanorum emolumento est detectus. Et pari Zelo quendam Caesaraugustanorum freti }\textit{Californiae}\protect\index{Ortsregister}{Kalifornien (California)}\textit{ vicinarumque regionem explorationem propriis sumtibus Heroico sane ausu non ita pridem aggredi coepisse audio, qua quidem brevi iter nobis huc usque desideratum in }\textit{Oceanum septentrionalem}\protect\index{Ortsregister}{Nordmeer (Oceanus septentrionales)}\textit{ detecturum speramus.}\edtext{}{\lemma{\textit{speramus.}}\Bfootnote{\textsc{A. Kircher, }\cite{00067}a.a.O., S.~293.}} \pend 
\pstart \edtext{[Filamenta]}{\lemma{Filamentorum}\Afootnote{\textit{\ L \"{a}ndert Hrsg.}}} ex foliis aloes ad aliquid suspendendum maximi usus sunt enim et tenacissima, et tenuissima ut prope visum fugiant.\edtext{}{\lemma{fugiant.}\Bfootnote{\textsc{A. Kircher, }\cite{00067}a.a.O., S.~308.}}
\pend 
\pstart Part. 6. probl. 2.\edtext{}{\lemma{probl. 2.}\Bfootnote{\textsc{A. Kircher, }\cite{00067}a.a.O., S.~358. }} pater Kircherus\protect\index{Namensregister}{\textso{Kircher} (Kircherus), Athanasius SJ 1602\textendash 1680} refert ex Stevino\protect\index{Namensregister}{\textso{Stevin} (Stevinus), Simon 1548\textendash 1620} et describit pyxin admodum commodam in gradus 360 divisam.
\pend 
\pstart P. Christoph. Burrus\protect\index{Namensregister}{\textso{Burrus,} Christoph  SJ} Ulyssipone\protect\index{Ortsregister}{Lissabon (Ulyssipolis)} agens cum videretur sibi ex ratione variationis magneticae reperisse longitudinum\protect\index{Sachverzeichnis}{longitudo} inveniendarum rationem ejus rei praemium 50000 ducatorum a Rege Catholico petiit, sed frustra. Haec P. Kircher lib. 2. parte 6. cap. 1. probl. 6.\edtext{}{\lemma{probl. 6.}\Bfootnote{\textsc{A. Kircher}, \cite{00067}a.a.O., S.~359. }} 
\pend 
\pstart P. Kircher d. l. problem. 7.\edtext{}{\lemma{problem. 7.}\Bfootnote{\textsc{A. Kircher, } \cite{00067}a.a.O., S.~362.}} inquit temporis exacti difficilem esse inventionem per horologia\protect\index{Sachverzeichnis}{horologium} sine sole\protect\index{Sachverzeichnis}{sol} et stellis\protect\index{Sachverzeichnis}{stella}. Et ideo plerique autores incumbunt in inventionem alicujus automatis, quod perpetuo moveatur cum sole\protect\index{Sachverzeichnis}{sol} ac stellis\protect\index{Sachverzeichnis}{stella}. Si darentur exacta ejusmodi Automata possent \edtext{differentiae longitudinis }{\lemma{possent}\Afootnote{ \textit{ (1) }\ tempus \textit{ (2) }\ differentiae longitudinis  \textit{ L}}} diversorum locorum exacte notari, dum sciremus quanto temporis intervallo a priore loco discessissemus, et ibi conferendo cum sole\protect\index{Sachverzeichnis}{sol}, sciremus longitudinis\protect\index{Sachverzeichnis}{longitudo} differentiam. Et P. Kircher\protect\index{Namensregister}{\textso{Kircher} (Kircherus), Athanasius SJ 1602\textendash 1680} putavit rem instituendam per Motum perennem naturalem, quem ipse flatu venti procurat.
\pend 
\pstart \edtext{\edlabel{Klammern1start}[IV. in}{{\xxref{Klammern1start}{Klammern1end}}\lemma{[...]}\Afootnote{\textit{Klammern von Leibniz}}} praeclare inventa sunt seminaria artificiorum 1. attractio ferri. 2. versio ad Polum\protect\index{Sachverzeichnis}{polus}. 3. inclinatio\protect\index{Sachverzeichnis}{inclinatio} \edtext{ab horizonte}{\lemma{}\Afootnote{ab horizonte \textit{ erg.} \textit{ L}}} pro latitudine\protect\index{Sachverzeichnis}{latitudo}, sub elevatione Poli\protect\index{Sachverzeichnis}{elevatio!poli} inventa a Gilberto\protect\index{Namensregister}{\textso{Gilbert} (Gilbertus), William 1544\textendash 1603}\edtext{}{\lemma{Gilberto}\Bfootnote{\textsc{W. Gilbert, }\cite{00053}a.a.O., S.~13\textendash15. }}. 4. correctio variationis a Polo\protect\index{Sachverzeichnis}{polus} per P. Jacobum 
 Grandamicum\protect\index{Namensregister}{\textso{Grandami} (Grandamicus), Jacques SJ 1588\textendash 1672}, examinata et approbata a P. Zucchio\protect\index{Namensregister}{\textso{Zucchi} (Zucchius), Niccol\`{o} SJ 1586\textendash 1670}, Kirchero\protect\index{Namensregister}{\textso{Kircher} (Kircherus), Athanasius SJ 1602\textendash 1680}, Schotto\protect\index{Namensregister}{\textso{Schott} (Schottus), Caspar SJ 1608\textendash 1666}.]\edlabel{Klammern1end} 
 \pend 
