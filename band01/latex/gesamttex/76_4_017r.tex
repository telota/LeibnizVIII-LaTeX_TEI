      
               
                \begin{ledgroupsized}[r]{120mm}
                \footnotesize 
                \pstart                
                \noindent\textbf{\"{U}berlieferung:}   
                \pend
                \end{ledgroupsized}
            
              
                            \begin{ledgroupsized}[r]{114mm}
                            \footnotesize 
                            \pstart \parindent -6mm
                            \makebox[6mm][l]{\textit{$E^1$}}\textsc{G. W. Leibniz}, \textit{Notitia Opticae Promotae}, Frankfurt a. M. 1671. LH XXXVII 2 Bl. 17\textendash 18. 1 Bog. 8\textsuperscript{o}. 4 S. 
 Bl. 16 r\textsuperscript{o} Titelseite. F\"{u}r die Wiedergabe des Drucks in unserer Reihe wird das Typoskript den Regeln der Akademie-Ausgabe angepasst. Kursiv gesetzte Hervorhebungen in der Druckvorlage werden gesperrt wiedergegeben.\\Kein Eintrag in KK 1 oder Cc 2. \pend
                            \end{ledgroupsized}
              
                            \begin{ledgroupsized}[r]{114mm}
                            \footnotesize 
                            \pstart \parindent -6mm
                            \makebox[6mm][l]{\textit{$E^2$}}\textsc{G. W. Leibniz}, \textit{Notitia Opticae Promotae}, in: \textsc{Dutens} III, S.~14f. \pend
                            \end{ledgroupsized}
              
                            \begin{ledgroupsized}[r]{114mm}
                            \footnotesize 
                            \pstart \parindent -6mm
                            \makebox[6mm][l]{\textit{$E^3$}}\textsc{G. W. Leibniz}, \textit{Notice de l'optique avanc\'{e}e}, in: \textit{Oeuvres concernant la physique}, hrsg. von \textsc{J. Peyroux}, Paris 1985, S. 5f.\pend
                            \end{ledgroupsized}
                \vspace*{8mm}
                \pstart 
                \normalsize
            \centering [17 r\textsuperscript{o}] NOTITIA\\  OPTICAE PROMOTAE.\\  Autore\\  G. G. L. L.\\  FRANCOFURTI,\\ Apud Joh. David. Zunnerum.\\  1671. \pend \vspace{1.0ex} \pstart