[46\textendash052, 074\textsuperscript{o}] Bei den folgenden St\"{u}cken handelt es sich um ein Textcorpus, das Leibniz nachtr\"{a}glich strukturiert hat. In der urspr\"{u}nglichen Fassung wurden die  \"{U}berlegungen zum Problem der L\"{a}ngengradbestimmung sukzessive auf 16 Seiten im Folioformat niedergeschrieben. Diese hat Leibniz sp\"{a}ter auf Bl. 46 r\textsuperscript{o} mit dem Zusatz Intra finem anni 1668 et initium 1669 sowie den st\"{u}ckkonstituierenden \"{U}berschriften \textit{Longit. 1}, \textit{Longit. 2.} usw. versehen. Bl. 74 v\textsuperscript{o} enth\"{a}lt den Entwurf zu einem \textit{Instrumentum longitudinum}, auf den in  \textit{Longit 2.} Bezug genommen wird. Wir ordnen die Beschreibung dieses Instruments zur L\"{a}ngengradbestimmung als eigenst\"{a}ndiges St\"{u}ck als N. 2\textsubscript{2} unmittelbar vor \textit{Longit 2.} ein.