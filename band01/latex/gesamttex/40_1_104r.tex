[104 r\textsuperscript{o}]  quia Tubus in Experimento Torricelliano\protect\index{Sachverzeichnis}{experimentum Torricellianum}  est plenus, ideo quanto altior est lapsus, seu major compressio  tanto etiam \edtext{plus est}{\lemma{etiam}\Afootnote{ \textit{ (1) }\ fortius est \textit{ (2) }\ plus est \textit{ L}}} Mercurii\protect\index{Sachverzeichnis}{mercurius} comprimentis.  Quam rationem verissimam esse hoc experimento  confirmari potest.\footnote{\textit{In der rechten Spalte}: \textso{Exp. 13.}} \edtext{Esto Tubus \textit{AB} apertus  sed claudibilis in \textit{A} infundatur ei Mercurius  per orificium \textit{B} ita tamen ut non perveniat usque ad \textit{A}, sed inter \textit{A} et Mercurium maneat spatium Mercurio vacuum}{\lemma{potest.}\Afootnote{ \textit{ (1) }\ Sumatur Mercu \textit{ (2) }\ Si Tubus \textit{AB} Mercurio\protect\index{Sachverzeichnis}{mercurius|textit} plenus non sit  \textit{(a)}\ ex majore altitudine Mercurius\protect\index{Sachverzeichnis}{mercurius|textit} suspensus manebit, et tanto quidem  \textit{(aa)}\ a latere \textit{A} sit vero a la \textit{(bb)}\ in \textit{A}  \textit{(b)}\ invertendus plenus non sit  \textit{(aa)}\ sed \textit{(bb)}\ habeatque Mercurium\protect\index{Sachverzeichnis}{mercurius|textit} in fundo versus \textit{B} spatio vacuo\protect\index{Sachverzeichnis}{spatium vacuum|textit} (id est aere) existente. \textit{ (3) }\ Esto Tubus \textit{AB} apertus  sed  \textit{(a)}\ firmabilis in \textit{A} \textit{(b)}\ claudibilis in \textit{A} infundatur ei  \textit{(aa)}\ aer \textit{(bb)}\ Mercurius [...] vacuum \textit{ L}}}  seu aere plenum, claudatur apertura \textit{A}  Tubusque statuatur erectus \edtext{orificio \textit{B}}{\lemma{erectus}\Afootnote{ \textit{ (1) }\ in vase \textit{C} \textit{ (2) }\ orificio \textit{B} \textit{ L}}} deorsum verso \textit{C}, \edtext{Mercurius suspensus  manebit}{\lemma{\textit{C},}\Afootnote{ \textit{ (1) }\ Mercurius\protect\index{Sachverzeichnis}{mercurius|textit} non effluet \textit{ (2) }\ Mercurius suspensus  manebit \textit{ L}}} ultra altitudinem consuetam altitudine  tanta, quanta est spatii in Tubo Vacui relicti. Hujus  Experimenti manifesta ratio est. Nam si Tubus esset  plenus, Mercurius\protect\index{Sachverzeichnis}{mercurius} fuisset depressus ad altitudinem consuetam, \edtext{seu aer compressus ad gradum solitum}{\lemma{}\Afootnote{seu [...] solitum \textit{ erg.} \textit{ L}}}  ergo cum non sit plenus tantum aberit \edtext{aer a compressione ordinaria}{\lemma{aberit}\Afootnote{ \textit{ (1) }\ ab aeris compressione \textit{ (2) }\ aer a compressione ordinaria \textit{ L}}}, \edtext{  quantum ponderis abest Tubo minus pleno}{\lemma{ordinaria,}\Afootnote{ \textit{ (1) }\ quantum deest ponderis plenitudini \textit{ (2) }\ quantum [...] pleno \textit{ L}}}; ergo tanto \edtext{minus vincetur ejus resistentia}{\lemma{tanto}\Afootnote{ \textit{ (1) }\ amplius  virium \textit{ (2) }\ minus vincetur ejus resistentia \textit{ L}}}, quantum Tubo  ponderis deest, ac proinde tantum pondus ultra solitum  sustinebit, quantum deest Tubo. Quod si Mercurius\protect\index{Sachverzeichnis}{mercurius}  Tubo \textit{A} immissus sit\footnote{\textit{In der rechten Spalte}: \textso{Exp. 14.}} \edtext{a latere \textit{A}}{\lemma{sit}\Afootnote{ \textit{ (1) }\ usque ad \textit{A} \textit{ (2) }\ a latere \textit{A} \textit{ L}}}  non vero a latere \textit{B} \edtext{ut \textit{AI} eadem evenient ut  in Tubo Torricelliano ordinario, si modo \textit{IB} abscissum  fingas, Mercurius scilicet eousque delabetur infra \textit{I}}{\lemma{\textit{B}}\Afootnote{ \textit{ (1) }\ aut in medio positus neque  latus \textit{A} neque latus \textit{B} attingat   \textbar\ ut \textit{A} vel \textit{MI} \textit{ erg.}\ \textbar\ , eadem regula manet,  perinde enim est ac si \textit{BI}  abscindatur, nihil enim confert descendetque Mercurius\protect\index{Sachverzeichnis}{mercurius|textit}  per altitudinem tantam \textit{ (2) }\ ut [...] delabetur  \textbar\ infra \textit{I} \textit{ erg.}\ \textbar\  \textit{ L}}}, donec  27 \edtext{(30)}{\lemma{}\Afootnote{(30) \textit{ erg.} \textit{ L}}} pollices ultra \textit{I} emineant. Unde si totus Mercurius\protect\index{Sachverzeichnis}{mercurius} non major sit 27. (vel 30.) pollicibus  nihil delabetur in Tubo.\footnote{\textit{In der rechten Spalte}: \textso{Exp. 15.}}\pend
 \pstart  Quod si Mercurius\protect\index{Sachverzeichnis}{mercurius}  in medio pendeat \edtext{et nec \textit{A} nec \textit{B}}{\lemma{pendeat}\Afootnote{ \textit{ (1) }\ inter \textit{A} et \textit{B} \textit{ (2) }\ et nec \textit{A} nec \textit{B} \textit{ L}}} attingat ut \textit{MI} fingendum est supra \textit{AM}  infra \textit{IB} esse abscissa;\footnote{\textit{In der rechten Spalte}: \textso{Exp. 16.}} et in genere semper  cogitandum, quasi Tubus tantae esset longitudinis solum  quanta a Mercurio\protect\index{Sachverzeichnis}{mercurius} impletur, ita semper ex illa  longitudine ut \textit{MI} descendet quicquid  in ea est ultra illos 27 \edtext{pollices, etsi ob Tubi longitudinem  ad fundum \textit{B} pervenire non possit.}{\lemma{pollices,}\Afootnote{ \textit{ (1) }\ residuum in Tubo  paulum infra \textit{I} descendet e \textit{ (2) }\ etsi [...] possit. \textit{ L}}} Ita Mercurium\protect\index{Sachverzeichnis}{mercurius purgatus}  etiam ab aere non purgatum  vel ex summo Tubo, vel inter duos aeres pendulum\protect\index{Sachverzeichnis}{pendulum} habebimus, nec opus erit ad Baroscopium\protect\index{Sachverzeichnis}{baroscopium}  vase subjecto. \pend
  \pstart       
                 Eadem omnia evenient si Tubus  simplex utrinque cui aliquid Mercurii\protect\index{Sachverzeichnis}{mercurius} immissum  sit sumatur.\footnote{\textit{In der rechten Spalte}: \textso{Exp. 17.}} Nihil enim refert Tubus \textit{AB} ipse, aut vas \textit{D} in quod orificium ejus apertum \textit{B} desinit, sit clausum.  Semper enim in spatio Tubi a Mercurio\protect\index{Sachverzeichnis}{mercurius} impleto non  nisi 27 (30) pollices remanebunt.\footnote{\textit{In der rechten Spalte}: \textso{Exp. 18.}} Hinc Mercurius\protect\index{Sachverzeichnis}{mercurius} 