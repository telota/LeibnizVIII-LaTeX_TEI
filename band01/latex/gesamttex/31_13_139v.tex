
\pstart  J'oserois m\^{e}me dire que \edtext{presque}{\lemma{presque}\Afootnote{\textit{ erg.} \textit{ L}}} tous les coups qui entrent entre \textit{AC} et \textit{BD} peuuent estre cont\'{e}s pour rien, puisque ceux qui viennent de \textit{G} par \textit{AC} \'{e}galent et d\'{e}truisent \edtext{la pluspart de}{\lemma{d\'{e}truisent}\Afootnote{ \textit{ (1) }\ m\^{e}me \textit{ (2) }\ la pluspart de \textit{ L}}} ceux qui viennent de \textit{H} par \textit{BD} les uns cherchant \edtext{une}{\lemma{une}\Afootnote{ \textit{ erg.} \textit{ L}}} sortie par o\`{u} les autres trouuent \edtext{leur}{\lemma{}\Afootnote{leur \textit{ erg.} \textit{ L}}} entr\'{e}e. \edtext{Mais il n'est pas necessaire de s'\'{e}tendre d'avantage}{\lemma{entr\'{e}e.}\Afootnote{ \textit{ (1) }\ Il y a point d'obstacle \textit{ (2) }\ Donc  \textit{(a)}\ point d \textit{(b)}\ rien qui \textit{ (3) }\ Mais [...] d'avantage \textit{ L}}} la dessus. Nous voyons par l\`{a}, que rien empeche que le mouuement interieur de la liqueur ne porte la placque inferieure libre \textit{CD} vers la superieure fixe \textit{AB}, s'il y auroit m\^{e}me un poids\protect\index{Sachverzeichnis}{poids} attach\'{e} \`{a} la dite placque parce qu'il est certain que le mouuement de la liqueur \'{e}gale ce poids\protect\index{Sachverzeichnis}{poids}, s'il est vray, qu'il peut so\^{u}tenir la dite placque \edtext{avec le poids, d\'{e}ja joincte \`{a} la superieure}{\lemma{placque}\Afootnote{ \textit{ (1) }\ d\'{e}ja attach\'{e}e \textit{ (2) }\ avec [...] superieure \textit{ L}}}. Il reste seulement de faire \edtext{l'elision du peu de fluide\protect\index{Sachverzeichnis}{fluide}}{\lemma{l'elision}\Afootnote{ \textit{ (1) }\ de finir \textit{ (2) }\  du fluide ou matiere f \textit{ (3) }\ du peu de fluide \textit{ L}}} qui se trouue entre les deux placques\protect\index{Sachverzeichnis}{deux placques} mais on peut faire voir par l'experience que la force necessaire \`{a} une telle elision surtout dans le vuide\protect\index{Sachverzeichnis}{vide} \edtext{et dans l'air}{\lemma{}\Afootnote{et dans l'air \textit{ erg.} \textit{ L}}} est imperceptible, et peut seurement estre cont\'{e}e pour rien. Il est donc manifeste enfin, que l'iss\"{u}e de nostre \edtext{experience nous fera juger infalliblement}{\lemma{experience}\Afootnote{ \textit{ (1) }\ decidera entierement \textit{ (2) }\ nous fera juger infalliblement \textit{ L}}} de la verit\'{e} de cette Hypothese, laquelle se trouue asseur\'{e}ment en danger puisqu'il \edtext{est demontr\'{e}, que}{\lemma{puisqu'il}\Afootnote{ \textit{ (1) }\ faut, que \textit{ (2) }\ est demontr\'{e}, que \textit{ L}}} le mouuement \edtext{de la liqueur ambiente}{\lemma{mouuement}\Afootnote{ \textit{ (1) }\ interieur \textit{ (2) }\ de la liqueur ambiente \textit{ L}}} ne pourra pas so\^{u}tenir la pesanteur de la placque inferieure, s'il ne \edtext{peut aussi la so\^{u}lever}{\lemma{peut}\Afootnote{ \textit{ (1) }\ pas  \textit{(a)}\ la so\^{u}tenir \textit{(b)}\  la so\^{u}lever ny \textit{ (2) }\ aussi la so\^{u}lever \textit{ L}}} pour la joindre \`{a} la superieure.
\pend