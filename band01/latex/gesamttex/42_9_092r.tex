[92 r\textsuperscript{o}]  ac sese fortissime explicans, divulsioni laminarum\protect\index{Sachverzeichnis}{laminae politae}, quippe  denuo compressurae resistit. Eadem causa est cur siphonis\protect\index{Sachverzeichnis}{sipho}\selectlanguage{polutonikogreek} ἑτερομήκους\selectlanguage{latin} phaenomenon, quo aqua ex vase aliquo elicitur,  quamdiu crus siphonis\protect\index{Sachverzeichnis}{sipho} extra aquam positum descendit infra aquae \edtext{superficiem,}{\lemma{superficiem,}\Afootnote{\textbar\ in vacuo \textit{ gestr.}\ \textbar\ quod \textit{ L}}} quod \edtext{a nostri tempori philosophis ad columnae aereae pressionem}{\lemma{quod}\Afootnote{ \textit{ (1) }\ ab omnibus ad aeris gravitatem\protect\index{Sachverzeichnis}{gravitas!aeris|textit} \textit{ (2) }\ a [...] pressionem \textit{ L}}} relatum  est, pendeat disjunctive sive a gravitate\protect\index{Sachverzeichnis}{gravitas} sive ab Elatere\protect\index{Sachverzeichnis}{elater}.  Nam si fit in aere libero gravitati\protect\index{Sachverzeichnis}{gravitas} columnae aereae debetur; \edtext{si fit}{\lemma{debetur;}\Afootnote{ \textit{ (1) }\ quae tota \textit{ (2) }\ si fit \textit{ L}}} in aere clauso debetur ejus Elaterio\protect\index{Sachverzeichnis}{elaterium} \edtext{comprimi ultra renuentis}{\lemma{comprimi}\Afootnote{ultra renuentis \textit{ erg.} \textit{ L}}}; si fit  in Recipiente exhausto, fit materiae residuae Elaterio\protect\index{Sachverzeichnis}{elaterium} sese ob vicinorum corporum pressionem cessantem libere  explicantis. Prorsus quemadmodum constat\edtext{}{\lemma{}\Afootnote{constat  \textbar\ in eodem Recipiente \textit{ gestr.}\ \textbar\ aerem \textit{ L}}}  aerem in vesica quadam flaccida contentum expandere  sese et vesicam inflare, ubi primum ab aere circumjecto premi  desiit, quod fit quando aer circumjectus exhauritur, aut quando in montem excelsum ascenditur, ubi minor est aeris pressio\protect\index{Sachverzeichnis}{pressio!aeris}.\pend \pstart  Ex his jam constat praeclara haec duo Experimenta novissime  ab Hugenio\protect\index{Namensregister}{\textso{Huygens} (Hugenius, Vgenius, Hugens, Huguens), Christiaan 1629\textendash 1695} publicata (laminarum\protect\index{Sachverzeichnis}{laminae politae} et siphonis\protect\index{Sachverzeichnis}{sipho} iniquicruri \edtext{in Vacuo quod vocant seu Recipiente exausto}{\lemma{in}\Afootnote{[...] exausto \textit{ erg.} \textit{ L}}}\edtext{}{\lemma{exausto}\Bfootnote{\textsc{Chr. Huygens, }\cite{00062}a.a.O., S.~139f. (\textit{HO} VII, S.~205f.).}}) nulla  alia causa indigere, quam Elaterio\protect\index{Sachverzeichnis}{elaterium} corporis \edtext{aere communi subtilioris}{\lemma{corporis}\Afootnote{ \textit{ (1) }\ subtilis \textit{ (2) }\ aere communi subtilioris \textit{ L}}} residui,  in \edtext{Recipiente,}{\lemma{Recipiente,}\Afootnote{\textbar\ quod nihil aliud esse videri,  quam aerem attenuatum, postea dicam, \textit{ gestr.}\ \textbar\ ab \textit{ L}}} ab Elaterio\protect\index{Sachverzeichnis}{elaterium} aeris  nihil differente.\pend \pstart \edtext{ Ostendam aliquando fusius ab eodem aeris Elaterio Gravitati mixto omnia frigoris  phaenomena posse manifestissime derivari. Nam data  aeris massa rarefacta plusque spatii occupans, necesse est atmosphaeram  aut totam elevet in liquidum aethera\protect\index{Sachverzeichnis}{aether}, aut tantundem comprimat,  quantum ipsa dilatatur; unde cum aer apud nos rarefit,  apud alios, aut etiam in suprema regione comprimitur.\pend \pstart  Resistit autem aer gravitate sua elevationi, Elaterio compressioni;  quare ubi cessat vis dilatans restituit omnia in statum aequilibrii\protect\index{Sachverzeichnis}{aequilibrium}  prioris, massamque aeris antea rarefactam quantum potest comprimit,  id est fit frigus. Quoties aer vasi inclusus summe incalescit, necesse  est vasis ipsius latera aut aperiri non nihil aut comprimi,}{\lemma{differente.}\Afootnote{ \textit{ (1) }\ Ostendam aliquando fusius, quod obiter hoc loco monebo, ab eodem aeris Elaterio\protect\index{Sachverzeichnis}{elaterium|textit}   \textbar\ gravitati\protect\index{Sachverzeichnis}{gravitas|textit} mixto \textit{ erg.}\ \textbar\  omnia \textso{frigoris}\protect\index{Sachverzeichnis}{frigus|textit} phaenomena rectissime derivari. Nam caloris\protect\index{Sachverzeichnis}{calor|textit}  vis aeris particulas   \textbar\ invicem separat ac \textit{ erg.}\ \textbar\  plurimum inter se  \textbar\ invicem \textit{ gestr.}\ \textbar\  spatii relinquere cogit;  hinc sequuntur duo, primum ut  \textit{(a)}\ aer totus \textit{(b)}\ plus spatii \textit{(c)}\ pars \textit{(d)}\ massa\protect\index{Sachverzeichnis}{massa|textit} quaedam  aeris data, plus quam ante spatii occupet, quod non nisi invita  columnae aereae incumbentis gravitate\protect\index{Sachverzeichnis}{gravitas|textit} fieri potest; deinde ut  exiguae aeris particulae et quasi bullae comprimantur: nam  si columna renititur, necesse est   \textbar\ singulas \textit{ erg.}\ \textbar\  aeris particulas comprimi, ut calor\protect\index{Sachverzeichnis}{calor|textit} inter eas spatium faciat; contra si earum particularum  \textit{(aa)}\ Elaterium\protect\index{Sachverzeichnis}{elaterium|textit} \textit{(bb)}\ Elater\protect\index{Sachverzeichnis}{elater|textit} compressioni renititur, necesse est columnam aeream atmosphaerae\protect\index{Sachverzeichnis}{atmosphaera|textit} elevari: cum autem concurrant renisus uterque, tum Elaterii\protect\index{Sachverzeichnis}{elaterium|textit} tum gravitatis\protect\index{Sachverzeichnis}{gravitas|textit}; necesse est   \textbar\ in aere libero \textit{ erg.}\ \textbar\  effectum inter eos dividi, et  partim comprimi aeris particulas, partim totam massam\protect\index{Sachverzeichnis}{massa|textit} datam  plus spatii occupare, seu columnam aerem incumbentem elevare.  In aere vero vase quodam concluso, qui  majus spatium utique occupare non potest, necesse est  vel latera vasis, vel ipsas aeris particulas vel potius utrasque comprimi; \textit{ (2) }\ Ostendam [...] Elaterio  \textbar\ Gravitati mixto \textit{ erg.}\ \textbar\ omnia [...] Nam  \textbar\ calore \textit{ gestr.}\ \textbar\ data [...] aer \textit{(a)}\ nostri climatis \textit{(b)}\ apud [...] comprimitur. Resistit [...] comprimi, \textit{ L}}}  unde fit ut ampulla vitrea quae valde incaluit, frigidae  subito immersa rumpatur, quia ejus compressio subito laxatur, at  inaequaliter; hinc partes \edtext{aliae}{\lemma{partes}\Afootnote{ \textit{ (1) }\ quaedam \textit{ (2) }\ aliae \textit{ L}}} se dilatant, aliae nondum  sequi possunt unde ruptura: si ab omni latere aequaliter, licet  subito refrigeraretur, non facile rumperetur.\pend \pstart  Scio quosdam contra aeris \edtext{gravitatem\protect\index{Sachverzeichnis}{gravitas!aeris} ratiocinatos}{\lemma{gravitatem}\Afootnote{ \textit{ (1) }\ ita \textit{ (2) }\ ratiocinatos \textit{ L}}} ex \edtext{Thermometri  phaenomeno}{\lemma{ex}\Afootnote{ \textit{ (1) }\ ipsa \textit{ (2) }\ Thermometri  phaenomeno \textit{ L}}}. Habent Thermometrum\protect\index{Sachverzeichnis}{thermometrum} quod inversum vocant, quia \edtext{aqua in eo}{\lemma{quia}\Afootnote{ \textit{ (1) }\ aer in eo \textit{ (2) }\ aqua in eo \textit{ L}}} ascendit aereque altior est; ejus haec constructio  est. \edtext{Sumitur}{\lemma{est.}\Afootnote{ \textit{ (1) }\ Esto \textit{ (2) }\ Sumitur \textit{ L}}} Ampulla vitrea \edtext{aqua semiplena}{\lemma{vitrea}\Afootnote{ \textit{ (1) }\ aere semi \textit{ (2) }\ aqua semiplena \textit{ L}}},  per orificium immittitur canna oblonga vitrea, in aquam  usque prope ad fundum pertingens. Orificium Ampullae  caemento conjungitur cannae, ne aer exspirare  possit, ita aer in ampulla \edtext{calore\protect\index{Sachverzeichnis}{calor}}{\lemma{}\Afootnote{calore\protect\index{Sachverzeichnis}{calor} \textit{ erg.} \textit{ L}}} dilatatus aquam ejusdem ampullae  deprimit, aqua depressa in \edtext{cannam supra apertam invita columnae aereae incumbentis gravitate}{\lemma{in}\Afootnote{ \textit{ (1) }\ canalem ascendit \textit{ (2) }\ cannam [...] gravitate ascendit. \textit{ L}}}