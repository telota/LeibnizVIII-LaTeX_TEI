\footnotesize\pstart Bei den folgenden zwei St\"{u}cken handelt es sich um Varianten der Ausarbeitung eines und desselben Themas. Mit Ausnahme der Einleitung weichen die Versionen deutlich von einander ab, so dass wir sie separat wiedergeben. Gegenstand der beiden St\"{u}cke ist Leibniz' Auseinandersetzung mit der These Descartes'\protect\index{Namensregister}{\textso{Descartes} (Cartesius, des Cartes, Cartes.), Ren\'{e} 1596\textendash 1650}, dass sich das Licht im dichteren Medium schneller bewegt als im d\"{u}nneren. In der urspr\"{u}nglichen Fassung diskutiert er das von Descartes\protect\index{Namensregister}{\textso{Descartes} (Cartesius, des Cartes, Cartes.), Ren\'{e} 1596\textendash 1650} entworfene Modell zur Begr\"{u}ndung dieses Zusammenhangs und gibt in dem zweiten St\"{u}ck eine eigene Erkl\"{a}rung daf\"{u}r, die auf der Unterscheidung von conatus simplex und conatus continue reparatus beruht. Die Datierung ergibt sich aus N. 51. In diesem Text wird mit Blick auf das Brechungsgesetz "une demonstration nouuelle toute claire et mecanique" angek\"{u}ndigt. Es ist davon auszugehen, dass das vorliegende St\"{u}ck die Durchf\"{u}hrung dieser Ank\"{u}ndigung darstellt.\pend\normalsize