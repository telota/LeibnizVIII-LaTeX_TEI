
 \pstart  Methodus generalis ex his patet computandi compressiones\protect\index{Sachverzeichnis}{compressio} tensionesque\protect\index{Sachverzeichnis}{tensio}. Data esse debet vis comprimens, seu \edtext{quantum scilicet spatii res Elastica}{\lemma{seu}\Afootnote{ \textit{ (1) }\ magnitudo rei Elasticae, \textit{ (2) }\ quantum scilicet spatii res Elastica \textit{ L}}} data antea occupaverit libera, et nunc occupet compressa; data etiam esse debet rei novae comprimendae tendendaeque, de qua eventum calculare volumus, vis resistens, seu quantum spatii occupet libera; restat \edtext{investigandum}{\lemma{}\Afootnote{investigandum \textit{ erg.} \textit{ L}}} quantum occupare debeat compressa. \edtext{Exempli causa}{\lemma{compressa.}\Afootnote{ \textit{ (1) }\ Sumatur ejus pars aequalis parti tensi corpori \textit{ (2) }\ Exempli causa \textit{ L}}} vis\protect\index{Sachverzeichnis}{vis!aeris compressi} data compressit aerem datum naturaliter spatium duorum pedum implentem, in spatium unius pedis, quaeritur vis eadem aerem unius pedis spatium implentem, in quod spatium comprimet? Respondetur in spatium quartae pedis. \edtext{Non tantum in spatium dimidii ejus. Quia dimidium ejus potentiae}{\lemma{pedis.}\Afootnote{ \textit{ (1) }\ Quia vis \textit{ (2) }\ Non [...] potentiae \textit{ L}}} sufficit ad comprimendum in spatium dimidii pedis, ut patet cum duplum ejus in spatium unius pedis se toto, ergo dimidium ejus in spatium dimidii pedis se dimidio compressit.
 [112 r\textsuperscript{o}] Ex his intelligi potest, si scimus quantum potentia data comprimere possit corpus quoddam datum, et quaeritur quantum comprimere possit aliud datum cujus ratio cognita est ad prius datum; tunc ita procedendum: \edtext{determinetur quantum}{\lemma{procedendum:}\Afootnote{ \textit{ (1) }\ cogitentur quantum \textit{ (2) }\ exploretur \textit{ (3) }\ determinetur quantum \textit{ L}}} utrumque corpus impleat spatii in statu naturali\protect\index{Sachverzeichnis}{status naturalis}: Quod si jam secundum minus spatii \edtext{naturaliter}{\lemma{}\Afootnote{naturaliter \textit{ erg.} \textit{ L}}} occupat quam primum, tunc sumatur pars \edtext{primi v. g. dimidia aequale}{\lemma{pars}\Afootnote{ \textit{ (1) }\ secundi aequalis \textit{ (2) }\ primi v. g. dimidia aequale \textit{ L}}} spatium occupans, ac proinde si homogenea sunt corpora, ut suppono (nam si heterogenea ut aer et lana, calculo ad homogeneitatem redigenda sunt) aequalis secundo. Cumque totum primi, compressum sit in spatium cognitum, cognoscetur etiam in quantum spatium compressa sit dicta pars primi, dimidia scilicet in spatium, quod ita est ad totum spatium ut ipsa \edtext{pars}{\lemma{}\Afootnote{pars \textit{ erg.} \textit{ L}}} est ad totum \edtext{suum}{\lemma{}\Afootnote{suum \textit{ erg.} \textit{ L}}} corpus, nempe primum \edtext{seu in spatii partem dimidiam a parte potentiae eandem quoque ad totam potentiam rationem servante, seu dimidia}{\lemma{}\Afootnote{  \textbar\ seu in  \textit{ (1) }\ spatium dimidium \textit{ (2) }\ spatii partem dimidiam \textit{ erg.}\ \textbar\  a [...] dimidia \textit{ erg.} \textit{ L}}}. Jam cum secundum corpus non sit nisi tantundem, comprimetur et ipsum a \edtext{parte dicta potentiae}{\lemma{a}\Afootnote{ \textit{ (1) }\ vi data \textit{ (2) }\ parte dicta potentiae \textit{ L}}} in idem spatium, \textso{sed} prioris spatii dimidium. Ergo a dupla potentia comprimetur in spatium quadruplum. Hinc compressiones\protect\index{Sachverzeichnis}{compressio} erunt in duplicata corporum ratione.
\pend