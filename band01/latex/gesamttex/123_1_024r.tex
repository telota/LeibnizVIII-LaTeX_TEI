   
        
        \begin{ledgroupsized}[r]{120mm}
        \footnotesize 
        \pstart        
        \noindent\textbf{\"{U}berlieferung:}  
        \pend
        \end{ledgroupsized}
      
       
              \begin{ledgroupsized}[r]{114mm}
              \footnotesize 
              \pstart \parindent -6mm
              \makebox[6mm][l]{\textit{L}}Notiz: LH XXXV 15, 6 Bl. 24. 1 Bl. 4\textsuperscript{o}. 1 S., R\"{u}ckseite leer. Alle R\"{a}nder be\-schnitten, der untere Rand nach rechts hin aufsteigend.\\Cc 2, Nr. 507 \pend
              \end{ledgroupsized}
        %\normalsize
        \vspace*{5mm}
        \begin{ledgroup}
        \footnotesize 
        \pstart
      \noindent\footnotesize{\textbf{Datierungsgr\"{u}nde}: Die Datierung erfolgt aufgrund des Wasserzeichens, das sich auch auf dem Papier einer Sammlung von mathematischen Problemen findet, die sich \"{u}ber die Bl\"{a}tter LH XXXV 2, 1 Bl. 293\textendash302, 312\textendash313 erstreckt. Ein Teil dieser Sammlung wurde in \textit{LSB} VII, 1 N. 35 gedruckt. Wir \"{u}bernehmen die dort angegebene Datierung.}
        \pend
        \end{ledgroup}
      
        \vspace*{8mm}
        \pstart 
        \normalsize
      [24 r\textsuperscript{o}] Morini\protect\index{Namensregister}{\textso{Morin} (Morinus), Jean-Baptiste 1583\textendash 1656} liber \textit{De Longitudinibus} \edtext{}{\lemma{liber}\Bfootnote{\textsc{J. B. Morin, }\cite{00080}\textit{Longitudinum terrestrium scientia}, Paris 1634.}}non est contemnendus: Omnes Astronomi, et inter hos celeberrimi viri, Longomontanus\protect\index{Namensregister}{\textso{Longomontanus,} Christen Sørensen 1562\textendash 1647}, Hortensius\protect\index{Namensregister}{\textso{Hortensius,} Martinus 1605\textendash 1639}, ipse Gassendus\protect\index{Namensregister}{\textso{Gassendi} (Gassendus), Pierre 1592\textendash 1655}, Gaulterius\protect\index{Namensregister}{\textso{Gaultier} (Gaulterius), Jacques}, Vallesius\protect\index{Namensregister}{\textso{Valesius} (Vallesius), Jacobus}, imo in Epistola ad Belligrandium\protect\index{Namensregister}{\textso{Beaugrand,} Jean de 1595\textendash 1640} Galilaeus\protect\index{Namensregister}{\textso{Galilei} (Galilaeus, Galileus), Galileo 1564\textendash 1642} confessi sunt, omnia quae attulerit praeclare explicata, Geometriceque demonstrata. \edtext{}{\lemma{demonstrata.}\Bfootnote{\textsc{J. B. Morin}, \cite{00080}\textit{Longitudinum terrestrium scientia}, S.~173. }}P. Furnerius\protect\index{Namensregister}{\textso{Fournier} (Fournerius), Georges SJ 1595\textendash 1652} dixit \edtext{in \textit{Hydrogra$\phi$ia}}{\lemma{}\Afootnote{in \textit{Hydrogra$\phi$ia} \textit{ erg.} \textit{ L}}} eum optime de Astronomia meritum, ob radicalia ejus fundamenta exactissime excussa; \edtext{}{\lemma{excussa;}\Bfootnote{\textsc{G. Fournier, }\cite{00046}\textit{L'Hydrographie}, Paris 1643, S.~589. }}Longomontanus\protect\index{Namensregister}{\textso{Longomontanus,} Christen Sørensen 1562\textendash 1647} erat Morini\protect\index{Namensregister}{\textso{Morin} (Morinus), Jean-Baptiste 1583\textendash 1656} admirator et \edtext{approbator, }{\lemma{approbator,}\Bfootnote{\textsc{J. B. Morin, }	\cite{00080}\textit{Longitudinum terrestrium scientia}, S.~179f. }}donec a Dano quodam qui Morinum\protect\index{Namensregister}{\textso{Morin} (Morinus), Jean-Baptiste 1583\textendash 1656} Parisiis\protect\index{Ortsregister}{Paris (Parisii)} viderat fuit edoctus Morinum\protect\index{Namensregister}{\textso{Morin} (Morinus), Jean-Baptiste 1583\textendash 1656} ipsum non esse observatorem, nec nisi in Musaeo suo ista speculando invenisse. Ab eo tempore ausus est scribere in Morinum\protect\index{Namensregister}{\textso{Morin} (Morinus), Jean-Baptiste 1583\textendash 1656}. Judices Morino\protect\index{Namensregister}{\textso{Morin} (Morinus), Jean-Baptiste 1583\textendash 1656} a Richelio\protect\index{Namensregister}{\textso{Richelieu,} Armand Jean Du Plessis 1585\textendash 1642} \edtext{dati,}{\lemma{dati,}\Bfootnote{\textsc{J. B. Morin, } \cite{00080}\textit{Longitudinum terrestrium scientia}, S.~127.}} Pascalius\protect\index{Namensregister}{\textso{Pascal} (Pascalius), \'{E}tienne 1588\textendash 1651} praeses, Mydorgius\protect\index{Namensregister}{\textso{Mydorge} (Mydorgius), Claude 1585\textendash 1647}, Beaugrand\protect\index{Namensregister}{\textso{Beaugrand,} Jean de 1595\textendash 1640}, Herigonus\protect\index{Namensregister}{\textso{H\'{e}rigone} (Herigonus), Pierre 1580\textendash 1643} etc. \edtext{}{\lemma{etc.}\Bfootnote{\textsc{J. B. Morin, }	\cite{00080}\textit{Longitudinum terrestrium scientia}, S.~134. }}et ex Nauarchis Beaulieu\protect\index{Namensregister}{\textso{Boulliau} (Bullialdus), Ismael 1605\textendash 1694} aliique primum in publico conventum approbaverant, cum quod responderent non haberent, postea mutavere sententiam magno suo dedecore, nec sine invidiae manifesta suspicione scripto edito, quo satis levia nec nisi generalia de difficultate praxeos et observationum, item, quod alii similes methodos dudum iniissent, quae tamen immensum differebant, continebantur. Memorabilis est ipsa actio seu collatio, ob rerum praeclaram a Morino\protect\index{Namensregister}{\textso{Morin} (Morinus), Jean-Baptiste 1583\textendash 1656} discussarum copiam, ipsamque methodum, qua eos in nassam confessionis pertraxit. \edtext{Plerique}{\lemma{pertraxit.}\Afootnote{ \textit{ (1) }\ Coegit \textit{ (2) }\ Plerique \textit{ L}}} Mathematici coacti sunt fateri praeclaras esse ante omnia methodos Morini\protect\index{Namensregister}{\textso{Morin} (Morinus), Jean-Baptiste 1583\textendash 1656} ad longitudines\protect\index{Sachverzeichnis}{longitudo} terrestres. \edtext{De maritimis}{\lemma{De}\Afootnote{ \textit{ (1) }\ terrestribus \textit{ (2) }\ maritimis \textit{ L}}} dubitavere, at Beaulieu\protect\index{Namensregister}{\textso{Boulliau} (Bullialdus), Ismael 1605\textendash 1694} aliique Nauarchi dixere, primum sibi sufficere si fit inventio qua non minus quam 2 gradibus erretur. \edtext{}{\lemma{erretur.}\Bfootnote{\textsc{J. B. Morin, }	\cite{00080}\textit{Longitudinum terrestrium scientia}, S.~194. }}Deinde, se non desperare exacte fieri a nautis posse, observationes quae in terra fieri possunt. Memorabile est quod ab iisdem nautis dictum, si qua methodus longitudinum\protect\index{Sachverzeichnis}{longitudo} nullam \edtext{aliam pateretur}{\lemma{aliam}\Afootnote{ \textit{ (1) }\ haberet \textit{ (2) }\ pateretur \textit{ L}}} difficultatem, quam calculi scientiaeve Astronomicae aut Geometricae etc. se ea fore contentissimos, satisque diligentiae ad eam a fundamentis comprehendendam allaturos. Notabile quod refert Morinus\protect\index{Namensregister}{\textso{Morin} (Morinus), Jean-Baptiste 1583\textendash 1656} de quodam olitore Osia Feronc\'{e}\protect\index{Namensregister}{\textso{Feronce} (Feronc\'{e}), Ozias } Gallo, praeclaro observatore, Gassendo\protect\index{Namensregister}{\textso{Gassendi} (Gassendus), Pierre 1592\textendash 1655} aliisque noto. \edtext{}{\lemma{noto.}\Bfootnote{\textsc{J. B. Morin, } \cite{00080}\textit{Longitudinum terrestrium scientia}, S.~177. Vgl. auch \textsc{J. B. Morin, }\cite{00138}\textit{Lettres}, Paris 1635, S.~20.}}
      \pend 
      \pstart Refert Morinus\protect\index{Namensregister}{\textso{Morin} (Morinus), Jean-Baptiste 1583\textendash 1656} tria inventa, quorum ope exiguis Instrumentis tantum effici possit, quantum ingentibus illis Tychonicis habebat ad manus quendam Ferrier\protect\index{Namensregister}{\textso{Ferrier} }, insignem instrumentorum Parisiis\protect\index{Ortsregister}{Paris (Parisii)} artificem.\edtext{}{\lemma{artificem.}\Bfootnote{\textsc{J. B. Morin, }	\cite{00080}\textit{Longitudinum terrestrium scientia}, S.~187. }} Quidam\footnote{\textit{Interlinear \"{u}ber} Vernier\protect\index{Namensregister}{\textso{Vernier,} Pierre 1580\textendash 1637} \textso{nisi fallor nobilis Burgundus:} Est per divisionem circuli contrariam adhibita ala.} \textso{Vernier}\protect\index{Namensregister}{\textso{Vernier,} Pierre 1580\textendash 1637}\textso{ nisi fallor nobilis Burgundus} detexerat modum quendam talem. Item alius per circulos transversales. Affert et tertium cujus nunc non memini.\pend \pstart Affert Morinus\protect\index{Namensregister}{\textso{Morin} (Morinus), Jean-Baptiste 1583\textendash 1656} pinnulas quasdam observationibus aptatas, inter alias quarum ope inveniri possit centrum lunae, quaecunque sit ejus aetas, quo scilicet opus est.\edtext{}{\lemma{est.}\Bfootnote{\textsc{J. B. Morin, } \cite{00080}\textit{Longitudinum terrestrium scientia}, S.~195. }}\pend \pstart Novus nuntius sidereus Morini\protect\index{Namensregister}{\textso{Morin} (Morinus), Jean-Baptiste 1583\textendash 1656},\edtext{}{\lemma{Morini,}\Bfootnote{\textsc{J. B. Morin, }\cite{00080}\textit{Longitudinum terrestrium scientia}, S.~207. }} seu observationes stellarum interdiu, ope Telescopii, notat cincinnos a stellis tunc abscissos esse, et ideo non nisi maximas sic videri. Disputatio est inter Frommium\protect\index{Namensregister}{\textso{Fromm} (Frommius), Georg 1605\textendash 1651} \edtext{\edlabel{longomontanostart}pro Longomontano\protect\index{Namensregister}{\textso{Longomontanus,} Christen Sørensen 1562\textendash 1647} scribentem\edlabel{longomontanoend}}{{\xxref{longomontanostart}{longomontanoend}}\lemma{}\Afootnote{pro Longomontano\protect\index{Namensregister}{\textso{Longomontanus,} Christen Sørensen 1562\textendash 1647} scribentem \textit{ erg.} \textit{ L}}} et Morinum\protect\index{Namensregister}{\textso{Morin} (Morinus), Jean-Baptiste 1583\textendash 1656} an fumus faciat refractionem\protect\index{Sachverzeichnis}{refractio}, ut et nubes etc. videri enim non nisi interspersa esse corpora, ut clathri, etc.\edtext{}{\lemma{etc.}\Bfootnote{\textsc{G. Fromm, }\cite{00119}\textit{Dissertatio astronomica}, Kopenhagen 1642.}}\pend \pstart Laudat Morinus\protect\index{Namensregister}{\textso{Morin} (Morinus), Jean-Baptiste 1583\textendash 1656} turrim Astronomicam jussu Regis Christiani IV\textsuperscript{ti}\protect\index{Namensregister}{\textso{D\"{a}nemark:} Christian IV., K\"{o}nig von D\"{a}nemark 1588\textendash 1648} extructam Hafniae\protect\index{Ortsregister}{Kopenhagen (Hafnia)}.\pend \pstart Morinus\protect\index{Namensregister}{\textso{Morin} (Morinus), Jean-Baptiste 1583\textendash 1656} ait ex omnibus instrumentis Astronomicis, sufficere solum quadrantem. Cum aliquando Telescopio instruit, contra disputaverat nonnihil Longomontanus\protect\index{Namensregister}{\textso{Longomontanus,} Christen Sørensen 1562\textendash 1647}.\pend \pstart Morinus\protect\index{Namensregister}{\textso{Morin} (Morinus), Jean-Baptiste 1583\textendash 1656} ostendit omnia resolvi tantum in veram solis parallaxin\protect\index{Sachverzeichnis}{parallaxis} eam sub Tropico nullo negotio haberi posse, et \edtext{illic habitantibus}{\lemma{et}\Afootnote{ \textit{ (1) }\ nationibus \textit{ (2) }\ illic habitantibus \textit{ L}}}, quasi jure naturali \edtext{debere restitutionem}{\lemma{debere}\Afootnote{ \textit{ (1) }\ confectiones \textit{ (2) }\ restitutionem \textit{ L}}} tabularum Astronomicarum\protect\index{Sachverzeichnis}{tabulae!Astronomicae}. Habita vera parallaxi\protect\index{Sachverzeichnis}{parallaxis} seu loco solis, haberi \edtext{loca omnium}{\lemma{haberi}\Afootnote{ \textit{ (1) }\ ejus \textit{ (2) }\ loca omnium \textit{ L}}} fixarum, seu correctionem Tabulae Tychonicae de illis et horum ope haberi restitutionem parallaxeos\protect\index{Sachverzeichnis}{parallaxis} Lunaris etc. Diggesaeum Anglum praeclara coepisse de parallaxibus\protect\index{Sachverzeichnis}{parallaxis} sed ingeniosas ejus demonstrationes Geometricas demtis paucis a praxi remotas.\edtext{}{\lemma{remotas.}\Bfootnote{\textsc{J. B. Morin}, \cite{00080}\textit{Longitudinum terrestrium scientia}, S.~214. }} Morinus\protect\index{Namensregister}{\textso{Morin} (Morinus), Jean-Baptiste 1583\textendash 1656} attulit methodum parallaxeos\protect\index{Sachverzeichnis}{parallaxis} solaris reperiendae a Beaunio\protect\index{Namensregister}{\textso{Beaune} (Beaunius), Florimond de 1601\textendash 1652} et Robervallio\protect\index{Namensregister}{\textso{Roberval} (Robervallius), Gilles Personne de 1602\textendash 1675} approbatam, qui et ejus aequationem temporis probavere,\edtext{}{\lemma{probavere,}\Bfootnote{\textsc{J. B. Morin, }	\cite{00080}\textit{Longitudinum terrestrium scientia}, S.~267.}} cujus inventor fuit Keplerus\protect\index{Namensregister}{\textso{Kepler} (Keplerus), Johannes 1571\textendash 1630},\edtext{}{\lemma{Keplerus,}\Bfootnote{\textsc{J. B. Morin}, \cite{00080}\textit{Longitudinum terrestrium scientia}, S.~265. }} sed qui perfectionem ejus ipse nondum norat. Longomontanus\protect\index{Namensregister}{\textso{Longomontanus,} Christen Sørensen 1562\textendash 1647} eam initio spreverat, sed a Morino\protect\index{Namensregister}{\textso{Morin} (Morinus), Jean-Baptiste 1583\textendash 1656} edoctus, amplexus est. Morinus\protect\index{Namensregister}{\textso{Morin} (Morinus), Jean-Baptiste 1583\textendash 1656} demonstravit quicquid propemodum a coelo in hoc negotio expectari potest. \edtext{Sua}{\lemma{potest.}\Afootnote{ \textit{ (1) }\ Quae \textit{ (2) }\ Sua \textit{ L}}} de parallaxibus\protect\index{Sachverzeichnis}{parallaxis} Morinus\protect\index{Namensregister}{\textso{Morin} (Morinus), Jean-Baptiste 1583\textendash 1656} diu secreta habuit, cum caetera edidisset, tantum amicorum et inter eos Hortensii\protect\index{Namensregister}{\textso{Hortensius,} Martinus 1605\textendash 1639} hortatu conpulsus ea quoque, atque ita suum opus Astronomicum integre publicavit. Herigonus\protect\index{Namensregister}{\textso{H\'{e}rigone} (Herigonus), Pierre 1580\textendash 1643} invenerat quiddam utile ad \rightmoon\ \edtext{in}{\lemma{\rightmoon}\Afootnote{ \textit{ (1) }\ intra \textit{ (2) }\ in \textit{ L}}} meridiano\protect\index{Sachverzeichnis}{meridianus}, invenit Morinus\protect\index{Namensregister}{\textso{Morin} (Morinus), Jean-Baptiste 1583\textendash 1656} methodum idem praestandi extra meridianum\protect\index{Sachverzeichnis}{meridianus}, sed ne famae sui prioris inventi detraheret, siluit, interea \edtext{[Michael Florent]}{\Afootnote{Carolus\textit{\ L \"{a}ndert Hrsg.}}} van Langren\protect\index{Namensregister}{\textso{Langren,} Michael Florent van ca. 1598\textendash 1675} Regis Hispaniae\protect\index{Namensregister}{\textso{Spanien: Philipp IV.}, K\"{o}nig von Spanien 1621\textendash 1665} Geogra$\upphi$us libro Hispanico, Verdat \edtext{etc.}{\lemma{etc.}\Bfootnote{\textsc{M. F. van Langren, }\cite{00286}\textit{La verdadera longitud por mar y tierra}, Br\"{u}ssel 1644.}} idem dedit. In eo careri potest parallaxibus\protect\index{Sachverzeichnis}{parallaxis}\footnote{\textit{Interlinear \"{u}ber} parallaxibus: \Denarius} puto etc. lunaribus, sed non restitutione Tabularum. Methodus quam praescripsit Morinus\protect\index{Namensregister}{\textso{Morin} (Morinus), Jean-Baptiste 1583\textendash 1656}, Astronomiam, si quando velit aliquis princeps a fundamentis restituendi non est contemnenda. Patrem De Fe\protect\index{Namensregister}{\textso{Duliris} (De Fe), L\'{e}onard SJ 1588\textendash 1656} dicit suum esse plagiarium, ne intellexisse quidem principia.\edtext{}{\lemma{principia.}\Bfootnote{\textsc{J. B. Morin, }\cite{00137}\textit{La science des longitudes}, Paris 1647, S. 4.}}\pend \pstart Morinus\protect\index{Namensregister}{\textso{Morin} (Morinus), Jean-Baptiste 1583\textendash 1656} excelluit in usu Trigonometrico, ejusque rei ope invenit et observavit consequentias, quas nemo putasset ex datis duci posse. Laudat Kepleri\protect\index{Namensregister}{\textso{Kepler} (Keplerus), Johannes 1571\textendash 1630} conceptum; quem nobilem vocat, de Ellipsibus. Bullialdus\protect\index{Namensregister}{\textso{Boulliau} (Bullialdus), Ismael 1605\textendash 1694} demonstare volebat motum terrae, in \textit{Astronomia $\phi$ilolaica}\edtext{}{\lemma{in}\Bfootnote{\textsc{I. Boulliau, }\cite{00013}\textit{Astronomia philolaica}, Paris 1645, 2. Buch. }},\footnote{\textit{Interlinear \"{u}ber Astronomia}: \Denarius} at Morinus\protect\index{Namensregister}{\textso{Morin} (Morinus), Jean-Baptiste 1583\textendash 1656} ei respondit in \edtext{\textso{Tychone Brahaeo,}}{\lemma{Brahaeo,}\Bfootnote{\textsc{J. B. Morin, }\cite{00136}\textit{Tycho Brahaeus}, Paris 1642.}} ut vocat librum suum et videtur ipse Bullialdus\protect\index{Namensregister}{\textso{Boulliau} (Bullialdus), Ismael 1605\textendash 1694} postea non instituisse illi ratiocinio, Bullialdum\protect\index{Namensregister}{\textso{Boulliau} (Bullialdus), Ismael 1605\textendash 1694} ait sibi vindicasse aequationem temporis Morini\protect\index{Namensregister}{\textso{Morin} (Morinus), Jean-Baptiste 1583\textendash 1656}. Compertum esse, ait Morinus\protect\index{Namensregister}{\textso{Morin} (Morinus), Jean-Baptiste 1583\textendash 1656}, observationes nobilis cujusdam, de perpendiculo ex filo \edtext{cannabino}{\lemma{}\Afootnote{cannabino \textit{ erg.} \textit{ L}}} suspenso compertas fuisse futiles, et non comparuisse, cum filum esset solidum, argenteum etc.\pend 