[145 r\textsuperscript{o}] un Vase, comme dans un Recipient de verre, elle ne soutient pas en verit\'{e}, l'effort de toute la masse, mais elle est contrainte par quelqu'autre chose, dont la resistence vaut bien l'effort de toute la masse, s\c{c}avoir par la fermet\'{e} du Vase continent. Quoyque cette fermet\'{e} soit plustost une resistence passive, qu'une force active; \edtext{puisqu'elle ne cherche pas de}{\lemma{active;}\Afootnote{ \textit{ (1) }\ ne cherchant pas \textit{ (2) }\ puisqu'elle ne cherche pas de \textit{ L}}} dilater ses limites, estant contante de se maintenir, et même de \edtext{reflechir}{\lemma{de}\Afootnote{ \textit{ (1) }\ repousser \textit{ (2) }\ reflechir \textit{ L}}} par \textso{un ressort born\'{e}} les corps, qui la choquent; au lieu que le ressort d'une petite portion d'air, est \textso{sans bornes,} ou au moins sans autres que celles de toute l'atmosphere\protect\index{Sachverzeichnis}{atmosph\`{e}re}. On voit par l\`{a} que la fermet\'{e} ou tenacit\'{e} des corps\protect\index{Sachverzeichnis}{corps!sensible} sensibles, comme du verre, ou même, \edtext{de cette pellicule si mince d'une}{\lemma{même,}\Afootnote{ \textit{ (1) }\ d'une petite \textit{ (2) }\ de [...] d'une \textit{ L}}} bulle d'air n\'{e}e sur la superficie de l'eau est capable de soutenir l'effort de toute la masse, \edtext{ne pouuant pas prendre}{\lemma{masse,}\Afootnote{ \textit{ (1) }\ parce qu'il est compens\'{e} \textit{ (2) }\ ne pouuant pas prendre \textit{ L}}} ais\'{e}ment un plus petit volume: de sorte que le Ressort de la masse en la dissipant, ne gagneroit rien, ce qui fait, qu'il ne la dissipe pas, la nature ne faisant rien en vain.\pend
 \pstart Il faut pourtant se garder icy d'une paralogisme. Car une bulle donn\'{e}e quoyqu'elle soûtienne l'effort \edtext{de tout l'air}{\lemma{l'effort}\Afootnote{ \textit{ (1) }\ de toute la masse \textit{ (2) }\ de tout l'air \textit{ L}}} n'a pas pourtant des forces \'{e}gales \`{a} celles du tout, car une autre bulle \'{e}gale en quantit\'{e} et \edtext{degrez}{\lemma{}\Afootnote{degrez  \textbar\ \`{a} elle \textit{ gestr.}\ \textbar\ , est \textit{ L}}}, est aussi \'{e}gale en forces \`{a} la donn\'{e}e: la donn\'{e}e donc estant \'{e}gal\'{e}e d'une partie du tout, ne peut pas \'{e}galer le tout, en forces. On demandera donc comment la Bulle donn\'{e}e peut soûtenir le choc avec des forces in\'{e}galles? La raison est, parce que la Masse \edtext{entiere ne se sert pas}{\lemma{entiere}\Afootnote{ \textit{ (1) }\ n'employe p \textit{ (2) }\ ne se sert pas \textit{ L}}} de toutes ses forces, quoyqu'elle employe toutes ses parties. Car chaque partie de la masse, employe seulement une partie de ces forces, s\c{c}avoir autant de forces, \edtext{\`{a} proportion}{\lemma{}\Afootnote{\`{a} proportion \textit{ erg.} \textit{ L}}} qu'elle gagneroit de place si l'on pourroit chasser ou contraindre dans un point, la bulle donn\'{e}e; employant le reste de ses forces contre le reste de la masse. Donc les forces employ\'{e}es \edtext{contre la bulle}{\lemma{}\Afootnote{contre  \textit{ (1) }\ les non employ\'{e}es \textit{ (2) }\ la bulle \textit{ erg.} \textit{ L}}} aux non \edtext{employ\'{e}es, sont}{\lemma{}\Afootnote{employ\'{e}es,  \textbar\ ou employ\'{e}es \textit{ gestr.}\ \textbar\ sont \textit{ L}}} comme la bulle donn\'{e}e, \`{a} toute la masse; ou comme ce que la bulle gagnante gagneroit sur la donn\'{e}e, \`{a} ce qu'elle gagneroit sur toute la masse. Toutes ses forces donc de toutes les parties ramass\'{e}es en un ne sont plus que les forces de la bulle donn\'{e}e. \edtext{Et pour la proposition mise en}{\lemma{donn\'{e}e.}\Afootnote{ \textit{ (1) }\ Et par consequent il y avoit un paralogisme en \textit{ (2) }\ Et [...] en \textit{ L}}} avant, qu'une petite bulle puisse soutenir un grand poids\protect\index{Sachverzeichnis}{poids} est fausse. Car le poids\protect\index{Sachverzeichnis}{poids} la presse dans un plus petit espace, quand l'air ambient ne pouuoit pas. Pour le montrer plus