[85 v\textsuperscript{o}] ita ut omnes paralleli radii intra altitudinem perpendicularis \rule[-9mm]{0mm}{9mm}\\
%\begin{edarrayc}
%&^{\rotatebox{180}{$\propto$}}\displaystyle\frac{3}{5}&\\
%&^{\rotatebox{180}{$\propto$}}\displaystyle\frac{5}{13}&\\
%&^{\rotatebox{180}{$\propto$}}\displaystyle\frac{7}{25}&\\
%\edbeforetab{x posit}{}
%&^{\rotatebox{180}{$\propto$}}\displaystyle\frac{9}{41}&\\
%&^{\rotatebox{180}{$\propto$}}\displaystyle\frac{31}{481}&\\
%&^{\rotatebox{180}{$\propto$}}\displaystyle\frac{49}{1201}&
%\edatleft{\{}{5\baselineskip}
%\edatright[\textrm{tendant ad diametrum intra longitudinem}]{\}}{5\baselineskip}
%\end{edarrayc}
 \renewcommand{\arraystretch}{2.3}
\hspace{-8mm}\parbox{1cm}{\textit{x} \edtext{[posita]}{\lemma{posit}\Afootnote{\textit{\ L \"{a}ndert Hrsg. }}}}
$\left\{
\begin{array}{r}
\hspace{-5mm}^{\rotatebox{180}{$\propto$}}\displaystyle\frac{3}{5}\\
\hspace{-5mm}^{\rotatebox{180}{$\propto$}}\displaystyle\frac{5}{13}\\
\hspace{-5mm}^{\rotatebox{180}{$\propto$}}\displaystyle\frac{7}{25}\\
\hspace{-5mm}^{\rotatebox{180}{$\propto$}}\displaystyle\frac{9}{41}\\
\hspace{-3mm}^{\rotatebox{180}{$\propto$}}\displaystyle\frac{31}{481}\\
\hspace{-2mm}^{\rotatebox{180}{$\propto$}}\displaystyle\frac{49}{1201}
\end{array}\right\}$\parbox{1.5cm}{tendant ad dia-\\metrum intra longitudinem}$\left\{\begin{array}{rll}
\displaystyle\frac{429}{231}&\hspace{-3mm}-\displaystyle\frac{1}{5,}\displaystyle\frac{873}{231}&\hspace{-5mm}^{\rotatebox{180}{$\propto$}}\displaystyle\frac{}{5,}\displaystyle\frac{272}{231}\\
\displaystyle\frac{429}{231}&\hspace{-3mm}-\displaystyle\frac{5}{13,}\displaystyle\frac{300}{231}&\hspace{-5mm}^{\rotatebox{180}{$\propto$}}\displaystyle\frac{}{13,}\displaystyle\frac{277}{231}\\
\displaystyle\frac{429}{231}&\hspace{-3mm}-\displaystyle\frac{10}{25,}\displaystyle\frac{447}{231}&\hspace{-5mm}^{\rotatebox{180}{$\propto$}}\displaystyle\frac{}{25,}\displaystyle\frac{278}{231}\\
\displaystyle\frac{429}{231}&\hspace{-3mm}-\displaystyle\frac{17}{41,}\displaystyle\frac{310}{231}&\hspace{-5mm}^{\rotatebox{180}{$\propto$}}\displaystyle\frac{}{41,}\displaystyle\frac{279}{231}\\
\displaystyle\frac{429}{231}&\hspace{-3mm}-\displaystyle\frac{206}{481,}\displaystyle\frac{070}{231}&\hspace{-5mm}^{\rotatebox{180}{$\propto$}}\displaystyle\frac{}{481,}\displaystyle\frac{279}{231}\\
\displaystyle\frac{429}{231}&\hspace{-3mm}-\displaystyle\frac{514}{1201,}\displaystyle\frac{950}{231}&\hspace{-3mm}^{\rotatebox{180}{$\propto$}}\displaystyle\frac{}{1201,}\displaystyle\frac{279}{231}\\
\end{array}\right\}$\parbox{1.5cm}{quae longitudo minor est}$
\left\{\begin{array}{c}
\displaystyle\frac{1}{4}\\
\displaystyle\frac{1}{10}\\
\displaystyle\frac{1}{20}\\
\displaystyle\frac{1}{33}\\
\displaystyle\frac{1}{398}\\
\displaystyle\frac{1}{994}
\end{array}\right\}
$\advanceline{5}.
\rule[0mm]{0mm}{10mm}Ex quibus patet quanto \textit{x} sive \textit{BF} minor est, tanto etiam  punctum \textit{I} longius distare ab \textit{N}, hoc est quanto radius aliquis magis distat ab axe aut vertice \textit{D}, tanto etiam remotius a vertice axem secare.\pend \pstart Deinde si concipiatur \textit{IDB}, circa axem \textit{DI} rotatam, figuram vitri describere, facile etiam inveniri potest magnitudo minimi plani ad angulos rectos ad \textit{DK} erecti, in quod omnes radii qui \textit{DI} sunt paralleli, atque contenti intra Cylindrum illum ab \textit{ABF}, circa axem \textit{DFN} rotata, descriptum, incidunt; (quod planum postea vocabitur focus:) sed cum non necesse  habeamus scire minimi hujus plani magnitudinem ut ad  propositum perveniamus, satis erit si tantum alterius cujusdam, quod longe quam hoc majus est, atque in quo 