\pend \pstart [p.~155] [...] in maioribus tamen angulis inclinationis\protect\index{Sachverzeichnis}{angulus!inclinationis}, falsum esse constat; in his enim angulus refractionis\protect\index{Sachverzeichnis}{angulus!refractionis} maior est subtriplo anguli inclinationis\protect\index{Sachverzeichnis}{angulus!inclinationis}; quod mihi aliisque, ex luculentis experimentis compertum est.\footnote{\textit{Leibniz unterstreicht}: in maioribus [...] compertum est.} [...] Sed his omissis, quae physica sunt, sit quadrans ellipticus, AR,\footnote{\textit{Am Rand mit Tinte}: fig. 127.} centro F, foci CG; [...].\selectlanguage{latin}