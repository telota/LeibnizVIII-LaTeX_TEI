\pstart In einem Brief vom 13. Mai 1676 an H. Bond\protect\index{Namensregister}{\textso{Bond,} Henry 1600?\textendash 1678} (\textit{LSB} III, 1 N. 80) teilt Leibniz mit, er habe jetzt auch den von Henricus Philippus\protect\index{Namensregister}{\textso{Philippes} (Philippus), Henry ?\textendash 1677} herausgegebenen \cite{00002}\textit{Sea-mans Kalender} gelesen, in dem Bonds\protect\index{Namensregister}{\textso{Bond,} Henry 1600?\textendash 1678} Entdeckung der Variation der Magnetlinien referiert w\"{u}rde. Genau diesen Sachverhalt er\"{o}rtert Leibniz u. a. in den drei Texten zur Hydrographie. Insbesondere versucht er, diesen Befund mit der Idee eines um zwei Ebenen drehbar gelagerten Kompasses beim Navigieren zu ber\"{u}cksichtigen. Es ist anzunehmen, dass Leibniz durch die Lekt\"{u}re von \cite{00002}\textit{Sea-mans Kalender} zu den nachstehenden \"{U}berlegungen angeregt wurde. Daf\"{u}r spricht auch, dass sich das Wasserzeichen des f\"{u}r die Aufzeichnungen verwendeten Papiers mit dem von \textit{LSB} VII, 3 N. 72 deckt. Wir orientieren uns f\"{u}r die Datierung an \textit{LSB} VII, 3 N. 72 und gehen von einer Entstehung der Texte in der 2. H\"{a}lfte des Jahres 1676 aus.\pend