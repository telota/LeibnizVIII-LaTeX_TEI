\pend \pstart [p.~154] [...] alioquin plus aequo distrahuntur radij\protect\index{Sachverzeichnis}{radius}. 4. si cavum\footnote{\textit{Leibniz unterstreicht mit Tinte}: cavum} loco 3. lentis\protect\index{Sachverzeichnis}{lens}\footnote{\textit{Doppelt mit Tinte unterstrichen}: 3. lentis} apponatur, contrahitur quidem campus, sed obiectum paulo maius apparet. [...] secundum est, lumen, diffundi facilius per corpus densum diaphanum, quam per rarum, puta per vitrum, quam per aera. Tertium est, sinus angulorum refractorum\protect\index{Sachverzeichnis}{angulus!refractionis}, esse vt sinus angulorum reciprocorum inclinationis;\footnote{\textit{Leibniz unterstreicht mit Tinte}: lumen [...] inclinationis} [...].