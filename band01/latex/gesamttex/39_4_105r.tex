      
               
                \begin{ledgroupsized}[r]{120mm}
                \footnotesize 
                \pstart                
                \noindent\textbf{\"{U}berlieferung:}   
                \pend
                \end{ledgroupsized}
            
              
                            \begin{ledgroupsized}[r]{114mm}
                            \footnotesize 
                            \pstart \parindent -6mm
                            \makebox[6mm][l]{\textit{L}}Konzept: LH XXXVII 3 Bl. 105\textendash106. 1 Bogen 2\textsuperscript{o}. 4 S. zweispaltig. Linke Spalte beschrieben, Bl. 106 v\textsuperscript{o} auch 3/4 der rechten Spalte beschrieben.\\
 Cc 2, Nr. 479 \pend
                            \end{ledgroupsized}
                %\normalsize
                \vspace*{5mm}
                \begin{ledgroup}
                \footnotesize 
                \pstart
            \noindent\footnotesize{\textbf{Datierungsgr\"{u}nde}: Die Datierung erfolgt aufgrund des Wasserzeichens, das sich u. a. auf dem Texttr\"{a}ger der Exzerpte aus Otto von Guerickes \cite{00055}\textit{Experimenta nova} findet. Auch inhaltlich kn\"{u}pft Leibniz an seine Guericke-Lekt\"{u}re an. Das betrifft insbesondere seine \"{U}berlegungen zum Einfluss des Windes auf den Luftdruck. Wir gehen daher von einem Entstehungszeitraum des St\"{u}ckes aus, der sich mit dem der Guericke-Exzerpte deckt.}
                \pend
                \end{ledgroup}
            
                \vspace*{8mm}
                \pstart 
                \normalsize
            [105 r\textsuperscript{o}] \edtext{ Ex quo primum Vacui quod vocant  Experimentum a Torricellio\edtext{}{\lemma{Torricellio}\Bfootnote{\textsc{E. Torricelli, }\cite{00107}\textit{Brief an Ricci vom 11. Juni 1644}. \cite{00032} Florenz 1663, S.~20f. (\textit{TO} III, S.~186\textendash188).}} detectum per orbem  increbuit observatum est duobus potissimis modis Hydrargyri altitudinem variare}{\lemma{}\Afootnote{ \textit{ (1) }\ Constat inter omnes, inde a detecto per Torricellium\protect\index{Namensregister}{\textso{Torricelli} (Torricellius), Evangelista 1608\textendash 1647|textit} primo Vacui\protect\index{Sachverzeichnis}{vacuum!Torricellianum|textit} quod vocant, experimento, ex  quo observatum est, non   \textbar\ pro \textit{ erg.}\ \textbar\  altitudine locorum tantum,  sed et   \textbar\ pro aeris \textit{ erg.}\ \textbar\  tempestate Hydrargyri\protect\index{Sachverzeichnis}{hydrargyrus|textit} elevationem variare,  quaesitam esse \textit{ (2) }\  Primus Torrice\protect\index{Namensregister}{\textso{Torricelli} (Torricellius), Evangelista 1608\textendash 1647|textit} \textit{ (3) }\  Primus Galilaeus\protect\index{Namensregister}{\textso{Galilei} (Galilaeus, Galileus), Galileo 1564\textendash 1642|textit}  rem a Mechanicis tantum observatam, at in philosophorum scholis tunc ignotam  in literas retulit, liquores scilicet \textit{ (4) }\ Ex [...] increbuit  \textbar\ a Pascalio, Canuto,  aliisque \textit{ gestr.}\ \textbar\ observatum [...] variare \textit{ L}}}, tum ob variam  locorum altitudinem aut depressionem, (manifestum enim erat \edtext{minorem  esse aeris columnam}{\lemma{erat}\Afootnote{ \textit{ (1) }\ minus aeris   \textbar\ pondus \textit{ erg.}\ \textbar\  sustineri \textit{ (2) }\ minorem  esse aeris columnam \textit{ L}}}, quae ex montis vertice,  quam quae ex vallis profunditate assurgens, a Mercurio\protect\index{Sachverzeichnis}{mercurius} sustineretur)\rule[-2cm]{0cm}{0.5cm} tum vero \edtext{ob\edlabel{diff105r1}}{\lemma{ob}\xxref{diff105r1}{diff105r3}\Afootnote{ \textit{ (1) }\ differentem  ipsius \textit{ (2) }\ aeris [...] differentem \textit{ L}}} aeris ponderositatem, prout ille oneraretur  vaporibus, aut sustineretur ventis, aut alias  temporum mutationes subiret, differentem.\edtext{\edlabel{diff105r3}}{\lemma{differentem.}\xxref{diff105r1}{diff105r2}\Afootnote{ \textit{ (1) }\ Sed  frustra quanquam in eam rem summo studio  sit inquisitum, regula tamen in publico comparuit hactenus nulla. Et quanquam  variis artibus qua ad conjiciendas \textit{ (2) }\ Inde [...] Barometro,  \textbar\ (sic [...] coepit;) \textit{ erg.}\ \textbar\ in [...] posset,  \textbar\ futurus \textit{ erg.}\ \textbar\  Aeris status. \textit{ L}}}
            \pend 
            \pstart Inde in magnam omnes spem erecti  sunt inveniendae regulae, per quam ex Barometro, (sic enim ab eo tempore appellari Tubus ille Torricellianus coepit;)  in aliquot saltem dies praedici posset, futurus Aeris status.
            \edtext{\edlabel{diff105r2}Sed quanquam Barometrum illud Torricelliano experimento  nixum, ad eam nunc perfectionem et ut sic dicam, sensibilitatem adductum sit,  ut si ex summo domus in imum descendas, appareat ascensus Mercurii, quanquam etiam a  multis annis exacta instituta sint experimenta  diariaque etiam condita, quibus quotidianus aeris et Tubi Torricelliani status conferrentur, nihil tamen  hactenus in publico apparuit,}{\lemma{status.}\Afootnote{ \textit{ (1) }\ Sed hactenus nihil tale in publico apparuit, et  \textit{(a)}\ quae \textit{(b)}\ quanquam \textit{ (2) }\  Sed quanquam  \textit{(a)}\ refertae sint artes \textit{(b)}\ Barometrum [...] perfectionem  \textbar\ et ut sic dicam, sensibilitatem \textit{ erg.}\ \textbar\ adductum [...] descendas, \textit{(aa)}\ sentiri  possit \textit{(bb)}\ appareat [...] apparuit, \textit{ L}}} quod satisfaceret  tantae inquisitioni.
            \pend 
            \pstart  Quae res mihi \edtext{suspicionem fecit}{\lemma{mihi}\Afootnote{ \textit{ (1) }\ persuadere coepit \textit{ (2) }\ suspicionem fecit \textit{ L}}}, Barometrum\protect\index{Sachverzeichnis}{barometrum}  quale hactenus adhibuimus non sufficere indicio tempestatum, \edtext{quid enim quod modo inesset, ex  eo non eruerent viri}{\lemma{tempestatum,}\Afootnote{ \textit{ (1) }\ quid enim ex eo non eruissent vi \textit{ (2) }\  nam si quid tale \textit{ (3) }\ quid enim   \textbar\ quod modo inesset, \textit{ erg.}\ \textbar\  ex  eo non eruerent viri \textit{ L}}} tanti, qui in ejus mutationibus  ad regulam revocandis laboravere.
            \pend 
            