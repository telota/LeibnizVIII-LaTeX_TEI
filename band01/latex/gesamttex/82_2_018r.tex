\pstart \begin{center}[18 r\textsuperscript{o}] Remedia:\end{center} 
\pend 
\vspace{1.0ex}
\pstart\noindent\hangindent=10mm (1) \edtext{\textso{fortificatio acus magneticae }\protect\index{Sachverzeichnis}{acus!magnetica}}{\lemma{(1)}\Afootnote{ \textit{ (1) }\ acus\protect\index{Sachverzeichnis}{acus!magnetica|textit} potest ita forti \textit{ (2) }\ \textso{fortificatio acus magneticae} \textit{ L}}}ut vim acquirat decuplo imo centuplo majorem. Unde sequitur pyxidem\protect\index{Sachverzeichnis}{pyxis} posse fieri satis magnam, satisque accurate subdivisam. Satis item virium in acu\protect\index{Sachverzeichnis}{acus!magnetica} \edtext{fore}{\lemma{acu}\Afootnote{ \textit{ (1) }\ esse \textit{ (2) }\ fore \textit{ L}}} ad ductus in mappa describendos. \edtext{Magni ad rem nauticam momenti haec fortificandarum acuum inventio est.}{\lemma{}\Afootnote{Magni [...] est. \textit{ erg.} \textit{ L}}} 
\pend
\pstart\noindent\hangindent=10mm (2) \textso{Ductus} possunt \edtext{fieri subtiles}{\lemma{possunt}\Afootnote{ \textit{ (1) }\ subtilissimo \textit{ (2) }\ fieri subtiles \textit{ L}}}, levique attactu 
\pend
\pstart\noindent\hangindent=10mm (3) \edtext{acus, utcunque jactatione perturbata sit, restituit se ipsam in lineam flexumque}{\lemma{(3)}\Afootnote{ \textit{ (1) }\ Omnis pyxi\protect\index{Sachverzeichnis}{pyxis|textit} \textit{ (2) }\ acus \textit{ L}}} priorem veri ergo flexus emergent semper ex \edlabel{perturbatisstart}perturbatis.
\pend
\pstart\noindent\hangindent=10mm \edtext{(4) Quod\edlabel{perturbatisend}}{{\xxref{perturbatisstart}{perturbatisend}}\lemma{perturbatis.}\Afootnote{ \textit{ (1) }\ In quo acus\protect\index{Sachverzeichnis}{acus!magnetica|textit} \textit{ (2) }\  \textbar\ Pendula ea in re acubus sunt inferiora, perturbatio enim in iis semel admissa postea non compensatur. \textit{ gestr.}\ \textbar\ (4) \textit{ L}}} declinationes\protect\index{Sachverzeichnis}{declinatio} attinet, etsi supponeremus nullum hic ex ipsa pyxide\protect\index{Sachverzeichnis}{pyxis} remedium esse, constat tamen earum observationem pene quotidianam non esse difficilem, et in longissimis itineribus Nautas quosdam acus\protect\index{Sachverzeichnis}{acus!magnetica} declinationem\protect\index{Sachverzeichnis}{declinatio} singulis propemodum diebus annotasse. Quare nihil aliud eo casu ad rei Hydrographicae perfectionem restabit, quam ut \edtext{declinatio diligenter \edlabel{obistart}observetur.}{\lemma{ut}\Afootnote{ \textit{ (1) }\ diligenter observentur declinationes\protect\index{Sachverzeichnis}{declinatio|textit} \textit{ (2) }\ linea \textit{ (3) }\ declinatio diligenter observetur \textit{ L}}}\edtext{}{{\xxref{obistart}{obiend}}
\lemma{observetur.}\Afootnote{\textit{ (1) }\ Hae \textit{ (2) }\ Et sequitur [...]\textso{ Longitudines}\protect\index{Sachverzeichnis}{longitudo}\textso{ dentur.} \textit{ erg.} \textit{ L}}}
\pend
\pstart\noindent\hangindent=10mm\hspace{10mm}Et sequitur ergo ex hac machina (sine ulla constituta declinationum\protect\index{Sachverzeichnis}{declinatio} Theoria universali) id quod hactenus irrito conatu quaesitum est, ut \edlabel{solisstart}\textso{solis}\protect\index{Sachverzeichnis}{sol}\textso{ observatis declinationibus}\protect\index{Sachverzeichnis}{declinatio}\textso{ Longitudines}\protect\index{Sachverzeichnis}{longitudo}\edlabel{obiend} \edtext{\textso{ dentur.}\edlabel{solisend}}{{\xxref{solisstart}{solisend}}\lemma{\textso{solis }[...]\textso{ dentur:}}\Afootnote{\textit{doppelt unterstrichen}}}
\newline Constat \edtext{plurimos eorum}{\lemma{Constat}\Afootnote{ \textit{ (1) }\ eos \textit{ (2) }\ plurimos eorum \textit{ L}}} qui nobis longitudines\protect\index{Sachverzeichnis}{longitudo} promisere, declinationes\protect\index{Sachverzeichnis}{declinatio} observari praesupposuisse. 
\pend 
\pstart\noindent\hangindent=10mm (5) Accedit quod declinatio\protect\index{Sachverzeichnis}{declinatio} mutatur non per saltus, sed paulatim, potest ergo continue error machinae emendari; et quamvis uno alterove die non possit observari declinatio\protect\index{Sachverzeichnis}{declinatio}, interea tamen, sic satis aestimari ex praecedentibus potest, errore postea ex sequentibus \edtext{observationibus}{\lemma{}\Afootnote{\hspace{4mm}observationibus \textbar\ exacte \textit{ gestr.}\ \textbar\ emendato. \textit{ L}}} emendato.
 \pend
\pstart\noindent\hangindent=10mm (6) \edtext{ Et potest ratio institui, ut machina continue emendet se ipsam, nullo}{\lemma{(6)}\Afootnote{ \textit{ (1) }\ Emendationis hujus continuae facilis quaedam ratio haberi potest, ut pondus quoddam aut elaterium\protect\index{Sachverzeichnis}{elater|textit} rotis\protect\index{Sachverzeichnis}{rota|textit} recte proportionatis applicatum pyxidem\protect\index{Sachverzeichnis}{pyxis|textit} tantundem circiter retroagat, quantum acus\protect\index{Sachverzeichnis}{acus!magnetica|textit} interim ea die declinando processit ita eodem res redibit, quasi nulla esset declinatio\protect\index{Sachverzeichnis}{declinatio|textit} \textit{ (2) }\ Et [...] nullo \textit{ L}}} [\textit{Satz bricht ab}]
\pend 
\pstart\noindent\hangindent=10mm (7) \edtext{Est et alia Emendatio}{\lemma{(7)}\Afootnote{ \textit{ (1) }\ Sunt et aliae Emendationes auxiliatrices \textit{ (2) }\ Est et alia Emendatio \textit{ L}}}. Nam si acus\protect\index{Sachverzeichnis}{acus!magnetica} et Navis\protect\index{Sachverzeichnis}{navis} eodem declinant, v. g. utraque a Septentrione in Orientem, potest haberi ratio determinandi in ipsa pyxide\protect\index{Sachverzeichnis}{pyxis} quis flexus sit a navi\protect\index{Sachverzeichnis}{navis}, quis ab acu\protect\index{Sachverzeichnis}{acus!magnetica}. 
\pend
\pstart\noindent\hangindent=10mm (8) Cum item ope \textso{pyxidis}\protect\index{Sachverzeichnis}{pyxis}\textso{ inclinatoriae} determinari semper possit Latitudo \edtext{exacte}{\lemma{}\Afootnote{exacte \textit{ erg.} \textit{ L}}} et Machina itidem Latitudinem\protect\index{Sachverzeichnis}{latitudo} determinet, \edtext{qualitercunque}{\lemma{}\Afootnote{qualitercunque \textit{ erg.} \textit{ L}}} collatio pyxidis\protect\index{Sachverzeichnis}{pyxis} inclinatoriae cum Machina Hydrographica\protect\index{Sachverzeichnis}{machina!hydrographica}, dabit nobis \edtext{praecise}{\lemma{}\Afootnote{praecise \textit{ erg.} \textit{ L}}}
[18 v\textsuperscript{o}] quantum a latitudine\protect\index{Sachverzeichnis}{latitudo} aberraverimus. Hinc autem poterit calculo satis subtili supputari quantum et in Longitudine\protect\index{Sachverzeichnis}{longitudo} Machina exerraverit. Constat enim de effectu quoad Latitudinem\protect\index{Sachverzeichnis}{latitudo}, constat item de proportione mutatae longitudinis\protect\index{Sachverzeichnis}{longitudo} ad mutatam latitudinem\protect\index{Sachverzeichnis}{latitudo}. Hinc supputabitur ex dato errore latitudinis\protect\index{Sachverzeichnis}{latitudo} error longitudinis\protect\index{Sachverzeichnis}{longitudo}. Semper enim latitudo\protect\index{Sachverzeichnis}{latitudo} et longitudo\protect\index{Sachverzeichnis}{longitudo} sunt sibi \edtext{complemento}{\lemma{sibi}\Afootnote{ \textit{ (1) }\ proportionales, id est \textit{ (2) }\ complemento \textit{ L }}} ad angulum rectum, ac proinde quanto minor est Latitudo\protect\index{Sachverzeichnis}{latitudo} tanto \edtext{[major]}{\lemma{}\Afootnote{minor \textit{\ L \"{a}ndert Hrsg.}}} est longitudo\protect\index{Sachverzeichnis}{longitudo}, et contra. Haec machinae Hydrographicae\protect\index{Sachverzeichnis}{machina!hydrographica} rectificatio est universalis, a coelo et sole\protect\index{Sachverzeichnis}{sol} independens, semper in potestate. Et si inclinationis\protect\index{Sachverzeichnis}{inclinatio} mutatio continue observabitur calculus rectificandarum quoque longitudinum\protect\index{Sachverzeichnis}{longitudo} ita exactus erit, ut vix gradu aberrari posse putem.\pend