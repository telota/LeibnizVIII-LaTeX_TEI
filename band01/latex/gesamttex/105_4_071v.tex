[71 v\textsuperscript{o}] Nam ex hac observatione praecise colligetur \textso{aetas }\textso{Lunae,}\protect\index{Sachverzeichnis}{luna} quod ita ostendo.\pend \pstart Constat Lunae\protect\index{Sachverzeichnis}{luna} revolutionem a primi mobilis revolutione 13. gradibus relinqui, ac proinde Lunam\protect\index{Sachverzeichnis}{luna} continue per totum Mensem, situm in sphaera fixarum\protect\index{Sachverzeichnis}{sphaera!fixarum} mutare, ergo dato loco Lunae\protect\index{Sachverzeichnis}{luna} in sphaera fixarum\protect\index{Sachverzeichnis}{sphaera!fixarum}, dabitur tempus mensis seu aetas Lunae\protect\index{Sachverzeichnis}{luna}.\pend \pstart Videamus tantum an mutatio illa loci  seu situs \edtext{Lunae\protect\index{Sachverzeichnis}{luna}}{\lemma{}\Afootnote{Lunae\protect\index{Sachverzeichnis}{luna} \textit{ erg.} \textit{ L}}}, in respectu fixarum\protect\index{Sachverzeichnis}{stella!fixa}, exiguo tempore sit sensibilis, ad tempus mensis satis accurate supputandum; quod ita inveniemus: cum Luna\protect\index{Sachverzeichnis}{luna} singulis 24. horis primi mobilis, 13. gradibus retardet, ergo una hora retardabit \edtext{seu a puncto sphaerae fixarum\protect\index{Sachverzeichnis}{sphaera!fixarum} sub quo antea fuerat, praevenietur }{\lemma{}\Afootnote{BITTE UEBERPRUEFEN!!! seu a puncto sphaerae fixarum\protect\index{Sachverzeichnis}{sphaera!fixarum} sub quo antea fuerat, praevenietur  \textit{ erg.} \textit{ L}}}13/24\textsuperscript{tis} gradus, et minuto horae \edtext{differet}{\lemma{}\Afootnote{differet \textit{ erg.} \textit{ L}}} 13/1440\textsuperscript{mis} gradus id est amplius paulo, quam parte gradus centesima vigesima \edtext{(nam @@@ G R A F I K @@@)}{\lemma{}\Afootnote{(nam @@@ G R A F I K @@@) \textit{ erg.} \textit{ L}}}, seu paulo amplius dimidio  minuto primo gradus, quod \edtext{scilicet}{\lemma{quod}\Afootnote{ \textit{ (1) }\ scil. \textit{ (2) }\ scilicet \textit{ L}}} in circulo latitudinis\protect\index{Sachverzeichnis}{latitudo} seu Parallelo\protect\index{Sachverzeichnis}{circulus parallelus} in quo Luna\protect\index{Sachverzeichnis}{luna} versatur, est \edtext{aestimandum. Idque tanto}{\lemma{est}\Afootnote{ \textit{ (1) }\ aestimanda, \textit{ (2) }\ aestimandum, quod tanto \textit{ (3) }\ aestimandum. Idque tanto \textit{ L}}} exactius poterit \edtext{per observationes}{\lemma{poterit}\Afootnote{ \textit{ (1) }\ observationibus \textit{ (2) }\ per observationes \textit{ L}}} aestimari, quanto et ipsa \textso{Luna}\protect\index{Sachverzeichnis}{luna} \textso{aequatori}\protect\index{Sachverzeichnis}{aequator}\textso{ est proprior,} et nos propriores \textso{ipsi }\textso{Lunae.}\protect\index{Sachverzeichnis}{luna} Quanquam certe, quod ad primum casum attinet, \edtext{licet Luna\protect\index{Sachverzeichnis}{luna} ab aequatore\protect\index{Sachverzeichnis}{aequator} sit remotior}{\lemma{}\Afootnote{BITTE UEBERPRUEFEN!!! licet Luna\protect\index{Sachverzeichnis}{luna} ab aequatore\protect\index{Sachverzeichnis}{aequator} sit remotior \textit{ erg.} \textit{ L}}}, paralleli aequatoris\protect\index{Sachverzeichnis}{aequator}, seu circuli latitudinis\protect\index{Sachverzeichnis}{latitudo} intra aequatorem\protect\index{Sachverzeichnis}{aequator}, in quibus scilicet solis\protect\index{Sachverzeichnis}{sol} luna\protect\index{Sachverzeichnis}{luna} versari potest, non admodum inter se et ab aequatore\protect\index{Sachverzeichnis}{aequator} magnitudine differant; et quantum ad secundum, vicissim nos quanto magis a Luna\protect\index{Sachverzeichnis}{luna} versus Polum\protect\index{Sachverzeichnis}{polus} absumus, seu quanto major Poli elevatio\protect\index{Sachverzeichnis}{elevatio!poli} est, eo minus exactitudine  observationum indigeamus, quia error circa  longitudines in Parallelo\protect\index{Sachverzeichnis}{circulus parallelus} minore, seu ab aequatore \protect\index{Sachverzeichnis}{aequator} remotione minus est sensibilis.\pend \pstart Determinato ergo loco Lunae\protect\index{Sachverzeichnis}{luna} in sphaera fixarum\protect\index{Sachverzeichnis}{sphaera!fixarum}, quanta fieri potest exactitudine; patet per Ephemerides\protect\index{Sachverzeichnis}{ephemeris} ad institutum istud debite accomodatas, determinatum esse aetatem Lunae\protect\index{Sachverzeichnis}{luna}, seu tempus mensis\edtext{. Et}{\lemma{mensis}\Afootnote{ \textit{ (1) }\ , et \textit{ (2) }\ . Et \textit{ L}}} omnium 