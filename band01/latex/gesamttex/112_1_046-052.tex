\pstart Bei den folgenden St\"{u}cken handelt es sich um ein Textcorpus, das Leibniz nachtr\"{a}glich strukturiert hat. In der urspr\"{u}nglichen Fassung wurden die \"{U}berlegungen zum Problem der L\"{a}ngengradbestimmung sukzessive auf 16 Seiten im Folioformat niedergeschrieben. Diese hat Leibniz sp\"{a}ter auf Bl. 46 r\textsuperscript{o} mit dem Zusatz \textit{Intra finem anni 1668 et initium 1669} versehen, worauf unsere Datierung beruht. Ebenfalls nachtr\"{a}glich wurden die st\"{u}ckkonstituierenden \"{U}berschriften \textit{Longit. 1}, \textit{Longit. 2} usw. hinzugef\"{u}gt. Bl. 74 v\textsuperscript{o} enth\"{a}lt den Entwurf zu einem \textit{Instrumentum longitudinum}, auf den in \textit{Longit. 2} Bezug genommen wird. Wir ordnen die Beschreibung dieses Instruments zur L\"{a}ngengradbestimmung als eigenst\"{a}ndiges St\"{u}ck unmittelbar vor \textit{Longit. 2} ein.\pend