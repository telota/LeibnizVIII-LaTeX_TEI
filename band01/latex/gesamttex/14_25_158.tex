\pend \pstart [p.~158] [...] Alterum est, vt adhibita speculi\protect\index{Sachverzeichnis}{speculum} caui opera, colligantur radij\protect\index{Sachverzeichnis}{radius} ab obiecto profecti in dato foco\protect\index{Sachverzeichnis}{focus}, et iuxta hunc lens\protect\index{Sachverzeichnis}{lens} oculo\protect\index{Sachverzeichnis}{oculus} admoueatur; sed profecto idem incommodum obstat; multi enim radij\protect\index{Sachverzeichnis}{radius} ab aliis obiectis circumpositis profecti, et in memoratam lentem\protect\index{Sachverzeichnis}{lens} illapsi repercussorum a speculo\protect\index{Sachverzeichnis}{speculum} radiorum\protect\index{Sachverzeichnis}{radius} ordinem confundunt.\footnote{\textit{Am Rand angestrichen:} Alterum [...] confundunt.} 3. aliqui vitrum obiectivum\protect\index{Sachverzeichnis}{vitrum!objectivum}, obducto plumbo\protect\index{Sachverzeichnis}{plumbum}, ad reflexionem\protect\index{Sachverzeichnis}{reflexio} adhibent, sic enim longitudo tubi suppletur, non tamen confusio obiecti,\footnote{\textit{Leibniz unterstreicht:} 3. aliqui [...] confusio obiecti \\ \textit{Am Rand mit Tinte:} ita postea Neuton\protect\index{Namensregister}{\textso{Newton} (Neuton, Neutonus), Isaac 1642\textendash 1727}} licet eiusdem moles valde augeatur, quo scilicet speculum\protect\index{Sachverzeichnis}{speculum} sphaerae maioris est; [...].