\pend \pstart [p.~123] [...] igitur si obiectiui loco statuatur  hoc vitrum, telescopium\protect\index{Sachverzeichnis}{telescopium} erit 40. palmorum, ac proinde in eadem proportione crescet obiecti apparentis moles,  [p.~124] quod distincte quidem, minus tamen clarum apparebit,  in ea scilicet proportione, qua apertura lentis\protect\index{Sachverzeichnis}{lens} AC minor est apertura lentis\protect\index{Sachverzeichnis}{lens} descriptae radio 40. palmos longo,  vt perspicuum est.\footnote{\textit{Am Rand angestrichen}: quod distincte [...] perspicuum est.}\pend \newpage \pstart  [...] hinc si obiectum plus  aequo splendeat contrahenda est apertura, vt fit in  venere, ne basis proiecta confundatur; sic autem inutilis veneris et stellarum\protect\index{Sachverzeichnis}{stella} coma rescinditur,  seu tondetur, contracta scilicet, obiectiui apertura.\footnote{\textit{Am Rand doppelt angestrichen und unterstrichen}: et stellarum\protect\index{Sachverzeichnis}{stella} [...] apertura.}