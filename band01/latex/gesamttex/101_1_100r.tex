[100 r\textsuperscript{o}] conatus\protect\index{Sachverzeichnis}{conatus} corporis \edtext{\textit{E}}{\lemma{}\Afootnote{\textit{E} \textit{ erg.} \textit{ L}}} medio novo jam salvo conatu\protect\index{Sachverzeichnis}{conatus} priore immersi, ex conatu\protect\index{Sachverzeichnis}{conatus} in horizontali \textit{gt}, et perpendiculari \textit{gu} ac proinde quo tempore pervenisset corpus \textit{E} in \textit{h} si medium fuisset homogeneum eodem tempore \edtext{nunc}{\lemma{nunc}\Afootnote{\textit{ erg.} \textit{ L}}} perventurum esse in \textit{x} punctum intersectionis \edtext{parallelarum}{\lemma{intersectionis}\Afootnote{ \textit{ (1) }\ perpendicularium \textit{ (2) }\ parallelarum \textit{ L}}} ex horizontalis et perpendicularis extremis ductarum \textit{tx} et \textit{ux}. Hinc sequitur \edtext{si corpus incidens statim ab initio totum medio novo immergi cogitetur,}{\lemma{sequitur}\Afootnote{ \textit{ (1) }\ motum in medio densiore non refringi tantum, sed et retardari etiam \textit{ (2) }\ si corpus incidens statim \textit{(a)}\ medio novo totum \textit{(b)}\ ab [...] cogitetur, \textit{ L}}} atque ita progressus ejus ulterior in eo aestimetur, quo casu tantum resistentiae est contra lineam horizontalem quantum contra perpendicularem, nihilominus  \edtext{refractionem\protect\index{Sachverzeichnis}{refractio} fore}{\lemma{refractionem}\Afootnote{ \textit{ (1) }\ esse \textit{ (2) }\ fore \textit{ L}}} versus horizontalem seu a perpendiculari, quando conatus\protect\index{Sachverzeichnis}{conatus} horizontalis \textit{gq} est major conatu\protect\index{Sachverzeichnis}{conatus} perpendicularis \textit{gr}. Nam si idem \edtext{\textit{rs} detrahitur inaequalibus \textit{gq} et \textit{gr}}{\lemma{idem}\Afootnote{ \textit{ (1) }\ detrahitur inaequalibus, plus \textit{ (2) }\ \textit{rs} [...] \textit{gr} \textit{ L}}} proportione detrahitur minori. Idem enim minoris dimidium esse potest, quod \pend\pstart\noindent majoris non nisi tertia pars est. Sed videamus quid eventurum sit, si horizontalis sit minor perpendiculari, quod fit, quoties linea incidentiae \edtext{cadit supra}{\lemma{incidentiae}\Afootnote{ \textit{ (1) }\ non cadit infra \textit{ (2) }\ cadit supra \textit{ L}}} lineam anguli 45 graduum. Esto \textit{qg} linea incidentiae 45 graduum, assumatur  linea incidentiae \textit{qg} quae sit tantum supra lineam 45 grad. quantum \textit{fg} est infra. \pend 