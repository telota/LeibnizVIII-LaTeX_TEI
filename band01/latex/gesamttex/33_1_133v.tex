\pstart [133 v\textsuperscript{o}] \edlabel{objectionsend}\textso{Object. 6.}
 Si la pression unitive des corps sensibles est plus forte, que la pression de l'atmosphere\protect\index{Sachverzeichnis}{atmosph\`{e}re}, par le phenomene 7. \edtext{(puisque en cas de concours}{\lemma{7.}\Afootnote{ \textit{ (1) }\ veu que \textit{ (2) }\ (puisque en cas de concours \textit{ L}}} le Mercure purg\'{e}\protect\index{Sachverzeichnis}{mercure!purg\'{e}} est suspendu \edtext{dans l'air libre}{\lemma{}\Afootnote{dans l'air libre \textit{ erg.} \textit{ L}}} jusques \`{a} 70 pouces, \edtext{outre le double de l'ordinaire}{\lemma{}\Afootnote{outre le double de l'ordinaire \textit{ erg.} \textit{ L}}} et l'ordinaire \edtext{restant}{\lemma{}\Afootnote{restant \textit{ erg.} \textit{ L}}} seulement jusques \`{a} 27 pouces), \edtext{il arrivera le même avec les deux plaques dans l'air libre,}{\lemma{pouces),}\Afootnote{ \textit{ (1) }\ pourquoy n'exerce-t-il \textit{ (2) }\ pourquoy en cas de con \textit{ (3) }\ et con \textit{ (4) }\ d'ou vient il, qu'il arrive pas le même, quand on  \textit{(a)}\ suspend un po \textit{(b)}\ tache de desunir deux placques unies dans l'air libre, \textit{ (5) }\  que le même n'arrive pas puisque l'air n'y peut pas arriver entre deux, \textit{ (6) }\ il [...] plaques \textit{(a)}\ et même \textit{(b)}\ dans l'air libre, \textit{ L}}} et elles ne seront pas seulement soûten\"{u}es de l'atmosphere\protect\index{Sachverzeichnis}{atmosph\`{e}re}, mais aussi de l'Effort du Mouuement general. Car comme le Mercure\protect\index{Sachverzeichnis}{mercure!purg\'{e}} est purg\'{e} de l'air par l'art, de même les deux placques\protect\index{Sachverzeichnis}{deux placques} en sont purg\'{e}es par la nature, n'y estant point d'apparence, qu'il s'\edtext{en produise}{\lemma{s'}\Afootnote{ \textit{ (1) }\ engendre \textit{ (2) }\ en produise \textit{ L}}} entre deux.\pend 
 \pstart Il faut respondre \`{a} cette objection en luy accordant tout ce qu'elle pretend, s\c{c}avoir que les deux placques\protect\index{Sachverzeichnis}{deux placques} dans l'air libre demeurent attach\'{e}es l'une \`{a} l'autre par l'union de ces deux efforts ensemble: de l'air, et du mouuement general; ou bien il faut dire, que l'union de deux placques dans le vuide\protect\index{Sachverzeichnis}{vide}, a un autre principe que la suspension \edtext{de la liqueur purg\'{e}e}{\lemma{suspension}\Afootnote{ \textit{ (1) }\ de la \textit{ (2) }\ du mercure\protect\index{Sachverzeichnis}{mercure|textit} pur \textit{ (3) }\ de la liqueur purg\'{e}e \textit{ L}}}: ce qui seroit renoncer \`{a} nostre hypothese pour les deux placques\protect\index{Sachverzeichnis}{deux placques} \edtext{ou trouver quelque raison de la difference}{\lemma{}\Afootnote{ou [...] difference \textit{ erg.} \textit{ L}}}. \edtext{C'est donc \`{a} l'experience}{\lemma{difference.}\Afootnote{ \textit{ (1) }\ Mais il faut exa \textit{ (2) }\ C'est donc \`{a} l'experience \textit{ L}}} de determiner, par le calcul des forces necessaires \`{a} la separation de deux placques\protect\index{Sachverzeichnis}{deux placques} dans l'air libre, et dans le vuide\protect\index{Sachverzeichnis}{vide}, si la seule pression de l'atmosphere\protect\index{Sachverzeichnis}{atmosph\`{e}re} suffit pour cet effect dans l'air libre. Mais il faut considerer aussi, que si nostre hypothese est refut\'{e}e par l'experience, alors toutes les autres, qui expliquent le \rule[-1cm]{0cm}{0.5cm}phenomene de la liqueur et de deux placques\protect\index{Sachverzeichnis}{deux placques} par une même raison \edtext{ne seront pas moins refut\'{e}es}{\lemma{raison}\Afootnote{ \textit{ (1) }\ seront refut\'{e}es aussi, \textit{ (2) }\ ne seront  \textbar\ aussi \textit{ gestr.}\ \textbar\  pas moins refut\'{e}es \textit{ L}}}. Et que \edtext{si leurs auteurs}{\lemma{que}\Afootnote{ \textit{ (1) }\ s'ils \textit{ (2) }\ si leurs auteurs \textit{ L}}} viennent \`{a} trouver des moyens pour se sauuer\edtext{ leurs excuses nous serviront}{\lemma{sauuer}\Afootnote{ \textit{ (1) }\ ; cet inconvenient, \textit{ (2) }\ leurs raisons, \textit{ (3) }\ et ils \textit{ (4) }\  leurs excuses nous serviront \textit{ L}}} de même.\edlabel{memestart}
 \edtext{}{\lemma{même.}\xxref{memestart}{memeend}
\Afootnote{ \textit{ (1) }\ \textso{Object. 6.} Que la pression de l'atmosphere\protect\index{Sachverzeichnis}{atmosph\`{e}re|textit} \textit{ (2) }\ Comme par exemple,   \textbar\ si les experiences l'exigeroient, \textit{ erg.}\ \textbar\  on pourroit dire, que  \textit{(a)}\ de soy\textendash même \textit{(b)}\ la pression de l'atmosphere\protect\index{Sachverzeichnis}{atmosph\`{e}re|textit} n'est pas moindre que l'effort \textit{(c)}\ l'effort [...] celuy \textit{ L}}}
 \pend 
 \pstart Comme par exemple, si les experiences l'exigeroient, on pourroit dire, que l'effort de l'atmosphere n'est pas moindre que celuy\edlabel{memeend}
 du mouuement general, \edtext{et qu'ils reviennet \`{a} peu pres au même}{\lemma{}\Afootnote{et [...] même \textit{ erg.} \textit{ L}}}, et qu'en faisant l'experience \edtext{de la desunion}{\lemma{}\Afootnote{la  \textit{ (1) }\ dissolution \textit{ (2) }\ desunion \textit{ erg.} \textit{ L}}} des placques \edtext{dans le vuide, un surmonte la resistence}{\lemma{placques}\Afootnote{ \textit{ (1) }\ dans l'air libre, il suffit d'en surmonter  \textit{(a)}\ une \textit{(b)}\ un \textit{ (2) }\ un surmonte seulement la resistance de l'atmosphere\protect\index{Sachverzeichnis}{atmosph\`{e}re|textit}, et qu'en faisant la même experience  \textit{ (3) }\ dans le vuide, un surmonte  \textit{(a)}\ seulement celle \textit{(b)}\ la resistence \textit{ L}}} du Mouuement \edtext{general; et qu'en faisant la même experience dans l'air libre un surmonte seulement celle de l'Atmosphere l'air ne suffrant pas, que la force unitive generalle pousse les placques en même ligne comme luy, n'estant pas assez penetrable \`{a} la matiere subtile pour cet effect. Et ainsi}{\lemma{general;}\Afootnote{ \textit{ (1) }\ mais que  \textit{(a)}\ leurs forces \textit{(b)}\ s'approchant, la raison de ces deux effects semble estre la même. Et que pour \textit{ (2) }\ et [...] experience  \textbar\ des placques \textit{ gestr.}\ \textbar\  dans l'air  \textit{(a)}\ general \textit{(b)}\ libre [...] ne \textit{(aa)}\ laissant pas agir \textit{(bb)}\ suffrant pas, que  \textit{(aaa)}\ l'effort du mouuement general pour l'union \textit{(bbb)}\ la force unitive   \textbar\ generalle \textit{ erg.}\ \textbar\ pousse [...] ainsi \textit{ L}}}