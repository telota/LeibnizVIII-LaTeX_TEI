\pend \pstart [p.~159] [...] hinc segmentum minus videtur, et scipio ita fractus, vt angulus LVR sit oppositus oculo\protect\index{Sachverzeichnis}{oculus} si vero oculus\protect\index{Sachverzeichnis}{oculus} sit in ipso catheto, puta in Q, nec scipio fractus, nec segmentum immersum minus videtur; hoc quippe videtur sub angulo VQR.\footnote{\textit{Am Rand angestrichen}: hinc segmentum [...] angulo VQR.}\pend \pstart  III. Sit porro superficies aquae AB, scipio CDE,\footnote{\textit{Gedruckte Marginalie}: Fig. 130.} in situ inclinato,\footnote{\textit{Leibniz unterstreicht}: Sit porro superficies aquae \textit{und} situ inclinato} vt supra, sit oculus\protect\index{Sachverzeichnis}{oculus} in G, scilicet inter oculum\protect\index{Sachverzeichnis}{oculus} et cathetum, sit radius refractus\protect\index{Sachverzeichnis}{radius!refractus} IG, ducantur GOE, GIF, [...].