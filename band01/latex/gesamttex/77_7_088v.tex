[88 v\textsuperscript{o}] focum\protect\index{Sachverzeichnis}{focus} circuli segmentorum, quae magna satis sunt, admodum  parvum esse, minusque remotum ab ipso vitro, quam in vitro \textit{HRDBQ} secundae figurae, si supponamus diametrum \textit{OP}, aequalem esse diametro \textit{ND}, ita ut vitrum hoc,  magnitudinem habens hujus figurae focum\protect\index{Sachverzeichnis}{focus} habiturum sit  paleae circiter latitudine a proxima superficie distantem.  Atque non difficile in hoc vitro similibusque aliis, simili calculo  quali hic supra usus sum, inveniri potest. Ex quo sequitur  aut per haec sola vitra inter se, aut cum praecedentibus composita,  fieri posse microscopia\protect\index{Sachverzeichnis}{microscopium}, quorum ope, ratione longitudinis,  objecta incredibili magnitudine apparere debent; imo  etiam per unicum tale vitrum magna admodum ac  distincta apparitura sunt.\pend \pstart  Praeteriri etiam hic non debet calculus congregationis  radiorum axi \textit{KD} parallelorum, (vide 2 fig:) posito quod  per vitrum eo modo permeerit, donec ad circumferentium \textit{DB} pervenirent ubi per aeris superficiem transeuntes  refringantur, ac axi \textit{KD} versus \textit{A} producto occurrant: nam  quamvis idem Cylindrus radiorum non in tam parva axis  lineola congregetur, nec tam parvum focum\protect\index{Sachverzeichnis}{focus}, quam supra 