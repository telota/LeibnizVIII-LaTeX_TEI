[13 v\textsuperscript{o}] fera un effect tout contraire si l'on fait toucher les secondes surfaces d'un rubis \`{a} \edtext{[celle]}{\lemma{}\Afootnote{celle \textit{ erg.} \textit{ Hrsg. }\ }} de l'eau mise dans un seau, ou dans un vaisseau dont le fonds n'ait point d'eclat, car alors la vivacit\'{e} de la couleur s'effacera presque entierement. Car la proportion de la refraction\protect\index{Sachverzeichnis}{r\'{e}fraction} du rubis \`{a} celle de l'eau est fort petite, ainsi la pluspart de la lumiere\protect\index{Sachverzeichnis}{lumi\`{e}re} passe de la pierre dans l'eau sans reflexion\protect\index{Sachverzeichnis}{r\'{e}flexion}, le même se fera dans les \'{E}meraudes et saphirs et encor plus sensiblement dans des verres color\'{e}s. Car la refraction\protect\index{Sachverzeichnis}{r\'{e}fraction} de l'eau au verre est comme 9 \`{a} 8. Que si l'on met des feuilles sous les verres color\'{e}s, comme sous les \edtext{pierres pretieuses}{\lemma{les}\Afootnote{ \textit{ (1) }\ vitres color\'{e}s \textit{ (2) }\ pierres pretieuses \textit{ L}}}, ils pourront paroistre avec autant d'eclat si leur couleur est aussi belle \`{a} cause que la lumiere\protect\index{Sachverzeichnis}{lumi\`{e}re} color\'{e}e repassera toute entiere aussi bien que travers du verre qu'\`{a} travers de la pierre. \pend \pstart Mons. Trocut\protect\index{Namensregister}{\textso{Trocut}} m'a fait voir une petite pierre de cristal de roche taill\'{e}e \`{a} huit pans dans toute laquelle il paroiste une fort belle couleur de rubis d'orient, quoyque la couleur rouge, qu'il y a appliqu\'{e}e par le dessous soit d'une epaisseur imperceptible le tranchant des vives arrestes; les degr\'{e}s et facettes n'en estant point alter\'{e}. Cette pierre qui en elle même est toute blanche estant mise dans un chaston avec une feuille dessous de la même couleur rouge ressemble parfaitement au plus beau rubis d'orient qu'on puisse trouver. \pend \pstart \edtext{[+ NB\edlabel{Klammer1start}}{{\xxref{Klammer1start}{Klammer1end}}\lemma{}\Afootnote{[...] \textit{Klammern von Leibniz}}}. \textso{Les peintures de couleurs diaphanes }\edtext{\textso{mises sur le derriere du verre}}{\lemma{}\Afootnote{ \textso{mises} [...] \textso{verre} \textit{ erg.} \textit{ L}}}\edtext{}{\lemma{\textso{verre,}}\Afootnote{\textit{doppelt unterstrichen}}} que j'ay fait faire \`{a} Paris\protect\index{Ortsregister}{Paris (Parisii)} par M. Schick\protect\index{Namensregister}{\textso{Schick,} Peter}, et dont j'ay des echantillons, \edtext{seront}{\lemma{echantillons,}\Afootnote{ \textit{ (1) }\ pourront pa \textit{ (2) }\ seront \textit{ L}}} incomparablement plus belles, si on peignoit la feuille de derriere d'argent bruni, des mêmes couleurs principalement aux plus beaux endroits. +]\edlabel{Klammer1end}\selectlanguage{latin} NB.\pend 