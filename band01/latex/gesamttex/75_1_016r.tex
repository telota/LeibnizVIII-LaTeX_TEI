      
               
                \begin{ledgroupsized}[r]{120mm}
                \footnotesize 
                \pstart                
                \noindent\textbf{\"{U}berlieferung:}   
                \pend
                \end{ledgroupsized}
            
              
                            \begin{ledgroupsized}[r]{114mm}
                            \footnotesize 
                            \pstart \parindent -6mm
                            \makebox[6mm][l]{\textit{L}}Konzept: LH XXXVII 2 Bl. 16. 1 Bl. 8\textsuperscript{o}. 2 S.  Linke Seite von Bl. 16 r\textsuperscript{o} beschnitten. Papier por\"{o}s, so dass die Tinte \"{u}ber weite Strecken durchschl\"{a}gt.\\Kein Eintrag in KK 1 oder Cc 2. \pend
                            \end{ledgroupsized}
                \vspace*{8mm}
                \pstart 
                \normalsize
             \centering[16 r\textsuperscript{o}] Notitia Opticae Promotae\\ autore G. G. \edlabel{ggllstart} L. L. \pend \vspace{1.0ex} \pstart \edtext{Cum\edlabel{ggllend}}{{\xxref{ggllstart}{ggllend}}\lemma{L. L.}
             \Afootnote{ \textit{ (1) }\ Utilissimam Humanarum scientiarum \textit{ (2) }\ Cum \textit{ L}}} saepe mecum cogitarem quantum a perfectione Opticae \edtext{in res humanas}{\lemma{Opticae}\Afootnote{ \textit{ (1) }\ rebus humanis \textit{ (2) }\ in res humanas \textit{ L}}} redundare utilitatis necesse sit, pandente nobis natura arcanos sinus, faciemque mundi centuplicante, atque insensibiles illas machinas detegente, quibus pleraeque etiam in corporibus nostris in peius meliusque mutationes peraguntur: officii mei esse putavi \edtext{}{\lemma{}\Afootnote{putavi  \textbar\ cogitationes \textit{ gestr.}\ \textbar\ nonnihil \textit{ L}}}nonnihil temporis, quod mihi \edtext{plurimum}{\lemma{mihi}\Afootnote{ \textit{ (1) }\ imo \textit{ (2) }\ plurimum \textit{ L}}} distracto exiguum superest, impendere \textso{scientiae} sancti fructus; sed more scilicet atque instituto meo, quo \edtext{assuevi}{\lemma{quo}\Afootnote{ \textit{ (1) }\ periisse mihi \textit{ (2) }\  assuevi \textit{ L}}} \edtext{eam operam studiis}{\lemma{assuevi}\Afootnote{ \textit{ (1) }\ ea studia \textit{ (2) }\ eam operam studiis \textit{ L}}} insumtam pro perdita habere, qua didici tantum, \edtext{sed quod adjicerem non}{\lemma{tantum,}\Afootnote{ \textit{ (1) }\ non et \textit{ (2) }\ et \textit{ (3) }\ sed quod adjicerem non \textit{ L}}} inveni. \pend \pstart Audio tale quiddam alios nonnullos egregios viros animo concepisse, sed cum publicare refrigerint, quam ingressi sint illi viam mihi non constat. Occasio rem penitius scrutandi haec fuit: diu est, ut amici norunt \edtext{(ultra parallaxes\protect\index{Sachverzeichnis}{parallaxis})}{\lemma{}\Afootnote{(ultra parallaxes\protect\index{Sachverzeichnis}{parallaxis}) \textit{ erg.} \textit{ L}}}, quod mihi in mentem venit \textso{ratio,} quaedam \textso{Optica metiendi ex una statione distantias }\edtext{\textso{quibus ex pluribus stationibus metimur}}{\lemma{}\Afootnote{\textso{quibus ex pluribus stationibus metimur} \textit{ erg.} \textit{ L}}}\textso{ magnitudinesque veras objectorum,} ita comparata ut spes sit ad coelestia usque extendi posse, quando et fundamentum illud alii innititur, eousque vim notabiliter exerit. \pend \pstart  Hanc cum nuper poliendam resumerem fuit in optices interiora inquirendum paulo diligentius, atque inprimis cogitandum de Figuris quibusdam quas ego novis nominibus \edtext{(quando et res aliis intacta est)}{\lemma{}\Afootnote{(quando [...] est) \textit{ erg.} \textit{ L}}} \textso{Isoptricas}\protect\index{Sachverzeichnis}{isoptrica}\textso{, }\textso{Dioptricas,}\protect\index{Sachverzeichnis}{dioptrica} et \textso{Paro}