\pend \pstart [p.~138] XI. Vnum omiseram obseruatione dignum quod  scilicet ita plano lentis\protect\index{Sachverzeichnis}{lens} noui generis oculi\protect\index{Sachverzeichnis}{oculus} pupilla\protect\index{Sachverzeichnis}{pupilla} admoueatur, vt ab omnibus plani punctis aequidistet, nimirum interceptis lineis parallelis aequalibus, cum tamen  omnes lineae parallelae a conuexo lentis\protect\index{Sachverzeichnis}{lens} productae inaequales sint; praeterea intra lentem\protect\index{Sachverzeichnis}{lens} vtrimque conuexam omnes radij refracti\protect\index{Sachverzeichnis}{radius!refractus} inaequales sunt, cum tamen intra lentem\protect\index{Sachverzeichnis}{lens} noui generis aequales sint, physice omnia scilicet  aggregata refractorum v.g. CB NM LK: multum  autem confert huiusmodi radiorum refractorum\protect\index{Sachverzeichnis}{radius!refractus} aequalitas ad praescriptum refractionum\protect\index{Sachverzeichnis}{refractio} ordinem seruandum;  quod si vacuitas ACS\footnote{\textit{Gedruckte Marginalie}: Fig. 107.} aqua plena sit, radius DE erit  quidem refractus, sed parum admodum; vnde perinde  fere se habebit, atque si NE rectus esset, ac deinde refringeretur in EF, vnde dupla esset foci\protect\index{Sachverzeichnis}{focus} distantia. Sed  de his satis.\footnote{\textit{Am Rand angestrichen}: Vnum omiseram [...] his satis.}