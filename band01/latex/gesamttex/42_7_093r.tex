[93 r\textsuperscript{o}]  Sed quicquid ejus sit pro constituto certe habendum est: aerem  cum difficultate quadam in rimas corporum angustiores penetrare;  et quod aqua facit cum in sicco aut inter pinguia deprehenditur,  ut scilicet in guttam sese tornet, aegreque excurrat \edtext{in rivos}{\lemma{}\Afootnote{in rivos \textit{ erg.} \textit{ L}}}, id aeri ordinarium  ac solenne esse.\pend \pstart  Et \edtext{hanc unam observationem ad phaenomena Liquoris purgati}{\lemma{Et}\Afootnote{ \textit{ (1) }\ hoc jam unicum ad rationem  phaenomeni aere, \textit{ (2) }\ hanc [...] purgati \textit{ L}}} altius suspensi explicanda  sufficere credo. Finge \edtext{tibi}{\lemma{tibi}\Afootnote{ \textit{ erg.} \textit{ L}}} pro aere guttam aquae in \edtext{Tabula deprehensam,}{\lemma{in}\Afootnote{ \textit{ (1) }\ pavimento \textit{ (2) }\ Tabula deprehensam, \textit{ L}}}  pro Mercurio\protect\index{Sachverzeichnis}{mercurius} aliove liquore penetrando varias in Tabula  fossulas rimasque; \edtext{quemadmodum}{\lemma{}\Afootnote{quemadmodum \textit{ erg.} \textit{ L}}} si \edtext{pars Tabulae guttam}{\lemma{si}\Afootnote{ \textit{ (1) }\ eadem gutta \textit{ (2) }\ pars Tabulae guttam \textit{ L}}}  circumjacens madida \edtext{sit}{\lemma{madida}\Afootnote{ \textit{ (1) }\ supponatur \textit{ (2) }\ sit \textit{ L}}} gutta facile per  Tabulam diffunditur; ita si Mercurius\protect\index{Sachverzeichnis}{mercurius} aere jam tum rigatus sit,  facile ab aere penetratur; finge partem Tabulae guttam  attingentem esse siccam et ab aqua purgatam, videbis, guttam  non diffundi; eodem modo si Mercurius\protect\index{Sachverzeichnis}{mercurius!purgatus} ab aere sit purgatus,  ab aere difficulter penetrabitur. Madefiat \edtext{Tabula}{\lemma{Madefiat}\Afootnote{ \textit{ (1) }\ aqua \textit{ (2) }\  Tabula \textit{ L}}} denuo; en guttam statim se diffundentem; vicissim  irrigetur mercurius\protect\index{Sachverzeichnis}{mercurius} aere novo, circumfusus ille exclususque  \edtext{antea aer facile per Mercurium diffundetur.}{\lemma{antea}\Afootnote{ \textit{ (1) }\ facile per eum diffundetur \textit{ (2) }\ aer facile per Mercurium diffundetur \textit{ L}}} \edtext{Aut si loco}{\lemma{diffundetur}\Afootnote{ \textit{ (1) }\ ; item loco ir \textit{ (2) }\ . Aut si loco \textit{ L}}} madefactionis novae, Tabula  concutiatur, gutta\edtext{}{\lemma{}\Afootnote{gutta  \textbar\ eodem modo \textit{ gestr.}\ \textbar\ procurret \textit{ L}}} procurret in Tabulam etiam  siccam; ita si Tubi latera concutiuntur, aer tornatione  sua introitum impediente \edtext{excussus in Mercurium}{\lemma{excussus}\Afootnote{ \textit{ (1) }\ per  \textit{(a)}\ aerem \textit{(b)}\ Mercurium\protect\index{Sachverzeichnis}{mercurius|textit} \textit{ (2) }\  in Mercurium \textit{ L}}} exundabit. En facilem simplicemque  tot phaenomenorum \edlabel{expli93ra}\edtext{explicationem.}{\lemma{explicationem.}\xxref{expli93ra}{expli93rb}\Afootnote{ \textit{ (1) }\ Exundabit autem  pervadetque pars subtilior tantum \textit{ (2) }\ At cur non infra  pollices 27. subsidet. Necesse est  id fieri,  \textit{(a)}\ quia aer ultra sub \textit{(b)}\ hic aeris gravitas\protect\index{Sachverzeichnis}{gravitas!aeris|textit} \textit{ (3) }\  Experimentis novissimis   \textbar\ Pneumaticis \textit{ erg.}\ \textbar\  \textit{ L}}} \pend 
\pstart Experimentis novissimis Pneumaticis\edlabel{expli93rb} Illustris Hugenii\protect\index{Namensregister}{\textso{Huygens} (Hugenius, Vgenius, Hugens, Huguens), Christiaan 1629\textendash 1695} manifeste confectum  est, \edtext{quod antea rationibus tantum longinquis assequebamur,}{\lemma{est,}\Afootnote{ \textit{ (1) }\ vacuum Recipientem \textit{ (2) }\ quod [...] assequebamur, \textit{ L}}} Antliam Pneumaticam\protect\index{Sachverzeichnis}{antlia!pneumatica} utcunque  exhaustam non tamen prorsus vacuam esse, sed corpus aliquod  superesse debere, \edtext{quod effectus illos aeris pressioni ascriptos, (ut duarum Tabularum laevigatarum cohaesionem effectumque, siphonisque bicruri uno crure infra aquae in vase contentae superficiem descendentis altero aquam ex eo haurientis donec aquae superficies a cruris extra vas orificium usque deprimatur) praestet.}{\lemma{debere,}\Afootnote{ \textit{ (1) }\ quo duae laminae politae\protect\index{Sachverzeichnis}{laminae politae|textit} junctaeque  \textit{(a)}\ contineantur  quod antea \textit{(b)}\ contineantur quod  \textit{(aa)}\ aqua in siphone\protect\index{Sachverzeichnis}{sipho|textit}  \textit{(aaa)}\ aequicruro \textit{(bbb)}\ iniquicruro \textit{(bb)}\ aquam in siphonem\protect\index{Sachverzeichnis}{sipho|textit}  \textit{(aaa)}\ iniquicrurum \textit{(bbb)}\ bicrurum ascendere cogat \textit{ (2) }\ quod [...] ascriptos, \textit{(a)}\ praestet \textit{(b)}\ (ut duarum Tabularum laevigatarum cohaesionem   \textbar\ effectumque \textit{ erg.}\ \textbar\ , siphonisque bicruri  \textit{(aa)}\ aquam \textit{(bb)}\ ea \textit{(cc)}\ phaenomenon \textit{(dd)}\ uno crure  \textit{(aaa)}\ ex vase \textit{(bbb)}\ infra [...] superficies \textit{(aaaa)}\ in eadem sit \textit{(bbbb)}\ a [...] praestet. \textit{ L}}}  Et credo ipsum celeberrimum Gerickium\protect\index{Namensregister}{\textso{Guericke} (Gerickius, Gerick.), Otto v. 1602\textendash 1686} nostrum, \edtext{magni illius phaenomeni Pneumatici primum inventorem,}{\lemma{nostrum,}\Afootnote{\textit{ (1) }\ illustris \textit{ (2) }\ magni illius phaenomeni Pneumatici primum inventorem, \textit{ erg.} \textit{ L}}}  re intellecta, vacuo\edtext{}{\lemma{}\Afootnote{vacuo  \textbar\ suo \textit{ gestr.}\ \textbar\ quod \textit{ L}}} quod summum vocat renuntiaturum. \edtext{At ex iisdem Experimentis cum his quae dudum noveramus collatis discimus phaenomenon Torricellianum, quod nunc Baroscopium doctissimi quidam Viri appellant, a memoratis illis aeris pressioni itidem ascriptis longe lateque differre. Aere enim exhausto cessat Effectus Baroscopii, nam si Tubus Torricellianus in Vacuo quod vocant, id est Recipiente exhausto locetur, Mercurius aut aqua (modo aere non sint purgata, ut habeant, quo implere possint locum quem descendendo vacuum in Tubi summitate relinquunt) descendunt ex Tubo ad horizontem usque liquoris in vase subjecto stagnantis; at vero laminarum politarum cohaesio siphonisque bicruri effectus non cessant. Necesse est ergo diversas esse experimenti Torricelliani et caeterorum duorum phaenomenorum causas et phaenomenon Torricellianum ab Aeris pressione. At vero confectum mihi videtur illud quoque Laminarum politarum}{\lemma{renuntiaturum.}\Afootnote{ \textit{ (1) }\ De  reliquorum phaenomenorum causis, liceat mihi quando \textit{ (2) }\ Sed an  confugiendum nobis sit ad \textit{ (3) }\ materiae cujusdam aere subtilioris  pressio tam horum phaenomenorum \textit{ (4) }\ At videtur mihi  etiam ultra hinc ex his experimentis amplius quiddam  sequi, scilicet non oriri has corporum connexiones ab  ulla materiae pressione. \textit{ (5) }\ At de laminis\protect\index{Sachverzeichnis}{laminae politae|textit} p \textit{ (6) }\   \textbar\ At [...] collatis \textit{ (1) }\ sequitur ut \textit{ (2) }\ discimus phaenomenon Torricellianum, \textit{ (a) }\ quod \textit{ (aa) }\ caeteris aeris \textit{ (bb) }\ a duobis isti s \textit{ (b) }\ quod \textit{ (aa) }\ vulgo B \textit{ (bb) }\ nunc [...] appellant, \textit{ (aaa) }\ a caeteris \textit{ (bbb) }\ a \textit{ (aaaa) }\ dictis \textit{ (bbbb) }\ memoratis illis [...] diversas esse \textit{ (aaaaa) }\ Baroscopii et Tubi \textit{ (bbbbb) }\ experimenti [...] et \textit{ (aaaaa-a) }\ cum Baroscopium \textit{ (bbbbb-b) }\ phaenomenon Torricellianum ab Aeris pressione. \textit{ erg.}\ \textbar\ At [...] politarum \textit{ L}}} cohaesionem et Mercurii\protect\index{Sachverzeichnis}{mercurius} alteriusve liquoris  suspensionem \edtext{supra suum horizontem et antliarum suctionem}{\lemma{suspensionem}\Afootnote{ \textit{ (1) }\ in Tubo Torricelliano\protect\index{Sachverzeichnis}{Tubus!Torricellianus|textit} \textit{ (2) }\ supra [...] suctionem \textit{ L}}} non oriri ab eadem  causa, quod nostri temporis philosophis videbatur. Nam  in \edtext{Recipiente}{\lemma{in}\Afootnote{ \textit{ (1) }\ Tubo \textit{ (2) }\ Recipiente \textit{ L}}} exhausto aeris \edtext{residui}{\lemma{}\Afootnote{residui \textit{ erg.} \textit{ L}}} pressio non est  satis valida ad sustinendum liquorem suspensum, descendit  enim ad horizontem usque liquoris ejusdem in subjecto  vase stagnantis. Ergo nec \edtext{satis valida erit}{\lemma{nec}\Afootnote{ \textit{ (1) }\ valida esse t \textit{ (2) }\ satis valida erit \textit{ L}}} ad \edtext{impediendam laminarum\protect\index{Sachverzeichnis}{laminae politae} divulsionem}{\lemma{impediendam}\Afootnote{ laminarum\protect\index{Sachverzeichnis}{laminae politae} divulsionem \textit{ erg.} \textit{ L}}} sustinendumque  pondus laminae\protect\index{Sachverzeichnis}{laminae politae} inferiori appensum. Impeditur tamen  divulsio, ergo ab alia causa: \edtext{Eodem modo probatur a pressione  aeris non oriri siphonis}{\lemma{causa:}\Afootnote{ \textit{ (1) }\ At vero ab eadem causa oriri videtur connexio laminarum\protect\index{Sachverzeichnis}{laminae politae|textit},  et suspensio Liquoris\protect\index{Sachverzeichnis}{liquor!purgatus|textit} aere purgati, et attractio\protect\index{Sachverzeichnis}{attractio|textit}  per Antliam\protect\index{Sachverzeichnis}{antlia|textit} follemve diductum, et suspensio liquoris.  Ergo siphonis\protect\index{Sachverzeichnis}{sipho|textit} quoque  \textit{(a)}\ iniquicruri \textit{(b)}\ bicruri per cujus crus extra  vas descendens. \textit{ (2) }\ Eodem modo probatur  \textit{(a)}\ ab aeris pressione\protect\index{Sachverzeichnis}{pressio!aeris|textit} \textit{(b)}\ neque a gravitate\protect\index{Sachverzeichnis}{gravitas|textit} neque ab Elaterio\protect\index{Sachverzeichnis}{elaterium|textit} aeris oriri \textit{(c)}\ a pressione  aeris non oriri siphonis \textit{ L}}} bicruri phaenomenon, cujus alterum  crus \edtext{orificio suo}{\lemma{}\Afootnote{orificio suo \textit{ erg.} \textit{ L}}} aquam in vase stagnantem intrat, alterum infra  ejus horizontem descendit, et aquam tam diu effluere  cogit, donec orificium cruris extra vas \edtext{ non sit amplius infra ejus horizontem.}{\lemma{vas}\Afootnote{ \textit{ (1) }\ non descendat \textit{ (2) }\ non [...] horizontem. \textit{ L}}}\pend \pstart 