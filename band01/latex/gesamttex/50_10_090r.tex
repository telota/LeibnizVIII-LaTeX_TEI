\pstart [90 r\textsuperscript{o}] Verum quidem est mathematice loquendo radios hos per circulum hunc paulo magis dispergi debere, cum antea non ad unum punctum mathematicum\protect\index{Sachverzeichnis}{punctum!mathematicum} tenderent; sed dispersionem hanc tantam non esse, quin focus\protect\index{Sachverzeichnis}{focus}, sive minimum planum ad quod postea ex vitro egressi ac aerem transeuntes, tendent, et in quo congregabuntur, pro puncto mechanico\protect\index{Sachverzeichnis}{punctum!mechanicum} habendum sit, simili calculo, aut etiam mechanice, facile constare potest.\pend \pstart Sed cum planum supra inventum; cujus semidiameter est \textit{IN} indeterminatum sit, ac propius ad \textit{N} accedat aut magis ab eo removeatur, prout apertura, aut \textit{BF}, major aut minor sumitur, determinatum planum ejus loco quae remus. Concipiatur ex \textit{K}\footnote{\textit{An der Mittelfalz}: fig. 1.} erectam esse perpendicularem cui producta \textit{BI} occurrit in \textit{O}, erit itaque \textit{IF} ad \textit{FB}, ut \textit{IK} ad \textit{KO}, reperieturque \textit{KO} minor, cum \textit{FB} est \rule[0mm]{0mm}{5mm} $\displaystyle\frac{9}{41}\:$\rule[-4mm]{0mm}{10mm}
quam \rule[0mm]{0mm}{5mm} $\displaystyle\frac{1}{439}$;\rule[-4mm]{0mm}{10mm} cum \textit{FB} est $\displaystyle\frac{31}{481}$, quam $\displaystyle\frac{1}{17625}$; cum \textit{FB} est $\displaystyle\frac{49}{1201}$, quam $\displaystyle\frac{1}{69590}$, differentia igitur quae est inter hanc \textit{KO} et semidiametrum \textit{IM}\rule[0mm]{0mm}{5mm} praecedentis plani ita parva est ut consideratu digna non sit, nec conclusio exinde ducta mutetur. Cum autem \textit{NK} supra reperta sit aequalis $\displaystyle\frac{429}{321}$,\rule[-4mm]{0mm}{10mm} quae est $\displaystyle1\frac{6}{7}$, erit \textit{KD} aequalis $\displaystyle2\frac{6}{7}$; \rule[-4mm]{0mm}{10mm}