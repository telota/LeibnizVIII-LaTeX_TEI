[141 r\textsuperscript{o}] Que si l'on dit, que la liqueur purg\'{e}e tombe, quand le mouuement du liquide ambient la presse de deux costez, \edtext{et qu'il ne faut pas}{\lemma{costez,}\Afootnote{ \textit{ (1) }\ sans \textit{ (2) }\ et qu'il ne faut pas \textit{ L}}} se mettre en peine de la quantit\'{e} des vagues; comme il suffit, \edtext{dans l'experience de Torricelli, qu'on perce le haut du tuyau tant soit peu}{\lemma{dans}\Afootnote{ \textit{ (1) }\ le tuyau\protect\index{Sachverzeichnis}{tuyau de Torricelli|textit} \textit{ (2) }\ l'experience de Torricelli,  \textit{(a)}\ que peu d'air en \textit{(b)}\ qu'on [...] peu \textit{ L}}} pour faire tomber le Mercure\protect\index{Sachverzeichnis}{mercure}: \edtext{Je repondray qu'il y a une  grande difference; car la raison  est manifeste, pourquoy une  petite ouuerture vaut bien  une grande, quand}{\lemma{Mercure:}\Afootnote{ \textit{ (1) }\ je repondray, qu'on a  \textit{(a)}\ allegu\'{e} aupa \textit{(b)}\ so\^{u}tenu auparavant, en expliquant le phenomene des placques, que les placques quoyque poreuses, et quoyque par consequent  \textit{(aa)}\ pressez par le mouuement du fluide ambient dans leur sup \textit{(bb)}\ le mouuement du fluide ambient du cost\'{e} de la superficie interieure, aussi bien que de l'exterieure, ne se separent pas. On me repartira, et avec raison, que le m\^{e}me arriveroit dans l'air libre,  \textit{(aaa)}\ si les placques seroient \textit{(bbb)}\ que les placques si elles seroient perc\'{e}es de grands trous sensibles justement comme dans la \textso{fig. 2} ne se separeroient pas pour cela  \textit{(aaaa)}\ , et que la raison de la difference, pourquoy la placque superieure estant perc\'{e}e, l'inferieure ne tombe \textit{(bbbb)}\  . Mais je repliqueray aussi que la raison de la difference, de l'evenement, c'est a dire, pourquoy la placque superieure estant perc\'{e}e, l'inferieure ne tombe pas pour cela; et qu'au contraire, le tuyau estant perc\'{e} en haut,  \textit{(aaaaa)}\ l'eau \textit{(bbbbb)}\ la liqueur tombe incontinent; est sans doute la difference entre les corps solides\protect\index{Sachverzeichnis}{corps!solide|textit} ou roides, comme les placques, et entre les pliants, comme les liqueurs qui donnent passage partout apres une petite  \textbar\ entr\'{e}e ou \textit{ gestr.}\ \textbar\  ouuerture. Il faudroit donc que les pores du tuyau de verre fissent tomber la liqueur purg\'{e}e   \textbar\ ce qui est contre l'experience \textit{ erg.}\ \textbar\ , puisque la matiere\protect\index{Sachverzeichnis}{mati\`{e}re!subtile|textit} plus subtile que l'air, \`{a} ce qu'on dit n'y passe pas moins ais\'{e}ment, que l'air par des trous sensibles; ou il faut confesser, que la pression   \textbar\ du fluide ambient \textit{ erg.}\ \textbar\  de deux costez, sur la liqueur suspendue ne suffit pas, pour la faire tomber, si la quantit\'{e} des vagues ou coups n'est pas \'{e}galle. Et comme  \textit{(aaaaa-a)}\ il est impossible que tant de vagues ou coups se trouuent dans \textit{(bbbbb-b)}\ tant de vagues ou coups du fluide ambient ne viennent pas de la petite bulle d'air, que du grand Recipient par l'ouuerture de la phiole, la pression ne sera pas \'{e}gale, et par consequent la liqueur purg\'{e}e ne deuroit pas tomber. \textit{ (2) }\ Et il y a grande difference \textit{ (3) }\ Mais sans tant de reparties  \textit{(a)}\ il manifeste, \textit{(b)}\  il est m \textit{(c)}\ la difference \textit{(d)}\ il est manifest \textit{(e)}\ la raison est manifeste, \textit{ (4) }\ Je  \textit{(a)}\ dira \textit{(b)}\ respondray pourquoy une petite ouuerture  \textit{(aa)}\ de la \textit{(bb)}\ \'{e}gale une grande, qua \textit{(cc)}\ vaut une grande, quand \textit{ (5) }\ Je [...] quand \textit{ L}}}%
\footnote{\textit{Gestrichene Marginalie zur 1. der vorausgehenden Ersetzungen:} Il faut donc répondre, que la matiere fluide plus subtile que l'air passe bien par les pores du verre mais non sans difficulté, comme peut estre l'air passeroit aussi difficilement et ne feroit pas tomber le mercure du tuyau de Torricelli si le trou seroit trop petit, car nous sçavons que l'air ne passe pas aisément par tout.}
%Afootnote zur Footnote in 141v verschoben
%\edtext{}{\lemma{}\Afootnote{\textit{ (1) }On pourra répondre que \textit{ (2) } Il [...] aussi \textit{ (a) } dans \textit{ (b) } difficilement [...] car \textit{ (aa) } nous voyons \textit{ (bb) } nous sçavons [...] tout.}}