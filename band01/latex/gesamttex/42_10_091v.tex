[91 v\textsuperscript{o}]  non video cert⟨e⟩ quomodo hujus corporis \edtext{aere subtilioris}{\lemma{}\Afootnote{aere subtilioris \textit{ erg.} \textit{ L}}} praesentia aeris pressionem\protect\index{Sachverzeichnis}{pressio!aeris} \edtext{Mercuriique altitudinem}{\lemma{pressionem}\Afootnote{ \textit{ (1) }\ Tubique alti \textit{ (2) }\ Mercuriique altitudinem \textit{ L}}} augeat. \edtext{Inerit enim}{\lemma{augeat.}\Afootnote{ \textit{ (1) }\ Constat e \textit{ (2) }\ Inerit enim \textit{ L}}}  utique et aeri Mercurium\protect\index{Sachverzeichnis}{mercurius!purgatus}\edtext{}{\lemma{}\Afootnote{Mercurium  \textbar\ aere non purgatum \textit{ erg. u.}\  \textit{ gestr.}\ \textbar\ Tubi \textit{ L}}} Tubi \edtext{Torricelliani\protect\index{Sachverzeichnis}{Tubus!Torricellianus} communi  modo adhibitum, prementi,}{\lemma{Torricelliani}\Afootnote{ \textit{ (1) }\ ordinariem pr \textit{ (2) }\ communi  modo adhibitum, prementi, \textit{ L}}} qui tamen ad eam  altitudinem non pervenit, nec unquam, quod sciam, triginta  pollices excedit, etsi pro temporum locorumque ratione variet.\pend
 \pstart  Sequitur quarto \textso{causam }\edtext{}{\lemma{\textso{causam}}\Afootnote{\textbar\ \textso{altitudinis} \textit{ gestr.}\ \textbar\ \textso{Mercurii} \textit{ L}}}\edtext{\textso{Mercurii}\protect\index{Sachverzeichnis}{mercurius}\textso{ aut alterius liquoris}}{\lemma{\textso{Mercurii}}\Afootnote{ \textit{ (1) }\ \textso{aquaeve} \textit{ (2) }\ \textso{aut alterius liquoris} \textit{ L}}}\textso{ ultra  vasis inferius stagnantis horizontem solito}\edtext{\textso{ altius suspensi esse petendam.}\edlabel{peti1}}{\lemma{\textso{altius}}\Afootnote{ \textit{ (1) }\ \textso{suspensi, esse ab ipso aere petendam} \textit{ (2) }\ \textso{suspensi esse petendam.} \textit{ L}}}\footnote{\textit{Zur folgenden Streichung geh\"{o}rige Marginalie:} Ostendam suo loco omnia quoque frigoris phaenomena ab  aeris gravitate posse derivari. Aer enim calore rarefactus  plus spatii occupat ac proinde columnam atmosphaerae  incumbentem aut  etiam in opposi [\textit{Satz bricht ab}]}
%AFussnote bezieht sich auf Fussnote, deshalb manuell an das Ende der Seite gebracht{\lemma{?LEMMA?:liberatum}\Afootnote{ \textit{ (1) }\ frigus\protect\index{Sachverzeichnis}{frigus|textit} quoque \textit{ (2) }\ omnia [...] incumbentem \textit{(a)}\ attollit; at vero ubi vis dilata \textit{(b)}\ aut  etiam in opposi \textit{ L}}}}
\edtext{}{\lemma{\textso{petendam.}}\xxref{peti1}{peti2}\Afootnote{  \textbar\ Cum enim phaenomenon hoc eveniat tum demum, quoties Mercurius\protect\index{Sachverzeichnis}{mercurius!purgatus} aliusve liquor\protect\index{Sachverzeichnis}{liquor!purgatus} aere purgatus est, et aere admisso cesset: Non video  quomodo aeri ratio hujus effectus adimi possit. Etsi enim eveniat  in Recipiente exhausto, evenit tamen et in pleno; et  \textbar\ in ipso  ut postea dicam, quod in Recipiente utcunque \textit{ gestr.}\ \textbar\ Ex \textit{ L}}}
%Einfuegung AFussnote von oben mit deklarierten Zeilenzaehler
\edtext{}{\lemma{loco}\linenum{|11|||14|}\Afootnote{ \textit{ (1) }\ frigus\protect\index{Sachverzeichnis}{frigus|textit} quoque \textit{ (2) }\ omnia [...] incumbentem \textit{(a)}\ attollit; at vero ubi vis dilata \textit{(b)}\ aut  etiam in opposi \textit{ L}}}\pend 
\pstart  Ex his \edtext{\edlabel{peti2}quatuor phaenomeni circumstantiis}{\lemma{his}\Afootnote{ \textit{ (1) }\ tribus phaenomenis \textit{ (2) }\ quatuor phaenomeni circumstantiis \textit{ L}}}: primum quod experimentum non  succedit, nisi liquor\protect\index{Sachverzeichnis}{liquor!purgatus} sit aere purgatum, secundo quod aere  immisso cessat, tertio quod cessat tubi lateribus fortiuscule  percussis, quarto quod in aere pariter ordinario  et in \edtext{aere Recipientis Magdeburgici suctione attenuati}{\lemma{in}\Afootnote{ \textit{ (1) }\ Recipiente Magdeburgico\protect\index{Sachverzeichnis}{Recipiens!Magdeburgicum|textit} suctione attenuato \textit{ (2) }\ aere Recipientis Magdeburgici suctione attenuati \textit{ L}}}  evenit.\pend \pstart  Horum ut ratio intelligatur, ante omnia annotandum  puto. Tres esse primarias aeris qualitates, unde  tot\edtext{}{\lemma{}\Afootnote{tot  \textbar\ admiranda ejus \textit{ gestr.}\ \textbar\ experimenta \textit{ L}}} experimenta praeclara pendeant, (1) Gravitatem\protect\index{Sachverzeichnis}{gravitas},  (2) vim Elasticam\protect\index{Sachverzeichnis}{vis!elastica}, (3) tenacitatem seu partium  cohaesionem.\pend 
\pstart   De Gravitate aeris\protect\index{Sachverzeichnis}{gravitas!aeris} non est dubitandum postquam Gerickius\protect\index{Namensregister}{\textso{Guericke} (Gerickius, Gerick.), Otto v. 1602\textendash 1686} noster accuratissimis  observationibus definivit quanto Recipiens exhaustus sit pleno levior\edtext{}{\lemma{levior}\Bfootnote{\textsc{O. v. Guericke, }\cite{00055}\textit{Experimenta nova}, Amsterdam 1672, S.~101. }}  postquam constat Barometro\protect\index{Sachverzeichnis}{barometrum} \edtext{tum per Mercurium\protect\index{Sachverzeichnis}{mercurius} aquamve, tum etiam simplici aere}{\lemma{tum}\Afootnote{per [...] constructo \textit{ erg.} \textit{ L}}} \edtext{constructo diversos gravitatis gradus mensurari;}{\lemma{constructo}\Afootnote{ \textit{ (1) }\ aeris varieta \textit{ (2) }\ gravitatem\protect\index{Sachverzeichnis}{gravitas|textit} mutantis varietatis observari, et quod notatu dignum est \textit{ (3) }\ diversos gravitatis  \textbar\ aereae \textit{ gestr.}\ \textbar\  gradus mensurari; \textit{ L}}} et adeo  duo Hemisphaeria cuprea \edtext{aut laminas politas\protect\index{Sachverzeichnis}{laminae politae}}{\lemma{aut}\Afootnote{laminas politas\protect\index{Sachverzeichnis}{laminae politae} \textit{ erg.} \textit{ L}}} facilius divelli, quando  Index Barometri\protect\index{Sachverzeichnis}{barometrum} est
humilior; et aeris gravitatem\protect\index{Sachverzeichnis}{gravitas!aeris}  esse \edtext{minorem,}{\lemma{minorem,}\Bfootnote{\textsc{O. v. Guericke, }\cite{00055}a.a.O., S.~114.}} quoties vento agitatur et quasi portatur, quemadmodum  in aqua agitata corpora saepe natant, quae in quiescente subsidunt.\pend 
\pstart \textso{Vis aeris Elastica a}\protect\index{Sachverzeichnis}{vis!elastica} gravitate\protect\index{Sachverzeichnis}{gravitas} ejus omnino separanda est;  nam ubi primum \edtext{aer ab inferiore aere aliove superiore corpore incumbente  premi}{\lemma{primum}\Afootnote{ \textit{ (1) }\ corpus premi \textit{ (2) }\ aeri ab aere aliove incumbente \textit{ (3) }\ aer [...] premi \textit{ L}}} desinit, proprio nisu se expandit, instar lanae \edtext{alteriusve  corporis villosi.}{\lemma{lanae}\Afootnote{ \textit{ (1) }\ aliorumve corporum villosorum \textit{ (2) }\ alteriusve  corporis villosi. \textit{ L}}} \edtext{Conaturque si possit, restituere se in eam expansionem, quam habet aer summus minime pressus. Etsi}{\lemma{villosi.}\Afootnote{ \textit{ (1) }\ Etsi sit inter Gravitatem\protect\index{Sachverzeichnis}{gravitas|textit} et Elaterium\protect\index{Sachverzeichnis}{elaterium|textit} \textit{ (2) }\ Est tamen \textit{ (3) }\ Conaturque [...] Etsi \textit{ L}}} vero differant hae qualitates, est tamen inter eas congruentia admirabilis; cum enim aer in dato aliquo spatio  comprehensus tantum habeat \edtext{compressionis}{\lemma{habeat}\Afootnote{ \textit{ (1) }\ pressionis \textit{ (2) }\ compressionis \textit{ L}}}, quantum columna aerea  ei incumbens habet gravitatis\protect\index{Sachverzeichnis}{gravitas} (prorsus ac si \edtext{Elaterium\protect\index{Sachverzeichnis}{elaterium} pondere aliquo appenso tendas).}{\lemma{Elaterium}\Afootnote{ \textit{ (1) }\ quoddam  ingenti saxo ei imposito graves) \textit{ (2) }\ pondere aliquo appenso tendas). \textit{ L}}} Sequitur \edtext{amoto aere}{\lemma{Sequitur}\Afootnote{ \textit{ (1) }\ idem exhausto  aere \textit{ (2) }\ amoto aere \textit{ L}}} comprimente vim aeris Elasticam\protect\index{Sachverzeichnis}{vis!elastica} restituentem  tantam esse quanta antea gravitas\protect\index{Sachverzeichnis}{gravitas} columnae aereae comprimentis  fuerat. Et haec est ratio cur complura phaenomena, quod  Clarissimus Pascalius\protect\index{Namensregister}{\textso{Pascal} (Pascalius), Blaise 1623\textendash 1662} ad gravitatem aeris\protect\index{Sachverzeichnis}{gravitas!aeris} retulerat, ad vim ejus Elasticam\protect\index{Sachverzeichnis}{vis!elastica}  potius sint referenda, ut quod \edtext{globulus plumbeus}{\lemma{quod}\Afootnote{ \textit{ (1) }\ globus plu \textit{ (2) }\ globulus plumbeus \textit{ L}}} sclopetarius\protect\index{Sachverzeichnis}{sclopetum}  suctu evocatus tanta vi prorumpit per canalem, id enim  fieri necesse est, quia aer in canali minus premitur  ab aere incumbente, quando homo sugit; ac proinde sese  dilatat: item \edtext{quod duae laminae politae}{\lemma{quod}\Afootnote{ \textit{ (1) }\ corpora \textit{ (2) }\ duae laminae politae \textit{ L}}} cohaerent,  nam si hoc fieret ob solam gravitatem\protect\index{Sachverzeichnis}{gravitas} columnae aereae  non posset \edtext{experimentum fieri nisi in aere libero nunquam in vase clauso,}{\lemma{posset}\Afootnote{ \textit{ (1) }\ fieri in vase clauso  \textit{(a)}\ aere plen \textit{(b)}\ licet pleno \textit{ (2) }\ experimentum [...] libero \textit{(a)}\ nec unquam licet \textit{(b)}\ nunquam in vase clauso, \textit{ L}}}  necesse est ergo\edtext{}{\lemma{}\Afootnote{ergo  \textbar\ in vase clauso aere pleno \textit{ erg. u.}\  \textit{ gestr.}\ \textbar\ aer \textit{ L}}} \edtext{aer ipse Elaterem}{\lemma{ergo}\Afootnote{ \textit{ (1) }\ aeris ipsius inclusum \textit{ (2) }\ aer ipse Elaterem \textit{ L}}} in vase clauso tantum  obstare divulsioni laminarum\protect\index{Sachverzeichnis}{laminae politae}, quantum obstat in aere  libero columnae aereae gravitas\protect\index{Sachverzeichnis}{gravitas}. \edtext{Divulsione enim laminarum\protect\index{Sachverzeichnis}{laminae politae} in aere libero columna aerea per totam atmosphaerae altitudinem producta basi laminis\protect\index{Sachverzeichnis}{laminae politae} respondens est elevanda; in aere clauso aer vasis est comprimendus. Sed idem aer vasis in aere libero positus Elaterio\protect\index{Sachverzeichnis}{elaterium} suo renititur columna totius atmosphaerae\protect\index{Sachverzeichnis}{atmosphaera}, ultra compressurae, eadem ergo vi laminis\protect\index{Sachverzeichnis}{laminae politae} quoque divulsione compressuris obsistet. Et hoc adeo verum est, ut non tantum aer ordinarius vasi inclusus, sed et aer suctione summe attenuatus, qualis in Recipiente exhausto remanet (neque enim ad rem pertinet an illud corpus residuum appelles: aerem summe attenuatum, an corpus aere subtilius), idem praestet. Postquam enim ab alio aere, quippe exucto comprimi desiit, expandit sese, vi proprii Elateris\protect\index{Sachverzeichnis}{elater}: hinc fit}{\lemma{Divulsione}\Afootnote{enim laminarum\protect\index{Sachverzeichnis}{laminae politae} in aere libero columna aerea   \textbar\ per totam  \textit{ (1) }\ atmosphaeram\protect\index{Sachverzeichnis}{atmosphaera|textit} assurgens \textit{ (2) }\ atmosphaerae altitudinem producta \textit{ erg.}\ \textbar\  basi laminis\protect\index{Sachverzeichnis}{laminae politae} respondens est elevanda; in aere clauso  \textit{ (a) }\ columna aerea \textit{ (b) }\ altitudine vasis \textit{ (c) }\ aer [...] remanet  \textbar\ (neque [...] aere  \textbar\ ordinario \textit{ gestr.}\ \textbar\   subtilius) \textit{ erg.}\ \textbar\ , idem [...] fit \textit{ erg.} \textit{ L}}} 
quod adeo verum est, ut  corpus illud quod aere exhausto in Recipiente remanet,  cujuscunque tandem sit naturae, tantundem intus possit,  ad comprimendas laminas\protect\index{Sachverzeichnis}{laminae politae}, quantum totae columnae aereae gravitas\protect\index{Sachverzeichnis}{gravitas} antea potuerat. Ratio est quia illud ipsum  corpus antea ab ipso columnae aereae pondere in aere libero,  aut ab aeris Elaterio\protect\index{Sachverzeichnis}{elaterium} in vase clauso comprimebatur \edtext{et proinde nunc  liberatum}{\lemma{comprimebatur}\Afootnote{ \textit{ (1) }\ nunc vero liberatum \textit{ (2) }\ et proinde nunc  liberatum \textit{ L}}} tantam habet vim 
 Elasticam\protect\index{Sachverzeichnis}{vis!elastica} quanta vis comprimens fuerat, quare nunc liberatum