[9 v\textsuperscript{o}] et intelligi possunt singula foraminis puncta, vertices totidem conorum lucidorum, quorum bases circulares sunt in plano, hae bases, seu hi circuli lucidi, eo majores sunt, quo magis abest planum excipiens a foramine, ac proinde eo magis omnes simul in circulum unum degenerare videntur. \pend \pstart Haec est ratiocinatio Claudii Chalesii\protect\index{Namensregister}{\textso{Dechales} (Chalesius), Claude Fran\c{c}ois Milliet 1621\textendash 1678} quae mihi admodum ingeniosa \edtext{videtur. Videatur}{\lemma{videtur.}\Afootnote{ \textit{ (1) }\ Unde deducet etiam \textit{ (2) }\ Videatur \textit{ L}}} ejus \textit{Opticae} lib. 3. lemmate post prop. 14. item ipsa propositio 17.\edtext{}{\lemma{propositio 17.}\Bfootnote{\textsc{C. F. M. Dechales, }\cite{00124}\textit{Cursus seu mundus mathematicus}, Lyon 1674, S.~453. }} Unde notat si figura foraminis mutetur, vel etiam opacum aliquid in foramine suspendatur, non ideo mutat figuram solis. Idem contingit, si duo foramina sint sibi valde vicina. Semper autem supponendum est distantiam excipientis superficiei esse notabilem idem ostendit postea, etiam de figura non rotunda, ut si pars solis a corpore opaco, ut nube, horizonte, luna (in Eclipsi\protect\index{Sachverzeichnis}{eclipsis}) intercipiatur. Unde ingeniose porro deducit, cur pauciores sint digiti \edtext{eclipsati}{\lemma{}\Afootnote{eclipsati \textit{ erg.} \textit{ L}}} in solis radio per foramen excepto, quam in coelo, et maculae minores appareant, quam revera sunt, scilicet, quia non rotantur penumbrae, sed coincidentia tantum umbrarum. \pend 