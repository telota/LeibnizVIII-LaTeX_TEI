 \pagebreak \pstart [p.~86] [...] en apres tirez au del\`{a} de cette conduite, \`{a} autant de ses pieds loin d'elle, que vous voulez que la hauteur de l'oeil\protect\index{Sachverzeichnis}{oeil} en contienne, vne droite ZCX, qui luy soit paralelle; elle sera celle qu'on nomme communement, \textit{horisontale}, et M. D.\footnote{\textit{Leibniz erg\"{a}nzt} M. D. \textit{zu} Mons. Desargues} ligne du plan de l'oeil\protect\index{Sachverzeichnis}{oeil}\footnote{\textit{Am Rand gestrichen}: la figure est fautive}; Dauantage, menez des deux bouts \textit{q} et \textit{p}, duquel que vous voudrez des pieds de la conduite de front E\textit{l}GV, comme icy par exemple de celuy \textit{4}, au point qu'il vous plaira C, de la ligne horisontale ZCX, deux droites fuyantes \textit{q}C, \textit{p}C; elles vous regleront entr'elles deux, l'inegalit\'{e} continuelle qu'il doit y auoir entre les pieds de front de c\'{e}t exemple; c'est \`{a} dire qu'elles en forment l'eschelle des pieds de front:\footnote{\textit{Leibniz unterstreicht}: l'eschelle [...] front} [...].\pend