      
               
                \begin{ledgroupsized}[r]{120mm}
                \footnotesize 
                \pstart                
                \noindent\textbf{\"{U}berlieferung:}   
                \pend
                \end{ledgroupsized}
            
              
                            \begin{ledgroupsized}[r]{114mm}
                            \footnotesize 
                            \pstart \parindent -6mm
                            \makebox[6mm][l]{\textit{LiH}}Marginalien, An- und Unterstreichungen in \textsc{B. Cavalieri}, \cite{00021}\textit{Lo specchio ustorio}, Bologna 1650: Leibn. Marg. 163. \pend
                            \end{ledgroupsized}
                %\normalsize
                \vspace*{5mm}
                \begin{ledgroup}
                \footnotesize 
                \pstart
            \noindent\footnotesize{\textbf{Datierungsgr\"{u}nde}: Der Titel \cite{00021}\textit{Lo specchio ustorio} kommt in dem bislang publizierten Schrifttum Leibniz' nur ein einziges Mal vor. Es handelt sich dabei um N. 16, wo sich Leibniz einen Hinweis auf Cavalieri\protect\index{Namensregister}{\textso{Cavalieri} (Cavalierius), Bonaventura 1598\textendash 1647} notiert. Nimmt man das St\"{u}ck N. 15 hinzu, in dem sich Leibniz ebenfalls Ausz\"{u}ge anfertigt, die sich auf Brennspiegel beziehen. Diese \"{U}bereinstimmung gibt eine gewisse Wahrscheinlichkeit daf\"{u}r, dass die Marginalien um 1671 entstanden sind.}
                \pend
                \end{ledgroup}
            
                \vspace*{8mm}
                \pstart 
                \normalsize
           \selectlanguage{italian} \begin{center} [p.~22] Della terza\footnote{\textit{Leibniz unterstreicht:} terza. \textit{Dar\"{u}ber:} secunda potius Catoptrica proprietas} propriet\`{a} della Parabola.\\Cap. XI.\end{center} \pend \vspace{1.0ex}\pstart Sia la Parabola BAC nell'8.\ fig.\ il cui asse OA indiffinitamente prolongato verso A, come in X, e sia foco\protect\index{Sachverzeichnis}{fuoco} di detta Parabola il punto M, e da che parte si voglia fuori di essa\footnote{\textit{Leibniz unterstreicht:} fuori di essa} incontrino la superficie parabolica per essempio le rette linee TI, FK ne i punti I, K, le quali siano sempre per dritto al foco\protect\index{Sachverzeichnis}{fuoco} M; h\`{a} dunque la Parabola quest'altra mirabile propriet\`{a}, che dalli detti punti d'incidenza si partono le riflesse\protect\index{Sachverzeichnis}{riflesso} dalla Parabola per di fuori sempre parallele all'asse, cio\`{e} all'AO, le quali riflesse\protect\index{Sachverzeichnis}{riflesso} siano le IV, KY prolongate come si voglia in V, Y. Ne meno questa propriet\`{a} h\`{o} veduta in altri, se ben facilmente si dimostra, come hora s'intender\`{a}.\footnote{\textit{Darunter:} \`{e} coincidente in effetto, con la prima, e vero \`{e} un corollario della prima}\pend \newpage