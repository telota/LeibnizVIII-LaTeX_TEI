      
               
                \begin{ledgroupsized}[r]{120mm}
                \footnotesize 
                \pstart                
                \noindent\textbf{\"{U}berlieferung:}   
                \pend
                \end{ledgroupsized}
            
              
                            \begin{ledgroupsized}[r]{114mm}
                            \footnotesize 
                            \pstart \parindent -6mm
                            \makebox[6mm][l]{\textit{L}}Konzept: LH XXXVII 2 Bl. 13. 1 Bl. 8\textsuperscript{o}. 2 S. Oberer und rechter Rand von Bl. 13 r\textsuperscript{o} beschnitten. Auf derselben Seite oben links die Zeichnung \textit{[Fig. 1]}. Text umlaufend.\\Kein Eintrag in KK 1 oder Cc 2. \pend
                            \end{ledgroupsized}
                %\normalsize
                \vspace*{5mm}
                \begin{ledgroup}
                \footnotesize 
                \pstart
            \noindent\footnotesize{\textbf{Datierungsgr\"{u}nde}: Anhaltspunkte f\"{u}r die Datierung sind die Namen Trocut\protect\index{Namensregister}{\textso{Trocut}} und Schick\protect\index{Namensregister}{\textso{Schick,} Peter}. Von beiden werden sehr detailliert Ereignisse geschildert, die sich in Paris\protect\index{Ortsregister}{Paris (Parisii)} zugetragen haben. Dieser Zeitraum l\"{a}sst sich noch etwas genauer durch einen Brief an Melchior Friedrich v. Sch\"{o}nborn\protect\index{Namensregister}{\textso{Sch\"{o}nborn}, Melchior Friedrich v. 1644\textendash 1717|textit} eingrenzen, den Leibniz am 16. September 1673 in Paris\protect\index{Ortsregister}{Paris (Parisii)} verfasst hat. Darin heißt es, dass Herr Schick\protect\index{Namensregister}{\textso{Schick,} Peter} seine Zeit mit allerhand n\"{u}tzlichen Curiosit\"{a}ten zugebracht habe, wozu m\"{o}glicherweise auch der von Leibniz geschilderte Sachverhalt geh\"{o}rte. Wie u. a. aus den Briefen N. 217, 227, 229 und 237 in \textit{LSB} I, 1 hervorgeht, ist eine Besch\"{a}ftigung Schicks\protect\index{Namensregister}{\textso{Schick,} Peter} mit naturwissenschaftlichen Gegenst\"{a}nden vorher kaum anzunehmen, so dass wir die Entstehung unseres St\"{u}cks auf den Zeitraum zwischen Mitte 1673 und Ende 1676 datieren.}
                \pend
                \end{ledgroup}
            
                \vspace*{8mm}
                \pstart 
                \normalsize
[13 r\textsuperscript{o}] \selectlanguage{french}Les vitres des Eglises paroissent d'un beau rouge dans l'Eglise et ne te paroissent pas tant \`{a} ceux qui les regardent dehors en \textit{B} dont voicy la raison. Le rayon \textit{CD} \edtext{tombe sur}{\lemma{\textit{CD}}\Afootnote{ \textit{ (1) }\ passant par \textit{ (2) }\ tombant par \textit{ (3) }\ tombe sur \textit{ L}}} le verre rouge \edtext{\textit{LM}  il en reflechit }{\lemma{rouge}\Afootnote{ \textit{ (1) }\ \textit{DE} en fait ref \textit{ (2) }\ \textit{LM}  \textit{(a)}\ fait re \textit{(b)}\  il en reflechit  \textit{ L}}}un rayon foible \edtext{\textit{Df}}{\lemma{\textit{Df}}\Afootnote{ \textbar\ ce \textit{ gestr.}\ \textbar\ qui \textit{ L}}} qui ne prend point de couleur, le principal du rayon \textit{DE} penetre le verre \edtext{et en prend la couleur}{\lemma{}\Afootnote{et en prend la couleur \textit{ erg.} \textit{ L}}}, et rencontrant la surface interieure \textit{NM} une partie assez foible, mais color\'{e}e est reflechi\'{e} et fait le rayon [\textit{Eg}]\edtext{}{\Afootnote{\textit{EB}\textit{\ L \"{a}ndert Hrsg. } }}; mais la plus considerable passe et fait le rayon \textit{EA}. \edtext{Et comme ceux qui sont dans l'Eglise voyent}{\lemma{\textit{EA.}}\Afootnote{ \textit{ (1) }\ C'est pourquoy on voit \textit{ (2) }\ Et [...] voyent \textit{ L}}} le rouge du verre par le moyen du rayon \textit{EA}, et ceux qui sont hors de l'Eglise par le moyen du rayon [\textit{Eg}]\edtext{}{\Afootnote{\textit{EB}\textit{\ L \"{a}ndert Hrsg. } }}, on voit bien pourquoy. Le rouge paroist plus foible, quand il est regard\'{e} hors de l'Eglise. \pend \pstart  Les rubis les emeraudes et les autres pierres pretieuses color\'{e}es font paroistre la couleur bien plus fortement par reflexion\protect\index{Sachverzeichnis}{r\'{e}flexion} que les verres parce que la proportion de la refraction\protect\index{Sachverzeichnis}{r\'{e}fraction} est plus grande dans ces pierres que dans le verre.\pend \pstart  
   \begin{wrapfigure}{l}{0.4\textwidth}
   \includegraphics[width=0.4\textwidth]
%Zeitz   {images/37_2_13r1}
{images/37_2_13r}
   \\\begin{center}\textit{[Fig. 1]}\end{center}
   \end{wrapfigure}
Si on suppose que cette proportion soit comme de 5 \`{a} 3 dans le rubis on $\langle$tr$\rangle$ ouvera par le calcul que le rayon le plus oblique qui pourra passer du dedans d'un rubis dans l'air fera un angle d'incidence\protect\index{Sachverzeichnis}{angle!d'incidence}  de 36 d. 53 m. et que si cet angle est de 36 d. 54 m. le rayon se reflechira entierement comme il fait dans le verre quand cet angle est de 41 d. 49 m. On peut donc tailler un rubis d'une maniere que la pluspart de ces rayons qui y entreront se reflechiront entierement sur les secondes surfaces et prendre une vivacit\'{e} de couleurs par le double passage qu'ils feront \`{a} travers la matiere color\'{e}e, ce qui n'arrivera pas \`{a} un verre color\'{e} taill\'{e} de même parce que sa refraction\protect\index{Sachverzeichnis}{r\'{e}fraction} estant moins forte, il laissera passer bien plus de rayons. C'est par cette raison qu'on met des feuilles d'argent bruni teintes d'un beau rouge au dessous des rubis afin de faire repasser par les yeux le reste de la lumiere qui les a traverses. On