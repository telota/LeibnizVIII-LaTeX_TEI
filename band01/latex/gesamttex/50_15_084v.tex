\pstart[84 v\textsuperscript{o}] Ad quod demonstrandum factum sic hoc \edtext{[schema]}{\lemma{scema}\Afootnote{\textit{\"{a}ndert Hrsg.}}} in quo \textit{N}, sit Centrum Circuli \textit{NDB}; \textit{ND} semidiameter; \textit{BF}, perpendicularis ad \textit{DN}; \textit{BC} tangens circulum in \textit{B}, \textit{LBNG} recta ad quam \textit{IG} et \textit{AL} perpendiculares sunt; \textit{AB} parallelae rectae \textit{DNI}. Si jam \edtext{\textit{AB}}{\lemma{jam}\Afootnote{ \textit{ (1) }\ \textit{AD} \textit{ (2) }\ \textit{AB} \textit{ L}}} sumatur pro quolibet  radiorum per aerem transeuntium, et in \textit{B} in vitrum  circularis hujus figurae incidentium, calculo inveniendum  est; primo, punctum illud in producta diametro versus quod  radius iste refractus tendit; quod si pro eo statuatur \textit{I}, invenienda est longitudo lineae \textit{NL}. Quae ut inveniatur,  sit $BN\: ^{\rotatebox{180}{$\propto$}} \:1 ; BF\: ^{\rotatebox{180}{$\propto$}} \:x ; NI\: ^{\rotatebox{180}{$\propto$}}\: z ; AL\: ^{\rotatebox{180}{$\propto$}}\: y ; AB\: ^{\rotatebox{180}{$\propto$}}\: BI$.
Tam  porro manifestum est triangula \textit{ALB}, \textit{BFN}, \textit{IGN}, proportionalia esse, ac propterea \textit{BF}, \textit{x}, habere eandem rationem  ad \textit{BN}, \textit{I}, ut \textit{AL}, \textit{y}; ad \textit{AB}, aut \textit{BI}, $\displaystyle \frac{y}{x}$\rule[-4mm]{0mm}{10mm}; ergo quadratum  super $\displaystyle BI\: ^{\rotatebox{180}{$\propto$}}\frac{yy}{xx}$, unde subtractum quadratum super \textit{BF},\rule[-4mm]{0mm}{10mm} relinquitur quadratum super $\displaystyle IF \:^{\rotatebox{180}{$\propto$}}\frac{yy}{xx}-xx$, ergo $\displaystyle IF \:^{\rotatebox{180}{$\propto$}}\sqrt{\frac{yy}{xx}-xx}$\rule[-4mm]{0mm}{10mm}, unde subtrahatur $FN\: ^{\rotatebox{180}{$\propto$}}\sqrt{1-xx}$, relinquetur $\displaystyle NI \:^{\rotatebox{180}{$\propto$}}Z \:^{\rotatebox{180}{$\propto$}} \sqrt{\frac{yy}{xx}-xx}-\sqrt{1-xx}$.\rule[-4mm]{0mm}{10mm} Porro \textit{BN}, 1, est ad \textit{BF}, \textit{x}, ut \textit{NI}, \textit{z} ad $IG\: ^{\rotatebox{180}{$\propto$}} xz$. \rule[0mm]{0mm}{5mm} Cum itaque ratio \textit{AL} ad \textit{IG} sit communis mensura refractionis  omnium radiorum, \edtext{ut apparet ex secundo capite Dioptrices  praedicti Do\textsuperscript{ni} Descartes\protect\index{Namensregister}{\textso{Descartes} (Cartesius, des Cartes, Cartes.), Ren\'{e} 1596\textendash 1650}; cognita refractionum vitri}{\lemma{Descartes}\Bfootnote{\textsc{R. Descartes}, \cite{00038}\textit{La dioptrique}, Leiden 1637, S.~21 (\textit{DO} VI, S.~101).}}