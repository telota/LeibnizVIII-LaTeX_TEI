[55 r\textsuperscript{o}] Esto navis\protect\index{Sachverzeichnis}{navis} \textit{AB} vel \textit{Ab} 
\edtext{Polus}{\lemma{\textit{Ab}}\Afootnote{ \textit{ (1) }\ in qua \textit{ (2) }\ Polus \textit{ L}}} \textit{C} aut ultra in recta \textit{AC} quantum satis producta. Cogitetur ea navis\protect\index{Sachverzeichnis}{navis} fluctibus jactari utcunque, motus autem centrum esse \edtext{unicum}{\lemma{esse}\Afootnote{ \textit{ (1) }\ \textit{A} \textit{ (2) }\ unicum \textit{ L}}} circa quod \edtext{immotum}{\lemma{}\Afootnote{immotum \textit{ erg.} \textit{ L}}} inter tot jactationes navim\protect\index{Sachverzeichnis}{navis} quoties cursum flectat, se circumagere necesse sit, et ei centro centrum verticitatis acus\protect\index{Sachverzeichnis}{acus!magnetica} aut magnetis\protect\index{Sachverzeichnis}{magnes}, coincidere, aut certe utrumque centrum verticitatis navis\protect\index{Sachverzeichnis}{navis} aut magnetis\protect\index{Sachverzeichnis}{magnes} incidere in rectam eandem horizonti perpendicularem, velut axem communem. Quod \edtext{punctum ponatur esse ubilibet}{\lemma{Quod}\Afootnote{ \textit{ (1) }\ quia \textit{ (2) }\ punctum \textit{(a)}\ quia ubilibet \textit{(b)}\ ponatur esse ubilibet \textit{ L}}}, in \textit{A}. Et \edtext{acus}{\lemma{Et}\Afootnote{ \textit{ (1) }\ ponatur momento \textit{ (2) }\ acus \textit{ L}}} magnetica \protect\index{Sachverzeichnis}{acus!magnetica} \edtext{centro \textit{A}}{\lemma{magnetica}\Afootnote{ \textit{ (1) }\ in \textit{A} \textit{ (2) }\ centro \textit{A} \textit{ L}}} ponatur respicere polum\protect\index{Sachverzeichnis}{polus}, \edtext{vel praecise}{\lemma{polum,}\Afootnote{ \textit{ (1) }\ vel quod \textit{ (2) }\ vel praecise \textit{ L}}} vel qualibet declinatione\protect\index{Sachverzeichnis}{declinatio}, versus \textit{C} neque enim hoc loco refert polum\protect\index{Sachverzeichnis}{polus}, praecise an aliud vicinum ei punctum respiciat acus\protect\index{Sachverzeichnis}{acus!magnetica} cum \edtext{sufficiat}{\lemma{cum}\Afootnote{ \textit{ (1) }\ respiciat \textit{ (2) }\ sufficiat \textit{ L}}} saltem eam momento dato determinatum aliquod mundi punctum respicere: Manifestum est eo momento quo navis\protect\index{Sachverzeichnis}{navis} centro \textit{A} ex situ \textit{AB} \edtext{vertitur}{\lemma{\textit{AB}}\Afootnote{ \textit{ (1) }\ transit \textit{ (2) }\ vertitur \textit{ L}}} in situm \textit{Ab} directionem magnetis\protect\index{Sachverzeichnis}{magnes} aut acus\protect\index{Sachverzeichnis}{acus!magnetica} designare \textso{angulum flexionis}\protect\index{Sachverzeichnis}{angulus!flexionis}\edtext{}{\lemma{\textso{angulum flexionis}}\Afootnote{\textit{doppelt unterstrichen}}}, cum enim \textso{angulus directionis}\protect\index{Sachverzeichnis}{angulus!directionis}\edtext{}{\lemma{\textso{angulus directionis}}\Afootnote{\textit{doppelt unterstrichen}}} antea fuerit \textit{CAB} nunc est \textit{CAb} ac proinde angulus flexionis\protect\index{Sachverzeichnis}{angulus!flexionis} \edtext{\textit{BAb}}{\lemma{}\Afootnote{\textit{BAb} \textit{ erg.} \textit{ L}}} est differentia angulorum directionis\protect\index{Sachverzeichnis}{angulus!directionis}. \edtext{Angulum}{\lemma{directionis.}\Afootnote{ \textit{ (1) }\ Si vero \textit{ (2) }\ Angulum \textit{ L}}} directionis\protect\index{Sachverzeichnis}{angulus!directionis} voco qui componitur ex linea motus navis\protect\index{Sachverzeichnis}{navis}, \edtext{}{\lemma{}\Afootnote{navis, \textbar\ et \textit{ gestr.}\ \textbar\ [AB, \textit{ L}}}\edtext{[\textit{AB}, vel \textit{Ab}]}{\lemma{\textit{CAB}, vel \textit{CAb}}\Afootnote{\textit{\ L \"{a}ndert Hrsg. }}} et linea directionis \textit{AC} seu quae ducitur a puncto verticitatis magneticae ad punctum mundi quod respicit magnes\protect\index{Sachverzeichnis}{magnes}. \edtext{Posito autem Angulum flexionis\protect\index{Sachverzeichnis}{angulus!flexionis} esse differentiam Angulorum directionis\protect\index{Sachverzeichnis}{angulus!directionis}, potest instrumento designatus haberi, quia anguli directionis\protect\index{Sachverzeichnis}{angulus!directionis} ipsi se designant, magnete\protect\index{Sachverzeichnis}{magnes} flexa utcunque navi\protect\index{Sachverzeichnis}{navis} situm suum retinente, ac proinde ab iis navibus\protect\index{Sachverzeichnis}{navis}, partibus quas antea respiciebat abeunte, quod in ipsa nave\protect\index{Sachverzeichnis}{navis} sentiri, et vel ab homine vel ab instrumento notari potest. De modo autem notandi postea.}{\lemma{}\Afootnote{Posito [...] notandi \textit{ (1) }\ mox \textit{ (2) }\ postea. \textit{ erg.} \textit{ L}}} Sed quoniam navis\protect\index{Sachverzeichnis}{navis} non habet unum centrum verticitatis, ut magnes\protect\index{Sachverzeichnis}{magnes}; sed varie jactatur in 