\pend \pstart [p.~102] [...] hinc si  vel semel vnus radius reflectatur, infinities reflectetur:  vt autem luminis solaris\protect\index{Sachverzeichnis}{lumen!solaris} vis multiplicetur, collectis huiusmodi radiis opera speculi parabolici\protect\index{Sachverzeichnis}{speculum!parabolicum}, ita hoc statuatur,  vt E v.g. sit communis vtriusque focus\protect\index{Sachverzeichnis}{focus}, vis luminis  multiplicabitur in E per repetitam reflexionem\protect\index{Sachverzeichnis}{reflexio} in conuexo, reflexio\protect\index{Sachverzeichnis}{reflexio} facile habetur, vt in conuexo\footnote{\textit{Leibniz unterstreicht}: vt autem [...] in conuexo} circuli; illud tantum singulare, quod si directus tendit ad alterum focum\protect\index{Sachverzeichnis}{focus}, ab altero reflexus directe procedere videtur; sic  NG reflectitur in GT; vtrum vero ad reflectendum sonum\protect\index{Sachverzeichnis}{sonus}, seu vocem aeque aptum sit hoc speculi\protect\index{Sachverzeichnis}{speculum} genus,  de sono\protect\index{Sachverzeichnis}{sonus} minime articulato concederem vltro, de voce  articulata omnino negarem\footnote{\textit{Leibniz unterstreicht}: de sono [...] negarem}, et perspicuum est; quia  omnes articulationes confunduntur in E.