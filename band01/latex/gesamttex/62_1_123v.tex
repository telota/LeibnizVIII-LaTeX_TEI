[123 v\textsuperscript{o}] H. Tenzel\protect\index{Namensregister}{\textso{Tenzel}, ()} hat commission von Dreßden\protect\index{Ortsregister}{Dresden}  gehabt bey seiner Reise nach Gotha\protect\index{Ortsregister}{Gotha} und Arnstad\protect\index{Ortsregister}{Arnstadt} wegen der goldw\"{a}scher in der Schwarze\protect\index{Ortsregister}{Schwartza} erkundigung einzuziehen\edtext{. Es}{\lemma{einzuziehen}\Afootnote{ \textit{ (1) }\ , hat einen \textit{ (2) }\  . Es \textit{ L}}} cessiret iezo die goldwascherey, weil der Graf von Arnstadt\protect\index{Namensregister}{\textso{Christian G\"{u}nther II.}, Graf von Schwarzburg-Arnstadt (1616\textendash 1666)} sie seinen Vettern 
 von Rudolstad\protect\index{Ortsregister}{Rudolstadt}\protect\index{Namensregister}{\textso{Carl G\"{u}nther}, Graf von Schwarzburg-Rudolstadt (1576\textendash 1630)} \protect\index{Namensregister}{\textso{Ludwig G\"{u}nther}, Graf von Schwarzburg-Rudolstadt (1581\textendash 1646)} \protect\index{Namensregister}{\textso{Albrecht G\"{u}nther}, Graf von Schwarzburg-Rudolstadt (1582\textendash 1634)} nicht allein laßen, sondern ein drittheil davon haben will. Es ist ein bauer alda, welcher saget er \edtext{wolle}{\lemma{wolle}\Afootnote{ \textit{ (1) }\ mit den f\"{u}ßen \textit{ (2) }\ gleich [...] f\"{u}ßen \textit{ L}}} gleich wißen ob goldsand, wenn er mit bloßen f\"{u}ßen im waßer gehet, habe es vor 10 thlr. gelernet, praetendirt auch so viel. Es ist auch ein Mann alda, welcher versichert, er wolle den Sand mit Nuzen waschen, wenn nur vor 3 pfennig gold sich im Zentner finde. Es ist noch ein dritter so zuvor in ChurSachs. diensten geweßen, welcher der will bey Dreßden\protect\index{Ortsregister}{Dresden} ein reiches \edtext{goldwerck}{\lemma{reiches}\Afootnote{ \textit{ (1) }\ bergwerck \textit{ (2) }\ goldwerck \textit{ L}}} angeben. Die beyden leztern d\"{u}rffen aus Landen gehen, weil sie dienste suchen. Es hat H. Tenzelius\protect\index{Namensregister}{\textso{Tenzel} (H. Tenzelius), } deswegen eine relation an den H. v. Tsch.\protect\index{Namensregister}{\textso{Tschirnhaus} (H.v.Tch., H.v.Tsch.), Ehrenfried Walther v. 1651\textendash 1708} geschicket umb es dem Gh. Stadhalter zu uberreichen, es ist aber nichts darauff resolvirt worden.\selectlanguage{latin}\pend \pstart \edtext{}{\lemma{worden.}\Afootnote{ \textbar\ Cartesius\protect\index{Namensregister}{\textso{Descartes} (Cartesius, des Cartes, Cartes.), Ren\'{e} 1596\textendash 1650} praeter mentem suam in methodo dixit animalia \textit{ gestr.}\ \textbar\ Wenn \textit{ L}}} \selectlanguage{ngerman}Wenn die gl\"{a}ßer lange im tiegel  gehalten werden, etliche Wochen, so werden sie immer clarer und bekommen endtlich eine  vollkommene clarheit, alle blasen sezen sich. Die wellen sind etwas besonders, \edtext{welche schwehr gewesen}{\lemma{besonders,}\Afootnote{ \textit{ (1) }\ solche  hindern z \textit{ (2) }\ welche schwehr gewesen \textit{ L}}} wegzunehmen  man siehet sie am besten am liecht in puncto confusionis oder foco\protect\index{Sachverzeichnis}{focus}. Sie schaden zwar nichts bey den objectivis\protect\index{Sachverzeichnis}{objectivum} der Telescopiorum\protect\index{Sachverzeichnis}{telescopium}, denn wenn da nur das ocular\protect\index{Sachverzeichnis}{ocular} guth, werden alle vitia corrigiret. H. Titre\protect\index{Namensregister}{\textso{Titre,} } seel. sagte mit einem guthen perspectiv m\"{u}ße man durch einen thaler sehen, das ist ein thaler hindert nicht daß die imaginum praevalentia vom objecto sich zeige. Also hindern auch wellen und blasen wenig. Allein bey Microscopiis\protect\index{Sachverzeichnis}{microscopium} hindern  die wellen. Weil sich nun bey denen  gemeinen gr\"{u}nen gl\"{a}sern keine wellen  finden, so hat solches gelegenheit geben der  sach nach zu dencken, also das man es endtlich gefunden. \edtext{gefunden.}{\lemma{gefunden.}\Afootnote{ \textit{ (1) }\ Die tafeln werden doch ein baar \textit{ (2) }\ Die \textit{(a)}\ perspectiv \textit{(b)}\ objectiv\protect\index{Sachverzeichnis}{objectivum|textit} \textit{ (3) }\ Die [...] baar \textit{ L}}} Die vitra aus tafeln werden doch ein baar pollices dick seyn m\"{u}ßen\edtext{. Wenn}{\lemma{m\"{u}ßen}\Afootnote{ \textit{ (1) }\ und \textit{ (2) }\ . Wenn \textit{ L}}} sie groß seyn sollen, sie auch eine gewiße  starcke von nothen haben auch die curvatura etwas  mit sich bringet. \pend \pstart  Ein Stab eisen wird in etlichen schlagen gl\"{u}end  wenn man auf das Eck schl\"{a}get bald zur rechten  bald zur lincken seiten. Allein das Eisen  muß praepariret seyn, sonst gehet es nicht an.  Nehmlich man muß ihn zuvor in kasten gestecket und mit ihn haben erkalten laßen  so wird er \edtext{}{\lemma{}\Afootnote{er  \textbar\ gleichsam wie \textit{ gestr.}\ \textbar\ stahl. \textit{ L}}} stahl. Wie denn  auch die caementation mit Kohlen den stahl gibt  doch muß man $\langle - \rangle $ nehmlich vor Eisen dazu nehmen. Die H\"{a}user so zu bauen daß sie nicht leicht brenen  wie dann zu Paris\protect\index{Ortsregister}{Paris (Parisii)} die Incendia holzarm, nehmlich  kein holz zu den boden, decken und sonst  zu nehmen, sondern haben Estriche kalck mit ziegeln ist sonderlich treflich gegen feuer, auch ein holzern haus wird dadurch nicht leicht brenen  etwa nur der giebel und alles sich salviren laßen. Zu Paris\protect\index{Ortsregister}{Paris (Parisii)} sieht man wie die treppen mit kalck und ziegeln, \edtext{}{\lemma{}\Afootnote{ziegeln, \textbar\ also \textit{ gestr.}\ \textbar\ in \textit{ L}}} in einen holzern selbst ist dergestalt leicht alles zu salviren weil die Estriche dem  feuer wiederstehen. \edtext{$\langle - \rangle $ selbst wiederstehet}{\lemma{wiederstehen.}\Afootnote{ \textit{ (1) }\ Estriche selbst wiederstehen \textit{ (2) }\ $\langle - \rangle$ selbst wiederstehet \textit{ L}}}  den feuer. Daher die Salzkothen nicht leicht brennen. D. Richter\protect\index{Namensregister}{\textso{Richter,} D.} ein guther Chymicus hat stucke  6pfundig so nicht $\langle - \rangle $, haben ihr zundloch $\langle - \rangle \langle - \rangle $\selectlanguage{latin}\pend 