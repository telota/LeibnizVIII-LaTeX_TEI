\pend \pstart [p.~153] [...] poterit etiam obiectum vltra augeri, si secunda lens\protect\index{Sachverzeichnis}{lens} CP tantulum remoueatur ab obiectiua\protect\index{Sachverzeichnis}{objectivum} A; item EF, a DC, vt supra dictum est; tunc enim crescit ratio distantiarum; ac proinde iuxta regulam supra traditam, obiectum augetur; vnde concludo, hoc microscopij\protect\index{Sachverzeichnis}{microscopium} genus optimum esse, et primum, saltem a me, visum, fabricatum fuisse Augustae Vindelicorum\protect\index{Ortsregister}{Augsburg (Augusta Vindelicorum)}; illius autem copia mihi facta est a clarissimo viro, meique amantissimo et omnium literatorum amore, et cultu dignissimo D. de Monconis\protect\index{Namensregister}{\textso{Monconys} (D. de Monconis), Balthasar de 1611\textendash 1665},\footnote{\textit{Leibniz unterstreicht:} D. de Monconis\\ \textit{Am Rand angestrichen:} obiectiua A [...] Monconis} quem hic honoris et grati animi ergo, post amara illius fata, deplorari potius, quam appellari a me par fuit.\pend \pstart  VIII. Duo non sunt omittenda ad rem hanc pertinentia primum est, lentem obiectiuam\protect\index{Sachverzeichnis}{lens!objectiva} paulo molliorem, id est, maioris sphaerae esse debere, vt obiectum sub minore quidem mole, sed cum maiore campo, vt aiunt, videatur; cuius ratio ex dictis facile intelligitur; simili autem telescopio\protect\index{Sachverzeichnis}{telescopium} Diuinius\protect\index{Namensregister}{\textso{Divini} (Divinius), Eustachio 1610\textendash 1685} noster vti solet, ad legendas attritorum numismatum inscriptiones\footnote{\textit{Leibniz unterstreicht}: simili autem [...] inscriptiones\\
\textit{Am Rand mit Tinte}: Posci solet lens\protect\index{Sachverzeichnis}{lens} n$\langle$---$\rangle$ si satis haberi p$\langle$---$\rangle$ campi}: alterum est statui posse vitrum\protect\index{Sachverzeichnis}{vitrum} cavum in locum lentis\protect\index{Sachverzeichnis}{lens} ocularis, sed tubum contrahendum esse, tunc autem res huius microscopij\protect\index{Sachverzeichnis}{microscopium} ad telescopium\protect\index{Sachverzeichnis}{telescopium} reducitur\footnote{\textit{Leibniz unterstreicht}: tunc [...] reducitur}; et eodem modo \edtext{demonstratur.}{\lemma{p$\langle$---$\rangle$}\Afootnote{ \textit{ (1) }\ agrorum \textit{ (2) }\ campi \textit{ L}}}