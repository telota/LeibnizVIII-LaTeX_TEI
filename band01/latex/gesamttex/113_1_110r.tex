      
               
                \begin{ledgroupsized}[r]{120mm}
                \footnotesize 
                \pstart                
                \noindent\textbf{\"{U}berlieferung:}   
                \pend
                \end{ledgroupsized}
             @@@ Keine Article Tradition gefunden. @@@
                \vspace*{8mm}
                \pstart 
                \normalsize
            [110 r\textsuperscript{o}] De legibus refractionis\protect\index{Sachverzeichnis}{lex!refractionis} \pend \pstart @@@ G R A F I K @@@% \begin{wrapfigure}{l}{0.4\textwidth}                    
                %\includegraphics[width=0.4\textwidth]{../images/De+legibus+refractionis/LH037%2C02_110r/files/100048.png}
                        %\caption{Bildbeschreibung}
                        %\end{wrapfigure}
                        %@ @ @ Dies ist eine Abstandszeile - fuer den Fall, dass mehrere figures hintereinander kommen, ohne dass dazwischen laengerer Text steht. Dies kann zu einer Fahlermeldung fuehren. @ @ @ \\
                     Ante omnia constat radium perpendicularem non refringi. \pend \pstart  Videamus, an \edtext{sumendo}{\lemma{an}\Afootnote{ \textit{ (1) }\ ponend \textit{ (2) }\ sumendo \textit{ L}}} quod experientia certum  est, refractionem\protect\index{Sachverzeichnis}{refractio} ex aere in aquam esse ad  perpendicularem, concludi possit quaenam sit lex refractionis\protect\index{Sachverzeichnis}{lex!refractionis}. \edtext{\textso{Primum dispiciamusquomodo fiat refractio in angulo  videri posset eam perinde  fieri ac si esset}}{\lemma{?LEMMA?:refractionis.}\Afootnote{ \textit{ (1) }\ \textso{Ante omnia pono }\textso{refractionem}\protect\index{Sachverzeichnis}{refractio|textit}\textso{ ad angulos rectos esse nullam,  deinde} \textit{ (2) }\ \textso{Ante omnia pono fig} \textit{ (3) }\ \textso{ }\textso{Refractionem}\protect\index{Sachverzeichnis}{refractio|textit}\textso{ in angulo perinde fieri ac si fieret} \textit{ (4) }\ \textso{Primum dispiciamus} \textit{(a)}\ \textso{an verum sit} \textit{(b)}\ \textso{quomodo [...] esset} \textit{ L}}}\textso{ ad rectam angulum bisecanti perpendicularem.  Sit }\edtext{\textso{in}}{\lemma{Sit}\Afootnote{ \textit{ (1) }\ \textso{angulus} \textit{ (2) }\ \textso{in} \textit{ L}}}\textso{ fig. 1. }@@@ G R A F I K @@@% \begin{wrapfigure}{l}{0.4\textwidth}                    
                %\includegraphics[width=0.4\textwidth]{../images/De+legibus+refractionis/LH037%2C02_110r/files/100116.png}
                        %\caption{Bildbeschreibung}
                        %\end{wrapfigure}
                        %@ @ @ Dies ist eine Abstandszeile - fuer den Fall, dass mehrere figures hintereinander kommen, ohne dass dazwischen laengerer Text steht. Dies kann zu einer Fahlermeldung fuehren. @ @ @ \\
                    \textso{ prisma vitreum, }\edtext{\textso{ cujus sectio axi  perpendicularis sit }\textit{\textso{ABC}}\textso{ et in eo plano radius }\textit{\textso{RB}}}{\lemma{vitreum,}\Afootnote{ \textit{ (1) }\ \textso{cujus sectio axi perpendicularis} \textit{(a)}\ \textso{sit tria} \textit{(b)}\ \textso{facta a plano radii }\textit{\textso{RB}} \textit{ (2) }\ \textso{cujus [...] \textit{RB}} \textit{ L}}}\textso{ occurrat }\edtext{\textso{triangulo}}{\lemma{occurrat}\Afootnote{ \textit{ (1) }\ \textso{an} \textit{ (2) }\ \textso{triangulo} \textit{ L}}}\textso{ in ipso angulo }\textit{\textso{B}}\textso{, ajo }\textso{refractionem}\protect\index{Sachverzeichnis}{refractio}\textso{ perinde fieri  ac si esset ad rectam }\textit{\textso{DBE}}\textso{ quae sit perpendicularis ad }\textit{\textso{BF}}\textso{, angulum }\textit{\textso{ABC}}\textso{ bisecantem.}\pend \pstart \edtext{\textso{Sed}}{\lemma{?LEMMA?:bisecantem.}\Afootnote{ \textit{ (1) }\ \textso{Hoc} \textit{ (2) }\ \textso{Sed} \textit{ L}}}\textso{ jam video  id esse falsum. Ponatur enim }\textit{\textso{RBA}}\textso{ esse situ in directum, et }\textit{\textso{ABC}}\textso{ esse angulum rectum, utique nulla ipsius radii }\textit{\textso{RB}}\textso{  fiet }\textso{refractio}\protect\index{Sachverzeichnis}{refractio}\textso{, quae tamen utique contingeret, si consideraretur radius ut perpendicularis ad }\textit{\textso{DBE}}\textso{. Itaque quaerenda }\textso{refractio}\protect\index{Sachverzeichnis}{refractio}\textso{  tum secundum rectam }\textit{\textso{CB}}\textso{ tum secundum rectam }\textit{\textso{AB}}\textso{, angulusque bisecandus}.\pend \pstart \edtext{Sit}{\lemma{?LEMMA?:bisecandus.}\Afootnote{ \textit{ (1) }\ Ante omnia autem constat radium perpendicularem non refringi  \textit{ (2) }\ Sit \textit{ L}}} porro in fig. 2. @@@ G R A F I K @@@% \begin{wrapfigure}{l}{0.4\textwidth}                    
                %\includegraphics[width=0.4\textwidth]{../images/De+legibus+refractionis/LH037%2C02_110r/files/100217.gif}
                        %\caption{Bildbeschreibung}
                        %\end{wrapfigure}
                        %@ @ @ Dies ist eine Abstandszeile - fuer den Fall, dass mehrere figures hintereinander kommen, ohne dass dazwischen laengerer Text steht. Dies kann zu einer Fahlermeldung fuehren. @ @ @ \\
                     circulus descriptus centro \textit{A} radio \textit{AB}.  Ponatur semicirculo \textit{BDC} dari vis radium \textit{FA} refringendi, quae  vis sit ut \textit{a} et praeterea adhuc dari semicirculo \textit{DCE}\textit{ vim refringendi ut }\textit{b}\textit{ erit portioni communi seu sectori }\textit{ADC}\textit{ data }\textit{vis refringendi}\protect\index{Sachverzeichnis}{vis!refringendi}\textit{ ut }\textit{a+b.}\pend \pstart \textit{Refractio}\protect\index{Sachverzeichnis}{refractio}\textit{ ita fiet ut primum quaeramus }\textit{ quae sit }\textit{refractio}\protect\index{Sachverzeichnis}{refractio}\textit{ secundum separatricem }\textit{BC}\textit{ et resistentiam }\textit{a}\textit{ deinde quae secundum separatricem }\textit{DE}\textit{ et resistentiam }\textit{b}\textit{ Angulus }\textit{ inventus bisecetur.}\pend \pstart \textit{Quod si jam ponamus radium }\textit{ repercussum perpendiculariter eadem via redire qua venit }\textit{ considerationem aliquam hinc provenire necesse est. }\pend \pstart \textit{  Ponatur ex medio }\textit{BFE}\textit{ in medium }\textit{BDC}\textit{ cognita }\textit{refractio}\protect\index{Sachverzeichnis}{refractio}\textit{, et secundum }\textit{ eam seu secundum resistentiam }\textit{a}\textit{ (b.) } \textit{ (1) }\ \textit{refractionem}\protect\index{Sachverzeichnis}{refractio|textit} \textit{ (2) }\ \textit{radium} \textit{ L}}}\textit{FA}\textit{ iri refractum in }\textit{AG}\textit{ (}\textit{AH}\textit{) ergo bisecto angulo }\textit{GAH}\textit{ per }\textit{AL}\textit{ erit }\textit{ radius } \textit{ (1) }\ \textit{refractus }\textit{GAH} \textit{ (2) }\ \textit{FA}\textit{ refractus} \textit{ L}}}\textit{ in }\textit{AL}\textit{. Nunc rursus invertendo }\textit{ si radius }\textit{LA}\textit{ ponatur incidere in } \textit{ (1) }\ \textit{duas superficies} \textit{ (2) }\ \textit{duo }\textit{} \textit{ L}}}\textit{ media }\textit{DBE}\textit{ et }\textit{BEC}\textit{, } \textit{ (1) }\ \textit{illiusque flexio} \textit{ (2) }\ \textit{ }\textit{refringendi vis}\protect\index{Sachverzeichnis}{vis!refringendi|textit} \textit{ L}}}\textit{ (sed in contrariam }\textit{ priori partem) cujusque medii sit data, habebitur modo priori et rectae }\textit{AF}\textit{ ubi nota aliam plane esse relationem medii }\textit{DAC}\textit{ ad }\textit{DBE}@@@ G R A F I K @@@% \begin{wrapfigure}{l}{0.4\textwidth}                    
                %\includegraphics[width=0.4\textwidth]{../images/De+legibus+refractionis/LH037%2C02_110r/files/100396.png}
                        %\caption{Bildbeschreibung}
                        %\end{wrapfigure}
                        %@ @ @ Dies ist eine Abstandszeile - fuer den Fall, dass mehrere figures hintereinander kommen, ohne dass dazwischen laengerer Text steht. Dies kann zu einer Fahlermeldung fuehren. @ @ @ \\
                     \textit{ (1) }\ \textit{vel }\textit{}\textit{DE} \textit{ (2) }\ \textit{et} \textit{ L}}}\textit{BEC}\textit{; quam }\textit{BFE}\textit{ ad }\textit{BDC}\textit{ et }\textit{DCE}\textit{. }\pend \pstart \textit{Si jam plures adhibeantur }\textit{ hujusmodi radii puto aliquid hinc duci posse. }\pend \pstart \textit{ a }\textsuperscript{\textit{2}}\textit{.}\pend 