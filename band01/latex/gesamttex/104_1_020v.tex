[20~v\textsuperscript{o}] Sine declinatione\protect\index{Sachverzeichnis}{declinatio} acus\protect\index{Sachverzeichnis}{acus!magnetica}, quia declinante acu\protect\index{Sachverzeichnis}{acus!magnetica}, annulus ipse cum ea declinavit. Et differentia inter annulum et circulum monstrabit acus\protect\index{Sachverzeichnis}{acus!magnetica} declinationes\protect\index{Sachverzeichnis}{declinatio}. Ut \edtext{\textit{a} annulus modo moveatur, cum}{\lemma{\textit{a}}\Afootnote{ \textit{ (1) }\ Tabula moveatur modo in \textit{ (2) }\ annulus modo moveatur, cum \textit{ L}}} acu\protect\index{Sachverzeichnis}{acus!magnetica}, modo non, effici potest, vel si semper fortiter prematur ab acu\protect\index{Sachverzeichnis}{acus!magnetica}, sit \textit{a} connexio inter Tabulam et annulum, \edtext{ut annulus}{\lemma{annulum,}\Afootnote{ \textit{ (1) }\ talis qualem \textit{ (2) }\ ut annulus \textit{ L}}} possit ire sine \edtext{circulo versus \textit{c} non sine circulo}{\lemma{sine}\Afootnote{ \textit{ (1) }\ acu\protect\index{Sachverzeichnis}{acus!magnetica|textit} versus \textit{c} non sine acu\protect\index{Sachverzeichnis}{acus!magnetica|textit} \textit{ (2) }\ circulo [...] circulo \textit{ L}}} versus \textit{d}. Hujus rei non difficilis procuratio est. Alia etiam methodus esse potest in connexione acus\protect\index{Sachverzeichnis}{acus!magnetica} cum annulo, ut quando acus\protect\index{Sachverzeichnis}{acus!magnetica} movetur versus \textit{c} annulum abripiat, quippe tum styli extremitatem \edtext{annulo}{\lemma{extremitatem}\Afootnote{ \textit{ (1) }\ ab eo latere \textit{ (2) }\ annulo \textit{ L}}} (nonnihil \edtext{puncti dato}{\lemma{}\Afootnote{puncti dato \textit{ erg.} \textit{ L}}} rugoso) \edtext{ita}{\lemma{}\Afootnote{ita \textit{ erg.} \textit{ L}}} obvertens \edtext{ut flecti styli}{\lemma{obvertens}\Afootnote{ \textit{ (1) }\ inquam \textit{ (2) }\ ut flecti styli \textit{ L}}}, extremitas non possit. Sed quando acus\protect\index{Sachverzeichnis}{acus!magnetica} movetur versus \textit{d} obvertet aliam styli extremitatem flexibilem et ideo \edtext{annulum}{\lemma{ideo}\Afootnote{ \textit{ (1) }\ Tabulam \textit{ (2) }\ annulum \textit{ L}}} relinquet. Ideo styli extremitas debet esse flexilis in unam tantum partem et