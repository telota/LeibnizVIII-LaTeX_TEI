   
        
        \begin{ledgroupsized}[r]{120mm}
        \footnotesize 
        \pstart        
        \noindent\textbf{\"{U}berlieferung:}  
        \pend
        \end{ledgroupsized}
      
       
              \begin{ledgroupsized}[r]{114mm}
              \footnotesize 
              \pstart \parindent -6mm
              \makebox[6mm][l]{\textit{L}}Konzept: LH XXXV 15, 6 Bl. 51\textendash52. 1 Bog. 2\textsuperscript{o}. 4 S., zweispaltig. Linke Spalte fortlaufender Text. Bl. 51 v\textsuperscript{o} rechte Spalte Erg\"{a}nzungen. \\KK 1, Nr. 193 E \pend
              \end{ledgroupsized}
        \vspace*{8mm}
        \pstart 
        \normalsize
\begin{center}[51 r\textsuperscript{o}] Longitud. 4.\end{center}\pend \vspace{1.0ex} \pstart Si nulla ratione effici possit, ut res rei insistens ad sustentaculi gyrationem circa axem non gyretur. Ultimum est in magnete\protect\index{Sachverzeichnis}{magnes} refugium; et, si P. Grandamici \protect\index{Namensregister}{\textso{Grandami} (Grandamicus), Jacques SJ 1588\textendash 1672} inventum\edtext{}{\lemma{inventum}\Bfootnote{Vgl. \cite{00264}N. 2\protect\raisebox{-0.5ex}{\tiny{4}}, S.~\pageref{inventum}.}} verum est, quod et Nicolaus Zucchius\protect\index{Namensregister}{\textso{Zucchi} (Zucchius), Niccol\`{o} SJ 1586\textendash 1670}, et Athanasius Kircherus\protect\index{Namensregister}{\textso{Kircher} (Kircherus), Athanasius SJ 1602\textendash 1680}, et Gasp. Schottus\protect\index{Namensregister}{\textso{Schott} (Schottus), Caspar SJ 1608\textendash 1666} examinarunt et approbarunt aeque certum, et si subtilius; et in usu majorem attentionem requirens. Grandamicus\protect\index{Namensregister}{\textso{Grandami} (Grandamicus), Jacques SJ 1588\textendash 1672} igitur invenit: si Terrella polo\protect\index{Sachverzeichnis}{polus} alterutro suberi imponatur, et ita in aqua fluctuet, certum meridianum\protect\index{Sachverzeichnis}{meridianus} sine variatione compositurum ad meridianum\protect\index{Sachverzeichnis}{meridianus} \edtext{loci}{\lemma{meridianum}\Afootnote{ \textit{ (1) }\ mundi, et i \textit{ (2) }\ loci \textit{ L}}}. Eo posito omnis quae in tabula designatoria fiet a linea recta flexio calculo Loxodromiae\protect\index{Sachverzeichnis}{loxodromia} perfecte corrigi potest, quia cum variis Magnetis\protect\index{Sachverzeichnis}{magnes} declinationibus nihil amplius negotii est, nam si ex subere \edtext{circulari}{\lemma{}\Afootnote{circulari \textit{ erg.} \textit{ L}}} emineant acumina sursum, ea poterunt circummovere Tabulam designatoriam, et in eundem cum \edtext{terrella}{\lemma{cum}\Afootnote{ \textit{ (1) }\ magnete\protect\index{Sachverzeichnis}{magnes|textit} \textit{ (2) }\ terrella \textit{ L}}} disponere. Illa difficultas restat, quod Terrella in aqua \edtext{in navi}{\lemma{aqua}\Afootnote{ \textit{ (1) }\ ad praxin \textit{ (2) }\ in navi \textit{ L}}} librari commode constanterque non est, quia aqua ad quemlibet navis\protect\index{Sachverzeichnis}{navis} motum turbata situm terrellae et tabulae perpetuo turbabit. Puto tamen aliam librandi rationem non adeo difficilem fore, et fortasse simpliciter rem effici posse, si magnes\protect\index{Sachverzeichnis}{magnes} \edtext{vel levissimo suberi insistat, suber intra crassitiem suam stylum}{\lemma{magnes}\Afootnote{ \textit{ (1) }\ tenuissimae laminae insistat, ea stylum \textit{ (2) }\ vel [...] stylum \textit{ L}}} orthogonalem recipiat, ita tamen ut circa eum libere gyrari possit, et tamen \edtext{a stylo non}{\lemma{tamen}\Afootnote{ \textit{ (1) }\ a subere non \textit{ (2) }\ a stylo non \textit{ L}}} perforetur, sed illa styli quasi vagina intus ferro munita sit contra perforationem. Si tamen hoc ad gyrationem et librationem non sufficiat, accedat haec industria: Magnes\protect\index{Sachverzeichnis}{magnes} polo\protect\index{Sachverzeichnis}{polus} superiore suspendatur aliquo filo (quod consultissime mutatione praxium Kircherianarum fiet ex tenuissimis Aloes foliorum filamentis) et tamen insistat suberi, ita nimirum ut neque filum remittat, sed prorsus extendatur, et tamen \edtext{magnes\protect\index{Sachverzeichnis}{magnes} }{\lemma{}\Afootnote{magnes\protect\index{Sachverzeichnis}{magnes} \textit{ erg.} \textit{ L}}}suberi sit immediatus, et ita erit in medio insistentiae et pensionis, nec vel subere vel filo ablato vel posito magis descendet, vel ascendet. Hic status admirandae sane considerationis et nescio an hactenus satis observatus librationibus est aptissimus. Suberi autem insistat Terrella in ferrea aliqua suberis quasi patinula, ut tanto melius contorqueatur quam si simpliciter plano insistat, et ut sit in eo firma. Quomodocunque autem fiat libratio ex subere promineant qui tabulam circummoveant aculei. Et, ut tanto fortius tabula moveatur, sint plures Magnetes\protect\index{Sachverzeichnis}{magnes} perpendiculariter 