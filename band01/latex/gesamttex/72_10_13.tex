      
               
                \begin{ledgroupsized}[r]{120mm}
                \footnotesize 
                \pstart                
                \noindent\textbf{\"{U}berlieferung:}   
                \pend
                \end{ledgroupsized}
            
              
                            \begin{ledgroupsized}[r]{114mm}
                            \footnotesize 
                            \pstart \parindent -6mm
                            \makebox[6mm][l]{\textit{LiH}}Marginalien, An- und Unterstreichungen in \textsc{I. Barrow}, \textit{Lectiones opticae}, London 1672: Leibn. Marg. 0. Unterstreichungen mit Bleistift werden, da sie nicht eindeutig Leibniz zugeordnet werden k\"{o}nnen, nicht aufgenommen. \pend
                            \end{ledgroupsized}
                %\normalsize
                \vspace*{5mm}
                \begin{ledgroup}
                \footnotesize 
                \pstart
            \noindent\footnotesize{\textbf{Datierungsgr\"{u}nde}: Leibniz hat sich die mit den \cite{00144}\textit{Lectiones geometricae} zusammengebundenen \cite{00144}\textit{Lectiones opticae} w\"{a}hrend seiner Reise Anfang 1673 nach London gekauft. Nach Paris zur\"{u}ckgekehrt, hat er, wie aus dem Brief an Oldenburg vom 26. April 1673 (\textit{LSB} III, 1 N. 17) hervorgeht, zun\"{a}chst die \cite{00144}\textit{Lectiones opticae} studiert. Wir gehen davon aus, dass die Marginalien sowie die An- und Unterstreichungen in der Zeit zwischen der R\"{u}ckkehr aus London und der Erw\"{a}hnung der \cite{00144}\textit{Lectiones opticae} in dem Brief an Oldenburg entstanden sind.}
                \pend
                \end{ledgroup}
            
                \vspace*{8mm}
                \pstart 
                \normalsize
            \selectlanguage{latin} [p.~13] Cum autem hujusmodi motum circularem obeundo punctum B descripserit arcum B$\upbeta$, et punctum D arcum D$\updelta$, hoc est quando recta BD obtinuerit situm $\upbeta$$\updelta$, etiam ipsum punctum D speculo\protect\index{Sachverzeichnis}{speculum} impinget ad $\updelta$ reditumque\footnote{\textit{Leibniz unterstreicht:} reditumque} proinde per arcum $\updelta$D, scilicet ipsius quoque jam interciso cursu, molietur; [...].\selectlanguage{latin}
            \pend