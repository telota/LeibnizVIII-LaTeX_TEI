[141 v\textsuperscript{o}] il est question de la pression \edtext{du ressort ou du poids de l'air}{\lemma{pression}\Afootnote{ \textit{ (1) }\ de l'air\protect\index{Sachverzeichnis}{pression de l'air|textit} \textit{ (2) }\ du [...] l'air \textit{ L}}}, et \edtext{non}{\lemma{}\Afootnote{non \textit{ erg.} \textit{ L}}} pas  quand il est question de la pression du  mouuement des liqueurs. Car la pression de l'air\protect\index{Sachverzeichnis}{pression de l'air} ne consiste pas dans un mouuement, mais dans un effort, qui n'agit, que quand il trouue une in\'{e}galit\'{e} ou difformit\'{e}, \edtext{par consequent}{\lemma{difformit\'{e},}\Afootnote{ \textit{ (1) }\ \`{a} laquelle \textit{ (2) }\ par consequent \textit{ L}}} le ressort de la moindre bulle d'air \edtext{dans sa consistence ordinaire}{\lemma{}\Afootnote{dans sa consistence ordinaire \textit{ erg.} \textit{ L}}} \'{e}gale l'effort \edtext{de toute l'atmosphere}{\lemma{l'effort}\Afootnote{ \textit{ (1) }\ que \textit{ (2) }\ le ressort de toute l'a \textit{ (3) }\ de toute l'atmosphere \textit{ L}}} sur elle, et fait par consequent tant d'effort sur un lieu voisin que toute l'atmosphere\protect\index{Sachverzeichnis}{atmosph\`{e}re}; puisque par une regle generalle que j'ay demontr\'{e}e \edtext{ailleurs}{\lemma{ailleurs}\Bfootnote{Vgl. \cite{00268}N. 48\protect\raisebox{-0.5ex}{\notsotiny 4}.}}, comme les \edtext{liqueurs contin\"{u}es ne pesent l'un contre l'autre qu'\`{a} raison}{\lemma{liqueurs}\Afootnote{ \textit{ (1) }\ ne pesent que par \textit{ (2) }\ contin\"{u}es [...] raison \textit{ L}}} de leurs hauteurs, sans \edtext{qu'il faille}{\lemma{}\Afootnote{qu' \textit{ (1) }\ on aye besoin d \textit{ (2) }\ il faille \textit{ erg.} \textit{ L}}} avoir \'{e}gard \`{a} leur \'{e}paisseur; de m\^{e}me les liqueurs contin\"{u}es ne font ressort, l'un contre l'autre, \edtext{qu'\`{a} raison des degrez de l'espece de leur}{\lemma{l'autre,}\Afootnote{ \textit{ (1) }\ que par leur \textit{ (2) }\ qu'\`{a} [...] leur \textit{ L}}} consistence, sans \edtext{qu'il soit necessaire d'}{\lemma{}\Afootnote{qu'il soit necessaire d' \textit{ erg.} \textit{ L}}}avoir \'{e}gard \`{a} la quantit\'{e}. Mais quand on \edtext{veut employer}{\lemma{on}\Afootnote{ \textit{ (1) }\ veut rendre raison \textit{ (2) }\ veut employer \textit{ L}}}%Afootnote zu footnote aus 141r
\edtext{}{\lemma{}\linenum{|14|||18|}\Afootnote{\textit{ (1) }On pourra répondre que \textit{ (2) } Il [...] aussi \textit{ (a) } dans \textit{ (b) } difficilement [...] car \textit{ (aa) } nous voyons \textit{ (bb) } nous sçavons [...] tout. \textit{ L}}}
les vagues ou mouuements actuels, il faut estimer la quantit\'{e} et le nombre des coups,\rule[-0.5cm]{0cm}{0.5cm} puisqu'un coup est independant de l'autre; ne so\^{u}tenant pas l'effort de tous les autres, comme \edtext{une petite portion d'air so\^{u}tient}{\lemma{comme}\Afootnote{ \textit{ (1) }\ une bulle so\^{u}tient \textit{ (2) }\ une petite portion d'air so\^{u}tient \textit{ L}}} \edtext{celuy de toute l'atmosphere}{\lemma{celuy}\Afootnote{ \textit{ (1) }\ de l'effort \textit{ (2) }\ du ressort \textit{ (3) }\  de toute l'atmosphere \textit{ L}}}.
\pend
\pstart Il y a bien d'autres difficult\'{e}s car je ne voy pas comment ce fluide ambient, puisse entrer ais\'{e}ment dans la bulle, car s'il passe ais\'{e}ment \`{a} travers du verre et de la liqueur purg\'{e}e\protect\index{Sachverzeichnis}{liqueur!purg\'{e}e}, la liqueur tombera, estant \'{e}galement press\'{e}e de deux costez, comme le Mercure\protect\index{Sachverzeichnis}{mercure} tombe, quand le Tuyau de\protect\index{Sachverzeichnis}{tuyau de Torricelli}  Torricelli vient d'estre perc\'{e} en haut. De plus si l'on s'imagine deux colomnes d'eau comme il est dit \textso{dans la lettre imprim\'{e}e susmentionn\'{e}e}\edtext{}{\lemma{susmentionn\'{e}e}\Bfootnote{\textsc{P.-D. Huet, }\cite{00163}a.a.O. Vgl. auch \cite{00267}N. 48.}}, \textit{AC}, et \textit{AD} dont l'une pese contre l'autre, jusqu'\`{a} ce que la colomne \textit{AC} estant so\^{u}lev\'{e}e par la pression de la bulle \textit{B} la colomne \textit{AD} descend, et par consequent la phiole se vuide; alors je ne voy pas, pourquoy la bulle ne puisse produire cet effect,\edtext{}{\lemma{}\Afootnote{effect,  \textbar\ m\^{e}me \textit{ gestr.}\ \textbar\ devant \textit{ L}}} devant quelle arrive \`{a} la hauteur \textit{B} sans insister sur ce que la bulle par \edtext{son mouuement}{\lemma{par}\Afootnote{ \textit{ (1) }\ la pression de \textit{ (2) }\ son mouuement \textit{ L}}} interieur presse autant vers en haut que vers embas. On dit bien de choses \textso{dans cette lettre} au sujet de la bulle, dont je ne comprends pas la raison, par exemple, qu'il ne faut conter la colomne \textit{BD} que depuis \textit{B}, et \`{a} l'autre \textit{AC} neantmoins d\'{e}puis \textit{A} quoyque je ne voye pas sujet d'aucune difference. Item je ne voy pas pourquoy il faut conter \textit{BC} d\'{e}puis \textit{B} quand la bulle est arriv\'{e}e en \textit{B}, ayant auparavant pris \textit{AC} d\'{e}puis \textit{A}. Item pourquoy \edtext{la colomne}{\lemma{}\Afootnote{la colomne \textit{ erg.} \textit{ L}}} \textit{AC} a est\'{e} plus pesante que l'autre