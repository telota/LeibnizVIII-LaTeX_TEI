[65 v\textsuperscript{o}] observationes requirant altitudines sideris\protect\index{Sachverzeichnis}{sidus} observandi  aequales, aut observationem unicam, sed sideris\protect\index{Sachverzeichnis}{sidus}  in summa altitudine positi, seu meridianum\protect\index{Sachverzeichnis}{meridianus} transeuntis.  Unde nec Refractionum\protect\index{Sachverzeichnis}{refractio} error magnopere metuendus  est hac methodo nostra. \edtext{Nam}{\lemma{nostra.}\Afootnote{ \textit{ (1) }\ Cum eni \textit{ (2) }\ Nam \textit{ L}}} \edtext{interdiu}{\lemma{}\Afootnote{interdiu \textit{ erg.} \textit{ L}}} solis\protect\index{Sachverzeichnis}{sol} supra horizontem altius evecti refractio\protect\index{Sachverzeichnis}{refractio}  non est magni momenti. Et noctu si unum  sidus apparet, apparent etiam plura, eaque  diversae ascensionis, ac proinde refractionum\protect\index{Sachverzeichnis}{refractio} quoque  differentium, ut proinde duo simul observari, et inde  collatione facta error exterminari possit. \pend \pstart  Contra si solis\protect\index{Sachverzeichnis}{sol} ortus occasusque observandus est, vereor  ne difficile sit verum ortus occasusque tempus reperire, @@@ G R A F I K @@@ @@@ G R A F I K @@@ quia constat solem\protect\index{Sachverzeichnis}{sol} \edtext{in horizonte sensibili libero mari definiti  surgentem aut cadentem}{\lemma{?LEMMA?:solem}\Afootnote{ \textit{ (1) }\ in aperto mari surgentem aut  cadentem \textit{ (2) }\ in [...] cadentem \textit{ L}}}, miram sui speciem iri, extremi  marginis undis exhibere, ac falsa sui varieque detorta.\footnote{  \textbar\  \textit{ (1) }\ Uti \textit{ (2) }\ Commodissimum autem est eligi stellam\protect\index{Sachverzeichnis}{stella}  ejusdem paralleli\protect\index{Sachverzeichnis}{circulus parallelus} cum nostro, aut saltem  vicini, eamque meridiano\protect\index{Sachverzeichnis}{meridianus} non nimis 5 vicinam. \textit{ gestr.}\ \textbar\ } Imagine sibi ipsi jam \edtext{summerso}{\lemma{jam}\Afootnote{ \textit{ (1) }\ depresso \textit{ (2) }\ summerso \textit{ L}}}, \edtext{diu}{\lemma{}\Afootnote{diu \textit{ erg.} \textit{ L}}} superesse  at postea etiam specie illa fallaci subito evanescente  noctem celerrime ac velut de improviso ingruere, \edtext{brevissimo ac parum notabili Crepusculo}{\lemma{ingruere,}\Afootnote{ \textit{ (1) }\ Crepusculo exiguo admodum \textit{ (2) }\ brevissimo ac parum notabili Crepusculo \textit{ L}}} interjecto, \edtext{praeterquam quod saepissime accidet ut una observatione facta, altera  altitudinis aequalis ab aeris navisve\protect\index{Sachverzeichnis}{navis} statu impediatur.}{\lemma{}\Afootnote{BITTE UEBERPRUEFEN!!! praeterquam [...] aerisnavisve\protect\index{Sachverzeichnis}{navis} statu impediatur. \textit{ erg.} \textit{ L}}} \pend \pstart \edtext{Turbare non debet quod hoc loco ad Inventionem Longitudinum\protect\index{Sachverzeichnis}{longitudo}, Latitudines\protect\index{Sachverzeichnis}{latitudo} inventas praerequirimus; nam et alioquin Latitudinum\protect\index{Sachverzeichnis}{latitudo} quoque inventio jam tum necessaria est ad cursum navis\protect\index{Sachverzeichnis}{navis} gubernandum, et ut Latitudo\protect\index{Sachverzeichnis}{latitudo} sine Longitudine\protect\index{Sachverzeichnis}{longitudo}, ita contra Longitudo\protect\index{Sachverzeichnis}{longitudo} quoque sine latitudine\protect\index{Sachverzeichnis}{latitudo} non sufficit.  Ut taceam hoc loco spem esse magnam, inveniri posse, aut ad perfectionem deduci quam primum inventionem Latitudinis\protect\index{Sachverzeichnis}{latitudo} seu elevationis Poli\protect\index{Sachverzeichnis}{elevatio!poli} universalem, ab omni observatione coelesti, ac proinde aeris marisque injuria independentem, de quo alias fusius dicendi locus erit.}{\lemma{}\Afootnote{BITTE UEBERPRUEFEN!!! Turbare [...] InventionemLongitudinum\protect\index{Sachverzeichnis}{longitudo}, Latitudines\protect\index{Sachverzeichnis}{latitudo} inventas praerequirimus; nam   \textbar\ et \textit{ erg.}\ \textbar\  alioquin Latitudinum\protect\index{Sachverzeichnis}{latitudo} quoque inventio jam tum necessaria est ad cursum navis\protect\index{Sachverzeichnis}{navis} gubernandum, et ut Latitudo\protect\index{Sachverzeichnis}{latitudo} sine Longitudine\protect\index{Sachverzeichnis}{longitudo}, ita contra Longitudo\protect\index{Sachverzeichnis}{longitudo} quoque sine latitudine\protect\index{Sachverzeichnis}{latitudo} non sufficit. \pend \pstart  Ut taceam hoc loco spem esse magnam, inveniri posse, \edtext{aut}{\lemma{posse,}\Afootnote{ \textit{ (1) }\ et \textit{ (2) }\ aut \textit{ L}}} ad perfectionem deduci quam primum inventionem Latitudinis\protect\index{Sachverzeichnis}{latitudo} seu elevationis Poli\protect\index{Sachverzeichnis}{elevatio!poli}  \textbar\ universalem \textit{ erg.}\ \textbar\ , ab omni observatione coelesti, ac proinde aeris marisque injuria independentem, de quo alias fusius dicendi locus erit. \textit{ erg.} \textit{ L}}} @@@ G R A F I K @@@@@@ G R A F I K @@@Problema ergo meum ita concipitur: \textso{Data }\edtext{\textso{Latitudine  [l]oci navis Horologique exacto, et accedente}}{\lemma{Data}\Afootnote{ \textit{ (1) }\ \textso{Horologio}\protect\index{Sachverzeichnis}{horologium|textit}\textso{ exacto, dataque} \textit{ (2) }\ \textso{Latitudine [...] accedente} \textit{ L}}}\textso{ unica, quacunque, }\textso{sideris}\textso{ }\protect\index{Sachverzeichnis}{sidus}\textso{   (motus explorati,) cujuscunque }\edtext{\textso{(in horizonte navis motum satis sensibilem habentis, seu polo non nimis vicini)}}{\lemma{}\Afootnote{BITTE UEBERPRUEFEN!!!  \textit{ (1) }\ \textso{(inprimis }\textso{meridiano}\protect\index{Sachverzeichnis}{meridianus|textit} \textit{ (2) }\ \textso{meridiano}\protect\index{Sachverzeichnis}{meridianus|textit}\textso{ non nimis vicini} \textit{ (3) }\ \textso{(in [...] vicini)} \textit{ erg.} \textit{ L}}}\textso{ observatione  Longitudines}\protect\index{Sachverzeichnis}{longitudo}\textso{, ac }\edtext{\textso{per consequens}}{\lemma{ac}\Afootnote{ \textit{ (1) }\ \textso{quod idem est} \textit{ (2) }\ \textso{per consequens} \textit{ L}}}\textso{ locum }\textso{navis}\protect\index{Sachverzeichnis}{navis}\textso{,   reperire.}\pend \pstart \edtext{Problema jam propositum  ita solvetur}{\lemma{?LEMMA?:reperire.}\Afootnote{ \textit{ (1) }\ Hoc ita fiet: Dato Horologio\protect\index{Sachverzeichnis}{horologium|textit} exacto datur locus sideris \protect\index{Sachverzeichnis}{sidus|textit} in coelo, quodcunque   sit, in  \textit{(a)}\ ordin \textit{(b)}\ respectu, ad locum \textit{(aa)}\ , ubi ho \textit{(bb)}\  discessus,   ad quem horologium\protect\index{Sachverzeichnis}{horologium|textit} direximus, qui \textit{ (2) }\ Cum \textit{ (3) }\ Datur enim   hora loci discessus, ac per consequens situs omnium siderum\protect\index{Sachverzeichnis}{sidus|textit},  \textit{(a)}\ qui \textit{(b)}\ posito quod nobis loci  discessus  latitudo\protect\index{Sachverzeichnis}{latitudo|textit} et longitudo\protect\index{Sachverzeichnis}{longitudo|textit} cognita sit. Datur vero praeterea  circulus aequatori\protect\index{Sachverzeichnis}{aequator} parallelus\protect\index{Sachverzeichnis}{circulus parallelus|textit} in quo navis\protect\index{Sachverzeichnis}{navis|textit} nunc versatur, Latitudine\protect\index{Sachverzeichnis}{latitudo|textit} quippe navis\protect\index{Sachverzeichnis}{navis|textit} data. Ac datur denique  angulus quem facit sidus\protect\index{Sachverzeichnis}{sidus|textit} cujus aspectus  nobis conceditur (et cujus per priora, locum praesentem  in mundo, scimus) ad horizontem navis\protect\index{Sachverzeichnis}{navis|textit},  \textit{(aa)}\  seu ad  circulum parallelum navis\protect\index{Sachverzeichnis}{navis|textit}. Idem  enim est angulus ad horizontem navis\protect\index{Sachverzeichnis}{navis|textit}, et ad  \textit{(aaa)}\ parallelum \textit{(bbb)}\   tangentem circuli paralleli navis\protect\index{Sachverzeichnis}{navis|textit}. Quia \textit{(bb)}\ seu  ad tangentem globi telluris\protect\index{Sachverzeichnis}{tellus|textit} in puncto navis\protect\index{Sachverzeichnis}{navis|textit}, aut ad  radium ductum ex centro terrae\protect\index{Sachverzeichnis}{terra|textit} in punctum navis\protect\index{Sachverzeichnis}{navis|textit}.  \textit{(aaa)}\ Facit aut \textit{(bbb)}\ Cognitum autem est quem angulum faciat  Circulus Parallelus datus ejusve tangens, aut radius ad radium  aut tangentem telluris\protect\index{Sachverzeichnis}{tellus|textit} a quo producto tangitur  aut secatur. Ergo   \textbar\ cognoscetur, \textit{ erg.}\ \textbar\  quem angulum faciat linea ex sidere\protect\index{Sachverzeichnis}{sidus|textit}  ducta seu radius sideris \protect\index{Sachverzeichnis}{sidus|textit} ad Parallelum\protect\index{Sachverzeichnis}{circulus parallelus|textit}. Jam  \textit{(aaaa)}\ idem  punctum cognitum  \textit{(bbbb)}\  ex uno puncto extra circulum  (aut saltem extra circuli centrum) posito ad eundem  circulum non possunt duci duae lineae eundem angulum  facientes.  \textit{(aaaaa)}\ Est ergo \textit{(bbbbb)}\ Determinato ergo circulo, et puncto  extra $\lbrack$circulum,$\rbrack$ [et] angulo  \textit{(aaaaaa)}\ puncti  \textit{(bbbbbb)}\ lineae circulum puncto connectentis  ad circulum (id est ad circuli radium vel tangentem) determinatum erit  punctum in circulo, ad quod linea connectens facit angulum  datum. \textit{ (4) }\  Hoc ita fiet \textit{ (5) }\ Problema jam propositum  ita solvetur \textit{ L}}}.  {ordin? unklare Lesung}  [Rechnungen nicht zum Text geh\"{o}rig.] @@@ G R A F I K @@@% \begin{wrapfigure}{l}{0.4\textwidth}                    
                %\includegraphics[width=0.4\textwidth]{../images/De+longitudinum+determinatione+scheda+prima/LH035%2C15%2C06_065v/files/100570.png}
                        %\caption{Bildbeschreibung}
                        %\end{wrapfigure}
                        %@ @ @ Dies ist eine Abstandszeile - fuer den Fall, dass mehrere figures hintereinander kommen, ohne dass dazwischen laengerer Text steht. Dies kann zu einer Fahlermeldung fuehren. @ @ @ \\
                     @@@ G R A F I K @@@% \begin{wrapfigure}{l}{0.4\textwidth}                    
                %\includegraphics[width=0.4\textwidth]{../images/De+longitudinum+determinatione+scheda+prima/LH035%2C15%2C06_065v/files/100572.png}
                        %\caption{Bildbeschreibung}
                        %\end{wrapfigure}
                        %@ @ @ Dies ist eine Abstandszeile - fuer den Fall, dass mehrere figures hintereinander kommen, ohne dass dazwischen laengerer Text steht. Dies kann zu einer Fahlermeldung fuehren. @ @ @ \\
                     @@@ G R A F I K @@@% \begin{wrapfigure}{l}{0.4\textwidth}                    
                %\includegraphics[width=0.4\textwidth]{../images/De+longitudinum+determinatione+scheda+prima/LH035%2C15%2C06_065v/files/100574.png}
                        %\caption{Bildbeschreibung}
                        %\end{wrapfigure}
                        %@ @ @ Dies ist eine Abstandszeile - fuer den Fall, dass mehrere figures hintereinander kommen, ohne dass dazwischen laengerer Text steht. Dies kann zu einer Fahlermeldung fuehren. @ @ @ \\
                     \pend 