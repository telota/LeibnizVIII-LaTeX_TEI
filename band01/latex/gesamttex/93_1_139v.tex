[139~v\textsuperscript{o}] vous\footnote{\textit{Am oberen Rand:}\\ $AC \sqcap x. \hspace{2mm} (A)C \sqcap y. \hspace{2mm} AB \sqcap a.\ Ergo\ 2ax - x^2 \sqcap y^2 \\\hspace{70mm} a \sqcap y^2 + x^2 \smallsmile 2x$} avez trouu\'{e} trois points \textit{(A)}, \textit{A}, \textit{((A))} de l'arc de cercle \textit{(A)A((A))} et par consequent il est ais\'{e} d'en trouuer le centre \textit{B} et \edtext{}{\lemma{}\Afootnote{et  \textbar\ par consequent \textit{ gestr.}\ \textbar\ la \textit{ L}}}la longueur \textit{AB}, ou \textit{CB} en tra\c{c}ant des lignes sur le papier proportionelles \`{a} celles qu'on a fait effectivement; ce qui se fait en mesurant la ligne \edtext{\textit{AC}, item la ligne \textit{(A)C} vel}{\lemma{ligne}\Afootnote{ \textit{ (1) }\ \textit{AC} vel \textit{ (2) }\ \textit{AC}, [...] vel \textit{ L}}} \textit{((A))C} et les transportant sur le papier avec une echelle.\pend
\pstart Mais pour s\c{c}avoir la longueur de \textit{AB} en nombres, mesurez les lignes \textit{AC}, \textit{(A)C}. Supposez que \textit{AC} fasse 5 pouces, et que \textit{(A)C} en fasse dix. Multipliez 5 par soy même, et vous aurez 25. Multipliez aussi 10 par soy même et vous aurez 100. La somme de 25 et de 100 fait 125. Divisez cette somme \edtext{par deux fois 5 ou}{\lemma{}\Afootnote{par deux fois 5 ou \textit{ erg.} \textit{ L}}} par 10, et vous aurez $\protect\begin{array}{l} \cancel{1}\cancel{2}5\\\cancel{1}\cancel{0}0 \hspace{5.5pt}f\hspace{5.5pt}12\protect\displaystyle\protect\frac{1}{2}\\~~~\cancel{1}\protect\end{array}$\rule[-9mm]{0mm}{23mm} pouces qui est la longueur de \textit{AB} dont ostant 5 pouces, longueur de \textit{AC}. Il vous en restera, $7\displaystyle\frac{1}{2}$\rule[-4mm]{0mm}{10mm} longueur de \textit{BC}. Autre exemple: Supposez que \textit{AC} \edtext{donne}{\lemma{\textit{AC}}\Afootnote{ \textit{ (1) }\ fasse \textit{ (2) }\ donne \textit{ L}}} 1 pouce et \textit{(A)C} en contienne 9. \textit{CB} en aura 40 suivant le calcul que \edtext{voicy:}{\lemma{voicy}\Bfootnote{Trotz eines Fl\"{u}chtigkeitsfehlers in der letzten Zeile endet die Rechnung mit dem korrekten Ergebnis.}}\\
$\protect\begin{array}{l}1\\1\\\overline{1} \protect\end{array}\hspace{5mm} \protect\begin{array}{r}9\\9\\\overline{81} \protect\end{array}\\
\vspace{2mm}\rule{2mm}{0mm} 1 \hspace{5.5pt} + \hspace{5.5pt} 81 \sqcap \underline{82}\\
\rule{22mm}{0mm}2 \hspace{5.5pt} f \hspace{5.5pt} 41\\
\rule{30mm}{0mm} 40 - 1 \sqcap 40 \sqcap CB.$
\pend 