   
        
        \begin{ledgroupsized}[r]{120mm}
        \footnotesize 
        \pstart        
        \noindent\textbf{\"{U}berlieferung:}  
        \pend
        \end{ledgroupsized}
      
       
              \begin{ledgroupsized}[r]{114mm}
              \footnotesize 
              \pstart \parindent -6mm
              \makebox[6mm][l]{\textit{L}}Konzept: LH XXXV 15, 6 Bl. 20\textendash21. 1 Bog. 8\textsuperscript{o}. 1 S. auf Bl. 21 r\textsuperscript{o}, R\"{u}ckseite leer. Auf den verbleibenden Seiten von Bl. 20 N. 7. \\Cc 2, Nr. 1556 C \pend
              \end{ledgroupsized}
              
        %\normalsize
        \vspace*{5mm}
        \begin{ledgroup}
        \footnotesize 
        \pstart
      \noindent\footnotesize{\textbf{Datierungsgr\"{u}nde}: Unser St\"{u}ck befindet sich zusammen mit N. 7 auf einem Bogen. Wir gehen daher von einem gemeinsamen Entstehungszeitraum f\"{u}r beide St\"{u}cke aus.}
        \pend
        \end{ledgroup}
      
        \vspace*{8mm}
        \pstart 
        \normalsize
      [21 r\textsuperscript{o}] \selectlanguage{french}(I) Ces problemes supposent tousjours \textso{le lieu du soleil,\linebreak donn\'{e}}. Mais il ne peut pas estre donn\'{e} \edtext{tousjours}{\lemma{}\Afootnote{tousjours \textit{ erg.} \textit{ L}}} sur la mer\protect\index{Sachverzeichnis}{mer}, que par un Horloge\protect\index{Sachverzeichnis}{horloge} de la derniere exactitude, [montr\'{e}]\edtext{}{\Afootnote{mont\'{e}\textit{\ L \"{a}ndert Hrsg. } }} continuellement d\'{e}puis la sortie du port; pour s\c{c}avoir quelle heure il seroit, si nous serions encor au lieu du d\'{e}part. Un tel horloge\protect\index{Sachverzeichnis}{horloge} est le fondement de tout ce qu'on a trouu\'{e} \edtext{jusqu'\`{a} la}{\lemma{jusqu'\`{a} la}\Afootnote{Es fehlt ein Substantiv wie connaissance \textit{Hrsg.}}} de plus veritable pour les longitudes\protect\index{Sachverzeichnis}{longitude}. Mais il y a bien de difficultez, puisqu'on n'est pas assez asseur\'{e} du succês des pendules\protect\index{Sachverzeichnis}{pendule} mêmes sur la mer\protect\index{Sachverzeichnis}{mer}.\pend \pstart (II) Trouuer la declinaison de l'\'{e}guille aimant\'{e}e\protect\index{Sachverzeichnis}{aiguille aimant\'{e}e}\protect\index{Sachverzeichnis}{aiguille aimant\'{e}e|see{acus}}, et trouuer la ligne meridienne, c'est la même chose. \pend \pstart (III) Si la declinaison du soleil\protect\index{Sachverzeichnis}{declinaison du soleil} ou son lieu dans le Zodiaque\protect\index{Sachverzeichnis}{zodiaque} est donn\'{e} et l'heure courante \edtext{trouu\'{e}e}{\lemma{}\Afootnote{trouu\'{e}e \textit{ erg.} \textit{ L}}}, aussi les longitudes\protect\index{Sachverzeichnis}{longitude} sont d\'{e}couuertes. \edtext{Et si les latitudes\protect\index{Sachverzeichnis}{latitude} sont conn\"{u}es aussi, le lieu du navire\protect\index{Sachverzeichnis}{navire} est precisement connu.}{\lemma{}\Afootnote{Et si les [...] connu. \textit{ erg.} \textit{ L}}}\pend \pstart (IV) Les problemes donc auroient pû estre conceûs de cette sorte: (1) \edtext{S\c{c}achant la latitude du lieu present et l'heure du lieu du d\'{e}part du navire, trouuer le lieu present}{\lemma{S\c{c}achant}\Afootnote{ \textit{ (1) }\ les latitudes\protect\index{Sachverzeichnis}{latitude|textit} et l'heure du lieu du d\'{e}part du navire\protect\index{Sachverzeichnis}{navire|textit}, trouuer les longitudes\protect\index{Sachverzeichnis}{longitude|textit} \textit{ (2) }\ la [...] navire \textit{(a)}\ , trouuer la longitude\protect\index{Sachverzeichnis}{longitude|textit} du lieu present \textit{(b)}\ (ou la longitude\protect\index{Sachverzeichnis}{longitude|textit}), trouuer le lieu present \textit{(c)}\ , trouuer le lieu present \textit{ L}}}, aux rayons du soleil \edtext{parce que l'heure du lieu du d\'{e}part confer\'{e}e avec les rayons du soleil peut donner les longitudes\protect\index{Sachverzeichnis}{longitude}. Et les longitudes\protect\index{Sachverzeichnis}{longitude} et latitudes\protect\index{Sachverzeichnis}{latitude} ensemble donnent les lieux precisement.}{\lemma{}\Afootnote{parce [...] avec \textit{ (1) }\ l'heure \textit{ (2) }\ les rayons [...] precisement. \textit{ erg.} \textit{ L}}} (2) S\c{c}achant la ligne meridienne du lieu present, et l'heure du lieu du d\'{e}part trouuer la latitude\protect\index{Sachverzeichnis}{latitude} du lieu present aux rayons\pend\pstart\noindent du soleil. Ou s\c{c}achant la ligne meridienne et la longitude\protect\index{Sachverzeichnis}{longitude} trouuer la latitude\protect\index{Sachverzeichnis}{latitude} aux rayons du soleil. (3) S\c{c}achant la latitude\protect\index{Sachverzeichnis}{latitude} et la longitude\protect\index{Sachverzeichnis}{longitude} trouuer la ligne meridienne aux rayons du soleil.\pend \pstart (V) Il faut tousjours avoir trois poincts pour determiner le lieu o\`{u} \edtext{nous sommes}{\lemma{o\`{u}}\Afootnote{ \textit{ (1) }\ (1) le \textit{ (2) }\ nous sommes \textit{ L}}}. Le lieu o\`{u} nous sommes se determine par comparaison avec le lieu du d\'{e}part. Si l'on s\c{c}auroit la distance du lieu du d\'{e}part, et les angles qu'on \edtext{}{\lemma{}\Afootnote{qu'on \textbar\ y \textit{ gestr.}\ \textbar\ a \textit{ L}}}a fait en cheminant, le lieu du navire\protect\index{Sachverzeichnis}{navire} sera determin\'{e}. Si l'on s\c{c}aura, \edtext{l'angle qui}{\lemma{s\c{c}aura,}\Afootnote{ \textit{ (1) }\ le lieu du soleil que nous voyons, en comparaison du lieu du d\'{e}part. \textit{ (2) }\ l'angle qui \textit{ L}}} [\textit{Satz bricht ab.}] \pend \pstart Nous avons tousjours determin\'{e} le centre de la terre. Par consequent l'horison. Par consequent l'angle du soleil \`{a} l'horison, si nous le voyons. Nous s\c{c}avons aussi par l'horloge\protect\index{Sachverzeichnis}{horloge}, l'angle que le même soleil fait en même temps, \`{a} un autre horison, du d\'{e}part.\selectlanguage{latin}\pend 