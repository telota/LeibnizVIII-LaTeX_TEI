[88 r\textsuperscript{o}] ac Telescopia\protect\index{Sachverzeichnis}{telescopium}, requirant: cum hoc iis qui sciunt quo modo  praenominatus D. des Cartes\protect\index{Namensregister}{\textso{Descartes} (Cartesius, des Cartes, Cartes.), Ren\'{e} 1596\textendash 1650} ad haec conficienda hyperbola  utatur, notum satis esse debeat. Ad cujus itaque dioptricam\protect\index{Sachverzeichnis}{dioptrica}  appello, in qua fundamenta horum omnium firmissima  jacta sunt. Verum quidem est in praedicta dioptrica\protect\index{Sachverzeichnis}{dioptrica} telescopia\protect\index{Sachverzeichnis}{telescopium}  ac microscopia\protect\index{Sachverzeichnis}{microscopium} non ex pluribus quam ex duobus lentibus\protect\index{Sachverzeichnis}{lens}  vitreis composita esse, cum ad eundem effectum aliquando  tribus Circularibus lentibus\protect\index{Sachverzeichnis}{lens!circularis} opus habeamus; aut etiam  quaedam ex pluribus componere possimus, sed eum et  hic iis, qui recte intelligunt quo modo ex duobus  componi possint, nulla difficultas superesse possit,  addere aliquid hac de re supervacuum diximus, atque  eo magis, quod semper minor vitrorum numerus, quando  idem effectus per eum haberi poterit, eligendus sit.  Unicum adhuc tantum verbum superaddam de iis vitris  circularibus quae ab utraque parte connexa sunt, quorum  exemplum in figura tertia\footnote{fig. 3.} exhibetur per \textit{MONP}, in quo \textit{O} est centrum, ex quo ducta est \textit{MPN}, et \textit{P}, ex quo \textit{NOM},  semidiametris existentibus aequalibus: in talibus nempe vitris 