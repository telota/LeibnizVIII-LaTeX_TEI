[150 v\textsuperscript{o}] \selectlanguage{french}De ces phenomenes on peut tirer premierement\footnote{\selectlanguage{french}Et, si l'on respond que la pression n'est pas \'{e}gale, \`{a} cause que les parties de la liqueur purg\'{e}e\protect\index{Sachverzeichnis}{liqueur!purg\'{e}e}, qui \edtext{respondent}{\lemma{?LEMMA?:coulomne}\Afootnote{ \textit{ (1) }\ sont  \textit{(a)}\ purg\'{e}es \textit{(b)}\ jointes \textit{ (2) }\ respondent \textit{ L}}} aux parties continues du verre, entre les pores, ne sont pas press\'{e}es de deux costez, et qu'ainsi il y a plus de matiere pressante de l'une que de de l'autre cost\'{e}, il s'ensuit donc qu'il faut considerer la quantit\'{e} de l'ouuerture, ou de la matiere subtile\protect\index{Sachverzeichnis}{mati\`{e}re!subtile} qui presse; si cette in\'{e}galit\'{e} est la cause de la suspension. Et par consequent le peu de la matiere, qui se trouue dans la bulle, n'\'{e}gale pas toute la pression de la matiere subtile vers la surface interieure du verre, de la matiere subtile qui entre librement par l'ouuerture du tuyau em bas. On a donc le choix ou pour\selectlanguage{latin}}  ces consequences.\pend \pstart \textso{Consequence 1.} Que la crainte du vuide\protect\index{Sachverzeichnis}{vide} n'y contribue rien \edtext{Autrement}{\lemma{rien}\Afootnote{ \textit{ (1) }\ icy. \textit{ (2) }\ comme d' autres l'on fait voir assez clairement. \textit{ (3) }\ Autrement \textit{ L}}} la difference du Recipient plein ou \'{e}puis\'{e}, de la liqueur naturelle ou purg\'{e}e\protect\index{Sachverzeichnis}{liqueur!purg\'{e}e} \edtext{ne changeroit pas}{\lemma{?LEMMA?:purg\'{e}e}\Afootnote{ \textit{ (1) }\ n'y feroit rien \textit{ (2) }\ ne changeroit pas \textit{ L}}} les phenomenes.\pend \pstart \textso{Conseq. 2.} Que la resistence\edtext{}{\lemma{}\Afootnote{resistence  \textbar\ des placques \textit{ erg. u.}\  \textit{ gestr.}\ \textbar\ de \textit{ L}}} de l'air est la cause du phaenom. \textso{1.}\edtext{}{\lemma{}\Afootnote{1.  \textbar\ et 2. \textit{ gestr.}\ \textbar\ comme \textit{ L}}} comme cela paroist par le phaenomene \textso{2.} et \textso{3.}\pend \pstart \textso{Conseq. 3.} Que l'attachement \edtext{de deux placques\protect\index{Sachverzeichnis}{deux placques}}{\lemma{}\Afootnote{de deux placques\protect\index{Sachverzeichnis}{deux placques} \textit{ erg.} \textit{ L}}} dans le vuide\protect\index{Sachverzeichnis}{vide}, ne provient ny d'\edtext{une certaine gl\"{u}e insensible}{\lemma{d'}\Afootnote{ \textit{ (1) }\ un Gluten \textit{ (2) }\ une certaine gl\"{u}e insensible \textit{ L}}}, ny d'une autre raison, qui se puisse trouuer dans les corps \edtext{unis mêmes}{\lemma{corps}\Afootnote{ \textit{ (1) }\ mêmes qui sont attachez ensemble \textit{ (2) }\ unis mêmes \textit{ L}}}, mais d'une pression exterieure. La raison \edtext{en}{\lemma{}\Afootnote{en \textit{ erg.} \textit{ L}}} est, parce qu'autrement la separation transversale \edtext{de deux points correspondants, et attachez ensemble dans les placques}{\lemma{}\Afootnote{BITTE UEBERPRUEFEN!!! de deux points  \textit{ (1) }\ des placques \textit{ (2) }\ correspondants, et attachez ensemble dans les placques \textit{ erg.} \textit{ L}}} seroit \edtext{aussi}{\lemma{seroit}\Afootnote{ \textit{ (1) }\ autant \textit{ (2) }\ aussi \textit{ L}}} difficile que la directe: contre le phaenom. 9. car on a trouu\'{e} que les deux placques\protect\index{Sachverzeichnis}{deux placques} \edtext{glissent ais\'{e}ment l'une sur}{\lemma{?LEMMA?:placques}\Afootnote{ \textit{ (1) }\ sont   \textbar\ ais\'{e}ment \textit{ erg.}\ \textbar\  mobiles l'une  \textbar\ sur \textit{ gestr.}\ \textbar\  \textit{ (2) }\ glissent ais\'{e}ment l'une   \textbar\ sur \textit{ erg.} \textit{ Hrsg. }\  \textit{ L}}} l'autre (même dans le vuide,) \edtext{pendant qu'elles resistent \`{a} la separation perpendiculaire}{\lemma{vuide,)}\Afootnote{ \textit{ (1) }\ mais pas ais\'{e}es \`{a} separer perpendiculairement \textit{ (2) }\ pendant [...] perpendiculaire \textit{ L}}}.\pend \pstart \textso{Conseq. 4.} Il s'ensuit donc qu'il \edtext{reste}{\lemma{qu'il}\Afootnote{ \textit{ (1) }\ y a \textit{ (2) }\ reste \textit{ L}}} tousjours\footnote{\selectlanguage{french}Si la liqueur se d\'{e}tache quand la bulle ne touche pas le verre, experience. Quand on l'\'{e}prouvera avec le Mercure\protect\index{Sachverzeichnis}{mercure}, jusqu'\`{a} la hauteur possible, s\c{c}avoir, si le surplus se d\'{e}tachera seulement, et le reste demeurera suspendu en haut.Experiences \`{a} faire, sur cette matiere avec les deux placques\protect\index{Sachverzeichnis}{deux placques}\edtext{en haut}{\lemma{respondent}\Afootnote{ \textit{ (1) }\ comme \textit{ (2) }\ en haut \textit{ L}}} ou le Mercure purg\'{e}\protect\index{Sachverzeichnis}{mercure!purg\'{e}} d'air. Il faut les placques dans le vuide\protect\index{Sachverzeichnis}{vide}\edtext{laisser}{\lemma{haut}\Afootnote{ \textit{ (1) }\ tremper \textit{ (2) }\ laisser \textit{ L}}} dans une liqueur comme dans l'eau, pour voir s'il y aura une difference.Experience \`{a} faire, percer le Tuyau dans le vuide\protect\index{Sachverzeichnis}{vide}, ou le rompre entierement en haut, pour voir, s'il tombera alors, principalement estant demeur\'{e} longtemps en repos. Experience \`{a} faire avec les placques perc\'{e}es.\selectlanguage{latin}\selectlanguage{ngerman}Ob die platten zusamen gestossen werden, wenn sie einander nahe.Experience, cum antlia suctoria, ob man im vacuo aere purgato pumpen koenne, hoc refutabit gluten. Id item ob man schwehrer pumpe wenn der liquor lange dran gewesen.\selectlanguage{latin}\selectlanguage{french}Il faut faire l'experience si avec le Mercure purg\'{e}\protect\index{Sachverzeichnis}{mercure!purg\'{e}} d'air le Mercure\protect\index{Sachverzeichnis}{mercure} ne tombe pas devant que la bulle est arriv\'{e}e \`{a} 27 pouces de hauteur.\selectlanguage{latin}} \edtext{quelque matiere}{\lemma{tousjours}\Afootnote{ \textit{ (1) }\ un corps \textit{ (2) }\ quelque matiere \textit{ L}}} dans la cavit\'{e} du Recipient \edtext{ dont on a tir\'{e} l'air}{\lemma{Recipient}\Afootnote{ \textit{ (1) }\ \'{e}puis\'{e} \textit{ (2) }\  dont on a tir\'{e} l'air \textit{ L}}}; \edtext{qui puisse}{\lemma{l'air;}\Afootnote{ \textit{ (1) }\ pour pouuoir \textit{ (2) }\ qui puisse \textit{ L}}} exercer cette pression sur les deux corps attachez ensemble.\pend \pstart \edtext{}{\lemma{}\Afootnote{ensemble.  \textbar\ \textso{Conseq. 5.} \textit{ gestr.}\ \textbar\ Je \textit{ L}}} Je ne dis pas pourtant, qu'il y a des pores dans le verre, \edtext{pour le passage de cette matiere car}{\lemma{verre,}\Afootnote{ \textit{ (1) }\ ny de \textit{ (2) }\ puisque \textit{ (3) }\ pour [...] car \textit{ L}}} on peut expliquer tout cela par la seule propagation des pressions, laquelle passe partout \edtext{jusqu'}{\lemma{}\Afootnote{jusqu' \textit{ erg.} \textit{ L}}} \`{a} l'indefini.\pend \pstart\edtext{ \textso{Conseq. 5.} Enfin il faut aussi que cette pression se fasse par un mouuement, ou par un effort;  Il reste \`{a} present de rendre raison de la maniere de cette pression par une Hypothese.}{\lemma{?LEMMA?:l'indefini.}\Afootnote{ \textit{ (1) }\ \textso{Conseq. 5.} Il faut aussi que la pression  \textit{(a)}\ dans \textit{(b)}\ de cette matiere ne soit plus forte que celle de l'air, car autrement la placque ne tomberoit pas, ayant surmont\'{e} la colomne de l'air contre le calcul. Et neantmoins elle semble estre plus forte, \`{a} cause qu'elle soustient le Mercure\protect\index{Sachverzeichnis}{mercure|textit} \`{a} une  \textit{(c)}\ plus grande hauteur \textit{(d)}\ hauteur outre le double de l'ordinaire. Outre le double dis je, pour ne dire pas  \textit{(e)}\ que ceux  \textit{(f)}\ ce d \textit{(g)}\ qu'il suffisse de joindre ces deux pressions ensemble. Mais il faut examiner la pression des placques: car peut estre qu'il y a deux pressions, et qu'il y faut plus de force que pour surmonter l'air, comme avec le Mercure purg\'{e}\protect\index{Sachverzeichnis}{mercure!purg\'{e}|textit}. \textit{ (2) }\  \textso{Conseq. 5.} Enfin il faut   \textbar\ aussi \textit{ erg.}\ \textbar\ que [...] effort; \textit{(a)}\ pour le mouuement, \textit{(b)}\ d'une matiere  \textit{(aa)}\ plus subtile \textit{(bb)}\ moins grossiere que l'air sensible.Il [...] Hypothese. \textit{ L}}}\pend \pstart \edtext{\textso{Consequ. 6.} J'ose dire, d'avantage que le phenomene 7. (et par consequent les autres non plus) ne peut pas estre expliqu\'{e} par un mouuement d'une matiere subtile, mais seulement par un effort compens\'{e}  de l'atmosphaere sur nous, ou de la coulomne}{\lemma{?LEMMA?:Hypothese.}\Afootnote{ \textit{ (1) }\ \textso{Consequ. 6.} J'ose  \textbar\ même \textit{ gestr.}\ \textbar\ dire, [...] 7.  \textbar\ (et [...] plus) \textit{ erg.}\ \textbar\ ne [...] compens\'{e} \textit{(a)}\ et insensible, sans \textit{(b)}\ comme est celuy de  \textbar\ l'adversaire \textit{ erg. u.}\  \textit{ gestr.}\ \textbar\ l'atmosphaere [...] coulomne \textit{ L}}}\selectlanguage{latin}(weiter mit fol. 151r, die Klammer wird auf dem n\"{a}chsten Blatt geschlossen}