[65 r\textsuperscript{o}] aestus marinos eodem modo incrementa variare.  Quod si ergo \edtext{Inclinationes quoque}{\lemma{ergo}\Afootnote{ \textit{ (1) }\ idem in \textit{ (2) }\ Inclinationes quoque \textit{ L}}} magneticae\protect\index{Sachverzeichnis}{magnes}  eandem proportionem servare detegentur, facilis  erit Latitudinis\protect\index{Sachverzeichnis}{latitudo} calculus, \edtext{ex sinuum Canone}{\lemma{calculus,}\Afootnote{ \textit{ (1) }\ modo Circulus in  quo Acus inclinationes\protect\index{Sachverzeichnis}{acus!inclinatoria} describuntur, possit esse  tantus, ut ad minuta usque  \textit{(a)}\ secunda dividi q \textit{(b)}\ aut ultra dividi queat \textit{ (2) }\ ex sinuum Canone \textit{ L}}}. Sin  minus, \edtext{peculiaris  illa progressionis ratio erit exacte  observanda, ut Tabula}{\lemma{minus,}\Afootnote{ \textit{ (1) }\ nova \textit{ (2) }\ separata quaedam Tabula \textit{ (3) }\ peculiaris [...] Tabula \textit{ L}}} qualiscunque condi queat. \pend \pstart  Sed difficultas \edtext{in praxi non contemnenda  offertur}{\lemma{difficultas}\Afootnote{ \textit{ (1) }\ haec oritur, \textit{ (2) }\ in praxi non contemnenda  offertur \textit{ L}}}. \edtext{Cum enim quodlibet Meridiani minutum  miliario Italico respondeat, apparet opus esse,  ut}{\lemma{offertur.}\Afootnote{ \textit{ (1) }\ Constat enim ad exactam opus esse u \textit{ (2) }\  necessarium \textit{ (3) }\ Cum [...] apparet \textit{(a)}\ rationis \textit{(b)}\ opus esse,  ut \textit{ L}}} \edtext{ Acus Inclinatoria\protect\index{Sachverzeichnis}{acus!inclinatoria}}{\lemma{ut}\Afootnote{ \textit{ (1) }\ Circulus \textit{ (2) }\  Acus Inclinatoria \textit{ L}}} minuta, minimum  prima, monstret. Sed ad hoc praestandum opus  est circulo ingenti, ac proinde etiam acu  tam longa, qualis \edtext{nec paratu, nec conservatu,  nec motu facilis.}{\lemma{qualis}\Afootnote{ \textit{ (1) }\ raro \textit{ (2) }\ nec [...] facilis. \textit{ L}}}\pend \pstart Sed non est, quod hic haereamus \textso{reperi} enim \textso{rationem }\edtext{\textso{applicandi}}{\lemma{rationem}\Afootnote{ \textit{ (1) }\ \textso{efficiendi} \textit{ (2) }\ \textso{applicandi} \textit{ L}}}\textso{  acum  magneticam}\protect\index{Sachverzeichnis}{acus!magnetica}\textso{ tam horizontalem }\edtext{\textso{seu vulgarem}}{\lemma{}\Afootnote{\textso{seu vulgarem} \textit{ erg.} \textit{ L}}}\textso{ quam  verticalem seu inclinatoriam, ad Circulos tam ingentes, ut ad minuta usque secunda }\edtext{\textso{commode}}{\lemma{}\Afootnote{\textso{commode} \textit{ erg.} \textit{ L}}}\textso{ subdividi  possint.} Quod inventum maximi ad rem nauticam  geographicamque, et sciotericam, et omnino omnem  cui usus pyxidis magneticae\protect\index{Sachverzeichnis}{pyxis!magnetica} intervenit momenti  suo loco proponam. Et credo nec Inclinationes\protect\index{Sachverzeichnis}{inclinatio}  nec declinationes\protect\index{Sachverzeichnis}{declinatio} Magneticas ad regulam reductum  iri, nisi pyxis\protect\index{Sachverzeichnis}{pyxis} ad minuta usque secunda subdivisa,  minimasque etiam variationes accurate monstratura  adhibeatur. Qualis hactenus ne proposita quidem  a quoquam, nedum constructa est. \pend \pstart  ♀ \edtext{Quod}{\lemma{♀}\Afootnote{ \textit{ (1) }\ Quod \textit{ (2) }\ Cum \textit{ (3) }\ Quanquam \textit{ (4) }\ Quod \textit{ L}}} ergo \edtext{hoc modo}{\lemma{}\Afootnote{hoc modo \textit{ erg.} \textit{ L}}} Illustris Hugenius\protect\index{Namensregister}{\textso{Huygens} (Hugenius, Vgenius, Hugens, Huguens), Christiaan 1629\textendash 1695} Longitudines\protect\index{Sachverzeichnis}{longitudo} \edtext{ investigatur Latitudinibus licet}{\lemma{?LEMMA?:Longitudines}\Afootnote{ \textit{ (1) }\ investigat sine Latitudinibus\protect\index{Sachverzeichnis}{latitudo|textit} licet declinationeque\protect\index{Sachverzeichnis}{declinatio|textit} solis \textit{ (2) }\  investigatur Latitudinibus licet \textit{ L}}} incognitis, sed observationibus duabus \edtext{iisque non quibuslibet, sed altitudinum  aequalium; id ego putem praestari  posse observatione unica}{\lemma{duabus}\Afootnote{ \textit{ (1) }\ id ego praestari  etiam posse puto  Latitudine\protect\index{Sachverzeichnis}{latitudo|textit} inventa, observatione  alia accedente unica, \textit{ (2) }\ putem tamen ego idem praestari observatione  tantum, \textit{ (3) }\ iisque [...] unica \textit{ L}}}  eaque sive solis\protect\index{Sachverzeichnis}{sol}, sive alterius sideris\protect\index{Sachverzeichnis}{sidus} polo\protect\index{Sachverzeichnis}{polus} non nimium vicini, cujuscunque, et quocunque  in loco sit sidus\protect\index{Sachverzeichnis}{sidus}, sive meridiano\protect\index{Sachverzeichnis}{meridianus} loci vicinum, sive  ab eo remotum, sed Latitudine\protect\index{Sachverzeichnis}{latitudo} loci supposita;  quam certe jam tum alias investigari debere, ad locum navis\protect\index{Sachverzeichnis}{navis} cursumque \edtext{definiendum constat.}{\lemma{cursumque}\Afootnote{ \textit{ (1) }\ inest \textit{ (2) }\ definiendum constat. \textit{ L}}} Quo  facto non tantum unam observationem lucrabimur  et calculo nautarum conjecturali, erroribus obnoxio, sed \edtext{positis duabus observationibus dictis, naveque interea progrediente  ad cognoscendam locorum observationis cujusque differentiam,}{\lemma{sed}\Afootnote{ \textit{ (1) }\ ad cognoscendam  locorum observationis cujusque differentiam,  positis duabus observationibus naveque interea progrediente, \textit{ (2) }\ positis [...] differentiam, \textit{ L}}}  necessario, poterimus carere. Sed etiam \edtext{indefinitam habebimus observandi  libertatem}{\lemma{etiam}\Afootnote{ \textit{ (1) }\ quod  liberam habebimus \textit{ (2) }\ indefinitam habebimus observandi  libertatem \textit{ L}}}, quoties unius tantum sideris\protect\index{Sachverzeichnis}{sidus}, explorati motus, momentaneus\protect\index{Sachverzeichnis}{motus!momentaneus} ut sic dicam aspectus conceditur. Quod rarissime  deesse potest. Cum contra duae illae prioris methodi 