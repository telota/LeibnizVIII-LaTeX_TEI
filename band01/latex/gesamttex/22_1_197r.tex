      
               
                \begin{ledgroupsized}[r]{120mm}
                \footnotesize 
                \pstart                
                \noindent\textbf{\"{U}berlieferung:}   
                \pend
                \end{ledgroupsized}
            
              
                            \begin{ledgroupsized}[r]{114mm}
                            \footnotesize 
                            \pstart \parindent -6mm
                            \makebox[6mm][l]{\textit{L}}Konzept: LH XXXVIII Bl. 197. 1 Bl. 2\textsuperscript{o}. 1 S. zweispaltig. Auf dem Blatt befanden sich urspr\"{u}nglich folgende, nicht von Leibniz stammende Textfragmente. Vorderseite gegenl\"{a}ufig: ad pr. t. testamenta. 26. R\"{u}ckseite: S ins. vmc. Die Marginalie beginnt in der letzten Zeile der linken Spalte und wird gegenl\"{a}ufig in 2/3 der rechten Spalte fortgesetzt. Der verbleibende Teil wird von 14 Zeilen eines gestrichenen Textes ausgef\"{u}llt, der inhaltlich nicht zur Reihe VIII geh\"{o}rt und daher hier nicht wiedergegeben wird.\\KK 1, Nr. 971\textsuperscript{b} \pend
                            \end{ledgroupsized}
                %\normalsize
                \vspace*{5mm}
                \begin{ledgroup}
                \footnotesize 
                \pstart
            \noindent\footnotesize{\textbf{Datierungsgr\"{u}nde}: Der Text wird datiert nach dem Wasserzeichen des verwendeten Papiers, das sich auch in den Textträgern des Stücks \textit{LSB} VI, 3 N. 38 findet. Die dort gegebene Datierungsbegr\"{u}ndung wird \"{u}bernommen.}
                \pend
                \end{ledgroup}
            
                \vspace*{8mm}
                \pstart 
                \normalsize
            [197 r\textsuperscript{o}] Possunt multa adhuc construi organa mirabilia pro Typographia\protect\index{Sachverzeichnis}{typographia}, ut aeque prope celeriter ac scriptura fieri possit non lapsu cistularum, sondern mit einem Wellbaum\protect\index{Sachverzeichnis}{Wellbaum} da das begehrte in casu pulsen herausgienge und imprimirte, aber stracks wieder drinn wer. Der welbaum\protect\index{Sachverzeichnis}{Wellbaum} m\"{u}ste nicht, sondern das papyr\protect\index{Sachverzeichnis}{Papier} unter ihm richtig fort gehen. Oder wohl gar uber ihm und der welbaum\protect\index{Sachverzeichnis}{Wellbaum} hinauf oder das Papyr\protect\index{Sachverzeichnis}{Papier} hinunter gedruckt werden, damit man dem Welbaum\protect\index{Sachverzeichnis}{Wellbaum} ne\"{u}e schwarze geben kondte, denn were drucken und sezen eins. Es were guth wenn an einem Zacken alle buchstaben weren oder zum wenigsten draus minimis varietatibus konten formirt werden, \edtext{welches}{\lemma{werden,}\Afootnote{ \textit{ (1) }\ dahehr \textit{ (2) }\ welches \textit{ L}}} mit gegenwertigen Alphabet schwehr w\"{u}rde zugehen und mußte man daher ein ander alphabet machen, da alle buchstaben auseinander, oder homogeneis varie locatis angulis saltem aut linearum numero vel multitudine variante, componirt wurden, wie die Tachygra$\upphi$ia
            \protect\index{Sachverzeichnis}{tachygraphia} mit sich bringt. Diß were denn eine Tachytypia\protect\index{Sachverzeichnis}{tachytypia} doch durffte es vielleicht mit gemeinen alphabet nicht schwehr seyn, wenn alle buchstaben in der n\"{a}he leicht konten einander furgeschoben werden. Weren concentrice umb einander herumb, der da schlegt, stunde in der mitten oder potius weils nicht weit seyn wird noch darf darneben und schl\"{u}ge allezeit. Der herein fallende k\"{a}me ansonsten in ein loch zu stehen, zu drucken nach dem er gedruckt und die hand remittirt kame stracks wieder hinauf an seinen orth. Oder konten alle fest an einer herumb gehenden kleinen kugel seyn, so aber vielleicht etwas langsamer. Imo non puto. Ein ieder griff oben were mit dem punct der kugel connex so heraus soll, und dr\"{u}ckte ihn ins loch. Besser wenn es unvergleichlich. Wenn alles abc. bestunde in auf und zu gethanen, linien etc. wie mans den oben machte, so th\"{a}te es unten auch scilicet man besteckte die ubrigen das sie nichts thun k\"{o}nten. Das beste mittel, das alle mahl der buchstabe so sehr soll oben in die Mitten gezogen w\"{u}rde. Konte ein buchstab \edtext{zugleich}{\lemma{buchstab}\Afootnote{ \textit{ (1) }\ f\"{u}r \textit{ (2) }\ zugleich \textit{ L }\ \hspace{10mm} 21 \hspace{3mm} ut \textit{ (1) }\ quaelibet lineae \textit{ (2) }\ cuilibet lineae \textit{ L}}} etliche auf sich haben, welche seite man wolte hinauf oder hinabkehren, als \includegraphics[width=0.15\textwidth]{images/38_197r} muste alles seyn das es fest were und nicht wancken k\"{o}nte.\footnote{\textit{Beginnt am unteren Blattrand neben dem letzten Satz, wird ab} finem destinatum \textit{in der rechten Spalte in gegenl\"{a}ufiger Schreibrichtung zur ersten Spalte weitergef\"{u}hrt}: NB. Sed haec non assequuntur finem destinatum. Denn dergestalt druckte man nur eins mit einem Saz. Ergo sit ita ut cuilibet lineae parti correspondeatur foramine, unde emittatur quod lubet in quolibet foramine sint literae omnes. Resurgant omnia per tubulos suos in locum priorem affusa aqua relegendo vestigia, uti sunt delapsa, si nimirum foramina eadem retro applicentur per singulas lineas. Sed hac extensu difficilia credo tamen adhuc possibilia. Esset res ita commode, si foramen esset variabilis magnitudinis, aut potius multorum tuborum et foraminum in unum foramen impressionis tendentium, sed omnia foramina tubique sint inaequales, non magnitudine sed figura, ita ut nullum possit ascendere per magnitudinem et foramen alterius. Ita unumquodque ascendet per foramen suum. Sed dubito an ita! non enim forte inveniet sed alibi haerebit, et si multum agitetur. Alioquin esset haec commodissima ratio et emittendi quae velis, et ea restituendi in statum priorem, sed sphaera superior unde delabuntur, debet ita multis modis esse major. Vereor ut procedat, quia multorum ascensus esse debet nimis obliquus. Summa nondum hoc succedit.} Ein ieder buchstabe lief selbst wird liberatus ausm mittel in sein eigen loch. Haec ratio videtur non difficilis. Man darff also die buchstaben nicht wieder austheilen, item pressen und sezen ist eins.
            \pend 