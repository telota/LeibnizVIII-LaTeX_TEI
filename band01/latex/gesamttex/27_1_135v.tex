[135 v\textsuperscript{o}] c'est une marque que l'union des placques ne se fait pas par \edtext{le mouuement de la liqueur en tous sens}{\lemma{par}\Afootnote{ \textit{ (1) }\ l'union des placques en tous sens \textit{ (2) }\ le [...] sens \textit{ L}}}, car il est \`{a} croire que le mouuement n'est pas d'une même force en toutes les \edlabel{liqu135v1}liqueurs\protect\index{Sachverzeichnis}{liqueur}.%\edtext{}{\lemma{.}\Bfootnote{Ende der Streichung}}
\pend 
\pstart
 \edtext{\edlabel{liqu135v2}}{\lemma{liqueurs.}\xxref{liqu135v1}{liqu135v2}\Afootnote{ \textbar\ \textso{Experience 5}\textsuperscript{me}. S\c{c}avoir  \textit{ (1) }\ si deux corps\protect\index{Sachverzeichnis}{corps|textit} \textit{ (2) }\ la même force qui tient les corps ensemble dans le vuide, les joint ensemble, s'ils ne sont que peu eloign\'{e}z l'un de l'autre, comme si une placque est suspend\"{u}e   \textbar\ fermement \textit{ erg.}\ \textbar\  d'une anse en-haut, l'autre dessous justement, eloign\'{e}es tres peu appuy\'{e}e simplement sur quelque support, sans attachement, s\c{c}avoir si celle d'embas s'elevera vers celle d'en-haut, si le mouuement de la liqueur\protect\index{Sachverzeichnis}{liqueur} en tous sens est la raison des phenomenes de l'attachement dans le vuide\protect\index{Sachverzeichnis}{vide}, cela arrivera. Car il y a tres peu de coups entre deux, et beaucoup par dehors. 
\textso{Experience 6}\textsuperscript{me}. S\c{c}avoir si la force unitive peut estre vainc\"{u}e, ou sinon plustost, il arrive quelque changement   \textbar\ dans la matiere \textit{ erg.}\ \textbar\  par la force de la change: comme par exemple, si le Mercure\protect\index{Sachverzeichnis}{mercure!purg\'{e}} purg\'{e} estant trop haut, cela fait engendrer une bulle par force, pour faire tomber le Mercure\protect\index{Sachverzeichnis}{mercure}, et si les deux placques\protect\index{Sachverzeichnis}{deux placques} se courbent en quelque fa\c{c}on pour  \textit{ (a) }\ donner la libert\'{e} \`{a} \textit{ (b) }\ se pouuoir separer, et se remettent par apres par leur ressort. Je me souuiens d'avoir remarqu\'{e} ailleurs des moyens pour s'en \'{e}claircir. \textit{ gestr.}\ \textbar\ \textso{Experience} \textit{ L}}}
 \textso{Experience 7}\textsuperscript{me}. S\c{c}avoir si la bulle ne fait rien, dans le tuyau de la liqueur purg\'{e}e\protect\index{Sachverzeichnis}{liqueur!purg\'{e}e} \edtext{sinon quand elle arrive}{\lemma{purg\'{e}e}\Afootnote{ \textit{ (1) }\ quand elle n'arrive pas \textit{ (2) }\ sinon quand elle arrive \textit{ L}}} \`{a} la hauteur du niveau ordinaire, de laquelle la liqueur\protect\index{Sachverzeichnis}{liqueur} apres estre tomb\'{e}e d'enhaut, demeurera neantmoins suspend\"{u}e. On peut eprouuer cela avec le Mercure\protect\index{Sachverzeichnis}{mercure!purg\'{e}} purg\'{e} dans l'air libre, car il faut que la bulle monte \`{a} la hauteur de 27 pouces.\pend 
\pstart \textso{Experience 8}\textsuperscript{me}. S\c{c}avoir s'il faut que la bulle touche la superficie interieure du tuyau, afin que la liqueur purg\'{e}e \protect\index{Sachverzeichnis}{liqueur!purg\'{e}e} se puisse d\'{e}tacher. On peut essayer le contraire, en laissant entrer une bulle, et en la faisant monter au milieu du tube, estant appuy\'{e} \`{a} quelque petite perche ou fil de fer\protect\index{Sachverzeichnis}{fer} appuy\'{e} au fond, et n'arrivant\footnote{\textit{In der rechten Spalte}: La perche se peut etendre en-haut pour s\c{c}avoir si la bulle glissera plus outre.}  pas jusque en-haut ny a la surperficie interieure du tuyau, il faut même prendre garde, si elle s'etendra \edtext{le long de ce fil}{\lemma{s'etendra}\Afootnote{ \textit{ (1) }\ \`{a} la perche \textit{ (2) }\ le long de ce fil \textit{ L}}} subitement, et si par apres elle montera plus haut, quand elle peut. \pend
 \pstart  \textso{Experience 9}\textsuperscript{me}. S\c{c}avoir \textso{s'il faut qu'il y ait seulement une bulle d'une cost\'{e}.} On peut essayer le contraire, en faisant entrer deux bulles d'air en même temps. Item en faisant entrer une seule en milieu, laquelle monte par une perche qui arrive en haut du tuyau, si alors l'eau tombe neantmoins, la raison allegu\'{e}e dans la lettre \`{a} Mons. Chouet\protect\index{Namensregister}{\textso{Chouet,} Jean-Robert 1642\textendash 1731} ne pourra pas estre juste.\edtext{}{\lemma{juste.}\Bfootnote{Vgl. hierzu \cite{00267}N. 48.}} Car elle suppose, que la bulle se trouue d'un cost\'{e} du tuyau.\pend \pstart \textso{Exper. 10}\textsuperscript{me}. S\c{c}avoir si non seulement la liqueur\protect\index{Sachverzeichnis}{liqueur!purg\'{e}e} purg\'{e}e demeure suspend\"{u}e d'une plus grande hauteur qu'\`{a} l'ordinaire; mais si elle peut même estre elev\'{e}e plus haut qu'\`{a} l'ordinaire, par une pompe\protect\index{Sachverzeichnis}{pompe} ou par un siphon \`{a} jambes inegales. Cette experience refutera la Gl\"{u}e, dont \edtext{quelques-uns}{\lemma{quelques-uns}\Bfootnote{Diese Ansicht wurde u. a. von  Niccol\`{o} Zucchi\protect\index{Namensregister}{\textso{Zucchi} (Zucchius), Niccol\`{o} SJ 1586\textendash 1670} vertreten.}} se servent pour expliquer le phenomene.  \pend 