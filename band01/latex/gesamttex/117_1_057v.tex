[57 v\textsuperscript{o}]   Ad nostram delineationem longitudinum\protect\index{Sachverzeichnis}{longitudo} opus, quod et efficitur ut aliquid nach proportion im verjungten masstab soviel zur\"{u}ckgehe, als das schiff\protect\index{Sachverzeichnis}{Schiff} vor sich, und denn etwas, das stets ad certum locum weise, a quo cognita distantia nostra a data re, et modis flexionis situs \edtext{ portus a quo abiimus vel ad quem tendimus, et omnino locus noster cognoscitur}{\lemma{situs}\Afootnote{ \textit{ (1) }\ rei cogno  \textit{ (2) }\ portus [...] cognoscitur \textit{ L}}}. \pend \pstart \edtext{}{\lemma{}\Bfootnote{ See marginal note fol. LH035,15,06\_057r}} \pend 